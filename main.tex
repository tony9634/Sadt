%!TEX bibfile = bibliografia.bib

%Questo è il preambolo, dove si inseriscono i pacchetti e le impostazioni che servono per compilare il documento. Quanto scritto dopo il simbolo '%' è solo un commento e serve a fini dimostrativi.
\documentclass[a4paper, 12, oneside]{report}
%\usepackage[13pt]{extsizes}
\linespread{1.5} %interlinea
\setlength{\parindent}{0pt} % niente rientro dopo punto a capo
\setlength{\parskip}{0.75em}
\pagestyle{plain}
\usepackage{geometry} %margini
\geometry{a4paper, top=2.5cm, bottom=2cm, left=2cm, right=2cm, bindingoffset=5mm}
\usepackage{graphicx}

\usepackage{listings}
\usepackage{xcolor}
\usepackage{tikz}
\usepackage{tikz-3dplot}
\usetikzlibrary{angles,quotes,calc,positioning}
\usetikzlibrary{decorations.pathmorphing}
\usetikzlibrary{decorations.markings}
\definecolor{darkgreen}{rgb}{0,0.5,0}
\tdplotsetmaincoords{70}{120}
\tikzset{
    fLines/.style={thin,->}
}

\lstset{
  language=Matlab,
  basicstyle=\ttfamily\small,
  keywordstyle=\color{blue},
  commentstyle=\color{green!60!black},
  stringstyle=\color{red},
  numbers=left,
  numberstyle=\tiny,
  stepnumber=1,
  numbersep=5pt,
  backgroundcolor=\color{gray!10},
  frame=single,
  breaklines=true,
  captionpos=b,
  showspaces=false,
  showstringspaces=false,
  showtabs=false,
  tabsize=2
}
\usepackage{multicol} %più colonne
\usepackage{ragged2e} %allineamento testo
\usepackage{caption}
\usepackage{pdfpages}
\usepackage{csquotes}
\usepackage{amsmath}
\usepackage{mathtools}
\usepackage{amssymb}
\usepackage{esint}
\usepackage{pgfplotstable}
\usepackage{pgfplots}
\pgfplotsset{compat=1.7}
\captionsetup{textfont=sl}
\maxdeadcycles=300

\usepackage{morefloats}

\usepackage[italian]{babel} %lingua principale

\usepackage{minitoc} %mini sommario a inizio capitolo
\nomtcrule 
\addto{\captionsitalian}{% Making babel aware of special titles
  \renewcommand{\mtctitle}{Sommario}
}
\setcounter{secnumdepth}{3}
\usepackage[sorting=none,backend=biber,refsection=chapter]{biblatex} %bibliografia
\defbibheading{bibbysubsect}{}
\addbibresource{bibliografia.bib}

\usepackage{hyperref}
\usepackage{comment}
\usepackage{xpatch}
\usepackage{blindtext}


\makeatletter

\xpatchcmd{\@makeschapterhead}{%
  \Huge \bfseries  #1\par\nobreak%
}{%
  \Huge \bfseries\centering #1\par\nobreak%
}{\typeout{Patched makeschapterhead}}{\typeout{patching of @makeschapterhead failed}}

\xpatchcmd{\@makechapterhead}{%
  \huge\bfseries \@chapapp\space \thechapter
}{%
  \huge\bfseries\centering \@chapapp\space \thechapter
}{\typeout{Patched @makechapterhead}}{\typeout{Patching of @makechapterhead failed}}

\makeatother
\usepackage{csvsimple}
\usepackage{fancyhdr}
\usepackage[export]{adjustbox}

\usepackage{booktabs} %per le tabelle
\usepackage{multirow}
%\usepackage[table,xcdraw]{xcolor}

\usepackage{lscape}
\usepackage{wrapfig}
\usepackage{pgffor}
\usepackage{subcaption}

\usepackage[version=4]{mhchem}
%\usepackage{chemfig}
\usepackage{calc} % per calcoli sulle lunghezze

\usepackage{chemfig}
\usepackage{longtable}
\usepackage{array}
\usepackage{enumitem}
\usepackage{supertabular}
\usepackage{breqn}


\newcolumntype{C}[1]{>{\centering\arraybackslash}p{#1}}
%\renewcommand{\arraystretch}{1.4} % <-- Aggiunge spazio tra le righe

\newcommand{\gbar}{\mathrlap{\raisebox{0.5ex}{\rule{0.5em}{0.1pt}}}\gamma}

% --- definizioni personalizzate ---
\newcommand{\imagesperrow}{3} % immagini per riga
\newcommand{\imgwidth}{0.3\textwidth} % larghezza di ciascuna immagine
\newcounter{imgcount}


%Da qui in poi inizia il documento.
\begin{document}
\begin{center}
    \Huge\textbf{Università degli Studi di Napoli “Federico II”}
    
    \vspace{1cm}

    \includegraphics[width=0.40\textwidth]{media/Logo.pdf}\\[1cm]
    
    \Huge\textbf{Appunti del Corso Di Strumentazione Avanzata per La Diagnosi e Terapia}
    
    \vspace{1cm}
    
    \large A.A. 2019/2020
    
    \vspace{1cm}
    
    \textbf{Docente: Mario Sansone}\\
    (Beppe Vessicchio)
    
    \vfill
    
    \begin{flushright}
    \textbf{A cura di}\\
    \textit{
    Abagnale Francesca M54000961\\
    Ausilio Antonio M54000985\\
    Senese Cristian M54000955}
    \end{flushright}
\end{center}

\newpage


\dominitoc
\tableofcontents

\begin{center}
\vfill
    \chapter{Meccanica classica}
    \label{blx:refsection\therefsection}
\vfill

\minitoc
\newpage
\end{center}
\justify

\section{Cenni di meccanica classica}\label{cenni-di-meccanica-classica}

La \textbf{meccanica} è la branca della fisica che studia il moto dei corpi materiali \cite{landau1994meccanica}. In base alle caratteristiche fisiche della materia considerata, sono state sviluppate diverse teorie meccaniche, suddivise principalmente in:

\begin{itemize}
\item
 \textbf{Meccanica classica}: descrive sistemi di dimensioni macroscopiche che si muovono a velocità molto inferiori rispetto a quella della luce;
\item
 \textbf{Meccanica statistica}: applicabile a sistemi costituiti da un numero elevato di particelle, delle quali si analizzano le proprietà medie;
\item
 \textbf{Meccanica relativistica}: tratta sistemi non quantistici che si muovono a velocità prossime a quella della luce;
\item
 \textbf{Meccanica quantistica}: si occupa di sistemi su scala atomica e subatomica, dove gli effetti quantistici risultano predominanti.
\end{itemize}

\section{Meccanica newtoniana}\label{meccanica-newtoniana}
La \textbf{meccanica classica} si fonda sulla descrizione dei fenomeni fisici secondo l'approccio introdotto da Newton nel XVII secolo. Essa interpreta il moto della natura in termini di forze \(\vec{F}\) e accelerazioni \(\vec{a}\). Il punto di vista \textbf{newtoniano è di tipo locale}: conoscendo le forze agenti su una particella e il loro andamento temporale, è possibile determinare il moto di quest'ultima.

L'equazione fondamentale della meccanica newtoniana è:

\[\vec{F} = m\vec{a}\]

dove \(m\) è la massa della particella (assunta costante) e \(\vec{a}\) è la sua accelerazione. L'accelerazione è definita come la derivata della velocità rispetto al tempo:

\[\vec{a} = \dfrac{d\vec{v}}{dt}\]

Si definisce la \textbf{quantità di moto} (o \textbf{momento lineare}) \(\vec{p}\) come il prodotto tra la massa e la velocità della particella:

\[\vec{p} = m\vec{v}\]

Il vettore \(\vec{p}\) ha la stessa direzione e verso del vettore velocità \(\vec{v}\). L'equazione fondamentale della meccanica può essere riscritta in termini di quantità di moto:

\[\vec{F} = m\vec{a} = m\dfrac{d\vec{v}}{dt} = \dfrac{d}{dt}(m\vec{v}) = \dfrac{d\vec{p}}{dt}\]

Questa forma è valida anche nel caso in cui la massa \(m\) non sia costante nel tempo, ad esempio, nel caso di corpi che perdono massa come una navetta spaziale.

L'equazione può essere estesa a un sistema di \(n\) particelle, con masse \(\{ m_{1},m_{2},\ldots,m_{n}\}\) e forze \(\{{\vec{f}}_{1},{\vec{f}}_{2},\ldots,{\vec{f}}_{n}\}\), ottenendo:

\[
{\vec{F}}_{\text{tot}} = \sum_{k = 1}^{n}{\vec{f}}_{k} = \sum_{k = 1}^{n}\dfrac{d{\vec{p}}_{k}}{dt}
\]

Questa espressione rappresenta il teorema della \textbf{dinamica dei sistemi di particelle}, il quale afferma che la somma delle forze agenti su un sistema di particelle è uguale alla derivata temporale della somma delle quantità di moto delle singole particelle.

\begin{figure}[ht]
\centering
\includegraphics[width=2.65748in,height=2.12598in,alt={Immagine che contiene linea Il contenuto generato dall'IA potrebbe non essere corretto.}]{media/1_Meccanica/image2.pdf}\caption{Sistema di particelle soggetto a forze esterne}
\end{figure}

Per una particella soggetta a una forza \(\vec{F}\), si definisce \textbf{momento torcente} (o \textbf{momento della forza}) \(\vec{N}\) (talvolta indicato anche con \(\vec{\tau}\)):

\[\vec{N} = \vec{r} \times \vec{F}\]

dove \(\vec{r}\) è il vettore posizione della particella rispetto a un polo (origine del sistema di riferimento).

\begin{figure}[ht]
\centering
\includegraphics[width=2.62626in,height=1.47606in,alt={Immagine che contiene linea, schizzo, diagramma Il contenuto generato dall'IA potrebbe non essere corretto.}]{media/1_Meccanica/image3.pdf}\caption{Definizione del momento torcente rispetto a un punto fisso}
\end{figure}

Analogamente, si definisce \textbf{momento angolare} (o \textbf{quantità di moto angolare}) \(\vec{L}\):

\[\vec{L} = \vec{r} \times \vec{p}\]

\begin{figure}[ht]
\centering
\includegraphics[width=2.71741in,height=1.60327in,alt={Immagine che contiene testo, diagramma, design Il contenuto generato dall'IA potrebbe non essere corretto.}]{media/1_Meccanica/image4.pdf}\caption{Momento angolare di una particella rispetto a un'origine fissa}
\end{figure}

Sostituendo la definizione di quantità di moto:

\[\vec{L} = \vec{r} \times (m\vec{v}) = m(\vec{r} \times \vec{v})\]

Applicando la derivata rispetto al tempo:

\[\begin{aligned}
\dfrac{d\vec{L}}{dt} & = \dfrac{d}{dt}(\vec{r} \times \vec{p}) = \dfrac{d\vec{r}}{dt} \times \vec{p} + \vec{r} \times \dfrac{d\vec{p}}{dt}
\end{aligned}\]

Nel caso in cui il polo sia fisso:

\[\dfrac{d\vec{r}}{dt} \times \vec{p} = \vec{v} \times \vec{p} = \vec{0}\]

perché \(\vec{v}\) e \(\vec{p}\) sono paralleli. Resta quindi:

\[\dfrac{d\vec{L}}{dt} = \vec{r} \times \dfrac{d\vec{p}}{dt} = \vec{r} \times \vec{F} = \vec{N}\]

Questa equazione prende il nome di \textbf{teorema del momento angolare}.

Per un sistema di particelle, il momento torcente totale è:

\[{\vec{N}}_{\text{tot}} = \sum_{k = 1}^{n}{{\vec{r}}_{k} \times {\vec{f}}_{k}} = \sum_{k = 1}^{n}{{\vec{r}}_{k} \times \dfrac{d{\vec{p}}_{k}}{dt}} = \dfrac{d\vec{L}}{dt}\]

Conoscendo le forze agenti, è possibile determinare il moto della particella integrando l'equazione fondamentale:

\[\vec{F} = \dfrac{d\vec{p}}{dt}\]

Integrando nel tempo tra un istante iniziale \(t_{0}\) e un istante generico \(t\), si ottiene:

\[\int_{t_{0}}^{t}{\vec{F}\, dt} = \vec{p}(t) - \vec{p}(t_{0}) = m\lbrack\vec{v}(t) - \vec{v}(t_{0})\rbrack\]

Assumendo \(\vec{v}(t_{0}) = \vec{0}\), si ha:

\[\int_{t_{0}}^{t}{\vec{F}\, dt} = m\vec{v}(t)\]

Integrando una seconda volta, si ottiene lo spostamento:

\[\int_{t_{0}}^{t}{\left( \int_{t_{0}}^{t'}{\vec{F}\, dt''} \right)dt'} = m\lbrack\vec{s}(t) - \vec{s}(t_{0})\rbrack\]

dove \(\vec{s}(t)\) è il vettore spostamento.

Il modello newtoniano descrive accuratamente molti fenomeni osservati nella sua epoca, come il moto dei pianeti, il comportamento dei corpi sotto l'azione di forze, e le interazioni meccaniche quotidiane.

\section{Meccanica lagrangiana}\label{meccanica-lagrangiana}
La \textbf{meccanica lagrangiana} è una formulazione matematica della meccanica introdotta nel XVIII secolo da Joseph-Louis Lagrange, come riformulazione della meccanica newtoniana \cite{arnold1992matematici, landau1994meccanica}.

Questa descrizione parte da un punto di vista globale e si propone di determinare il moto di un sistema minimizzando una funzione chiamata \emph{azione}, che dipende dall'intero percorso del sistema. Il modello lagrangiano è particolarmente utile, poiché consente di descrivere non solo fenomeni della meccanica classica, ma anche situazioni della \textbf{meccanica quantistica}.

Consideriamo un sistema composto da \(N\) particelle. La descrizione del loro moto secondo la meccanica newtoniana richiede l'uso di uno spazio tridimensionale cartesiano: a ciascuna particella \(m_{i}\) è associata una terna di coordinate \((x_{i},y_{i},z_{i})\), che variano nel tempo. Questo approccio porta alla necessità di risolvere \(3N\) equazioni differenziali per determinare il moto di tutte le particelle.

Tuttavia, spesso le particelle sono soggette a vincoli che limitano il loro moto a determinate traiettorie o superfici. In questi casi, è possibile descrivere il sistema utilizzando \emph{coordinate generalizzate} \(q_{i}\), con \(i = 1,2,\ldots,s\), dove \(s\) rappresenta il numero dei \emph{gradi di libertà} del sistema.

Tra tutte le curve che collegano un punto \(\vec{A}\) al tempo \(t_{0}\) con un altro punto \(\vec{B}\) al tempo \(t_{1}\), esiste una traiettoria unica che rende stazionaria l'azione, ovvero l'integrale della funzione lagrangiana nel tempo.

\begin{figure}[ht]
\centering
\includegraphics[width=3.9759in,height=1.98795in,alt={Immagine che contiene calligrafia, linea Il contenuto generato dall'IA potrebbe non essere corretto.}]{media/1_Meccanica/image5.pdf}\caption{Esempio di moto su traiettoria a azione stazionaria tra due punti nel tempo.}
\end{figure}
\subsection{Lemma 1: Principio di Azione Stazionaria di Hamilton}\label{lemma-1-principio-di-minimizzazione}

Il moto della particella, ovvero la traiettoria \(q_{i}(t)\) e la velocità con cui essa viene percorsa \({\dot{q}}_{i}(t)\), deve rendere stazionario l'integrale \cite{arnold1992matematici}:

\[S = \int_{t_{0}}^{t_{1}}{L\left( q_{i},{\dot{q}}_{i},t \right)\, dt}\]

L'integrale \(S\) è detto \emph{azione}, \(t\) è la variabile temporale, mentre \(L\) è la funzione \emph{lagrangiana}, o semplicemente \emph{lagrangiana}, del sistema di particelle. Dal punto di vista dimensionale, la lagrangiana è omogenea all'energia, e quindi ha le stesse dimensioni del joule:

\[\lbrack L\rbrack = \lbrack J\rbrack\]

dove \(\lbrack L\rbrack\) indica le dimensioni fisiche della lagrangiana e \(\lbrack J\rbrack\) quelle dell'energia, ovvero:

\[\lbrack J\rbrack = kg \cdot m^{2} \cdot s^{- 2}\]

\subsection{Lemma 2: equazione di Eulero-Lagrange}\label{lemma-2-equazione-di-eulero-lagrange}

Per descrivere il moto di una particella o di un sistema di particelle, la lagrangiana deve soddisfare l'equazione di Eulero-Lagrange \cite{landau1994meccanica}:

\begin{align*}
\dfrac{d}{dt}\left(\dfrac{\partial L}{\partial\dot{q}_i}\right) - \dfrac{\partial L}{\partial q_i} &= 0, \ i=1,2,\dots,s
\end{align*}

Poiché l'azione \(S\) deve essere stazionaria, una sua variazione \(\delta S\), dovuta a una perturbazione dello spostamento \(\delta\vec{q}\) e della velocità \(\delta\dot{\vec{q}}\), è tale che:

\[S + \delta S = \int_{t_{0}}^{t_{1}}{L\left( \vec{q} + \delta\vec{q},\dot{\vec{q}} + \delta\dot{\vec{q}} \right)dt}\]

Dove \(\vec{q}\) è una traiettoria che rende stazionaria l'azione \(S\). Il Principio di Hamilton (o di Minima Azione) stabilisce che la traiettoria rende l'azione stazionaria o un estremo (ovvero $\delta S=0$). Non è garantito che sia un minimo poiché può essere un massimo o un punto di sella.

Siccome le variazioni di spostamento e velocità sono molto minori delle rispettive qualità:

\[
\delta q \ll q,\ \delta\dot{q} \ll \dot{q}
\]

è possibile sviluppare la lagrangiana in serie di Taylor, arrestando lo sviluppo al primo ordine, nell'intorno del punto \(\left( \vec{q},\dot{\vec{q}} \right)\):

\[L\left( \vec{q} + \delta\vec{q},\dot{\vec{q}} + \delta\dot{\vec{q}} \right) \simeq L\left( \vec{q},\dot{\vec{q}} \right) + \vec{\nabla}L \cdot \left( \delta\vec{q},\delta\dot{\vec{q}} \right) =\]

Per definizione di gradiente si ha:

\[= L\left( \vec{q},\dot{\vec{q}} \right) + \left( \dfrac{\partial L}{\partial\vec{q}},\dfrac{\partial L}{\partial\dot{\vec{q}}} \right) \cdot \left( \delta\vec{q},\ \delta\dot{\vec{q}} \right) =\]

Svolgendo l'operazione di prodotto scalare si ha:

\[= L\left( \vec{q},\dot{\vec{q}} \right) + \dfrac{\partial L}{\partial\vec{q}}\delta\vec{q} + \dfrac{\partial L}{\partial\dot{\vec{q}}}\ \delta\dot{\vec{q}}\]

Dove il gradiente è un vettore colonna mentre le coordinate generalizzate dei vettori riga. Sostituendo nell'espressione della variazione dell'azione, si ottiene:

\[S + \delta S = \int_{t_{0}}^{t_{1}}{\left\lbrack L\left( \vec{q},\dot{\vec{q}} \right) + \dfrac{\partial L}{\partial\vec{q}}\delta\vec{q} + \dfrac{\partial L}{\partial\dot{\vec{q}}}\ \delta\dot{\vec{q}} \right\rbrack dt}\]

Per la linearità dell'integrale si scrive:

\[S + \delta S = \int_{t_{0}}^{t_{1}}{L\left( \vec{q},\dot{\vec{q}} \right)dt} + \int_{t_{0}}^{t_{1}}{\left( \dfrac{\partial L}{\partial\vec{q}}\delta\vec{q} + \dfrac{\partial L}{\partial\dot{\vec{q}}}\ \delta\dot{\vec{q}} \right)dt}\]

Dove:

\[S = \int_{t_{0}}^{t_{1}}{L\left( \vec{q},\dot{\vec{q}} \right)dt}\]

Da cui si ottiene:

\[S + \delta S = S + \int_{t_{0}}^{t_{1}}{\left( \dfrac{\partial L}{\partial\vec{q}}\delta\vec{q} + \dfrac{\partial L}{\partial\dot{\vec{q}}}\ \delta\dot{\vec{q}} \right)dt}\]

Semplificando \(S\), si ottiene un'espressione per la variazione dell'azione \(\delta S\):

\[\delta S = \int_{t_{0}}^{t_{1}}{\left( \dfrac{\partial L}{\partial\vec{q}}\delta\vec{q} + \dfrac{\partial L}{\partial\dot{\vec{q}}}\ \delta\dot{\vec{q}} \right)dt}\]

La traiettoria perturbata \(\delta\vec{q}\) ha in comune con la traiettoria \(\vec{q}\) il punto iniziale \(\vec{A}\) all'istante \(t_{0}\) e il punto di fine \(\vec{B}\) all'istante \(t_{1}\), ne discende che:

\[\begin{cases}
\vec{q}\left( t_{0} \right) = \vec{q}\left( t_{0} \right) + \delta\vec{q}\left( t_{0} \right) = \vec{A} \\
\vec{q}\left( t_{1} \right) = \vec{q}\left( t_{1} \right) + \delta\vec{q}\left( t_{1} \right) = \vec{B}
\end{cases} \]

Non variando il punto iniziale, risulta che nei punti iniziali non vi sono perturbazioni:

\[\delta\vec{q}\left( t_{0} \right) = \vec{0},\ \ \delta\vec{q}\left( t_{1} \right) = \vec{0}\]

\begin{figure}[ht]
\centering
\includegraphics[width=5.17616in,height=2.58808in,alt={Immagine che contiene Carattere, linea, diagramma Il contenuto generato dall'IA potrebbe non essere corretto.}]{media/1_Meccanica/image6.pdf}\caption{Perturbazione della traiettoria}
\end{figure}

Se risulta che:

\[\dot{\vec{q}} = \dfrac{d\vec{q}}{dt}\]

Allora, deve accadere che:

\[\delta\dot{\vec{q}} = \dfrac{d}{dt}\left( \delta\vec{q} \right)\]

Dunque, la variazione di azione può essere espressa come:

\[\delta S = \int_{t_{0}}^{t_{1}}{\left( \dfrac{\partial L}{\partial\vec{q}}\delta\vec{q} + \dfrac{\partial L}{\partial\dot{\vec{q}}}\ \delta\dot{\vec{q}} \right)dt} = \int_{t_{0}}^{t_{1}}{\dfrac{\partial L}{\partial\vec{q}}\delta\vec{q}dt} + \int_{t_{0}}^{t_{1}}{\dfrac{\partial L}{\partial\dot{\vec{q}}}\ \dfrac{d}{dt}\left( \delta\vec{q} \right)dt}\]

Si considera la quantità:

\[\dfrac{d}{dt}\left( \dfrac{\partial L}{\partial\dot{\vec{q}}}\delta\vec{q} \right)\]

Questa può essere riscritta ricorrendo alle proprietà del prodotto:

\[\dfrac{d}{dt}\left( \dfrac{\partial L}{\partial\dot{\vec{q}}}\delta\vec{q} \right) = \left( \dfrac{d}{dt}\dfrac{\partial L}{\partial\dot{\vec{q}}} \right)\delta\vec{q} + \dfrac{\partial L}{\partial\dot{\vec{q}}}\dfrac{d}{dt}\left( \delta\vec{q} \right)\]

Da cui è possibile ricavare:

\[\dfrac{\partial L}{\partial\dot{\vec{q}}}\dfrac{d}{dt}\left( \delta\vec{q} \right) = \dfrac{d}{dt}\left( \dfrac{\partial L}{\partial\dot{\vec{q}}}\delta\vec{q} \right) - \left( \dfrac{d}{dt}\dfrac{\partial L}{\partial\dot{\vec{q}}} \right)\delta\vec{q}\]

Sostituendo nel secondo integrale della variazione dell'azione si ottiene:

\[\delta S = \int_{t_{0}}^{t_{1}}{\left\lbrack \dfrac{\partial L}{\partial\vec{q}}\delta\vec{q} + \dfrac{d}{dt}\left( \dfrac{\partial L}{\partial\dot{\vec{q}}}\delta\vec{q} \right) - \left( \dfrac{d}{dt}\dfrac{\partial L}{\partial\dot{\vec{q}}} \right)\delta\vec{q} \right\rbrack dt}\]

Raccogliendo \(\delta\vec{q}\) tra il primo e l'ultimo termine, si ha:

\[\delta S = \int_{t_{0}}^{t_{1}}{\left\lbrack \left( \dfrac{\partial L}{\partial\vec{q}} - \dfrac{d}{dt}\dfrac{\partial L}{\partial\dot{\vec{q}}} \right)\delta\vec{q} + \dfrac{d}{dt}\left( \dfrac{\partial L}{\partial\dot{\vec{q}}}\delta\vec{q} \right) \right\rbrack dt} = \int_{t_{0}}^{t_{1}}{\left( \dfrac{\partial L}{\partial\vec{q}} - \dfrac{d}{dt}\dfrac{\partial L}{\partial\dot{\vec{q}}} \right)\delta\vec{q}dt} + \int_{t_{0}}^{t_{1}}{\dfrac{d}{dt}\left( \dfrac{\partial L}{\partial\dot{\vec{q}}}\delta\vec{q} \right)dt}\]

Si considera l'ultimo integrale. Risulta che:

\[\int_{t_{0}}^{t_{1}}{\dfrac{d}{dt}\left( \dfrac{\partial L}{\partial\dot{\vec{q}}}\delta\vec{q} \right)dt} = \left. \ \ \dfrac{\partial L}{\partial\dot{\vec{q}}}\delta\vec{q} \right|_{t_{0}}^{t_{1}}\ \]

Ma, poiché \(\delta\vec{q}\left( t_{0} \right) = \delta\vec{q}\left( t_{1} \right) = \vec{0}\), l'integrale è nullo. In definitiva, si ottiene:

\[\delta S = \int_{t_{0}}^{t_{1}}{\left( \dfrac{\partial L}{\partial\vec{q}} - \dfrac{d}{dt}\dfrac{\partial L}{\partial\dot{\vec{q}}} \right)\delta\vec{q}dt}\]

Dato che l'azione deve essere stazionaria, la sua variazione deve essere nulla \(\delta S = 0\). Per cui si ha:

\[\int_{t_{0}}^{t_{1}}{\left( \dfrac{\partial L}{\partial\vec{q}} - \dfrac{d}{dt}\dfrac{\partial L}{\partial\dot{\vec{q}}} \right)\delta\vec{q}dt} = 0\]

Se l'integrale è nullo, allora la funzione integranda deve essere nulla:

\[\left( \dfrac{\partial L}{\partial\vec{q}} - \dfrac{d}{dt}\dfrac{\partial L}{\partial\dot{\vec{q}}} \right)\delta\vec{q} = 0\]

Se la variazione della traiettoria, \(\delta\vec{q}\), è non nulla, si ritrova l'equazione di Eulero-Lagrange.

\subsubsection{Particella in coordinate cartesiane}\label{particella-in-coordinate-cartesiane}

Per una \textbf{particella libera} (cioè in assenza di forze conservative, dove l'energia potenziale è nulla) in moto in uno spazio cartesiano, la lagrangiana \(L\) è data da:

\[
L = \dfrac{1}{2}mv^{2}
\]

Dove

\[
v^{2} = {\dot{x}}^{2} + {\dot{y}}^{2} + {\dot{z}}^{2}
\]

In altre parole, in questo caso la lagrangiana coincide con l'energia cinetica della particella. Inoltre, la descrizione lagrangiana si riduce alla modellazione newtoniana, dato che il sistema di riferimento corrisponde con quello cartesiano.

Tuttavia, nella pratica non è sempre conveniente descrivere il moto in coordinate cartesiane. In presenza di vincoli o geometrie particolari, è utile ricorrere alle coordinate generalizzate \(\vec{q}\) e \(\dot{\vec{q}}\). È quindi necessario esprimere l'energia del sistema in funzione di queste coordinate.

\subsection{Lemma 3: energia cinetica in coordinate generalizzate}\label{lemma-3-energia-cinetica-in-generalizzate}

Per un sistema di \(N\) particelle nelle coordinate generalizzate \(\vec{q}\) e \(\dot{\vec{q}}\), l'energia cinetica è:

\[
T = T\left( q_{i},{\dot{q}}_{i} \right),\ \ i = 1,\ \ldots,N
\]

Ovvero l'energia cinetica è una funzione anche delle coordinate generalizzate \(q_{i}\) oltre che della velocità \({\dot{q}}_{i}\).

Si considerano \(N\) funzioni \(f_{i}\) che legano le coordinate cartesiane \(\left( x_{i},y_{i},z_{i} \right)\) con le coordinate generalizzate \(\left( q_{1},q_{2},\ldots,q_{s} \right)\). Ad esempio, si considera:

\[
x_{i} = f_{i}\left( q_{1},\ q_{2},\ \ldots,\ q_{s} \right),\ \ i = 1,2,\ldots,N
\]

Dove:

\[q_{k} = q_{k}(t),\ k = 1,2,\ldots,s\]

Si deriva \(x_{i}\) rispetto al tempo al fine da ottenere la velocità:

\[{\dot{x}}_{i} = \dfrac{d}{dt}f_{i}\left( q_{1},q_{2},\ldots,\ q_{s} \right),\ \ i = 1,2,\ldots,N\]

Per la derivata della funzione composta si ha:

\[{\dot{x}}_{i} = \dfrac{d}{dt}f_{i}\left( q_{1},\ q_{2},\ \ldots,\ q_{s} \right) = \sum_{k = 1}^{s}{\dfrac{\partial f_{i}}{\partial q_{k}}\dfrac{dq_{k}}{dt}} = \sum_{k = 1}^{s}{\dfrac{\partial f_{i}}{\partial q_{k}}{\dot{q}}_{k}}\]

Si eleva al quadrato \({\dot{x}}_{i}\), ottenendo:

\[{\dot{x}}_{i}^{2} = \left( \sum_{k = 1}^{s}{\dfrac{\partial f_{i}}{\partial q_{k}}{\dot{q}}_{k}} \right)^{2} = \sum_{k = 1}^{s}{\dfrac{\partial f_{i}}{\partial q_{k}}{\dot{q}}_{k}}\sum_{k = 1}^{s}{\dfrac{\partial f_{i}}{\partial q_{k}}{\dot{q}}_{k}}\]

È possibile esprimere il secondo membro come doppia sommatoria sugli indici \(k\) e \(j\):

\[{\dot{x}}_{i}^{2} = \sum_{k = 1}^{s}{\sum_{j = 1}^{s}{\dfrac{\partial f_{i}}{\partial q_{j}}{\dot{q}}_{j}\dfrac{\partial f_{i}}{\partial q_{k}}{\dot{q}}_{k}}} = \sum_{k = 1}^{s}{\sum_{j = 1}^{s}{\dfrac{\partial f_{i}}{\partial q_{j}}\dfrac{\partial f_{i}}{\partial q_{k}}{\dot{q}}_{j}}{\dot{q}}_{k}}\]

L'energia cinetica totale del sistema è:

\[T = \dfrac{1}{2}\sum_{i = 1}^{N}{m_{i}{\dot{x}}_{i}}^{2} = \dfrac{1}{2}\sum_{i = 1}^{N}m_{i}\sum_{k = 1}^{s}{\sum_{j = 1}^{s}{\dfrac{\partial f_{i}}{\partial q_{j}}\dfrac{\partial f_{i}}{\partial q_{k}}{\dot{q}}_{j}}{\dot{q}}_{k}}\]

Si definiscono coefficienti metrici:

\[a_{kj}(q) = \sum_{i = 1}^{N}{m_{i}\dfrac{\partial f_{i}}{\partial q_{j}}\dfrac{\partial f_{i}}{\partial q_{k}}}\]

Allora l'energia cinetica può essere scritta come forma quadratica nelle velocità generalizzate:

\[T = \dfrac{1}{2}\sum_{k = 1}^{s}{\sum_{j = 1}^{s}a_{kj}(q){\dot{q}}_{k}{\dot{q}}_{j}}\]

Dato che \(a_{kj}\) è un parametro dipendente dalla posizione generalizzata \(q\), l'energia cinetica totale dipende dalla velocità e dalla posizione generalizzate.

\subsection{Lemma 4: energia potenziale in coordinate generalizzate}\label{lemma-4-energia-potenziale-in-coordinate-generalizzate}

È conveniente esprimere anche l'energia potenziale in funzione delle coordinate generalizzate. Si considera un sistema di \(N\) particelle. L'energia di interazione tra le particelle è indicata con \(U\left( q_{i} \right),\ i = 1,2,\ldots,N\) ed è denotata come energia potenziale. In meccanica classica questa quantità dipende solamente dalla posizione.

In particolare, se le coordinate generalizzate \(q_{i}\) descrivono completamente la configurazione del sistema, allora l'energia potenziale può essere scritta come:

\[U = U(q_{1},q_{2},\ldots,q_{s})\]

dove \(s\) è il numero di coordinate generalizzate.

La meccanica classica assume che le interazioni tra le particelle avvengano istantaneamente, ovvero non considera gli effetti di propagazione dei cambiamenti nei campi di forza. Tuttavia, secondo la teoria della relatività di Einstein, le interazioni si propagano attraverso campi con velocità finita, non superiore alla velocità della luce \(c\). Questo implica che la descrizione classica è un'approssimazione valida solo quando le velocità coinvolte sono molto inferiori a \(c\) e gli effetti di ritardo possono essere trascurati.

\subsection{Lemma 5: forma della lagrangiana}\label{lemma-5-forma-della-lagrangiana}

La lagrangiana può essere espressa come funzione dell'energia cinetica \(T\) e dell'energia potenziale \(U\), secondo la relazione \cite{arnold1992matematici, landau1994meccanica}:

\[
L\left( q_{i},{\dot{q}}_{i} \right) = T\left( q_{i},{\dot{q}}_{i} \right) - U\left( q_{i} \right),\quad i = 1,\ldots,N
\]

La lagrangiana \(L\left( q_{i},{\dot{q}}_{i} \right)\) è, dunque, una funzione delle coordinate generalizzate posizione e velocità, che a loro volta dipendono dal tempo:

\[L : TQ \times \mathbb{R} \rightarrow \mathbb{R}\]

Dove \(TQ\) indica il \textbf{fibrato tangente} dello spazio delle configurazioni \(Q\), dove:

\begin{itemize}
\item Un elemento di \(Q\) è semplicemente una configurazione \(\vec{q}\);
\item Un elemento di \(TQ\) è una coppia \(\left( \vec{q},\dot{\vec{q}} \right)\), ovvero traiettoria e velocità della particella.
\end{itemize}

Si considera l'azione \(S\), data per definizione da:

\[S = \int_{t_{1}}^{t_{2}}{L\left( q_{i},{\dot{q}}_{i} \right)dt}\]

L'azione dipende dalle coordinate generalizzate, funzioni del tempo, dunque, è un funzionale poiché associa a ogni funzione \(\vec{q}(t)\) un valore numerico. Dunque, l'azione è definita nello spazio vettoriale delle traiettorie generalizzate \(\mathbb{V =}\left\{ \vec{q}(t) \right\}\) a valori in \(\mathbb{R}\):

\[S:\mathbb{V \rightarrow R}\]
\subsection{Lemma 6: legame tra lagrangiana ed equazioni di Newton}\label{lemma-6-legame-tra-lagrangiana-ed-equazioni-di-newton}

Per individuare una correlazione tra la meccanica newtoniana e quella lagrangiana si scrive l'equazione di Eulero-Lagrange in coordinate cartesiane. In generale, l'equazione di Eulero-Lagrange può essere espressa come:

\[\dfrac{d}{dt}\dfrac{\partial L}{\partial{\dot{q}}_{i}} - \dfrac{\partial L}{\partial q_{i}} = 0\]

È noto che la lagrangiana è data da:

\[L\left( q_{i},{\dot{q}}_{i} \right) = T\left( q_{i},{\dot{q}}_{i} \right) - U\left( q_{i} \right)\]

Per cui, è possibile scrivere:

\[\dfrac{d}{dt}\dfrac{\partial}{\partial{\dot{q}}_{i}}\left\lbrack T\left( q_{i},{\dot{q}}_{i} \right) - U\left( q_{i} \right) \right\rbrack - \dfrac{\partial}{\partial q_{i}}\left\lbrack T\left( q_{i},{\dot{q}}_{i} \right) - U\left( q_{i} \right) \right\rbrack = 0\]

Passando alle coordinate cartesiane si ha:

\[\dfrac{\partial}{\partial q_{i}} = \dfrac{\partial x_{i}}{\partial q_{i}}\dfrac{\partial}{\partial x_{i}} = \dfrac{\partial}{\partial x_{i}}\]

Questa semplificazione è vera solo se si assume che la $i$-esima coordinata generalizzata $q_i$ coincida con la $i$-esima coordinata cartesiana $x_i$. Analogo procedimento può essere eseguito per passare da \({\dot{q}}_{i}\) a \({\dot{x}}_{i}\).

In coordinate cartesiane, l'energia cinetica dipende solamente dalla velocità \({\dot{x}}_{i}\) mentre l'energia potenziale solamente dalla posizione \(x_{i}\). L'equazione di Eulero-Lagrange è:

\[\dfrac{d}{dt}\dfrac{\partial}{\partial{\dot{x}}_{i}}\left\lbrack T\left( {\dot{x}}_{i} \right) - U\left( x_{i} \right) \right\rbrack - \dfrac{\partial}{\partial x_{i}}\left\lbrack T\left( {\dot{x}}_{i} \right) - U\left( x_{i} \right) \right\rbrack = 0 \Leftrightarrow \dfrac{d}{dt}\dfrac{\partial}{\partial{\dot{x}}_{i}}T\left( {\dot{x}}_{i} \right) + \dfrac{\partial}{\partial x_{i}}U\left( x_{i} \right) = 0\]

Da cui si ottiene:

\[\dfrac{d}{dt}\dfrac{\partial}{\partial{\dot{x}}_{i}}T\left( {\dot{x}}_{i} \right) = - \dfrac{\partial}{\partial x_{i}}U\left( x_{i} \right)\]

Siccome l'energia cinetica dipende solo dalla velocità, risulta:

\[\dfrac{\partial}{\partial x_{i}}T\left( {\dot{x}}_{i} \right) = 0\]

È possibile scrivere:

\[\dfrac{\partial}{\partial x_{i}}U\left( x_{i} \right) = \dfrac{\partial}{\partial x_{i}}\left\lbrack T\left( {\dot{x}}_{i} \right) - U\left( x_{i} \right) \right\rbrack = - \dfrac{\partial L}{\partial x_{i}}\]

L'energia cinetica per un sistema di \(N\) particelle, in coordinate cartesiane, è:

\[T = \dfrac{1}{2}\sum_{i = 1}^{N}{m_{i}{\dot{x}}_{i}^{2}}\]

Applicando la derivata rispetto a \({\dot{x}}_{i}\) si ottiene:

\[\dfrac{\partial T}{\partial{\dot{x}}_{i}} = \dfrac{\partial T}{\partial{\dot{x}}_{i}}\left( \dfrac{1}{2}\sum_{i = 1}^{N}{m_{i}{\dot{x}}_{i}^{2}} \right) = \dfrac{1}{2}\dfrac{\partial T}{\partial{\dot{x}}_{i}}\left( m_{1}{\dot{x}}_{1}^{2} + m_{2}{\dot{x}}_{2}^{2} + \ldots + m_{i}{\dot{x}}_{i}^{2} + \ldots + m_{N}{\dot{x}}_{N}^{2} \right) = m_{i}{\dot{x}}_{i}\]

Derivando tale quantità rispetto al tempo si ottiene il primo membro dell'equazione di Eulero-Lagrange in coordinate cartesiane:

\[\dfrac{d}{dt}\dfrac{\partial T}{\partial{\dot{x}}_{i}} = \dfrac{d}{dt}\left( m_{i}{\dot{x}}_{i} \right) = m_{i}{\ddot{x}}_{i}\]

Dunque, poiché:

\[\dfrac{d}{dt}\dfrac{\partial}{\partial{\dot{x}}_{i}}T\left( {\dot{x}}_{i} \right) + \dfrac{\partial}{\partial x_{i}}U\left( x_{i} \right) = 0\]

risulta:

\[m_{i}{\ddot{x}}_{i} = - \dfrac{\partial}{\partial x_{i}}U\left( x_{i} \right)\]

Dalla relazione:

\[L\left( q_{i},{\dot{q}}_{i} \right) = T\left( q_{i},{\dot{q}}_{i} \right) - U\left( q_{i} \right)\]

È possibile scrivere che:

\[\dfrac{\partial}{\partial x_{i}}U\left( x_{i} \right) = \dfrac{\partial}{\partial x_{i}}\left\lbrack T\left( {\dot{x}}_{i} \right) - L\left( x_{i},{\dot{x}}_{i} \right) \right\rbrack = - \dfrac{\partial L}{\partial x_{i}}\]

Per cui si ottiene:

\[
m_{i}{\ddot{x}}_{i} = - \dfrac{\partial}{\partial x_{i}}U\left( x_{i} \right) = - \dfrac{\partial L}{\partial x_{i}}
\]

Dal secondo principio della dinamica è noto che:

\[
m_{i}{\ddot{x}}_{i} = f_{i} = \dfrac{dp_{i}}{dt}
\]

Dove \(f_{i}\) è la forza agente mentre \(p_{i}\) la quantità di moto.

Ne discende che la quantità di moto \(p_{i}\) e la forza \(f_{i}\) sono legate all'energia potenziale \(U\) e alla lagrangiana \(L\) dalle relazioni \cite{arnold1992matematici, landau1994meccanica}:

\[
\begin{cases}
\displaystyle p_{i} = \dfrac{\partial L}{\partial{\dot{q}}_{i}} \\
\displaystyle f_{i} = - \dfrac{\partial L}{\partial q_{i}}
\end{cases}
\]

Le due relazioni sono note come definizioni fondamentali della meccanica lagrangiana. La prima è detta momento coniugato o generalizzato, mentre la seconda forza generalizzata.

Si considera l'equazione di Eulero-Lagrange e si sostituisce la relazione per il momento coniugato generalizzato:

\[\dfrac{d}{dt}\dfrac{\partial L}{\partial{\dot{q}}_{i}} - \dfrac{\partial L}{\partial q_{i}} = 0 \Leftrightarrow \dfrac{d}{dt}p_{i} = \dfrac{\partial L}{\partial q_{i}}\]

Da cui risulta:

\[{\dot{p}}_{i} = \dfrac{\partial L}{\partial q_{i}}\]

\subsection{Principio di conservazione}\label{principio-di-conservazione}

Le leggi di conservazione dell'energia, del momento lineare e del momento angolare sono una conseguenza delle simmetrie fondamentali dello spazio e del tempo.

In particolare, queste leggi di conservazione sono connesse all'invarianza delle leggi fisiche del sistema rispetto a determinate trasformazioni. Questo legame profondo fu dimostrato da Emmy Noether nel 1915 \cite{arnold1992matematici}. Secondo il suo teorema, a ogni simmetria continua e differenziabile delle leggi fisiche corrisponde una quantità conservata:

\begin{itemize}
\item
  Dall'omogeneità temporale discende la conservazione dell'energia. L'omogeneità temporale implica che le leggi fisiche non cambiano se un fenomeno viene traslato nel tempo;
\item
  Dall'omogeneità spaziale deriva la conservazione del momento lineare (o quantità di moto). L'omogeneità spaziale implica che le leggi fisiche non variano se il sistema viene traslato nello spazio;
\item
  Dall'isotropia dello spazio discende la conservazione del momento angolare. L'isotropia spaziale implica che le leggi fisiche non cambiano se il sistema viene ruotato nello spazio.
\end{itemize}

\subsection{Lemma 7: conservazione dell'energia}\label{lemma-7-conservazione-dellenergia}

L'energia totale \(E = T + U\) si conserva grazie alla proprietà di omogeneità temporale. Infatti, se il tempo è omogeneo e il sistema isolato, per definizione la lagrangiana non dipende esplicitamente dal tempo; invece, tale dipendenza è presente nelle coordinate generalizzate \cite{landau1994meccanica}:

\[L = L\left( \vec{q},\dot{\vec{q}} \right),\ \ \vec{q} = \vec{q}(t),\dot{\vec{q}} = \dot{\vec{q}}(t)\]

La derivata rispetto al tempo della lagrangiana, per la derivata delle funzioni composte, può essere espressa come:

\[\dfrac{d}{dt}L\left( \vec{q},\dot{\vec{q}} \right) = \dfrac{\partial L}{\partial\vec{q}}\dfrac{d\vec{q}}{dt} + \dfrac{\partial L}{\partial\dot{\vec{q}}}\dfrac{d\dot{\vec{q}}}{dt}\]

Dove i prodotti tra vettori sono da intenderi come prodotti scalari. Dunque, ricorrendo alla simbologia sulle derivate temporali, si può scrivere:

\[\dfrac{d}{dt}L\left( \vec{q},\dot{\vec{q}} \right) = \dfrac{\partial L}{\partial\vec{q}}\dot{\vec{q}} + \dfrac{\partial L}{\partial\dot{\vec{q}}}\vec{\ddot{q}}\]

Si considera la quantità:

\[\dfrac{d}{dt}\left( \dfrac{\partial L}{\partial\dot{\vec{q}}}\dot{\vec{q}} \right)\]

Svolgendo la derivata si ottiene:

\[\dfrac{d}{dt}\left( \dfrac{\partial L}{\partial\dot{\vec{q}}}\dot{\vec{q}} \right) = \dfrac{d}{dt}\left( \dfrac{\partial L}{\partial\dot{\vec{q}}} \right)\dot{\vec{q}} + \dfrac{\partial L}{\partial\dot{\vec{q}}}\dfrac{d\dot{\vec{q}}}{dt} = \dfrac{d}{dt}\left( \dfrac{\partial L}{\partial\dot{\vec{q}}} \right)\dot{\vec{q}} + \dfrac{\partial L}{\partial\dot{\vec{q}}}\vec{\ddot{q}}\]

Si isola il termine contenente \(\vec{\ddot{q}}\):

\[\dfrac{\partial L}{\partial\dot{\vec{q}}}\vec{\ddot{q}} = \dfrac{d}{dt}\left( \dfrac{\partial L}{\partial\dot{\vec{q}}}\dot{\vec{q}} \right) - \dfrac{d}{dt}\left( \dfrac{\partial L}{\partial\dot{\vec{q}}} \right)\dot{\vec{q}}\]

Si sostituisce questo risultato nella derivata temporale della lagrangiana:

\[\dfrac{dL}{dt} = \dfrac{\partial L}{\partial\vec{q}}\dot{\vec{q}} + \dfrac{\partial L}{\partial\dot{\vec{q}}}\vec{\ddot{q}} = \dfrac{\partial L}{\partial\vec{q}}\dot{\vec{q}} + \dfrac{d}{dt}\left( \dfrac{\partial L}{\partial\dot{\vec{q}}}\dot{\vec{q}} \right) - \dfrac{d}{dt}\left( \dfrac{\partial L}{\partial\dot{\vec{q}}} \right)\dot{\vec{q}}\]

Si porta il termine \(\dfrac{d}{dt}\left( \dfrac{\partial L}{\partial\dot{\vec{q}}}\dot{\vec{q}} \right)\) al primo membro:

\[\dfrac{dL}{dt} - \dfrac{d}{dt}\left( \dfrac{\partial L}{\partial\dot{\vec{q}}}\dot{\vec{q}} \right) = \dfrac{\partial L}{\partial\vec{q}}\dot{\vec{q}} - \dfrac{d}{dt}\left( \dfrac{\partial L}{\partial\dot{\vec{q}}} \right)\dot{\vec{q}} \Leftrightarrow \dfrac{d}{dt}\left( L - \dfrac{\partial L}{\partial\dot{\vec{q}}}\dot{\vec{q}} \right) = \left\lbrack \dfrac{\partial L}{\partial\vec{q}} - \dfrac{d}{dt}\left( \dfrac{\partial L}{\partial\dot{\vec{q}}} \right) \right\rbrack\dot{\vec{q}}\]

Per l'equazione di Eulero-Lagrange, il secondo membro è nullo:

\[\dfrac{\partial L}{\partial\vec{q}} - \dfrac{d}{dt}\left( \dfrac{\partial L}{\partial\dot{\vec{q}}} \right) = \vec{0}\]

Dunque:

\[\dfrac{d}{dt}\left( L - \dfrac{\partial L}{\partial\dot{\vec{q}}}\dot{\vec{q}} \right) = 0\]

Integrando rispetto al tempo si ottiene:

\[L - \dfrac{\partial L}{\partial\dot{\vec{q}}}\dot{\vec{q}} = cost\]

Per definizione di lagrangiana, risulta che:

\[L = T - U\]

Inoltre, l'energia potenziale non dipende dalla velocità, dunque:

\[\dfrac{\partial L}{\partial\dot{\vec{q}}} = \dfrac{\partial}{\partial\dot{\vec{q}}}(T - U) = \dfrac{\partial L}{\partial\dot{\vec{q}}}\]

Dalle relazioni tra meccanica lagrangiana e newtoniana è possibile scrivere:

\[\dfrac{\partial L}{\partial\dot{\vec{q}}} = {\vec{p}}^{T} = m{\dot{\vec{q}}}^{T}\]

Moltiplicando ambo i membri per \(\dot{\vec{q}}\), si ha:

\[\dfrac{\partial L}{\partial\dot{\vec{q}}}\dot{\vec{q}} = m{\dot{\vec{q}}}^{T}\dot{\vec{q}} = m{\dot{q}}^{2}\]

Per sistemi dove l'energia cinetica è una funzione omogenea di secondo grado rispetto alle velocità, è possibile affermare che:

\[m{\dot{q}}^{2} = 2T\]

Per cui:

\[L - \dfrac{\partial L}{\partial\dot{\vec{q}}}\dot{\vec{q}} = cost \Leftrightarrow T - U - 2T = cost\]

A meno di un segno, risulta:

\[
T + U = cost
\]

In definitiva, si è dimostrato che la conservazione dell'energia totale (T+U) è una diretta conseguenza della simmetria temporale delle leggi fisiche.

\subsection{Lemma 8: conservazione della quantità di moto}\label{lemma-8-conservazione-della-quantituxe0-di-moto}

Il momento lineare:

\[\vec{p} = \sum_{i = 1}^{N}{\vec{p}}_{i} = \sum_{i = 1}^{N}{m\vec{v}}_{i}\]

Si conserva lungo la direzione per cui l'energia potenziale resta invariata, quindi lungo traiettorie equipotenziali.

Si suppone che lo spazio sia omogeneo, dunque, una qualsiasi traslazione \(\delta\vec{r}\) del sistema non deve cambiare la lagrangiana; in altre parole, deve risultare che \(\delta L = 0\).

La variazione della lagrangiana può essere espressa come:

\[\delta L = \dfrac{\partial L}{\partial\vec{r}}\delta\vec{r}\]

Si considera l'equazione di Eulero-Lagrange \cite{landau1994meccanica}:

\[\dfrac{\partial L}{\partial\vec{q}} - \dfrac{d}{dt}\left( \dfrac{\partial L}{\partial\dot{\vec{q}}} \right) = \vec{0}\]

Passando a coordinate cartesiane, \(\vec{q}\) coincide con la posizione \(\vec{r}\), mentre \(\dot{\vec{q}}\) con la velocità \(\vec{v}\). L'equazione di Eulero-Lagrange può essere scritta come:

\[\dfrac{\partial L}{\partial\vec{r}} - \dfrac{d}{dt}\left( \dfrac{\partial L}{\partial\vec{v}} \right) = \vec{0} \Leftrightarrow \dfrac{\partial L}{\partial\vec{r}} = \dfrac{d}{dt}\left( \dfrac{\partial L}{\partial\vec{v}} \right)\]

Per la proprietà di omogeneità spaziale, la derivata rispetto alla posizione della lagrangiana è nulla:

\[\dfrac{\partial L}{\partial\vec{r}} = \vec{0}\]

Per cui risulta che:

\[\dfrac{d}{dt}\left( \dfrac{\partial L}{\partial\vec{v}} \right) = \vec{0}\]

Integrando rispetto al tempo si ottiene:

\[
\dfrac{\partial L}{\partial\vec{v}} = \vec{const}
\]

In altre parole, il gradiente della lagrangiana rispetto alla velocità è costante.

In coordinate cartesiane, è possibile scrivere che:

\[L = \dfrac{1}{2}\sum_{i = 1}^{N}{m_{i}v_{i}^{2}} - U\left( x_{1},x_{2}\ldots,x_{N} \right) = \dfrac{1}{2}\sum_{i = 1}^{N}{m_{i}{\vec{v}}_{i} \cdot {\vec{v}}_{i}} - U\left( x_{1},x_{2}\ldots,x_{N} \right)\]

Derivando rispetto a \(\vec{v}\), si ottiene:

\[
\dfrac{\partial L}{\partial\vec{v}} = \dfrac{\partial}{\partial\vec{v}}\left( \dfrac{1}{2}\sum_{i = 1}^{N}{m_{i}v_{i}^{2}} - U\left( x_{1},x_{2}\ldots,x_{N} \right) \right) = \sum_{i = 1}^{N}{m_{i}{\vec{v}}_{i}} = \sum_{i = 1}^{N}{\vec{p}}_{i} = \vec{const}
\]

La quantità di moto o momento lineare, in definitiva, si conserva in ipotesi di omogeneità spaziale.

\subsection{Lemma 9: conservazione del momento angolare}\label{lemma-9-conservazione-del-momento-angolare}

La proiezione del momento angolare \cite{landau1994meccanica}:

\[\vec{L} = \sum_{i = 1}^{N}{{\vec{r}}_{i} \times {\vec{p}}_{i}}\]

si conserva lungo direzioni in cui il potenziale \(U\) presenta delle simmetrie. Ad esempio, se il potenziale ha simmetria cilindrica, la proiezione del momento angolare lungo quest'asse si conserva.

Si suppone che lo spazio sia isotropo, dunque, una rotazione \(\delta\vec{r} = d\vec{\varphi} \times \vec{r}\) non deve modificare la lagrangiana. Il termine $\varphi$ rappresenta l'asse di una simmetria parziale del potenziale.

Anche la velocità \(\vec{v}\) subisce una rotazione \(\delta\vec{v} = d\vec{\varphi} \times \vec{v}\) dovuta a \(d\vec{\varphi}\). La variazione \(\delta L\) della lagrangiana, dovuta alla rotazione \(d\vec{\varphi}\), può essere espressa come differenziale:

\[\delta L = \dfrac{\partial L}{\partial\vec{v}} \cdot \delta\vec{v} + \dfrac{\partial L}{\partial\vec{r}} \cdot \delta\vec{r}\]

Per l'ipotesi di isotropia, la variazione \(\delta L = 0\), per cui si ha:

\[\dfrac{\partial L}{\partial\vec{v}} \cdot \delta\vec{v} + \dfrac{\partial L}{\partial\vec{r}} \cdot \delta\vec{r} = 0\]

Si sostituiscono le relazioni per le variazioni di spostamento e velocità in termini di \(d\vec{\varphi} \times \vec{r}\):

\[\delta\vec{r} = d\vec{\varphi} \times \vec{r},\ \ \delta\vec{v} = d\vec{\varphi} \times \vec{v}\]

Si ottiene:

\[\dfrac{\partial L}{\partial\vec{v}} \cdot \left( d\vec{\varphi} \times \vec{v} \right) + \dfrac{\partial L}{\partial\vec{r}} \cdot \left( d\vec{\varphi} \times \vec{r} \right) = 0\]

Esplicitando le derivate, si ha:

\[\sum_{i = 1}^{N}\left\lbrack \dfrac{\partial L}{\partial{\vec{v}}_{i}} \cdot \left( d\vec{\varphi} \times {\vec{v}}_{i} \right) + \dfrac{\partial L}{\partial{\vec{r}}_{i}} \cdot \left( d\vec{\varphi} \times {\vec{r}}_{i} \right) \right\rbrack = 0\]

Dati tre vettori generici \(\vec{a}\), \(\vec{b}\) e \(\vec{c}\), si dimostra:

\[
\left( \vec{a} \times \vec{b} \right) \cdot \vec{c} = \left( \vec{b} \times \vec{c} \right) \cdot \vec{a} = \left( \vec{c} \times \vec{a} \right) \cdot \vec{b}
\]

Applicando tale relazione, è possibile scrivere \(d\vec{\varphi}\) in prodotto scalare con l'operazione di prodotto vettoriale tra gli altri due vettori:

\[\sum_{i = 1}^{N}\left\lbrack \dfrac{\partial L}{\partial{\vec{v}}_{i}} \cdot \left( d\vec{\varphi} \times {\vec{v}}_{i} \right) + \dfrac{\partial L}{\partial{\vec{r}}_{i}} \cdot \left( d\vec{\varphi} \times {\vec{r}}_{i} \right) \right\rbrack = \sum_{i = 1}^{N}\left\lbrack d\vec{\varphi} \cdot \left( {\vec{v}}_{i} \times \dfrac{\partial L}{\partial{\vec{v}}_{i}} \right) + d\vec{\varphi} \cdot \left( {\vec{r}}_{i} \times \dfrac{\partial L}{\partial{\vec{r}}_{i}} \right) \right\rbrack = 0\]

Dall'equivalenza con la meccanica classica, è noto che:
\[
\begin{cases}
\displaystyle \dfrac{\partial L}{\partial{\vec{v}}_{i}} = {\vec{p}}_{i} \\
\displaystyle \dfrac{\partial L}{\partial{\vec{r}}_{i}} = {\vec{f}}_{i} = \displaystyle \dfrac{d{\vec{p}}_{i}}{dt}
\end{cases}
\]

Dunque, si ha:

\[\sum_{i = 1}^{N}\left( {\vec{v}}_{i} \times {\vec{p}}_{i} + {\vec{r}}_{i} \times \dfrac{d{\vec{p}}_{i}}{dt} \right) \cdot d\vec{\varphi} = 0\]

Poiché \({\vec{p}}_{i} = m_{i}{\vec{v}}_{i}\), la quantità di moto è parallela alla velocità, dunque, il loro prodotto vettorale è nullo:

\[{\vec{v}}_{i} \times {\vec{p}}_{i} = \vec{0}\]

Resta, dunque:

\[\sum_{i = 1}^{N}\left( {\vec{r}}_{i} \times \dfrac{d{\vec{p}}_{i}}{dt} \right) \cdot d\vec{\varphi} = 0\]

Si considera la quantità:

\[\dfrac{d}{dt}\left( {\vec{r}}_{i} \times {\vec{p}}_{i} \right)\]

Applicando le proprietà delle derivate, si ha:

\[\dfrac{d}{dt}\left( {\vec{r}}_{i} \times {\vec{p}}_{i} \right) = \dfrac{d{\vec{r}}_{i}}{dt} \times {\vec{p}}_{i} + {\vec{r}}_{i} \times \dfrac{d{\vec{p}}_{i}}{dt}\]

La derivata temporale della posizione coincide con la velocità istantanea:

\[\dfrac{d{\vec{r}}_{i}}{dt} = {\vec{v}}_{i}\]

Siccome \({\vec{v}}_{i} \times {\vec{p}}_{i} = \vec{0}\), risulta:

\[\dfrac{d}{dt}\left( {\vec{r}}_{i} \times {\vec{p}}_{i} \right) = {\vec{r}}_{i} \times \dfrac{d{\vec{p}}_{i}}{dt}\]

Sostituendo questo risultato nella relazione:

\[\sum_{i = 1}^{N}\left( {\vec{r}}_{i} \times \dfrac{d{\vec{p}}_{i}}{dt} \right) \cdot d\vec{\varphi} = 0\]

Si ha:

\[
\sum_{i = 1}^{N}\left( {\vec{r}}_{i} \times \dfrac{d{\vec{p}}_{i}}{dt} \right) \cdot d\vec{\varphi} = \dfrac{d}{dt}\sum_{i = 1}^{N}\left( {\vec{r}}_{i} \times {\vec{p}}_{i} \right) \cdot d\vec{\varphi} = 0
\]

Da questa relazione discende la conservazione del momento angolare lungo la direzione \(d\vec{\varphi}\) di simmetria del potenziale \(U\).
\subsection{Lagrangiana per pendolo}\label{lagrangiana-per-pendolo}

Si considera un corpo di massa \(m\) sospeso a un filo di lunghezza \(l\) nel campo gravitazionale con accelerazione \(g\).

\begin{figure}[ht]
\centering
\includegraphics[width=1.62881in,height=2.32892in,alt={Immagine che contiene schizzo, diagramma, linea, disegno Il contenuto generato dall'IA potrebbe non essere corretto.}]{media/1_Meccanica/image7.pdf}\caption{Pendolo semplice}
\end{figure}

Il pendolo possiede un solo grado di libertà, ovvero la rotazione intorno al proprio polo. Sia \(\vartheta\) l'angolo di cui la massa \(m\) è inclinata rispetto la verticale. La funzione lagrangiana data da:

\[L\left( \vartheta,\dot{\vartheta} \right) = T - U\]

Nel moto del pendolo, la velocità \(v\) è legata alla velocità angolare \(\dot{\vartheta}\), dovuta allo spostamento angolare, dalla relazione:

\[v = l\dot{\vartheta}\]

L'energia cinetica del sistema si scrive, quindi, come:

\[T = \dfrac{1}{2}m\left( l\dot{\vartheta} \right)^{2}\]

Per l'energia potenziale, \(U\), la componente verticale è data dalla proiezione della posizione della massa rispetto alla verticale, ovvero \(l\cos\vartheta\). La differenza di altezza rispetto al punto di equilibrio è data da:

\[\Delta h = l - l\cos\vartheta = l(1 - cos\vartheta)\]

Dunque, l'energia potenziale è data da:

\[U = mgl(1 - cos\vartheta)\]

Esplicitando l'energia potenziale e cinetica, la lagrangiana è data da:

\[L\left( \vartheta,\dot{\vartheta} \right) = T - U = \dfrac{1}{2}m\left( l\dot{\vartheta} \right)^{2} - mgl(1 - cos\vartheta)\]

La funzione lagrangiana deve soddisfare l'equazione di Eulero-Lagrange:

\[\dfrac{d}{dt}\dfrac{\partial L}{\partial\dot{\vartheta}} - \dfrac{\partial L}{\partial\vartheta} = 0\]

Sostituendo l'equazione ottenuta per \(L\), si ha:

\[\dfrac{d}{dt}\left( ml^{2}\dot{\vartheta} \right) + mgl\sin\vartheta = 0\]

\[ml^{2}\ddot{\vartheta} + \ mgl\sin\vartheta = 0\]

Semplificando \(m\) ed \(l\) si ha:

\[\ddot{\vartheta} + \dfrac{g}{l}\sin\vartheta = 0\]

Si definisce pulsazione naturale del sistema:

\[
\omega = \sqrt{\dfrac{g}{l}}
\]

Con questa definizione, l'equazione può essere scritta come:

\[
\ddot{\vartheta} + \omega^{2}\sin\vartheta = 0
\]

Tale equazione non ammette soluzione in forma chiusa a meno di considerare l'approssimazione per piccole oscillazioni:

\[
\vartheta \ll 1
\]

In questo caso, l'equazione si scrive come:

\[
\ddot{\vartheta} + \omega^{2}\vartheta = 0
\]

In definitiva, si ottiene l'equazione dell'oscillatore armonico.

\subsection{Lagrangiana per il doppio pendolo}\label{lagrangiana-per-il-doppio-pendolo}

Si vuole scrivere la lagrangiana per un doppio pendolo, costituito da due masse, \(m_{1}\) e \(m_{2}\) connesse tra loro. La prima massa è collegata al fulcro mediante un cavo di lunghezza \(l_{1}\); la seconda è connessa a \(m_{1}\) mediante un cavo di lunghezza \(l_{2}\)

\begin{figure}[ht]
\centering
\includegraphics[width=1.49765in,height=2.0292in,alt={Immagine che contiene diagramma, linea, design Il contenuto generato dall'IA potrebbe non essere corretto.}]{media/1_Meccanica/image8.pdf}\caption{Doppio pendolo}
\end{figure}

Si proiettano le componenti delle lunghezze \(l_{1}\) e \(l_{2}\) sugli assi cartesiani:

\[l_{1}: \begin{cases}
x_{1} = l_{1}\sin\vartheta_{1} \\
y_{1} = - l_{1}\cos\vartheta_{1}
\end{cases}\ \]

\[l_{2}: \begin{cases}
x_{2} = x_{1} + l_{2}\sin\vartheta_{2} \\
y_{2} = y_{1} - l_{2}\cos\vartheta_{2}
\end{cases}\]

Le componenti della velocità possono essere valutate, derivando rispetto al tempo le equazioni ottenute:

\[
\begin{cases}
\displaystyle{\dot{x}}_{1} = l_{1}\dfrac{d}{dt}\sin\vartheta_{1} = l_{1}\cos\vartheta_{1}\dfrac{d\vartheta_{1}}{dt} \\
\displaystyle{\dot{y}}_{1} = - l_{1}\dfrac{d}{dt}\cos\vartheta_{1} = l_{1}\sin\vartheta_{1}\dfrac{d\vartheta_{1}}{dt}
\end{cases}
\]

\[
\begin{cases}
\displaystyle{\dot{x}}_{2} = l_{1}\cos\vartheta_{1}\dfrac{d\vartheta_{1}}{dt} + l_{2}\cos\vartheta_{2}\dfrac{d\vartheta_{2}}{dt} \\
\displaystyle{\dot{y}}_{2} = l_{1}\sin\vartheta_{1}\dfrac{d\vartheta_{1}}{dt} + l_{2}\sin\vartheta_{2}\dfrac{d\vartheta_{2}}{dt}
\end{cases} \]

La configurazione del sistema può essere determinata noti i parametri \(\vartheta_{1}\) e\(\vartheta_{2}\), dunque, il sistema presenta \(2\) gradi di libertà. Infatti, rispetto al fulcro, le due masse possono ruotare relativamente, dunque, l'energia cinetica comprende sia la velocità di transizione che rotazione:

\[T = \dfrac{1}{2}m_{1}v_{1}^{2} + \dfrac{1}{2}m_{2}v_{2}^{2}\]

La velocità al quadrato della massa \(m_{1}\) è data da:

\[
v_{1}^{2} = {\dot{x}}_{1}^{2} + {\dot{y}}_{1}^{2} = \left( l_{1}\cos\vartheta_{1}\dfrac{d\vartheta_{1}}{dt} \right)^{2} + \left( l_{1}\sin\vartheta_{1}\dfrac{d\vartheta_{1}}{dt} \right) = l_{1}^{2}{\dot{\vartheta}}_{1}^{2}\left( \cos^{2}\vartheta_{1} + \sin^{2}\vartheta_{1} \right) \Leftrightarrow v_{1}^{2} = l_{1}^{2}{\dot{\vartheta}}_{1}^{2}
\]

La velocità al quadrato della massa \(m_{2}\) è data da:

\[\begin{aligned}
v_{2}^{2} & = {\dot{x}}_{2}^{2} + {\dot{y}}_{2}^{2} = \left( l_{1}\cos\vartheta_{1}\,{\dot{\vartheta}}_{1} + l_{2}\cos\vartheta_{2}\,{\dot{\vartheta}}_{2} \right)^{2} + \left( l_{1}\sin\vartheta_{1}\,{\dot{\vartheta}}_{1} + l_{2}\sin\vartheta_{2}\,{\dot{\vartheta}}_{2} \right)^{2} \\
 & = l_{1}^{2}{\dot{\vartheta}}_{1}^{2}\left( \cos^{2}\vartheta_{1} + \sin^{2}\vartheta_{1} \right) + l_{2}^{2}{\dot{\vartheta}}_{2}^{2}\left( \cos^{2}\vartheta_{2} + \sin^{2}\vartheta_{2} \right) + 2l_{1}l_{2}{\dot{\vartheta}}_{1}{\dot{\vartheta}}_{2}\left( \cos\vartheta_{1}\cos\vartheta_{2} + sin\vartheta_{1}\sin\vartheta_{2} \right) \\
 & = l_{1}^{2}{\dot{\vartheta}}_{1}^{2} + l_{2}^{2}{\dot{\vartheta}}_{2}^{2} + 2l_{1}l_{2}{\dot{\vartheta}}_{1}{\dot{\vartheta}}_{2}cos(\vartheta_{1} - \vartheta_{2})
\end{aligned}\]

Ricorrendo alle identità trigonometriche, è possibile scrivere:

\[v_{2}^{2} = l_{1}^{2}{\dot{\vartheta}}_{1}^{2} + l_{2}^{2}{\dot{\vartheta}}_{2}^{2} + 2l_{1}l_{2}{\dot{\vartheta}}_{1}{\dot{\vartheta}}_{2}\cos\left( \vartheta_{1} - \vartheta_{2} \right)\]

L'energia cinetica si scrive come:

\[T = \dfrac{1}{2}m_{1}l_{1}^{2}{\dot{\vartheta}}_{1}^{2} + \dfrac{1}{2}m_{2}\left\lbrack l_{1}^{2}{\dot{\vartheta}}_{1}^{2} + l_{2}^{2}{\dot{\vartheta}}_{2}^{2} + 2l_{1}l_{2}{\dot{\vartheta}}_{1}{\dot{\vartheta}}_{2}\cos\left( \vartheta_{1} - \vartheta_{2} \right) \right\rbrack\]

Bisogna valutare anche l'energia potenziale. Quest'ultima è data dalla somma delle energie potenziali delle due masse:

\[U = m_{1}g\Delta h_{1} + m_{2}g\Delta h_{2}\]

Dove:

\[\Delta h_{1} = l_{1} - l_{1}\cos\vartheta_{1} = l_{1}\left( 1 - cos\vartheta_{1} \right),\ \ \Delta h_{2} = l_{1} + l_{2} - l_{1}\cos\vartheta_{1} - l_{2}\cos\vartheta_{2} = l_{1}\left( 1 - cos\vartheta_{1} \right) + l_{2}\left( 1 - cos\vartheta_{2} \right)\]

Sostituendo, si ottiene:

\[U = m_{1}gl_{1}\left( 1 - cos\vartheta_{1} \right) + m_{2}gl_{1}\left( 1 - cos\vartheta_{1} \right) + m_{2}gl_{2}\left( 1 - cos\vartheta_{2} \right)\]

La lagrangiana per questo sistema è data da:

\begin{align*}
L\left( \vartheta_{1},\vartheta_{2},{\dot{\vartheta}}_{1},{\dot{\vartheta}}_{2} \right) 
&= \dfrac{1}{2}m_{1}l_{1}^{2}{\dot{\vartheta}}_{1}^{2} 
+ \dfrac{1}{2}m_{2}\left( l_{1}^{2}{\dot{\vartheta}}_{1}^{2} + l_{2}^{2}{\dot{\vartheta}}_{2}^{2} + 2l_{1}l_{2}{\dot{\vartheta}}_{1}{\dot{\vartheta}}_{2}\cos\left( \vartheta_{1} - \vartheta_{2} \right) \right) +\\
&\quad - m_{1}gl_{1}\left( 1 - \cos\vartheta_{1} \right) 
- m_{2}gl_{1}\left( 1 - \cos\vartheta_{1} \right) 
- m_{2}gl_{2}\left( 1 - \cos\vartheta_{2} \right)
\end{align*}

La funzione lagrangiana deve soddisfare l'equazione di Eulero-Lagrange:

\[\dfrac{d}{dt}\dfrac{\partial L}{\partial\dot{\vec{\vartheta}}} - \dfrac{\partial L}{\partial\vec{\vartheta}} = \vec{0}\]

L'equazione si traduce in due equazioni, relative a \(\vartheta_{1}\) e \(\vartheta_{2}\):

\[ \begin{cases}
\displaystyle \dfrac{d}{dt}\left( \dfrac{\partial L}{\partial{\dot{\vartheta}}_{1}} \right) - \dfrac{\partial L}{\partial\vartheta_{1}} = 0 \\
\displaystyle \dfrac{d}{dt}\left( \dfrac{\partial L}{\partial{\dot{\vartheta}}_{2}} \right) - \dfrac{\partial L}{\partial\vartheta_{2}} = 0
\end{cases} \]

Dove le derivate sono:

\[\begin{cases}
\displaystyle\dfrac{\partial L}{\partial{\dot{\vartheta}}_{1}} = \left( m_{1} + m_{2} \right)l_{1}^{2}{\dot{\vartheta}}_{1} + m_{2}l_{1}l_{2}{\dot{\vartheta}}_{2}\cos\left( \vartheta_{1} - \vartheta_{2} \right) \\
\displaystyle\dfrac{\partial L}{\partial{\dot{\vartheta}}_{2}} = m_{2}l_{2}^{2}{\dot{\vartheta}}_{2} + m_{2}l_{1}l_{2}{\dot{\vartheta}}_{1}\cos\left( \vartheta_{1} - \vartheta_{2} \right) \\
\displaystyle\dfrac{\partial L}{\partial\vartheta_{1}} = - m_{2}l_{1}l_{2}{\dot{\vartheta}}_{1}{\dot{\vartheta}}_{2}\sin\left( \vartheta_{1} - \vartheta_{2} \right) + \left( m_{1} + m_{2} \right)gl_{1}\sin\vartheta_{1} \\
\displaystyle\dfrac{\partial L}{\partial\vartheta_{2}} = m_{2}l_{1}l_{2}{\dot{\vartheta}}_{1}{\dot{\vartheta}}_{2}\sin\left( \vartheta_{1} - \vartheta_{2} \right) + m_{2}gl_{2}\sin\vartheta_{2}
\end{cases}\]

Eseguendo le derivate temporali e riarrangiando i termini, si ottengono le due equazioni:

\[
\begin{cases}
\left( m_{1} + m_{2} \right)l_{1}^{2}{\ddot{\vartheta}}_{1} + m_{2}l_{1}l_{2}{\ddot{\vartheta}}_{2}\cos\left( \vartheta_{1} - \vartheta_{2} \right) + m_{2}l_{1}l_{2}{\dot{\vartheta}}_{2}^{2}\sin\left( \vartheta_{1} - \vartheta_{2} \right) + \left( m_{1} + m_{2} \right)gl_{1}\sin\vartheta_{1} = 0 \\
m_{2}l_{2}{\ddot{\vartheta}}_{2} + m_{2}l_{1}l_{2}{\ddot{\vartheta}}_{1}\cos\left( \vartheta_{1} - \vartheta_{2} \right) - m_{2}l_{1}l_{2}{\dot{\vartheta}}_{1}^{2}\sin\left( \vartheta_{1} - \vartheta_{2} \right) + m_{2}gl_{2}\sin\vartheta_{2} = 0
\end{cases}
\]

Risolte le due equazioni, si ottiene la traiettoria, descritta da \(\vartheta_{1}\) e \(\vartheta_{2}\), del doppio pendolo.

\includegraphics[width=6.25in,height=2.38333in,alt={Sequenza temporale del moto del doppio pendolo in 10 istanti successivi.}]{media/1_Meccanica/image9.pdf}

La Figura 1.8 mostra una sequenza di fotogrammi che rappresentano l'evoluzione temporale del doppio pendolo. Ogni riquadro corrisponde a un istante successivo, e le posizioni delle due masse sono tracciate in base agli angoli \(\vartheta_{1}(t)\) e \(\vartheta_{2}(t)\).

Il comportamento del sistema è altamente non lineare e sensibile alle condizioni iniziali: anche piccole variazioni iniziali possono produrre traiettorie molto diverse. Questo fenomeno è noto come \textbf{caos deterministico}.

La traiettoria delle masse non segue un percorso regolare, ma mostra oscillazioni complesse e interazioni dinamiche tra i due bracci del pendolo. Questo rende il doppio pendolo un sistema ideale per lo studio della dinamica non lineare.

\section{Descrizione hamiltoniana}\label{descrizione-hamiltoniana}

La descrizione hamiltoniana privilegia le variabili momento lineare o quantità di moto \(\vec{p}\) e la posizione generalizzata della particella \(\vec{q}\) \cite{landau1994meccanica}. Questa teoria sfrutta una funzione \(H\) detta hamiltonina, data da:

\[
H\left( \vec{p},\vec{q} \right) = E = T + U
\]

Questa relazione è valida se:

\begin{itemize}
\item
 Nel sistema vi sono solo forze conservative;
\item
 La Lagrangiana non dipende esplicitamente dal tempo, dunque, il sistema è isolato;
\item
 L'energia cinetica è una funzione quadratica omogenea delle velocità generalizzate.
\end{itemize}

Privilegiando la quantità di moto e la posizione generalizzata della particella, l'approccio hamiltoniano permette di descrivere l'evoluzione del sistema nel tempo come una traiettoria nello spazio posizione-quantità di moto, detto spazio delle fasi.

\subsection{Lemma 10: equazioni di Hamilton}\label{lemma-9-equazione-di-hamilton}

Si considera il differenziale della funzione hamiltoniana:

\[dH\left( \vec{p},\vec{q} \right) = \sum_{i = 1}^{N}\left( \dfrac{\partial H}{\partial p_{i}}dp_{i} + \dfrac{\partial H}{\partial q_{i}}dq_{i} \right)\]

Si differenzia anche la lagrangiana:

\[dL\left( \vec{q},\dot{\vec{q}} \right) = \sum_{i = 1}^{N}\left( \dfrac{\partial L}{\partial q_{i}}dq_{i} + \dfrac{\partial L}{\partial{\dot{q}}_{i}}d{\dot{q}}_{i} \right)\]

Combinando l'equazione del momento coniugato generalizzato con quella di Eulero-Lagrange, si ottiene:

\[
\begin{cases}
\displaystyle p_{i} = \dfrac{\partial L}{\partial{\dot{q}}_{i}} \\
\displaystyle {\dot{p}}_{i} = \dfrac{\partial L}{\partial q_{i}}
\end{cases}
\]

Sostituendo le relazioni precedenti, il differenziale della lagrangiana può essere scritto come:

\[dL\left( \vec{q},\dot{\vec{q}} \right) = \sum_{i = 1}^{N}\left( {\dot{p}}_{i}dq_{i} + p_{i}d{\dot{q}}_{i} \right)\]

Si considera la quantità:

\[d\left( p_{i}{\dot{q}}_{i} \right) = p_{i}d{\dot{q}}_{i} + dp_{i}{\dot{q}}_{i}\]

Da cui si ottiene:

\[p_{i}d{\dot{q}}_{i} = d\left( p_{i}{\dot{q}}_{i} \right) - dp_{i}{\dot{q}}_{i}\]

Si sostituisce questo risultato nel differenziale della lagrangiana:

\[dL\left( \vec{q},\dot{\vec{q}} \right) = \sum_{i = 1}^{N}\left( {\dot{p}}_{i}dq_{i} + p_{i}d{\dot{q}}_{i} \right) = \sum_{i = 1}^{N}\left\lbrack {\dot{p}}_{i}dq_{i} + d\left( p_{i}{\dot{q}}_{i} \right) - {\dot{q}}_{i}dp_{i} \right\rbrack = \sum_{i = 1}^{N}\left( {\dot{p}}_{i}dq_{i} - {\dot{q}}_{i}dp_{i} \right) + \sum_{i = 1}^{N}{d\left( p_{i}{\dot{q}}_{i} \right)}\]

Riordinando i termini, si scrive:

\[dL - \sum_{i = 1}^{N}{d\left( p_{i}{\dot{q}}_{i} \right)} = \sum_{i = 1}^{N}\left( {\dot{p}}_{i}dq_{i} - {\dot{q}}_{i}dp_{i} \right)\]

Grazie alla proprietà di linearità del differenziale, si può scrivere:

\[d\left( L - \sum_{i = 1}^{N}{p_{i}{\dot{q}}_{i}} \right) = \sum_{i = 1}^{N}\left( {\dot{p}}_{i}dq_{i} - {\dot{q}}_{i}dp_{i} \right)\]

Moltiplicando per \(- 1\), si ottiene:

\[d\left( \sum_{i = 1}^{N}{p_{i}{\dot{q}}_{i}} - L \right) = \sum_{i = 1}^{N}\left( {\dot{q}}_{i}dp_{i} - {\dot{p}}_{i}dq_{i} \right)\]

Risulta che:

\[\sum_{i = 1}^{N}{p_{i}{\dot{q}}_{i}} = \sum_{i = 1}^{N}{m_{i}{\dot{q}}_{i}{\dot{q}}_{i}} = \sum_{i = 1}^{N}{m_{i}{\dot{q}}_{i}^{2}} = 2T\]

Di conseguenza:

\[\sum_{i = 1}^{N}{p_{i}{\dot{q}}_{i}} - L = 2T - T + U = E\]

Con questo risultato, è possibile scrivere:

\[dE = \sum_{i = 1}^{N}\left( {\dot{q}}_{i}dp_{i} - {\dot{p}}_{i}dq_{i} \right)\]

Per definizione di hamiltoniana, è possibile scrivere:

\[dE = dH = \sum_{i = 1}^{N}\left( {\dot{q}}_{i}dp_{i} - {\dot{p}}_{i}dq_{i} \right)\]

Confrontando questo risultato con il differenziale dell'hamiltoniana:

\[dH\left( \vec{p},\vec{q} \right) = \sum_{i = 1}^{N}\left( \dfrac{\partial H}{\partial p_{i}}dp_{i} + \dfrac{\partial H}{\partial q_{i}}dq_{i} \right)\]

Si ottiene:

\[
\sum_{i = 1}^{N}\left( {\dot{q}}_{i}dp_{i} - {\dot{p}}_{i}dq_{i} \right) = \sum_{i = 1}^{N}\left( \dfrac{\partial H}{\partial p_{i}}dp_{i} + \dfrac{\partial H}{\partial q_{i}}dq_{i} \right)
\]

Confrontati i coefficienti dei due polinomi, si ottengono le equazioni, note come \textbf{equazioni canoniche di Hamilton} \cite{arnold1992matematici}:

\[
\begin{cases}
\displaystyle {\dot{q}}_{i} = \dfrac{\partial H}{\partial p_{i}} \\
\displaystyle {\dot{p}}_{i} = - \dfrac{\partial H}{\partial q_{i}}
\end{cases}
\]

A meno di un segno, queste equazioni sono simmetriche rispetto a quelle relative alla lagrangiana.


\subsection{Parentesi di Poisson}\label{parentesi-di-poisson}

Si considera una qualunque grandezza \(f\), funzione delle coordinate generalizzate \(\vec{q}\) e del momento lineare \(\vec{p}\):

\[
f = f\left( \vec{q},\vec{p} \right)
\]

La sua derivata temporale è valutata mediante la proprietà delle derivate dalle funzioni composte:

\[\dfrac{df}{dt} = \sum_{i = 1}^{N}\left( \dfrac{\partial f}{\partial q_{i}}\dfrac{dq_{i}}{dt} + \dfrac{\partial f}{\partial p_{i}}\dfrac{dp_{i}}{dt} \right) = \sum_{i = 1}^{N}\left( \dfrac{\partial f}{\partial q_{i}}{\dot{q}}_{i} + \dfrac{\partial f}{\partial p_{i}}{\dot{p}}_{i} \right)\]

Per le proprietà della funzione di Hamilton:

\[\begin{cases}
\displaystyle {\dot{q}}_{i} = \dfrac{\partial H}{\partial p_{i}} \\
\displaystyle {\dot{p}}_{i} = - \dfrac{\partial H}{\partial q_{i}}
\end{cases}
\]

La derivata temporale della funzione \(f\), può essere scritta come:

\[\dfrac{df}{dt} = \sum_{i = 1}^{N}\left( \dfrac{\partial f}{\partial q_{i}}\dfrac{dq_{i}}{dt} + \dfrac{\partial f}{\partial p_{i}}\dfrac{dp_{i}}{dt} \right) = \sum_{i = 1}^{N}\left( \dfrac{\partial f}{\partial q_{i}}{\dot{q}}_{i} + \dfrac{\partial f}{\partial p_{i}}{\dot{p}}_{i} \right) = \sum_{i = 1}^{N}\left( \dfrac{\partial f}{\partial q_{i}}\dfrac{\partial H}{\partial p_{i}} - \dfrac{\partial f}{\partial p_{i}}\dfrac{\partial H}{\partial q_{i}} \right)\]

Per semplificare la notazione si introduce la **parentesi di Poisson** \cite{arnold1992matematici}:

\[
\left\{ f,H \right\} = \sum_{i = 1}^{N}\left( \dfrac{\partial f}{\partial q_{i}}\dfrac{\partial H}{\partial p_{i}} - \dfrac{\partial f}{\partial p_{i}}\dfrac{\partial H}{\partial q_{i}} \right)
\]

\subsection{Hamiltoniana per sistema con un grado di libertà}\label{hamiltoniana-per-sistema-con-un-grado-di-libertuxe0}

Si vuole valutare la funzione di Hamilton per un sistema con un grado di libertà immerso in un potenziale quadratico, come la forza di richiamo elastica. Questa condizione si applica anche nei punti di minimo del potenziale \(U\), in cui vale un'approssimazione del secondo ordine. Per definizione, l'hamiltoniana è data da:

\[H = E = T + U\]

Dove:

\[T = \dfrac{1}{2}mv^{2}\]

Si scrive l'energia cinetica \(T\) in funzione della quantità di moto. Risulta:

\[p = mv\]

Elevando al quadrato, si ottiene:

\[p^{2} = m^{2}v^{2}\]

Isolando la velocità:

\[v^{2} = \dfrac{p^{2}}{m^{2}}\]

Si sostituisce questo risultato nell'energia cinetica:

\[
T = \dfrac{1}{2}mv^{2} = \dfrac{1}{2}m\dfrac{p^{2}}{m^{2}} = \dfrac{p^{2}}{2m}
\]

Il potenziale, invece, dipende solamente dalla posizione, per cui è dato da:

\[
U = \dfrac{1}{2}kq^{2}
\]

L'hamiltoniana può essere scritta come:

\[
H = \dfrac{p^{2}}{2m} + \dfrac{1}{2}kq^{2}
\]

Si applicano le proprietà dell'hamiltoniana, dunque, si eseguono le derivate parziali:

\[
\begin{cases}
\displaystyle {\dot{q}}_{i} = \dfrac{\partial H}{\partial p_{i}} \\
\displaystyle {\dot{p}}_{i} = - \dfrac{\partial H}{\partial q_{i}}
\end{cases} \Leftrightarrow \begin{cases}
\displaystyle \dot{q} = \dfrac{p}{m} \\
\dot{p} = - kq
\end{cases}
\]

Risolvendo il sistema, si ottiene l'andamento della traiettoria generalizzata \(q\). A tale scopo si deriva la prima equazione rispetto al tempo:

\[\ddot{q} = \dfrac{\dot{p}}{m}\]

Sostituendo la seconda equazione, si ottiene l'equazione dell'oscillatore armonico, la cui soluzione è nota:

\[
\ddot{q} = - \dfrac{k}{m}q
\]

\section{Metodo di Eulero}\label{metodo-di-eulero}

Si considera il sistema di equazioni differenziali del secondo ordine, nelle funzioni incognite \(y_{1}\) e \(y_{2}\):

\[
\begin{cases}
{\ddot{y}}_{1} = f_{1}\left( {\dot{y}}_{i},{\dot{y}}_{2},y_{1},y_{2} \right) \\
{\ddot{y}}_{2} = f_{2}\left( {\dot{y}}_{i},{\dot{y}}_{2},y_{1},y_{2} \right)
\end{cases}
\]

Dove \(f_{1}\) e \(f_{2}\) sono due funzioni qualsiasi che legano la derivata seconda di una funzione incognite con le derivate prime e funzioni incognite stesse. Per rendere il sistema del primo ordine si considerano due variabili ausiliarie, \(A_{1}\) e \(A_{2}\), definite come:

\[
\begin{cases}
A_{1} = {\dot{y}}_{1} \\
A_{2} = {\dot{y}}_{2}
\end{cases}
\]

Si ottiene così un sistema del primo ordine con quattro funzioni incognite, \(A_{1}\), \(A_{2}\), \(y_{1}\) e \(y_{2}\):

\[
\begin{cases}
A_{1} = {\dot{y}}_{1} \\
A_{2} = {\dot{y}}_{2} \\
{\dot{A}}_{1} = f_{1}\left( A_{1},A_{2},y_{1},y_{2} \right) \\
{\dot{A}}_{2} = f_{2}\left( A_{1},A_{2},y_{1},y_{2} \right)
\end{cases}
\]

Il sistema può essere risolto mediante il metodo di Eulero degli elementi finiti. Si considera la prima equazione:

\[A_{1} = \dfrac{dy_{1}}{dt} \Leftrightarrow dy_{1} = A_{1}dt\]

Passando agli incrementi finiti, è possibile approssimare l'equazione:

\[
\Delta y_{1} \simeq A_{1}\Delta t
\]

Per tutte le altre equazioni è possibile procedere allo stesso modo:

\[
\begin{cases}
\Delta y_{1} \simeq A_{1}\Delta t \\
\Delta y_{2} \simeq A_{2}\Delta t \\
\Delta A_{1} \simeq f_{1}\left( A_{1},A_{2},y_{1},y_{2} \right)\Delta t \\
\Delta A_{2} \simeq f_{2}\left( A_{1},A_{2},y_{1},y_{2} \right)\Delta t
\end{cases}
\]

Dividendo l'intervallo temporale in intervalli sufficientemente piccoli è possibile determinare la soluzione approssimata del sistema. Tale metodo è in grado di fornire soluzioni valide solamente se le funzioni incognite non variano troppo rapidamente rispetto gli intervalli di tempo scelti per l'analisi \(\Delta t\). Con funzioni rapidamente variabili sono possibili anche errori importanti.

\subsection{Risoluzione sistema con MatLab}\label{risoluzione-sistema-con-matlab}

Si considera il seguente sistema di equazioni differenziali del secondo ordine, nelle funzioni incognite \(y_{1}\) e \(y_{2}\) \cite{landau1994meccanica}:

\[
\begin{cases}
{\ddot{y}}_{1} = - 5.5y_{1} + 1.1y_{2} \\
{\ddot{y}}_{2} = 1.1y_{1} - 1.2y_{2}
\end{cases}
\]

Ponendo \(z_{1} = {\dot{y}}_{1},z_{2} = {\dot{y}}_{2}\), si ottiene il sistema:

\[
\begin{cases}
{\dot{y}}_{1} = z_{1} \\
{\dot{y}}_{2} = z_{2} \\
{\dot{z}}_{1} = - 5.5y_{1} + 1.1y_{2} \\
{\dot{z}}_{2} = 1.1y_{1} - 1.2y_{2}
\end{cases}
\]

Si pone il sistema in forma matriciale:

\[
\begin{pmatrix}
{\dot{y}}_{1} \\
{\dot{y}}_{2} \\
{\dot{z}}_{1} \\
{\dot{z}}_{2}
\end{pmatrix} = \begin{pmatrix}
0 & 0 & 1 & 0 \\
0 & 0 & 0 & 1 \\
- 5.5 & 1.1 & 0 & 0 \\
1.1 & - 1.2 & 0 & 0
\end{pmatrix}\begin{pmatrix}
y_{1} \\
y_{2} \\
z_{1} \\
z_{2}
\end{pmatrix}
\]

Utilizzando le variabili ausiliarie \(z_{1}\) e \(z_{2}\), è possibile avere una matrice quadrata. Si pone:

\[\textbf{A} = \begin{pmatrix}
- 5.5 & 1.1 \\
1.1 & - 1.2
\end{pmatrix}\]

La matrice dei termini noti \(\textbf{C}\) può essere scritta come:

\[\textbf{C} = \begin{pmatrix}
0 & 0 & 1 & 0 \\
0 & 0 & 0 & 1 \\
- 5.5 & 1.1 & 0 & 0 \\
1.1 & - 1.2 & 0 & 0
\end{pmatrix} = \begin{pmatrix}
{\textbf{0}}_{2 \times 2} & {\textbf{I}}_{2 \times 2} \\
\textbf{A} & \textbf{0}_{2 \times 2}
\end{pmatrix}\]

Dove \({\textbf{I}}_{2 \times 2}\) è la matrice identità \(2 \times 2\), mentre \(\textbf{0}_{2 \times 2}\) è la matrice nulla \(2 \times 2\).

Definendo \(\vec{y}\) il vettore delle funzioni incognite:

\[\vec{y} = \begin{pmatrix}
y_{1} \\
y_{2} \\
z_{1} \\
z_{2}
\end{pmatrix}\]

Il sistema può essere scritto come:

\[
\dot{\vec{y}} = \textbf{C}\vec{y}
\]

La soluzione di questa equazione è del tipo:

\[
\vec{y} = \vec{k}\exp\left( \lambda\textbf{I}t \right)
\]

Sostituendo nel sistema, si ottiene:

\[\lambda\vec{k}\textbf{I}\exp\left( \lambda\textbf{I}t \right) = \textbf{C}\vec{k}\exp\left( \lambda\textbf{I}t \right)\]

Poiché la funzione esponenziale è sempre non nulla, esiste l'inversa a \(\vec{k}\exp\left( \lambda\textbf{I}t \right)\), si ha:

\[\lambda\textbf{I} = \textbf{C} \Leftrightarrow \textbf{C} - \lambda\textbf{I} = \textbf{0}\]

Affinché il sistema ammetta soluzioni non banali bisogna porre:

\[
\det\left( \textbf{C} - \lambda\textbf{I} \right) = 0
\]

Di conseguenza, \(\lambda\) sono gli autovalori della matrice dei coefficienti. Si calcolano, dunque, gli autovalori:

\[\det\left( \textbf{C} - \lambda\textbf{I} \right) = \begin{vmatrix}
- \lambda & 0 & 1 & 0 \\
0 & - \lambda & 0 & 1 \\
- 5.5 & 1.1 & - \lambda & 0 \\
1.1 & - 1.2 & 0 & - \lambda
\end{vmatrix} = 0\]

La cui soluzioni sono:

\[
\begin{cases}
\lambda_{1} = j2.4011 \\
\lambda_{2} = - j2.4011 \\
\lambda_{3} = j0.9669 \\
\lambda_{4} = - j0.9669
\end{cases}
\]

Di conseguenza, le soluzioni sono del tipo:

\[
y_{i} = k_{i,1}\cos\left( \omega_{i,1}t + \vartheta_{i,1} \right) + k_{i,2}\cos\left( \omega_{i,2}t + \vartheta_{i,2} \right),\ i = 1,2
\]

Dove \(k_{i,j}\) e \(\vartheta_{i,j}\), con \(i,j = 1,2\) sono costanti ottenute imponendo le condizioni iniziali; mentre \(\omega_{i,1}\) sono le pulsazioni naturali del sistema.

Per ottenere la soluzione si ricorre a MATLAB. Per prima cosa, si pulisce l'ambiente.

\begin{lstlisting}
clear all
close all
\end{lstlisting}

Si definiscono i parametri del sistema. La matrice dei coefficienti è definita come globale perché deve essere letta anche da una funzione, richiamata dalla \emph{main function}.

\begin{lstlisting}
global A
A=[-5.5 1.1;1.1 -1.2]; %matrice dei coefficienti del sistema
Ts=0.001;
t_span=0:Ts:200;
y0 = {.5,.5,0,0}'; %si usa il trasposto perche' e' necessario avere un vettore colonna
\end{lstlisting}

Si risolve il sistema mediante ode45, il risolutore di equazioni differenziali. Bisogna utilizzare una funzione che implementa il sistema di equazioni differenziali.

\begin{lstlisting}
eq=@sistema\_f;
[t,s]=ode45(eq,t\_span,y0);
a\_val=eig(A);
w=abs(a\_val);
w1=sqrt(w(1))
w2=sqrt(w(2))
\end{lstlisting}

risulta che:

\[\omega_{1} = 2.4011\]

\[\omega_{2} = 0.9669\]

Si plottano le due funzioni e i picchi spettrali.

\begin{lstlisting}
subplot(1,2,1)
plot(t,[s(:,1) s(:,2)])
subplot(1,2,2)
L = length(t);
fax = (1/Ts)*(0:L-1)/L; %Si normalizza l'asse delle frequenze
plot(fax,abs(fft(s(:,1:2))))
set(gca,'xlim',[0 1])
\end{lstlisting}

\begin{figure}[ht]
\centering
\includegraphics[width=4.93333in,height=3.95833in,alt={P754\#yIS1}]{media/1_Meccanica/image10.pdf}\caption{Andamento delle soluzioni e relativa risposta spettrale}
\end{figure}

Come si vede dalla trasformata di Fourier, le soluzioni del sistema accoppiato contengono due frequenze di oscillazione ben distinte (\(\omega_{1}\) e \(\omega_{2}\)). A causa dell'interazione tra le due funzioni, entrambe le soluzioni \(y_{1}\) e \(y_{2}\) sono una combinazione di queste due frequenze, che sono differenti dal caso non interagente.

\begin{center}
\vfill
    \chapter{Meccanica relativistica}
    \label{blx:refsection\therefsection}
\vfill

\minitoc
\newpage
\end{center}
\justify

\section{Discordanza tra Meccanica ed Elettromagnetismo}\label{discordanza-meccanica-elettromagnetismo}

Verso la fine del XIX secolo, emerse una profonda discordanza tra le leggi della \textbf{meccanica classica}, basate sulle \textbf{trasformazioni di Galileo}, e quelle dell'\textbf{elettromagnetismo}, descritte dalle equazioni di Maxwell. Le equazioni di Maxwell prevedevano che la luce, e più in generale ogni radiazione elettromagnetica, si propagasse nel vuoto con una velocità costante $c$, pari a circa $3 \times 10^8$ m/s, in \textbf{tutti i sistemi di riferimento inerziali}. Tuttavia, se si applicavano le \textbf{trasformazioni galileiane per la composizione delle velocità}, un osservatore in moto relativo $v$ avrebbe dovuto misurare una velocità $c' = c \pm v$. Questo risultato era in palese contraddizione con la previsione di Maxwell che $c$ fosse una \textbf{costante universale}, rendendo le leggi dell'elettromagnetismo non invarianti sotto trasformazioni galileiane, a differenza di quelle della meccanica.

L'\textbf{esperimento di Michelson-Morley}, condotto nel 1887, mirava originariamente a misurare la velocità della Terra rispetto al presunto \textbf{etere luminifero}. Il celebre \textbf{risultato nullo} dell'esperimento dimostrò l'impossibilità di rilevare tale moto, fornendo una cruciale evidenza sperimentale che supportò l'idea che la \textbf{velocità della luce $c$ è la stessa} in tutti i sistemi di riferimento inerziali, indipendentemente dal moto della sorgente o dell'osservatore. Questa evidenza sperimentale non era conciliabile con la fisica basata sulle trasformazioni di Galileo, dimostrando la necessità di una profonda \textbf{revisione delle fondamenta della fisica classica} e portando alla formulazione della Teoria della Relatività Ristretta.

\subsection{Trasformazioni galileiane}\label{trasformazioni-galileiane}

Per comprendere le previsioni della meccanica classica si considerino due sistemi di riferimento inerziali, $K$ e $K^'$, in moto relativo uniforme l'uno rispetto all'altro con velocità costante $\vec{v}$. Si assume che il moto del sistema $K'$ rispetto a $K$ avvenga unicamente lungo l'asse $x$, in modo che gli assi restino paralleli e che $K$ e $K'$ coincidano all'istante $t=t'=0$.

\begin{figure}[h!]
\centering
\begin{tikzpicture}[>=stealth,scale=1.2]

% Assi principali (sistema non primato)
\draw[->] (0,-0.2,0) -- (2,-0.2,0) node[below] {$x$};
\draw[->] (0,-0.2,0) -- (0,2,0) node[left] {$y$};
\draw[->] (0,-0.2,0) -- (-1,-1,0) node[left] {$z$};

% Assi primati
\draw[->] (1,0,0) -- (3,0,0) node[below] {$x'$};
\draw[->] (1,0,0) -- (1,2,0) node[left] {$y'$};
\draw[->] (1,0,0) -- (0,-1,0) node[left] {$z'$};

% Velocità V
\draw[->] (1,1.7,0) -- (2,1.7,0) node[right] {$V$};

% Punti
\node[circle,fill,inner sep=2pt] (B) at (1.3,0) {};
\node[circle,fill,inner sep=2pt] (A) at (1.8,0) {};
\node[circle,fill,inner sep=2pt] (C) at (2.3,0) {};

\node[above] at (B) {B};
\node[above] at (A) {A};
\node[above] at (C) {C};

% Frecce attorno ad A
\draw[<->] (1.3,0.45) -- (2.3,0.45);

\end{tikzpicture}
\caption{Sistemi di riferimento in moto relativo}
\label{fig:2_SistRef}
\end{figure}

L'ipotesi fondamentale della meccanica classica è il \textbf{tempo assoluto}, ovvero che il tempo scorra nello stesso modo per tutti gli osservatori inerziali ($t = t'$). In queste condizioni, le \textbf{Trasformazioni Galileiane di Posizione} si scrivono come:

\[
\begin{cases}
x = x' + vt \\
y = y' \\
z = z' \\
t = t'
\end{cases}
\]

Derivando queste relazioni rispetto al tempo, si ottengono le \textbf{Trasformazioni Galileiane di Velocità} o regola di composizione classica delle velocità:

\[
\begin{cases}
v_{x} = v'_{x} + v \\
v_{y} = v'_{y} \\
v_{z} = v'_{z}
\end{cases}
\]

Da queste relazioni, si evince che se una radiazione elettromagnetica viaggia con velocità $c$ lungo l'asse $x'$ nel sistema $K'$ (quindi $v'_{x} = c$), un osservatore nel sistema $K$ misurerebbe una velocità $v_x$ pari a $c + v$ (se $K'$ si allontana da $K$). Questa previsione, basata sul \textbf{Principio di Relatività Galileiana} e sull'ipotesi di \textbf{tempo assoluto}, è in \textbf{netto contrasto} con il risultato nullo dell'esperimento di Michelson-Morley e con la costanza della velocità della luce \cite{kittel1965meccanica}.

Questo profondo conflitto teorico-sperimentale evidenziò che l'ipotesi classica di $t=t'$ non è valida quando si considerano fenomeni ad alta velocità \cite{kittel1965meccanica}.

\section{Relatività ristretta}\label{relativita-ristretta}

Assumendo che il principio di relatività sia valido, all'inizio del '900 Einstein dimostrò che il concetto di tempo non è assoluto ma \textbf{relativo}. Egli si basò su due postulati fondamentali:

\begin{enumerate}
    \item Le leggi della fisica sono le stesse in tutti i sistemi di riferimento inerziali (Principio di Relatività).
    \item La velocità della luce nel vuoto ($c$) è la stessa in tutti i sistemi di riferimento inerziali, indipendentemente dal moto della sorgente o dell'osservatore (Principio di Invarianza di $c$).
\end{enumerate}

Si considerano due sistemi inerziali $K$ e $K'$, di cui il primo fermo, mentre il secondo in moto lungo la direzione delle $x$ positive con velocità costante $v$. Il secondo postulato di Einstein implica la non assolutezza del tempo. Per dimostrarlo, si considera l'esperimento mentale dei tre punti. Siano $A$, $B$ e $C$ tre punti sull'asse delle $x'$, con $A$ e $C$ equidistanti da $B$, sorgente luminosa. I tre punti sono solidali con il sistema di riferimento $K'$.

\begin{itemize}
    \item Per un osservatore solidale con $K'$: Egli vede i tre punti fermi. Poiché la luce procede con velocità $c$ in entrambe le direzioni e le distanze $BA$ e $BC$ sono uguali, la luce giunge \textbf{contemporaneamente} sui punti $A$ e $C$.
    \item Per un osservatore solidale con $K$: I tre punti si muovono con velocità $v$ nel verso positivo delle $x$. In particolare, $A$ si muove verso la luce emessa da $B$, mentre $C$ si allontana da essa. Poiché la velocità della luce deve essere $c$ anche in $K$, essa, dovendo percorrere cammini diversi, raggiunge prima il punto $A$ e, successivamente, il punto $C$.
\end{itemize}

Da questo esempio discende che, cambiando sistema di riferimento, gli eventi che in un sistema di riferimento $K'$ sono contemporanei, non lo sono nel sistema di riferimento $K$. Si conclude che la \textbf{simultaneità} è un concetto relativistico legato al sistema di riferimento considerato.

Per rispettare il principio di invarianza della velocità della luce, le trasformazioni tra i sistemi $K$ e $K'$ devono essere diverse da quelle galileiane. A tale scopo si supponga che un'onda luminosa sferica si propaghi partendo dall'origine dei due sistemi all'istante $t=t'=0\ \text{s}$ e si propaghi con velocità $c$. L'equazione che descrive il fronte d'onda deve essere la stessa nei due sistemi:

\[
\begin{cases}
x^{2} + y^{2} + z^{2} = r^{2} \\
{x'}^{2} + {y'}^{2} + {z'}^{2} = {r'}^{2}
\end{cases} 
\]

dove, i raggi sono dati da:

\[
r = ct, \quad \text{e} \quad r' = ct'
\]

Dato che gli assi $y$ e $z$ sono perpendicolari al moto, si ha $y=y'$ e $z=z'$. Per cui, è possibile scrivere:


\[
\begin{cases}
x^{2} + y^{2} + z^{2} = c^{2}t^{2} \\
{x'}^{2} + {y'}^{2} + {z'}^{2} = c^{2}{t'}^{2}
\end{cases} 
\]

Siccome gli eventi non sono simultanei nei due sistemi di riferimento, \(t' \neq t\).

Einstein postulò un legame di tipo lineare tra le coordinate, un'ipotesi necessaria per la ricerca della trasformazione, nella forma:

\[
\begin{cases}
x' = ax + bt \\
t' = gx + et
\end{cases}
\]

Si sostituisce queste relazioni nell'equazione che descrive l'espansione dell'onda in \(K'\):

\[
\left(ax + bt\right)^{2} + y^{2} + z^{2} = c^{2}\left(gx + et\right)^{2}
\]

Sviluppando i quadrati:

\[
a^{2}x^{2} + 2abxt + b^{2}t^{2} + y^{2} + z^{2} = c^{2}g^{2}x^{2} + 2c^{2}gext + c^{2}e^{2}t^{2}
\]

Raccogliendo, si ha:

\[
\left( a^{2} - c^{2}g^{2} \right)x^{2} + \left( 2ab - 2c^{2}ge \right)xt + y^{2} + z^{2} = \left( c^{2}e^{2} - b^{2} \right)t^{2}
\]

Questa equazione descrive l'onda nel sistema di riferimento \(K\), quindi, deve essere paragonata all'equazione:

\[
x^{2} + y^{2} + z^{2} = c^{2}t^{2}
\]

Questa equazione deve essere proporzionale a $x^{2} + y^{2} + z^{2} - c^{2}t^{2} = 0$. Uguagliando i coefficienti, si ottiene il sistema di equazioni:

\[
\begin{cases}
a^{2} - c^{2}g^{2} = 1 \quad \text{(coeff. di } x^2) \\
2ab - 2c^{2}ge = 0 \quad \text{(coeff. di } xt) \\
c^{2}e^{2} - b^{2} = c^{2} \quad \text{(coeff. di } t^2)
\end{cases}
\]

Il sistema ottenuto presenta tre equazioni, nelle incognite $a$, $b$, $e$ e $g$. È, dunque, necessario aggiungere una quarta equazione che leghi i coefficienti alla velocità relativa $v$. A tale scopo, si considera un punto fermo nel sistema $K'$, per il quale $dx' = 0$. Dalla trasformazione di $x'$, differenziando, si ha:

\[
dx' = a\ dx + b\ dt = 0 \Leftrightarrow \dfrac{dx}{dt} = - \dfrac{b}{a}
\]

La quantità $dx/dt$ è la velocità di un punto fermo in $K'$ misurata in $K$. Poiché il sistema $K'$ si muove rispetto a $K$ con velocità $v$, si deve avere:

\[
\dfrac{dx}{dt} = v
\]

Pertanto, è valida la relazione:

\[
v = - \dfrac{b}{a} \Leftrightarrow b = -av
\]

Si ottiene il sistema risolvibile:

\[
\begin{cases}
a^{2} - c^{2}g^{2} = 1 \quad \text{(Eq. 1)} \\
ab - c^{2}ge = 0 \quad \text{(Eq. 2)} \\
c^{2}e^{2} - b^{2} = c^{2} \quad \text{(Eq. 3)} \\
b = - av \quad \text{(Eq. 4)}
\end{cases}
\]

Si considera la terza equazione:

\[c^{2}e^{2} - b^{2} = c^{2}\]

Si sostituisce l'ultima:

\[c^{2}e^{2} - v^{2}e^{2} = c^{2}\]

Da cui è possibile ricavare \(e\):

\[e^{2} = \dfrac{c^{2}}{c^{2} - v^{2}} = \dfrac{1}{1 - \left( \dfrac{v}{c} \right)^{2}}\]

\[e = \dfrac{1}{\sqrt{1 - \left( \dfrac{v}{c} \right)^{2}}}\]

Noto \(e\) è possibile risalire a \(b\) dalla quarta equazione:

\[b = - ve = - \dfrac{v}{\sqrt{1 - \left( \dfrac{v}{c} \right)^{2}}}\]

Dalla seconda equazione è possibile ricavare \(g\) in funzione di \(a\):

\[c^{2}ge = ab \Leftrightarrow g = \dfrac{1}{c^{2}}a\dfrac{b}{e}\]

Per la quarta equazione, si ha:

\[g = - \dfrac{v}{c^{2}}a\]

Si sostituisce questo risultato nella prima equazione:

\[a^{2} - c^{2}g^{2} = 1 \Leftrightarrow a^{2} - c^{2}\dfrac{v^{2}}{c^{4}}a^{2} = 1 \Leftrightarrow \left( 1 - \dfrac{v^{2}}{c^{2}} \right)a^{2} = 1 \Leftrightarrow a = \dfrac{1}{\sqrt{1 - \left( \dfrac{v}{c} \right)^{2}}}\]

Noto \(a\), è possibile risalire a \(g\):

\[g = - \dfrac{v}{c^{2}}a = - \dfrac{v}{c^{2}}\dfrac{1}{\sqrt{1 - \left( \dfrac{v}{c} \right)^{2}}}\]

I coefficienti determinati sono, dunque:

\[
\begin{cases}
a = \dfrac{1}{\sqrt{1 - \left( \dfrac{v}{c} \right)^{2}}} \\
b = - \dfrac{v}{\sqrt{1 - \left( \dfrac{v}{c} \right)^{2}}} \\
g = - \dfrac{v}{c^{2}}\dfrac{1}{\sqrt{1 - \left( \dfrac{v}{c} \right)^{2}}} \\
e = \dfrac{1}{\sqrt{1 - \left( \dfrac{v}{c} \right)^{2}}}
\end{cases}
\]

Dato che il rapporto della velocità di movimento del sistema di riferimento sulla velocità della luce compare frequentemente, si pone:

\[
\dfrac{v}{c} = \beta
\]

La quantità:

\[
\gamma = \dfrac{1}{\sqrt{1 - \left( \dfrac{v}{c} \right)^{2}}} = \dfrac{1}{\sqrt{1 - \beta^{2}}}
\]

È detto \textbf{fattore di Lorentz}. Con queste definizioni, i coefficienti delle equazioni di composizione possono essere scritti come:

\[
\begin{cases}
a = \gamma \\
b = - v\gamma \\
 g = - \dfrac{v}{c^{2}}\gamma \\
e = \gamma
\end{cases} \Leftrightarrow \begin{cases}
a = \gamma \\
b = - \beta\gamma c \\
 g = - \dfrac{\beta}{c}\gamma \\
e = \gamma
\end{cases} 
\]

Le leggi di trasformazioni tra il sistema \(K\) e \(K'\) sono dette \textbf{trasformazioni di Lorentz} e sono \cite{landau1975campi,feynman1964vol1}:

\[
\begin{cases}
x' = ax + bt \\
y' = y \\
z' = z \\
t' = gx + et
\end{cases} \Leftrightarrow \begin{cases}
x' = \gamma (x - vt) \\
y' = y \\
z' = z \\
t' = \gamma \left( t - \dfrac{vx}{c^{2}} \right)
\end{cases}
\]

\subsection{Composizione delle velocità}\label{composizione-delle-velocituxe0}

Si determinano, ora, le leggi di trasformazione per la velocità; a tale scopo si differenziano le trasformazioni di Lorentz:

\[
\begin{cases}
dx' = \gamma (dx - v\,dt) \\
dy' = dy \\
dz' = dz \\
dt' = \gamma \left( dt - \dfrac{v}{c^{2}}\,dx \right)
\end{cases}
\]

Dividendo una delle variazioni spaziali per $dt'$ si ottengono le velocità lungo gli assi nel sistema $K'$.

La velocità $v_{x'}$ è data dal rapporto $dx'/dt'$. Sostituendo le espressioni differenziali:
\[
v_{x'} = \dfrac{dx'}{dt'} = \dfrac{\gamma(dx - v\,dt)}{\gamma\left( dt - \dfrac{v}{c^{2}}\,dx \right)} = \dfrac{dx - v\,dt}{dt - \dfrac{v}{c^{2}}\,dx}
\]

Al secondo membro si divide numeratore e denominatore per $dt$ per far comparire le velocità $v_x = dx/dt$:

\[
v_{x'} = \dfrac{\dfrac{dx}{dt} - v}{1 - \dfrac{v}{c^{2}}\dfrac{dx}{dt}}
\]

Posto $v_{x} = dx/dt$ come la componente di velocità lungo l'asse \(x\) con cui si muove un fenomeno nel sistema \(K\), si ottiene la formula di composizione relativistica:

\[
v_{x'} = \dfrac{v_{x} - v}{1 - \dfrac{v_{x}v}{c^{2}}}
\]

Lungo \(y'\), la velocità è data da:

\[
v_{y'} = \dfrac{dy'}{dt'} = \dfrac{dy}{\gamma\left( dt - \dfrac{v}{c^{2}}\,dx \right)}
\]

Moltiplicando e dividendo per \(dt\) al secondo membro, si ha:

\[
v_{y'} = \dfrac{\dfrac{dy}{dt}}{\gamma\left( 1 - \dfrac{v}{c^{2}}\dfrac{dx}{dt} \right)}
\]

Dove compare la componente di velocità lungo l'asse \(x\) con cui si muove un fenomeno nel sistema \(K\):

\[
\dfrac{dy}{dt} = v_{y} \Rightarrow v_{y'} = \dfrac{v_{y}}{\gamma\left( 1 - \dfrac{v_{x}v}{c^{2}} \right)}
\]

Analogamante, per la componente lungo \(z'\) si ha:

\[
v_{z'} = \dfrac{dz'}{dt'} = \dfrac{dz}{\gamma\left( dt - \dfrac{v}{c^{2}}\,dx \right)} = \dfrac{v_{z}}{\gamma\left( 1 - \dfrac{v_{x}v}{c^{2}} \right)}
\]

Si considera il caso in cui la velocità $v_{x} = c$ (un raggio di luce nel sistema $K$). Nel sistema $K'$, la velocità $v_{x'}$ diventa:

\[
v_{x'} = \left. \ \dfrac{v_{x} - v}{1 - \dfrac{v_{x}v}{c^{2}}} \right|_{v_{x} = c} = \dfrac{c - v}{1 - \dfrac{cv}{c^{2}}} = \dfrac{c - v}{1 - \dfrac{v}{c}} = \dfrac{c - v}{\dfrac{c - v}{c}} = c
\]

Le trasformazioni di Lorentz per la composizione delle velocità verificano, quindi, il \textbf{Principio di Invarianza della Velocità della Luce}, risolvendo la contraddizione presente nelle leggi di composizione galileiane.

\subsection{Fattore di Lorentz}\label{fattore-di-lorentz}

Il \textbf{fattore di Lorentz} ($\gamma$) è definito come:

\[
\gamma = \dfrac{1}{\sqrt{1 - \left( \dfrac{v}{c} \right)^{2}}}
\]

dove $v$ è la velocità relativa tra i sistemi di riferimento e $c$ è la velocità della luce.

Le velocità considerate nella maggior parte delle applicazioni pratiche sono molto minori della velocità della luce:

\[
\dfrac{v}{c} \ll 1
\]

In questo caso, il fattore $\left(v/c\right)^{2}$ è trascurabile, e il fattore di Lorentz tende all'unità:

\[
\gamma \simeq 1 \quad \text{per } v \ll c
\]

In questo limite, la relatività ristretta ricade nella \textbf{meccanica classica} (o meccanica newtoniana). La meccanica classica, in ultima analisi, è un caso particolare della meccanica relativistica di Einstein, valido quando gli effetti relativistici sono minimi.

Al crescere della velocità $v$, il fattore di Lorentz, sempre positivo, tende a crescere. Nel caso limite in cui la velocità si avvicina alla velocità della luce ($v \rightarrow c$), il denominatore tende a zero:

\[
\lim_{v \to c} \gamma = \lim_{v \to c} \dfrac{1}{\sqrt{1 - \left( \dfrac{v}{c} \right)^{2}}} \rightarrow \infty
\]

\begin{figure}[h!]
\centering
\begin{figure}[h!]
\centering
\begin{tikzpicture}

\begin{axis}[
    axis lines=middle,
    xlabel={$V$},
    ylabel={$\gamma$},
    xmin=0, xmax=1.2,
    ymin=1, ymax=6,
    xtick={0.2,0.4,0.6,0.8, 1},
    xticklabels={$0.2c$, $0.4c$, $0.6c$, $0.8c$, $c$},
    ytick={1},
    domain=0:0.99,
    samples=200,
    smooth,
    width=11cm,
    height=7cm,
    axis line style={->},
    tick style={black},
]

% Curva gamma
\addplot[black, thick] {1/sqrt(1-x^2)};

% Linea tratteggiata vicino a c
\draw[dashed] (axis cs:0.99,0) -- (axis cs:0.99,10);

\end{axis}

\end{tikzpicture}
\caption{Andamentodel fattorediLorentz al variare della velocità}
\label{fig:2_GammaFactor}
\end{figure}
\caption{Andamento del fattore di Lorentz al variare della velocità}
\label{fig:2_GammaFactor}
\end{figure}


La velocità della luce ($c$) rappresenta un \textbf{limite irraggiungibile} per qualsiasi corpo massivo secondo la meccanica relativistica.

\subsection{Dilatazione dei tempi}\label{dilatazione-dei-tempi}

Si considerano due sistemi di riferimento inerziali \(K\) e \(K'\), in moto relativo tra loro con velocità \(v\) lungo la direzione delle \(x\) positive. Si suppone che \(K\) sia fermo.

Per ricavare $dt$ in funzione di $dt'$ e $dx'$, si parte dalle Trasformazioni di Lorentz:

\[
\begin{cases}
dx' = \gamma(dx - \beta c)dt \\
dy' = dy \\
dz' = dz \\
dt' = \gamma\left( - \dfrac{\beta}{c}dx + dt \right)
\end{cases}
\]

Si considerano solamente la prima e la quarta equazione. Tramite queste si ricavano le coordinate del sistema di riferimento fisso \(K\) in funzione di quelle relative a \(K'\)

\[
\begin{cases}
dx' = \gamma(dx - \beta c)dt \\
dt' = \gamma\left( - \dfrac{\beta}{c}dx + dt \right)
\end{cases}  \Leftrightarrow \begin{cases}
\dfrac{1}{\gamma}dx' = dx - \beta c\ dt \\
\dfrac{1}{\gamma}dt' = - \dfrac{\beta}{c}\ dx + dt
\end{cases} \Leftrightarrow \begin{cases}
dx = \dfrac{1}{\gamma}dx' + \beta c\ dt \\
dt = \dfrac{1}{\gamma}dt' + \dfrac{\beta}{c}\ dx
\end{cases}
\]

Si sostituisce la prima equazione nella seconda:

\[
dt = \dfrac{1}{\gamma}dt' + \dfrac{\beta}{c}\ dx \Leftrightarrow dt = \dfrac{1}{\gamma}dt' + \dfrac{\beta}{c}\ \left( \dfrac{1}{\gamma}dx' + \beta c\ dt \right)
\]

Risolvendo, si ha:

\[
dt = \dfrac{1}{\gamma}dt' + \dfrac{\beta}{\gamma c}\ dx' + \beta^{2}dt
\]

Si portano i termini contenenti \(dt\) al primo membro:

\[
\left( 1 - \beta^{2} \right)dt = \dfrac{1}{\gamma}dt' + \dfrac{\beta}{\gamma c}\ dx'
\]

Isolando \(dt\), si ha:

\[
dt = \dfrac{1}{\gamma\left( 1 - \beta^{2} \right)}\left( dt' + \dfrac{\beta}{c}\ dx' \right)
\]

Per definizione, il fattore di Lorentz può essere scritto come:

\[
\gamma = \dfrac{1}{\sqrt{1 - \beta^{2}}}
\]

Sostituendo questo risultato nell'espressione per \(dt\) si ottiene:

\[
dt = \dfrac{\sqrt{1 - \beta^{2}}}{\left( 1 - \beta^{2} \right)}\left( dt' + \dfrac{\beta}{c}\ dx' \right) = \dfrac{1}{\sqrt{1 - \beta^{2}}}\left( dt' + \dfrac{\beta}{c}\ dx' \right) = \gamma\left( dt' + \dfrac{\beta}{c}\ dx' \right)
\]

Noto \(dt\), è possibile ricavare \(dx\) dalla prima equazione:

\[
dx = \dfrac{1}{\gamma}dx' + \beta c\ dt = \dfrac{1}{\gamma}dx' + \beta c\gamma\left( dt' + \dfrac{\beta}{c}\ dx' \right) = \dfrac{1}{\gamma}dx' + \beta c\gamma\ dt' + \beta^{2}\gamma\ dx'
\]

Si raccoglie \(\gamma\) al secondo membro:

\[
dx = \gamma\left\lbrack \left( \dfrac{1}{\gamma^{2}} + \beta^{2} \right)dx' + \beta c\ dt' \right\rbrack
\]

Dove, per definizione del fattore di Lorentz:

\[
\dfrac{1}{\gamma^{2}} + \beta^{2} = 1 - \beta^{2} + \beta^{2} = 1
\]

Con questo risultato, si ottiene l'espressione per \(dx\):

\[
dx = \gamma\left( dx' + \beta c\ dt' \right)
\]

Le due equazioni, per \(dx\) e \(dt\), sono, in definitiva:

\[
\begin{cases}
dx = \gamma\left( dx' + \beta c\ dt' \right) \\
dt = \gamma\left( dt' + \dfrac{\beta}{c}\ dx' \right)
\end{cases}
\]

Si considera un orologio o un fenomeno solidale col sistema di riferimento \(K'\), ovvero fermo rispetto a esso. Ne discende che \(dx' = 0\), in quanto la variazione lungo l'asse \(x\) è nulla, essendo, appunto, il punto fermo. Nel sistema di riferimento \(K\), la variazione temporale è data da:

\[
dt = \left. \gamma\left( dt' + \dfrac{\beta}{c}\ dx' \right) \right|_{dx' = 0} = \gamma\ dt'
\]

Si definisce tempo proprio \(\tau\) il tempo che una particela vede scorrere nel sistema di riferimento in cui è in quiete. Con questa definizione, la relazione può essere scritta come:

\[
dt = \gamma\ d\tau
\]

L'intervallo di tempo $dt$ misurato in $K$ (dove l'orologio è in moto) è maggiore dell'intervallo di tempo proprio $d\tau$ (dove l'orologio è fermo). Ciò significa che, per l'osservatore in $K$, \textbf{gli orologi in movimento scorrono più lentamente} del suo orologio. Questo è l'effetto noto come dilatazione dei tempi.

Per apprezzare tale effetto, noto come dilatazione dei tempi, il fattore di Lorentz \(\gamma\) deve essere significativamente maggiore di \(1\). Tale condizione si verifica quando la velocità con cui si muove \(K'\) rispetto a \(K\) deve essere prossima alla velocità della luce.

Per apprezzare tale effetto, il fattore di Lorentz \(\gamma\) deve essere significativamente maggiore di \(1\).

\begin{itemize}
 \item \textbf{Limite Classico} ($v \ll c$): In questo limite, il fattore di Lorentz è circa unitario ($\gamma \simeq 1$), dunque la dilatazione dei tempi non è apprezzabile:
    
 \[
 dt \simeq d\tau
 \]
 \item \textbf{Limite Relativistico} ($v \to c$): Nel caso limite in cui \(K'\) si muove alla velocità della luce, il fattore di Lorentz tende all'infinito ($\gamma \to \infty$):
 \[
 dt = \left. \ \gamma\ d\tau \right|_{v \to c} \rightarrow \infty
 \]
    Ciò implica che, per un osservatore esterno, il tempo di un oggetto che viaggia alla velocità della luce (se fosse possibile) si fermerebbe.
\end{itemize}

A differenza degli intervalli di tempo misurati nei vari sistemi di riferimento, il tempo proprio \(\tau\) è univocamente determinato noto \(\gamma\), ovvero la velocità relativa di \(K'\) rispetto al sistema di riferimento \(K\), in cui si effettua la misura.

\subsection{Contrazione delle lunghezze}\label{contrazione-delle-lunghezze}

Si considerino due sistemi di riferimento inerziali $K$ e $K'$, in moto relativo con velocità $v$ (con $\beta = v/c$) lungo la direzione positiva dell'asse $x$. Si assuma che il sistema $K$ sia fermo e si osservano due punti nel sistema $K'$ nello stesso istante; pertanto la variazione temporale è nulla: $dt' = 0$.

Nel sistema in moto $K'$ si misura una distanza $dx'$, mentre un osservatore nel sistema fisso $K$ misura una distanza $dx$, legata a $dx'$ dalle trasformazioni di Lorentz:

\[
\begin{cases}
dx = \gamma\left( dx' + \beta c\ dt' \right) \\
 dt = \gamma\left( dt' + \dfrac{\beta}{c}\ dx' \right)
\end{cases} 
\]

dove $\gamma = 1/\sqrt{1-\beta^2}$.

Per misurare la lunghezza di un oggetto in movimento, l'osservatore nel sistema $K$ deve determinare le posizioni delle sue estremità simultaneamente nel proprio sistema, cioè $dt=0$. Impostando quindi la seconda equazione a zero si ottiene:

\[
dt = \gamma\left( dt' + \dfrac{\beta}{c}\ dx' \right) \Leftrightarrow\ \gamma\left( dt' + \dfrac{\beta}{c}\ dx' \right) = 0
\]

Da cui si ricava:

\[
dt' = - \dfrac{\beta}{c}\ dx'
\]

Sostituendo questo risultato nella prima equazione:

\[
dx = \gamma\left( dx' + \beta c\, dt' \right)
     = \gamma\left( dx' + \beta c \left( - \dfrac{\beta}{c}\, dx' \right) \right)
     = \gamma\left( dx' - \beta^2 dx' \right).
\]

Raccogliendo $dx'$:

\[
dx = \gamma\, dx' (1 - \beta^2).
\]

Poiché $1 - \beta^2 = \gamma^{-2}$, si ottiene:

\[
dx = \gamma dx' \left(\dfrac{1}{\gamma^2}\right) 
\]

Semplificando $\gamma$ si ottiene la relazione che lega la lunghezza nel sistema $K$, ovvero $dx$, con quella nel sistema $K'$, $dx'$:

\[
dx = \dfrac{1}{\gamma} dx'
\]

Poiché \(\gamma > 1,v > 0\), la lunghezza misurata nel sistema $K$, in cui l'oggetto è in moto, risulta dunque minore rispetto alla misura $dx'$ ottenuta nel sistema $K'$, in cui l'oggetto è in quiete. La distanza tra due punti è massima nel sistema in cui essi sono fermi; inoltre, a differenza delle misure ottenute negli altri sistemi, essa è univocamente determinata noto $\gamma$, cioè la velocità relativa tra i sistemi.

Infine, noto $\gamma$ e $dx'$, è sempre possibile determinare la corrispondente misura $dx$ in qualsiasi sistema di riferimento inerziale.

\subsection{Energia e quantità di moto relativistiche}\label{energia-e-quantituxe0-di-moto-relativistiche}

Si ricavano le relazioni di energia e quantità di moto nella teoria della Relatività Ristretta partendo dal **Principio di Azione Stazionaria** (o di Hamilton). L'azione $S$ di una particella libera è definita come l'integrale della Lagrangiana $L$ rispetto al tempo coordinato $t$:

\[
S = \int_{t_1}^{t_2}{L dt}
\]

L'azione $S$ deve essere uno \textbf{scalare di Lorentz}, ovvero deve avere lo stesso valore in tutti i sistemi di riferimento inerziali. Affinché $S$ sia invariante, la Lagrangiana $L$ non può essere una semplice estensione della Lagrangiana classica.

Si osserva che l'elemento di linea $ds$ e il tempo proprio $d\tau$ sono invarianti di Lorentz:

\[
ds^2 = c^2 dt^2 - d\vec{x} \cdot d\vec{x} = c^2 d\tau^2
\]

Il tempo proprio $d\tau$ è legato al tempo misurato in un sistema di riferimento inerziale ($dt$) tramite il fattore di Lorentz $\gamma$:

\[
d\tau = \frac{1}{\gamma} dt = \sqrt{1 - \frac{v^{2}}{c^{2}}}dt
\]

Per garantire l'invarianza dell'azione, l'integrando deve essere una quantità proporzionale al tempo proprio, l'unico intervallo di tempo universale per la particella. Si definisce quindi l'azione come:

\[
S = \int_{\tau_1}^{\tau_2}{\alpha d\tau}
\]

dove $\alpha$ è una costante scalare da determinare.

Sostituendo l'espressione per $d\tau$, l'azione può essere riscritta in funzione del tempo coordinato $t$:

\[
S = \int_{t_1}^{t_2}{\alpha \sqrt{1 - \frac{v^{2}}{c^{2}}}dt}
\]

Da cui si ricava la Lagrangiana Relativistica $L^r$:

\[
L^{r} = \alpha \sqrt{1 - \frac{v^{2}}{c^{2}}}
\]

\[S = \int{\alpha d\tau} = \int{\alpha\sqrt{1 - \dfrac{v^{2}}{c^{2}}}dt}\]

La meccanica relativistica deve inglobare la meccanica classica, nel caso limite in cui \(v \ll c\). In questa condizione l'azione deve essere uguale a quella classica, ovvero, la quantità \(\alpha\) deve coincidere con la lagrangiana classica. Per \(v \ll c\), risulta che:

\[\sqrt{1 - \dfrac{v^{2}}{c^{2}}} \simeq \left( 1 - \dfrac{1}{2}\dfrac{v^{2}}{c^{2}} \right),\ \ v \ll c\]

Con questo risultato l'azione, nel limite classico, può essere scritta come:

\[s = \int{\alpha\sqrt{1 - \dfrac{v^{2}}{c^{2}}}dt} \simeq \int{\alpha\left( 1 - \dfrac{1}{2}\dfrac{v^{2}}{c^{2}} \right)dt}\]

Nell'approssimazione classica, l'azione è data da:

\[S = \int{L_{class}dt}\]

L'integrando coincide con la lagrangiana relativistica $L_{appr}^r$, nel limite delle meccanica classica, in cui le velocità in gioco sono molto minori della velocità della luce. Questa quantità deve coincidere con la lagrangiana classica ($L_{class}$), a meno di una costante additiva, in quanto le equazioni di Eulero-Lagrange dipendono solamente dalla derivata della lagrangiana.

Per una particella libera in assenza di un campo di potenziale, la lagrangiana coincide con l'energia cinetica, per cui l'azione è data da:

\[S = \int{L_{class}dt} = \int{\dfrac{1}{2}m_{0}v^{2}dt}\]

Dove \(m_{0}\) è la massa inerziale. Confrontando l'azione ottenuta nel limite classico, per \(v \ll c\), con quella scritta nella teoria classica, si uguagliano i termini in \(v^{2}\): poiché la parte costante non influenza le equazioni di Eulero-Lagrange. Si ottiene:

\[- \dfrac{1}{2}\alpha\dfrac{v^{2}}{c^{2}} = \dfrac{1}{2}m_{0}v^{2}\]

Da cui si ricava \(\alpha\) come:

\[\alpha = - m_{0}c^{2}\]

Con questo risultato si ha la certezza che, nel limite \(v \ll c\), la lagrangiana relativistica e quella classica coincidano, a meno di una costante \(- m_{0}c^{2}\). Infatti, risulta che:

\[
L_{appr}^r=\alpha\left( 1 - \dfrac{1}{2}\dfrac{v^{2}}{c^{2}} \right) = - m_{0}c^{2}\left( 1 - \dfrac{1}{2}\dfrac{v^{2}}{c^{2}} \right) = - m_{0}c^{2} + \dfrac{1}{2}m_{0}v^{2} = - m_{0}c^{2} + L_{class}
\]

La costante non influenza il risultato dell'equazione di Eulero-Lagrange, dunque, può essere trascurata senza problemi.

La lagrangiana deve rispettare il principio di Fermat del minor tempo, secondo il quale, tra tutti i possibili percorsi che uniscono due punti,
un raggio di luce segue il cammino che richiede il minor tempo e che, di conseguenza, rende l'azione stazionaria.

L'azione relativistica può essere espressa come:

\[s = \int{\alpha\sqrt{1 - \dfrac{v^{2}}{c^{2}}}dt} = \int{- m_{0}c^{2}\sqrt{1 - \dfrac{v^{2}}{c^{2}}}dt}\]

La lagrangiana relativistica è, dunque:

\[L^{r} = - m_{0}c^{2}\sqrt{1 - \dfrac{v^{2}}{c^{2}}} = - m_{0}c^{2}\sqrt{1 - \dfrac{\vec{v} \cdot \vec{v}}{c^{2}}}\]

Dall'equivalenza tra meccanica newtoniana e meccanica lagrangiana è possibile ricavare la definizione di momento lineare generalizzato alla meccanica relativistica:

\[\vec{p} = \dfrac{\partial L}{\partial\vec{v}} = \dfrac{\partial}{\partial\vec{v}}\left( - m_{0}c^{2}\sqrt{1 - \dfrac{\vec{v} \cdot \vec{v}}{c^{2}}} \right) = \dfrac{m_{0}\vec{v}}{\sqrt{1 - \dfrac{v^{2}}{c^{2}}}}\]

Da questa relazione è possibile osservare che la massa relativistica è legata alla massa inerziale dalla relazione:

\[m = \dfrac{m_{0}}{\sqrt{1 - \dfrac{v^{2}}{c^{2}}}} = \gamma m_{0}\]

\(m_{0}\) è detta massa a riposo e rappresenta la quantità di materia che possiede un corpo da fermo. La massa \(m\) di una particella dipende, invece, dalla velocità con cui si muove. Nel limite classico, \(v \ll c\), il fattore di Lorentz è prossimo all'unità, per cui, la massa relativistica \(m\) coincide con la massa a riposo:

\[m \simeq m_{0},\ \ v \ll c\]

Se la velocità \(v\) tende a raggiungere la velocità della luce, il fattore di Lorentz tende a diventare infinito; di conseguenza, anche la massa relativistica tende a divergere:

\[m \rightarrow \infty,\ \ v \rightarrow \infty\]

\begin{figure}[ht]
\centering
\begin{tikzpicture}

\begin{axis}[
    axis lines=middle,
    xlabel={$v$},
    ylabel={$m$},
    xmin=0, xmax=1,
    ymin=0, ymax=8.5,
    xtick={1},
    xticklabels={},
    ytick={1,2,3,4,5,6,7,8},
    yticklabels={$m_0$, $2m_0$, $3m_0$, $4m_0$, $5m_0$, $6m_0$, $7m_0$, $8m_0$},
    width=10cm,
    height=8cm,
    domain=0:0.99,
    samples=200,
    smooth,
    axis line style={->},
    tick style={black},
]

% Curva massa relativistica m = gamma m0
\addplot[red, thick] {1/sqrt(1 - x^2)};

% Linea orizzontale della massa a riposo
\addplot[red, thick] coordinates {(0,1) (1,1)};

% Linea verticale in v = c
\draw[dashed] (axis cs:1,0) -- (axis cs:1,8.5);

% Etichetta "massa relativistica"
\node at (axis cs:0.55,3.3) {\small massa relativistica};

% Etichetta "c" posizionata più in alto
\node at (axis cs:0.98,1.5) {$c$};

\end{axis}

\end{tikzpicture}


\caption{Andamento della massa relativistica in funzione della velocità}
\label{fig:2_MassRelav}
\end{figure}

La divergenza della massa relativistica al crescere della velocità spiega perché non sia possibile superare la velocità della luce \(c\). Infatti, l'energia fornita a una particella in parte ne aumenta la velocità e in parte ne accresce la massa relativistica. Di conseguenza, per superare la velocità della luce \(c\) è necessario fornire un'energia infinita, violando il principio di conservazione dell'energia.

La quantità di moto, a differenza del limite classico, è legata alla massa relativistica, che a sua volta dipende dalla velocità. Di conseguenza, la quantità di moto non dipende più in modo lineare dalla velocità:

\[\vec{p} = m\vec{v}\]

Essa presenta, invece, una relazione più complessa, poiché anche la massa dipende dalla velocità:

\[\vec{p} = \dfrac{m_{0}\vec{v}}{\sqrt{1 - \dfrac{v^{2}}{c^{2}}}}\]

La massa \(m\), nel piano \(p - v\), rappresenta la pendenza della curva quantità di moto in funzione della velocità. Nella teoria relativistica, non si ha una retta. L'andamento della quantità di moto relativistica è lineare nel limite classico, dunque per \(v \ll c\).

\begin{figure}[h!]
\centering
\begin{tikzpicture}
\begin{axis}[
    width=15cm,
    height=12cm,
    xlabel={velocità [m/s]},
    ylabel={quantità di moto [kg$\cdot$m/s]},
    xmin=0, xmax=3e8,
    ymin=0, ymax=2e-21,
    xtick={0,0.5e8,1e8,1.5e8,2e8,2.5e8,3e8},
    xticklabel style={/pgf/number format/fixed},
    yticklabel style={/pgf/number format/fixed},
    scaled y ticks=false,
    scaled x ticks=false,
    y label style={at={(axis description cs:-0.1,.5)},rotate=180},
    x label style={at={(axis description cs:0.5,-0.1)}},
    legend style={draw=none},
]

% Parametri
\def\m{9.11e-31} % massa elettrone
\def\c{3e8}      % velocità della luce

% Curva relativistica
\addplot[blue, thick, domain=0:2.99e8, samples=200] 
    {(\m*x)/sqrt(1-(x/\c)^2)};
\addlegendentry{Relativistica}

% Curva classica
\addplot[red, dashed, domain=0:3e8, samples=100] 
    {\m*x};
\addlegendentry{Approssimazione classica}

% Linea verticale c
\addplot[green, dashed] coordinates {(3e8,0) (3e8,2e-21)};

% Etichetta
\node at (axis cs:2.5e8,1.5e-22) {approssimazione classica};

\end{axis}
\end{tikzpicture}

\caption{Andamento della massa relativistica in funzione della velocità}
\label{fig:2_MassaRelativistica}
\end{figure}

Per valutare l'energia relativistica si ricorre alla definizione di lagrangiana:

\[L^{r} = T - U\]

L'energia totale del sistema è data da:

\[E = T + U\]

Nota la lagrangiana è possibile ricavare l'energia totale del sistema. Sommando, infatti, le due equazioni, si ha:

\[L^{r} + E = T - U + T + U = 2T \Leftrightarrow E = 2T - L^{r}\]

È possibile scrivere che:

\[2T = \vec{p} \cdot \vec{v} = \dfrac{\partial L}{\partial\vec{v}} \cdot \vec{v}\]

Con questo risultato l'energia totale può essere espressa come:

\[E = \dfrac{\partial L}{\partial\vec{v}} \cdot \vec{v} - L\]

Sostituendo le espressioni ricavate per la quantità di moto e lagrangiana nell'ambito della teoria relativistica, si ha:

\[E = \dfrac{\partial L}{\partial\vec{v}} \cdot \vec{v} - L = \dfrac{m_{0}\vec{v}}{\sqrt{1 - \dfrac{v^{2}}{c^{2}}}} \cdot \vec{v} + m_{0}c^{2}\sqrt{1 - \dfrac{v^{2}}{c^{2}}} = \dfrac{m_{0}v^{2}}{\sqrt{1 - \dfrac{v^{2}}{c^{2}}}} + m_{0}c^{2}\sqrt{1 - \dfrac{v^{2}}{c^{2}}}\]

Si esegue il minimo comune multiplo al secondo membro e, successivamente, si svolgono i prodotti:

\[E = \dfrac{m_{0}v^{2} + m_{0}c^{2}\left( 1 - \dfrac{v^{2}}{c^{2}} \right)}{\sqrt{1 - \dfrac{v^{2}}{c^{2}}}} = \dfrac{m_{0}v^{2} + m_{0}c^{2} - m_{0}v^{2}}{\sqrt{1 - \dfrac{v^{2}}{c^{2}}}}\]

L'energia totale è, quindi, data da:

\[E = \dfrac{m_{0}}{\sqrt{1 - \dfrac{v^{2}}{c^{2}}}}c^{2}\]

Questa relazione rappresenta una delle equazioni più note di Einstein e della relatività ristretta \cite{landau1994meccanica,feynman1964vol1}. Il termine \(m_{0}c^{2}\) rappresenta un termine energetico, legato allo stato di quiete della particella. Infatti, nel limite classico, è possibile approssimare in serie di Taylor il denominatore:

\[
\dfrac{1}{\sqrt{1 - \dfrac{v^{2}}{c^{2}}}}\simeq 1 + \dfrac{1}{2}\dfrac{v^{2}}{c^{2}}
\]

Sostituendo tale risultato nell'equazione per l'energia $E$, si ottiene:

\[
E = \dfrac{m_{0}}{\sqrt{1 - \dfrac{v^{2}}{c^{2}}}}c^{2}\simeq m_{0}c^{2}\left(1 + \dfrac{1}{2}\dfrac{v^{2}}{c^{2}}\right)
\]

Svolgendo i prodotti, si ricava l'espressione per l'energia totale nel limite classico:

\[
E\simeq m_{0}c^{2} +\dfrac{1}{2}m_{0}v^{2} ,\ \ v \ll c
\]

L'energia totale della particella è data dalla somma dell'energia a riposo ($m_{0}c^{2}$) e del termine di energia cinetica caratteristico della meccanica classica per una particella libera.

Se, invece, la velocità \(v\) approssima quella della luce, l'energia tende a divergere:

\[E \rightarrow \infty,\ \ v \rightarrow \infty\]

Questo risultato ribadisce il concetto che, per portare una particella con massa a riposo \(m_{0}\) da ferma alla velocità della luce \(c\), bisogna fornire un'energia infinita.

Anche in meccanica relativistica valgono i teoremi di conservazione, tuttavia, i principi di conservazione della massa e dell'energia sono sostituiti dal principio di conservazione della massa-energia.

\section{Quadrivettori}\label{quadrivettori}

Lo spazio-tempo o \textbf{cronotopo} è lo spazio quadridimensionale introdotto da Einstein nella relatività ristretta, composto da tre coordinate spaziali e una temporale. Ogni fenomeno fisico è descritto da eventi nello spazio-tempo del tipo \((ct,x,y,z)\), detti quadrivettori. È necessario introdurre come prima componente \(ct\) in modo da avere delle quantità dimensionalmente omogenee nel quadrivettore; inoltre, il quadrivettore è denotato con l'apice \(\alpha\):

\[s^{\alpha} = \begin{pmatrix}
ct, x, y, z
\end{pmatrix}
\]

Si definisce \textbf{intervallo spazio-temporale} $s^2$ la quantità:

\[
s^{2} = c^{2}t^{2} - \left( x^{2} + y^{2} + z^{2} \right)
\]

Il termine $x^{2} + y^{2} + z^{2}$ coincide con la distanza euclidea. L'intervallo $s^{2}$ è una quantità \text{invariante} rispetto alle trasformazioni di Lorentz \cite{landau1994meccanica}. L'invarianza dell'intervallo rispetto al quadrato della distanza euclidea è dovuta al fatto che lo spazio-tempo non è piatto, ma quadridimensionale e definito dalla \textbf{metrica di Minkowski}, non euclidea. La definizione di \(s\) è scelta in modo da ottenere equazioni simili alla meccanica classica.

Si definisce quadrivettore velocità $u^\alpha$ la derivata del quadrivettore spostamento $s^{\alpha}$ rispetto al \textbf{tempo proprio} $\tau$ della particella:

\[
{u}^{\alpha} = \dfrac{d{s}^{\alpha}}{d\tau} =
\begin{pmatrix}
 c\dfrac{dt}{d\tau},  \dfrac{dx}{d\tau},  \dfrac{dy}{d\tau},  \dfrac{dz}{d\tau}
\end{pmatrix}
\]

La variazione del \textbf{tempo proprio} $d\tau$ è legata alla variazione del tempo $dt$, osservata in un qualsiasi sistema di riferimento inerziale, dalla relazione:

\[
d\tau = \dfrac{1}{\gamma}dt \Leftrightarrow d\tau = \sqrt{1 - \dfrac{v^{2}}{c^{2}}}dt
\]

Il quadrivettore velocità, espresso in termini della velocità classica $\vec{v} = (v_x, v_y, v_z)$, è:

\[
{u}^{\alpha} = \dfrac{d{s}^{\alpha}}{d\tau} = \dfrac{1}{\sqrt{1 - \dfrac{v^{2}}{c^{2}}}}\dfrac{d{s}^{\alpha}}{dt} = \gamma\dfrac{d{s}^{\alpha}}{dt}
\]

Svolgendo l'operazione di derivata, il quadrivettore velocità si esprime come:

\[
{u}^{\alpha} = \gamma \begin{pmatrix}
 c\dfrac{dt}{dt},  \dfrac{dx}{dt},   \dfrac{dy}{dt},  \dfrac{dz}{dt}
\end{pmatrix} = \gamma \begin{pmatrix}
c, v_{x}, v_{y}, v_{z}
\end{pmatrix}
\]

In meccanica, il vettore velocità lungo i tre assi si esprime come:

\[
\vec{v} = \begin{pmatrix}
v_{x}, v_{y}, v_{z}
\end{pmatrix}
\]

Il quadrivettore velocità può essere espresso come:

\[{u}^{\alpha} = \gamma
\begin{pmatrix}
c,\vec{v}
\end{pmatrix}
\]

È possibile esprimere il quadrivettore in termini di quantità di moto ed energia. Infatti, per definizione di quantità di moto, risulta:

\[
\vec{p} = m_{0} \vec{v} \Leftrightarrow \vec{v} = \dfrac{1}{m_{0}}\vec{p}
\]

L'energia, invece, può essere espressa come:

\[
E = m_{0} c^{2} \Leftrightarrow c = \dfrac{E}{m_{0}c}
\]

Il quadrivettore velocità, espresso in termini energetici, è dato da:

\[{u}^{\alpha} = \gamma \begin{pmatrix}
 \dfrac{E}{m_{0}c}, \dfrac{1}{m_{0}}\vec{p}
\end{pmatrix}
\]

È possibile definire anche l'operazione di moltiplicazione tra quadrivettori. A tale scopo si adopera la notazione di Einstein secondo la quale ogni indice che compare all'interno di un fattore più di una volta viene sommato al variare di tutti i possibili valori che l'indice può assumere. Ad esempio, il prodotto vettoriale tra vettori  \(n\)-dimensionali, mediante la notazione di Einstein, è:

\[
\vec{x}\times\vec{y} = \sum_{i=1}^{n}{\left(\sum_{j=1}^{n}{\left(\sum_{k=0}^{n}{\left(\varepsilon_{ijk} x_{j} y_{k} \vec{e}_{i}\right)}\right)}\right)} = \varepsilon_{ijk} x^{j} y^{k} \vec{e}^{i}
\]

Dove \(\varepsilon_{ijk}\) è il simbolo di Levi-Civita e \(\vec{e}_{i}\) è la \(i\)-esimo vettore della base canonica di \(\mathbb{R}^{3}\).

Si applica tale notazione al fine di valutare il modulo del quadrivettore velocità:

\[
u_{\alpha}u^{\alpha} = \gamma^{2}\left( c^{2} - v^{2} \right) = c^{2}\gamma^{2}\left( 1 - \dfrac{v^{2}}{c^{2}} \right)
\]

Dopo aver raccolto \(c^{2}\), per definizione di \(\gamma\), è possibile scrivere:

\[
u_{\alpha}u^{\alpha} = c^{2}\gamma^{2}\gamma^{- 2} = c^{2}
\]

Il modulo quadro del quadrivettore velocità è la velocità della luce al quadrato, $c^{2}$, ed è \textbf{invariante rispetto alle trasformazioni di Lorentz}.

Si valuta il modulo quadro del quadrivettore quantità di moto:

\[
{p}^{\alpha} = m_{0}{u}^{\alpha} = m_{0}\gamma\begin{pmatrix}
c,\vec{v}
\end{pmatrix} = \begin{pmatrix}
m_{0}\gamma c,m_{0}\gamma\vec{v}
\end{pmatrix}
\]

Il termine \(m_{0}\gamma\) rappresenta la massa relativistica \(m\), dunque, la relazione \({p}^{\alpha}\) può essere scritta come:

\[
 {p}^{\alpha} = \begin{pmatrix}
m c,m\vec{v}
\end{pmatrix}
\]

Si è visto che \(E=mc^{2}\Leftrightarrow mc=E/c\) e \(\vec{p}=m\vec{v}\). Da queste relazione si evince che il quadrivettore quantità di moto è:

\[
{p}^{\alpha} = \begin{pmatrix}
 \dfrac{E}{c}, \vec{p}
\end{pmatrix}
\]

Si valuta il modulo del quadrivettore quantità di moto:

\[
p^{\alpha}p^{\alpha} = \dfrac{E^{2}}{c^{2}} - p^{2}
\]

dove \(E = \gamma m_{0}c^{2}\) e \(p = \gamma m_{0}v\), per cui:

\[
p_{\alpha}p^{\alpha} = \dfrac{E^{2}}{c^{2}} - p^{2} = \dfrac{\gamma^{2}m^{2}_{0}c^{4}}{c^{2}} - \gamma^{2}m^{2}_{0}v^{2} = \gamma^{2}m^{2}_{0}c^{2} - \gamma^{2}m^{2}_{0}v^{2}
\]

Raccogliendo \(\gamma^{2}m^{2}_{0}c^{2}\) si ha:

\[
p^{\alpha}p^{\alpha} = \gamma^{2}m^{2}_{0}c^{2}\left( 1 - \dfrac{v^{2}}{c^{2}} \right)
\]

Per definizione del fattore di Lorentz si ha:

\[
p_{\alpha}p^{\alpha} = m^{2}_{0}c^{2}
\]

Siccome \(E = m_{0}c^{2}\), la quantità di moto può essere espressa anche in termini energetici:

\[
p_{\alpha}p^{\alpha} = m_{0}E
\]

Infine, il vettore quantità di moto \(\vec{p}\) può essere espresso in termini energetici, infatti:

\[
E = m_{0}c^{2} \Leftrightarrow m_{0} = \dfrac{E}{c^{2}}
\]

Per cui:

\[
\vec{p} = m_{0}\vec{v} \Leftrightarrow \vec{p} = \dfrac{E}{c^{2}}\vec{v}
\]

Tale equazione è molto utile nel campo della medicina radiologica in cui le particelle raggiungono quasi la velocità della luce. In questo caso, la quantità di moto è data da:

\[
p = \dfrac{E}{c^{2}}c = \dfrac{E}{c}
\]

Questa relazione descrive la quantità di moto dei fotoni. 

\subsection{Legge di trasformazione dei quadrivettori}\label{legge-di-trasformazione-dei-quadrivettori}

Per comodità è possibile scrivere le trasformazioni di Lorentz in forma matriciale; così è più semplice determinare il modo in cui un quadrivettore in un sistema di riferimento \(K\) si trasforma in un quadrivettore nel sistema di riferimento \(K'\).

Si parte dalle trasformazioni di Lorentz:

\[ \begin{cases}
 t' = \gamma\left( - \dfrac{\beta}{c}x + t \right) \\
x' = \gamma(x - \beta ct) \\
y' = y \\
z' = z
\end{cases} 
\]

Si moltiplicano entrambi i membri della prima equazione, in modo da ottenere le quantità \(ct\) e \(ct'\) presenti nei quadrivettori dello spostamento:

\[
\begin{cases}
ct' = \gamma ct - \beta\gamma x \\
x' = - \gamma\beta ct + \gamma x \\
y' = y \\
z' = z
\end{cases}
\]

Si pone \({s'}^{\alpha}\) il quadrivettore spostamento nel sistema di riferimento \(K'\) e \({s}^{\alpha}\) il quadrivettore spostamento nel sistema di riferimento \(K\):

\[{s'}^{\alpha} =\begin{pmatrix}
ct' \\
x' \\
y' \\
z'
\end{pmatrix},\ \ {s}^{\alpha} = \begin{pmatrix}
ct \\
x \\
y \\
z
\end{pmatrix}
\]

La matrice di trasformazione di Lorentz è data da:

\[
\boldsymbol{\Lambda} = \begin{pmatrix}
\gamma & - \beta\gamma & 0 & 0 \\
- \beta\gamma & \gamma & 0 & 0 \\
0 & 0 & 1 & 0 \\
0 & 0 & 0 & 1
\end{pmatrix}
\]

Le trasformazioni di Lorentz in forma matriciale si scrivono come:

\[
\begin{pmatrix}
ct' \\
x' \\
y' \\
z'
\end{pmatrix} = \begin{pmatrix}
\gamma & - \beta\gamma & 0 & 0 \\
 - \beta\gamma & \gamma & 0 & 0 \\
0 & 0 & 1 & 0 \\
0 & 0 & 0 & 1
\end{pmatrix} \begin{pmatrix}
ct \\
x \\
y \\
z
\end{pmatrix}
\]

In forma compatta, si ha:

\[
{s'}^{\alpha} =\boldsymbol{\Lambda} {s}^{\alpha}
\]

Derivando rispetto al tempo si ottengono le trasformazioni della velocità per passare dal sistema \(K\) a \(K'\):

\[
\dfrac{d{{s}'}^{\alpha}}{dt} = \boldsymbol{\Lambda}\dfrac{d{s}^{\alpha}}{dt}
\]

La matrice di trasformazione di Lorentz è costante rispetto al tempo poiché la velocità relativa tra i due sistemi di riferimento è fissata, dunque, i termini \(\beta\) e \(\gamma\) sono costanti.

Si è visto che:

\[
\gamma\dfrac{d{s}^{\alpha}}{dt} = {u}^{\alpha} \Leftrightarrow \dfrac{d{s}^{\alpha}}{dt} = \dfrac{1}{\gamma}{u}^{\alpha} =\gamma\left( \dfrac{E}{m_{0}c},\dfrac{1}{m_{0}}\vec{p} \right)
\]

Per cui è possibile scrivere le equazioni di composizione della velocità in forma matriciale:

\[
\begin{pmatrix}
 \dfrac{E'}{m_{0}c} \\
 \dfrac{p_{x'}}{m_{0}} \\
 \dfrac{p_{y'}}{m_{0}} \\
 \dfrac{p_{z'}}{m_{0}}
\end{pmatrix} = \begin{pmatrix}
\gamma & - \beta\gamma & 0 & 0 \\
 - \beta\gamma & \gamma & 0 & 0 \\
0 & 0 & 1 & 0 \\
0 & 0 & 0 & 1
\end{pmatrix} \begin{pmatrix}
 \dfrac{E}{m_{0}c} \\
 \dfrac{p_{x}}{m_{0}} \\
 \dfrac{p_{y}}{m_{0}} \\
 \dfrac{p_{z}}{m_{0}}
\end{pmatrix} \Leftrightarrow  \begin{pmatrix}
 \dfrac{E'}{m_{0}c} \\
v_{x'} \\
v_{y'} \\
v_{z'}
\end{pmatrix} = \begin{pmatrix}
\gamma & - \beta\gamma & 0 & 0 \\
 - \beta\gamma & \gamma & 0 & 0 \\
0 & 0 & 1 & 0 \\
0 & 0 & 0 & 1
\end{pmatrix} \begin{pmatrix}
 \dfrac{E}{m_{0}c} \\
v_{x} \\
v_{y} \\
v_{z}
\end{pmatrix}
\]

Moltiplicando entrambi i membri per \(m_{0}\) si ottiene la trasformazione della quantità di moto:

\[
\begin{pmatrix}
 \dfrac{E'}{c} \\
p_{x'} \\
p_{y'} \\
p_{z'}
\end{pmatrix}  = \begin{pmatrix}
\gamma & - \beta\gamma & 0 & 0 \\
 - \beta\gamma & \gamma & 0 & 0 \\
0 & 0 & 1 & 0 \\
0 & 0 & 0 & 1
\end{pmatrix} \begin{pmatrix}
 \dfrac{E}{c} \\
p_{x} \\
p_{y} \\
p_{z}
\end{pmatrix}
\]

Le componenti dei quadrivettori spostamento, velocità e quantità di moto variano in base al sistema di riferimento, tuttavia, il loro quadrato è costante, dunque, è invariante rispetto alle trasformazioni di Lorentz.

\section{Urto anelastico}\label{urti-anelastico}

Un urto anelastico è un particolare tipo di urto in cui si conserva solamente la quantità di moto del sistema, mentre l'energia cinetica è non si conserva ma si trasforma in massa a riposo. Fondamentalmente le due particelle dopo l'urto si fondono in un'unica particella, come, ad esempio, due nuclei di idrogeno collidono, producendo un nucleo di elio.

Si pone l'origine del sistema di riferimento \(S\) nel centro di massa delle due particelle. Se le due particelle di masse identiche \(m\) hanno velocità uguali ed oppose, risulta che la somma delle quantità di moto deve essere nulla:

\[{\vec{p}}_{1} + {\vec{p}}_{2} = \gamma m_{0}{\vec{v}}_{1} + \gamma m_{0}{\vec{v}}_{2} = \vec{0}\]

Poiché:

\[m_{1} = m_{2} = m_{0},\ \ {\vec{v}}_{1} = - {\vec{v}}_{2}\]

\begin{figure}[ht]
\centering
\begin{tikzpicture}[>=Stealth, font=\small]

% Prima riga: particelle A e B
% Particella A
\fill[blue] (-3,1.5) circle (0.3);
\node[below=4pt] at (-3,1.4) {\textbf{A}};
\node[above=4pt] at (-3,1.6) {$m$};

% Freccia A
\draw[->, thick] (-2.6,1.5) -- (-1,1.5);
\node[above=4pt] at (-1.8,1.5) {$u'_A = u'_0$};

% Particella B
\fill[red] (3,1.5) circle (0.3);
\node[below=4pt] at (3,1.4) {\textbf{B}};
\node[above=4pt] at (3,1.6) {$m$};

% Freccia B
\draw[->, thick] (2.6,1.5) -- (1,1.5);
\node[above=4pt] at (1.8,1.5) {$u'_B = -u'_0$};

% Asse x' superiore
\draw[->, thick] (-4,0.7) -- (4,0.7) node[right] {$x'$};

% Linea separatrice
\draw[thick] (-4,0) -- (4,0);

% Seconda riga: particella C
\fill[blue] (-0.3,-1.5) circle (0.3);
\fill[red] (0.3,-1.5) circle (0.3);
\node[below=4pt] at (0,-1.6) {\textbf{C}};
\node[above=6pt] at (0,-1.4) {$M_0$};

% Velocità C
\node[right=8pt] at (0.6,-1.5) {$u'_C = 0$};

% Asse x' inferiore
\draw[->, thick] (-4,-2.3) -- (4,-2.3) node[right] {$x'$};

\end{tikzpicture}
\label{fig:2_UrtoAnelastico}
\caption{Urto anelastico tra due particelle}
\end{figure}

L'energia di una particella secondo la meccanica relativistica è data da:

\[E = \gamma m_{0}c^{2}\]

Siccome le velocità, in modulo, e le masse delle due particelle sono uguali, anche le energie coincidono, ovvero:

\[E_{1} = E_{2} = E\]

Dopo l'urto la particella risultate è in quiete nel centro di massa, per cui il fattore di Lorentz è unitario. L'energia della particella risultante, di massa a riposo \(M_{0}\), è, dunque:

\[E = M_{0}c^{2},\ \ \gamma = 1\]

Per la conservazione dell'energia, la somma delle energie delle particelle prima dell'urto deve essere uguale all'energia della particella dopo l'urto, \(E_{R}\):

\[E_{R} = E_{1} + E_{2} = 2E\]

Esplicitando le equazioni, si ha:

\[2\gamma m_{0}c^{2} = M_{0}c^{2}\]

Semplificando \(c^{2}\), si ha:

\[M_{0} = 2\gamma m_{0} = 2\dfrac{m_{0}}{\sqrt{1 - \dfrac{v^{2}}{c^{2}}}}\]

Poiché le due particelle, prima dell'urto, erano in moto, il fattore di Lorentz è maggiore dell'unità, di conseguenza, la massa \(M_{0}\) della particella risultante a valle dell'urto è maggiore della somma delle due masse a riposo iniziali. Questo fenomeno può essere spiegato ammettendo che l'energia cinetica posseduta dalle particelle prima dell'urto sia convertita in massa a riposo.

\section{Conservazione dell'energia relativistica}\label{conservazione-dellenergia-relativistica}

Il principio di conservazione dell'energia può essere verificato a partire a partire dal principio di conservazione del momento lineare e applicando le leggi di trasformazioni di Lorentz al quadrivettore energia-momento. Si considera un sistema di riferimento \(S'\) con origine non coincidente con il centro di massa del sistema. Supponendo che il moto avvenga solamente lungo una direzione, è possibile scrivere:

\[p_{x_{1}'} + p_{x_{2}'} = p_{x_{3}'}\]

Dove \(1\), \(2\) e \(3\) distinguono le tre particelle prima e dopo l'urto. Si applica la trasformazione di Lorentz per la quantità di moto:

\[
\begin{pmatrix}
 \dfrac{E'}{c} \\
p_{x'} \\
p_{y'} \\
p_{z'}
\end{pmatrix} = \begin{pmatrix}
\gamma & - \beta\gamma & 0 & 0 \\
 - \beta\gamma & \gamma & 0 & 0 \\
0 & 0 & 1 & 0 \\
0 & 0 & 0 & 1
\end{pmatrix} \begin{pmatrix}
 \dfrac{E}{c} \\
p_{x} \\
p_{y} \\
p_{z}
\end{pmatrix}
\]

Da cui si ricavano le equazioni:

\[
\begin{cases}
 \dfrac{E'}{c} = \gamma\dfrac{E}{c} - \beta\gamma p_{x} \\
 p_{x'} = - \beta\gamma\dfrac{E}{c} + p_{x}
\end{cases}
\]

Utilizzando la seconda equazione è possibile riscrivere il bilancio della quantità di modo:

\[p_{x_{1}'} + p_{x_{2}'} = p_{x_{3}'} \Leftrightarrow - \beta\gamma\dfrac{E_{1}}{c} + \gamma p_{x_{1}} - \beta\gamma\dfrac{E_{2}}{c} + \gamma p_{x_{2}} = - \beta\gamma\dfrac{E_{3}}{c} + \gamma p_{x_{3}}\]

Se la particella risultante è in quiete nel sistema \(S\), la sua quantità di moto \(p_{x_{3}}\) è nulla. Raccogliendo la primo membro si ha:

\[- \beta\gamma\dfrac{E_{1}}{c} - \beta\gamma\dfrac{E_{2}}{c} + \gamma\left( p_{x_{1}} + p_{x_{2}} \right) = - \beta\gamma\dfrac{E_{3}}{c}\]

Per il principio di identità di polinomi deve risultare che:

\[\begin{cases}
p_{x_{1}} + p_{x_{2}} = 0 \\
 - \beta\gamma\dfrac{E_{1}}{c} - \beta\gamma\dfrac{E_{2}}{c} = - \beta\gamma\dfrac{E_{3}}{c}
\end{cases} 
\]

Semplificando i termini comuni nella seconda equazione, si ha:

\[E_{1} + E_{2} = E_{3}\]

Da cui discende la conservazione dell'energia.

\section{Esempio carica in moto in campo elettrico}\label{esempio-carica-in-moto-in-campo-elettrico}

Si considera una particella carica in un campo elettrico \({\vec{E}}_{ext}\) uniforme in una regione dello spazio e costante nel tempo. La carica è soggetta alla forza di Lorentz:

\[\vec{F} = q{\vec{E}}_{ext}\]

Per la meccanica newtoniana, risulta:

\[\vec{F} = \dfrac{d\vec{p}}{dt} = \dot{\vec{p}}\]

Per cui la legge di Lorentz si scrive anche:

\[\dot{\vec{p}} = q{\vec{E}}_{ext}\]

\begin{figure}[ht]
\centering
\begin{tikzpicture}[>=Stealth, font=\large]

% Piastre
\draw[thick] (-3,-4) rectangle (-2,4);
\draw[thick] (3,-4) rectangle (2,4);

% Cariche sulle piastre
\foreach \y in {-3.5,-2.5,-1.5,-0.5,0.5,1.5,2.5,3.5} {
    \node at (-2.5,\y) {\textbf{+}};
    \node at (2.5,\y) {\textbf{-}};
}

% Linee di campo elettrico
\foreach \y in {-3.5,-2.5,-1.5,-0.5,0.5,1.5,2.5,3.5} {
    \draw[->, red, thick] (-2,\y) -- (2,\y);
}

% Etichetta E
\node[red, above] at (0,4.2) {\textbf{E}};
\draw[->, red, thick] (0,4.1) -- (0,3.7);

% Delta V
\node[above] at (0,5) {$\Delta V$};
\draw[decorate, decoration={brace, amplitude=10pt}] (-3,4.6) -- (3,4.6);

% Particella positiva
\draw[thick] (-1.5,0) circle (0.4);
\node at (-1.5,0) {\textbf{+}};
\draw[->, thick] (-1.1,0) -- (-0.3,0);
\node[above] at (-0.85,-0.5) {$\vec{v}$};

% Distanza d
\draw[<->, thick] (-1.8,-4.5) -- (1.8,-4.5);
\node[below] at (0,-4.5) {$d$};

\end{tikzpicture}

\label{fig:2_CaricaCampo}
\caption{Particella immersa in un campo elettrico costante e uniforme}
\end{figure}

Si suppone che il moto avvenga solamente lungo l'asse \(x\):

\[{\dot{p}}_{x} = qE_{ext,x}\]

Si integra tale equazione rispetto al tempo. In ipotesi di velocità iniziale nulla, si ha:

\[p_{x} = \int_{0}^{t}{qE_{ext,x}dt'} = qE_{ext,x}t\]

Nella teoria classica, la velocità della particella cresce linearmente nel tempo. Infatti, dividendo l'equazione per la quantità di moto appena ottenuta per la massa, risulta:

\[v_{x} = \dfrac{q}{m}E_{ext,x}t\]

Secondo la meccanica relativistica, la velocità non può aumentare indefinitamente, in quando la velocità della luce è un limite invalicabile. A tale scopo, si studia lo stesso fenomeno mediante un approccio relativistico. Il modulo quadro del quadrivettore quantità di moto è dato da:

\[p_{\alpha}p^{\alpha} = \dfrac{E^{2}}{c^{2}} - p_{x}^{2}\]

Dove \(p_{\alpha}p^{\alpha} = m^{2}_{0}c^{2}\), per cui si ha:

\[m^{2}_{0}c^{2} = \dfrac{E^{2}}{c^{2}} - p_{x}^{2}\]

Da questa equazione, si ricava l'energia della particella \(E\):

\[E = c\sqrt{m^{2}_{0}c^{2} + p^{2}}\]

La quantità di moto è data da \(p_{x} = qE_{ext,x}t\), per cui si ottiene:

\[E = c\sqrt{m^{2}_{0}c^{2} + q^{2}E_{ext,x}^{2}t^{2}}\]

La velocità della particella è legata all'energia dalla relazione:

\[\vec{p} = \dfrac{E}{c^{2}}\vec{v}\]

Proiettando questa equazione sull'asse del moto, si ha:

\[p_{x} = \dfrac{E}{c^{2}}v_{x} \Leftrightarrow v_{x} = \dfrac{c^{2}}{E}p_{x}\]

Sostituendo le relazioni per \(p_{x}\) e l'energia, si ricava

\[v_{x} = \dfrac{c^{2}}{E}p_{x} = \dfrac{c^{2}qE_{ext,x}t}{c\sqrt{m^{2}_{0}c^{2} + q^{2}E_{ext,x}^{2}t^{2}}} = \dfrac{cqE_{ext,x}t}{\sqrt{m^{2}_{0}c^{2} + q^{2}E_{ext,x}^{2}t^{2}}}\]

Per \(t \rightarrow \infty\) la velocità tende a quella della luce:

\[\lim_{t \rightarrow \infty}v_{x} = \lim_{t \rightarrow \infty}\dfrac{cqE_{ext,x}t}{\sqrt{m^{2}_{0}c^{2} + q^{2}E_{ext,x}^{2}t^{2}}} \rightarrow \dfrac{cqE_{ext,x}t}{qE_{ext,x}t} = c\]

Tale equazione è coerente con il principio di limite teorico invalicabile per la velocità della luce. Infatti, affinché una particella di massa a riposo \(m_{0}\) raggiunga la velocità della luce deve essere accelerata per un tempo indefinito.

Per piccoli intervalli temporali, è valida l'approssimazione classica in cui la velocità è proporzionale al tempo \(t\) mediante una costante di proporzionalità:

\[v_{x} \simeq \dfrac{q}{m}E_{ext,x}t,\ \ t \ll \dfrac{m_{0}c}{qE_{ext,x}}\]

\begin{figure}[ht]
\centering
\begin{tikzpicture}
\begin{axis}[
    width=14cm,
    height=8cm,
    xlabel={tempo [s]},
    ylabel={velocità},
    xmin=0, xmax=0.01,
    ymin=0, ymax=4e8,
    xtick={0,0.001,0.002,0.003,0.004,0.005,0.006,0.007,0.008,0.009,0.01},
    yticklabel style={/pgf/number format/fixed},
    scaled y ticks=false,
    scaled x ticks=false,
    y label style={at={(axis description cs:-0.2,.5)},rotate=180},
    x label style={at={(axis description cs:0.5,-0.1)}},
]

% Parametri
\def\c{3e8} % velocità della luce
\def\a{1.5e11} % accelerazione costante (esempio)

% Curva relativistica (approssimazione)
\addplot[blue, thick, domain=0:0.01, samples=200] 
    {\c*(1 - exp(-\a*x/\c))}; % modello esponenziale per saturazione
\addlegendentry{Relativistica}

% Curva classica
\addplot[green, dashed, domain=0:0.0025, samples=100] 
    {\a*x};
\node[above, rotate=66] at (axis cs:0.002,2.8e8) {approssimazione classica};

% Linea limite c
\addplot[red, dash dot, domain=0:0.01] {\c};
\node[above] at (axis cs:0.008,3e8) {limite c};

\end{axis}
\end{tikzpicture}

\label{fig:2_velocitàTempo}
\caption{Andamento della velocità in funzione del tempo per particella accelerata}
\end{figure}

Nel limite classico, il concetto di massa inerziale coincide con quello di massa a riposo relativistico.
\begin{center}
\vfill
    \chapter{Elettromagnetismo}
    \label{blx:refsection\therefsection}
\vfill

\minitoc
\newpage
\end{center}
\justify


\section{Cenni di elettromagnetismo}\label{cenni-di-elettromagnetismo}

Le \textbf{equazioni di Maxwell}, formalizzate verso la metà dell'Ottocento dal fisico James Clerk Maxwell, non solo sintetizzarono le leggi dell'elettricità e del magnetismo note all'epoca (come le leggi di Faraday e Ampère), ma predissero l'esistenza delle onde elettromagnetiche e dimostrarono che la luce stessa è un'onda elettromagnetica \cite{landau1975campi, purcell1985elettromagnetismo, feynman1964vol2, halliday2004fisica2}. Queste equazioni sono ancora valide in molte applicazioni pratiche moderne, con poche modifiche rispetto alla formulazione originale.

Il concetto di \textbf{spazio-tempo} (nella teoria \textbf{Relatività Ristretta}) fu introdotto da Albert Einstein nel 1905 proprio per risolvere il conflitto tra la meccanica newtoniana (che era invariante solo rispetto alle trasformazioni Galileiane) e le equazioni di Maxwell (che sono invece invarianti rispetto alle trasformazioni di Lorentz).

Il campo elettromagnetico è descritto in modo formale dai due campi vettoriali: il \textbf{campo elettrico} ($\vec{E}$) e il \textbf{campo di induzione magnetica} ($\vec{B}$). Questi due campi non sono entità separate e indipendenti, ma sono le manifestazioni di un unico tensore elettromagnetico.

Il campo magnetico ($\vec{B}$) è essenzialmente una manifestazione relativistica del campo elettrico ($\vec{E}$). Ossia, il campo magnetico osservato in un sistema di riferimento $K$ è in parte il risultato della trasformazione del campo elettrico e delle cariche in moto osservate in un altro sistema $K'$

Quando una carica $q$ si muove in una regione di spazio, essa è soggetta alla Forza di Lorentz ($\vec{F} = q(\vec{E} + \vec{v} \times \vec{B})$) \cite{landau1975campi, purcell1985elettromagnetismo, feynman1964vol2, halliday2004fisica2}. Se si considera un sistema di riferimento $K'$ solidale con la carica ($\vec{v}' = 0$), su di essa agisce solo la forza elettrica ($\vec{F}' = q\vec{E}'$). Pertanto, in tale sistema, qualsiasi forza percepita dalla carica deve essere attribuita al solo campo elettrico trasformato $\vec{E}'$, mentre il termine magnetico è nullo.

\section{Equazioni di Maxwell nel vuoto in forma locale}\label{equazioni-di-maxwell-nel-vuoto-in-forma-locale}

Sebbene sia possibile descrivere l'interazione elettromagnetica mediante l'uso del solo campo elettrico, per semplicità si ammette l'esistenza anche del campo magnetico. In quest'ottica, il campo elettrico e il campo di induzione magnetica sono due manifestazioni dello stesso campo: il campo elettromagnetico, descritto, in assenza di mezzo materiale (nel vuoto), dalle equazioni differenziali in forma locale:

\[\begin{cases}
 \vec{\nabla} \cdot \vec{E} = \dfrac{\rho}{\varepsilon_{0}} & \text{(Legge di Gauss per l'Elettricità)} \\
\vec{\nabla} \cdot \vec{B} = 0 & \text{(Legge di Gauss per il Magnetismo)} \\
 \vec{\nabla} \times \vec{E} = - \dfrac{\partial\vec{B}}{\partial t} & \text{(Legge di Faraday-Neumann-Lenz)} \\
 \vec{\nabla} \times \vec{B} = \mu_{0}\left( \vec{J} + \varepsilon_{0}\dfrac{\partial\vec{E}}{\partial t} \right) & \text{(Teorema di Ampère-Maxwell)}
\end{cases}
\]

La prima equazione è la legge di Gauss per il campo elettrico e afferma che la divergenza del campo elettrico è proporzionale alla densità di carica $\rho$ presente nel volume di riferimento, rapportata alla costante dielettrica nel vuoto \(\varepsilon_{0}\).

La seconda equazione è la legge di Gauss per il magnetismo, secondo cui la divergenza del campo di induzione magnetica $\vec{B}$ è nulla. Questa legge implica che le linee di campo magnetico non hanno sorgenti o pozzi (non esistono monopoli magnetici) e che il campo $\vec{B}$ è solenoidale. Dal punto di vista fisico, questa legge implica che le linee di campo magnetico entrano ed escono dal volumetto \(d\Omega\) su cui si calcola la divergenza, che, di conseguenza, è nulla \figurename~\ref{fig:CampSolenoidale}.

La terza equazione è la legge di Faraday-Neumann-Lenz e stabilisce che un campo di induzione magnetica variabile nel tempo ($\partial\vec{B}/\partial t$) induce un campo elettrico rotazionale ($\vec{\nabla} \times \vec{E}$). In altre parole, la variazione del flusso del campo magnetico attraverso un circuito elettrico genera una forza elettromotrice indotta e, di conseguenza, una corrente elettrica indotta

\begin{figure}[ht]
\centering
\includegraphics[width=1.90691in,height=1.76389in,alt={P1154\#yIS1}]{media/3_Elettromagnetismo/image18.pdf}\caption{Flusso campo magnetico attraverso superficie chiusa}\label{fig:CampSolenoidale}
\end{figure}

La quarta equazione è il teorema di Ampère-Maxwell, secondo cui il rotore del campo $\vec{B}$ è determinato dalla densità di corrente di conduzione ($\vec{J}$) più la corrente di spostamento ($\varepsilon_{0}\partial\vec{E}/\partial t$). La corrente di spostamento fu introdotta da Maxwell per garantire la conservazione della carica. Secondo questo teorema, dal punto di vista globale, un campo magnetico indotto intorno a un circuito chiuso qualsiasi è proporzionale alla corrente elettrica concatenata \(\mu_{0}\vec{J}\) al circuito più la corrente di spostamento (\(\mu_{0}\varepsilon_{0}\partial\vec{E}/\partial t\)) attraverso la superficie chiusa.

\subsection{Equazione di continuità della carica}
Dalle equazioni di Maxwell è possibile ricavare la legge di continuità della carica. A tale scopo si applica la divergenza all'equazione di Ampere-Maxwell:

\[
\vec{\nabla} \cdot \vec{\nabla} \times \vec{B} = \mu_{0}\vec{\nabla} \cdot \vec{J} + \mu_{0}\varepsilon_{0}\vec{\nabla} \cdot \dfrac{\partial\vec{E}}{\partial t}
\]

Il primo membro è la divergenza di un rotore che, per qualsiasi campo vettoriale derivabile due volte, è sempre nulla, ovvero \(\vec{\nabla} \cdot \vec{\nabla} \times \vec{B} = 0\). Si ottiene:

\[\mu_{0}\vec{\nabla} \cdot \vec{J} + \mu_{0}\varepsilon_{0}\vec{\nabla} \cdot \dfrac{\partial\vec{E}}{\partial t} = 0\]

È possibile scambiare il simbolo di derivata temporale con quello di divergenza:

\[\mu_{0}\vec{\nabla} \cdot \vec{J} + \mu_{0}\varepsilon_{0}\dfrac{\partial}{\partial t}\left( \vec{\nabla} \cdot \vec{E} \right) = 0\]

La divergenza del campo elettrico è data dalla prima equazione di Maxwell. Semplificando anche \(\mu_{0}\), si ottiene:

\[\vec{\nabla} \cdot \vec{J} + \varepsilon_{0}\dfrac{\partial}{\partial t}\left( \dfrac{\rho}{\varepsilon_{0}} \right) = 0\]

Dunque, si è ottenuta la legge di conservazione della carica:

\[\vec{\nabla} \cdot \vec{J} + \dfrac{\partial\rho}{\partial t} = 0\]

Tale equazione afferma che, se la carica in un volume varia nel tempo, deve esistere una corrente in ingresso o in uscita dal volume che sostenga tale variazione di carica.

\subsection{Risoluzione delle equazioni di Maxwell omogenee}\label{risoluzione-delle-equazioni-maxwell-omogenee}

Si considera il caso di assenza di sorgenti, ovvero, si vuole risolvere le equazioni di Maxwell lontano dalle sorgenti di campo \(\vec{J}\) e \(\rho\)

\[\begin{cases}
 \vec{\nabla} \cdot \vec{E} = \dfrac{\rho}{\varepsilon_{0}} \\
\vec{\nabla} \cdot \vec{B} = 0 \\
\vec{\nabla} \times \vec{E} = - \dfrac{\partial\vec{B}}{\partial t}  \\
\vec{\nabla} \times \vec{B} = \mu_{0}\left( \vec{J} + \varepsilon_{0}\dfrac{\partial\vec{E}}{\partial t} \right)
\end{cases}
\]

Si vuole risolvere il sistema di equazioni in modo da determinare il campo induzione magnetica \(\vec{B}\). A tale scopo si applica il rotore all'equazione di Ampere-Maxwell:

\[\vec{\nabla} \times \vec{\nabla} \times \vec{B} = \mu_{0}\varepsilon_{0}\vec{\nabla} \times \dfrac{\partial\vec{E}}{\partial t}\]

Il rotore del rotore può essere scritto come:

\[\vec{\nabla} \times \vec{\nabla} \times = \vec{\nabla}\left( \vec{\nabla} \cdot \  \right) - \nabla^{2}\ \]

Dove \(\nabla^{2}\) è l'operatore laplaciano che, scritto in coordinate cartesiane, è dato da:

\[\nabla^{2} = \Delta = \dfrac{\partial^{2}}{\partial x^{2}} + \dfrac{\partial^{2}}{\partial y^{2}} + \dfrac{\partial^{2}}{\partial z^{2}}\]

L'equazione di Ampere-Maxwell si può scrivere, invertendo l'operatore derivata temporale col rotore, come:

\[\vec{\nabla} \times \vec{\nabla} \times \vec{B} = \mu_{0}\varepsilon_{0}\vec{\nabla} \times \dfrac{\partial\vec{E}}{\partial t} \Leftrightarrow \vec{\nabla}\left( \vec{\nabla} \cdot \ \vec{B} \right) - \nabla^{2}\vec{B} = \mu_{0}\varepsilon_{0}\dfrac{\partial}{\partial t}\left( \vec{\nabla} \times \vec{E} \right)\]

La divergenza del campo induzione magnetica è nulla, inoltre, sostituendo la terza equazione di Maxwell si ha:

\[- \nabla^{2}\vec{B} = \mu_{0}\varepsilon_{0}\dfrac{\partial}{\partial t}\left( - \dfrac{\partial\vec{B}}{\partial t} \right) \Leftrightarrow - \nabla^{2}\vec{B} = - \ \mu_{0}\varepsilon_{0}\dfrac{\partial^{2}\vec{B}}{\partial t^{2}}\]

L'equazione:

\[\nabla^{2}\vec{B} - \ \mu_{0}\varepsilon_{0}\dfrac{\partial^{2}\vec{B}}{\partial t^{2}} = 0\]

È detta equazione delle onde o di d'Alembert. Si definisce operatore di d'Alembert come:

\[\square = \nabla^{2} - \dfrac{1}{c^{2}}\dfrac{\partial^{2}}{\partial t^{2}}\]

Dove \(c\) è la velocità di propagazione dell'onda. In questo caso, \(c\) coincide con la velocità della luce nel vuoto. Dalle equazioni di Maxwell discende che:

\[c = \dfrac{1}{\sqrt{\mu_{0}\varepsilon_{0}}} \simeq 3.00 \cdot 10^{8}\ m/s\]

\subsection{Risoluzione delle equazioni di Maxwell in presenza di sorgenti}\label{risoluzione-delle-equazioni-di-maxwell}

Si vuole risolvere le equazioni di Maxwell in prossimità delle sorgenti \(\vec{J}\) e \(\rho\). In questo modo è possibile ricostruire l'intero campo elettromagnetico generato dalle sorgenti.

\[\begin{cases}
 \vec{\nabla} \cdot \vec{E} = \dfrac{\rho}{\varepsilon_{0}} \\
\vec{\nabla} \cdot \vec{B} = 0 \\
\vec{\nabla} \times \vec{E} = - \dfrac{\partial\vec{B}}{\partial t}  \\
\vec{\nabla} \times \vec{B} = \mu_{0}\left( \vec{J} + \varepsilon_{0}\dfrac{\partial\vec{E}}{\partial t} \right)
\end{cases}
\]

Dall'equazione di Gauss per il magnetismo, risulta che il campo induzione magnetica è solenoidale, quindi, è possibile definire un potenziale vettore \(\vec{A}\) tale che:

\[\vec{\nabla} \times \vec{A} = \vec{B}\]

Si sostituisce tale definizione nella terza equazione di Maxwell:

\[\vec{\nabla} \times \vec{E} = - \dfrac{\partial\vec{B}}{\partial t} = - \dfrac{\partial}{\partial t}\left( \vec{\nabla} \times \vec{A} \right)\]

L'operatore \(\vec{\nabla}\) è indipendente dalla derivata temporale, dunque, è lecita la loro inversione:

\[\vec{\nabla} \times \vec{E} = - \vec{\nabla} \times \dfrac{\partial\vec{A}}{\partial t}\]

Portando tutto al primo membro e applicando la linearità dell'operatore rotore, è possibile scrivere:

\[\vec{\nabla} \times \left( \vec{E} + \dfrac{\partial\vec{A}}{\partial t} \right) = \vec{0}\]

Il campo \(\vec{E} + \partial\vec{A}/\partial t\) è irrotazionale, per cui, è possibile definire un potenziale scalare \(\phi\) tale che:

\[\vec{E} + \dfrac{\partial\vec{A}}{\partial t} = - \vec{\nabla}\phi\]

Il potenziale scalare $\phi$ è generalmente definito con un segno negativo. Nell'elettrostatica, il potenziale scalare concide con la tensione $\vec{E} = -\vec{\nabla}V$.

In presenza di un campo magnetostatico, la derivata temporale di \(\vec{B}\) è nulla, per cui il potenziale scalare coincide con il potenziale elettrico:

\[\vec{E} = - \vec{\nabla}\phi\]

La forza elettrica ($\vec{F}$) deve essere opposta al gradiente del potenziale scalare ($\phi$) perché il potenziale $\phi$ rappresenta l'energia potenziale elettrica per unità di carica (ossia, la tensione o voltaggio).In fisica, una forza conservativa (come la forza elettrica) tende sempre a muovere un oggetto nella direzione in cui l'energia potenziale diminuisce più rapidamente. Il gradiente ($\vec{\nabla}\phi$) punta, per definizione, nella direzione in cui il potenziale aumenta più rapidamente. Per questo motivo, la relazione fondamentale tra campo elettrico ($\vec{E}$) e potenziale elettrico ($\phi$) deve includere un segno negativo

 potenziali vettore $\vec{A}$ e scalare $\phi$ non sono univocamente determinati. Per tale motivo, si adotta una condizione di Gauge per disaccoppiare le equazioni differenziali. La Gauge di Lorentz, data da:

\[\vec{\nabla} \cdot \vec{A} + \mu_0\varepsilon_{0}\dfrac{\partial\phi}{\partial t} = 0\]

è invariante quando si passa da un sistema di riferimento inerziale a un altro tramite le trasformazioni di Lorentz,il che è un requisito fondamentale della Relatività Ristretta. Per tale motivo si parla di condizione covariante in relatività.

Sotto la condizione di Gauge di Lorentz, le Equazioni di Maxwell si riducono a due equazioni d'onda disaccoppiate per $\vec{A}$ e $\phi$ con termini sorgente. La soluzione di queste equazioni porta ai potenziali ritardati.

Si dimostra che, per una sorgente puntiforme, il potenziale vettore, nel dominio della frequenza, è dato da:

\[\vec{A} = \dfrac{\mu_{0}}{4\pi}\dfrac{\vec{J}}{\left| \vec{r} - {\vec{r}}_{0} \right|}e^{- jk\left| \vec{r} - {\vec{r}}_{0} \right|}\]

Dove \({\vec{r}}_{0}\) è la posizione in cui sono posizionate le sorgenti, mentre \(\vec{r}\) è il punto nello spazio in cui si valuta il campo. La quantità:

\[g\left( \vec{r} \right) = \dfrac{1}{4\pi}\dfrac{1}{\left| \vec{r} \right|}e^{- jk\left| \vec{r} \right|}\]

È la funzione di Green o risposta impulsiva del mezzo, lineare e isotropo, visto come un sistema ingresso-uscita. Si definisce il numero d'onda (o costante di propagazione nel vuoto) come:

\[k = \omega\sqrt{\mu_{0}\varepsilon_{0}}=\dfrac{\omega}{c}\]

\subsection{Equazioni di Maxwell nel vuoto in forma globale}\label{equazioni-di-maxwell-nel-vuoto-in-forma-globale}

Le equazioni di Maxwell in forma locale non possono essere utilizzate in presenza di brusche variazioni della superficie dei volumi sui quali calcolare i campi. In questo caso si ricorre alla forma globale o integrale:

\[\begin{cases}
 \oiint_{S}{\vec{E} \cdot d\vec{S}\ } = \dfrac{Q}{\varepsilon_{0}} \\
 \oiint_{S}{\vec{B} \cdot d\vec{S}\ } = 0  \\
 \oint_{\partial S}{\vec{E} \cdot d\vec{s}} = - \dfrac{\partial}{\partial t}\int_{\Sigma}{\vec{B} \cdot d\vec{\Sigma}}  \\
 \oint_{\partial S}{\vec{B} \cdot d\vec{s}} = \mu_{0}I + \mu_{0}\varepsilon_{0}\dfrac{\partial}{\partial t}\int_{\Sigma}{\vec{E} \cdot d\vec{\Sigma}}
\end{cases}
\]

Dove \(S\) è una superficie all'interno della quale si vuole valutare il campo, mentre \(\Sigma\) è una superficie aperta avente per contorno \(\partial S\). \(Q\) è la quantità di carica contenuta in \(S\) e \(I\) la corrente che attraversa \(\Sigma\).

\section{Equazioni di Maxwell nel mezzo}\label{equazioni-di-maxwell-nel-mezzo}

In presenza di un mezzo materiale si introducono i vettori di induzione elettrica \(\vec{D}\) e di campo magnetico \(\vec{H}\). Il vettore \(\vec{B}\) è detto vettore di induzione magnetica, mentre \(\vec{E}\) rappresenta il campo elettrico.

I campi di induzione descrivono il comportamento del materiale a seguito dell'applicazione dei rispettivi campi. Con l'introduzione dei campi \(\vec{E}\), \(\vec{D}\), \(\vec{H}\) e \(\vec{B}\), le equazioni di Maxwell in forma locale si scrivono come:

\[
\begin{cases}
\vec{\nabla} \cdot \vec{D} = \rho_{lib} \\
\vec{\nabla} \cdot \vec{B} = 0 \\
\vec{\nabla} \times \vec{E} = - \dfrac{\partial \vec{B}}{\partial t} \\
\vec{\nabla} \times \vec{H} = \vec{J}_{lib} + \dfrac{\partial \vec{D}}{\partial t}
\end{cases}
\]

Dove \(\rho_{lib}\) e \(\vec{J}_{lib}\) sono le sorgenti libere ovvero le cariche e correnti che possono muoversi liberamente all’interno o all’esterno di un materiale, non vincolate alla struttura atomica o molecolare del mezzo. Queste sorgenti possono essere controllate direttamente, ad esempio, applicando una tensione o facendo passare una corrente in un conduttore. In contrapposizione, ci sono le sorgenti vincolate (o di legame), che derivano dal comportamento microscopico delle molecole e atomi del materiale. I vettori di polarizzazione e di magnetizzazione dipendono da tali sorgenti.

Il vettore di induzione elettrica è definito come:

\[
\vec{D} = \varepsilon_{0}\vec{E} + \vec{P}
\]

dove \(\vec{P}\) è il vettore di polarizzazione; esso descrive come i dipoli elettrici contenuti nel materiale si orientano sotto l'effetto del campo elettrico.

L'induzione magnetica è invece definita come:

\[
\vec{B} = \mu_{0}(\vec{H} + \vec{M})
\]

dove \(\vec{M}\) è il vettore di magnetizzazione e descrive come i dipoli magnetici nel materialesi orientano sotto l’azione del campo magnetico applicato H \(\vec{H}\).

Per i mezzi lineari e isotropi esistono due costanti, dette suscettibilità elettrica \(\chi_{e}\) e suscettibilità magnetica \(\chi_{m}\), che rappresentano i coefficienti di proporzionalità, rispettivamente, tra i campi di polarizzazione e magnetizzazione e i relativi campi applicati:

\[
\vec{P} = \varepsilon_{0}\chi_{e}\vec{E}, \qquad \vec{M} = \chi_{m}\vec{H}
\]

Sostituendo il vettore di polarizzazione nella definizione del vettore di induzione elettrica si ottiene:

\[
\vec{D} = \varepsilon_{0}\vec{E} + \vec{P} = \varepsilon_{0}\vec{E} + \varepsilon_{0}\chi_{e}\vec{E} = \varepsilon_{0}\left( 1 + \chi_{e} \right)\vec{E}
\]

Si definisce la costante dielettrica del mezzo, per un materiale lineare e isotropo, come:

\[
\varepsilon = \varepsilon_{0}\left( 1 + \chi_{e} \right)
\]

Con questa definizione si ha:

\[
\vec{D} = \varepsilon\vec{E}
\]

Analogamente, per il campo di induzione magnetica:

\[
\vec{B} = \mu_{0}\vec{H} + \mu_{0}\vec{M} = \mu_{0}\vec{H} + \mu_{0}\chi_{m}\vec{H} = \mu_{0}\left( 1 + \chi_{m} \right)\vec{H}
\]

Si definisce la permeabilità magnetica del mezzo, per un materiale lineare e isotropo, come:

\[
\mu = \mu_{0}\left( 1 + \chi_{m} \right)
\]

Il campo di induzione magnetica si scrive quindi come:

\[
\vec{B} = \mu\vec{H}
\]

\begin{figure}[ht]
\centering
\includegraphics[width=3.19044in,height=2.4711in,alt={P1240\#yIS1}]{media/3_Elettromagnetismo/image19.pdf}
\caption{Dipolo elettrico e magnetico.}
\end{figure}

Infine, la densità di corrente totale si suddivide in una componente libera, associata alle cariche mobili, e in una componente vincolata, dovuta alla polarizzazione e alla magnetizzazione del materiale. Si scrive dunque:

\[
\vec{J} = \vec{J}_{lib} + \vec{J}_{vinc}
\]

La densità di corrente vincolata (o di legame) è data da:

\[
\vec{J}_{vinc} = \dfrac{\partial \vec{P}}{\partial t} + \vec{\nabla} \times \vec{M}
\]

Per un mezzo lineare e isotropo, sostituendo le relazioni costitutive si ottiene:

\[
\vec{J}_{vinc} = \varepsilon_0 \chi_e \dfrac{\partial \vec{E}}{\partial t} + \left(\vec{\nabla} \times \left(\chi_{m} \vec{H}\right)\right)
\]

Se il mezzo è omogeneo, $\chi_m$ è una costante spaziale. In questa ipotesi, la densità di corrente vincolata si scrive come:

\[
\vec{J}_{vinc} = \varepsilon_0 \chi_e \dfrac{\partial \vec{E}}{\partial t} + \chi_{m}\left(\vec{\nabla} \times \vec{H}\right)
\]

La densità di corrente totale \(\vec{J}\) dipende dunque sia dal campo elettrico sia dal campo magnetico. Nella quarta equazione di Maxwell, tuttavia, la densità di corrente si riferisce solo alla parte libera:

\[
\vec{\nabla} \times \vec{H} = \vec{J}_{lib} + \dfrac{\partial \vec{D}}{\partial t}
\]

In definitiva, i campi fondamentali che descrivono le forze elettromagnetiche sono \(\vec{E}\) e \(\vec{B}\), mentre \(\vec{D}\) e \(\vec{H}\) sono campi ausiliari introdotti per semplificare le equazioni di Maxwell in presenza di un mezzo materiale (come un dielettrico o un magnete), inglobando negli stessi gli effetti microscopici di polarizzazione (\(\vec{P}\)) e magnetizzazione (\(\vec{M}\)).

\section{Forza di Lorentz}\label{forza-di-lorentz}

I campi elettrico \(\vec{E}\) e di induzione magnetica \(\vec{B}\) esercitano una forza su una particella di carica \(q\) che si muove con velocità \(\vec{v}\) in una regione di spazio contenente tali campi. L’espressione della forza totale è:

\[
\vec{F} = q\vec{E} + q\vec{v} \times \vec{B}
\]

Questa interazione è nota come \textbf{forza di Lorentz}. Essa si compone di due contributi distinti:

\begin{itemize}
    \item \textbf{Forza elettrica}: \(\vec{F}_{E} = q\vec{E}\).
    Questa forza è parallela o antiparallela al campo elettrico, a seconda del segno della carica \(q\), ed è indipendente dalla velocità \(\vec{v}\) della particella.
    
    \item \textbf{Forza magnetica}: \(\vec{F}_{B} = q(\vec{v} \times \vec{B})\).
    Questa forza è sempre perpendicolare sia alla velocità \(\vec{v}\) sia al campo di induzione magnetica \(\vec{B}\). Poiché la direzione della forza è perpendicolare allo spostamento della particella, la forza magnetica \textit{non compie lavoro} sulla carica.
\end{itemize}

\section{Parallelismo tra campo elettrico e magnetico}\label{parallelismo-tra-campo-magnetico-ed-elettrico}

Le equazioni di Maxwell possono essere rese formalmente simmetriche introducendo le \textit{cariche magnetiche} e le \textit{correnti magnetiche}, concetti ipotetici non osservati sperimentalmente ma utili per evidenziare l’analogia tra fenomeni elettrici e magnetici.

Si indichi con \(U\) l’energia potenziale di un dipolo elettrico \(\vec{d}\) o magnetico \(\vec{\mu}\), immerso rispettivamente in un campo elettrico o magnetico.  
Conoscendo \(U\), è possibile determinare la forza \(\vec{F}\) e il momento torcente \(\vec{N}\) agente sul dipolo.

Per il campo elettrico:

\[
\begin{cases}
 U = - \vec{d} \cdot \vec{E} \\
 \vec{F} = - \vec{\nabla}U = \vec{\nabla}\left( \vec{d} \cdot \vec{E} \right) \\
 \vec{N} = \vec{d} \times \vec{E} \\
 \phi(\vec{r}) = \dfrac{1}{4\pi\varepsilon_0}\int \dfrac{\rho(\vec{r}')}{R}\,dV' \\
 \vec{E} = - \vec{\nabla}\phi \\
 \nabla^{2}\phi = -\dfrac{\rho}{\varepsilon_{0}}
\end{cases}
\]

Dove:
\begin{itemize}
    \item \(\vec{d}\) è il \textbf{momento di dipolo elettrico}, vettore diretto dalla carica negativa a quella positiva, proporzionale al modulo della carica e alla distanza di separazione;
    \item \(\vec{E}\) è il \textbf{campo elettrico}, che rappresenta la forza per unità di carica esercitata nello spazio;
    \item \(\phi\) è il \textbf{potenziale elettrico scalare}, da cui il campo deriva come gradiente negativo;
    \item \(\rho\) è la \textbf{densità di carica elettrica}, sorgente del campo elettrico;
    \item \(\varepsilon_0\) è la \textbf{costante dielettrica del vuoto}, che stabilisce la relazione tra campo elettrico e densità di carica.
\end{itemize}

Per il campo magnetico risulta:

\[
\begin{cases}
 U = - \vec{\mu} \cdot \vec{B} \\
 \vec{F} = - \vec{\nabla}U = \vec{\nabla}\left( \vec{\mu} \cdot \vec{B} \right) \\
 \vec{N} = \vec{\mu} \times \vec{B} \\
 \vec{A}(\vec{r}) = \dfrac{\mu_0}{4\pi}\int \dfrac{\vec{J}(\vec{r}')}{R}\,dV' \\
 \vec{B} = \vec{\nabla} \times \vec{A} \\
 \nabla^{2}\vec{A} = -\mu_{0}\vec{J}
\end{cases}
\]

Dove:
\begin{itemize}
    \item \(\vec{\mu}\) è il \textbf{momento di dipolo magnetico}, legato al moto circolare di cariche (correnti) e orientato secondo la regola della mano destra rispetto al verso della corrente;
    \item \(\vec{B}\) è il \textbf{campo magnetico}, che descrive l’azione sui dipoli magnetici e sulle particelle cariche in movimento;
    \item \(\vec{A}\) è il \textbf{potenziale vettore magnetico}, da cui si ricava il campo magnetico mediante il rotore;
    \item \(\vec{J}\) è la \textbf{densità di corrente elettrica}, sorgente del campo magnetico;
    \item \(\mu_0\) è la \textbf{permeabilità magnetica del vuoto}, che regola l’intensità del campo magnetico generato dalle correnti.
\end{itemize}

Queste relazioni mostrano il parallelismo formale tra le grandezze elettriche \((\vec{d}, \vec{E}, \phi, \rho, \varepsilon_0)\) e le corrispondenti magnetiche \((\vec{\mu}, \vec{B}, \vec{A}, \vec{J}, \mu_0)\).

\begin{table}[h!]
\centering
\begin{tabular}{@{} c c l @{}}
\toprule
\textbf{Elettrico} & \textbf{Magnetico} & \textbf{Descrizione} \\ 
\midrule
$\vec{d}$ & $\vec{\mu}$ & Momento di dipolo (separazione di cariche o corrente circolare) \\
$\vec{E}$ & $\vec{B}$ & Campo vettoriale (forza per unità di carica o effetto su dipoli) \\
$\phi$ & $\vec{A}$ & Potenziale (scalare o vettoriale) da cui deriva il campo \\ 
$\rho$ & $\vec{J}$ & Sorgente del campo (cariche o correnti) \\
$\varepsilon_0$ & $\mu_0$ & Costanti fondamentali del vuoto \\
\bottomrule
\end{tabular}
\caption{Confronto tra le grandezze elettriche e magnetiche e loro significato fisico.}
\label{tab:parallelo-elettromagnetico}
\end{table}

Questa analogia sottolinea la profonda simmetria formale tra elettricità e magnetismo, unificate nella teoria elettromagnetica di Maxwell.

\section{Forza e coppia su una piccola spira}\label{forza-e-coppia-su-una-piccola-spira}

Si vuole determinare la forza totale agente su una piccola spira di superficie \(S\), percorsa da una corrente \(I\), immersa in un campo di induzione magnetica uniforme.

\begin{figure}[ht]
\centering
\includegraphics[width=0.35\textwidth]{media/3_Elettromagnetismo/image20.pdf}
\caption{Spira percorsa da corrente in un campo magnetico.}
\end{figure}

Il momento magnetico dipolare è dato da:
\[
\vec{\mu} = I S\,\hat{\imath}_n
\]
dove \(\hat{\imath}_n\) è la normale alla superficie \(S\), orientata in modo da vedere la corrente ruotare in senso antiorario.

Si consideri un tratto elementare \(d\vec{s}\) della spira; su di esso agisce la forza di Lorentz, poiché le cariche al suo interno sono messe in moto dalla corrente elettrica:
\[
d\vec{F} = I\,d\vec{s} \times \vec{B}
\]
Integrando su tutta la lunghezza della spira si ottiene:
\[
\vec{F} = \oint_{\partial S} I\,d\vec{s} \times \vec{B}
\]
Poiché la corrente e il campo \(\vec{B}\) sono costanti, possono essere portati fuori dal segno di integrale:
\[
\vec{F} = I\left( \oint_{\partial S} d\vec{s} \right) \times \vec{B}
\]
Siccome la spira è in equilibrio, la somma di tutti i contributi elementari è nulla; di conseguenza la forza risultante è nulla:
\[
\vec{F} = \vec{0}
\]
ovvero:
\[
I\left( \oint_{\partial S} d\vec{s} \right) \times \vec{B} = \vec{0}
 \Longleftrightarrow
\left( \oint_{\partial S} d\vec{s} \right) \times \vec{B} = \vec{0}
\]

Anche se la forza netta agente sulla spira è nulla, possono comunque esistere momenti torcenti. Poiché la forza agente sulla spira elementare è nulla, il momento delle forze risulta indipendente dal polo scelto. La coppia agente sull’elemento infinitesimo è data da:
\[
d\vec{N} = \vec{r} \times d\vec{F}
\]
La coppia, o momento torcente, può essere denotata anche con \(\vec{\tau}\). Integrando lungo la spira si ottiene:
\[
\vec{N} = \oint_{\partial S} \vec{r} \times d\vec{F}
\]
Sostituendo la forza di Lorentz \(d\vec{F} = I\,d\vec{s} \times \vec{B}\), si ha:
\[
\vec{N} = I \oint_{\partial S} \vec{r} \times (d\vec{s} \times \vec{B})
\]
Dati tre vettori, è valida l’identità:
\[
\vec{a} \times (\vec{b} \times \vec{c}) = (\vec{a}\cdot\vec{c})\,\vec{b} - (\vec{a}\cdot\vec{b})\,\vec{c}
\]
Applicando tale relazione, l'espressione del momento si scrive come:
\[
\vec{N} = I \oint_{\partial S} \big[(\vec{r}\cdot\vec{B})\,d\vec{s} - \vec{B}\,(\vec{r}\cdot d\vec{s})\big]
\]
Poiché il vettore \(\vec{r}\) appartiene alla spira, risulta \(d\vec{s} = d\vec{r}\), e dunque:
\[
\vec{N} = I \oint_{\partial S} \big[(\vec{r}\cdot\vec{B})\,d\vec{r} - \vec{B}\,(\vec{r}\cdot d\vec{r})\big]
\]
Il termine \(\vec{r}\cdot d\vec{r}\) può essere riscritto come:
\[
\vec{r}\cdot d\vec{r} = \dfrac{1}{2}\,d(\vec{r}\cdot\vec{r})
\]
per cui:

\[\vec{N} = I\oint_{\partial S}{\left( \vec{r} \cdot \vec{B} \right)d\vec{r}} - I\oint_{\partial S}{\vec{B}\left( \vec{r} \cdot d\vec{r} \right)} = I\oint_{\partial S}{\left( \vec{r} \cdot \vec{B} \right)d\vec{r}} - \dfrac{1}{2}I\vec{B}\oint_{\partial S}{d\left( \vec{r} \cdot \vec{r} \right)}\]

L'ultimo termine è l'integrale esteso a una linea chiusa di una forma differenziale esatta, dunque, è nullo:

\[\oint_{\partial S}{d\left( \vec{r} \cdot \vec{r} \right)} = 0\]

Pertanto, la coppia risulta:

\[\vec{N} = I\oint_{\partial S}{\left( \vec{r} \cdot \vec{B} \right)d\vec{r}}\]

Per risolvere tale integrale si considera la quantità \(d\left( \left( \vec{r} \cdot \vec{B} \right)\vec{r} \right)\); si applicano le proprietà del differenziale:

\[d\left( \left( \vec{r} \cdot \vec{B} \right)\vec{r} \right) = \left( d\vec{r} \cdot \vec{B} \right)\vec{r} + \left( \vec{r} \cdot \vec{B} \right)d\vec{r} + \left( \vec{r} \cdot d\vec{B} \right)\vec{r}\]

Dato che il campo è uniforme, \(d\vec{B} = 0\). Per cui:

\[d\left( \left( \vec{r} \cdot \vec{B} \right)\vec{r} \right) = \left( d\vec{r} \cdot \vec{B} \right)\vec{r} + \left( \vec{r} \cdot \vec{B} \right)d\vec{r}\]

Si ricava la quantità \(\left( \vec{r} \cdot \vec{B} \right)d\vec{r}\), presente nell'espressione della coppia:

\[\left( \vec{r} \cdot \vec{B} \right)d\vec{r} = d\left( \left( \vec{r} \cdot \vec{B} \right)\vec{r} \right) - \left( d\vec{r} \cdot \vec{B} \right)\vec{r}\]

Per applicare nuovamente la proprietà sul prodotto vettore di tre vettori, si aggiunge e sottrae \(\left( \vec{r} \cdot \vec{B} \right)d\vec{r}\) al secondo membro:

\[\left( \vec{r} \cdot \vec{B} \right)d\vec{r} = d\left( \left( \vec{r} \cdot \vec{B} \right)\vec{r} \right) - \left( d\vec{r} \cdot \vec{B} \right)\vec{r} - \left( \vec{r} \cdot \vec{B} \right)d\vec{r} + \left( \vec{r} \cdot \vec{B} \right)d\vec{r}\]

Dove:

\[\left( \vec{r} \cdot \vec{B} \right)d\vec{r} - \left( d\vec{r} \cdot \vec{B} \right)\vec{r} = \vec{r} \times d\vec{r} \times \vec{B}\]

Per cui si ha:

\[\left( \vec{r} \cdot \vec{B} \right)d\vec{r} = d\left( \left( \vec{r} \cdot \vec{B} \right)\vec{r} \right) + \vec{r} \times d\vec{r} \times \vec{B} - \left( \vec{r} \cdot \vec{B} \right)d\vec{r} \Leftrightarrow 2\left( \vec{r} \cdot \vec{B} \right)d\vec{r} = d\left( \left( \vec{r} \cdot \vec{B} \right)\vec{r} \right) + \vec{r} \times d\vec{r} \times \vec{B}\]

In definitiva, si ha:

\[\left( \vec{r} \cdot \vec{B} \right)d\vec{r} = \dfrac{1}{2}d\left( \left( \vec{r} \cdot \vec{B} \right)\vec{r} \right) + \dfrac{1}{2}\vec{r} \times d\vec{r} \times \vec{B}\]

Sostituendo tale risultato nell’espressione della coppia si ha::

\[\vec{N} = I\oint_{\partial S}{\left( \vec{r} \cdot \vec{B} \right)d\vec{r}} = \dfrac{1}{2}\ I\oint_{\partial S}{d\left( \left( \vec{r} \cdot \vec{B} \right)\vec{r} \right)} + \dfrac{1}{2}\ I\oint_{\partial S}{\vec{r} \times d\vec{r} \times \vec{B}}\]

Il primo termine è nullo, essendo una circuitazione di una forma differenziale esatta, quindi:

\[\oint_{\partial S}{d\left( \left( \vec{r} \cdot \vec{B} \right)\vec{r} \right)} = \vec{0}\]

Per cui si ha:

\[\vec{N} = \dfrac{1}{2}\ I\oint_{\partial S}{\vec{r} \times d\vec{r} \times \vec{B}}\]

Siccome il campo è costante lungo tutto il percorso di integrazione, può essere portato all'esterno del simbolo di integrale:

\[\vec{N} = \dfrac{1}{2}\ I\left( \oint_{\partial S}{\vec{r} \times d\vec{r}} \right) \times \vec{B}\]

Si definisce il \textit{momento magnetico} della spira:

\[\vec{\mu} = \dfrac{1}{2}\ I\oint_{\partial S}{\vec{r} \times d\vec{r}} = IS{\hat{\imath}}_{n}\]

La coppia \(\vec{N}\) può, in definitiva, essere espressa come:

\[\vec{N} = \vec{\mu} \times \vec{B}\]

Nel caso più generale, il momento magnetico di una distribuzione continua di corrente in un volume \(V\) è dato da:

\[\vec{\mu} = \dfrac{1}{2}\int_{V}{\vec{r} \times \vec{J}dV}\]

\subsection{Momento magnetico di un anello rotante}\label{momento-magnetico-di-un-anello-rotante}

Si vuole calcolare il momento magnetico di un anello sottile di raggio \(R\), massa \(M\) e carica \(q\) distribuita uniformemente, che ruota con velocità angolare \(\omega\) attorno ad un asse ortogonale al piano in cui giace l'anello e passante per il centro.

\begin{figure}[ht]
\centering
\includegraphics[width=1.34709in,height=2.1703in,alt={P1332\#yIS1}]{media/3_Elettromagnetismo/image21.pdf}\caption{Anello rotante nel campo magnetico}
\end{figure}

L’anello, ruotando, genera una corrente la cui intensità media è:

\[I = \dfrac{q}{T}\]

Dove \(T\) è il periodo dell'oscillazione dato da:

\[T = \dfrac{2\pi}{\omega}\]

Per cui la corrente è data da:

\[I = \dfrac{q}{T} = q\dfrac{\omega}{2\pi}\]

L'anello è assimilabile a una spira percorsa da corrente, dunque, il suo momento magnetico è dato da:

\[\vec{\mu} = IS{\hat{\imath}}_{n} = q\dfrac{\omega}{2\pi}\pi R^{2}{\hat{\imath}}_{n}\]

Semplificando, si ottiene il momento magnetico:

\[\vec{\mu} = \dfrac{1}{2}q\omega R^{2}{\hat{\imath}}_{n}\]

\subsection{Momento magnetico per un guscio sferico}\label{momento-magnetico-per-un-guscio-sferico}

Si vuole calcolare il momento magnetico di uno strato sferico sottile di raggio \(R\), massa \(M\) e carica \(Q\) distribuita uniformemente, che ruota con velocità angolare \(\omega\) attorno ad un asse passante per il centro della sfera.

\begin{figure}[ht]
\centering
\includegraphics[width=2.625in,height=1.991in,alt={P1346\#yIS1}]{media/3_Elettromagnetismo/image22.pdf}\caption{Guscio sferico}
\end{figure}

Si pone l'asse \(z\) coincidente con l'asse di rotazione, per cui \(\vec{\omega}=\omega\,\hat{z}\). Si suddivide la superficie sferica in anelli infinitesimi, ciascuno giacente su un piano ortogonale all'asse \(z\). In corrispondenza dell'angolo polare \(\vartheta\) l'anello ha:
\[
\text{raggio: } r=R\sin\vartheta,\qquad
\text{lunghezza: } 2\pi R\sin\vartheta,\qquad
\text{larghezza (sulla sfera): } R\,d\vartheta.
\]

L'elemento di area dell'anello è quindi:
\[
dS = (2\pi R\sin\vartheta)(R\,d\vartheta)=2\pi R^{2}\sin\vartheta\,d\vartheta.
\]

Se \(\sigma\) è la densità superficiale di carica uniforme, la carica dell'anello è \(dq=\sigma\,dS\) e la corrente associata alla sua rotazione è:
\[
dI=\dfrac{dq}{T}=\dfrac{\sigma\,dS}{T}=\dfrac{\omega\sigma}{2\pi}\,2\pi R^{2}\sin\vartheta\,d\vartheta
=\omega\sigma R^{2}\sin\vartheta\,d\vartheta,
\]
con \(T=2\pi/\omega\) periodo di rotazione della sfera. L'area della spira (anello) è \(S=\pi (R\sin\vartheta)^{2}\), dunque il momento magnetico elementare vale:
\[
d\vec{\mu}=S\,dI\,\hat{z}
= \pi R^{2}\sin^{2}\vartheta\;\omega\sigma R^{2}\sin\vartheta\,d\vartheta\;\hat{z}
= \pi\omega\sigma R^{4}\sin^{3}\vartheta\,d\vartheta\;\hat{z}.
\]

Per le ipotesi fatte sul sistema di riferimento, risulta che \({\hat{\imath}}_{n} = {\hat{\imath}}_{z}\). Infatti, il momento magnetico è diretto come \(\vec{\omega}\). Svolgendo i prodotti, si ha:

\[d\vec{\mu} = \pi\omega\sigma R^{4}\sin^{3}\vartheta\ d\vartheta{\hat{\imath}}_{z}\]

Il momento magnetico è ottenuto integrando l'equazione ottenuta su tutti i possibili valori assunti da \(\vartheta\), ovvero:

\[\vec{\mu} = \int_{0}^{\pi}{\pi\omega\sigma R^{4}\sin^{3}\vartheta\ d\vartheta{\hat{\imath}}_{z}} = \pi\omega\sigma R^{4}\int_{0}^{\pi}{\sin^{3}\vartheta\ d\vartheta}{\hat{\imath}}_{z}\]

Si risolve l'integrale. Per le relazioni trigonometriche è possibile scrivere:

\[\int_{0}^{\pi}{\sin^{3}\vartheta d\vartheta} = \int_{0}^{\pi}{\left( \dfrac{3\sin\vartheta - \sin{3\vartheta}}{4} \right)d\vartheta} = \dfrac{1}{4}\left( 3\int_{0}^{\pi}{\sin\vartheta d\vartheta} - \int_{0}^{\pi}{\sin{3\vartheta}d\vartheta} \right)\]

Dove:

\[3\int_{0}^{\pi}{\sin\vartheta d\vartheta} = 3\left\lbrack - \cos\vartheta \right\rbrack_{0}^{\pi} = 3\left( - \cos\pi + \cos 0 \right) = 3(1 + 1) = 6\]

\[\int_{0}^{\pi}{\sin{3\vartheta}d\vartheta} = - \dfrac{1}{3}\left\lbrack \cos{3\vartheta} \right\rbrack_{0}^{\pi} = - \dfrac{1}{3}\left( \cos{3\pi} - \cos 0 \right) = - \dfrac{1}{3}( - 1 - 1) = \dfrac{2}{3}\]

Nel complesso, l'integrale è dato da:

\[\int_{0}^{\pi}{\sin^{3}\vartheta d\vartheta} = \dfrac{1}{4}\left( 6 - \dfrac{2}{3} \right) = \dfrac{1}{4}\left( \dfrac{18 - 2}{3} \right) = \dfrac{1}{4}\dfrac{16}{3} = \dfrac{4}{3}\]

Il momento magnetico è dato da:

\[\vec{\mu} = \dfrac{4}{3}\ \pi\omega\sigma R^{4}{\hat{\imath}}_{z}\]

La superficie totale della sfera è data da:

\[S = 4\pi R^{2}\]

Il prodotto della densità superficiale di carica \(\sigma\) per la superficie \(S\) restituisce la carica globale \(Q\). Il momento magnetico può essere scritto come:

\[\vec{\mu} = \dfrac{1}{3}\ \omega QR^{2}{\hat{\imath}}_{z}\]

%\input{Image/Ring}

\subsection{Momento magnetico per una sfera}\label{momento-magnetico-per-una-sfera}

Si vuole calcolare il momento magnetico di una sfera raggio \(R\), massa \(M\) e carica \(Q\) distribuita uniformemente, che ruota con velocità angolare \(\omega\) attorno ad un asse passante per il centro della sfera.

\begin{figure}[ht]
\centering
\includegraphics[width=1.375in,height=1.73838in,alt={P1372\#yIS1}]{media/3_Elettromagnetismo/image23.pdf}\caption{Sfera rotante}
\end{figure}

Si utilizzano le coordinate sferiche \(r\), \(\vartheta\), \(\varphi\). Si assume come polo il centro \(O\) della sfera carica e come asse polare il diametro parallelo alla velocità angolare\(\omega\). Dove:

\[r \in \left[ 0; R\right],\ \vartheta \in \left[ 0;\pi\right],\ \varphi \in \left[ 0;2\pi\right]\]

Si indica con \(\rho\) la densità di carica volumetrica. Il volumetto \(dV\) contiene una carica \(dq\):

\[dq = \rho dV\]

Un elemento di volume $dV$ contribuisce alla densità di corrente volumetrica $\vec{J}$, data da:

\[\vec{J} = \rho \vec{v}\]

Poiché la sfera ruota con velocità angolare \(\vec{\omega}\) e la carica ha densità \(\rho\), la densità di corrente volumetrica è:

\[\vec{J} = \rho \vec{v} = \rho \left( \vec{\omega} \times \vec{r}\right)\]

per calcolare il momento magnetico di un corpo di volume \(V\) con densità di corrente \(J\), si usa la formula generale:

\[\vec{\mu} = \dfrac{1}{2}\int_{V}{\vec{r} \times \vec{J}dV}\]

Sostituendo la densità di corrente dell'elemento di volume infinitesimo nella definizione del momento magnetico si ha:

\[\vec{\mu} = \dfrac{1}{2}\int_{V}{\vec{r} \times \left(\rho \left( \vec{\omega} \times \vec{r}\right)\right)dV}\]

Nell'ipotesi di densità di carica costante in tutto il volume, è possibile portare all'esterno del simbolo di integrale \(\rho\):


\[\vec{\mu} = \dfrac{1}{2}\rho\int_{V}{\vec{r} \times \left( \vec{\omega} \times \vec{r}\right)dV}\]

Si considera il triplo prodotto vettoriale all'interno dell'integrale e si utilizza l'identità nota:

\[
\vec{a} \times (\vec{b} \times \vec{c}) = (\vec{a} \cdot \vec{c})\vec{b} - (\vec{a} \cdot \vec{b})\vec{c}
\]

Ponendo \(\vec{a} = \vec{r}\), \(\vec{b} = \vec{\omega}\), e \(\vec{c} = \vec{r}\), l'espressione \(\vec{r} \times \vec{\omega} \times \vec{r} \) diventa:

\[
\vec{r} \times \vec{\omega} \times \vec{r} = (\vec{r} \cdot \vec{r})\vec{\omega} - (\vec{r} \cdot \vec{\omega})\vec{r}
\]

Ricordando che \(\vec{r} \cdot \vec{r} = r^2\), l'ultima relazione può essere scritta come:

\[
\vec{r} \times \vec{\omega} \times \vec{r} = r^{2}\vec{\omega}-\left(\vec{r}\cdot\vec{\omega}\right)\vec{r}
\]

Per ipotesi, la rotazione avviene lungo la direzione \(\hat{\imath}_{z}\), per cui \(\vec{\omega}=\omega \hat{\imath}_{z}\). Inoltre, è possibile esprimere il vettore posizione in coordinate sferiche:

\[
\vec{r} = r (\sin\vartheta \cos\varphi\,\hat{\imath}_x + \sin\vartheta \sin\varphi\,\hat{\imath}_y + \cos\vartheta\,\hat{\imath}_z)
\]

Il prodotto scalare tra il vettore posizione e la velocità angolare avviene solamente tre le componenti lungo \(\hat{\imath}_z\):

\[
\vec{r}\cdot \vec{\omega} = r (\sin\vartheta \cos\varphi\,\hat{\imath}_x + \sin\vartheta \sin\varphi\,\hat{\imath}_y + \cos\vartheta\,\hat{\imath}_z) \cdot \omega\hat{\imath}_z = r\omega\cos\vartheta\
\]

Il prodotto vettoriale triplo si scrive come:

\[
\vec{r} \times \vec{\omega} \times \vec{r} = {r}^2\omega\hat{\imath}_z - \left(\omega r \cos\vartheta\right)\left(r (\sin\vartheta \cos\varphi\,\hat{\imath}_x + \sin\vartheta \sin\varphi\,\hat{\imath}_y + \cos\vartheta\,\hat{\imath}_z)\right)
\]

Svolgendo i prodotti, si ricava:

\[
\vec{r} \times \vec{\omega} \times \vec{r} = {r}^2\omega\hat{\imath}_z - r^{2}\omega\left( \sin{\vartheta}\cos{\varphi}\,\cos\vartheta\hat{\imath}_{x} + \sin\vartheta \sin\varphi\,\cos\vartheta\hat{\imath}_y+\cos^2\vartheta\hat{\imath}_z\right)
\]

Sostituendo questo risultato nell'espressione vettoriale per il momento magnetico, si ottiene:

\[
\vec{\mu} = \dfrac{1}{2}\rho\int_{V}{ \left({r}^2\omega\hat{\imath}_z - r^{2}\omega\left( \sin{\vartheta}\cos{\varphi}\,\cos\vartheta\hat{\imath}_{x} + \sin\vartheta \sin\varphi\,\cos\vartheta\hat{\imath}_y+\cos^2\vartheta\hat{\imath}_z\right)\right)dV}
\]

Si raccoglie \(\omega\) che, essendo costante rispetto la posizione, può essere portata fuori dal simbolo di integrale:

\[
\vec{\mu} = \dfrac{1}{2}\rho\omega \int_{V}{ \left({(r}^2\hat{\imath}_z - r^{2}\left( \sin{\vartheta}\cos{\varphi}\,\cos\vartheta\hat{\imath}_{x} + \sin\vartheta \sin\varphi\,\cos\vartheta\hat{\imath}_y+\cos^2\vartheta\hat{\imath}_z\right)\right)dV}
\]

Si divide l'integrale in base alle componenti spaziali delle coordinate sferiche:

\[
\vec{\mu} = \dfrac{1}{2}\rho\omega \left(\int_{V}{ \left(r^2\,\hat{\imath}_z - r\cos^{2}{\vartheta} \right)\hat{\imath}_z\,dV} - \int_{V}{\left( r^{2} \sin{\vartheta}\cos{\varphi}\,\cos\vartheta\hat{\imath}_{x} \right) \,dV} - \int_{V}{\left( r^2sin\vartheta \sin\varphi\,\cos\vartheta\hat{\imath}_y\right) \,dV}\right)
\]

In coordinate sferiche, l'elemento di volume infinitesimo è il prodotto delle tre lunghezze elementari:

\[
dV = 
\underbrace{dr}_{\text{spessore radiale}}\,
\underbrace{(r\,d\vartheta)}_{\text{altezza meridiana}}\,
\underbrace{(r\sin\vartheta\,d\varphi)}_{\text{lunghezza azimutale}}
= r^{2}\sin\vartheta\,dr\,d\vartheta\,d\varphi
\]

Si risolve l'integrale contenente la componente lungo la direzione \(\hat{\imath}_x\). Utilizzando le coordinate sferiche e integrando su tutto il volume, si ottiene:

\[
\int_{V}{\left( r^{2} \sin{\vartheta}\cos{\varphi}\,\cos\vartheta \right) \,dV}=\int_{0}^{R}{\int_{0}^{\pi}{\int_{0}^{2\pi}{r^{2} \sin{\vartheta}\cos{\varphi}\,\cos\vartheta \left(r^{2}\sin\vartheta\right)d\varphi}d\vartheta}dr}=
\]

Per la linearità dell'operatore integrale è possibile scrivere:

\[
=\int_{0}^{R}{r^{4}\int_{0}^{\pi}{\sin^{2}{\vartheta}\cos{\vartheta}\int_{0}^{2\pi}{cos{\varphi}\,d\varphi}d\vartheta}dr}
\]

Nella relazione individuata il \(cos{\varphi}\) è integrato sul suo periodo, pertanto il suo risultato è nullo. Di conseguenza anche l'integrale complessivo è nullo:

\[
\int_{V}{\left( r^{2} \sin{\vartheta}\cos{\varphi}\,\cos\vartheta \right) \,dV}=\int_{0}^{R}{r^{4}\int_{0}^{\pi}{\sin^{2}\cos{\vartheta}\int_{0}^{2\pi}{cos{\varphi}\,d\varphi}d\vartheta}dr}=0
\]

Per l'integrazione lungo la componente \(\hat{\imath}_y\) vale un discorso analogo, dunque, anche questo contributo è nullo. Le componenti $\mu_x$ e $\mu_y$ del momento magnetico si annullano a causa della simmetria di rotazione della sfera attorno all'asse $\hat{\imath}_{z}$, espressa matematicamente dall'integrazione su $\varphi$ (l'angolo azimutale) che annulla i termini $\cos\varphi$ e $\sin\varphi$. Il momento magnetico totale è quindi solo in direzione $\hat{\imath}_z$:

\[
\vec{\mu} = \mu_z\,\hat{\imath}_z
\]

Al fine di valutare l'espressione del momento magnetico, bisogna risolvere l'integrale lungo l'asse di rotazione:

\[
\mu_{z} = \dfrac{1}{2}\rho\omega\int_{V}{\left(r^{2} - r^{2}\cos^2{\vartheta} \right) dV} = \dfrac{1}{2}\rho\omega\int_{V}{r^{2}\left(1 -\cos^2{\vartheta} \right) dV}
\]

Si utilizza l'identità trigonometrica \(1 - \cos^2{\vartheta} = \sin^2{\vartheta}\) e si sostituisce l'espressione del volumetto elementare \(dV = r^{2}\sin\vartheta drd\vartheta d\varphi\):

\[
\mu_{z} = \dfrac{1}{2}\rho\omega\int_{V}{r^{2}\sin^2{\vartheta} dV}=\dfrac{1}{2}\rho\omega\int_{0}^{R}{\int_{0}^{\pi}{\int_{0}^{2\pi}{r^{2}\sin^2{\vartheta} \left(r^{2}\sin\vartheta drd\vartheta d\varphi\right)}}}
\]

Riorganizzando i termini e separando gli integrali, si ricava:

\[
\mu_{z} = \dfrac{1}{2}\rho\omega \left(\int_{0}^{2\pi} d\varphi\right) \left(\int_{0}^{\pi} \sin^3{\vartheta} d\vartheta\right) \left(\int_{0}^{R} r^{4} dr\right)
\]

L'integrale sulla coordinata azimutale è:

\[
\int_{0}^{2\pi} d\varphi = 2\pi
\]

L'integrale sul raggio, è invece dato da:

\[
\int_{0}^{R} r^{4} dr = \left[ \dfrac{r^{5}}{5} \right]_{0}^{R} = \dfrac{R^{5}}{5}
\]

L'integrale sulla coordinata polare è:

\[\int_{0}^{\pi}{\sin^{3}\vartheta d\vartheta} = \int_{0}^{\pi}{\left( \dfrac{3\sin\vartheta - \sin{3\vartheta}}{4} \right)d\vartheta} = \dfrac{1}{4}\left( 3\int_{0}^{\pi}{\sin\vartheta d\vartheta} - \int_{0}^{\pi}{\sin{3\vartheta}d\vartheta} \right)\]

\[\int_{0}^{\pi}{\sin^{3}\vartheta d\vartheta} = \int_{0}^{\pi}{\left( \dfrac{3\sin\vartheta - \sin{3\vartheta}}{4} \right)d\vartheta} = \dfrac{1}{4}\left( 3\int_{0}^{\pi}{\sin\vartheta d\vartheta} - \int_{0}^{\pi}{\sin{3\vartheta}d\vartheta} \right)\]

Dove:

\[3\int_{0}^{\pi}{\sin\vartheta d\vartheta} = 3\left\lbrack - \cos\vartheta \right\rbrack_{0}^{\pi} = 3\left( - \cos\pi + \cos 0 \right) = 3(1 + 1) = 6\]

\[\int_{0}^{\pi}{\sin{3\vartheta}d\vartheta} = - \dfrac{1}{3}\left\lbrack \cos{3\vartheta} \right\rbrack_{0}^{\pi} = - \dfrac{1}{3}\left( \cos{3\pi} - \cos 0 \right) = - \dfrac{1}{3}( - 1 - 1) = \dfrac{2}{3}\]

Nel complesso, l'integrale è dato da:

\[
\int_{0}^{\pi} \sin^3{\vartheta} d\vartheta = \dfrac{4}{3}
\]


Di conseguenza, il momento magnetico della sfera è dato da:

\[
\mu_{z} = \dfrac{1}{2}\rho\omega \left( 2\pi \right) \left( \dfrac{4}{3} \right) \left( \dfrac{R^{5}}{5} \right) = \dfrac{4\pi}{15}\rho\omega R^5
\]

La carica totale \(Q\) è data dal prodotto della densità volumetrica \(\rho\) per il volume della sfera \(V=4\pi R^{3}/3\):

\[
Q=\rho V\Leftrightarrow \rho = \dfrac{Q}{V}
\]

Sostituendo il volume della sfera, si ottiene:
\[
Q=\rho V\Leftrightarrow \rho = \dfrac{Q}{V}=\dfrac{Q}{\dfrac{4}{3}\pi R^3}=\dfrac{3Q}{4\pi R^3}
\]

Sostituendo l'espressione di \(\rho\) nell'espressione per il momento magnetico lungo l'asse di rotazione, si ottiene:

\[
\mu_{z} = \dfrac{4\pi}{15}\omega R^5 \left( \dfrac{3Q}{4\pi R^3} \right)
\]

Semplificando:

\[
\mu_{z} = \dfrac{1}{5} Q \omega R^2
\]

Dato che il momento magnetico della sfera rotante è diretto lungo l'asse di rotazione, si ha:
\[
\vec{\mu} = \dfrac{1}{5} Q \omega R^{2}\hat{\imath}_z
\]

\section{Forza su piccola spira in un campo disomogeneo}\label{forza-su-piccola-spira-in-un-campo-disomogeneo}

La forza agente su una spira elementare, percorsa da corrente, immersa in un campo magnetico generico può essere espressa come:

\[\vec{F} = \vec{\nabla}\left( \vec{\mu} \cdot \vec{B} \right)\]

Dove \(\vec{\mu}\) è il momento magnetico della spira. Per un atomo il momento magnetico è costante, dunque, può essere portato all'esterno del simbolo di gradiente, producendo l'operatore:

\[
(\vec{\mu} \cdot \vec{\nabla}) = \mu_x\dfrac{\partial}{\partial x} + \mu_y\dfrac{\partial}{\partial y} + \mu_z\dfrac{\partial}{\partial z}
\]

L'espressione per la forza si riduce a:

\[
\vec{F} = \left(\vec{\mu} \cdot \vec{\nabla}\right) \vec{B}
\]

L'equazione, scritta in forma estesa, è:

\[
\begin{pmatrix}
F_{x} \\
F_{y} \\
F_{z}
\end{pmatrix} = \begin{pmatrix}
     \mu_x\dfrac{\partial}{\partial x} + \mu_y\dfrac{\partial}{\partial y} + \mu_z\dfrac{\partial}{\partial z}
\end{pmatrix}\begin{pmatrix}
B_{x} \\
B_{y} \\
B_{z}
\end{pmatrix} = \begin{pmatrix}
 \mu_x\dfrac{\partial B_x}{\partial x} + \mu_y\dfrac{\partial B_x}{\partial y} + \mu_z\dfrac{\partial B_x}{\partial z} \\
 \mu_x\dfrac{\partial B_y}{\partial x} + \mu_y\dfrac{\partial B_y}{\partial y} + \mu_z\dfrac{\partial B_y}{\partial z} \\
 \mu_x\dfrac{\partial B_z}{\partial x} + \mu_y\dfrac{\partial B_z}{\partial y} + \mu_z\dfrac{\partial B_z}{\partial z}
\end{pmatrix}
\]

Si suppone che il campo sia diretto solamente lungo \(z\), direzione lungo cui è variabile. In altre parole, il campo è dato da:

\[\vec{B} = B_{z}(z){\hat{\imath}}_{z}\]

La forza, in questo contesto, può essere espressa come:
\[
\begin{pmatrix}
F_{x} \\
F_{y} \\
F_{z}
\end{pmatrix} = \begin{pmatrix}
      \mu_x\dfrac{\partial}{\partial x} + \mu_y\dfrac{\partial}{\partial y} + \mu_z\dfrac{\partial}{\partial z}
\end{pmatrix}\begin{pmatrix}
0 \\
0 \\
B_{z}\left(z\right)
\end{pmatrix} = \begin{pmatrix}
0 \\
0 \\
 \mu_z\dfrac{\partial B_z}{\partial z}
\end{pmatrix}
\]

La componente non nulla della forza è solamente quella lungo \(z\), data da:

\[F_{z} = \mu_{z}\dfrac{\partial B_{z}}{\partial z}\]

La forza che agisce sulla spira dipende dalla proiezione del momento magnetico \(\vec{\mu}\) lungo \(z\) e dal gradiente del campo magnetico \(\vec{B}\) lungo \(z\).

Questa relazione mostra che un dipolo magnetico immerso in un campo non uniforme subisce una forza che tende a spostarlo verso le regioni di campo più intenso se il momento magnetico è positivo, \(\mu_z>0\), o meno intenso se \(\mu_z<0\).

\subsection{Esperimento di Stern-Gerlach}\label{esperimento-di-stern-gerlach}

L'esperimento di Stern-Gerlach fornisce una prova sperimentale della quantizzazione del momento magnetico dell'atomo.

Nel modello classico l'elettrone era visto come un pianeta orbitante intorno al nucleo, dunque, caratterizzato da un momento angolare dato dalla rivoluzione dell'elettrone intorno al nucleo e intorno al proprio asse.

\begin{figure}[ht]
\centering
\includegraphics[width=3.21429in,height=2.53278in,alt={P1423\#yIS1}]{media/3_Elettromagnetismo/image24.pdf}\caption{Modello planetario dell'atomo}
\end{figure}

È noto, inoltre, che la massa del nucleo è di molti ordini di grandezza maggiore rispetto quella dell'elettrone; dunque, è possibile ritenere il nucleo fermo rispetto all'elettrone.

Il moto dell'elettrone può essere considerato circolare con raggio \(r\). L'elettrone viaggia con velocità \(\vec{v}\). Per definizione la corrente è:

\[I = \dfrac{\mathrm{\Delta}q}{\mathrm{\Delta}t}\]

Dove, la carica coincide con quella dell'elettrone, mentre \(\mathrm{\Delta}t\) coincide con il periodo di rivoluzione della particella carica:

\[I = \dfrac{\mathrm{\Delta}q}{\mathrm{\Delta}t} = \dfrac{e}{2\pi\dfrac{r}{v}} = \dfrac{ev}{2\pi r}\]

Data la piccola corrente generata, esiste un momento magnetico \(\vec{\mu}\) ortogonale al piano sul quale l'elettrone esegue la sua orbita, con verso diretto in modo da vedere la corrente ruotare in senso antiorario. Per convenzione sulla corrente, l'elettrone deve ruotare in senso orario.

Il momento magnetico è dato da:

\[\vec{\mu} = IS{\hat{\imath}}_{n}\]

Si considera il modulo, si sostituisce l'espressione della corrente prodotta dall'elettrone e la superficie della spira descritta :

\[\mu = IS = - \dfrac{ev}{2\pi r}\pi r^{2} = - \dfrac{evr}{2}\]

Dove il segno meno è dovuto alla convenzione sulle correnti.

Sull'elettrone agisce un momento angolare \(\vec{L}\), dovuto all'orbita circolare descritto dall'elettrone, anch'esso ortogonale al piano dell'orbita, dato da:

\[\vec{L} = m\vec{r} \times \vec{v}\]

Si considera il modulo del momento angolare:

\[L = mrv\]

Si moltiplicano ambo i membri per \(e/2\):

\[\dfrac{e}{2}L = \dfrac{e}{2}mrv\]

Sostituendo l'espressione del momento magnetico, si ha:

\[\dfrac{e}{2}L = m\mu\]

Ricavano il momento magnetico in funzione del momento angolare, si ha:

\[\mu = \dfrac{e}{2m}L\]

In generale, se la particella ha carica \(q\) negativa, risulta:

\[\vec{\mu} = - \dfrac{q}{2m}\vec{L}\]

Dove \(\vec{\mu}\) è opposto a \(\vec{L}\) per la convenzione sulle correnti.

Sebbene l'ultima equazione sia stata ricavata nell'ambito della fisica classica, è valida anche in meccanica quantistica, in cui il punto di vista è completamente diverso.

Sebbene l'esperimento venga oggi usato per dimostrare lo spin, al tempo l'obiettivo era dimostrare l'esistenza del momento magnetico intrinseco e la sua quantizzazione spaziale. I due fisici eseguirono l'esperimento sugli atomi di argento, emessi da una sorgente. Questi atomi venivano deflessi da un campo magnetico non omogeneo, variabile lungo \(z\). L'argento è stato scelto proprio perché il suo momento angolare totale, derivante dagli elettroni di valenza, è dovuto essenzialmente a un singolo elettrone $s$-orbitale, semplificando l'interpretazione dei risultati.

La meccanica classica prevede che momento magnetico degli atomi di argento sia distribuito statisticamente in tutte le direzioni. Sugli atomi di argento agisce una forza data da:

\[F_{i} = {\mu_{i}}_{z}\dfrac{\partial B_{z}}{\partial z}\ \]

Dunque, ogni atomo subisce una deflessione dovuta alla proiezione del suo momento magnetico lungo l'asse \(z\) e dal gradiente del campo magnetico lungo lo stesso asse.

Nella teoria classica, dato che ogni atomo possiede un momento magnetico orientato casualmente, Stern e Gerlach si aspettavano di ottenere una linea retta compresa tra un massimo e minimo. Tutte le posizioni compresi tra questi due valori hanno tutti la stessa probabilità.

Tuttavia, i due scienziati rilevarono solo due punti di arrivo. I due dedussero che gli orientamenti dei momenti magnetici degli atomi non disposti in modo casuale ma in maniera quantizzata.

Questo risultato non può essere previsto dalla meccanica classica, ma viene spiegato dalla meccanica quantistica.

\begin{figure}[ht]
\centering
\includegraphics[width=5.47196in,height=2.05093in,alt={P1456\#yIS1}]{media/3_Elettromagnetismo/image25.pdf}\caption{Esperimento di Stern-Gerlach}
\end{figure}

\subsection{Concetto di spin}\label{concetto-di-spin}

Il concetto di spin è inglobato nella meccanica quantistica e può essere visualizzato come la rotazione dell'elettrone intorno al suo asse. Lo spin determina un momento angolare e un momento magnetico. I due parametri sono antiparalleli e legati dal rapporto giromagnetico ($\gamma$).

\[\vec{\mu} = - \dfrac{q}{2m}\vec{L} = - \gamma_{e}\vec{L}\]

Il concetto di elettrone rotante non può essere considerato valido poiché, nel contesto della meccanica quantistica, l'elettrone è privo di estensione superficiale. Lo spin è una proprietà intrinseca della particella, non legata a rotazione spaziale.

In meccanica quantistica, ogni particella elementare o composta che possieda un momento angolare intrinseco (spin $\vec{S}$) o orbitale ($\vec{L}$) è associata a un momento magnetico $\vec{\mu}$. Il momento magnetico intrinseco ($\vec{\mu}_s$) di una particella è sempre proporzionale al suo spin ($\vec{S}$):

\[
\vec{\mu}_s = g \dfrac{q}{2m} \vec{S}
\]
Dove:
\begin{itemize}
    \item $q$ e $m$ sono, rispettivamente, la carica e la massa della particella (ad esempio, elettrone, protone, muone, ecc.);
    \item $\vec{S}$ è il momento angolare di spin della particella;
    \item $g$ è il fattore giromagnetico ($g$-factor) o fattore di Landé, un numero adimensionale che indica di quanto il momento magnetico devia dal valore classico ($q/2m$)
\end{itemize}

Per l'elettrone, la formula è:
\[
\vec{\mu}_{e} = - g_{e} \dfrac{e}{2m_{e}} \vec{S}
\]

Il fattore $e/2m_{e}$ è l'unità naturale del magnetismo per l'elettrone ed è chiamato magnetone di Bohr ($\mu_B$). Il fattore $g_e$ per l'elettrone libero è $g_e \approx 2.0023$ (la piccola deviazione da $g=2$ è spiegata dall'Elettrodinamica Quantistica, QED).

Per i nucleoni e nuclei, si usa la massa del protone ($m_p$) come riferimento per l'unità magnetica, poiché i momenti magnetici nucleari sono molto più piccoli:

\[
\vec{\mu}_{N} = g_{N} \dfrac{e}{2m_{p}} \vec{I}
\]
Dove:
\begin{itemize}
    \item \(e/2m_{p}\) è il magnetone nucleare ($\mu_N$);
    \item $\vec{I}$ è il momento angolare di spin nucleare;
    \item $g_N$ è il fattore $g$ nucleare, che è caratteristico di ciascun nucleo.
\end{itemize}

Per un atomo in uno stato quantico definito, il momento magnetico totale ($\vec{\mu}_{tot}$) è proporzionale al momento angolare totale ($\vec{J} = \vec{L} + \vec{S}$):

\[
\vec{\mu}_{tot} = - g \dfrac{e}{2m_{e}} \vec{J}
\]
In questo caso, il fattore $g$ è il fattore di Landé ($g$) dell'atomo, il cui valore dipende da come si accoppiano $\vec{L}$ e $\vec{S}$ (ad esempio, tramite l'accoppiamento $LS$)

In sintesi, la relazione $\vec{\mu} = g q/2m \vec{S}$ è la formula unificatrice in meccanica quantistica, con il fattore $g$ che incorpora la natura specifica della particella o del sistema considerato.

\section{Campo magnetico prodotto da un momento magnetico}\label{campo-magnetico-prodotto-da-un-momento-magnetico}

Si vuole determinare il campo creato da un momento magnetico di una piccola spira percorsa da corrente. A tale scopo si considerano le equazioni di Maxwell per il campo induzione magnetica nel vuoto:

\[\begin{cases}
\vec{\nabla} \cdot \vec{B} = 0 \\
\vec{\nabla} \times \vec{B} = \mu_{0}\left( \vec{J} + \varepsilon_{0}\dfrac{\partial\vec{E}}{\partial t} \right)
\end{cases}
\]

Se la lunghezza d'onda del campo incidente \(\lambda\) è molto maggiore della dimensione lineare dell'oggetto, è possibile ritenere il campo elettromagnetico lentamente variabile sulla superficie della spira, dunque:

\[\dfrac{\partial\vec{E}}{\partial t} \simeq 0\]

È possibile scrivere:

\[\begin{cases}
\vec{\nabla} \cdot \vec{B} = 0 \\
\vec{\nabla} \times \vec{B} = \mu_{0}\vec{J}
\end{cases}
\]

Siccome il campo induzione magnetica è solenoidale, è possibile definire un potenziale vettore, tale che:

\[\vec{B} = \vec{\nabla} \times \vec{A}\]

Si sostituisce la definizione del potenziale vettore nella seconda equazione:

\[\vec{\nabla} \times \vec{B} = \mu_{0}\vec{J} \Leftrightarrow \vec{\nabla} \times \vec{\nabla} \times \vec{A}\]

Il rotore del rotore può essere scritto come:

\[\vec{\nabla} \times \vec{\nabla} \times = \vec{\nabla}\left( \vec{\nabla} \cdot \  \right) - \nabla^{2}\ \]

Dunque, si ottiene:

\[\vec{\nabla}\left( \vec{\nabla} \cdot \vec{A} \right) - \nabla^{2}\vec{A} = \mu_{0}\vec{J}\]

Il potenziale vettore non è univocamente definito, dunque, è possibile imporre la condizione, detta gauge di Coulomb:

\[\vec{\nabla} \cdot \vec{A} = 0\]

Si ottiene che il laplaciano del campo vettore è dato da:

\[\nabla^{2}\vec{A} = - \mu_{0}\vec{J}\]

Si dimostra che la soluzione è del tipo:

\[\vec{A} = \dfrac{\mu_{0}}{4\pi}\int_{V}\dfrac{\vec{J}\left( {\vec{r}}' \right)}{\left| \vec{r} - {\vec{r}}' \right|}dV'\]

Per una piccola spira, lontano da essa, si utilizza lo sviluppo in serie di multipoli per $1/\left| \vec{r} - {\vec{r}}' \right|$ e si considera solo il termine di dipolo magnetico. Si dimostra che il potenziale vettore in un punto di osservazione $\vec{R}$ è dato da::

\[\vec{A}\left( \vec{r} \right) = \dfrac{\mu_{0}}{4\pi}\dfrac{\vec{\mu} \times \vec{R}}{R^{3}}\]

Dove \(\vec{R}\) è il vettore che congiunge il centro della piccola spira col punto di osservazione.

\begin{figure}[ht]
\centering
\includegraphics[width=2.69167in,height=2.03333in,alt={P1496\#yIS1}]{media/3_Elettromagnetismo/image26.pdf}\caption{Campo prodotto da una spira elementare}
\end{figure}

L'iterazione di una spira con un campo magnetico \(\vec{B}\) esterno è descritta dall'energia potenziale \(U\), data da:

\[U = - \vec{\mu} \cdot \vec{B} = -\mu B\cos\beta\]

Con \(\beta\) angolo formato dal campo magnetico e il momento magnetico. L'energia potenziale è nulla quando il campo magnetico è ortogonale al momento magnetico.

Il momento magnetico immerso in un campo magnetico subisce l'effetto di una coppia data da:

\[\vec{\tau} = \vec{\mu} \times \vec{B}\]

Con \(\vec{\tau}\) momento torcente.

\section{Moto del momento magnetico in campo magnetico}\label{moto-del-momento-magnetico-in-campo-magnetico}

Si considera una particella, come un elettrone, un atomo o un nucleo, immerso in un campo magnetico \(\vec{B}\). Per semplicità si analizza il sistema mediante una descrizione classica e non relativistica.

Ogni atomo o particella subatomica, immerso in un campo magnetico subisce un momento torcente \(\vec{\tau}\) dato da:

\[\vec{\tau} = \vec{\mu} \times \vec{B}\]

Il momento angolare rispetta la seconda legge di Newton:

\[\dfrac{d\vec{L}}{dt} = \vec{\tau}\]

Sostituendo l'espressione per il momento angolare in funzione del momento magnetico si ha:

\[\dfrac{d\vec{L}}{dt} = \vec{\mu} \times \vec{B}\]

Il momento magnetico è legato al momento angolare da un fattore di proporzionalità \(\gamma\):

\[\vec{\mu} = \gamma\vec{L}\]

In caso di elettroni, la costante di proporzionalità coincide con il rapporto giromagnetico:

\[\gamma = - \gamma_{e}\]

In caso di atomo con:

\[\gamma = - g\dfrac{q}{2m}\]

Mentre per un nucleo con:

\[\gamma = g\dfrac{q}{2m_{p}}\]

La seconda legge di Newton può essere scritta come:

\[\dfrac{d\vec{L}}{dt} = \vec{\mu} \times \vec{B} =  \gamma\vec{L} \times \vec{B}\]

\(\gamma\) è positivo se momento angolare e momento magnetico sono paralleli, negativo se antiparalleli.

Dall'equazione ottenuta si nota che la derivata del momento angolare deve essere perpendicolare al momento angolare stesso. Ne discende che il modulo di \(\vec{L}\) è costante. In altre parole, la punta del vettore momento angolare \(\vec{L}\) giace su una sfera e percorre una traiettoria circolare nel piano perpendicolare a \(\vec{B}\). Tale modo è detto precessione del momento magnetico o del momento angolare.

\begin{figure}[ht]
\centering
\includegraphics[width=2.30833in,height=2.58576in,alt={P1524\#yIS1}]{media/3_Elettromagnetismo/image27.pdf}\caption{Moto di precessione del momento magnetico}
\end{figure}

Confrontando l'equazione ottenuta per il momento angolare:

\[\dfrac{d\vec{L}}{dt} = \gamma\vec{L} \times \vec{B} = - \gamma\vec{B} \times \vec{L}\]

Con la relazione generale che lega la rotazione di un vettore alla velocità angolare:

\[d\vec{v} = \vec{\Omega} \times \vec{v}dt\]

è evidente che la velocità del moto di precessione è data da:

\[\vec{\Omega} = - \gamma\vec{B}\]

Con questa definizione, è possibile scrivere:

\[\dfrac{d\vec{L}}{dt} = \vec{\Omega} \times \vec{L}\]

\section{Diamagnetismo}\label{diamagnetismo}

I materiali diamagnetici sono caratterizzati da atomi che, se immersi in un campo magnetico \(\vec{B}\), sviluppano un momento magnetico aggiuntivo tale che il momento magnetico risultante si oppone al campo applicato.

Si suppone di applicare lentamente un campo magnetico a un atomo di materiale diamagnetico. Il suo nucleo, avendo una massa molto maggiore dell'elettrone può essere ritenuto fermo, mentre l'elettrone ruota interno al nucleo. Questo movimento può essere assimilato a una spira di raggio \(r\) percorsa da corrente centrata sul nucleo.

\begin{figure}[ht]
\centering
\includegraphics[width=2.05833in,height=1.91724in,alt={P1537\#yIS1}]{media/3_Elettromagnetismo/image28.pdf}\caption{Nucleo immerso in un campo magnetico}
\end{figure}

Il campo magnetico \(\vec{B}\), variabile nel tempo e nello spazio, si concatena con la spira, provocando la generazione di una forza elettromagnetica data dalla legge di Faraday:

\[\oint_{C}{\vec{E} \cdot d\vec{C}} = - \dfrac{\partial}{\partial t}\int_{S}{\vec{B} \cdot d\vec{S}}\]

Dove \(S\) è la superficie della spira e \(\partial S = C\) il suo contorno. Se il campo \(\vec{B}\) è costante sulla superficie della spira può essere portato fuori dal simbolo di integrale:

\[\vec{E} \cdot \oint_{C}{d\vec{C}} = - \dfrac{\partial}{\partial t}\vec{B} \cdot \int_{S}{d\vec{S}}\]

Ne discende che il campo elettrico è anch'esso uniforme sulla spira e parallelo, in ogni punto, a \(d\vec{C}\). Per una circonferenza risulta:

\[E2\pi r = \pi r^{2}\left( - \dfrac{\partial B}{\partial t} \right)\]

Dato che \(B\) è una funzione solo del tempo, la derivata parziale si riduce a una totale. Semplificando \(\pi r\) si ottiene:

\[E = - \dfrac{r}{2}\dfrac{dB}{dt}\]

Il campo elettrico produce un momento torcente sull'elettrone, dato dalla relazione:

\[\vec{\tau} = q\vec{r} \times \vec{E}\]

Dato che il campo elettrico è ortogonale al raggio, essendo ortogonale anche al campo magnetico \(\vec{B}\), il modulo del momento torcente, esplicitando anche la carica dell'elettrone, può essere espresso come:

\[\tau = - erE\]

Sostituendo il campo elettrico prima determinato, si ha:

\[\tau = e\dfrac{r^{2}}{2}\dfrac{dB}{dt}\]

Per la seconda legge di Newton \(dL\backslash dt = \tau\), si ottiene:

\[\dfrac{dL}{dt} = e\dfrac{r^{2}}{2}\dfrac{dB}{dt}\]

Si integra tra \(t_{0}\), tempo di applicazione del campo magnetico, e \(t_{1}\) tempo in cui il campo \(B\) raggiunge il suo valore massimo:

\[\int_{t_{0}}^{t_{1}}{\dfrac{dL}{dt}dt} = e\dfrac{r^{2}}{2}\int_{t_{0}}^{t_{1}}{\dfrac{dB}{dt}dt}\]

Risolvendo si ha:

\[L\left( t_{1} \right) - L\left( t_{0} \right) = e\dfrac{r^{2}}{2}\left\lbrack B\left( t_{1} \right) - B\left( t_{0} \right) \right\rbrack\]

Inizialmente il campo è nullo \(B\left( t_{0} \right) = 0\), per cui la differenza di momento angolare è data da:

\[\mathrm{\Delta}L = e\dfrac{r^{2}}{2}B\]

Dove \(B\) è il massimo valore del campo.

Il momento magnetico dell'elettrone intorno al nucleo è antiparallelo al momento angolare. Le due quantità sono legate dal rapporto giromagnetico:

\[\mathrm{\Delta}\mu = - \dfrac{e}{2m}\mathrm{\Delta}L\]

Sostituendo l'espressione della differenza del momento angolare, si ottiene:

\[\mathrm{\Delta}\mu = - \dfrac{e^{2}r^{2}}{4m}B\]

In linea di principio l'orbita descritta dall'elettrone dovrebbe variare per l'applicazione del campo magnetico.

\begin{figure}[ht]
\centering
\includegraphics[width=2.30915in,height=2.64167in,alt={P1568\#yIS1}]{media/3_Elettromagnetismo/image29.pdf}\caption{Elettrone in equilibrio sull'orbita}
\end{figure}

Un elettrone orbitante è mantenuto, infatti, in equilibrio dalla forza centripeta (\(F_{c}\)) e dell'iterazione coulombiana con il nucleo (\(F_{e}\)):

\[F_{e} = F_{c}\]

Sostituendo le relative espressioni nell'ipotesi che il nucleo sia composto da un solo elettrone, si ha:

\[m\dfrac{v^{2}}{r} = \dfrac{1}{4\pi\varepsilon_{0}}\dfrac{e^{2}}{r^{2}}\]

La variazione di velocità indotta dal campo elettrico dopo l'introduzione del campo \(dB\) è data dalla forza che il campo esercita:

\[F = \dfrac{dp}{dt} = m\dfrac{dv}{dt}\]

La forza può essere espressa in termini di campo elettrico. Dalla definizione di campo elettrico, per l'elettrone risulta:

\[E = \dfrac{F}{q} \Leftrightarrow F = - eE\]

Dunque, la variazione di quantità di moto può essere espressa come:

\[- eE = m\dfrac{dv}{dt}\]

Il campo elettrico, per la legge di Faraday, è legato al campo magnetico dalla relazione:

\[E = - \dfrac{r}{2}\dfrac{d\vec{B}}{dt}\]

Dunque, si ha:

\[- eE = m\dfrac{dv}{dt} \Leftrightarrow e\dfrac{r}{2}\dfrac{d\vec{B}}{dt} = m\dfrac{dv}{dt}\]

Si analizza la sola variazione di velocità subita dall'elettrone:

\[dv = \dfrac{er}{2m}dB\]

A causa della variazione di velocità, anche l'accelerazione centripeta \(\alpha\) varia. Ricorrendo allo sviluppo in serie di Taylor, trascurando gli ordini superiori al primo, si ha:

\[\alpha + d\alpha = \dfrac{v^{2}}{r} + d\left( \dfrac{v^{2}}{r} \right) = \dfrac{v^{2}}{r} + \dfrac{2vdv}{r}\  + o\left( \dfrac{1}{r^{2}}dr \right) + o\left( dv^{2} \right)\]

Dopo l'applicazione del campo magnetico, bisogna considerare anche la forza di Lorentz nel bilancio delle forze sull'elettrone. La forza di Lorentz e quella elettrostatica sono concordi, dunque, affinché l'elettrone sia in equilibrio, la velocità deve aumentare per incrementare la forza centripeta:

\[\dfrac{mv^{2}}{r} + \dfrac{2mvdv}{r} = \dfrac{1}{4\pi\varepsilon_{0}}\dfrac{e^{2}}{r^{2}} + evdB\]

Si è visto che:

\[dv = \dfrac{er}{2m}dB\]

Sostituendo nel bilancio, si ha:

\[\dfrac{mv^{2}}{r} + \dfrac{2mv}{r}\dfrac{er}{2m}dB = \dfrac{1}{4\pi\varepsilon_{0}}\dfrac{e^{2}}{r^{2}} + evdB \Leftrightarrow \dfrac{mv^{2}}{r} + vedB = \dfrac{1}{4\pi\varepsilon_{0}}\dfrac{e^{2}}{r^{2}} + evdB\]

Per cui si ottiene l'equazione dell'equilibrio prima dell'applicazione del campo esterno:

\[\dfrac{mv^{2}}{r} = \dfrac{1}{4\pi\varepsilon_{0}}\dfrac{e^{2}}{r^{2}}\]

Dopo l'applicazione del campo magnetico, l'equilibrio dell'elettrone è ottenuto mediante una sola variazione della velocità di rotazione. La variazione del raggio è influente per ordini superiori, dunque, in prima analisi, è perfettamente trascurabile. La cancellazione dei termini di ordine superiore ($\pm evdB$) dimostra che, se si ignora la variazione di raggio, la variazione di velocità indotta ($2mvdv/r$) è esattamente ciò che è richiesto dalla forza aggiuntiva di Lorentz ($evdB$) per mantenere l'equilibrio (cioè, $F_{centripeta} = F_{Lorentz} + F_{Coulomb}$).

Per i materiali diamagnetici la permeabilità relativa è circa unitaria, \(\mu_{r} \simeq 1\), con valori leggermente inferiori all'unità in modo da opporsi al campo applicato. Inoltre, essendo un fenomeno legato all'orbita elettronica, il diamagnetismo è presente in tutti i materiali.

\section{Paramagnetismo}\label{paramagnetismo}

A differenza del diamagnetismo sempre presente in ogni materiale, il paramagnetismo è presente solamente in quei materiali i cui atomi o molecole hanno un momento magnetico permanente ($\vec{\mu} \neq 0$), dovuto alla presenza di elettroni spaiati negli orbitali atomici.

Per il principio di esclusione di Pauli, gli elettroni negli orbitali atomici si dispongono con spin antiparallelo, così da cancellare gli effetti degli spin stessi. Negli atomi in cui sono presenti elettroni spaiati, ovvero con orbitali atomici non completamente riempiti, la cancellazione degli spin non avviene, dunque, lo spin netto dell'atomo è diverso da zero. 

A livello macroscopico, gli spin dei vari costituenti del materiale si sommano, dando luogo alla magnetizzazione macroscopica. Quest'ultima è influenzata dalla presenza o meno di un campo magnetico.

In assenza di un campo magnetico esterno, il momento magnetico di ogni atomo è orientato in modo casuale a causa dell'agitazione termica, dunque, la magnetizzazione macroscopica media ($\vec{M}$) del materiale è nulla.

Applicato un campo magnetico esterno, invece, i singoli momenti magnetici (spin) subiscono un momento torcente che tende ad allinearli nella direzione del campo. Essi iniziano a precedere attorno alla direzione del campo con la frequenza di Larmor:

\[\omega = \gamma B\]

\begin{figure}[ht]
\centering
\includegraphics[width=3.81482in,height=2.1701in,alt={P1604\#yIS1}]{media/3_Elettromagnetismo/image30.pdf}\caption{Spin nei materiali paramagnetici}
\end{figure}

Il paramagnetismo è presente solo in alcune sostanze ed è caratterizzato da una permeabilità magnetica relativa \(\mu_{r} \simeq 1\), lievemente maggiore dell'unità.

Si osservi che il vettore di magnetizzazione ($M$) è definito come la somma vettoriale dei momenti magnetici per unità di volume:

\[
\vec{M} = \dfrac{1}{V}\sum_{i}\vec{\mu}_i
\]

L'interazione del campo esterno con i momenti magnetici determina la magnetizzazione macroscopica $M$ del materiale, che in equilibrio termico è generalmente debole a causa del disorientamento termico.
\begin{center}
\vfill
    \chapter{Meccanica quantistica}
    \label{blx:Quantistica\therefsection}
\vfill

\minitoc
\newpage
\end{center}
\justify

\section{Crisi della Fisica Classica e Nascita della Meccanica Quantistica}
\label{crisi-meccanica-quantistica}

All'inizio del Novecento, il modello matematico basato sulla fisica classica (Meccanica Newtoniana ed Elettrodinamica di Maxwell) si rivelò inadeguato a descrivere con successo numerosi fenomeni osservati su scala microscopica. Queste \textbf{incongruenze} tra teoria ed esperimento segnarono l'inizio di una profonda revisione dei fondamenti della fisica.

La fisica classica non era in grado di spiegare in modo soddisfacente i seguenti fenomeni, che richiedevano l'introduzione della \textbf{quantizzazione}:
\begin{itemize}
\item
    \textbf{La Stabilità dell'Atomo di Rutherford}: Secondo l'elettrodinamica classica, un elettrone in orbita attorno al nucleo (in moto accelerato) dovrebbe emettere continuamente radiazione elettromagnetica, perdendo energia e spiraleggiando verso il nucleo nel giro di una frazione di secondo. Al contrario, gli atomi sono \textbf{strutture stabili};
\item
    \textbf{La Distribuzione Spettrale della Radiazione del Corpo Nero}: La teoria classica (legge di Rayleigh-Jeans) prevedeva che l'energia emessa da un corpo nero divergesse alle alte frequenze, portando a una quantità di energia infinita (la cosiddetta \textbf{catastrofe ultravioletta} o \textbf{Rayleigh-Jeans catastrophe}), in netto contrasto con i dati sperimentali. Questo problema fu risolto da Planck (1900) postulando che l'energia fosse emessa in pacchetti discreti ($E=h\nu$);
\item
    \textbf{L'Effetto Fotoelettrico}: La luce incidente su una superficie metallica può liberare elettroni solo se la sua frequenza supera un valore soglia specifico, indipendentemente dall'intensità della luce stessa. La teoria ondulatoria classica non poteva spiegare questa dipendenza dalla frequenza. Il fenomeno fu spiegato da Einstein (1905) introducendo il concetto di \textbf{fotone} (quanto di luce).
\end{itemize}

Parallelamente o immediatamente dopo la risoluzione di questi problemi, una serie di esperimenti fornirono ulteriori e definitive prove a sostegno della necessità di una nuova teoria quantistica:

\begin{itemize}
\item
    \textbf{Spettri di Emissione e Assorbimento Atomico}: Gli atomi emettono o assorbono luce solo a \textbf{frequenze discrete} e ben definite, anziché su uno spettro continuo. Questo risultato fu la prova diretta che l'energia degli elettroni all'interno dell'atomo è \textbf{quantizzata} in livelli specifici (modello di Bohr, 1913);
\item
    \textbf{Esperimento di Stern-Gerlach (1922)}: Un fascio di atomi (come l'argento) fatto passare attraverso un campo magnetico non uniforme si divide in un numero finito e discreto di componenti (storicamente, due). Ciò dimostrò la \textbf{quantizzazione spaziale} del momento angolare e, successivamente, l'esistenza dello \textbf{spin} (momento angolare intrinseco) dell'elettrone, che non ha alcun analogo classico;
\item
    \textbf{Diffrazione di Elettroni e Doppio Slit (Davisson-Germer, 1927)}: L'osservazione che particelle come gli elettroni producano figure di interferenza tipiche delle onde confermò la \textbf{dualità onda-particella} proposta da De Broglie (1924), evidenziando che le particelle microscopiche si comportano come onde in determinate circostanze.
\end{itemize}

Queste difficoltà e scoperte portarono alla necessità di rivedere i fondamenti della fisica. Nacque così la \textbf{Meccanica Quantistica}, una nuova teoria che descrive il comportamento della materia e dell'energia su scala microscopica, introducendo concetti fondamentali come la \textbf{quantizzazione}, la \textbf{probabilità} e la \textbf{dualità onda-particella} \cite{messiah1961quantum, wichmann1971quantistica, feynman1965vol3}.

\subsection{Stabilità dell'Atomo e Collasso Elettronico (Critica al Modello di Rutherford)}\label{stabilituxe0-dellelettrone}

Il problema della stabilità atomica rappresenta una delle incongruenze più acute tra teoria classica e osservazione.
Secondo l'elettrodinamica classica, formalizzata dalle equazioni di Maxwell, una carica elettrica in \textbf{moto accelerato} (come l'elettrone in orbita attorno al nucleo nel modello planetario di Rutherford) dovrebbe emettere continuamente energia sotto forma di \textbf{radiazione elettromagnetica}. 

Questo rilascio continuo di energia implicherebbe una rapida \textbf{perdita di energia cinetica} da parte dell'elettrone e una conseguente riduzione del raggio orbitale, portando al \textbf{collasso} dell'elettrone sul nucleo in un tempo stimato di circa $10^{-11}$ secondi.

Tuttavia, l'evidenza sperimentale dimostra che gli atomi sono strutture intrinsecamente \textbf{stabili}, che esistono indefinitamente nel tempo senza collassare. Questa lampante discrepanza richiese l'abbandono dei principi classici per la descrizione della materia a livello microscopico.

La prima risoluzione teorica di questa crisi fu proposta da \textbf{Niels Bohr nel 1913}, attraverso l'introduzione di \textbf{postulati quantistici}: l'elettrone può occupare solo determinate \textbf{orbite stazionarie} (o stati quantici) in cui, per definizione, \textbf{non emette alcuna radiazione}. L'emissione o l'assorbimento di energia avviene solo durante la transizione tra questi livelli discreti, sotto forma di quanti di energia (fotoni), in accordo con la relazione di Planck ($E = h\nu$).

\subsection{Radiazione del Corpo Nero e la Quantizzazione di Planck}\label{corpo-nero}

Il \textbf{corpo nero} è un oggetto ideale che assorbe completamente tutta la radiazione elettromagnetica incidente, senza rifletterla né trasmetterla. Di conseguenza, per la conservazione dell'energia, quando riscaldato, emette radiazione termica a ogni lunghezza d'onda, e la distribuzione di questa energia è funzione esclusiva della sua temperatura.

Sebbene il corpo nero sia un'astrazione teorica, può essere approssimato sperimentalmente da una cavità con pareti interne nere e una piccola apertura. La radiazione che entra ha una probabilità estremamente bassa di uscire, simulando il comportamento ideale.

\begin{figure}[ht]
\centering
\resizebox{0.5\textwidth}{!}{%
\begin{tikzpicture}[scale=1.3]

  \shade[top color=black!60,bottom color=black!80,shading angle=10] % background
    (7:1) arc (7:355:1);
  
  \fill[thick,black,postaction=decorate, % rough inner surface
    decoration={markings,mark=between positions 0.55 and 1 step 0.03 with {
                  \node[transform shape,inner sep=1pt]
                  (hit\pgfkeysvalueof{/pgf/decoration/mark info/sequence number}) {};
    }}]
    (7:1) arc (7:353:1) --++ (-7:-0.18)
    decorate[decoration={random steps,segment length=2,amplitude=1pt}]
        {arc (-7:-353:0.82)} -- cycle;

% Raggi che entrano e colpiscono la circonferenza
\draw[red,thick,->] (2.3,0) -- (0.9,0); % ingresso
\draw[red,thick,->] (0.9,0) -- (-0.1,-0.8); % primo rimbalzo
\draw[red,thick,->] (-0.1,-0.8) -- (0.6,0.5); % secondo rimbalzo
\draw[red,thick,->] (0.6,0.5) -- (-0.8,-0.1); % terzo rimbalzo
\draw[red,thick,->] (-0.8,-0.1) -- (-0.2,-0.51); % quarto rimbalzo

% Etichetta raggio
\node[red] at (3.5,0) {Raggio incidente};

\end{tikzpicture}
}
\caption{Modello pratico di corpo nero, come una cavità con una piccola apertura}
\label{fig:4_CorpoNero}
\end{figure}


Secondo la fisica classica, in particolare la teoria basata sulla legge di \textbf{Rayleigh-Jeans}, l'intensità della radiazione emessa per unità di frequenza ($I(\nu, T)$) dovrebbe aumentare indefinitamente con l'aumento della frequenza ($\nu$). Questa previsione teorica, che prevedeva una divergenza dell'energia alle alte frequenze, è nota come \textbf{catastrofe ultravioletta} e contraddiceva clamorosamente i risultati sperimentali.

\begin{figure}[ht]
\centering
\resizebox{0.5\textwidth}{!}{%

% BLACK BODY - 3000, 4000, 5000K, Wien's displacement law
\begin{tikzpicture}
% redraw axis on top
\makeatletter \newcommand{\pgfplotsdrawaxis}{\pgfplots@draw@axis} \makeatother
\pgfplotsset{axis line on top/.style={after end axis/.append code={\pgfplotsdrawaxis}}
}

% CUSTOM COLORS
% See https://tikz.net/blackbody_color/
\definecolor{1000K}{rgb}{1,0.0337,0}
\definecolor{2000K}{rgb}{1,0.2647,0.0033}
\definecolor{3000K}{rgb}{1,0.4870,0.1411}
\definecolor{4000K}{rgb}{1,0.6636,0.3583}
\definecolor{5000K}{rgb}{1,0.7992,0.6045}
\definecolor{6000K}{rgb}{1,0.9019,0.8473}
\definecolor{8000K}{rgb}{0.7874,0.8187,1}
\definecolor{10000K}{rgb}{0.6268,0.7039,1}
\pgfdeclareverticalshading{rainbow}{100bp}{
  color(0bp)=(red); color(25bp)=(red); color(35bp)=(yellow);
  color(45bp)=(green); color(55bp)=(cyan); color(65bp)=(blue);
  color(75bp)=(violet); color(100bp)=(violet)
}
\colorlet{myred}{red!70!black}
\colorlet{mygreen}{green!70!black}
\colorlet{mydarkgreen}{green!55!black}

% PLANCK & RAYLEIGH-JEANS
% 2hc^2/lambda^5 = 2 * 6.62607015e-34 * 299792458^2
%                = 1.191042972e-16
%    W.m -> kW.nm: 1.191042972e26
%  hc/k lambda T = 6.62607015e-34*299792458/(1.38064852e-23)
%                = 0.01438777378
%         m -> nm: 0.01438777378e9
% 2ckT/lambda^4  = 2 * 299792458 * 1.38064852e-23
%                = 8.278160269e-15
%    W.m -> kW.nm: 8.278160269e18
\pgfmathdeclarefunction{planck}{2}{%
  \pgfmathparse{1.191042972e26/(#1^5)/(exp(0.01439e9/(#1*#2))-1)}%
}
\pgfmathdeclarefunction{rayleighjeans}{2}{%
  \pgfmathparse{8.278160269e18*#2/(#1^4)}%
}
\pgfmathdeclarefunction{wien}{2}{%
  \pgfmathparse{1.191042972e26/(#1^5)*exp(-0.01439e9/(#1*#2))}%
}
\pgfmathdeclarefunction{lampeak}{1}{% % Wien's displacement law
  \pgfmathparse{2.898e6/#1}%
}
  \message{^^JBlack body, Wien's displacement law}
  \def\N{60}
  \def\xmax{3100}
  \def\ymax{1.36e10}
  \def\tick#1#2{\draw[thick] (#1+.01*\ymax) -- (#1-.01*\ymax) node[below=-.5pt,scale=0.75] {#2};}
  \begin{axis}[
      every axis plot/.style={
        very thick,mark=none,samples=\N,domain=5:\xmax,smooth},
      xmin=(-.05*\xmax), xmax=(1.05*\xmax),
      ymin=(-.08*\ymax), ymax=(1.08*\ymax),
      restrict y to domain=0:\ymax,
      axis lines=middle,
      axis line style=thick,
      %enlargelimits=upper, % extend the axes a bit to the right and top
      tick style={black,thick},
      ticklabel style={scale=0.8},
      %xtick style={draw=none},xticklabels=none,
      max space between ticks=26,
      xlabel={Wavelength $\lambda$ [nm]},
      ylabel={Power $P$ [kW/sr\,m$^2$\,nm]},
      xlabel style={at={(rel axis cs:0.5,0)},below=-1pt,font=\small},
      ylabel style={at={(rel axis cs:-0.11,0.5)},rotate=90},
      width=9cm, height=7cm,
      %clip=false
      tick scale binop=\times,
      every y tick scale label/.style={at={(rel axis cs:0,1)},anchor=south}]
    ]
    
    % RAINBOW
    \shade[shading=rainbow,shading angle=90,opacity=0.5] (380,0) rectangle (740,\ymax);
    \node[above=-1pt,scale=0.8] at (200,\ymax) {\strut UV}; % 10 - 400 nm
    \node[above=-1pt,scale=0.8] at (570,\ymax) {\strut optical}; % 380 - 740 nm
    \node[above=-1pt,scale=0.8] at (920,\ymax) {\strut IR}; % 740 - 1050 nm
    
    % PLANCK
    \addplot[red]    {planck(x,3000)};
    \addplot[orange] {planck(x,4000)};
    \addplot[blue,samples=3*\N] {planck(x,5000)};
    \addplot[dashed,thick,blue,domain=1000:3500] {rayleighjeans(x,5000)};
    
    % MAXIMUM (Wien's displacement law)
    \addplot[mydarkgreen,thick,variable=T,domain=2200:4000,samples=40]
      ({lampeak(T)},{planck(lampeak(T),T)});
    \addplot[mydarkgreen,thick,variable=T,domain=4000:5200,samples=100]
      ({lampeak(T)},{planck(lampeak(T),T)});
    \fill[mydarkgreen!80!black] ({lampeak(3000)},{planck(lampeak(3000),3000)}) circle(1.5pt);
    \fill[mydarkgreen!80!black] ({lampeak(4000)},{planck(lampeak(4000),4000)}) circle(1.5pt);
    \fill[mydarkgreen!80!black] ({lampeak(5000)},{planck(lampeak(5000),5000)}) circle(1.5pt);
    
    % LABELS
    \node[above=0pt,scale=0.75,red] at (1150,{planck(1150,3000)}) {\SI{3000}{K}};
    \node[above right=-1pt,scale=0.75,orange!80!black] at (740,{planck(740,4000)}) {\SI{4000}{K}};
    \node[above right=-1pt,scale=0.75,blue] at (800,{planck(800,5000)}) {\SI{5000}{K}};
    \node[above right=-1pt,scale=0.75,blue] at (1500,{rayleighjeans(1500,5000)}) {\SI{5000}{K} Rayleigh-Jeans};
    \node[above right=-1pt,scale=0.75,blue] at (1600,{rayleighjeans(1430,5000)}) {Ultraviolet Catastrophe};
    
  \end{axis}
\end{tikzpicture}

}
\caption{Confronto tra la previsione classica (Rayleigh-Jeans) e i dati sperimentali che mostrano la catastrofe ultravioletta}
\label{fig:4_CatastrofeUV}
\end{figure}

In realtà, lo spettro di emissione del corpo nero dipende strettamente dalla temperatura, presentando un andamento continuo e finito: l'intensità raggiunge un massimo a una specifica lunghezza d'onda, che si sposta verso valori più corti all'aumentare della temperatura (Legge dello spostamento di Wien), per poi decrescere.

La risoluzione teorica di questo problema fondamentale fu proposta da \textbf{Max Planck nel 1900}, segnando la nascita della meccanica quantistica. Planck ipotizzò che l'energia scambiata tra gli oscillatori atomici delle pareti della cavità e la radiazione non fosse continua, ma avvenisse in pacchetti discreti o \textit{quanti} di energia.

L'energia emessa o assorbita è quindi un multiplo intero ($n$) di una quantità elementare, proporzionale alla frequenza ($\nu$) della radiazione:

\[
E = h\nu
\]

dove $E$ è l'energia del quanto (per $n=1$), $\nu$ è la frequenza della radiazione, e $h$ è la \textbf{costante di Planck}. Questa ipotesi permise a Planck di derivare una formula che descriveva correttamente l'intero spettro osservato.

\subsection{Effetto fotoelettrico}\label{effetto-fotoelettrico}

L'effetto fotoelettrico è un fenomeno di interazione tra la radiazione elettromagnetica e la materia, caratterizzato dall'emissione di elettroni (detti \textbf{fotoelettroni}) da una superficie, generalmente metallica, quando irradiata da energia elettromagnetica. 

Sperimentalmente, l'effetto si osserva ponendo due elettrodi metallici all'interno di un'ampolla sotto vuoto spinto. Quando l'elettrodo (catodo) viene irradiato, gli elettroni espulsi generano una \textbf{corrente elettrica} rilevabile sull'elettrodo opposto (anodo).

\begin{figure}[ht]
\centering
\includegraphics[width=2.98611in,height=2.19407in,alt={P1636\#yIS1}]{media/4_Quantiatica/image33.pdf}\caption{Circuito per l'effetto fotoelettrico}
\end{figure}

Gli elettroni emessi dall'elettrodo irradiato, dotati di energia cinetica non nulla, vengono accelerati dal campo elettrico generato da una batteria, raggiungendo il secondo elettrodo collegato al polo positivo. In questo modo, si chiude il circuito elettrico e si rileva una corrente.

L'energia cinetica massima ($K_{\max}$) dei fotoelettroni può essere misurata applicando una \textbf{tensione di arresto} ($V_s$): una differenza di potenziale opposta e sufficiente ad annullare la corrente, impedendo anche agli elettroni più energetici di raggiungere l'anodo. L'energia cinetica massima è quindi data da $K_{\max} = e V_s$.

Si applica una differenza di potenziale tale da bloccare la circolazione di corrente nel circuito, nonostante l'irraggiamento dell'elettrodo. In questa situazione, un elettrone, espulso per effetto fotoelettrico con una certa energia cinetica, si trova in un campo elettrico che lo decelera fino a farlo tornare sull'elettrodo che l'ha prodotto. Nota la tensione applicata, si riesce a determinare l'energia dell'elettrone espulso.

Secondo la teoria ondulatoria classica, l'energia trasferita agli elettroni e, di conseguenza, la loro energia cinetica massima ($K_{\max}$), avrebbe dovuto:
\begin{enumerate}
    \item Dipendere dall'\textbf{intensità} della radiazione (una luce più brillante avrebbe dovuto liberare elettroni più energetici).
    \item Manifestarsi dopo un certo \textbf{ritardo temporale} per accumulare l'energia necessaria.
\end{enumerate}
L'esperimento smentì entrambe le previsioni: l'emissione è \textbf{istantanea} e $K_{\max}$ dipende esclusivamente dalla \textbf{frequenza} ($\nu$) della radiazione incidente, e non dalla sua intensità. Inoltre, esiste una \textbf{frequenza di soglia} ($\nu_0$) al di sotto della quale l'emissione non avviene affatto, anche ad alta intensità.


Albert Einstein risolse questa incongruenza nel 1905, estendendo l'ipotesi di Planck: la luce non si comporta solo come un'onda, ma è composta da particelle discrete, i \textbf{fotoni} (quanti di luce), ciascuno con energia:

\[
E = h\nu
\]

dove $h$ è la costante di Planck e $\nu$ è la frequenza. L'emissione di un elettrone avviene in un processo di interazione \textbf{uno-a-uno} tra un singolo fotone e un singolo elettrone.

L'equazione fondamentale che descrive l'effetto fotoelettrico è:

$$K_{\max} = E - \phi = h\nu - \phi$$

dove $K_{\max}$ è l'energia cinetica massima dell'elettrone espulso, $h\nu$ è l'energia del fotone incidente, e $\mathbf{\phi}$ (o $W_0$) è la \textbf{funzione lavoro} (l'energia minima necessaria per estrarre l'elettrone dalla superficie metallica).

Questa spiegazione, che attribuiva alla luce una \textbf{natura corpuscolare} (dualità onda-particella), valse ad Einstein il Premio Nobel per la Fisica nel 1921.

\subsection{L'Esperimento della Doppia Fenditura e la Dualità Onda-Particella}\label{double-slit}

Mentre la crisi della fisica classica era iniziata con la natura quantizzata della luce (fotoni, 1905), l'esperimento della \textbf{doppia fenditura} (o \textbf{double slit}) ha esteso il dualismo alla \textbf{materia}, rivelando la natura ondulatoria di particelle come gli elettroni. Questo risultato ha messo definitivamente in crisi la visione strettamente corpuscolare della materia.

Nell'esperimento cruciale, un fascio di elettroni (o altre particelle) viene inviato contro una parete dotata di due fenditure parallele. Sullo schermo di rilevazione posto dietro le fenditure, anziché osservare le due bande nette previste per le particelle classiche, si forma una \textbf{figura di interferenza}: una sequenza di frange chiare (interferenza costruttiva) e scure (interferenza distruttiva), un comportamento tipico ed esclusivo dei fenomeni ondulatori.

\begin{figure}[ht]
\centering
\includegraphics[width=3.02778in,height=2.29079in,alt={P1647\#yIS1}]{media/4_Quantiatica/image34.pdf}\caption{Figura di interferenza della doppia fenditura}
\end{figure}

Il comportamento degli elettroni in questo esperimento può essere interpretato attraverso il principio di Huygens dei fenomeni ondulatori, secondo cui ogni punto \(d\Sigma\) di un fronte d'onda \(\Sigma\) può essere considerato come sorgente secondaria di onde sferiche. La perturbazione risultante in un punto dello spazio è data dalla sovrapposizione di tutte le onde secondarie che vi giungono.

Il risultato è sorprendente perché, anche inviando gli elettroni \textbf{uno alla volta}, il pattern di interferenza si accumula gradualmente, suggerendo che ogni particella interferisca in qualche modo \textbf{con sé stessa}.

Gli elettroni sono particelle dotate di massa, infatti, in altri esperimenti, come l'effetto fotoelettrici, gli elettroni interagiscono con i fotoni come particelle, attraverso urti elastici. Questi due comportamenti apparentemente contraddittori (particella in urti elastici come nell'effetto fotoelettrico; onda nella doppia fenditura) hanno portato alla formulazione del concetto di \textbf{dualismo onda-corpuscolo}.

Nel 1924, \textbf{Louis de Broglie} propose l'ipotesi che a ogni particella materiale, con quantità di moto $p$, fosse associata un'onda, definita \textbf{onda di materia}, la cui lunghezza d'onda ($\lambda$) è inversamente proporzionale alla quantità di moto stessa:

\[
\lambda = \dfrac{h}{p}
\]

dove $h$ è la costante di Planck. Questa ipotesi teorica fu confermata sperimentalmente nel 1927 dagli esperimenti di \textbf{Davisson e Germer} e da \textbf{G.P. Thomson} sulla diffrazione di elettroni, che rappresenta uno dei pilastri fondamentali della meccanica quantistica.

\subsection{Esperimento di Stern e Gerlach: La Quantizzazione dello Spin}\label{esperimento-di-stern-e-gerlach}

L'esperimento di Stern e Gerlach, condotto nel 1922, ha fornito una delle prove sperimentali più dirette e convincenti della \textbf{quantizzazione del momento angolare} e ha rivelato l'esistenza di una proprietà intrinseca delle particelle: lo \textbf{spin}.

Nel setup sperimentale, un fascio di atomi d'argento (il cui momento angolare deriva quasi interamente dall'unico elettrone nel guscio esterno) viene fatto passare attraverso un \textbf{campo magnetico fortemente non uniforme}. 

Secondo la fisica classica, il momento magnetico degli atomi, orientandosi in modo continuo rispetto al campo, avrebbe dovuto produrre un'unica linea o una \textbf{distribuzione continua} sullo schermo di rilevazione.

Tuttavia, ciò che si osserva è una \textbf{separazione netta} del fascio in \textbf{due sole componenti distinte}. Questo risultato indica inequivocabilmente che il momento magnetico degli atomi, e quindi il loro momento angolare totale $L$, non può assumere valori arbitrari (come previsto dalla classica), ma solo un insieme finito e discreto di orientamenti spaziali.

In particolare, per gli atomi d'argento, la separazione in due componenti dimostra la quantizzazione dello \textbf{spin} dell'elettrone, per cui la sua proiezione lungo la direzione del campo può assumere solo due valori: $m_s = +1/2$ (spin "su") e $m_s = -1/2$ (spin "giù").

Questo esperimento ha dimostrato che:

\begin{itemize}
\item Il momento angolare (e il momento magnetico associato) è \textbf{quantizzato} spazialmente (quantizzazione spaziale).
\item Lo \textbf{spin} è una proprietà quantistica intrinseca e fondamentale delle particelle subatomiche, priva di analogo classico.
\item Le misure di grandezze quantistiche non danno risultati continui, ma solo \textbf{valori discreti} (autovalori), confermando il carattere non deterministico e non continuo della realtà microscopica.
\end{itemize}

L'esperimento di Stern-Gerlach è un pilastro della meccanica quantistica, poiché mostra in modo diretto la discrezione degli stati quantici.

\subsection{Spettri di Assorbimento e di Emissione: Evidenza della Quantizzazione Atomica}\label{spettro-di-assorbimento}

Lo \textbf{spettro di assorbimento} di un elemento chimico è l'insieme delle radiazioni elettromagnetiche assorbite quando la sostanza viene esposta a una sorgente luminosa a spettro continuo. Le sostanze assorbono energia solamente a determinate frequenze specifiche, dando origine a uno \textbf{spettro a righe scure} caratteristico.

Nei primi anni del Novecento, l'incapacità della meccanica classica di prevedere questi spettri a righe rappresentava un'ulteriore prova della sua inadeguatezza. Secondo la fisica classica, un atomo dovrebbe assorbire o emettere radiazioni su tutte le frequenze. Invece, l'osservazione che l'assorbimento avvenisse solo a frequenze discrete, indipendentemente dall'intensità, suggeriva l'esistenza di un meccanismo interno \textbf{quantizzato}.

Lo spettro risultante, osservato a valle dell'esposizione, è dato dalla radiazione incidente privata delle lunghezze d'onda assorbite dal materiale. In altre parole, lo spettro presenta delle bande scure, corrispondenti alle frequenze assorbite.

\begin{figure}[ht]
\centering
\includegraphics[width=3.34722in,height=2.81542in,alt={P1658\#yIS1}]{media/4_Quantiatica/image35.pdf}\caption{Spettro di assorbimento e di emissione}
\end{figure}

Già prima della formulazione teorica, la regolarità degli spettri atomici fu descritta empiricamente. In particolare, per l'atomo di idrogeno, \textbf{Johannes Rydberg} propose una formula per calcolare le lunghezze d'onda delle righe spettrali (sia di assorbimento che di emissione):

\[
\dfrac{1}{\lambda} = R_{H}\left( \dfrac{1}{n_{1}^{2}} - \dfrac{1}{n_{2}^{2}} \right)
\]

dove $R_H$ è la costante di Rydberg e $n_1, n_2$ sono \textbf{numeri interi positivi} ($n_1, n_2 \in \mathbb{N}$, con $n_2 > n_1$). La dipendenza da numeri interi era un indizio potentissimo dell'esistenza di \textbf{livelli energetici discreti} all'interno dell'atomo.

Sebbene la formula di Rydberg fosse predittiva, non ne spiegava il fondamento fisico. La spiegazione teorica arrivò nel 1913 con il \textbf{modello atomico di Bohr}. Bohr postulò che:

\begin{itemize}
    \item L'energia dell'elettrone è \textbf{quantizzata} e confinata in livelli specifici ($E_n$);
    \item L'assorbimento (o l'emissione) di radiazione avviene solo quando l'elettrone compie una \textbf{transizione quantica} tra due di questi livelli discreti, con la frequenza del fotone emesso/assorbito data dalla differenza di energia tra i livelli: $\nu = (E_2 - E_1)/h$.
\end{itemize}

Il modello di Bohr fornì la prima giustificazione fisica della formula di Rydberg, legando i numeri interi $n_1$ e $n_2$ ai numeri quantici dei livelli energetici stazionari dell'atomo.

\section{Teoria della meccanica quantistica}\label{teoria-della-meccanica-quantistica}

La Meccanica Quantistica (MQ) costituisce il quadro teorico fondamentale per la descrizione del comportamento della materia e dell'energia su scala microscopica. In fisica teorica, la MQ viene sviluppata attraverso due principali formulazioni:

\begin{itemize}
    \item La MQ \textbf{non relativistica}: sufficiente per descrivere la struttura degli atomi, le molecole e la dinamica delle particelle che si muovono a velocità ben inferiori a quella della luce ($v \ll c$). Questa modellazione si basa sull'equazione di Schrödinger.
    \item La MQ \textbf{relativistica}: necessaria per trattare particelle che si muovono a velocità prossime a $c$ e per le interazioni ad alta energia, un campo noto come \textbf{Teoria Quantistica dei Campi} (QFT).
\end{itemize}

La teoria quantistica relativistica più celebre e di successo è l'\textbf{Elettrodinamica Quantistica} (QED, \textit{Quantum Electrodynamics}). Essa descrive con straordinaria precisione le interazioni elettromagnetiche, modellando il comportamento dei fotoni (mediatori del campo) e le loro interazioni con le particelle cariche, come gli elettroni. 

\begin{figure}[h!]
    \centering
    \resizebox{1\textwidth}{!}{
        \begin{tikzpicture}[>=stealth, thick]

%--- Direct Channel ---
% Fermions in ingresso
\node[left] (a1) at (-2,1) {$e^-$};
\node[left] (a2) at (-2,-1) {$e^-$};

% Fermions in uscita
\node[right] (b1) at (2,1) {$e^-$};
\node[right] (b2) at (2,-1) {$e^-$};

% Vertici
\node (v1) at (0,1) {};
\node (v2) at (0,-1) {};

% Linee fermioniche
\draw[->] (a1) -- (v1);
\draw[->] (a2) -- (v2);
\draw[->] (v1) -- (b1);
\draw[->] (v2) -- (b2);

% Fotone
\draw[decorate, decoration={snake}] (v1) -- node[right] {$\gamma$} (v2);

% Titolo
\node at (0,-2.2) {Direct Channel};

\end{tikzpicture}
\hspace{2cm}
\begin{tikzpicture}[>=stealth, thick]

%--- Exchange Channel ---
% Fermions in ingresso
\node[left] (c1) at (-2,1) {$e^-$};
\node[left] (c2) at (-2,-1) {$e^-$};

% Fermions in uscita
\node[right] (d1) at (2,1) {$e^-$};
\node[right] (d2) at (2,-1) {$e^-$};

% Vertici
\node (v3) at (0,1) {};
\node (v4) at (0,-1) {};

% Linee fermioniche con scambio
\draw[->] (c1) -- (v3);
\draw[->] (c2) -- (v4);
\draw[->] (v3) -- (d2);
\draw[->] (v4) -- (d1);

% Fotone
\draw[decorate, decoration={snake}] (v3) -- node[right] {$\gamma$} (v4);

% Titolo
\node at (0,-2.2) {Exchange Channel};

\end{tikzpicture}

    }
    \caption{Diagrammi di Feynman per lo scattering elettrone-elettrone (Møller). Il diagramma a sinistra rappresenta il canale diretto, e il diagramma a destra rappresenta il canale di scambio.}
    \label{fig:electron_electron_scattering_arranged}
\end{figure}

Una delle previsioni cruciali della formulazione relativistica della MQ (in particolare, l'equazione di Dirac del 1928, fondamento della QED) è l'esistenza delle \textbf{antiparticelle}, entità con carica e numeri quantici opposti rispetto alla materia ordinaria. Un esempio è il \textbf{positrone} ($e^{+}$), l'antiparticella dell'elettrone, scoperto sperimentalmente nel 1932.

La Meccanica Quantistica non relativistica si è sviluppata a partire dai contributi pionieristici di:
\begin{itemize}
    \item \textbf{Max Planck} (1900): con l'ipotesi della quantizzazione dell'energia ($E=h\nu$) per spiegare la radiazione del corpo nero.
    \item \textbf{Albert Einstein} (1905): con l'interpretazione quantistica dell'effetto fotoelettrico, introducendo il concetto di fotone.
    \item \textbf{Louis de Broglie} (1924): con la proposta della dualità onda-corpuscolo per la materia.
\end{itemize}

Questi contributi hanno posto le basi per la formulazione moderna della MQ, una teoria che abbandona il determinismo classico per descrivere il comportamento delle particelle microscopiche in termini \textbf{probabilistici}, introducendo concetti come la \textbf{funzione d’onda} ($\Psi$), il \textbf{principio di indeterminazione} (Heisenberg) e la \textbf{quantizzazione degli stati energetici} \cite{dirac1930principles, messiah1961quantum, wichmann1971quantistica, feynman1965vol3, landau1975quantistica_rel}.

\subsection{Quantizzazione della materia}\label{quantizzazione-della-materia}

L'ipotesi fondamentale della meccanica quantistica è che le grandezze fisiche a livello microscopico siano \textbf{quantizzate}: esse non variano in modo continuo, ma possono assumere solo \textbf{valori discreti} (autovalori), spesso multipli interi di una quantità fondamentale. Questo principio, inizialmente suggerito da Planck per l'energia della radiazione e da Einstein per il fotone, è cruciale per descrivere la struttura della materia.

Gli elettroni all'interno di un atomo occupano \textbf{livelli energetici stazionari} ($E_1, E_2, \ldots$) che rappresentano stati quantici definiti. Poiché gli elettroni possono occupare solo questi livelli quantizzati, non sono soggetti al collasso sul nucleo (come previsto dalla fisica classica), garantendo la \textbf{stabilità dell'atomo}.

Un fotone incidente sull'atomo può essere assorbito dall'elettrone solo se la sua energia è \textbf{esattamente uguale} alla differenza di energia tra due livelli ammissibili ($E_f$ e $E_i$):

\[
E_{\text{fotone}} = E_f - E_i
\]

L'energia del fotone è data dalla relazione di Planck:

\[
E = h\nu
\]

dove $E$ è l'energia del fotone, $\nu$ è la sua frequenza, e $h$ è la costante di Planck. Da questa condizione deriva il fenomeno degli \textbf{spettri a righe}: l'atomo assorbe o emette radiazione solo alle frequenze $\nu$ ben definite, come dimostrato dalla formula empirica di Rydberg.

Se il fotone viene assorbito, l'elettrone subisce una transizione (eccitazione) a un livello energetico superiore ($E_f$). Successivamente, l'elettrone può ritornare a un livello inferiore (rilassamento), emettendo un fotone con un'energia pari alla differenza energetica tra i due stati.

\subsection{Dualismo Onda-Particella (Ipotesi di De Broglie)}\label{dualismo-onda-particella} 
 
L'ipotesi di \textbf{Louis de Broglie}, formulata nel 1924, ha esteso il dualismo onda-corpuscolo (già riconosciuto per la luce) alla \textbf{materia}. Secondo questa ipotesi, a ogni particella materiale, come l'elettrone, è associata un'onda (l'\textbf{onda di materia}), descritta da una \textbf{funzione d'onda} $\Psi(\vec{r}, t)$, che racchiude le informazioni sullo stato quantico della particella. 
 
Le relazioni di De Broglie stabiliscono il legame quantitativo tra le proprietà corpuscolari (Energia $E$, Quantità di moto $p$) e le proprietà ondulatorie (Frequenza $\nu$, Lunghezza d'onda $\lambda$). 
 
L'energia ($E$) della particella è legata alla frequenza ($\nu$) e alla pulsazione ($\omega$) dell'onda associata dalla relazione di Planck: 

\[
E = h\nu
\]

dove \(E\) è l'energia associata all'onda, \(\nu\) è la frequenza, e \(h\) è la costante di Planck. La pulsazione è legata alla frequenza da:

\[
\omega = 2\pi\nu
\]

Sostituendo nella formula dell'energia, si ottiene:

\[
E = \dfrac{h}{2\pi}\omega = \hslash\omega
\]

dove:

\[
\hslash = \dfrac{h}{2\pi}
\]

è la \textbf{costante di Planck ridotta}. 

Alla particella è anche associata una quantità di moto \(p\). Per l'ipotesi di de Broglie, la quantità di moto ($p$) è legata al \textbf{numero d'onda} ($k$) e alla lunghezza d'onda ($\lambda$): 

\[
p = \hslash k
\]

dove \(k\) è il \textbf{numero d'onda} definito come:

\[
k = \dfrac{2\pi}{\lambda}
\]

Sostituendo la definizione di $\hslash$ e $k$, si ottiene la relazione fondamentale che quantifica il dualismo:

\[
\lambda = \dfrac{h}{p}
\]

Questa formula dimostra che maggiore è la quantità di moto della particella, minore è la lunghezza d'onda associata, spiegando perché gli effetti ondulatori non siano osservabili su scala macroscopica. 
 
Mentre le relazioni di De Broglie si applicano a tutte le particelle, il loro contesto energetico è definito dalla relatività. Per una particella con massa a riposo $m_0$, la relazione completa tra energia e quantità di moto è: 

\[
E^2 = (pc)^2 + (m_0c^2)^2
\]

In \textbf{regime non relativistico} ($v \ll c$), l'energia cinetica domina, e la relazione $\lambda = h/p$ è l'approssimazione valida. Il \textbf{fotone} ($m_0=0$) rappresenta il limite estremo, dove la relazione $p = h/\lambda$ è ottenuta direttamente dalle leggi dell'elettromagnetismo e di Planck ($p = E/c = h\nu/c$). 
 
Queste relazioni confermano in modo definitivo il \textbf{dualismo onda-corpuscolo} alla base della meccanica quantistica. 

\subsection{Funzione d'onda associata alla particella}\label{funzione-donda-associata-alla-particella}

L'ipotesi di de Broglie stabilisce che a ogni particella (elettrone, protone, ecc.) è associata una \textbf{funzione d'onda} $\Psi(\vec{r}, t)$, che ne descrive lo stato quantico. Nel caso più semplice di una \textbf{particella libera} in moto (ad esempio, lungo l'asse $x$), la funzione d'onda assume la forma di un'onda piana, espressa mediante un esponenziale complesso:

\[
\Psi(x,t) = A \exp\left( j(kx - \omega t) \right)
\]

dove $A$ è l'ampiezza dell'onda, \(k\) è il numero d'onda, \(\omega\) è la pulsazione, e \(j\) è l'unità immaginaria (\(j^2 = -1\)). Una qualsiasi forma d'onda può essere ottenuta come sovrapposizione di infinite onde piane.

I parametri \(k\) (numero d'onda) e \(\omega\) (pulsazione) possono essere espressi in funzione delle grandezze fisiche della particella:

\[
k = \dfrac{p}{\hslash} \quad \text{e} \quad \omega = \dfrac{E}{\hslash}
\]

Sostituendo i parametri ondulatori ($k$ e $\omega$) con le grandezze fisiche della particella (quantità di moto $p$ ed Energia $E$) tramite la relazione di De Broglie ($k = p/\hslash$ e $\omega = E/\hslash$), si ottiene:

\[
\Psi(x,t) = A \exp\left( \dfrac{j}{\hslash}\left(px - Et\right) \right)
\]

Nel caso tridimensionale, l'espressione è generalizzata utilizzando il prodotto scalare tra il vettore quantità di moto $\vec{p}$ e il vettore posizione $\vec{r}$:

\[
\Psi(\vec{r},t) = A \exp\left( \dfrac{j}{\hslash} \left(\vec{p} \cdot \vec{r} - Et\right) \right)
\]

Secondo l'\textbf{interpretazione di Born}, la funzione d'onda $\Psi(\vec{r}, t)$ non ha un significato fisico diretto, ma il suo modulo quadro, $|\Psi(\vec{r}, t)|^2$, rappresenta la \textbf{densità di probabilità} $P(\vec{r}, t)$ di trovare la particella nella posizione $\vec{r}$ al tempo $t$:

\[
P(\vec{r},t) = \left| \Psi(\vec{r},t) \right|^{2}
\]

Nel caso specifico dell'onda piana, la probabilità risulta uniforme in tutto lo spazio ($|\Psi|^2 = |A|^2 = \text{costante}$). Tuttavia, tale funzione si estende all'infinito e non è normalizzabile, pertanto non rappresenta una situazione fisica realistica di una particella localizzata. Per descrivere particelle localizzate, si utilizza il concetto di \textbf{pacchetti d’onda}, ottenuti come sovrapposizione di infinite onde piane con diverse lunghezze d'onda (e quindi diverse quantità di moto).

In questo contesto probabilistico, il concetto classico di orbita (inteso come traiettoria deterministica) non è più applicabile. La regione dello spazio in cui la probabilità di trovare l'elettrone è massima viene definita \textbf{orbitale} (per gli stati legati negli atomi).

\section{Principio di indeterminazione di Heisenberg}\label{principio-di-indeterminazione-di-heisenberg}

Il principio di indeterminazione di Heisenberg è coerente con l’interpretazione probabilistica della meccanica quantistica proposta da Born. Secondo questo principio, non è possibile conoscere con precisione e simultaneamente la posizione e la quantità di moto di una particella. Indicando con \({\Delta}x\) l’incertezza sulla posizione e con \({\Delta}p\) quella sulla quantità di moto, si ha:

\[
{\Delta}x {\Delta}p \geq \dfrac{\hslash}{2}
\]

Questo significa che, se la posizione viene determinata con estrema precisione (\({\Delta}x \rightarrow 0\)), l’incertezza sulla quantità di moto deve aumentare (\({\Delta}p \rightarrow \infty\)) per mantenere valida la disuguaglianza. Analogamente, maggiore è la precisione sulla misura della quantità di moto meno precisa è la conoscenza della posizione, ovvero \({\Delta}p \rightarrow 0\) allora \({\Delta}x \rightarrow \infty\).

Il principio può essere esteso anche alla coppia energia-tempo, come:

\[
{\Delta}E {\Delta}t \geq \dfrac{\hslash}{2}
\]

In questo contesto, $\Delta t$ rappresenta l'intervallo di tempo durante il quale l'energia del sistema è definita. Questo significa che una misura precisa dell’energia ($\Delta E \rightarrow 0$) richiede un intervallo di tempo $\Delta t$ sufficientemente lungo (tendente a infinito), mentre una misura molto rapida ($\Delta t$ piccolo) comporta una maggiore incertezza sull’energia ($\Delta E$ grande).

Questo principio rappresenta un limite fondamentale alla conoscenza dello stato di una particella e riflette la natura intrinsecamente probabilistica della meccanica quantistica.

L’indeterminazione può essere interpretata come una conseguenza del processo di misura. Per determinare la posizione di una particella, ad esempio un elettrone, è necessario interagire con essa, ad esempio mediante fotoni. Questa interazione modifica inevitabilmente lo stato della particella. Nel mondo macroscopico, tale effetto è trascurabile, poiché l’energia dei fotoni è molto inferiore rispetto a quella degli oggetti con cui interagiscono, e quindi non ne altera significativamente lo stato.

\section{Equazione di Schrödinger}\label{equazione-di-schruxf6dinger}

La meccanica ondulatoria di Schrödinger è una teoria quantistica non relativistica, valida per particelle che si muovono a velocità molto inferiori a quella della luce \(c\). In questa descrizione si trascurano fenomeni come la creazione e l'annichilazione di particelle, poiché le energie coinvolte in tali processi sono troppo elevate per essere trattate nel contesto non relativistico della meccanica ondulatoria. Si osservi che il fotone è l'unica particella che può essere creata e distrutta con semplicità, tramite i fenomeni di emissione e assorbimento. Questi fenomeni sono descritti dalla meccanica ondulatoria.

La teoria si applica principalmente agli stati stazionari delle particelle e rappresenta la base della meccanica quantistica non relativistica. Da essa sono state sviluppate estensioni relativistiche, come l’equazione di Klein-Gordon e quella di Dirac, quest’ultima capace di prevedere l’esistenza del positrone \cite{dirac1930principles, landau1975quantistica_rel}.

La funzione d’onda associata a una particella libera può essere espressa come:

\[
\Psi(\vec{r},t) = \exp\left( \dfrac{j}{\hslash}(\vec{p} \cdot \vec{r} - Et) \right)
\]

La funzione d’onda deve soddisfare un’equazione differenziale che descriva la sua evoluzione nel tempo. Per l'ipotesi di Schrödinger, la funzione d'onda contiene tutte le informazioni necessarie a definire il moto della particella, per questo motivo l'equazione che permette di ricavare la funzione d'onda deve essere un'equazione differenziale, contenente la sua derivata temporale al primo ordine. In questo modo è possibile ricavare la funzione d'onda in ogni istante temporale, noto l'istante iniziale.

Inoltre, la teoria di Schrödinger non considera gli effetti relativistici. Un primo tentativo di includere la relatività fu formulato da Klein-Gordon che, appunto, considerarono l'equazione con una derivata temporale al secondo ordine.

Dirac, infine, scrisse un'equazione relativistica corretta da cui fu possibile prevedere l'esistenza del positrone \cite{dirac1930principles}.

Per ricavare l'equazione di Schrödinger, si applica il gradiente alla funzione d'onda piana:

\[
\vec{\nabla}\Psi\left(\vec{r},t \right) = \dfrac{\partial\Psi}{\partial\vec{r}} = \dfrac{\partial}{\partial\vec{r}}\exp\left( \dfrac{j}{\hslash}\left( \vec{p} \cdot \vec{r} - Et \right) \right) = \dfrac{j}{\hslash}\vec{p}\exp\left( \dfrac{j}{\hslash}\left( \vec{p} \cdot \vec{r} - Et \right) \right)
\]

Il gradiente può essere riscritto come:

\[
\vec{\nabla}\Psi\left( \vec{r},t \right) = \dfrac{j}{\hslash}\vec{p}\,\Psi\left( \vec{r},t \right)
\]

Per valutare il laplaciano della funzione d'onda, si applica la divergenza al gradiente di \(\Psi\):

\[
\vec{\nabla} \cdot \vec{\nabla}\Psi\left( \vec{r},t \right) = \nabla^{2}\Psi\left( \vec{r},t \right) = \dfrac{j}{\hslash}\left( \vec{\nabla} \cdot \left(\vec{p}\Psi\left( \vec{r},t \right)\right) \right) = 
\]

Per le proprietà dell'operatore divergenza è possibile scrivere:

\[
 = \dfrac{j}{\hslash}\left(\Psi\left( \vec{r},t \right) \vec{\nabla} \cdot \vec{p} + \vec{p}\cdot \vec{\nabla}\Psi\left( \vec{r},t \right) 
 \right) = 
\]

Poiché il vettore quantità di moto per una particella libera è costante, la sua divergenza è nulla (\(\vec{\nabla} \cdot \vec{p} = 0\)), per cui risulta:

\[
\vec{\nabla} \cdot \vec{\nabla}\Psi\left( \vec{r},t \right) = \dfrac{j}{\hslash}\vec{p}\cdot\vec{\nabla}\,\Psi\left( \vec{r},t \right)
\]

Il laplaciano della funzione d'onda si scrive come:

\[
\nabla^{2}\Psi = \dfrac{j}{\hslash}\vec{p}\cdot\vec{\nabla}\,\Psi\left( \vec{r},t \right) = \dfrac{j}{\hslash}\vec{p}\cdot\left( \dfrac{j}{\hslash}\vec{p}\,\Psi\left( \vec{r},t \right) \right) = \left( \dfrac{j}{\hslash} \right)^{2}\Psi\left( \vec{r},t \right) \vec{p}\cdot\vec{p}
\]

Per le proprietà del prodotto scalare e delle potenze dell'unità immaginaria, si scrive:

\[
\nabla^{2}\Psi = - \left( \dfrac{p}{\hslash} \right)^{2}\Psi  
\]

La derivata temporale di \(\Psi\) è, invece:

\[\dfrac{\partial\Psi}{\partial t} = \dfrac{\partial}{\partial t}\exp\left( \dfrac{j}{\hslash}\left( \vec{p} \cdot \vec{r} - Et \right) \right) = - \dfrac{j}{\hslash}E\exp\left( \dfrac{j}{\hslash}\left( \vec{p} \cdot \vec{r} - Et \right) \right)
\]

Ovvero:

\[\dfrac{\partial\Psi}{\partial t} = - \dfrac{j}{\hslash}E\Psi\]

L'equazione di Schrödinger è ottenuta confrontando le due quantità ottenute:

\[
\begin{cases}
 \nabla^{2}\Psi = - \left( \dfrac{p}{\hslash} \right)^{2}\Psi \\
 \dfrac{\partial\Psi}{\partial t} = - \dfrac{j}{\hslash}E\Psi
\end{cases} 
\]

Si isola \(\Psi\) per entrambe le equazione:

\[
\begin{cases}
\Psi = - \dfrac{\hslash^{2}}{p^{2}}\nabla^{2}\Psi \\
\Psi = - \dfrac{\hslash}{jE}\dfrac{\partial\Psi}{\partial t}
\end{cases} \Leftrightarrow \begin{cases}
 \Psi = - \dfrac{\hslash^{2}}{p^{2}}\nabla^{2}\Psi \\
 \Psi = j\dfrac{\hslash}{E}\dfrac{\partial\Psi}{\partial t}
\end{cases}
\]

Uguagliando i secondi membri delle due equazioni si ottiene:

\begin{equation}
    - \dfrac{\hslash^{2}}{p^{2}}\nabla^{2}\Psi = j\dfrac{\hslash}{E}\dfrac{\partial\Psi}{\partial t} 
    \label{eq:eq1}
\end{equation}



Per una particella libera, l’energia totale o hamiltoniana  coincide con l'energia cinetica:

\[
H = E = T = \dfrac{1}{2}mv^{2} = \dfrac{p^{2}}{2m}
\]

Dall'ultima uguaglianza si isola il termine \(p^{2}\):

\[
p^{2} = 2mE
\]

Sostituendo questo risultato nell'equazione differenziale ottenuta, si ha;

\[
- \dfrac{\hslash^{2}}{2mE}\nabla^{2}\Psi = j\dfrac{\hslash}{E}\dfrac{\partial\Psi}{\partial t}
\]

Si semplifica \(\hslash\) ed \(E\), ottenendo l'equazione:

\[
\dfrac{\hslash}{2m}\nabla^{2}\Psi + j\dfrac{\partial\Psi}{\partial t} = 0
\]

Questa relazione rappresenta la forma standard dell’equazione di Schrödinger per una particella libera. 

Nel caso in cui la particella si trovi in un campo di potenziale,  dipendente dalla posizione e dal tempo \(U(\vec{r},t)\), l’equazione si modifica includendo il termine di energia potenziale al secondo membro:

\[
j\hslash\dfrac{\partial\Psi}{\partial t} = \left( -\dfrac{\hslash^{2}}{2m}\nabla^{2} + U(\vec{r},t) \right)\Psi
\]

Questa equazione descrive l’evoluzione temporale della funzione d’onda \(\Psi(\vec{r},t)\), che contiene tutte le informazioni sullo stato quantico della particella. La sua soluzione consente di determinare la probabilità di trovare la particella in una certa posizione e in un certo istante.

\section{Operatori in meccanica quantistica}\label{operatori-in-meccanica-quantistica}
In meccanica quantistica, a ogni grandezza fisica osservabile è associato un operatore matematico \(\hat{f}\) che agisce sullo spazio delle funzioni d'onda:

\[
\hat{f} : \Psi \rightarrow \varphi
\]

L'operatore \(\hat{f}\) è un'applicazione lineare che trasforma una funzione d'onda \(\Psi\) in un'altra funzione d'onda \(\varphi\), ovvero porta una particella da uno stato quantico iniziale a uno finale. I valori misurabili della grandezza fisica corrispondono agli autovalori dell'operatore \(\hat{f}\), ottenuti risolvendo l'equazione agli autovalori:

\[
\hat{f} \Psi = f \Psi
\]

In questo contesto, un’osservabile è una grandezza fisica misurabile, rappresentata da un operatore lineare, in genere complesso, che agisce sullo spazio degli stati quantici.

Essendo \(\hat{f}\) un operatore lineare, vale la proprietà:

\[
\hat{f}\left( c_{1}\Psi_{1} + c_{2}\Psi_{2} \right) = c_{1}\hat{f}\left( \Psi_{1} \right) + c_{2}\hat{f}\left( \Psi_{2} \right)
\]

In meccanica quantistica, la quantità di moto (o momento lineare) è associata all'operatore:

\[
\hat{\vec{p}} = -j\hslash\vec{\nabla}
\]

che, in coordinate cartesiane, si scrive:

\[
\hat{\vec{p}} = -j\hslash
\begin{pmatrix}
 \dfrac{\partial}{\partial x} \\
 \dfrac{\partial}{\partial y} \\
 \dfrac{\partial}{\partial z}
\end{pmatrix}
\]

L'energia di una particella libera è descritta dall'operatore hamiltoniano:

\[
\hat{H} = \dfrac{1}{2m} \hat{\vec{p}} \cdot \hat{\vec{p}} = \dfrac{1}{2m} (-j\hslash\vec{\nabla}) \cdot (-j\hslash\vec{\nabla}) = -\dfrac{\hslash^{2}}{2m}\nabla^{2}
\]

che, in coordinate cartesiane, si scrive:

\[
\hat{H} = -\dfrac{\hslash^{2}}{2m}\left( \dfrac{\partial^{2}}{\partial x^{2}} + \dfrac{\partial^{2}}{\partial y^{2}} + \dfrac{\partial^{2}}{\partial z^{2}} \right)
\]

Se la particella è immersa in un campo di potenziale \(U\left(\vec{r},t\right)\), l'operatore hamiltoniano, rappresentante l'energia totale della particella, si generalizza come:

\[
\hat{H} =  -\dfrac{\hslash^{2}}{2m}\nabla^{2} + U\left(\vec{r},t\right)
\]

Il momento angolare è associato all'operatore:

\[
\vec{\hat{L}} = \vec{r} \times \hat{\vec{p}} = \vec{r} \times \left( -j\hslash\vec{\nabla} \right) = -j\hslash\, \vec{r} \times \vec{\nabla}
\]

\begin{table}[ht]
    \centering
    \caption{Operatori Fondamentali in Meccanica Quantistica}
    \label{tab:operatori-quantistici}
    \begin{tabular}{|l|c|c|}
        \hline
        \textbf{Osservabile} & \textbf{Simbolo Classico} & \textbf{Operatore Quantistico ($\hat{f}$)} \\
        \hline
        Posizione & $\vec{r}$ & $\hat{\vec{r}} = \vec{r}$ \\
        \hline
        Quantità di Moto & $\vec{p}$ & $\hat{\vec{p}} = -j\hslash\vec{\nabla}$ \\
        \hline
        Energia Cinetica & $T = \dfrac{p^2}{2m}$ & $\hat{T} = -\dfrac{\hslash^{2}}{2m}\nabla^{2}$ \\
        \hline
        Energia Potenziale & $U(\vec{r},t)$ & $\hat{U} = U(\vec{r},t)$ \\
        \hline
        Energia Totale (Hamiltoniana) & $H = T + U$ & $\hat{H} = -\dfrac{\hslash^{2}}{2m}\nabla^{2} + U(\vec{r},t)$ \\
        \hline
        Momento Angolare & $\vec{L} = \vec{r} \times \vec{p}$ & $\vec{\hat{L}} = -j\hslash\, (\vec{r} \times \vec{\nabla})$ \\
        \hline
    \end{tabular}
\end{table}

Utilizzando la definizione dell'operatore hamiltoniano, è possibile riscrivere in forma compatta l'equazione di Schrödinger per una particella immersa in un campo potenziale variabile nel tempo e con la posizione:

\[
j\hslash\dfrac{\partial \Psi}{\partial t} = -\dfrac{\hslash^2}{2m} \nabla^2 \Psi + U(\vec{r},t)\Psi
\]

Poiché \(\hat{H} = -\hslash^{2}/{2m}\,\nabla^{2} + U\left(\vec{r},t\right)\), si può scrivere:

\[
\left( U\left( \vec{r},t \right) -\dfrac{\hslash^{2}}{2m}\nabla^{2}\right) \Psi = j\hslash\dfrac{\partial\Psi}{\partial t}
\]

Si ottiene così la forma operativa dell’equazione di Schrödinger:

\[
\hat{H}\Psi = j\hslash\dfrac{\partial\Psi}{\partial t}
\]

oppure, portando tutti i termini da un lato:

\[
\hat{H}\Psi - j\hslash\dfrac{\partial\Psi}{\partial t} = 0
\]

\section{Equazione di Schrödinger per stati stazionari}\label{equazione-di-schruxf6dinger-per-stati-stazionari}

La funzione d'onda \(\Psi\left( \vec{r},t \right)\), espressa come un'onda piana, può essere scritta come:

\[
\Psi\left( \vec{r},t \right) = \exp\left( \dfrac{j}{\hslash}\left( \vec{p} \cdot \vec{r} - Et \right) \right) = \exp\left( \dfrac{j}{\hslash}\vec{p} \cdot \vec{r} \right)\exp\left( - \dfrac{j}{\hslash}Et \right)
\]

Si definisce \(\phi\left( \vec{r} \right)\) come la parte della funzione d'onda dipendente dalla posizione:

\[
\phi\left( \vec{r} \right) = \exp\left( \dfrac{j}{\hslash}\vec{p} \cdot \vec{r} \right)
\]

L'equazione di Schrödinger si scrive quindi come:

\[
\Psi\left( \vec{r},t \right) = \phi\left( \vec{r} \right)\exp\left( - \dfrac{j}{\hslash}Et \right)
\]

Se l'energia del sistema è costante, l'hamiltoniana non dipende dal tempo. Si applica questo operatore alla funzione d'onda:

\[
\hat{H}\Psi = \hat{H}\left( \phi\left( \vec{r} \right)\exp\left( - \dfrac{j}{\hslash}Et \right) \right)
\]

Per definizione dell'operatore hamiltoniano, si ha:

\[
\hat{H}\Psi = \left( -\dfrac{\hslash^{2}}{2m}\nabla^{2} + V(\vec{r}) \right)\left( \phi\left( \vec{r} \right)\exp\left( - \dfrac{j}{\hslash}Et \right) \right)
\]

Poiché $V(\vec{r})$ è un operatore di moltiplicazione e l'esponenziale temporale non dipende dalla posizione (dunque, può essere portato all'esterno del Laplaciano) si ottiene:

\[
\hat{H}\Psi = - \dfrac{\hslash^{2}}{2m}\exp\left( - \dfrac{j}{\hslash}Et \right)\nabla^{2}\phi\left( \vec{r} \right) + V(\vec{r})\phi\left( \vec{r} \right) \exp\left( - \dfrac{j}{\hslash}Et \right)
\]

Raccogliendo opportunamente, si ha:

\[
\hat{H}\Psi = \exp\left( - \dfrac{j}{\hslash}Et \right)\left( -\dfrac{\hslash^{2}}{2m}\nabla^{2} + V(\vec{r}) \right)\phi\left( \vec{r} \right)
\]

Per definizione dell'operatore hamiltoniano, si ricava:

\[
\hat{H}\Psi = \exp\left( - \dfrac{j}{\hslash}Et \right)\hat{H}\phi\left( \vec{r} \right)
\]

Si calcola ora la derivata temporale di \(\Psi\):

\[
\dfrac{\partial\Psi}{\partial t} = \dfrac{\partial}{\partial t}\left( \phi\left( \vec{r} \right)\exp\left( - \dfrac{j}{\hslash}Et \right) \right)
\]

Poiché \(\phi\left( \vec{r} \right)\) non dipende dal tempo, ma solamente dalla posizione, può essere portata all'esterno del simbolo di derivata:

\[
\dfrac{\partial\Psi}{\partial t} = \phi\left( \vec{r} \right)\dfrac{\partial}{\partial t}\exp\left( - \dfrac{j}{\hslash}Et \right) = - \dfrac{j}{\hslash}E\phi\left( \vec{r} \right)\exp\left( - \dfrac{j}{\hslash}Et \right)
\]

Si considera l'equazione di Schrödinger in termini di hamiltoniana, \(\hat{H}\Psi - j\hslash\partial\Psi/\partial t = 0\). Si è visto che:

\[
\begin{cases}
\dfrac{\partial\Psi}{\partial t} = - \dfrac{j}{\hslash}E\phi\left( \vec{r} \right)\exp\left( - \dfrac{j}{\hslash}Et \right) \\
\hat{H}\Psi = \exp\left( - \dfrac{j}{\hslash}Et \right)\hat{H}\phi\left( \vec{r} \right)
\end{cases} 
\]

Sostituendo:

\[
\exp\left( - \dfrac{j}{\hslash}Et \right)\hat{H}\phi\left( \vec{r} \right) - j\hslash\left( - \dfrac{j}{\hslash}E\phi\left( \vec{r} \right)\exp\left( - \dfrac{j}{\hslash}Et \right) \right) = 0
\]

Svolgendo i prodotti, si ottiene:

\[
\exp\left( - \dfrac{j}{\hslash}Et \right)\hat{H}\phi\left( \vec{r} \right) - E\phi\left( \vec{r} \right)\exp\left( - \dfrac{j}{\hslash}Et \right) = 0
\]

Semplificando il termine esponenziale, si ottiene:

\[
\hat{H}\phi\left( \vec{r} \right) = E\phi\left( \vec{r} \right)
\]

Si ottiene un'equazioni agli autovettori e autovalori; infatti, è possibile scrivere:

\[\left( \hat{H} - E \right)\phi\left( \vec{r} \right) = 0\]

Dove \(E\) è l'energia totale del sistema supposta costante. L'energia \(E\) rappresenta gli autovalori dell'operatore hamiltoniano \(\hat{H}\). Questo risultato è coerente con gli esperimenti, poiché gli autovalori sono, in genere, un'infinità numerabile e, dunque, discreti.

La meccanica ondulatoria i Schrödinger prevede la quantizzazione dell'energia degli orbitali atomici. La soluzione dell'equazione agli autovalori permette di ottenere i livelli energetici del sistema e le autofunzioni \(\phi\left( \vec{r} \right)\), il cui modulo quadro rappresenta la probabilità che la particella del sistema si trovi un quel livello energetico.

Sia \(V\left( \vec{r} \right)\) l'energia potenziale a cui la particella è soggetta, ad esempio un elettrone attratto dal nucleo. L'operatore hamiltoniano si scrive:

\[
\hat{H} = - \dfrac{\hslash^{2}}{2m}\nabla^{2} + V\left( \vec{r} \right)
\]

Con \({\hat{E}}_{c}\) operatore energia cinetica:

\[
{\hat{E}}_{c} = - \dfrac{\hslash^{2}}{2m}\nabla^{2}
\]

L'operatore hamiltoniano è dato da:

\[
\hat{H} = \hat{E} = - \dfrac{\hslash^{2}}{2m}\nabla^{2} + V\left( \vec{r} \right)
\]

Nel caso dell'elettrone attratto dal nucleo, il potenziale è di tipo coulombiano:

\[
V\left( \vec{r} \right) = -\dfrac{1}{4\pi\varepsilon_{0}}\dfrac{Ze^{2}}{r}
\]

Dove \(Z\) è il numero atomico, ovvero il numero di protoni nel nucleo.

L'equazione agli autovettori dell'hamiltoniano si scrive come:


\[
\hat{H}\phi\left( \vec{r} \right) - \ E\phi\left( \vec{r} \right) = 0 \Leftrightarrow - \dfrac{\hslash^{2}}{2m}\nabla^{2}\phi + V\left( \vec{r} \right)\phi = E\phi
\]

Moltiplicando per \(- 1\), si ha:

\[
\dfrac{\hslash^{2}}{2m}\nabla^{2}\phi - V\left( \vec{r} \right)\phi + E\phi = 0
\]

Raccogliendo, si ha:

\[
\left( \dfrac{\hslash^{2}}{2m}\nabla^{2} + \left( E - V\left( \vec{r} \right) \right) \right) \phi = 0
\]

In generale, è possibile scrivere un'equazione agli autovalori per ogni operatore. I corrispondenti autovalori sono i valori che quell'operatore può assumere.

Ad esempio, gli autovalori del momento angolare sono i possibili valori del momento angolare ottenuti risolvendo l'equazione agli autovalori:

\[
\hat{L}\phi = L\phi
\]

In presenza di un campo magnetico, l'energia totale dell'elettrone o particella è data da:

\[
E = E_{c} + V\left( \vec{r} \right)
\]

Dove il potenziale è dato da:

\[
V\left( \vec{r} \right) = \vec{\mu} \cdot \vec{B}
\]

Il momento magnetico è legato al momento angolare dal fattore giromagnetico:

\[
\hat{\vec{\mu}} = - \gamma\hat{\vec{L}}
\]

Dunque, il potenziale può essere espresso come:

\[
V\left( \vec{r} \right) = - \gamma\hat{\vec{L}} \cdot \vec{B}
\]

In definitiva, l'operatore hamiltoniano si scrive:

\[
\hat{H} = - \dfrac{\hslash^{2}}{2m}\nabla^{2} - \gamma\hat{\vec{L}} \cdot \vec{B}
\]

\subsection{Buco di potenziale}\label{buco-di-potenziale}

Secondo la meccanica classica, l'energia cinetica di una particella è:

\[
T = \dfrac{1}{2}mv^{2}
\]

Mentre l'energia totale è data da:

\[
E = T + V
\]

Con \(V\) energia potenziale. Dalla relazione per l'energia totale è possibile valutare l'energia cinetica in funzione di quella totale e potenziale:

\[
T = E - V
\]

Nella teoria classica, la particella è in moto se l'energia cinetica è positiva, ovvero:

\[
T > 0 \Leftrightarrow E > V
\]

Pertanto, in ambito classico, il moto della particella può avvenire solo nelle regioni in cui \(E > V\).

Si considera ora una buca di potenziale unidimensionale definita da:

\[
V(x) = \begin{cases}
0 & 0 < x < a \\
\infty & x \leq 0 \text{ oppure } x \geq a
\end{cases}
\]

\begin{figure}[ht]
\centering
\includegraphics[width=2.67857in,height=0.86783in,alt={P1856C2T1\#yIS1}]{media/4_Quantiatica/image36.pdf}\caption{Buca di potenziale}
\end{figure}

La particella può muoversi solo all'interno della regione di spazio \(0 < x < a\), dunque, la funzione d'onda \(\phi\left( \vec{r} \right)\) è nulla all'esterno della buca di potenziale, in cui \(V \rightarrow \infty\).


L'equazione di Schrödinger ha soluzioni non nulle solo dove il potenziale è finito. In termini di hamiltoniana:

\[
\hat{H}\phi(x) = E\phi(x)
\]

Poiché la particella è libera all'interno della buca, l'operatore hamiltoniano nel caso generale è:

\[
\hat{H} = - \dfrac{\hslash^{2}}{2m}\nabla^{2}
\]

Il moto avviene solamente lungo l'asse \(x\), per cui l'operatore è:

\[
\hat{H} = - \dfrac{\hslash^{2}}{2m}\dfrac{d^{2}}{d x^{2}}
\]

L'equazione di Schrödinger per la buca di potenziale e, quindi:

\[
- \dfrac{\hslash^{2}}{2m}\dfrac{d^{2}\phi}{dx^{2}} = E\phi
\Leftrightarrow
\dfrac{d^{2}\phi}{dx^{2}} + \dfrac{2mE}{\hslash^{2}}\phi = 0
\]

L'equazione ottenuta coincide con l'oscillatore armonico, la cui soluzione è del tipo:

\[\phi(x) = A\cos\left( \sqrt{\dfrac{2mE}{\hslash^{2}}}x \right) + B\sin\left( \sqrt{\dfrac{2mE}{\hslash^{2}}}x \right)\]

Dove \(A\) e \(C\) sono due costanti dipendenti dalle condizioni al contorno, ottenute considerando la funzione d'onda continua nei punti \(x = a\) e \(x = b\):

\[
\begin{cases}
\phi(a) = 0 \\
\phi(0) = 0
\end{cases} 
\]

Si applica la prima condizione \(\phi(a) = 0 \):

\[
\phi(a) = A\cos\left( \sqrt{\dfrac{2mE}{\hslash^{2}}}0 \right) + B\sin\left( \sqrt{\dfrac{2mE}{\hslash^{2}}}0 \right) = 0
\]

Da cui risulta:

\[
\phi(0) = A = 0
\]

Si applica la seconda condizione al contorno:

\[
\phi(a) = B\sin\left( \sqrt{\dfrac{2mE}{\hslash^{2}}}a \right) = 0
\]

Le soluzioni dell'equazione:

\[
\sin\left( \sqrt{\dfrac{2mE}{\hslash^{2}}}a \right) = 0
\]

sono un'infinità numerabile:

\[
\sqrt{\dfrac{2mE_{n}}{\hslash^{2}}}a = n\pi,\ n\in\mathbb{N}
\]

Si ricava \(E_{n}\):

\[
\dfrac{a}{\hslash}\sqrt{2mE_{n}} = n\pi \Leftrightarrow \sqrt{2mE_{n}} = \dfrac{\hslash^{2}}{a^{2}}n\pi,\ n\in \mathbb{N}
\]

Si eleva al quadrato e si divide ambo i membro per \(2m\):

\[
E_{n} = \dfrac{\hslash^{2}}{2ma^{2}}n^{2}\pi^{2},\ n\in \mathbb{N}
\]

Gli \(E_{n}\) sono gli autovalori possibili dell'operatore hamiltoniano e, di conseguenza, i possibili livelli energetici che può assumere la particella in una buca di potenziale monodimensionale. Come si nota, i livelli energetici sono quantizzati, in accordo con le previsioni sperimentali.

Si considera, ora, il caso tridimensionale, ovvero la particella si muove in una scatola di potenziale. La funzione potenziale è data da:

\[
V(x,y,z) = \begin{cases}
0 & 0 < x < a,\ 0 < y < b,0 < z < c\  \\
\infty & altrove
\end{cases} 
\]

L'equazione di Schrödinger fornisce valori non nulli solamente nella regione di spazio in cui il potenziale è finito \(\left[ 0;a\right] \times \left[ 0;b\right] \times \left[0;c\right]\). In termini di hamiltoniano, costante nel tempo, si ha:

\[
\hat{H}\phi\left( \vec{r} \right) - E\phi\left( \vec{r} \right) = 0
\]

Esplicitando l'operatore hamiltoniano, si ottiene:

\[
- \dfrac{\hslash^{2}}{2m}\nabla^{2}\ \phi\left( \vec{r} \right) - E\phi\left( \vec{r} \right) = 0
\]

Ricavando il laplaciano di \(\phi\):

\[
\nabla^{2}\ \phi + \dfrac{2m}{\hslash^{2}}E\phi = 0
\]

In coordinate cartesiane, l'equazione è data:

\[
\dfrac{\partial^{2}\phi}{\partial x^{2}} + \dfrac{\partial^{2}\phi}{\partial y^{2}} + \dfrac{\partial^{2}\phi}{\partial z^{2}} + \dfrac{2m}{\hslash^{2}}E\phi = 0
\]

Si applica il metodo delle variabile separabili, secondo cui la soluzione è del tipo:

\[
\phi\left( \vec{r} \right) = \alpha(x)\beta(y)\gamma(z)
\]

Si sostituisce questa espressione nell'equazione di Schrödinger in coordinate cartesiane:

\[
\beta\gamma\dfrac{\partial^{2}\alpha}{\partial x^{2}} + \alpha\gamma\dfrac{\partial^{2}\beta}{\partial y^{2}} + \alpha\beta\dfrac{\partial^{2}\gamma}{\partial z^{2}} + \dfrac{2mE}{\hslash^{2}}\alpha\beta\gamma = 0
\]

Si divide per \(\phi\left( \vec{r} \right) = \alpha(x)\beta(y)\gamma(z)\):

\[\dfrac{1}{\alpha}\ \dfrac{\partial^{2}\alpha}{\partial x^{2}} + \dfrac{1}{\beta}\ \dfrac{\partial^{2}\beta}{\partial y^{2}} + \dfrac{1}{\gamma}\dfrac{\partial^{2}\gamma}{\partial z^{2}} + \dfrac{2mE}{\hslash^{2}} = 0\]

I tre termini dipendono solamente da una variabile spaziale, dunque, è possibile scrivere tre equazioni diverse:

\[
\begin{cases}
\dfrac{\partial^{2}\alpha}{\partial x^{2}} + k_{x}^{2}\alpha = 0 \\
\dfrac{\partial^{2}\beta}{\partial y^{2}} + k_{y}^{2}\beta = 0 \\
\dfrac{\partial^{2}\gamma}{\partial z^{2}} + k_{z}^{2}\gamma = 0
\end{cases}
\]

La soluzione delle equazioni è del tipo:

\[
f(q) = c\exp\left( - jk_{q}q \right),\ \ q = x,y,z
\]

Dunque, la integrale generale è del tipo:

\[
\phi(x,y,z) = \left( A_x \cos(k_x x) + B_x \sin(k_x x) \right) \cdot \left( A_y \cos(k_y y) + B_y \sin(k_y y) \right) \cdot \left( A_z \cos(k_z z) + B_z \sin(k_z z) \right)
\]

Dove:
\[
k_x = \sqrt{2m\dfrac{E_x}{\hslash^2}},\ k_y = \sqrt{2m\dfrac{E_y}{\hslash^2}}\, k_z = \sqrt{2m\dfrac{E_z}{\hslash^2}}
\]
mentre $A_i$ e $B_i$ sono costanti determinate dalle condizioni al contorno.

Le condizione da imporre riguardano la continuità della funzione d'onda ai bordi della buca di potenziale:

\[
\begin{cases}
\phi(0,y,z) = 0 \\
\phi(a,y,z) = 0 \\
\phi(x,0,z) = 0 \\
\phi(x,b,z) = 0 \\
\phi(x,y,0) = 0 \\
\phi(x,y,c) = 0
\end{cases} 
\]

Le condizioni al contorno portano a un'equazione del tipo:

\[
\sin\left( ak_{x} \right) = 0
\]

La cui soluzioni sono:

\[
k_{x} = n_{x}\dfrac{\pi}{a},\ \ n_{x}\in \mathbb{N}
\]

Da cui si ottengono gli autovalori lungo \(x\) dell'equazione di Schrödinger:

\[
E_{x} = k_{x}^{2}\dfrac{\hslash}{2m} = n_{x}^{2}\dfrac{\pi^{2}}{a^{2}}\dfrac{\hslash^{2}}{2m},\ \ n_{x}\in\mathbb{ N}
\]

Analogo risultato lo si ottiene per \(E_{y}\):

\[
E_{y} = n_{y}^{2}\dfrac{\pi^{2}}{b^{2}}\dfrac{\hslash^{2}}{2m},\ \ n_{y}\in\mathbb{ N}
\]

ed \(E_{z}\):

\[E_{z} = n_{z}^{2}\dfrac{\pi^{2}}{c^{2}}\dfrac{\hslash^{2}}{2m},\ \ n_{z}\in\mathbb{N}\]

La somma dei tre autovalori deve essere uguale all'energia totale:

\[
E_{n} = E_{x} + E_{z} + E_{z}
\]

Sostituendo i valori ottenuti si ottengono gli autovalori dell'operatore hamiltoniano per la geometria considerata:

\[
E_{n} = \left( \dfrac{n_{x}^{2}}{a^{2}} + \dfrac{n_{y}^{2}}{b^{2}} + \dfrac{n_{z}^{2}}{c^{2}} \right)\dfrac{\pi^{2}\hslash^{2}}{2m},\ \ n_{x},n_{y},n_{z}\in \mathbb{N}
\]

Allo stesso modo è possibile ottenere i livelli energetici per un atomo qualsiasi, come quello di idrogeno. In questo caso, il potenziale in cui è immerso l'elettrone è di tipo coulombiano:

\[V(r) = \dfrac{1}{4\pi\varepsilon_{0}}\dfrac{e^{2}}{r}\]

La risoluzione dell'equazione di Schrödinger in coordinate sferiche fornisce i livelli energetici dell'atomo.

Le previsioni teoriche, ovvero livelli energetici ricavati dalla risoluzione dell'equazione di Schrödinger, coincidono con l'energia dei livelli energetici dell'atomo. Ne discende che, mediante l'equazione di Schrödinger è possibile ottenere una previsione teorica anche per gli spettri di assorbimento. Infatti, nota l'energia degli orbitali, è nota anche l'energia, \(E = h\nu\), che il fotone deve possedere affinché sia assorbito dall'elettrone. L'energia del fotone deve essere maggiore della differenza dell'energia dei due livelli energetici coinvolti.

\subsection{Gradino di potenziale}\label{gradino-di-potenziale}

In meccanica classica il moto di una particella non può avvenire nella regione di spazio in cui la sua energia \(E\) è minore  del potenziale \(V\) che insiste in quella regione di spazio (\(E < V\)). In altre parole, se la particella non ha energia sufficiente, non riesce a superare il gradino di potenziale.

Si consideri una particella elementare, come un elettrone, con energia \(E\), lanciata verso un gradino di potenziale \(V_0\), definita come:

\[
V(x) = \begin{cases}
V_{0} & x > 0 \\
0 & x < 0
\end{cases} 
\]

\begin{figure}[ht]
\centering
\includegraphics[width=5.10848in,height=0.63328in,alt={P1933\#yIS1}]{media/4_Quantiatica/image37.pdf}\caption{Gradino di potenziale}
\end{figure}

È possibile scrivere l'equazione di Schrödinger per le due regioni dello spazio, in base al potenziale \(V(x)\). Essendo il moto monodimensionale, risulta:

\[
\begin{cases}
 \dfrac{\partial^{2}\phi}{\partial x^{2}} + \dfrac{2mE}{\hslash^{2}}\phi = 0 & x < 0 \\
 \dfrac{\partial^{2}\phi}{\partial x^{2}} + \dfrac{2m}{\hslash^{2}}\left( E - V_{0} \right)\phi = 0 & x > 0
\end{cases}
\]

La prima equazione presenta una soluzione del tipo:

\[
\phi(x) = A\exp\left( j\dfrac{\sqrt{2mE}}{\hslash}x \right) + C\exp\left( - j\dfrac{\sqrt{2mE}}{\hslash}x \right)
\]

All'interfaccia del gradino di potenziale, si genera un'onda riflessa che prosegue in verso retrogrado. Dunque, Nella regione di spazio \(x < 0\) vi sono due onde: una progressiva (o incidente) e una regressiva (o riflessa). Nell'equazione, $A$ è il coefficiente dell'onda incidente e $C$ quello della riflessa.

\begin{figure}[ht]
\centering
\includegraphics[width=5.25in,height=0.81494in,alt={P1940\#yIS1}]{media/4_Quantiatica/image38.pdf}\caption{Onda riflessa e trasmessa all'interfaccia}
\end{figure}

Nella regione \(x > 0\) supponendo che \(E < V_{0}\), l'esponenziale dell'onda deve essere reale e negativo, ovvero:

\[
\phi(x) = B\exp\left( - \dfrac{\sqrt{2m(V_{0} - E)}}{\hslash}x \right)
\]

Applicando la condizione di continuità della funzione d'onda all'interfaccia \(x = 0\), risulta che:

\[
\phi\left( 0^{-} \right) = \phi\left( 0^{+} \right)
\]

Ovvero:

\[
A + C = B
\]

Esiste, dunque, una probabilità non nulla di trovare la particella oltre il gradino di potenziale. Tuttavia, poiché l'onda nella regione di spazio \(x>0\) presenta un esponenziale reale, la probabilità di trovare la particella elementare oltre il gradino di potenziale decresce rapidamente con la distanza. Nonostante ciò, trovare la particella oltre il gradino di potenziale è un evento possibile, soprattutto in prossimità dell'interfaccia. 

\begin{figure}[ht]
\centering
\includegraphics[width=3.77117in,height=2.41667in,alt={P1949\#yIS1}]{media/4_Quantiatica/image39.pdf}\caption{Probabilità di rilevare la particella oltre il gradino di potenziale}
\end{figure}

L'effetto del gradino di potenziale è sfruttato nei dispositivi a semiconduttore.

\subsection{Effetto tunnel}\label{effetto-tunnel}

Si suppone che il potenziale sia confinato in una regione dello spazio, ovvero del tipo:

\[V(x) = \begin{cases}
0 & x < 0 \\
V_{0} & 0 \leq x \leq a \\
0 & x > 0
\end{cases} 
\]

\begin{figure}[ht]
\centering
\includegraphics[width=6.52905in,height=0.84849in,alt={P1955\#yIS1}]{media/4_Quantiatica/image40.pdf}\caption{Impulso di tensione}
\end{figure}

Si suppone che la particella provenga da sinistra, ovvero proceda nel verso delle \(x\) crescenti. Nella prima regione (\(x < 0\)), l'equazione di Schrödinger stazionaria è:

\[
\dfrac{d^2\phi}{dx^2} + \dfrac{2mE}{\hslash^2}\phi = 0
\]

La cui soluzione prevede due onde: una progressiva e una regressiva a causa dei fenomeni di riflessione:

\[
\phi_{x < 0}(x) = I\exp\left( j\dfrac{\sqrt{2mE}}{\hslash}x \right) + R\exp\left( - j\dfrac{\sqrt{2mE}}{\hslash}x \right)
\]

dove \(I\) è l'ampiezza dell'onda incidente e \(R\) quella dell'onda riflessa.

Nella regione intermedia (\(0 < x < a\)), dove il potenziale è costante e maggiore dell'energia della particella (\(V_0 > E\)), l'equazione diventa:

\[
\dfrac{d^{2}\phi}{d x^{2}} + \dfrac{2m}{\hslash^{2}}\left( E - V_{0} \right)\phi = 0,\ 0 < x < a
\]

Per la presenza dell'interfaccia successiva, per ottenere la soluzione completa, è necessario prevedere la presenza di due onde: una progressiva e una regressiva:

\[
\phi_{0 < x < a}(x) = \ A\exp\left( \dfrac{\sqrt{2m(V_0 - E)}}{\hslash}x \right) + B\exp\left( - \dfrac{\sqrt{2m(V_0 - E)}}{\hslash}x \right)
\]

Dove gli esponenziali sono reali a causa della condizione \(V_{0} > E\).

Nella regione \(x > a\), invece, si ha un'unica onda poiché non vi sono fenomeni di riflessione. L'equazione stazionaria è la stessa della regione per \(x < 0\):

\[\dfrac{d^{2}\phi}{d x^{2}} + \dfrac{2mE}{\hslash^{2}}\phi = 0,\ \ x > a\]

Dove la soluzione è:

\[
\phi_{x > a}(x) = S\exp\left( j\dfrac{\sqrt{2mE}}{\hslash}x \right)
\]

Oltre l'impulso di tensione di ampiezza \(V_{0}\) maggiore di \(E\) della particella, esiste una probabilità non nulla di trovare la particella. Tale fenomeno è noto come effetto tunnel e rappresenta uno dei risultati più controintuitivi e utilizzati della meccanica quantistica.

\begin{figure}[ht]
\centering
\includegraphics[width=4.07292in,height=1.87026in,alt={P1971\#yIS1}]{media/4_Quantiatica/image41.pdf}\caption{Effetto tunnel}
\end{figure}

L'effetto tunnel ha importanti applicazioni, tra cui il Microscopio a Scansione a Effetto Tunnel (STM) e i diodi tunnel.

%fenomeno della \textbf{scintillazione}, utilizzato per ridurre la dose di radiazione somministrata al paziente durante esami radiologici.

\section{Meccanica quantistica con notazione di Dirac}\label{meccanica-quantistica-con-notazione-di-dirac}

La \textbf{meccanica ondulatoria}, sviluppata principalmente da Schrödinger, e la \textbf{meccanica matriciale}, introdotta da Heisenberg, sono due formulazioni della meccanica quantistica elaborate nello stesso periodo storico. Dirac ha dimostrato l'equivalenza tra le due teorie attraverso un approccio algebrico basato sugli \textbf{spazi di Hilbert}, una generalizzazione dello spazio euclideo.

Ogni sistema microscopico è descritto, in ogni istante, da uno \textbf{stato quantico} rappresentato da un vettore nello spazio di Hilbert, indicato con \(\left| \varphi \right\rangle\), dove \(\varphi\) rappresenta lo stato del sistema.

Uno spazio di Hilbert \(H = \left( \mathbb{H}, \left\langle \cdot \middle| \cdot \right\rangle \right)\) è uno spazio vettoriale \textbf{complesso} dotato di un \textbf{prodotto interno sesquilineare} e \textbf{positivo definito}. Una forma sesquilineare è una funzione \(B: V \times V \rightarrow F\) lineare nel primo argomento e coniugata lineare nel secondo.  Essendo uno spazio lineare, vale il \textbf{principio di sovrapposizione}. Le principali proprietà dello spazio di Hilbert sono:

\begin{itemize}
  \item \textbf{Prodotto interno}: esiste un prodotto interno \(\left\langle \cdot \middle| \cdot \right\rangle\) tale che, detta \(d\) la distanza indotta dal prodotto interno, lo spazio metrico \(\left( \mathbb{H}, d \right)\) è \textbf{completo}, ovvero ogni successione di Cauchy converge in \(\mathbb{H}\).
  \item \textbf{Norma}: si può definire una norma associata al prodotto interno:
  \[
  \left\| \vec{v} \right\| = \sqrt{\left\langle \vec{v} \middle| \vec{v} \right\rangle}
  \]
  \item \textbf{Distanza}: la distanza tra due vettori è definita come:
  \[
  d\left( \vec{u}, \vec{v} \right) = \sqrt{\left\langle \vec{u} - \vec{v} \middle| \vec{u} - \vec{v} \right\rangle}
  \]
\end{itemize}

Le grandezze fisiche misurabili in un esperimento sono dette \textbf{osservabili} e corrispondono a \textbf{operatori hermitiani} (o della meccanica quantistica) che agiscono sullo spazio di Hilbert. Un operatore quantistico, come \(\hat{A}\), agisce su uno stato \(\left| \varphi \right\rangle\) per generare un nuovo stato \(\left| b \right\rangle\):

\[
\hat{A} \left| \varphi \right\rangle = \left| b \right\rangle
\]

L'operazione di misura \textbf{perturba} lo stato del sistema microscopico. In altre parole, la misura \textbf{modifica} lo stato quantico. Questo fenomeno è alla base del \textbf{principio di indeterminazione di Heisenberg}, secondo il quale non è possibile conoscere simultaneamente con precisione due grandezze coniugate (come energia e intervallo temporale). Il principio si esprime come:

\[
\Delta E \Delta t \geq \dfrac{\hslash}{2}
\]

Per calcolare l'energia di una particella, si applica l'\textbf{operatore hamiltoniano} \(\hat{H}\) allo stato \(\left| \varphi \right\rangle\):

\[
\hat{H} \left| \varphi \right\rangle = E \left| \varphi \right\rangle
\]

L'operatore hamiltoniano cambia lo stato del sistema. In questo caso, \(E\) è un \textbf{autovalore} dell'operatore \(\hat{H}\). Il sistema può assumere solo i valori energetici corrispondenti agli autovalori dell'operatore applicato.

Analogamente, per misurare il \textbf{momento angolare} si applica l'operatore \(\hat{L}\):

\[
\hat{L} \left| \varphi \right\rangle = L \left| \varphi \right\rangle
\]

Il momento angolare è \textbf{quantizzato}, quindi il sistema può assumere solo i valori corrispondenti agli autovalori dell'operatore \(\hat{L}\).

\subsection{Autovalori dell'operatore hamiltoniano}\label{autovalori-delloperatore-hamiltoniano}

Per un elettrone legato a un nucleo atomico, gli autovalori dell'operatore hamiltoniano rappresentano i possibili \textbf{livelli energetici} degli orbitali \(s\), \(p\), \(d\) e \(f\). Ogni orbitale può contenere al massimo \textbf{due elettroni con spin opposto}, secondo il \textbf{principio di esclusione di Pauli}, che stabilisce che due fermioni non possono occupare lo stesso stato quantico simultaneamente.

Questi livelli energetici sono discreti e quantizzati, e corrispondono alle soluzioni dell'equazione di Schrödinger per l'atomo. La struttura elettronica degli atomi è quindi determinata dagli autovalori dell'hamiltoniano, che definiscono le energie consentite per ciascun elettrone.

\subsection{Risultato dell'operatore di misura}\label{risultato-delloperatore-di-misura}

In generale, il risultato di una misura deve essere un \textbf{autovalore} dell'operatore di misura. Il valore della grandezza fisica in esame, in altre parole, deve essere una soluzione dell'equazione:

\[
\hat{A} \left| a \right\rangle = a_{n} \left| a \right\rangle
\]

Una misura può dare come esito uno qualsiasi degli autovalori \(a_{n}\) dell'operatore \(\hat{A}\). L'autovettore associato all'autovalore è detto \textbf{autostato}. Se lo stato iniziale del sistema è un autostato dell'operatore, la misura restituirà con certezza l'autovalore corrispondente. In altre parole, se un sistema si trova in un autostato \(\left| a_{n} \right\rangle\) con autovalore \(a_{n}\), allora il risultato della misura sarà proprio \(a_{n}\).

Se, invece, il sistema si trova in uno stato qualsiasi \(\left| \varphi \right\rangle\), questo può essere espresso come combinazione lineare degli autostati dell'operatore:

\[
\left| \varphi \right\rangle = \sum_{n} \varphi_{n} \left| a_{n} \right\rangle
\]

Gli autostati \(\left| a_{n} \right\rangle\) di un operatore di misura \(\hat{A}\)  formano una \textbf{base ortonormale} dello spazio di Hilbert, quindi ogni stato può essere scritto come loro combinazione lineare. Applicando l'operatore \(\hat{A}\) allo stato \(\left| \varphi \right\rangle\):

\[
\hat{A} \left| \varphi \right\rangle = \hat{A} \sum_{n} \varphi_{n} \left| a_{n} \right\rangle
\]

Per linearità dell'operatore sommatoria, è possibile scrivere:

\[
\hat{A}\left| \varphi \right\rangle = \sum_{n}^{}{\varphi_{n}\hat{A}\left| a_{n} \right\rangle}
\]

Dato che \(\left| a_{n} \right\rangle\) è autovettore dell'operatore applicato, \(\hat{A}\left| a_{n} \right\rangle = a_{n}\left| a_{n} \right\rangle\), risulta:

\[
\hat{A}\left| \varphi \right\rangle = \ \sum_{n}^{}{\varphi_{n}a_{n}\left| a_{n} \right\rangle}
\]

Ovvero, si è espresso lo stato \(\hat{A}\left| \varphi \right\rangle\) come combinazione degli autovettori dell'operatore \(\hat{A}\).

In meccanica quantistica non è possibile prevedere con certezza il risultato della misura, ma è possibile calcolare la \textbf{probabilità} che il sistema transiti dallo stato iniziale \(\varphi\) nello stato \(\left| a_{n} \right\rangle\), attraverso l'operatore di misura \(\hat{A}\). Questa è data dal modulo quadro del prodotto scalare:

\[
\left| \left\langle a_{n} \middle| \varphi \right\rangle \right|^2
\]

La notazione di Dirac (o bra-ket) distingue tra il vettore di stato nello spazio di Hilbert e il suo coniugato hermitiano:
\begin{itemize}
    \item Ket: La notazione $\left| \varphi \right\rangle$ è detta ket e rappresenta il vettore di stato del sistema nello spazio di Hilbert ($\mathbb{H}$).
    \item Bra: La notazione $\left\langle \varphi \right|$ è detta bra e rappresenta il vettore duale o coniugato hermitiano del ket $\left| \varphi \right\rangle$. Il bra appartiene allo spazio duale di $\mathbb{H}$ ($\mathbb{H}^*$).
\end{itemize}

Il prodotto scalare tra i due vettori è il bra-ket (la parola "bra-ket" deriva dalla contrazione delle due notazioni, $\left\langle \text{bra} \middle| \text{ket} \right\rangle$):

\[
\left\langle a_{n} \middle| \varphi \right\rangle
\]

Questo prodotto è un numero complesso che proietta lo stato iniziale $\left| \varphi \right\rangle$ sull'autostato $\left| a_{n} \right\rangle$, fornendo l'ampiezza di probabilità del risultato $a_n$.

Dirac assunse che il \textbf{valor medio} (o \textbf{valore di aspettazione}) di una grandezza fisica, associata all'operatore \(\hat{A}\), per un sistema nello stato \(\left| \varphi \right\rangle\), come:

\[
\left\langle \hat{A} \right\rangle = \left\langle \varphi \right| \hat{A} \left| \varphi \right\rangle
\]

Ad esempio, l'energia media \(\left\langle E \right\rangle\) si ottiene applicando l'operatore hamiltoniano:

\[
\left\langle E \right\rangle = \left\langle \varphi \right| \hat{H} \left| \varphi \right\rangle
\]

In generale, il vettore \(\varphi\) può essere espresso come combinazione lineare degli autostati dell'operatore di misura \(\hat{A}\):

\[
\left| \varphi \right\rangle = \sum_{n} \varphi_{n} \left| a_{n} \right\rangle
\]

Il valor medio dell'operatore di misura è, dunque:

\[
\left\langle \varphi \right|\hat{A}\left| \varphi \right\rangle = \left\langle \varphi \right|\hat{A}\sum_{n}^{}{\varphi_{n}\left| a_{n} \right\rangle
}
\]

Per la linearità si ha:

\[
\left\langle \varphi \right|\hat{A}\left| \varphi \right\rangle = \left\langle \varphi \right|\hat{A}\sum_{n}^{}{\varphi_{n}\left| a_{n} \right\rangle} = \left\langle \varphi \right|\sum_{n}^{}{\varphi_{n}\hat{A}\left| a_{n} \right\rangle}
\]

Anche il vettore bra può essere espresso mediante gli autovettori dell'operatore \(\hat{A}\), tuttavia, si rende necessario l'uso del complesso coniugato, in modo che il prodotto scalare sia reale:

\[
\left\langle \varphi \right| = \sum_{k} \varphi_{k}^{*} \left\langle a_{k} \right|
\]

Sostituendo nell'espresso per il valor medio si ha:

\[
\left\langle \varphi \right|\hat{A}\left| \varphi \right\rangle = \sum_{k}^{}{\varphi_{k}^{*}\left\langle a_{k} \right|}\sum_{n}^{}{\varphi_{n}\hat{A}\left| a_{n} \right\rangle} = \sum_{n}^{}{\sum_{k}^{}{\varphi_{k}^{*}\varphi_{n}\left\langle a_{k} \right|\hat{A}\left| a_{n} \right\rangle}}
\]

Poiché \(\hat{A} \left| a_{n} \right\rangle = a_{n} \left| a_{n} \right\rangle\), ovvero \(\left| a_{n} \right\rangle\) sono gli autostati dell'operatore \(\hat{A}\), si ha:

\[
\left\langle \varphi \right|\hat{A}\left| \varphi \right\rangle  = \sum_{n}^{}{\sum_{k}^{}{\varphi_{k}^{*}\varphi_{n}\left\langle a_{k} \right|\hat{A}\left| a_{n} \right\rangle}} = \sum_{n}^{}{\sum_{k}^{}{\varphi_{k}^{*}\varphi_{n}a_{n}\left\langle a_{k} \middle| a_{n} \right\rangle}}
\]

Si scelgono gli autostati dell'operatore \(\hat{A}\) in modo che siano ortonormali, ovvero sono ortogonali tra loro, mentre la loro norma è unitaria. Per cui risulta:

\[
\left\langle a_{k} \middle| a_{n} \right\rangle = \delta_{kn} =  \begin{cases}
1,\ \  & k = n \\
0,\ \  & k \neq n
\end{cases}
\]

Dunque, il valor medio si esprime come:

\[
\left\langle \varphi \right|\hat{A}\left| \varphi \right\rangle = \sum_{n}^{}{\left| \varphi_{n} \right|^{2}a_{n}}
\]

La probabilità che la misura dia come risultato \(a_{n}\) è quindi \(\left| \varphi_{n} \right|^2\).

Se la misura ha dato come risultato \(a_{k}\), il sistema si trova nello stato \(\left| a_{k} \right\rangle\). 

Se la misura ha dato come risultato \(a_{k}\), il sistema, dopo la misura, si trova nello stato corrispondente all'autovalore \(a_{k}\), ovvero nell'autostato \(\left| a_{k} \right\rangle\). Di conseguenza, lo stato precedente alla misura non è più conoscibile, ma quello successivo sì.

Se si ripete la misura in un tempo sufficientemente breve, il risultato sarà nuovamente \(a_{k}\), poiché il sistema si trova già in un autostato dell'operatore \(\hat{A}\). Tuttavia, se il tempo tra le due misure è troppo lungo, il sistema potrebbe decadere in uno stato diverso, rendendo il risultato imprevedibile.

\subsection{Evoluzione libera degli stati}\label{evoluzione-libera-degli-stati}

Se il sistema microscopico non viene perturbato, evolve liberamente secondo l'equazione differenziale deterministica di Schrödinger:

\[
\hat{H}\left| \varphi \right\rangle = j\hslash\dfrac{d}{dt}\left| \varphi \right\rangle
\]

Dove \(\hat{H}\) è l'hamiltoniana del sistema.
A seguito della misura dell'operatore \(\hat{H}\), il sistema microscopico collassa in un autostato \(\left| \varphi_{n} \right\rangle\) con autovalore \(E_{n}\), che rappresenta l'energia dello stato:

\[
\hat{H}\left| \varphi_{n} \right\rangle = E_{n}\left| \varphi_{n} \right\rangle
\]

L'equazione di Schrödinger si scrive:

\[
j\hslash\dfrac{d}{dt}\left| \varphi_{n} \right\rangle = \hat{H}\left| \varphi_{n} \right\rangle = E_{n}\left| \varphi_{n} \right\rangle
\]

La soluzione di questa equazione differenziale è:

\[
\left| \varphi_{n}(t) \right\rangle = \left| \varphi_{n}\left( t_{0} \right) \right\rangle\exp\left( -j\dfrac{E_{n}}{\hslash}\left( t - t_{0} \right) \right)
\]

Dove \(t_{0}\) è l'istante di misura.

Per uno stato generico \(\left| \varphi \right\rangle\), è possibile generalizzare questo risultato sviluppando lo stato iniziale come combinazione lineare degli autostati dell'operatore hamiltoniano, con coefficienti \(c_n = \left\langle \varphi_{n}(t_0) \middle| \varphi(t_0) \right\rangle\):

\[
\left| \varphi(t_0) \right\rangle = \sum_{n}^{}{c_{n}\left| \varphi_{n}\left( t_{0} \right) \right\rangle}
\]

L'evoluzione dello stato generico è data dall'evoluzione di ciascun autostato componente:

\[
\left| \varphi(t) \right\rangle = \sum_{n}^{}{c_{n}\left| \varphi_{n}\left( t_{0} \right) \right\rangle\exp\left\lbrack -j\dfrac{E_{n}}{\hslash}\left( t - t_{0} \right) \right\rbrack}
\]


È possibile giungere al \textbf{principio di indeterminazione di Heisenberg} osservando che, nello spazio vettoriale degli stati, gli operatori non soddisfano in generale la proprietà algebrica commutativa. Dati due operatori \(\hat{A}\) e \(\hat{B}\), si ha:

\[
\hat{A}\hat{B} \neq \hat{B}\hat{A}
\]

Ciò implica che la misura consecutiva di due grandezze fisiche diverse su un sistema microscopico non porta allo stesso risultato, indipendentemente dall'ordine con cui si effettuano le misure. Di conseguenza, non è possibile determinare simultaneamente quantità di moto e posizione, in accordo con il principio di Heisenberg:

\[
{\Delta}p{\Delta}x \geq \dfrac{\hslash}{2}
\]

In altre parole, una misura di posizione altera la misura della quantità di moto e viceversa. Questo effetto è legato alla necessità di perturbare il sistema per osservare una grandezza. Se si misura la posizione, non è più possibile ottenere altre informazioni sullo stato iniziale del sistema. Nota la posizione della particella, si dice che la \textbf{funzione d'onda collassa}, poiché la probabilità di ritrovare la particella in altre posizioni si annulla.

Nel contesto probabilistico della meccanica quantistica, il concetto deterministico di \textbf{traiettoria} perde di significato.

\subsection{Notazione di Dirac per lo spin}\label{notazione-di-dirac-per-lo-spin}

L'esperimento di Stern e Gerlach ha evidenziato che il momento angolare delle particelle microscopiche deve essere quantizzato. Affinché i risultati sperimentali siano in accordo con le previsioni teoriche, è necessario ammettere l'esistenza dell'operatore spin \(\hat{S}\), rappresentante il \textbf{momento angolare intrinseco} di una particella.

Nel caso di un protone o di un elettrone in un campo magnetico, gli stati possibili dello spin sono due, indicati con \(+\) e \(-\), in base alle due possibili orientazioni lungo la direzione \(z\).

Siano \(\left| + \right\rangle\) e \(\left| - \right\rangle\) i due stati dello spin, e \(\hat{S}_z\) l'operatore di spin lungo l'asse \(z\). Applicando la misura del momento magnetico \(\hat{S}_z\) a uno stato \(\left| \varphi \right\rangle\), si ha:


\[
\hat{S}_z \left| \varphi \right\rangle = s_z \left| \varphi \right\rangle
\]

dove \(s_z\) sono gli autovalori dell'operatore \(\hat{S}_z\), che possono assumere solo due valori:

\[
+ \dfrac{\hslash}{2}, \quad - \dfrac{\hslash}{2}
\]


I vettori \(\left| + \right\rangle\) e \(\left| - \right\rangle\) sono gli autovettori dell'operatore \({\hat{S}}_{z}\), dunque, risulta:

\[
\hat{S}_z \left| + \right\rangle = + \dfrac{\hslash}{2} \left| + \right\rangle, \quad
\hat{S}_z \left| - \right\rangle = - \dfrac{\hslash}{2} \left| - \right\rangle
\]

\begin{figure}[ht]
\centering
\includegraphics[width=2.72257in,height=2.79167in,alt={P2063\#yIS1}]{media/4_Quantiatica/image42.pdf}\caption{Orientazione dello spin}
\end{figure}

Un qualsiasi stato può essere espresso come combinazione lineare dei due autostati dell'operatore \({\hat{S}}_{z}\):

\[
\left| \varphi \right\rangle = \varphi_{+}\left| + \right\rangle + \varphi_{-}\left| - \right\rangle
\]

L'operazione di misura dello spin \({\hat{S}}_{z}\) lungo l'asse \(z\) è fondamentale poiché le proiezioni del momento magnetico della particella lungo gli altri assi si dimostra essere dipendente da \({\hat{S}}_{z}\).

Sia \(\left\langle \varphi \right|\) lo stato iniziale e \(\left| \psi \right\rangle\) lo stato finale. La quantità \(\left\langle \varphi \right| \hat{S}_z \left| \psi \right\rangle\) rappresenta la misura della proiezione del momento magnetico lungo \(z\).

Sia \(\left\langle \varphi \right|\) lo stato iniziale dello spin della particella e \(\left| \psi \right\rangle\) lo stato finale. La quantità \(\left\langle \varphi \right|{\hat{S}}_{z}\left| \psi \right\rangle\) indica rappresenta la misura della proiezione del momento magnetico lungo \(z\) di una particella inizialmente nello stato \(\left\langle \varphi \right|\). A fine della misura, la particella si porta nello stato \(\left| \psi \right\rangle\). Se lo stato iniziale coincide con l'autostato \(\left\langle + \right|\) e anche quello finale \(\left| + \right\rangle\), allora risulta:

\[
\left\langle + \right|{\hat{S}}_{z}\left| + \right\rangle = \left\langle + \right|\left( + \dfrac{\hslash}{2} \right)\left| + \right\rangle = + \dfrac{\hslash}{2}\left\langle + \middle| + \right\rangle
\]

Siccome gli autovettori sono ortonormali, risulta:

\[
\left\langle + \middle| + \right\rangle = 1
\]

Quindi, in definitiva, si ottiene:

\[
\left\langle + \right|{\hat{S}}_{z}\left| + \right\rangle = + \dfrac{\hslash}{2}
\]

Se lo stato iniziale coincide con l'autostato \(\left\langle - \right|\) e anche quello finale \(\left| - \right\rangle\), poiché gli autovettori sono ortonormali, si ha:

\[
\left\langle - \right|{\hat{S}}_{z}\left| - \right\rangle = \left\langle - \right| - \dfrac{\hslash}{2}\left| - \right\rangle = - \dfrac{\hslash}{2}\left\langle - \middle| - \right\rangle = - \dfrac{\hslash}{2}
\]

Se lo stato iniziale coincide con l'autostato \(\left\langle + \right|\) e anche quello finale \(\left| - \right\rangle\), risulta:

\[
\left\langle + \right|{\hat{S}}_{z}\left| - \right\rangle = \left\langle + \right| - \dfrac{\hslash}{2}\left| - \right\rangle = - \dfrac{\hslash}{2}\left\langle + \middle| - \right\rangle = 0
\]

Se lo stato iniziale coincide con l'autostato \(\left\langle - \right|\) e anche quello finale \(\left| + \right\rangle\), risulta:

\[
\left\langle - \right|{\hat{S}}_{z}\left| + \right\rangle = \left\langle - \right| + \dfrac{\hslash}{2}\left| + \right\rangle = + \dfrac{\hslash}{2}\left\langle - \middle| + \right\rangle = 0
\]

Siccome i vettori della base dell'operatore \({\hat{S}}_{z}\) sono due, è possibile associare il vettore \((1,0)^{T}\) all'autostato \(\left| + \right\rangle\) e \((0,1)^{T}\) all'autostato \(\left| - \right\rangle\). Dato che sono possibili quattro combinazioni, tra stato iniziale e finale dopo la misura, l'operatore \({\hat{S}}_{z}\) ha dimensione finita. È possibile associare una matrice alla trasformazione:

\[
\begin{pmatrix}
\left\langle + \right|{\hat{S}}_{z}\left| + \right\rangle & \left\langle + \right|{\hat{S}}_{z}\left| - \right\rangle \\
\left\langle - \right|{\hat{S}}_{z}\left| + \right\rangle & \left\langle - \right|{\hat{S}}_{z}\left| - \right\rangle
\end{pmatrix} = \begin{pmatrix}
 + \dfrac{\hslash}{2} & 0 \\
0 & - \dfrac{\hslash}{2}
\end{pmatrix} = \dfrac{\hslash}{2}\begin{pmatrix}
1 & 0 \\
0 & - 1
\end{pmatrix}
\]

La matrice:

\[
\boldsymbol{\sigma}_{z} = \begin{pmatrix}
1 & 0 \\
0 & - 1
\end{pmatrix}
\]

È detta matrice di Pauli e riassume i valori dello spin lungo l'asse \(z\) e consente di scrivere l'operatore di misura del momento magnetico come:

\[{\hat{S}}_{z} = \dfrac{\hslash}{2}{\boldsymbol{\sigma}}_{z}\]

Tramite la matrice di Pauli è possibile, inoltre, calcolare lo stato finale di una particella. Ad esempio, se lo stato iniziale di una particella è \(\left\langle + \right|\), lo stato finale può essere espresso come:

\[
{\hat{S}}_{z}\left| + \right\rangle = \dfrac{\hslash}{2}{\boldsymbol{\sigma}}_{z}\left| + \right\rangle = \dfrac{\hslash}{2}\begin{pmatrix}
1 & 0 \\
0 & - 1
\end{pmatrix}\left( \begin{array}{r}
1 \\
0
\end{array} \right) = \dfrac{\hslash}{2}\left( \begin{array}{r}
1 \\
0
\end{array} \right) = \dfrac{\hslash}{2}\left| + \right\rangle
\]

È possibile definire le matrici di Pauli anche per la misura del momento magnetico intrinseco lungo gli altri assi, come:

\[
\hat{S}_x = \dfrac{\hslash}{2}
\begin{pmatrix}
0 & 1 \\
1 & 0
\end{pmatrix}, \quad
\hat{S}_y = \dfrac{\hslash}{2}
\begin{pmatrix}
0 & -j \\
j & 0
\end{pmatrix}
\]

Se una particella si trova in uno stato \(\left| + \right\rangle\), autovettore di \({\hat{S}}_{z}\), allora lungo \(x\) si ha uno stato:

\[
{\hat{S}}_{x}\left| + \right\rangle = \dfrac{\hslash}{2}\begin{pmatrix}
0 & 1 \\
1 & 0
\end{pmatrix}\left( \begin{array}{r}
1 \\
0
\end{array} \right) = \dfrac{\hslash}{2}\left( \begin{array}{r}
0 \\
1
\end{array} \right) = \dfrac{\hslash}{2}\left| - \right\rangle
\]

Se, invece, lo stato iniziale è \(\left| - \right\rangle\), lungo l'asse \(x\), si ha:

\[{
\hat{S}}_{x}\left| - \right\rangle = \dfrac{\hslash}{2}\begin{pmatrix}
0 & 1 \\
1 & 0
\end{pmatrix}\left( \begin{array}{r}
0 \\
1
\end{array} \right) = \dfrac{\hslash}{2}\left( \begin{array}{r}
1 \\
0
\end{array} \right) = \dfrac{\hslash}{2}\left| + \right\rangle
\]

I vettori \(\left| + \right\rangle\) e \(\left| - \right\rangle\) non sono una base per l'operatore \({\hat{S}}_{x}\) poiché, l'applicazione di questo operatore a uno dei due vettori, non restituisce un vettore parallelo.

Analogamente, è possibile ripetere lo stesso discorso per l'operatore \({\hat{S}}_{y}\):

\[{\hat{S}}_{y}\left| + \right\rangle = \dfrac{\hslash}{2}\begin{pmatrix}
0 & - j \\
j & 0
\end{pmatrix}\left( \begin{array}{r}
1 \\
0
\end{array} \right) = j\dfrac{\hslash}{2}\left( \begin{array}{r}
0 \\
1
\end{array} \right) = j\dfrac{\hslash}{2}\left| - \right\rangle\]

\[{\hat{S}}_{y}\left| - \right\rangle = \dfrac{\hslash}{2}\begin{pmatrix}
0 & - j \\
j & 0
\end{pmatrix}\left( \begin{array}{r}
0 \\
1
\end{array} \right) = - j\dfrac{\hslash}{2}\left( \begin{array}{r}
1 \\
0
\end{array} \right) = - j\dfrac{\hslash}{2}\left| + \right\rangle\]

Anche per \({\hat{S}}_{y}\), i vettori \(\left| + \right\rangle\) e \(\left| - \right\rangle\) non sono una base per questo operatore.

Posizionando un campo magnetico \(\vec{B}\) diretto lungo l'asse \(z\) e misurando \(\hat{S}_z\), si ottiene un valore certo di \(S_z\) (\(\pm \hslash/2\)). Tuttavia, a causa della \textbf{non-commutatività} degli operatori di spin, ovvero:

\[
[\hat{S}_i, \hat{S}_j] \neq 0 \quad \text{per } i \neq j,
\]

lo stato risultante \(\left| + \right\rangle\) o \(\left| - \right\rangle\) non è un autostato di \(\hat{S}_x\) o \(\hat{S}_y\). Di conseguenza, non è possibile conoscere simultaneamente le componenti dello spin lungo gli assi \(x\) e \(y\): la misura di \(S_z\) rende le componenti \(S_x\) e \(S_y\) completamente \textbf{indeterminate}, con valore di aspettazione nullo e massima incertezza:

\[
\Delta S_x = \Delta S_y = \dfrac{\hslash}{2}
\]

Non è dunque possibile ricostruire tridimensionalmente il vettore momento intrinseco con precisione illimitata.

\subsection{Split dei livelli energetici}\label{split-dei-livelli-energetici}

Considerando un nucleo immerso in un campo magnetico \(\vec{B}\) diretto lungo l'asse \(z\), l'operatore hamiltoniano totale \(\hat{H}\) si compone dell'hamiltoniano imperturbato \(\hat{H}^{(0)}\) (energia cinetica e potenziale) e dell'energia potenziale magnetica \(\hat{U}\):

\[
\hat{H} = \hat{H}^{(0)} + \hat{U} = \left( \dfrac{{\hat{p}}^{2}}{2m} + \hat{V} \right) - \hat{\vec{\mu}} \cdot \vec{B}
\]

dove \(\hat{V}\) è l'energia potenziale coulombiana che agisce sul nucleo, \(\hat{\vec{\mu}}\) è l'operatore momento magnetico intrinseco e \(\hat{U}\) è l'energia potenziale magnetica:

\[
\hat{U} = - \hat{\mu} \cdot \vec{B}
\]

\begin{figure}[ht]
\centering
\includegraphics[width=1.94783in,height=2.39583in,alt={P2108\#yIS1}]{media/4_Quantiatica/image43.pdf}
\caption{Nucleo immerso in un campo magnetico}
\end{figure}

Proiettando lungo \(z\), il momento magnetico è legato all'operatore di spin dal rapporto giromagnetico \(\gamma\): \(\hat{\mu}_z = \gamma \hat{S}_z\). Poiché il campo magnetico ha solo componente lungo \(\hat{\imath}_z\), ovvero \(\vec{B} = B \hat{\imath}_z\), l'hamiltoniano si scrive:

\[
\hat{H} = \left( \dfrac{{\hat{p}}^{2}}{2m} + \hat{V} \right) - \gamma{\hat{S}}_{z}B
\]

Sia \(E_n^{(0)}\) l'autovalore di \(\hat{H}^{(0)}\) per uno stato orbitale \(\left| \varphi_n \right\rangle\), tale che:

\[
\left( \dfrac{{\hat{p}}^{2}}{2m} + \hat{V} \right)\left| \varphi_{n} \right\rangle = E_{n}^{(0)}\left| \varphi_{n} \right\rangle
\]

Assumendo che lo stato totale sia separabile in una parte orbitale e una di spin, \(\left| \psi_{n,s} \right\rangle = \left| \varphi_n \right\rangle \otimes \left| s_z \right\rangle\), l'applicazione dell'hamiltoniano produce:

\[
\hat{H}\left| \psi_{n,s} \right\rangle = E_{n}^{(0)}\left| \psi_{n,s} \right\rangle - \gamma B{\hat{S}}_{z}\left| \psi_{n,s} \right\rangle
\]

L'operatore \(\hat{S}_z\) possiede solo due autovalori per lo spin \(1/2\):

\[
s_z = \pm \dfrac{\hslash}{2}
\]

Sostituendo tali autovalori, l'energia totale si divide in due livelli:

\[
E = E_n^{(0)} - \gamma B s_z =
\begin{cases}
E_{\text{low}} = E_n^{(0)} - \dfrac{\hslash}{2} \gamma B & \text{per } s_z = +\dfrac{\hslash}{2} \text{ (Spin Up, } \left| + \right\rangle \text{)} \\
E_{\text{high}} = E_n^{(0)} + \dfrac{\hslash}{2} \gamma B & \text{per } s_z = -\dfrac{\hslash}{2} \text{ (Spin Down, } \left| - \right\rangle \text{)}
\end{cases}
\]

Per effetto del campo magnetico diretto lungo \(z\), ogni autovalore dell'energia si divide in due livelli energetici che differiscono di:

\[
\Delta E = E_{\text{high}} - E_{\text{low}} = \hslash \gamma B
\]

Questo fenomeno è noto come \textbf{split dei livelli energetici} o \textbf{effetto Zeeman nucleare}.

\begin{figure}[ht]
\centering
\includegraphics[width=5.59819in,height=1.7684in,alt={P2123\#yIS1}]{media/4_Quantiatica/image44.pdf}\caption{Split di livelli energetici di un nucleo immerso in un campo magnetico}
\end{figure}

L'hamiltoniano di un nucleo in un campo magnetico diretto lungo l'asse \(z\) contiene i termini \(\pm \dfrac{\hslash}{2} \gamma B\). Gli spin si orientano quindi in direzione parallela o antiparallela rispetto al campo magnetico applicato.

Si può trascurare il termine \(\hat{p}^{2}/2m\) poiché il nucleo, avendo massa elevata, può essere considerato fermo. Se il nucleo è in equilibrio, anche il potenziale coulombiano può essere trascurato. L'hamiltoniano si riduce quindi a:

\[
\hat{H}\left| \varphi_{n} \right\rangle = - \gamma B{\hat{S}}_{z}\left| \varphi_{n} \right\rangle
\]

Se lo stato del nucleo è \(\left| + \right\rangle\), allora:

\[
\hat{H}\left| + \right\rangle = - \gamma B{\hat{S}}_{z}\left| + \right\rangle = - \dfrac{\hslash}{2}\gamma B
\]

Per $\gamma > 0$ come nel protone, lo spin è parallelo al campo magnetico poiché il nucleo possiede la minima energia. Al contrario, se lo stato è \(\left| - \right\rangle\), risulta:

\[
\hat{H}\left| - \right\rangle = - \gamma B{\hat{S}}_{z}\left| - \right\rangle = \dfrac{\hslash}{2}\gamma B
\]

 Per $\gamma > 0$, lo spin è antiparallelo al campo magnetico poiché il suo nucleo possiede energia massima.

\subsection{Energia media di un nucleo in base allo spin}\label{energia-media-di-un-nucleo-in-base-allo-spin}

Uno stato qualsiasi del sistema microscopico è descritto dal vettore ket \(\left| \psi \right\rangle\), che può essere decomposto come combinazione lineare degli autostati dell'operatore spin \(\hat{S}_z\):

\[
\hat{H} \left| \psi \right\rangle = C_{+} \hat{H} \left| + \right\rangle + C_{-} \hat{H} \left| - \right\rangle
\]

Si calcola l'energia del sistema applicando l'operatore hamiltoniano:

\[
\hat{H}\left| \psi \right\rangle = \hat{H}\left( C_{+}\left| + \right\rangle + C_{-}\left| - \right\rangle \right) = C_{+}\hat{H}\left| + \right\rangle + C_{-}\hat{H}\left| - \right\rangle
\]

Si valuta il valor medio dell'energia del nucleo, mediante la definizione \(\left\langle \psi \right|\hat{H}\left| \psi \right\rangle\). Sostituendo l'equazione ottenuta per \(\hat{H}\left| \psi \right\rangle\), si ha:

\[
\left\langle \psi \right|\hat{H}\left| \psi \right\rangle = \left\langle \psi \right|\left( C_{+}\hat{H}\left| + \right\rangle + C_{-}\hat{H}\left| - \right\rangle \right) = C_{+}\left\langle \psi \right|\hat{H}\left| + \right\rangle + C_{-}\left\langle \psi \right|\hat{H}\left| - \right\rangle
\]

L'autostato bra \(\left\langle \psi \right|\) può essere espresso come combinazione lineare degli autostati dell'operatore \({\hat{S}}_{z}\), mediante coefficienti complessi e coniugati:

\[
\left\langle \psi \right| = C_{+}^{*}\left\langle + \right| + C_{-}^{*}\left\langle - \right|
\]

Sostituendo nella relazione per \(\left\langle \psi \right|\hat{H}\left| \psi \right\rangle\), si ha:

\[
\left\langle \psi \right|\hat{H}\left| \psi \right\rangle = C_{+}\left( C_{+}^{*}\left\langle + \right| + C_{-}^{*}\left\langle - \right| \right)\hat{H}\left| + \right\rangle + C_{-}\left( C_{+}^{*}\left\langle + \right| + C_{-}^{*}\left\langle - \right| \right)\hat{H}\left| - \right\rangle
\]

Svolgendo i prodotti, si ottiene:

\[
\left\langle \psi \right|\hat{H}\left| \psi \right\rangle = C_{+}C_{+}^{*}\left\langle + \right|\hat{H}\left| + \right\rangle + C_{+}C_{-}^{*}\left\langle - \right|\hat{H}\left| + \right\rangle + C_{-}C_{+}^{*}\left\langle + \right|\hat{H}\left| - \right\rangle + C_{-}C_{-}^{*}\left\langle - \right|\hat{H}\left| - \right\rangle
\]

Dove \(C_{+}C_{+}^{*} = \left| C_{+} \right|^{2}\) e \(C_{-}C_{-}^{*} = \left| C_{-} \right|^{2}\), dunque, è possibile scrive:

\[
\left\langle \psi \right|\hat{H}\left| \psi \right\rangle = \left| C_{+} \right|^{2}\left\langle + \right|\hat{H}\left| + \right\rangle + C_{+}C_{-}^{*}\left\langle - \right|\hat{H}\left| + \right\rangle + C_{-}C_{+}^{*}\left\langle + \right|\hat{H}\left| - \right\rangle + \left| C_{-} \right|^{2}\left\langle - \right|\hat{H}\left| - \right\rangle
\]

Per un nucleo immesso in un campo magnetico di ampiezza \(B_{0}\) diretto lungo \(z\), l'operatore hamiltoniano, trascurano la velocità del nucleo e il campo coulombiano, può essere espresso come:

\[
\hat{H} = - \gamma B_0 \hat{S}_z
\]

Il valor medio dell'energia del nucleo, può essere espressa come:

\[
\left\langle \psi \right|\hat{H}\left| \psi \right\rangle = - \gamma B_{0}\left( \left| C_{+} \right|^{2}\left\langle + \right|{\hat{S}}_{z}\left| + \right\rangle + C_{+}C_{-}^{*}\left\langle - \right|{\hat{S}}_{z}\left| + \right\rangle + C_{-}C_{+}^{*}\left\langle + \right|{\hat{S}}_{z}\left| - \right\rangle + \left| C_{-} \right|^{2}\left\langle - \right|{\hat{S}}_{z}\left| - \right\rangle \right)
\]

Dalla definizione di matrice di Pauli, risulta:

\[{\hat{S}}_{z}\left| + \right\rangle = \dfrac{\hslash}{2}{\boldsymbol{\sigma}}_{z}\left| + \right\rangle = \dfrac{\hslash}{2}\begin{pmatrix}
1 & 0 \\
0 & - 1
\end{pmatrix}\left( \begin{array}{r}
1 \\
0
\end{array} \right) = \dfrac{\hslash}{2}\left( \begin{array}{r}
1 \\
0
\end{array} \right) = \dfrac{\hslash}{2}\left| + \right\rangle\]

\[{\hat{S}}_{z}\left| - \right\rangle = \dfrac{\hslash}{2}{\boldsymbol{\sigma}}_{z}\left| - \right\rangle = \dfrac{\hslash}{2}\begin{pmatrix}
1 & 0 \\
0 & - 1
\end{pmatrix}\left( \begin{array}{r}
0 \\
1
\end{array} \right) = \dfrac{\hslash}{2}\left( \begin{array}{r}
0 \\
 - 1
\end{array} \right) = - \dfrac{\hslash}{2}\left| - \right\rangle\]

Inoltre, i vettori \(\left| + \right\rangle\) e \(\left| - \right\rangle\) sono ortonormali, dunque, risulta che:

\[\left\langle + \right|{\hat{S}}_{z}\left| + \right\rangle = \left\langle + \right|\dfrac{\hslash}{2}\left| + \right\rangle = \dfrac{\hslash}{2}\left\langle + \middle| + \right\rangle = \dfrac{\hslash}{2}\]

\[\left\langle - \right|{\hat{S}}_{z}\left| - \right\rangle = - \left\langle - \right|\dfrac{\hslash}{2}\left| - \right\rangle = - \dfrac{\hslash}{2}\left\langle - \middle| - \right\rangle = - \dfrac{\hslash}{2}\]

\[\left\langle - \right|{\hat{S}}_{z}\left| + \right\rangle = \left\langle - \right|\dfrac{\hslash}{2}\left| + \right\rangle = \dfrac{\hslash}{2}\left\langle - \middle| + \right\rangle = 0\]

\[\left\langle + \right|{\hat{S}}_{z}\left| - \right\rangle = - \left\langle + \right|\dfrac{\hslash}{2}\left| - \right\rangle = - \dfrac{\hslash}{2}\left\langle + \middle| - \right\rangle = 0\]

Si ottiene:

\[\left\langle \psi \right|\hat{H}\left| \psi \right\rangle = - \gamma B_{0}\left( \dfrac{\hslash}{2}\left| C_{+} \right|^{2} - \dfrac{\hslash}{2}\left| C_{-} \right|^{2} \right) = - \gamma B_{0}\dfrac{\hslash}{2}\left( \left| C_{+} \right|^{2} - \left| C_{-} \right|^{2} \right)\]

Il parametro \(\left| C_{+} \right|^{2}\) rappresenta la probabilità che il nucleo sia nello stato \(\left| + \right\rangle\) alla fine del processo di misura. Analogamente, \(\left| C_{-} \right|^{2}\) rappresenta la probabilità che il nucleo sia nello stato \(\left| - \right\rangle\).

In un insieme di \(N\) nuclei (come in un campione NMR), se si assume che la probabilità di trovare un singolo nucleo negli autostati \(|+\rangle\) o \(|-\rangle\) sia data dalle popolazioni termiche di equilibrio, il numero atteso di nuclei in ciascuno stato è, per lo stato \(|+\rangle\):

\[
N_{+} = N P_{+} = N\left| C_{+} \right|^{2}
\]

per lo stato \(|-\rangle\), invece:

\[
N_{-} = N P_{-} = N\left| C_{-} \right|^{2}
\]

Sia \(\Delta N = N_{+} - N_{-}\) la differenza di popolazione tra gli spin nello stato \(\left| + \right\rangle\) e quelli nello stato \(\left| - \right\rangle\):

\[
\Delta N = N_{+} - N_{-} = N\left| C_{+} \right|^{2} - N\left| C_{-} \right|^{2} = N \left( \left| C_{+} \right|^{2} - \left| C_{-} \right|^{2} \right)
\]

Dividendo per \(N\), si ottiene l'eccesso di popolazione normalizzato:

\[
\dfrac{\Delta N}{N} = \left| C_{+} \right|^{2} - \left| C_{-} \right|^{2}
\]

Sostituendo questa relazione nell'espressione del valor medio dell'energia \(\left\langle \psi \right|\hat{H}\left| \psi \right\rangle\), si ottiene:

\[
\left\langle \psi \right|\hat{H}\left| \psi \right\rangle = - \gamma B_{0}\dfrac{\hslash}{2}\dfrac{\Delta N}{N}
\]

L'energia media dei nuclei è data dalla differenza di nuclei nello stato \(\left| + \right\rangle\) rispetto a quelli nello stato \(\left| - \right\rangle\), rapportato al numero totale dei nuclei.

\subsection{Evoluzione temporale dello stato spin up}\label{evoluzione-temporale-dello-stato-left-mathbf-rightrangle}

A seguito della misura, il sistema evolve secondo una legge deterministica. Lo stato quantico ha un'evoluzione temporale descritta da:

\[
\left| \varphi_{n}(t) \right\rangle = \left| \varphi_{n}\left( t_{0} \right) \right\rangle\exp\left( -j\dfrac{E_{n}}{\hslash}\left( t - t_{0} \right) \right)
\]

Se il sistema si trova nello stato \(\left| \mathbf{+} \right\rangle\), l'energia associata è:

\[
E^{+} = - \dfrac{\hslash}{2}\gamma B_{0}
\]

Per cui, l'evoluzione temporale dello stato è:

\[
\left| + (t) \right\rangle = \left| + \left( t_{0} \right) \right\rangle\exp\left( - j \dfrac{1}{\hslash} \left( - \dfrac{\hslash}{2}\gamma B_{0} \right) \left( t - t_{0} \right) \right) = \left| + \left( t_{0} \right) \right\rangle\exp\left( + j\dfrac{\gamma B_{0}}{2}\left( t - t_{0} \right) \right)
\]

Analogamente, nello stato \(\left| \mathbf{-} \right\rangle\) il sistema possiede un energia data da:

\[E^{-} = \dfrac{\hslash}{2}\gamma B_{0}\]

Per cui, l'andamento dello stato è:

\[
\left| - (t) \right\rangle = \left| - \left( t_{0} \right) \right\rangle\exp\left( - j \dfrac{1}{\hslash} \left( + \dfrac{\hslash}{2}\gamma B_{0} \right) \left( t - t_{0} \right) \right) = \left| - \left( t_{0} \right) \right\rangle\exp\left( - j\dfrac{\gamma B_{0}}{2}\left( t - t_{0} \right) \right)
\]

Un qualsiasi stato \(\left| \psi \right\rangle\) del sistema può essere espresso come combinazione lineare degli autostati dell'operatore \({\hat{S}}_{z}\):

\[
\left| \psi(t) \right\rangle = C_{+}\left| + \left( t_{0} \right) \right\rangle\exp\left( + j\dfrac{\gamma B_{0}}{2}\left( t - t_{0} \right) \right) + C_{-}\left| - \left( t_{0} \right) \right\rangle\exp\left( - j\dfrac{\gamma B_{0}}{2}\left( t - t_{0} \right) \right)
\]

\subsection{Valore medio del momento magnetico sull'asse longitudinale}\label{valore-medio-del-momento-magnetico-su-mathbfz}

L'operatore momento angolare su \(z\), \({\hat{\mu}}_{z}\), è legato all'operatore di spin \({\hat{S}}_{z}\), dalla relazione:

\[{\hat{\mu}}_{z} = \gamma{\hat{S}}_{z}\]

È possibile calcolare il valor medio di tale operatore mediante la definizione:

\[\left\langle \psi \right|{\hat{\mu}}_{z}\left| \psi \right\rangle = \left\langle \psi \right|\gamma{\hat{S}}_{z}\left| \psi \right\rangle = \gamma\left\langle \psi \right|{\hat{S}}_{z}\left| \psi \right\rangle\]

È possibile esprimere gli stati come sovrapposizione degli autostati dell'operatore spin:

\[\left\langle \psi \right| = C_{+}^{*}\left\langle + \right| + C_{-}^{*}\left\langle - \right|\]

\[\left| \psi \right\rangle = C_{+}\left| + \right\rangle + C_{-}\left| - \right\rangle\]

Sostituendo queste due relazioni nel valor medio, si ottiene:

\[\left\langle \psi \right|{\hat{\mu}}_{z}\left| \psi \right\rangle = \left( C_{+}^{*}\left\langle + \right| + C_{-}^{*}\left\langle - \right| \right){\hat{S}}_{z}\left( C_{+}\left| + \right\rangle + C_{-}\left| - \right\rangle \right)\]

Si svolgono i prodotti:

\[\left\langle \psi \right|{\hat{\mu}}_{z}\left| \psi \right\rangle = \gamma\left( {C_{+}C}_{+}^{*}\left\langle + \right|{\hat{S}}_{z}\left| + \right\rangle + C_{+}C_{-}^{*}\left\langle - \right|{\hat{S}}_{z}\left| + \right\rangle + C_{-}C_{+}^{*}\left\langle + \right|{\hat{S}}_{z}\left| - \right\rangle + C_{-}C_{-}^{*}\left\langle - \right|{\hat{S}}_{z}\left| - \right\rangle \right)\]

Dove:
\[
\begin{cases}
    \left\langle + \right|{\hat{S}}_{z}\left| + \right\rangle = \dfrac{\hslash}{2}\left\langle + \middle| + \right\rangle = \dfrac{\hslash}{2} \\
    \left\langle - \right|{\hat{S}}_{z}\left| - \right\rangle = - \dfrac{\hslash}{2}\left\langle - \middle| - \right\rangle = - \dfrac{\hslash}{2} \\
     \left\langle + \right|{\hat{S}}_{z}\left| - \right\rangle = - \dfrac{\hslash}{2}\left\langle + \middle| - \right\rangle = 0 \\
     \left\langle - \right|{\hat{S}}_{z}\left| + \right\rangle = \dfrac{\hslash}{2}\left\langle - \middle| + \right\rangle = 0    
\end{cases}
\]

Per cui si ha:

\[\left\langle \psi \right|{\hat{\mu}}_{z}\left| \psi \right\rangle = \gamma\dfrac{\hslash}{2}\left( \left| C_{+} \right|^{2} - \left| C_{-} \right|^{2} \right)\]

\subsection{Andamento temporale dello stato momento magnetico lungo l'asse trasversale}\label{andamento-temporale-dello-stato-momento-magnetico-lungo-mathbfx}

Si vuole calcolare l'energia media nel tempo dell'operatore momento magnetico lungo \(x\), \({\hat{\mu}}_{x}\). Siccome il momento magnetico possiede solo due autostati, può assumere solamente due valori. Lo stato energetico dei due autostati \(\left| + (t) \right\rangle\) e \(\left| - (t) \right\rangle\) sono rispettivamente:

\[E^{-} = \dfrac{\hslash}{2}\gamma B_{0},\ \ E^{+} = - \dfrac{\hslash}{2}\gamma B_{0}\]

L'evoluzione temporale dei due autostati stazionari è:

\[\left| + (t) \right\rangle = \left| + \left( t_{0} \right) \right\rangle\exp\left\lbrack - j\dfrac{E^{+}}{\hslash}\left( t - t_{0} \right) \right\rbrack\]

\[\left| - (t) \right\rangle = \left| - \left( t_{0} \right) \right\rangle\exp\left\lbrack - j\dfrac{E^{-}}{\hslash}\left( t - t_{0} \right) \right\rbrack\]

Sia \(\left| \psi(t) \right\rangle\) un qualsiasi stato, esso può essere ottenuto come combinazione lineare dei due autostati stazionari:

\[\left| \psi\left( t_{0} \right) \right\rangle = C_{+}\left| + \left( t_{0} \right) \right\rangle + C_{-}\left| - \left( t_{0} \right) \right\rangle\]

Dunque, l'evoluzione temporale è:

\[\left| \psi(t) \right\rangle = C_{+}\left| + \left( t_{0} \right) \right\rangle\exp\left\lbrack - j\dfrac{E^{+}}{\hslash}\left( t - t_{0} \right) \right\rbrack + C_{-}\left| - \left( t_{0} \right) \right\rangle\exp\left\lbrack - j\dfrac{E^{-}}{\hslash}\left( t - t_{0} \right) \right\rbrack\]

Si valuta l'energia media dell'operatore \({\hat{\mu}}_{x}\), sostituendo la combinazione per \(\left| \psi(t) \right\rangle\), si ha:

\[\left\langle \psi(t) \right|{\hat{\mu}}_{x}\left| \psi(t) \right\rangle = \left\langle \psi(t) \right|{\hat{\mu}}_{x}C_{+}\left| + \left( t_{0} \right) \right\rangle\exp\left\lbrack - j\dfrac{E^{+}}{\hslash}\left( t - t_{0} \right) \right\rbrack + \left\langle \psi(t) \right|{\hat{\mu}}_{x}C_{-}\left| - \left( t_{0} \right) \right\rangle\exp\left\lbrack - j\dfrac{E^{-}}{\hslash}\left( t - t_{0} \right) \right\rbrack\]

L'operatore momento magnetico lungo \(x\) può essere espresso in termini dell'operatore momento angolare intrinseco lungo lo stesso asse, \({\hat{S}}_{x}\):

\[{\hat{\mu}}_{x} = \gamma{\hat{S}}_{x}\]

Mediante la matrice di Pauli è possibile valutare il risultato dell'applicazione di \({\hat{S}}_{x}\) agli autostati \(\left| + \right\rangle\) e \(\left| - \right\rangle\):

\[{\hat{\mu}}_{x}\left| + \right\rangle = \gamma{\hat{S}}_{x}\left| + \right\rangle = \gamma\dfrac{\hslash}{2}\begin{pmatrix}
0 & 1 \\
1 & 0
\end{pmatrix}\left( \begin{array}{r}
1 \\
0
\end{array} \right) = \gamma\dfrac{\hslash}{2}\left( \begin{array}{r}
0 \\
1
\end{array} \right) = \gamma\dfrac{\hslash}{2}\left| - \right\rangle\]

\[{\hat{\mu}}_{x}\left| - \right\rangle = \gamma{\hat{S}}_{x}\left| - \right\rangle = \gamma\dfrac{\hslash}{2}\begin{pmatrix}
0 & 1 \\
1 & 0
\end{pmatrix}\left( \begin{array}{r}
0 \\
1
\end{array} \right) = \gamma\dfrac{\hslash}{2}\left( \begin{array}{r}
1 \\
0
\end{array} \right) = \gamma\dfrac{\hslash}{2}\left| + \right\rangle\]

Per cui, l'energia media può essere scritta come:

\[\left\langle \psi(t) \right|{\hat{\mu}}_{x}\left| \psi(t) \right\rangle = \gamma\dfrac{\hslash}{2}\left\{ C_{+}\left\langle \psi(t) \middle| - \left( t_{0} \right) \right\rangle\exp\left\lbrack - j\dfrac{E^{+}}{\hslash}\left( t - t_{0} \right) \right\rbrack + C_{-}\left\langle \psi(t) \middle| + \left( t_{0} \right) \right\rangle\exp\left\lbrack - j\dfrac{E^{-}}{\hslash}\left( t - t_{0} \right) \right\rbrack \right\}\]

Il vettore bra può essere espresso come combinazione lineare degli autostati di \({\hat{S}}_{z}\):

\[\left\langle \psi(t) \right| = C_{+}^{*}\left\langle + (t) \right| + C_{-}^{*}\left\langle - (t) \right| = C_{+}^{*}\left\langle + \left( t_{0} \right) \right|\exp\left\lbrack j\dfrac{E^{+}}{\hslash}\left( t - t_{0} \right) \right\rbrack + C_{-}^{*}\left\langle - \left( t_{0} \right) \right|\exp\left\lbrack j\dfrac{E^{-}}{\hslash}\left( t - t_{0} \right) \right\rbrack\]

Sostituendo questa relazione nel calcolo dell'energia media, si ottengono i risultati:

\[C_{+}C_{+}^{*}\left\langle + \left( t_{0} \right) \middle| - \left( t_{0} \right) \right\rangle\exp\left\lbrack - j\dfrac{E^{+}}{\hslash}\left( t - t_{0} \right) \right\rbrack\exp\left\lbrack j\dfrac{E^{+}}{\hslash}\left( t - t_{0} \right) \right\rbrack = 0\]

\[C_{+}C_{-}^{*}\left\langle - \left( t_{0} \right) \middle| - \left( t_{0} \right) \right\rangle\exp\left\lbrack - j\dfrac{E^{+}}{\hslash}\left( t - t_{0} \right) \right\rbrack\exp\left\lbrack j\dfrac{E^{-}}{\hslash}\left( t - t_{0} \right) \right\rbrack = C_{+}C_{-}^{*}\exp\left\lbrack - j\dfrac{E^{+} - E^{-}}{\hslash}\left( t - t_{0} \right) \right\rbrack\]

\[C_{-}C_{+}^{*}\left\langle - \left( t_{0} \right) \middle| - \left( t_{0} \right) \right\rangle\exp\left\lbrack - j\dfrac{E^{-}}{\hslash}\left( t - t_{0} \right) \right\rbrack\exp\left\lbrack j\dfrac{E^{+}}{\hslash}\left( t - t_{0} \right) \right\rbrack = C_{-}C_{+}^{*}\exp\left\lbrack j\dfrac{E^{+} - E^{-}}{\hslash}\left( t - t_{0} \right) \right\rbrack\]

\[C_{-}^{*}C_{+}^{*}\left\langle - \left( t_{0} \right) \middle| + \left( t_{0} \right) \right\rangle\exp\left\lbrack - j\dfrac{E^{-}}{\hslash}\left( t - t_{0} \right) \right\rbrack\exp\left\lbrack j\dfrac{E^{-}}{\hslash}\left( t - t_{0} \right) \right\rbrack = 0\]

L'energia media del momento magnetico lungo \(x\) è, dunque:

\[\left\langle \psi(t) \right|{\hat{\mu}}_{x}\left| \psi(t) \right\rangle = \gamma\dfrac{\hslash}{2}\left\{ C_{+}C_{-}^{*}\exp\left\lbrack - j\dfrac{E^{+} - E^{-}}{\hslash}\left( t - t_{0} \right) \right\rbrack + C_{-}C_{+}^{*}\exp\left\lbrack j\dfrac{E^{+} - E^{-}}{\hslash}\left( t - t_{0} \right) \right\rbrack \right\}\]

Si pone \(C_{+}C_{-}^{*} = A\exp{j\beta}\), dove compaiono esplicitamente modulo, \(A = \left| C_{+}C_{-}^{*} \right|\), e fase, \(\angle C_{+}C_{-}^{*}\), del prodotto \(C_{+}C_{-}^{*}\). La relazione per il valor medio del momento magnetico lungo \(x\) può essere scritta tenendo conto delle proprietà dei numeri complessi: la somma di un numero complesso e del suo coniugato restituisce il doppio della parte reale del numero stesso. Questa relazione può essere scritta anche come:

\[\left\langle \psi(t) \right|{\hat{\mu}}_{x}\left| \psi(t) \right\rangle = \gamma\hslash Re\left\{ C_{-}C_{+}^{*}\exp\left\lbrack j\dfrac{E^{+} - E^{-}}{\hslash}\left( t - t_{0} \right) \right\rbrack \right\}\]

Si pone:

\[
\omega_0 = \dfrac{E^{-} - E^{+}}{\hslash} 
\]

Sostituendo le espressioni per le due energie, \(E^{-} = \hslash/2\gamma B_{0}\) e \(E^{+} = - \hslash/2\gamma B_{0}\), si ottiene:

\[
\omega_0 = \dfrac{E^{-} - E^{+}}{\hslash} = \dfrac{1}{\hslash}\left( \dfrac{\hslash}{2}\gamma B_{0} - \left( - \dfrac{\hslash}{2}\gamma B_{0} \right) \right) = \gamma B_0
\]

\(\omega_{0}\) è detta pulsazione di Larmor, quantità positiva se \(\gamma>0\).

In definitiva, il valor medio del momento magnetico lungo \(x\) è:

\[\left\langle \psi(t) \right|{\hat{\mu}}_{x}\left| \psi(t) \right\rangle = \gamma\hslash A\cos\left( \beta - \omega\left( t - t_{0} \right) \right)\]

\subsection{Transizioni di stato}\label{transizioni-di-stato}

Gli autostati dell'operatore di misura \(\hat{A}\) presentano energie costanti, rappresentante dagli autovalori dell'operatore di misura applicato.

Finché il sistema si trova in un autostato stazionario, ovvero il sistema non è perturbato, l'evoluzione temporale avviene in maniera determinista, secondo l'equazione:

\[\left| \psi(t) \right\rangle = \left| \psi\left( t_{0} \right) \right\rangle\exp\left( - j\dfrac{E_{n}}{\hslash}\left( t - t_{0} \right) \right)\]

In seguito a una misura, mediante l'applicazione dell'operatore $\hat{A}$), lo stato del sistema collassa in un autostato di $\hat{A}$, e la probabilità che lo stato $\left| \psi \right\rangle$ sia trovato nell'autostato $\left| \varphi \right\rangle$ è data da $|\langle \varphi | \psi \rangle|^2$.

Il passaggio da uno stato energetico all'altro, ovvero la transizione di stato, può avvenire in modo guidato a seguito di una sollecitazione elettromagnetica di un'opportuna frequenza. L'efficacia di tale transizione è determinata dall'elemento di matrice di transizione, come ad esempio $\langle \varphi | \hat{\mu}_x | \psi \rangle$ per l'interazione magnetica trasversale.

Se il sistema quantico si trova in un campo elettromagnetico oscillante, è possibile sviluppare una soluzione dell'equazione di Schrödinger tempo dipendente:

\[j\hslash\dfrac{\partial\psi}{\partial t} = \hat{H}\psi\]

I metodi approssimata per poter risolvere questa equazione sono abbastanza complessi. Questi metodi sono basati sull'assunzione che la soluzione approssimata al primo ordine, ottenuta considerando un operatore hamiltoniano senza perturbazione \({\hat{H}}_{0}\) a cui si aggiunge un termine del primo ordine dello sviluppo della perturbazione, \({\hat{H}}_{1}(t)\). Questo termine aggiuntivo rappresenta la probabilità di transizione dallo stato a energia \(l\) verso quello a energia \(k\). In simboli è possibile scrivere:

\[H_{kl} = \left\langle k \right|{\hat{H}}_{1}\left| l \right\rangle\]

I termini che esprimono le probabilità di transizione per unità di tempo, al tempo \(t\), da un livello energetico all'altro sono esprimibili secondo la \textbf{meccanica quantistica perturbativa nel tempo} (\textit{time-dependent perturbation theory}):

\[
W_{a \rightarrow b} =\dfrac{1}{\hslash^{2}}\left|\int_{-\infty}^{+\infty}H_{ab}\left(\tau\right)\,\exp\left(
j\dfrac{E_{b} - E_{a}}{\hslash}\,\tau\right)d\tau\right|^{2}
\]

dove:
\begin{itemize}
    \item \( W_{a \rightarrow b} \): probabilità di transizione (proporzionale alla probabilità che il sistema passi dallo stato \( a \) allo stato \( b \));
    \item \( H_{ab}(\tau) = \langle b | H'(\tau) | a \rangle \): elemento di matrice del termine perturbativo tra gli stati \( |a\rangle \) e \( |b\rangle \).
    \item \( E_a, E_b \): energie degli stati iniziale e finale del sistema non perturbato;
    \item \( \hslash \): costante di Planck ridotta, che lega energia e frequenza quantistica (\( E = \hslash \omega \)).
    \item l'esponenziale: fattore di fase oscillante dovuto alla differenza di energia tra i due stati.
\end{itemize}

La probabilità di transizione tra lo stato \(a\) a quello \(b\) dipende dalla trasformata di Fourier dell'hamiltoniana associata alla radiazione oscillante. \({\Delta}E = E_{b} - E_{a}\) è la differenza di energia tra i due stati. Se la perturbazione \({\hat{H}}_{1}\) è costituita da un campo di frequenze, con spettro concentrato intorno alla frequenza:

\[\omega_{ab} = \dfrac{E_{b} - E_{a}}{\hslash}\]

La probabilità \(W_{a \rightarrow b}\) risulta essere molto più alta rispetto al caso in cui lo spettro della perturbazione non è concentrato intorno a questa frequenza ma è disperso.


\begin{figure}[ht]
\centering
\includegraphics[width=\linewidth]{media/4_Quantiatica/image45.pdf}\caption{Spettro concentrato in alto e più disperso in basso}
\end{figure}

La probabilità i transizione inversa, \(W_{b \rightarrow a}\) assume lo stesso valore, a causa della presenza del modulo quadro:

\[
W_{b \rightarrow a} = \mathbf{\dfrac{1}{\hslash^2}}\left| \int_{- \infty}^{+ \infty}{H_{ba}(\tau)\exp\left( j\dfrac{E_{a} - E_{b}}{\hslash}\tau \right)d\tau} \right|^{2}
\]

L'elemento di matrice dovrebbe è $H_{ba} = \langle a | \hat{H}_1 | b \rangle$, ma poiché $|H_{ba}| = |H_{ab}|$, l'uso di $H_{ab}$ è accettabile se si intende $|H_{ab}|^2$. Tuttavia, per rigore, si usa l'elemento di matrice corretto per la transizione $b \to a$)

Se \(E_{a} > E_{b}\), è possibile avere le transizioni in entrambi i versi, ovvero, un sistema può sia passare dal livello energetico inferiore \(E_{b}\) a quello superiore \(E_{a}\) che il viceversa. Al primo passaggio è associato l'assorbimento di un fotone, con frequenza data dalla legge di Plank-Einstein:

\[h\nu = \left| E_{a} - E_{b} \right|\]

Il passaggio inverso avviene mediante l'emissione, detta stimolata, di un fotone con frequenza uguale a quella del fotone assorbito.

\begin{figure}[ht]
\centering
\includegraphics[width=6.69306in,height=3.00486in,alt={P2265\#yIS1}]{media/4_Quantiatica/image46.pdf}\caption{Transizioni tra livelli energetici mediante assorbimento ed emissione di fotoni}
\end{figure}

\begin{center}
\vfill
    \chapter{Meccanica statistica}
    \label{blx:refsection\therefsection}
\vfill

\minitoc
\newpage
\end{center}
\justify


\section{Cenni di meccanica statistica}
\label{cenni-di-meccanica-statistica}

La Meccanica Statistica è il fondamentale ponte teorico che collega il comportamento delle singole particelle a livello microscopico con le proprietà misurabili e osservabili dei sistemi a livello macroscopico (termodinamico). Essa utilizza i principi della probabilità e della statistica per trattare sistemi composti da un numero enorme (\(N \approx 10^{23}\)) di costituenti (atomi, molecole o particelle quantistiche), superando l'impossibilità pratica di applicare le leggi della meccanica classica o quantistica a ogni singola entità.

Il cuore della meccanica statistica risiede nell'idea di \textbf{ensemble statistico} e nel principio che il valore di equilibrio di una grandezza macroscopica \(M\) è dato dalla media sui microstati permessi (o dalla media temporale, secondo l'ipotesi ergodica). Formalmente:

\[
\left\langle M \right\rangle = \sum_{\text{microstati } i} M_i P_i
\]

dove \(P_i\) è la probabilità che il sistema si trovi nel microstato \(i\). Le probabilità \(P_i\) sono ricavate dalla \textbf{funzione di partizione} \(Z\), che funge da generatrice di tutte le grandezze termodinamiche.

La meccanica statistica consente di dedurre tutte le grandezze termodinamiche (macroscopiche) partendo dalle proprietà microscopiche (livelli energetici e Hamiltoniana del sistema). Le grandezze fondamentali che essa riesce a esprimere in termini microscopici includono:

\begin{table}[h!]
\centering
\caption{Grandezze macroscopiche derivate dalla meccanica statistica}
\label{tab:mecc_stat}
\scriptsize
\begin{tabular}{|l|c|c|>{\centering\arraybackslash}m{4.5cm}|}
\hline
\textbf{Grandezza} & \textbf{Simbolo} & \textbf{Relazione Statistica} & \textbf{Descrizione Microscopica} \\
\hline \hline
\textbf{Energia Interna} & \(U\) (o \(E\)) & \(\displaystyle U = \langle H \rangle = -\left(\frac{\partial \ln Z}{\partial \beta}\right)_{V, N}\) & Media dell'energia totale su tutti i microstati. \\
\hline
\textbf{Pressione} & \(P\) & \(\displaystyle P = \frac{1}{\beta}\left(\frac{\partial \ln Z}{\partial V}\right)_{T, N}\) & Legata all'impulso trasferito dagli urti delle particelle contro le pareti. \\
\hline
\textbf{Temperatura} & \(T\) & \(\displaystyle T = \frac{1}{k_B \beta}\) & Misura dell'agitazione termica media delle particelle. \\
\hline
\textbf{Entropia} & \(S\) & \(S = k_B \ln W\) & Logaritmo del numero \(W\) di microstati compatibili (disordine molecolare). \\
\hline
\textbf{Capacità Termica} & \(C_V\) & \(\displaystyle C_V = \left(\frac{\partial U}{\partial T}\right)_{V}\) & Misura della variazione dell'energia interna (media) con la temperatura. \\
\hline
\end{tabular}
\end{table}

In sintesi, la meccanica statistica traduce i parametri microscopici (posizioni, quantità di moto e interazioni delle \(N\) particelle) in grandezze macroscopiche, fornendo una giustificazione teorica per le leggi fenomenologiche della Termodinamica.

\section{Possibili stati di un sistema in base allo spin}\label{possibili-stati-di-un-sistema-in-base-allo-spin}

Si considerano due sistemi separati da un setto. Il sistema a destra possiede inizialmente due particelle, mentre quello a sinistra una sola. Si suppone di rimuovere il setto in modo da far interagire tra loro le tre particelle. Si osservano fenomeni diffusivi.

\begin{figure}[ht]
\centering
\includegraphics[width=4.88764in,height=2.22619in]{media/5_MecStatistica/image47.pdf}
\caption{Sistemi separati da un setto}
\end{figure}

Una particella può trovarsi in due stati possibili: a destra o a sinistra. Le possibili configurazioni del sistema complessivo, privato del setto, sono \(2^{N}\), dove \(2\) è il numero degli stati e \(N\) il numero delle particelle. Nel caso in esame si hanno i seguenti casi:


\begin{longtable}[]{@{}ccc@{}}
\caption{Possibili configurazioni del sistema con tre particelle}\tabularnewline
\toprule
Particella \(1\) & Particella \(2\) & Particella \(3\) \\
\midrule
\endfirsthead
\toprule
Particella \(1\) & Particella \(2\) & Particella \(3\) \\
\midrule
\endhead
S & S & S \\
S & S & D \\
S & D & S \\
S & D & D \\
D & S & S \\
D & S & D \\
D & D & S \\
D & D & D \\
\bottomrule
\end{longtable}

Di tutte le possibili configurazioni, una prevede che tutte le particelle si trovino a sinistra, una che tutte siano a destra, mentre tre prevedono due particelle a sinistra e una a destra. Analogamente, vi sono tre casi in cui due particelle sono a destra e una a sinistra.

Tutte le configurazioni elencate sono possibili, tuttavia quelle intermedie hanno maggiore probabilità di verificarsi. Infatti, la probabilità che tutte siano a destra o tutte a sinistra è:

\[
\frac{1}{8} = 0.125 = 12.5\%
\]

La probabilità di avere due particelle a sinistra e una a destra, o viceversa, è:

\[
\frac{3}{8} = 0.375 = 37.5\%
\]

\begin{figure}[ht]
\centering
\includegraphics[width=4.08333in,height=2.63031in,alt={P2317\#yIS1}]{media/5_MecStatistica/image48.pdf}\caption{Probabilità delle varie configurazioni}
\end{figure}

Si considera un sistema di \(N\) particelle. Per la sua descrizione sono presenti tre gradi di libertà per ciascuna particella, uno per ogni coordinata cartesiana. Se le particelle si muovono secondo le leggi della meccanica classica, la lagrangiana del sistema totale si scrive come:

\[
L = \frac{1}{2}\sum_{i = 1}^{N}{m_{i}v_{i}^{2}} - U\left( x_{1},y_{1},z_{1},\ldots,x_{N},y_{N},z_{N} \right)
\]

dove \(U\) rappresenta l'energia di interazione tra le particelle del sistema. Se si considera un grammo di una sostanza pura, esso contiene un numero di Avogadro di particelle:

\[
N_{A} = 6.022 \cdot 10^{23}\ \text{mol}^{-1}
\]

Risulta, dunque, molto complesso, se non impossibile, scrivere le equazioni del moto e risolverle tutte. A tale scopo è necessario conoscere le condizioni iniziali delle particelle.

Per studiare tali sistemi, si ricorre a una descrizione probabilistica, come nel caso delle tre particelle. Con un grammo di sostanza, vi sono \(2^{N_{A}}\) combinazioni possibili, ognuna con la propria probabilità di verificarsi.

Si considera ora un sistema composto da \(N\) spin non interagenti, ciascuno dei quali può assumere solamente due stati possibili: \(\left| + \right\rangle\) e \(\left| - \right\rangle\).

Si suppone che ogni spin occupi una posizione fissa nello spazio. Lo stato complessivo del sistema macroscopico è descritto da una sequenza di stati \(\left| + \right\rangle\) e \(\left| - \right\rangle\), che identificano lo stato di ciascuno spin.

Il sistema può assumere \(2^{N}\) configurazioni diverse. Si vuole determinare il numero massimo di combinazioni con \(k\) spin nello stato \(\left| + \right\rangle\). Il numero delle combinazioni con \(k\) spin nello stato \(\left| + \right\rangle\) coincide con il coefficiente binomiale:

\[
N_{k} = \binom{N}{k} = \frac{N!}{k!(N - k)!}
\]

La probabilità di avere \(k\) spin nello stato \(\left| + \right\rangle\) è ottenuta dividendo questo risultato per il numero totale delle configurazioni ammissibili del sistema \(2^{N}\):

\[
P(k\text{ spin nello stato } \left| + \right\rangle) = \frac{1}{2^{N}} \binom{N}{k} = \frac{1}{2^{k}2^{N - k}}\binom{N}{k}
\]


Il numero degli spin nello stato \(\left| + \right\rangle\) equivale a una distribuzione binomiale con probabilità di \(1/2\). Per cui la media è:

\[
m = \frac{N}{2}
\]

La varianza della distribuzione binomiale è:

\[
\sigma^{2} = Npq = N \cdot \frac{1}{2} \cdot \frac{1}{2} = \frac{N}{4}
\]

Dove \(p\) è la probabilità del caso voluto e \(q = 1 - p\) la probabilità del caso sfavorevole. 

All'aumentare del numero di particelle, il numero delle configurazioni intermedie aumenta, così come la loro probabilità di comparsa. Inoltre, la distribuzione binomiale tende a una distribuzione gaussiana, in cui il picco rappresenta la configurazione più probabile. Le configurazioni estreme, come tutti gli spin nello stato \(\left| + \right\rangle\) o \(\left| - \right\rangle\), hanno probabilità praticamente nulla.

Le fluttuazioni della relative a una determinata configurazione sono date da \(\sigma/m\), con \(m\) media. Sostituendo la relazione per media e varianza si ha:

\[
\frac{\sigma}{m} = \frac{\sqrt{N}/2}{N/2} = \frac{\sqrt{N}}{2}\frac{2}{N} = \frac{1}{\sqrt{N}}
\]

Se il numero delle particelle è dell'ordine del numero di Avogadro, la fluttuazione è:

\[
\frac{\sigma}{m} \simeq \frac{1}{\sqrt{10^{23}}} \simeq 10^{-11} \ll 1
\]

La distribuzione delle configurazioni ha una varianza molto ridotta, quindi la curva gaussiana è molto concentrata intorno al valore medio.


\begin{figure}[ht]
\centering
\includegraphics[width=3.66791in,height=2.66284in,alt={P2343\#yIS1}]{media/5_MecStatistica/image49.pdf}\caption{Variazione della distribuzione delle possibili configurazioni al variare del numero di spin}
\end{figure}

Se il sistema di \(N\) spin potesse essere osservato in tutti gli istanti di tempo, si vedrebbe quasi sempre una configurazione in cui \(k = N/2\), ovvero la metà degli spin è nello stato \(\left| + \right\rangle\) e l'altra metà nello stato \(\left| - \right\rangle\). In definitiva, è molto più probabile trovare il sistema in una configurazione con probabilità \(p = 1/2\), piuttosto che in una configurazione lontana da quella maggiormente favorevole.

\section{Approssimazione di Stirling}\label{approssimazione-di-Stirling}

Si considera un numero intero \(N\), sufficientemente grande, e si valuta la quantità:

\[
\ln{N!} = \ln{\prod_{n = 1}^{N}n}
\]

Per le proprietà dei logaritmi, si ha:

\[
\ln{\prod_{n = 1}^{N}n} = \sum_{n = 1}^{N}{\ln n}
\]

Per \(N \gg 1\), si può approssimare la somma con un integrale:

\[
\sum_{n = 1}^{N}{\ln n} \simeq \int_{1}^{N}{\ln x\,dx}
\]

Integrando per parti, si ha;


\[
\int_{1}^{N}{\ln xdx} = \left[ x\ln x \right]_{1}^{N} - \int_{1}^{N}{dx} = N\ln N - N + 1
\]

Per \(N \gg 1\), il termine costante può essere trascurato:

\[
\ln{N!} \simeq N\ln N - N
\]

Maggiore è il numero delle particelle, più l'approssimazione di Stirling risulta accurata. L'uguaglianza diventa esattamente valida nel limite \(N \rightarrow \infty\).

\subsection{Probabilità con approssimazione di Stirling}\label{probabilituxe0-con-approssimazione-di-Stirling}

L'approssimazione di Stirling può essere utilizzata per stimare la probabilità che un sistema composto da \(N\) particelle assuma una determinata configurazione, quando \(N \gg 1\) e \(k\) è dell'ordine di \(N/2\). Il coefficiente binomiale è:

\[
P\left( k\ \text{spin su}\ N\ \text{nello stato}\ \left| + \right\rangle \right) = \frac{1}{2^{N}} \binom{N}{k} = \frac{1}{2^{N}} \frac{N!}{k!(N - k)!}
\]

Applicando il logaritmo e le sue proprietà:

\[
\ln P = \ln\left( \frac{1}{2^{N}}\frac{N!}{k!(N - k)!} \right) = \ln N! - \ln k! - \ln(N - k)! - N\ln 2
\]

Si applica l'approssimazione di Stirling (\(\ln M! \simeq M\ln M - M\)):

\[
\ln P \simeq \underbrace{(N\ln N - N)}_{\ln N!} - \underbrace{(k\ln k - k)}_{\ln k!} - \underbrace{((N - k)\ln(N - k) - (N - k))}_{\ln(N-k)!} - N\ln 2
\]

Riordinando i termini è possibile scrivere:

\[
\ln{P} \simeq N\ln{N} - N -k\ln{k} + k - \left(N-k\right)\ln{\left(N-k\right)} + N - k-N\ln{2}
\]

I termini lineari in $N$ e $k$ si cancellano, quindi:

\[
\ln P \simeq N\ln N - \big[k\ln k + (N-k)\ln(N-k)\big] - N\ln 2.
\]

Si applicano le proprietà dei logaritmi ai termini $N\ln N - N\ln 2$:

\[
\ln P \simeq N\ln\!\frac{N}{2} - \left[k\ln k + (N-k)\ln(N-k)\right]
\]

Si definisce lo scarto $k$ dalla media $\mu = N/2$ di una quantità $\Delta s \ll N/2$:

\[
k = \frac{N}{2} + \Delta s \quad \text{e} \quad N - k = \frac{N}{2} - \Delta s
\]

Si può scrivere \(N - k\) come:

\[
N - k = N - \frac{N}{2} - \Delta s = \frac{N}{2} - \Delta s = \frac{N}{2}\left( 1 - \frac{\Delta s}{\frac{N}{2}} \right)
\]

Si applica il logaritmo a entrambi i membri:

\[
\ln(N - k) = \ln\left( \frac{N}{2}\left( 1 - \frac{{\Delta}s}{\frac{N}{2}} \right) \right) = \ln\frac{N}{2} + \ln\left( 1 - \frac{{\Delta}s}{\frac{N}{2}} \right)
\]

Siccome  \(\Delta s \ll N/2\), è possibile sviluppare in serie di Taylor il secondo logaritmo al secondo membro:

\[
\ln\left( 1 - \frac{2\Delta s}{N} \right) \simeq - \frac{2\Delta s}{N} - \frac{1}{2}\left(\frac{2\Delta s}{N}\right)^2
\]

Per cui si ottiene:

\[
\ln(N - k) = \ln\left(\frac{N}{2} - \Delta s\right) = \ln\frac{N}{2} + \ln\left(1 - \frac{2\Delta s}{N}\right) \simeq \ln\frac{N}{2} - \frac{2\Delta s}{N} - \frac{1}{2}\left(\frac{2\Delta s}{N}\right)^2
\]

Riscrivendo l'equazione si ottiene:

\[
\ln (N - k) \simeq \ln\frac{N}{2} - \frac{2\Delta s}{N} - \frac{2\Delta s^{2}}{N^{2}}
\]

Analogamente, si considera \(\ln\left( k \right)\) e si applica la relazione \(k=N/2+\Delta s\). Per le proprietà dei logaritmi si ha:

\[
\ln k = \ln\left(\frac{N}{2} + \Delta s\right) = \ln\frac{N}{2} + \ln\left(1 + \frac{2\Delta s}{N}\right) \simeq \ln\frac{N}{2} + \frac{2\Delta s}{N} - \frac{1}{2}\left(\frac{2\Delta s}{N}\right)^2
\]

Scrivendo diversamente, si ottiene:

\[
\ln k \simeq \ln\frac{N}{2} + \frac{2\Delta s}{N} - \frac{2\Delta s^{2}}{N^{2}}
\]

Si considerano, uno alla volta, i gruppi presenti dell'equazione per \(\ln P\):


\[
\ln P \simeq - \left[ k\ln k + (N-k)\ln(N-k) \right] + N\ln\frac{N}{2}
\]

Si parte dai termini contenenti il parametro \(k\), ovvero \(k\ln{k}\):

\[
\begin{aligned}
k\ln k 
&= \left(\frac{N}{2}+\Delta s\right)
   \left[\ln\!\frac{N}{2} + \frac{2\Delta s}{N} - \frac{1}{2}\!\left(\frac{2\Delta s}{N}\right)^2\right] \\
&= \frac{N}{2}\ln\!\frac{N}{2} + \Delta s\ln\!\frac{N}{2}
   + \frac{N}{2}\cdot\frac{2\Delta s}{N} + \Delta s\cdot\frac{2\Delta s}{N}
   - \frac{N}{2}\cdot\frac{1}{2}\!\left(\frac{4\Delta s^2}{N^2}\right)
\end{aligned}
\]

Eseguendo le somme e semplificando si ottiene si ottiene:

\[
k\ln k \simeq 
\frac{N}{2}\ln\!\frac{N}{2} + \Delta s\ln\!\frac{N}{2} + \Delta s + \frac{\Delta s^2}{N}
\]

Si considera ora la restante parte:

\[
\begin{aligned}
(N - k)\ln(N - k)
&= \left(\frac{N}{2} - \Delta s\right)
\left[
\ln\!\frac{N}{2}
- \frac{2\Delta s}{N}
- \frac{1}{2}\!\left(\frac{2\Delta s}{N}\right)^{2}
\right] \\
& \simeq \frac{N}{2}\ln\!\frac{N}{2}
- \Delta s\ln\!\frac{N}{2}
- \frac{N}{2}\cdot\frac{2\Delta s}{N}
+ \Delta s\cdot\frac{2\Delta s}{N}
- \frac{N}{2}\cdot\frac{1}{2}\!\left(\frac{4\Delta s^{2}}{N^{2}}\right)
\end{aligned}
\]

Svolgendo i seguenti passaggi:
\[
\frac{N}{2}\cdot\frac{2\Delta s}{N} = \Delta s,
\qquad
\frac{N}{2}\cdot\frac{1}{2}\!\left(\frac{4\Delta s^{2}}{N^{2}}\right) = \frac{\Delta s^{2}}{N}.
\]

si ottiene:

\[
\begin{aligned}
(N - k)\ln(N - k)\simeq \frac{N}{2}\ln\!\frac{N}{2}
- \Delta s\ln\!\frac{N}{2}
- \Delta s
+ \frac{\Delta s^{2}}{N}
\end{aligned}
\]

Si sostituiscono questi risultati nell'espressione per \(\ln P\):

\[
\ln P \simeq N\ln\!\frac{N}{2} - \left[k\ln k + (N-k)\ln(N-k)\right]
\]

Si sostituiscono le relative espressioni in \(k\ln k + (N-k)\ln(N-k)\):

\[
\begin{aligned}
k\ln k + (N-k)\ln(N-k)
&\simeq \frac{N}{2}\ln\!\frac{N}{2}
- \Delta s\ln\!\frac{N}{2}
- \Delta s
+ \frac{\Delta s^{2}}{N} + \\
& + N\ln\!\frac{N}{2} + \frac{2\Delta s^2}{N} + \frac{N}{2}\ln\!\frac{N}{2} + \Delta s\ln\!\frac{N}{2} + \Delta s + \frac{\Delta s^2}{N}
\end{aligned}
\]

Sommando i termini analoghi si ha:

\[
k\ln k + (N-k)\ln(N-k)\simeq N\ln\!\frac{N}{2} + \frac{2\Delta s^2}{N}.
\]

Si sostituisce tale risultato nell'espressione per la probabilità:

\[
\ln P \simeq 
N\ln\!\frac{N}{2} - 
\left[N\ln\!\frac{N}{2} + \frac{2\Delta s^2}{N}\right]
= -\frac{2\Delta s^2}{N}
\]

Dalla definizione di \({\Delta}s\) si ha che \(k = N/2 + {\Delta}s \Leftrightarrow {\Delta}s = k - N/2\). È possibile scrivere:

\[
\ln P \simeq - \frac{\left( k - \frac{N}{2} \right)^{2}}{\frac{N}{2}}
\]

Questa relazione può essere arrangiata in modo da evidenziare la varianza della distribuzione binomiale \(\sigma = N/4\), moltiplicando e dividendo per \(2\) il secondo membro:

\[
\ln P \simeq - \frac{1}{2}\frac{\left( k - \frac{N}{2} \right)^{2}}{\frac{N}{4}}
\]

Si applica l'esponenziale, in modo da ricavare la probabilità \(P\):

\[
P\left( k\ \text{spin su}\ N\ \text{nello stato}\ \left| + \right\rangle \right) \simeq \exp\left( - \frac{1}{2} \frac{(k - \frac{N}{2})^2}{\sigma^2} \right)
\]

La distribuzione ottenuta è di tipo gaussiano, con media \(\mu = N/2\) e varianza  \(\sigma^{2} = N/4\). Affinché la relazione sia effettivamente una gaussiana, è necessario introdurre un termine di normalizzazione \(A\);

\[
P\left( k\ \text{spin su}\ N\ \text{nello stato}\ \left| + \right\rangle \right) \simeq \exp\left( - \frac{1}{2} \frac{(k - \frac{N}{2})^2}{\sigma^2} \right)
\]

Nei passaggi precedenti, la costante di normalizzazione non è comparsa a causa delle approssimazioni introdotte da Stirling e dallo sviluppo in serie di Taylor. La costante di normalizzazione è:
\[
A = \frac{1}{\sqrt{2\pi \sigma^2}} = \frac{1}{\sqrt{2\pi (N/4)}} = \frac{2}{\sqrt{2\pi N}}
\]

Quindi la distribuzione che \(k\) spin siano nello stato up è:

\[
P\left( k\ \text{spin su}\ N\ \text{nello stato}\ \left| + \right\rangle \right) \simeq \frac{2}{\sqrt{2\pi N}} \exp\left( - \frac{1}{2} \frac{(k - \frac{N}{2})^2}{\sigma^2} \right)
\]

\section{Stati ammissibili}\label{stati-ammissibili}

Si considera un sistema composto da \(N\) particelle immerse in un potenziale. Un esempio di tale configurazione è un sistema di \(N\) spin immersi in un campo magnetico \(\vec{B}\). In questa situazione, gli spin sono soggetti a un'energia potenziale:

\[
U = - \vec{\mu} \cdot \vec{B}
\]

Se il campo magnetico ha solo componente lungo l'asse \(z\), ovvero \(\vec{B} = B_{0} \hat{i}_{z}\), l'energia potenziale diventa:

\[
U = \mp \mu_{z} B_{0}
\]

dove \(\mu_{z}\) è la proiezione del momento magnetico intrinseco sull'asse \(z\).

\begin{figure}[ht]
\centering
\includegraphics[width=1.7273in,height=1.75in,alt={P2413\#yIS1}]{media/5_MecStatistica/image50.pdf}\caption{Spin immerso in un campo magnetico}
\end{figure}

Secondo la meccanica quantistica, le particelle possono assumere solo due livelli energetici, \(\pm \mu_{z} B_{0}\), corrispondenti agli autovalori dell'operatore hamiltoniano in presenza di un campo magnetico:

\[
\hat{H} = - \gamma B_{0} \hat{S}_{z}
\]

L'operatore \(\hat{S}_{z}\) presenta due autovalori: \(\pm \hslash / 2\), che corrispondono a due possibili momenti magnetici. Il momento magnetico parallelo al campo ha energia minima \(- \gamma \hslash B_0 / 2\), mentre quello antiparallelo ha energia massima \(+ \gamma \hslash B_0 / 2\). I due stati sono indicati con \(\left| + \right\rangle\) e \(\left| - \right\rangle\).

Supponendo che gli spin siano non interagenti, l'energia totale del sistema è data dalla somma delle energie di ciascuna particella:

\[
U_{\text{tot}} = \sum_{n = 1}^{N} \left[ - \mu_{z} B_{0} s(n) \right] = \mathbf{- \mu_{z} B_{0} \sum_{n = 1}^{N} s(n)}
\]

dove \(s(n) = 1\) se lo spin si trova nello stato \(\left| + \right\rangle\), invece, \(s(n) = - 1\) se lo spin si trova nello stato \(\left| - \right\rangle\).

Se il sistema è isolato, non può scambiare energia con l'ambiente, quindi la sua energia totale rimane costante nel tempo. Il sistema può trovarsi in qualsiasi configurazione con \(k\) spin nello stato \(\left| + \right\rangle\), purché l'energia totale sia invariata. La somma si scrive come:

\[
\sum s(n) = k - (N-k) = 2k - N
\]

Nel caso in esame l'energia può essere espressa come multiplo della differenza tra il numero di spin nello stato \(\left| + \right\rangle\) e quelli nello stato \(\left| - \right\rangle\):

\[
{U_{\text{tot}} = - \mu_{z} B_{0} \sum_{n = 1}^{N} s(n)} \Rightarrow U_{\text{tot} = - \left[ k - (N - k) \right] \mu_{z} B_{0} = - (2k - N) \mu_{z} B_{0}}
\]

Il numero delle possibili configurazioni che il sistema può assumere con \(k\) spin nello stato \(\left| + \right\rangle\) è dato da:

\[
\binom{N}{k} = \frac{N!}{k!(N - k)!}
\]

Nell'ipotesi di spin non interagenti, il sistema è statico e permane nello stato iniziale. Tuttavia, nella realtà gli spin interagiscono tra loro, seppur per tempi brevi, scambiandosi energia. Così, uno spin con energia maggiore può trasferire energia a uno con energia inferiore, causando un cambio di stato per entrambi.

Si suppone che gli spin siano debolmente interagenti, ovvero che scambino una quantità di energia trascurabile rispetto all'energia totale del sistema. A causa di queste interazioni, il sistema non permane nello stato iniziale, ma evolve attraverso configurazioni diverse, purché l'energia totale rimanga costante. Le configurazioni che condividono la stessa energia totale sono dette \textbf{stati ammissibili}. In pratica, il sistema transita continuamente tra stati ammissibili, ciascuno dei quali ha uguale probabilità. Tutti gli stati ammissibili corrispondenti allo stesso stato macroscopico osservato sono equiprobabili; pertanto, osservando il sistema, esso può trovarsi in una qualsiasi di queste configurazioni.

\section{Entropia}\label{entropia}

Si considerano due sistemi \(\mathbb{S}_{1}\) e \(\mathbb{S}_{2}\), composti rispettivamente da \(N_{1}\) e \(N_{2}\) particelle e con energia \(U_{1}\) e \(U_{2}\). I due sistemi sono messi in contatto così che possano scambiare energia ma non materia; ovvero, i due sistemi sono messi in contatto termico ma non diffusivo.

\begin{figure}[ht]
\centering
\includegraphics[width=2.37891in,height=1.5625in,alt={P2432\#yIS1}]{media/5_MecStatistica/image51.pdf}\caption{Sistemi posti in contatto}
\end{figure}

Aver posto in contatto i due sistemi, si ottiene un sistema \(\mathbb{S}\) con \(N = N_{1} + N_{2}\) particelle e con energia totale \(U = U_{1} + U_{2}\). Si suppone che il sistema complessivo \(S\) sia isolato dall'ambiente esterno, così da non poter scambiare energia con l'ambiente.

Si considera il sistema \(\mathbb{S}_{1}\), la cui energia dipende dallo stato delle sue particelle. Sia \(s_{1}\) la variabile che enumera le configurazioni possibili per il sistema \(\mathbb{S}_{1}\), come, ad esempio, il numero degli spin nello stato \(\left| + \right\rangle\). L'energia del sistema considerato dipende dalla configurazione del sistema:

\[U_{1} = U_{1}\left( s_{1} \right)\]

La variabile \(s_{1}\) non identifica un'unica configurazione ma un insieme di configurazioni, aventi tutte la stessa energia totale \(U_{1}\left( s_{1} \right)\).

La numerosità degli stati ammissibili \(g\) è una funzione del numero di particelle e dell'energia del sistema:

\[g_{1}\left( U_{1}\left( s_{1} \right),N_{1} \right)\]

Analogo discorso vale per il sistema \(\mathbb{S}_{2}\), la cui numerosità degli stati ammissibili è:

\[g_{2}\left( U_{s}\left( s_{2} \right),N_{2} \right)\]

Ponendo in contatto i due sistemi l'energia totale, costante nel tempo, è data da:

\[U = U_{1}\left( s_{1} \right) + U_{2}\left( s_{2} \right)\]

È possibile descrivere l'energia del secondo sistema in funzione dell'energia totale e dell'energia del primo sistema:

\[U_{2}\left( s_{2} \right) = U - U_{1}\left( s_{1} \right)\]

La numerosità degli stati nel sistema \(\mathbb{S}_{2}\) può essere scritta come:

\[g_{2}\left( U - U_{1}\left( s_{1} \right),N_{2} \right)\]

Da questo ragionamento si evince che, aumentando l'energia del sistema \(U_{1}\), aumenta il numero delle sue configurazioni ammissibili; mentre il sistema \(\mathbb{S}_{2}\) riduce la sua energia, dunque, la numerosità dei suoi stati ammissibili si riduce.

Si fissa il valore di \(s_{1}\) del sistema \(\mathbb{S}_{1}\), in questo modo anche l'energia \(U_{1}\) è fissata. Il numero di configurazioni possibili per il sistema complessivo \(\mathbb{S =}\mathbb{S}_{1} + \mathbb{S}_{2}\) è dato dal prodotto delle numerosità:

\[g(U,N) = g_{1}\left( U_{1}\left( s_{1} \right),N_{1} \right)g_{2}\left( U - U_{1}\left( s_{1} \right),N_{2} \right)\]

Dove \(N = N_{1} + N_{2}\). Per ogni stato ammissibile del primo del primo sistema, il secondo può trovarsi in uno qualunque dei suoi stati ammissibili, il cui numero dipende dall'energia del sistema \(\mathbb{S}_{1}\). In altre parole, al variare dell'enumerazione \(s_{1}\) del primo sistema, varia anche la numerosità degli stati ammissibili del sistema totale. Esiste, di conseguenza, un massimo della funzione \(g(U,N)\), ottenuto ponendo uguale a \(0\) la derivata di \(g\) rispetto all'energia:

\[\frac{\partial g}{\partial U} = \frac{\partial g}{\partial U_{1}} = \frac{\partial}{\partial U_{1}}\left[ g_{1}\left( U_{1}\left( s_{1} \right),N_{1} \right)g_{2}\left( U - U_{1}\left( s_{1} \right),N_{2} \right) \right)\]

Per le proprietà delle derivate si ha:

\[\frac{\partial g}{\partial U_{1}} = \frac{\partial g_{1}}{\partial U_{1}}g_{2} - g_{1}\frac{\partial g_{2}}{\partial U_{1}}\]

Dove il meno è legato alla dipendenza di \(g_{2}\) da \(- U_{1}\). In condizione di massimo, deve risultare:

\[\frac{\partial g}{\partial U} = 0\]

Ovvero:

\[\frac{\partial g_{1}}{\partial U_{1}}g_{2} - g_{1}\frac{\partial g_{2}}{\partial U_{1}} = 0 \Leftrightarrow \frac{\partial g_{1}}{\partial U_{1}}g_{2} = g_{1}\frac{\partial g_{2}}{\partial U_{1}}\]

Si divide per \(g_{1}g_{2}\):

\[\frac{1}{g_{1}}\frac{\partial g_{1}}{\partial U_{1}} = \frac{1}{g_{2}}\frac{\partial g_{2}}{\partial U_{1}}\]

È noto che:

\[\frac{\partial}{\partial U_{1}}\left( \ln g_{1} \right) = \frac{1}{g_{1}}\frac{\partial g_{1}}{\partial U_{1}}\]

Dunque, è possibile scrivere:

\[\frac{\partial}{\partial U_{1}}\left( \ln g_{1} \right) = \frac{\partial}{\partial U_{1}}\left( \ln g_{2} \right)\]

Si definisce entropia di un sistema \(\sigma\), quantità adimensionale, come il logaritmo della molteplicità degli stati aventi tutti la stessa energia:

\[S = \sigma = \ln g\]

Tramite il concetto di entropia è possibile riscrivere la relazione precedente come:

\[\left. \ \frac{\partial\sigma}{\partial U} \right|_{1} = \left. \ \frac{\partial\sigma}{\partial U} \right|_{2}\]

L'entropia nel sistema \(\mathbb{S}_{1}\) è uguale a quella del sistema \(\mathbb{S}_{2}\) una volta raggiunto l'equilibrio termodinamico, rappresentato dalla configurazione più probabile, coincidente con il massimo della numerosità.

La legge zero della termodinamica classica afferma che, se due corpi sono posti a contatto, dopo un certo tempo, raggiungono la stessa temperatura. Con il ragionamento effettuato si è visto che due sistemi, messi in contatto termico ma non diffusivo, raggiungono la stessa variazione di entropia, rispetto l'energia \(U_{1}\). Da questo risultato si definisce la temperatura come:

\[\frac{\partial\sigma}{\partial U} = \frac{1}{k_{B}T}\]

Dove \(k_{B}\) è la costante di Boltzmann, utile affinché l'equazione sia valida dal punto di vista dimensionale:

\[k_{B} = 1.38 \cdot 10^{- 23}JK^{- 1}\]

L'entropia è legata al disordine del sistema: più un sistema ha un alto numero di configurazioni ammissibili e più è disordinato, poiché aumenta la sua numerosità.

\section{Distribuzione di Boltzmann}\label{distribuzione-di-boltzmann}

Si suppone che un sistema \(\mathbb{S}_{1}\) sia molto più piccolo del sistema \(\mathbb{S}_{2}\). I due sistemi sono posti in contatto termico tra loro. Il sistema \(\mathbb{S}_{2}\) è detto serbatoio termico poiché, ponendolo in contatto col piccolo sistema, la sua energia \(U\) varia di una quantità trascurabile, idealmente nulla.

\begin{figure}[ht]
\centering
\includegraphics[width=4.23666in,height=2.7381in,alt={P2477\#yIS1}]{media/5_MecStatistica/image52.pdf}\caption{Sistema molto più piccolo dell'altro messi in contatto termico}
\end{figure}

Si vuole collegare le variazioni di entropia del piccolo sistema con quelle del grande sistema, raggiunto l'equilibrio termico.

L'entropia è collegata alla probabilità di trovare il sistema \(\mathbb{S}_{1}\) in uno stato specifico \(s_{1}\), a cui corrisponde un'energia \(\varepsilon_{1}^{\alpha} \ll U\), mediante la numerosità degli stati. Per valutare questa probabilità si scrive la numerosità del sistema totale:

\[g\left( \varepsilon_{1}^{\alpha},N \right) = g_{1}\left( \varepsilon_{1}^{\alpha},N_{1} \right)g_{2}\left( U - \varepsilon_{1}^{\alpha},N_{2} \right)\]

Fissata l'energia e la configurazione del piccolo sistema (come lo stato di uno spin immerso nell'ambiente) risulta che:

\[g_{1}\left( \varepsilon_{1}^{\alpha},N_{1} \right) = 1\]

Le possibili configurazioni ammissibili dipendono solamente dal sistema \(\mathbb{S}_{2}\), poiché, appunto, la configurazione di \(\mathbb{S}_{1}\) è fissata:

\[g\left( \varepsilon_{1}^{\alpha},N \right) = g_{2}\left( U - \varepsilon_{1}^{\alpha},N_{2} \right)\]

Si considera un secondo valore di energia del sistema \(\mathbb{S}_{1}\), \(\varepsilon_{1}^{\beta}\). Fissato lo stato del sistema \(\mathbb{S}_{1}\), la numerosità degli stati ammissibili dal sistema complessivo è data da:

\[g\left( \varepsilon_{1}^{\beta},N \right) = g_{2}\left( U - \varepsilon_{1}^{\beta},N_{2} \right)\]

La numerosità è legata alla probabilità mediante un fattore di normalizzazione, dato dal numero totale delle configurazioni possibili del sistema, indipendente dall'energia. Calcolando il rapporto tra le due numerosità degli stati del sistema globale, si ottiene il rapporto tra la probabilità che il sistema \(\mathbb{S}_{2}\) sia nello stato \(g_{2}\left( U - \varepsilon_{1}^{\alpha} \right)\) e la probabilità che sia nello stato \(g_{2}\left( U - \varepsilon_{1}^{\beta} \right)\). In altre parole, il fattore di normalizzazione si elide:

\[\frac{P\left( \varepsilon_{1}^{\alpha} \right)}{P\left( \varepsilon_{1}^{\beta} \right)} = \frac{g_{2}\left( U - \varepsilon_{1}^{\alpha},N_{2} \right)}{g_{2}\left( U - \varepsilon_{1}^{\beta},N_{2} \right)}\]

L'entropia è legata alla numerosità degli stati mediante logaritmo:

\[\sigma = \ln g = \ln g_{2}\]

Applicando l'esponenziale si ha:

\[g_{2} = e^{\sigma}\]

Sostituendo questo risultato nel rapporto tra le probabilità si ottiene:

\[\frac{P\left( \varepsilon_{1}^{\alpha} \right)}{P\left( \varepsilon_{1}^{\beta} \right)} = \frac{g_{2}\left( U - \varepsilon_{1}^{\alpha},N_{2} \right)}{g_{2}\left( U - \varepsilon_{1}^{\beta},N_{2} \right)} = \frac{\exp\left[ \sigma\left( U - \varepsilon_{1}^{\alpha},N_{2} \right) \right)}{\exp\left[ \sigma\left( U - \varepsilon_{1}^{\beta},N_{2} \right) \right)}\]

Siccome \(\varepsilon_{1}^{\alpha},\varepsilon_{1}^{\beta} \ll U\), è possibile sviluppare in serie di Taylor, arrestato al primo ordine, la funzione entropia:

\[\sigma\left( U - \varepsilon_{1}^{i},N_{2} \right) \simeq \sigma(U) - \varepsilon_{1}^{i}\frac{\partial\sigma}{\partial U},\ \ i = \alpha,\beta\]

Per definizione di temperatura:

\[\frac{\partial\sigma}{\partial U} = \frac{1}{k_{B}T}\]

Lo sviluppo si scrive come:

\[\sigma\left( U - \varepsilon_{1}^{i},N_{2} \right) \simeq \sigma(U) - \varepsilon_{1}^{i}\frac{1}{k_{B}T},\ \ i = \alpha,\beta\]

Si sostituisce questo risultato nel rapporto tra le due probabilità:

\[\frac{P\left( \varepsilon_{1}^{\alpha} \right)}{P\left( \varepsilon_{1}^{\beta} \right)} = \frac{\exp\left[ \sigma\left( U - \varepsilon_{1}^{\alpha},N_{2} \right) \right)}{\exp\left[ \sigma\left( U - \varepsilon_{1}^{\beta},N_{2} \right) \right)} \simeq \frac{\exp\left[ \sigma(U) \right)\exp\left[ - \frac{\varepsilon_{1}^{\alpha}}{k_{B}T} \right)}{\exp\left[ \sigma(U) \right)\exp\left[ - \frac{\varepsilon_{1}^{\beta}}{k_{B}T} \right)}\]

Dato che l'entropia è la stessa per entrambe le configurazione, si semplifica \(\exp\left[ \sigma(U) \right)\):

\[\frac{P\left( \varepsilon_{1}^{\alpha} \right)}{P\left( \varepsilon_{1}^{\beta} \right)} \simeq \frac{\exp\left( - \frac{\varepsilon_{1}^{\alpha}}{k_{B}T} \right)}{\exp\left( - \frac{\varepsilon_{1}^{\beta}}{k_{B}T} \right)}\]

Dato che il rapporto tra le due probabilità \(P\left( \varepsilon_{1}^{\alpha} \right)\) e \(P\left( \varepsilon_{1}^{\beta} \right)\), restituisce il rapporto tra due esponenziali, la probabilità generica \(P(\varepsilon)\) deve essere proporzionale all'esponenziale:

\[P(\varepsilon) \propto \exp\left( - \frac{\varepsilon}{k_{B}T} \right)\]

Il termine esponenziale al secondo membro, \(\exp\left( - \frac{\varepsilon}{k_{B}T} \right)\), è detto fattore di Boltzmann.

Per ottenere la dipendenza esatta della probabilità dall'energia del piccolo sistema, \(\varepsilon\), si introduce un fattore di normalizzazione \(Z\), tale che:

\[P(\varepsilon) = \frac{1}{Z}\exp\left( - \frac{\varepsilon}{k_{B}T} \right)\]

\(Z\) è scelto in modo tale che la somma di tutte le probabilità, al variare dell'energia del sistema \(\mathbb{S}_{1}\) sia unitaria, ovvero:

\[Z:\sum_{n}^{}{P\left( \varepsilon_{n} \right)} = 1\]

Sostituendo l'espressione per la probabilità, si ha:

\[\frac{1}{Z}\sum_{n}^{}{\exp\left( - \frac{\varepsilon_{n}}{k_{B}T} \right)} = 1\]

Da cui:

\[Z = \sum_{n}^{}{\exp\left( - \frac{\varepsilon_{n}}{k_{B}T} \right)}\]

\(Z\) è noto come fattore di ripartizione e funge, come detto, da costante di normalizzazione. La probabilità di una configurazione del piccolo sistema con energia \(\varepsilon\) è:

\[P(\varepsilon) = \frac{\exp\left( - \frac{\varepsilon}{k_{B}T} \right)}{\sum_{n}^{}{\exp\left( - \frac{\varepsilon_{n}}{k_{B}T} \right)}}\]

\subsection{Magnetizzazione macroscopica}\label{magnetizzazione-macroscopica}

Mediante la formula di Boltzmann è possibile determinare la magnetizzazione macroscopica di un volumetto di materiale, contenente un numero di Avogadro di particelle, immerso in un campo magnetico diretto lungo \(z\).

Ogni spin presente nel volumetto possiede uno dei due livelli energetici:

\[\varepsilon_{1,2} = \pm \gamma\frac{\hslash}{2}B_{0}\]

\begin{figure}[ht]
\centering
\includegraphics[width=2.71424in,height=2.52778in,alt={P2523\#yIS1}]{media/5_MecStatistica/image53.pdf}\caption{Volume elementare immerso in un campo magnetico}
\end{figure}

Usando il fattore il fattore di Boltzmann, la probabilità che le particelle si trovino nello stato \(\left| + \right\rangle\) o \(\left| - \right\rangle\) sono date da:

\[P\left( \varepsilon_{1,2} \right) = \frac{1}{Z}\exp\left( - \frac{\varepsilon_{1,2}}{k_{B}T} \right)\]

Dove:

\[Z = \sum_{n}^{}{\exp\left( - \frac{\varepsilon_{n}}{k_{B}T} \right)} = \exp\left( \frac{\gamma\hslash B_{0}}{2k_{B}T} \right) + \exp\left( - \frac{\gamma\hslash B_{0}}{2k_{B}T} \right)\]

Il fattore di Boltzmann è dato dalla somma di due elementi poiché gli spin possono assumere solamente due livelli energetici.

Sia \(N\) il numero degli spin nell'unità di volume. Il momento di magnetizzazione è dato da:

\[M = N\gamma\frac{\hslash}{2}\left( P^{+} - P^{-} \right)\]

Dove \(P^{+}\) è il totale degli spin nello stato \(\left| + \right\rangle\), mentre \(P^{-}\) nello stato \(\left| - \right\rangle\). In altre parole, il vettore di magnetizzazione \(M\) è dato dal netto degli spin nello stato parallelo rispetto a quelli nello stato antiparallelo, rispetto al campo applicato.

Per la distribuzione di Boltzmann, risulta che:

\[P^{+} = \frac{\exp\left( \frac{\gamma\hslash B_{0}}{2k_{B}T} \right)}{\exp\left( \frac{\gamma\hslash B_{0}}{2k_{B}T} \right) + \exp\left( - \frac{\gamma\hslash B_{0}}{2k_{B}T} \right)},\ \ P^{-} = \frac{\exp\left( - \frac{\gamma\hslash B_{0}}{2k_{B}T} \right)}{\exp\left( \frac{\gamma\hslash B_{0}}{2k_{B}T} \right) + \exp\left( - \frac{\gamma\hslash B_{0}}{2k_{B}T} \right)}\]

Il vettore di magnetizzazione, dunque, è dato da:

\[M = N\gamma\frac{\hslash}{2}\left[ \frac{\exp\left( \frac{\gamma\hslash B_{0}}{2k_{B}T} \right) - \exp\left( - \frac{\gamma\hslash B_{0}}{2k_{B}T} \right)}{\exp\left( \frac{\gamma\hslash B_{0}}{2k_{B}T} \right) + \exp\left( - \frac{\gamma\hslash B_{0}}{2k_{B}T} \right)} \right)\]

Il secondo membro coincide con la tangente iperbolica, per cui:

\[M = N\gamma\frac{\hslash}{2}\tanh\left( \frac{\gamma\hslash B_{0}}{2k_{B}T} \right)\]

Con opportune temperature, risulta che:

\[\frac{\gamma\hslash B_{0}}{2k_{B}T} \ll 1 \Leftrightarrow \frac{\gamma\hslash B_{0}}{2} \ll k_{B}T\]

Il momento di magnetizzazione può essere scritto come:

\[M = N\gamma\frac{\hslash}{2}\tanh\left( \frac{\gamma\hslash B_{0}}{2k_{B}T} \right) \simeq \ N\gamma\frac{\hslash}{2}\frac{\gamma\hslash B_{0}}{2k_{B}T}\]

Da cui si ricava la legge di Curie:

\[M \simeq \ N\frac{\gamma^{2}\hslash^{2}}{4k_{B}T}B_{0}\]

Da questa relazione è possibile ricavare un'espressione per la suscettività magnetica \(\chi_{m}\):

\[\vec{M} = \chi_{m}\vec{H}\]

La magnetizzazione netta dipende dal campo applicato e dall'inverso della temperatura; gli altri parametri sono costanti, dunque, non è possibile agire su di essi. In risonanza magnetica, la temperatura non può essere resa piccola a piacere per non raffreddare eccessivamente il paziente. Si utilizzano, per tale motivo, campi magnetici molto elevati, dell'ordine di \(1.5\ T\) in diagnosti, \(3\ T\) in terapia e \(7\ T\) in ricerca.

\subsection{Legge di Planck}\label{legge-di-planck}

Si considera una cavità metallica rettangolare, con un piccolo foro sul centro di un lato. Questo oggetto approssima il comportamento del corpo nero. Dal foro fuoriesce una radiazione elettromagnetica, la cui distribuzione energetica è dipendente dalla temperatura della cavità.

\begin{figure}[ht]
\centering
\includegraphics[width=1.525in,height=2.12413in,alt={P2550\#yIS1}]{media/5_MecStatistica/image54.pdf}\caption{Approssimazione del corpo nero}
\end{figure}

Mediatane la meccanica classica non è possibile spiegare il comportamento dello spettro di emissione del corpo nero. Plank ipotizzò che la radiazione elettromagnetica avesse una natura quantizzata su livello energetico, data da:

\[E = nh\nu,\ \ n\mathbb{\in N}\]

Il comportamento del corpo nero può essere spiegato e descritto mediante l'uso della meccanica quantistica e statistica.

Nella cavità esistono dei modi di oscillazione del campo magnetico. Sia \(\omega\) la pulsazione di uno di questi modi.

L'ipotesi di Planck afferma che l'energia è quantizzata, dunque, può assumere valori multipli di una quantità fondamentale \(\hslash\omega\). I livelli energetici dei fotoni di questo modo sono dati da:

\[E_{s} = s\hslash\omega,\ \ s\mathbb{\in N}\]

Si considera un singolo fotone, visto come particella immessa in contatto con un sistema molto più grande costituito dai restanti fotoni. La probabilità che un fotone si trovi al livello energetico \(E_{s} = s\hslash\omega\), è data da:

\[P\left( E_{s} \right) = \frac{1}{Z}\exp\left( - \frac{s\hslash\omega}{k_{B}T} \right)\]

Dove il fattore di ripartizione \(Z\), nell'ipotesi che vi siano infiniti fotoni nella cavità, è dato da:

\[Z = \sum_{s = 0}^{\infty}{\exp\left( - \frac{s\hslash\omega}{k_{B}T} \right)}\]

Il fattore di ripartizione \(Z\) è una serie geometrica con ragione minore dell'unità, dunque, convergente:

\[Z = \sum_{s = 0}^{\infty}{\exp\left( - \frac{s\hslash\omega}{k_{B}T} \right)} = \frac{1}{1 - \exp\left( - \frac{\hslash\omega}{k_{B}T} \right)}\]

La probabilità che un fotone si trovi nel livello energetico \(s\) è:

\[P\left( E_{s} \right) = \frac{1}{Z}\exp\left( - \frac{s\hslash\omega}{k_{B}T} \right) = \exp\left( - \frac{s\hslash\omega}{k_{B}T} \right)\left[ 1 - \exp\left( - \frac{\hslash\omega}{k_{B}T} \right) \right)\]

Si calcola il valor medio dell'energia del modo con pulsazione \(\omega\):

\[\left\langle E \right\rangle = \sum_{s = 0}^{\infty}{E_{s}P\left( E_{s} \right)}\]

Dove \(E_{s} = s\hslash\omega\). Sostituendo le espressioni per\(E_{s}\) e \(P\left( E_{s} \right)\), si ricava:

\[\left\langle E \right\rangle = \sum_{s = 0}^{\infty}{E_{s}P\left( E_{s} \right)} = \sum_{s = 0}^{\infty}{s\hslash\omega\exp\left( - \frac{s\hslash\omega}{k_{B}T} \right)\left[ 1 - \exp\left( - \frac{\hslash\omega}{k_{B}T} \right) \right)} =\]

Per la linearità della sommatoria, si ha:

\[\left\langle E \right\rangle = \ \hslash\omega\left[ 1 - \exp\left( - \frac{\hslash\omega}{k_{B}T} \right) \right)\sum_{s = 0}^{\infty}{s\exp\left( - \frac{s\hslash\omega}{k_{B}T} \right)}\]

Si considera la quantità:

\[\sum_{s = 0}^{\infty}{s\exp( - xs)},\ \ x = \frac{s\hslash\omega}{k_{B}T}\]

È possibile scrivere che:

\[s\exp( - xs) = - \frac{d}{dx}\exp( - xs)\]

Per cui è possibile scrivere:

\[\left\langle E \right\rangle = \ \hslash\omega\left[ 1 - \exp\left( - \frac{\hslash\omega}{k_{B}T} \right) \right)\sum_{s = 0}^{\infty}\left[ - \frac{d}{dx}\exp\left( - \frac{s\hslash\omega}{k_{B}T} \right) \right)\]

Per la linearità della sommatoria e della derivata si ha:

\[\left\langle E \right\rangle = - \ \hslash\omega\left[ 1 - \exp\left( - \frac{\hslash\omega}{k_{B}T} \right) \right)\frac{d}{dx}\sum_{s = 0}^{\infty}{\exp( - xs)}\]

La serie risulta essere convergente in quanto serie geometrica con ragione in modulo minore dell'unità. Il valor medio può essere scritto come:

\[\left\langle E \right\rangle = - \ \hslash\omega\left[ 1 - \exp\left( - \frac{\hslash\omega}{k_{B}T} \right) \right)\frac{d}{dx}\left[ \frac{1}{1 - \exp( - x)} \right)\]

Svolgendo la derivata, si ottiene:

\[\left\langle E \right\rangle = - \ \hslash\omega\left[ 1 - \exp\left( - \frac{\hslash\omega}{k_{B}T} \right) \right)\frac{- \exp\left( - \frac{\hslash\omega}{k_{B}T} \right)}{\left[ 1 - \exp\left( - \frac{\hslash\omega}{k_{B}T} \right) \right)^{2}}\]

Semplificando:

\[\left\langle E \right\rangle = \ \hslash\omega\frac{\exp\left( - \frac{\hslash\omega}{k_{B}T} \right)}{\left[ 1 - \exp\left( - \frac{\hslash\omega}{k_{B}T} \right) \right)}\]

È possibile raccogliere il termine esponenziale \(\exp\left( - \frac{\hslash\omega}{k_{B}T} \right)\):

\[\left\langle E \right\rangle = \ \frac{\hslash\omega}{\left[ \exp\left( \frac{\hslash\omega}{k_{B}T} \right) - 1 \right)}\]

Tale equazione è la legge di Planck e rappresenta una prima logge in cui si applica la quantizzazione della materia, unita alle teoria di Boltzmann.

\subsection{Rumore termico o di Johnson-Nyquist}\label{rumore-termico-o-di-johnson-nyquist}

Il rumore termico o di Johnson-Nyquist è sempre presente nei componenti elettronici, come resistori o dispositivi a semiconduttore.

Si considera una linea di trasmissione chiusa alle estremità da due resistori uguali alle impedenza caratteristica della linea. In altre parole, le resistenze sono adattate alla linea. Sulla linea di trasmissione, cioè, viaggiano solamente onde progressive o modi, a causa dell'adattamento (\(\Gamma = 0\)), generate dalle resistenze, la cui energia dipende dalla temperatura a cui si trovano, in accordo con la legge di Planck. Poiché il sistema è adattato, la potenza trasferita sul carico è data da:

\[P = \frac{\left\langle V^{2} \right\rangle}{4R} = \left\langle I^{2} \right\rangle R\]

\begin{figure}[ht]
\centering
\includegraphics[width=4.11381in,height=1.43842in,alt={P2593\#yIS1}]{media/5_MecStatistica/image55.pdf}\caption{Linea di trasmissione adattata}
\end{figure}

Per ogni pulsazione del modo, \(\omega\), esistono due onde viaggianti in direzione opposte. Sia \(L\) la lunghezza della linea e \(c\) la velocità di propagazione del segnale sulla linea. Il tempo impiegato dall'onda per percorrere l'intera linea di trasmissione è:

\[{\Delta}t = \frac{L}{c}\]

I modi presenti sulla linea presentano delle frequenze multiple intere di una quantità \(\delta f\) data da:

\[\delta f = \frac{1}{{\Delta}t} = \frac{c}{L}\]

Le frequenze dei modi sono multiple di \(\delta f\)\emph{,} per cui in un certo intervallo di frequenze \({\Delta}f\) è presente un numero di modi dato da:

\[n_{modi} = \frac{{\Delta}f}{\delta f} = \frac{L}{c}{\Delta}f\]

Noto il numero di modi presenti nell'intervallo frequenziale \({\Delta}f\), è possibile valutare l'energia complessivamente presente sulla linea di trasmissione:

\[E = 2\frac{{\Delta}f}{\delta f}\left\langle E \right\rangle\]

Dove il \(2\) è dovuto alle due onde trasmesse dai due resistori.

Ogni singolo modo può essere considerato come un fotone immerso in un sistema, composto dagli altri fotoni, per cui è possibile applicare la legge di Planck per il calcolo dell'energia media:

\[\left\langle E \right\rangle = \ \frac{\hslash\omega}{\left[ \exp\left( \frac{\hslash\omega}{k_{B}T} \right) - 1 \right)}\]

Da cui:

\[E = 2\frac{{\Delta}f}{\delta f}\frac{\hslash\omega}{\left[ \exp\left( \frac{\hslash\omega}{k_{B}T} \right) - 1 \right)}\]

Per le normali frequenze utilizzate nella pratica elettrotecnica, risulta che:

\[\hslash\omega \ll k_{B}T \Leftrightarrow \frac{\hslash\omega}{k_{B}T} \ll 1\]

In questa ipotesi, è possibile approssimare l'esponenziale in serie di Taylor, arrestato al primo ordine:

\[\exp\left( \frac{\hslash\omega}{k_{B}T} \right) \simeq 1 + \frac{\hslash\omega}{k_{B}T}\]

Con questa approssimazione, l'energia media è:

\[\left\langle E \right\rangle = \ \frac{\hslash\omega}{\left[ \exp\left( \frac{\hslash\omega}{k_{B}T} \right) - 1 \right)} \simeq \ \frac{\hslash\omega}{\left[ 1 + \frac{\hslash\omega}{k_{B}T} - 1 \right)} = k_{B}T\]

L'energia complessiva, di conseguenza, è:

\[E = 2\frac{{\Delta}f}{\delta f}k_{B}T\]

A questa energia corrisponde una potenza data, per definizione, da:

\[P = \frac{E}{{\Delta}t} = 2\frac{1}{{\Delta}t}\frac{{\Delta}f}{\delta f}k_{B}T\]

Ma \({\Delta}t\) è l'inverso di \(\delta f\), dunque, il prodotto dei due termini è unitario. La potenza è, quindi:

\[P = \frac{E}{{\Delta}t} = 2{\Delta}fk_{B}T\]

Su un singolo resistore, si ritrova metà potenza, ovvero:

\[P_{R} = \frac{P}{2} = {\Delta}fk_{B}T\]

Per l'adattamento in potenza, la potenza trasferita al carico è:

\[P = \frac{\left\langle V^{2} \right\rangle}{4R}\]

Sostituendo l'espressione appena determinata per la potenza, è possibile ricavare \(\left\langle V^{2} \right\rangle\):

\[{\Delta}fk_{B}T = \frac{\left\langle V^{2} \right\rangle}{4R} \Leftrightarrow \left\langle V^{2} \right\rangle = 4R{\Delta}fk_{B}T\]

\(\left\langle V^{2} \right\rangle\) corrisponde al valor quadratico medio della tensione su un resistore, dovuto all'agitazione termica dei suoi portatori di carica, nella banda \({\Delta}f\). Il rumore termico presenta la stessa ampiezza in tutto il range frequenziale \({\Delta}f\) per cui può essere modellato come un rumore bianco.

Non approssimando l'esponenziale, il valor quadratico medio della tensione su un resistore è dato da:

\[\left\langle V^{2} \right\rangle = 4R{\Delta}f\ \frac{\hslash\omega}{\left[ \exp\left( \frac{\hslash\omega}{k_{B}T} \right) - 1 \right)}\]

La frequenza di taglio è:

\[f_{0} = \frac{k_{B}T}{2\pi\hslash}\]

Questa frequenza, a temperatura ambiente, è data da:

\[f_{0} = \frac{k_{B}T}{2\pi\hslash} = \frac{1.38 \cdot 10^{- 23}\frac{J}{K} \cdot 290\ K}{6.63 \cdot 10^{- 34}\ J \cdot s} = 6.05\ THz\]

Questa frequenza, come detto precedentemente, è molto maggiore di quelle ottenibili con l'attuale strumentazione elettronica. Per questo motivo si ricorre alla relazione approssimata.

\section{Distribuzione di Gibbs}\label{distribuzione-di-gibbs}

Si considerano due sistemi contenenti, rispettivamente, \(N_{1}\) e \(N_{2}\) particelle. I due sistemi sono posti in contatto si termico che diffusivo, ovvero possono scambiare sia materia che energia.

Il sistema totale ha energia data dalla somme delle singole energie iniziali dei due sistemi. \(U = U_{1} + U_{2}\). Si suppone, infine, che il sistema complessivo sia isolato dall'ambiente, così da conservare la propria energia.

\begin{figure}[ht]
\centering
\includegraphics[width=4.71575in,height=2.30556in,alt={P2637\#yIS1}]{media/5_MecStatistica/image56.pdf}\caption{Sistemi posti in contatto termico e diffusivo}
\end{figure}

Prima del contatto, i due sistemi possedevano un'entropia, rispettivamente, uguali a \(\sigma_{1}\) e \(\sigma_{2}\). Dopo il contatto, per ragioni probabilistiche, l'entropia dovrà essere massima.

Si suppone che il sistema \(\mathbb{S}_{1}\) sia molto più piccolo del sistema \(\mathbb{S}_{2}\), approssimabile come un serbatoio termico. Si vuole valutare la probabilità che il sistema \(\mathbb{S}_{1}\) sia in uno stato caratterizzato da \(N_{a}\) particelle ed energia \(U_{a}\).

Fissata la configurazione del sistema \(\mathbb{S}_{1}\), la probabilità che si verifichi questa condizione dipende dalle configurazioni ammissibili del sistema \(\mathbb{S}_{2}\), ovvero:

\[P\left( N_{a},U_{a} \right) = g_{2}\left( N_{2},U_{2} \right)\]

Siccome il numero delle particelle è costante, così come l'energia, è possibile scrivere:

\[N = N_{a} + N_{2} \Leftrightarrow N_{2} = N - N_{a},\ \ U = U_{a} + U_{2} \Leftrightarrow U_{2} = U - U_{a}\]

La numerosità degli stati ammissibili dal sistema \(\mathbb{S}_{2}\) può essere scritta come:

\[P\left( N_{a},U_{a} \right) \propto g_{2}\left( N - N_{a},U - U_{a} \right)\]

La probabilità che il sistema \(\mathbb{S}_{1}\) assuma un'altra configurazione, caratterizzata da un numero \(N_{b}\) di particelle da un'energia uguale a \(U_{b}\), è data da:

\[P\left( N_{b},U_{b} \right) \propto g_{2}\left( N - N_{b},U - U_{b} \right)\]

Il rapporto tra le due probabilità si scrive come:

\[\frac{P\left( N_{a},U_{a} \right)}{P\left( N_{b},U_{b} \right)} = \frac{g_{2}\left( N - N_{a},U - U_{a} \right)}{g_{2}\left( N - N_{b},U - U_{b} \right)}\]

Si considera il logaritmo di tale rapporto:

\[\log\left[ \frac{P\left( N_{a},U_{a} \right)}{P\left( N_{b},U_{b} \right)} \right) = \log\left[ \frac{g_{2}\left( N - N_{a},U - U_{a} \right)}{g_{2}\left( N - N_{b},U - U_{b} \right)} \right) = \log{g_{2}\left( N - N_{a},U - U_{a} \right)} - \log{g_{2}\left( N - N_{b},U - U_{b} \right)}\]

Per definizione di entropia:

\[\sigma(N,U) = \log{g(N,U)}\]

Il logaritmo del rapporto può essere espresso come:

\[\log\left[ \frac{P\left( N_{a},U_{a} \right)}{P\left( N_{b},U_{b} \right)} \right) = \sigma\left( N - N_{a},U - U_{a} \right) - \sigma\left( N - N_{b},U - U_{b} \right)\]

Siccome il sistema \(\mathbb{S}_{2}\) è molto più grande del sistema \(\mathbb{S}_{1}\), è possibile concludere che:

\[U \gg U_{a},U_{b},\ \ N \gg N_{a},N_{b}\]

È possibile sviluppare in serie di Taylor l'entropia, arrestando lo sviluppo al primo ordine:

\[\sigma\left( N - N_{i},U - U_{i} \right) = \ \sigma(N,U) - \left. \ \frac{\partial\sigma}{\partial N} \right|_{N}N_{i} - \left. \ \frac{\partial\sigma}{\partial U} \right|_{U}U_{i},\ \ i = a,b\]

Con questa approssimazione si ottiene:

\[\log\left[ \frac{P\left( N_{a},U_{a} \right)}{P\left( N_{b},U_{b} \right)} \right) = \sigma(N,U) - \left. \ \frac{\partial\sigma}{\partial N} \right|_{N}N_{a} - \left. \ \frac{\partial\sigma}{\partial U} \right|_{U}U_{a} - \sigma(N,U) + \left. \ \frac{\partial\sigma}{\partial N} \right|_{N}N_{b} - \left. \ \frac{\partial\sigma}{\partial U} \right|_{U}U_{b}\]

Elidendo \(\sigma(N,U)\) e raccogliendo:

\[\log\left[ \frac{P\left( N_{a},U_{a} \right)}{P\left( N_{b},U_{b} \right)} \right) = \left( N_{b} - N_{a} \right)\left. \ \frac{\partial\sigma}{\partial N} \right|_{N} + \left( U_{b} - U_{a} \right)\left. \ \frac{\partial\sigma}{\partial U} \right|_{U}\]

Si definisce la temperatura come:

\[\left. \ \frac{\partial\sigma}{\partial U} \right|_{U} = \frac{1}{k_{B}T}\]

Si definisce potenziale chimico \(\mu\) come il fattore di proporzionalità tra la derivata dell'entropia rispetto al numero di particelle e la temperatura:

\[\left. \ \frac{\partial\sigma}{\partial N} \right|_{N} = - \frac{\mu}{k_{B}T}\]

Con queste definizioni, si ha:

\[\log\left[ \frac{P\left( N_{a},U_{a} \right)}{P\left( N_{b},U_{b} \right)} \right) = - \left( N_{b} - N_{a} \right)\frac{\mu}{k_{B}T} + \left( U_{b} - U_{a} \right)\frac{1}{k_{B}T} = \left( N_{a} - N_{b} \right)\frac{\mu}{k_{B}T} - \left( U_{a} - U_{b} \right)\frac{1}{k_{B}T}\]

Si applica l'esponenziale, si ha:

\[\frac{P\left( N_{a},U_{a} \right)}{P\left( N_{b},U_{b} \right)} = \exp\left[ \left( N_{a} - N_{b} \right)\frac{\mu}{k_{B}T} \right)\exp\left[ - \left( U_{a} - U_{b} \right)\frac{1}{k_{B}T} \right) = \frac{\exp\left[ \left( N_{a} - N_{b} \right)\frac{\mu}{k_{B}T} \right)}{\exp\left[ - \left( U_{a} - U_{b} \right)\frac{1}{k_{B}T} \right)}\]

Quindi, la probabilità che il sistema \(\mathbb{S}_{1}\) si trovi in uno stato con energia \(U_{a}\) e con un numero di particelle \(N_{a}\) è proporzionale a:

\[P\left( N_{a},U_{a} \right) \propto \exp\left( \frac{\mu N_{a} - U_{a}}{k_{B}T} \right)\]

L'esponenziale nella relazione individuata è detto fattore di Gibbs. La probabilità esatta si esprime introducendo un fattore di normalizzazione \(Z\), tale che:

\[\frac{1}{Z}\sum_{N}^{}{\sum_{U}^{}{P(N,U)}} = 1\]

Sostituendo la relazione per la probabilità, si ricava:

\[\frac{1}{Z}\sum_{N}^{}{\sum_{U}^{}{\exp\left( \frac{\mu N - U}{k_{B}T} \right)}} = 1 \Leftrightarrow Z = \sum_{N}^{}{\sum_{U}^{}{\exp\left( \frac{\mu N - U}{k_{B}T} \right)}}\]

Il fattore di partizione \(Z\) così ottenuto è detto somma di Gibbs.

La probabilità che il piccolo sistema \(\mathbb{S}_{1}\) si trovi in uno stato \(\left( U_{a},N_{a} \right)\) è, dunque, data da:

\[P\left( N_{a},U_{a} \right) = \frac{\exp\left( \frac{\mu N_{a} - U_{a}}{k_{B}T} \right)}{\sum_{N}^{}{\sum_{U}^{}{\exp\left( \frac{\mu N - U}{k_{B}T} \right)}}}\]

La quantità \(\mu\) rappresenta la capacità di scambio diffusivo tra sistemi messi in contatto diffusivo e termico.

\section{Distribuzione di Fermi-Dirac}\label{distribuzione-di-fermi-dirac}

I fermioni sono particelle elementari con spin frazionario, ovvero uguale a \(\pm 1/2\). Due fermioni non possono occupare lo stesso livello energetico per il principio di esclusione di Pauli.

Si considera un piccolo sistema costituito da fermioni, come protoni o elettroni. Siano \(N = 0\) e \(N = 1\) gli stati ammissibili per un fermione, corrispondenti ai livelli energetici \(U = 0\) e \(U = \varepsilon\).

Un fermione o non è presente nel livello energetico, nel caso \(U = 0\) e \(N = 0\), o c'è, nel caso \(U = \varepsilon\) e \(N = 1\). la somma di Gibbs è data da:

\[Z = \sum_{N}^{}{\sum_{U}^{}{\exp\left( \frac{\mu N - U}{k_{B}T} \right)}} = \exp\left( \frac{\mu 0 - 0}{k_{B}T} \right) + \exp\left( \frac{\mu \cdot 1 - \varepsilon}{k_{B}T} \right) = 1 + \exp\left( \frac{\mu - \varepsilon}{k_{B}T} \right)\]

Si vuole determinare il numero medio di particelle del piccolo sistema \(\mathbb{S}_{1}\) con energia \(U = \varepsilon\). Per definizione di media, si ha:

\[\left\langle N(\varepsilon) \right\rangle = \frac{1}{Z}\sum_{N}^{}{\sum_{U}^{}{P(N,U)}} = \frac{\exp\left( \frac{\mu - \varepsilon}{k_{B}T} \right)}{1 + \exp\left( \frac{\mu - \varepsilon}{k_{B}T} \right)}\]

La somma si riduce a un solo elemento, in quanto solo un fermione può trovarsi in quel livello energetico. Raccogliendo il fattore di Gibbs, si ottiene:

\[\left\langle N(\varepsilon) \right\rangle = \frac{1}{1 + \exp\left( \frac{\mu - \varepsilon}{k_{B}T} \right)}\]

La distribuzione di Fermi-Dirac permette di conoscere il numero medio dei fermioni in un determinato livello energetico. La statistica è utilizzata nel campo dei semiconduttori o cristalli scintillatori al fine di valutare il numero di elettroni in un determinato livello energetico.

\subsection{Distribuzione di Bose-Einstein}\label{distribuzione-di-bose-einstein}

I bosoni sono particelle con spin nullo o intero. Un livello energetico può essere occupato da un numero qualsiasi di bosoni, dunque, queste particelle non rispettano il principio di esclusione di Pauli. Il fotone è uno dei bosoni più importante; esso presenta uno spin unitario.

Si considera un piccolo sistema di bosoni. Questo sistema può contenere un numero qualsiasi \(N\) di particelle, ciascuna con energia \(\varepsilon\). La somma di Gibbs, nel caso di energia fissata a \(U = \varepsilon\), si esprime come:

\[Z = \sum_{N}^{}{\sum_{U}^{}{\exp\left( \frac{\mu N - U}{k_{B}T} \right)}} = \sum_{N}^{}{\exp\left( \frac{N\mu - N\varepsilon}{k_{B}T} \right)} = \sum_{N}^{}\left[ \exp\left( \frac{\mu - \varepsilon}{k_{B}T} \right) \right)^{N}\]

La somma è convergente, in quanto serie geometrica con ragione, in modulo, minore dell'unità:

\[Z = \frac{1}{1 - \exp\left( \frac{\mu - \varepsilon}{k_{B}T} \right)}\]

Il numero medio delle particelle del sistema che si trovano nel livello energetico \(\varepsilon\) è:

\[\left\langle N(\varepsilon) \right\rangle = \frac{1}{Z}\sum_{N}^{}{\sum_{U}^{}{NP(N,U)}} = \left[ 1 - \exp\left( \frac{\mu - \varepsilon}{k_{B}T} \right) \right)\sum_{N}^{}{\sum_{U}^{}{N\exp\left( \frac{\mu N - U}{k_{B}T} \right)}}\]

Avendo fissata l'energia, risulta:

\[\left\langle N(\varepsilon) \right\rangle = \left[ 1 - \exp\left( \frac{\mu - \varepsilon}{k_{B}T} \right) \right)\sum_{N}^{}{N\exp\left( \frac{N\mu - N\varepsilon}{k_{B}T} \right)} = \left[ 1 - \exp\left( \frac{\mu - \varepsilon}{k_{B}T} \right) \right)\sum_{N}^{}{N\left[ \exp\left( \frac{\mu - \varepsilon}{k_{B}T} \right) \right)^{N}}\]

Si pone:

\[X = \exp\left( \frac{\mu - \varepsilon}{k_{B}T} \right)\]

Il numero medio di bosoni nel livello energetico \(\varepsilon\) può essere scritto come.

\[\left\langle N(\varepsilon) \right\rangle = (1 - X)\sum_{N}^{}{NX^{N}}\]

Si considera la quantità:

\[Z = \sum_{N}^{}X^{N}\]

Si deriva rispetto a \(X\):

\[\frac{dZ}{dX} = \frac{d}{dX}\sum_{N}^{}X^{N} = \sum_{N}^{}{\frac{d}{dX}\left( X^{N} \right)} = \sum_{N}^{}{NX^{N - 1}} = \sum_{N}^{}{NX^{N}X^{- 1}} =\]

Per la linearità dell'operatore sommatoria, si scrive:

\[\frac{dZ}{dX} = \frac{1}{X}\sum_{N}^{}{NX^{N}}\]

Da cui si ha:

\[\sum_{N}^{}{NX^{N}} = X\frac{dZ}{dX} = X\frac{d}{dX}\left[ \frac{1}{1 - X} \right) = \frac{X}{(1 - X)^{2}}\]

Il numero medio dei bosoni può essere scritto come:

\[\left\langle N(\varepsilon) \right\rangle = (1 - X)\sum_{N}^{}{NX^{N}} = (1 - X)\frac{X}{(1 - X)^{2}} = \frac{X}{1 - X} = \frac{1}{\frac{1}{X} - 1}\]

Sostituendo il valore di \(X\) si ottiene il numero medio dei bosoni nel livello energetico \(\varepsilon\):

\[\left\langle N(\varepsilon) \right\rangle = \frac{1}{\exp\left( - \frac{\mu - \varepsilon}{k_{B}T} \right) - 1}\]

Le distribuzioni di Bose-Einstein, Fermi-Dirac e Gibbs a temperatura ambiente convergono alla distribuzione di Boltzmann. Le statistiche sono sempre valide e descrivono il comportamento dei fermioni e bosoni.


\section{Introduzione alla risonanza magnetica}\label{introduzione-alla-risonanza-magnetica}

\subsection{Diverse metodiche di imaging}\label{diverse-metodiche-di-imaging}

Dal punto di vista estetico, i macchinari risonanza magnetica, tomografia computerizzata (TC) e PET presentano una forma a spirale. Inoltre, tutte queste metodiche di imaging prevedono un gantry, dove è presente la circuiteria necessaria per la produzione dello stimolo da fornire al paziente e la sua ricezione. Si osservi che nella PET, il segnale di stimolo è generato all'interno del paziente stesso, mediante l'iniezione di un mezzo di contrasto.

\begin{figure}
\centering
\includegraphics[width=4.35482in,height=2.05952in,alt={P2729\#yIS1}]{media/6_IntroMRI/image57.pdf}\caption{Figura .: Tipica struttura di una strumentazione di imaging}
\end{figure}

Il paziente viene posto su un tavolo in moto longitudinale, detto tavolo porta-paziente. Sebbene le applicazioni, dal punto di vista esterno, possano sembrare uguali, esse si basano su principi fisici diversi:

\begin{itemize}
\item
  La TC si basa sull'assorbimento, da parte del corpo umano, della radiazione a raggi X. L'assorbimento di questa radiazione è, essenzialmente, legato alla densità del tessuto biologico attraversato. Così, l'osso assorbe una percentuale molto maggiore dei tessuti moli. In particolare, quest'ultimi assorbono una percentuale di raggi X simile tra loro. Mediante TC si ottengono delle immagini con un alto contenuto di informazioni morfologiche dei tessuti;
\item
  La risonanza magnetica si basa sulla risonanza dei nuclei di idrogeno, contenuti nel corpo umano, per effetto di un campo magnetico esterno. La maggior parte dell'idrogeno nel corpo umano è contenuto negli atomi di acqua (\(H_{2}O\)), dunque, è possibile discriminare i tessuti biologici in base al suo contenuto acquoso. Con questa metodica, l'osso, essendo un tessuto duro, contiene una quantità di acqua molto più bassa dei tessuti molli, dunque, è poco visibile;
\item
  La PET si basa sull'emissione di fotoni \(\gamma\) da parte di un radiofarmaco iniettato nel paziente. La distribuzione di radiofarmaco fornisce informazione sul comportamento metabolico dei tessuti. Dunque, mentre la TC fornisce immagini con informazioni morfologiche, la PET fornisce immagini funzionali poiché, appunto, indicative delle attività metaboliche di un tessuto.
\end{itemize}

La strumentazione moderna prevede una combinazione della CT e della PET, ottenendo la PETT-CT, in modo da ottenere immagini con un alto contenuto informativo sia morfologico che funzionale.

La risonanza magnetica può fornire immagini sia morfologiche che funzioni. Queste ultime presentano la maggior applicazione nello studio dell'encefalo: in base alle aree attivate da un determinato stimolo, si genera una mappa a colori, indicante il funzionamento della corteccia cerebrale.

La MRI (\emph{magnetic resonance imaging}) funzionale è utilizzata per la diagnosi di malattie neurodegenerative. Questo tipo di studio non può essere eseguito mediante CT poiché la scatola cranica assorbe la maggior parte della radiazione X incidente. Ne risulta, dunque, un'immagine dell'encefalo poco definita.

La CT è essenzialmente utilizzata per ottenere informazioni di natura morfologica sugli organi interni in breve tempo. Le immagini CT presentano una buona risoluzione nel discriminare le ossa; tuttavia, la risoluzione peggiora con i tessuti molli.

Utilizzando una radiazione X, la CT possiede una pericolosità intrinseca, dovuta ai possibili effetti cancerogeni della radiazione ionizzante. I danni biologici si verificano con un certo andamento statistico, quindi, sono state prodotte delle specifiche normative che regolano la dose assorbita sia dal paziente che dal tecnico radiologo.

La tomografia computerizzata è la metodica di imaging più veloce poiché, mediante le attuali tecnologie, permette di ottenere delle scansioni \emph{total body} in pochi secondi. Per tale motivo la CT è la metodica di analisi più utilizzata per eseguire la diagnosi di una patologia in tempo breve, seppur la risoluzione grossolana per i tessuti molli.

La risonanza magnetica, a differenza delle altre metodiche, non utilizza radiazioni ionizzanti ma campi elettromagnetici statici, di grande intensità, e a radiofrequenza, dell'ordine dei \(mT\), Ai campi magnetici non sono associati particolari effetti nocivi, tuttavia, la loro intensità in risonanza magnetica è normata da norme nazionali e internazionali.

L'utilizzo del campo elettromagnetico consente di eseguire l'esame di risonanza magnetica ripetutamente. Alla risonanza magnetica, inoltre, è associata una bassa invasività e una grande flessibilità, poiché permette di ottenere sia immagini morfologiche che funzionali, ovvero, è possibile visualizzare un tessuto biologico in base alla sua composizione biochimica.

A differenza della CT, tuttavia, l'esame con risonanza magnetica richiede un lungo tempo di esecuzione, che si aggira intorno ai \(30\) minuti fino a un'ora. Ciò provoca anche un senso di disagio per il paziente, il quale deve restare immobile per un lungo periodo di tempo. Tuttavia, a causa dei normali movimento del paziente, si producono artefatti da movimento durante l'esecuzione dell'esame radiologico. Infine, la strumentazione usata per la risonanza magnetica è più complessa di quella della CT, poiché deve produrre campi statici e a radiofrequenza con determinate caratteristiche. Ne discende che la risonanza magnetica è più costosa della CT.

La PET condivide con la risonanza magnetica i lunghi tempo di analisi, infatti, al fine di ottenere delle immagini \emph{total body} sono richiesti di \(45\) minuti a un'ora. Storicamente, questa metodologia diagnostica nasce per eseguire lo studio metabolico del cervello. Attualmente, l'applicazione più frequente dalla PET è lo studio del comportamento metabolico dei tumori, al fine di evidenziare il suo stadio e la presenza di metastasi. Infatti, in presenza di diagnosi di tumore, è necessario eseguire la PET almeno una volta all'anno, al fine di rilevare precocemente la presenza di nuove metastasi, formatesi per problemi legati alla recidiva tumorale.

La PET si basa sull'emissione di radiazione \(\gamma\) da parte di radionuclidi eccitati. La diversa distribuzione del radio-metabolita permette di ottenere immagini indicanti la funzionalità degli organi interessati.

L'unione della PET con la CT permette di ottenere immagini con informazioni morfologiche e funzionali. La sola informazione funzionale, generalmente, risulta essere di difficile compressione poiché in questo tipo di immagini non sono evidenziate le strutture anatomiche che emettono quella data quantità di tracciante.

Lo svantaggio principale associato alla PET è la presenza di radiazione \(\gamma\) emessa dai radionuclidi. Il paziente, dunque, è una fonte di radiazioni \(\gamma\) finché il radiofarmaco non ha esaurito la sua radioattività, generalmente entro le \(24\) ore. In questo intervallo temporale, il paziente non può entrare in contatto con donne incinte e bambini.

\subsection{Storia della risonanza magnetica}\label{storia-della-risonanza-magnetica}

Con l'esperimento di Stern e Gerlach si dimostrò che il momento magnetico atomico è quantizzato, ovvero può assumere solamente due valori a cui corrispondono due determinati livelli energetici.

Negli anni '30 il fisico Felix Bloch sviluppò delle equazioni fenomenologiche, basate su una descrizione intermedia tra fisica classica e meccanica quantistica, per descrivere il comportamento degli spin immersi in un campo magnetico.

Grazie agli studi di Bloch, negli anni '70 il chimico Paul Christian Lauterbur mise appunto una tecnica per ottenere immagini di sezioni di un corpo mediante campi elettromagnetici a radiofrequenza. La prima risonanza magnetica fu eseguita su un limone. Da questa prima applicazione sono stati sviluppato dei macchinari commerciali, attualmente utilizzati in molti ambiti della diagnostica medica. A Lauterbur fu attribuita l'idea secondo la quale i gradienti del campo magnetico consentono l'individuazione dell'origine delle onde radio emesse dai nuclei dell'oggetto in esame, ottenendo, così, immagini bidimensionali.

\subsection{Introduzione al principio di risonanza magnetica}\label{introduzione-al-principio-di-risonanza-magnetica}

Il principio fisico su cui si basa la risonanza magnetica è descritto dalle equazioni di Bloch: un gran numero di protoni di idrogeno, immersi in un campo magnetico \(\overset{\underline{}}{B}\), producono una magnetizzazione netta \(\overset{\underline{}}{M}\), misurabile, a temperatura ambiente e all'equilibrio termodinamico, mediante la legge di Curie:

\[M \simeq \frac{N}{V}\frac{\gamma^{2}\hslash^{2}}{4k_{B}T}B_{0}\]

Dove \(M\) è il momento magnetico per unità di volume, il quale contiene \(N\) particelle con spin. La grandezza \(N/V\) è detta densità protonica ed è indicata con \(\rho\).

Dalla legge i Curie si evince che la magnetizzazione dipende dal campo magnetico applicato, dalla temperatura, dal rapporto giromagnetico e dal numero di spin presenti nel volume elementare sotto analisi. Ne discende che per aumentare il valore della magnetizzazione è possibile:

\begin{itemize}
\item
  Incrementare il valore del campo magnetico utilizzato. Tipicamente il valore utilizzato del campo magnetico è \(1.5\ T\). Questo valore è regolato da norme internazionali;
\item
  La temperatura non può essere ridotta a piacere poiché non è possibile raffreddare eccessivamente il paziente. Nella sala della risonanza magnetica la temperatura è mantenuta costante, intorno ai \(25\ {^\circ}C\), al fine di evitare fluttuazioni della magnetizzazione;
\item
  È possibile scegliere la sostanza di cui si vuole calcolare la magnetizzazione, selezionando quella con un rapporto giromagnetico maggiore. In linea di principio è possibile determinare immagini di risonanza magnetica anche degli elettroni, che presentano un rapporto giromagnetico \(\gamma_{e} = - 1.76 \cdot 10^{11}\ rad/Ts\), mentre quello del protone è \(\gamma_{p} = 2.68 \cdot 10^{8}\ rad/Ts\). Il rapporto di \(\gamma_{e}\) e \(\gamma_{p}\) è:
\end{itemize}

\[\gamma = \frac{\left| \gamma_{e} \right|}{\gamma_{p}} = \frac{1.76 \cdot 10^{11}\ \frac{rad}{Ts}}{2.68 \cdot 10^{8}\ \frac{rad}{Ts}} = 658\]

Il rapporto giromagnetico dell'elettrone è molto maggiore di quel del protone, ovvero del nucleo di idrogeno, quindi, il vettore magnetizzazione degli elettroni, all'equilibrio termodinamico, ha intensità maggiore rispetto a quello dei nuclei di idrogeno. Si osservi, tuttavia, che il fattore giromagnetico è presente nell'espressione della frequenza di precessione di Larmor, ovvero la frequenza con cui gli spin ruotano intorno all'asse individuato dal campo magnetico:

\[\omega_{0} = \gamma B_{0}\]

La frequenza del campo magnetico applicato aumenta al crescere del rapporto giromagnetico, in particolare, con un campo di \(1.5\ T\), per un elettrone, si ha:

\[f_{0} = 2\pi\omega_{0} = 2\pi 1.76 \cdot 10^{11}\ \frac{rad}{Ts}1.5\ T = 42\ GHz\]

Il campo irradiato dall'elettrone è, dunque, dell'ordine della decina di \(GHz\). Ciò determina una maggiore energia associata all'onda, che si deposita nei tessuti biologici. In generale, più l'onda si avvicina allo spettro dei raggi X, maggiore è il loro contenuto energetico e maggiori sono i possibili effetti biologici. Per tale motivo le radiofrequenze adoperate sono ottimizzate per il protone.

\begin{itemize}
\item
  Si potrebbe pensare di utilizzare un atomo o un composto che risuoni alle radiofrequenze ma con un fattore giromagnetico più alto; tuttavia, l'idrogeno ha una concentrare di \(88\ M = 88\ mol/V\), molto maggiore degli altri composti che presentano una concentrazione molare dell'ordine dei \(\mu M\) o \(mM\). L'uso della risonanza dei protoni consente di ottenere il giusto compromesso tra energia depositata nel paziente, dunque effetto biologico, e un elevato numero di spin per unità di volume, ottimizzando, di conseguenza, il valore della magnetizzazione all'equilibrio;
\item
  L'unico parametro che può essere modificato per aumentare la magnetizzazione è il campo statico esterno applicato, detto principale.
\end{itemize}

Il vettore di magnetizzazione è valutato su un volumetto elementare contenente un numero di Avogadro \(N_{A}\) di particelle. Ne discende che in risonanza magnetica il paziente può essere considerato come un insieme di volumetti elementari, ognuno dei quali possiede il proprio vettore di magnetizzazione \(d\overset{\underline{}}{M}\). La ricostruzione del momento magnetico permettere di eseguire l'imaging del corpo umano. Si osservi che non tutti i volumetti considerati possiedono lo stesso numero di particelle. In media, è possibile ritenere che il numero delle particelle sia pressocché lo stesso se si considerano volumetti elementari con dimensione lineare di \(1\ mm\).

Giunti all'equilibrio termodinamico, il vettore magnetizzazione non è direttamente misurabile, poiché non produce alcun segnale variabile nel tempo da captare con apposite antenne. Per ottenere un'immagine tomografica è necessario perturbare l'equilibrio termodinamico e registrare il segnale emesso dal corpo del paziente durante il ritorno all'equilibrio dei vettori di magnetizzazione di ogni singolo volumetto elementare in cui è scomponibile il paziente.

I protoni, ovvero i nuclei degli atomi di idrogeno, non sono presenti solamente nell'acqua ma sono legati anche ad altre molecole biologiche come proteine, acidi nuclei e lipidi. I nuclei di idrogeno contenuti in queste molecole non sono soggetti allo stesso campo magnetico principale imposto dall'esterno. Ciò è dovuto all'effetto di schermatura prodotto dalla molecola. Di conseguenza, gli spin di questi nuclei compieranno delle oscillazioni a frequenza diversa da quelle degli altri atomi di idrogeno. Eccitando opportunamente un tessuto, è possibile discriminare i suoi vari costituenti sulla base delle caratteristiche biochimiche.

Dalla meccanica quantistica è noto che la transizione tra due stati \(\left| \psi \right\rangle\) e \(\left| \varphi \right\rangle\) possiede un andamento nel piano \(xy\) dato da:

\[\gamma\hslash\cos\left( \beta - \omega_{0}t \right)\]

Dove \(\omega_{0} = \gamma B_{0}\) è detta pulsazione di Larmor.

Le previsioni della meccanica quantistica possono essere descritte in maniera più semplice considerando gli spin orientati in modo casuale. Quando si applica un campo magnetico, gli spin si orientano nella direzione del campo, mentre nel piano \(xy\), traverso all'asse del campo, si instaura un moto di precessione con pulsazione angolare \(\omega_{0} = \gamma B_{0}\). In altre parole, gli spin ruotano intorno all'asse \(z\), individuato dal campo magnetico, in senso orario con frequenza \(2\pi\omega_{0}\).

\begin{figure}
\centering
\includegraphics[width=2.4729in,height=1.38542in,alt={P2775\#yIS1}]{media/6_IntroMRI/image58.pdf}
\caption{Figura .: Orientamento degli spin a causa del campo}
\end{figure}

Questa assunzione, sebbene non sia esatta, consente di descrivere in modo semplice il comportamento degli spin immersi in un campo magnetico, ottenendo gli stessi risultati della meccanica quantistica.

\begin{figure}
\centering
\includegraphics[width=2.60417in,height=2.14214in,alt={P2778\#yIS1}]{media/6_IntroMRI/image59.pdf}\caption{Figura .: Moto di precessione}
\end{figure}

Si suppone di applicare un campo magnetico nella direzione \(z\), convenzionalmente coincidente con l'asse maggiore del tavolo porta-paziente. Si suppone di applicare uno stimolo \(B_{1}\) tale da ruotare il vettore di magnetizzazione all'equilibrio sull'asse \(y\).

\begin{figure}
\centering
\includegraphics[width=6.125in,height=2.40972in,alt={P2781\#yIS1}]{media/6_IntroMRI/image60.pdf}\caption{Figura .: Rotazione del vettore magnetizzazione a opera di uno stimolo esterno}
\end{figure}

Il vettore magnetizzazione globale, somma di tanti momenti magnetici intrinseci, presenta un andamento più complesso di un normale vettore; infatti, le componenti trasversali evolvono con una tempistica diversa dalle componenti longitudinali.

Rimosso lo stimolo, il vettore magnetizzazione torna all'equilibrio termodinamico, emettendo un segnale dato da:

\[s = - \frac{d}{dt}\int_{V}^{}{\overset{\underline{}}{M} \cdot {\overset{\underline{}}{B}}_{RF}dV}\]

Dove \({\overset{\underline{}}{B}}_{RF}\) è il campo che sarebbe erogato dall'antenna ricevente se percorsa da una certa corrente. Trascurando questo termine, il segnale registrato è proporzionale alla devirata del vettore magnetizzazione:

\[s \propto \frac{dM}{dt}\]

Generalmente l'eccitamento è di tipo sinusoidale, dunque:

\[\frac{dM}{dt} = \omega_{0}M = \gamma B_{0}M\]

Per la legge di Curie, scritta in termini di densità protonica:

\[M \simeq \frac{N}{V}\frac{\gamma^{2}\hslash^{2}}{4k_{B}T}B_{0} = \rho\frac{\gamma^{2}\hslash^{2}}{4k_{B}T}B_{0}\]

Il segnale registrato è proporzionale a:

\[s \propto \ \gamma B_{0}M = \rho\frac{\gamma^{3}\hslash^{2}}{4k_{B}T}B_{0}^{2}\]

Tralasciando i termini costante:

\[s \propto \ \rho\frac{\gamma^{3}}{T}B_{0}^{2}\]

\(\rho\) rappresenta il numero di protoni di idrogeno presenti nel volumetto considerato. Dato che il segnale dipende anche da \(B_{0}^{2}\), è fondamentale che il campo \(B_{0}\) sia molto elevato, in modo da ottenere un buon rapporto segnale-rumore o \emph{signal-to-noise ratio} (SNR).

\subsection{Risonanza magnetica come tecnica spettroscopica}\label{risonanza-magnetica-come-tecnica-spettroscopica}

La risonanza magnetica nasce negli anni '50-'60 con applicazioni spettroscopiche. Questa metodica, ancora oggi molto utilizzata, permette di valutare la composizione chimica del materiale irraggiato dal campo magnetico. In campo medico, la spettroscopia è utilizzata per valutare lo stato metabolico di un tessuto.

A causa dell'effetto della schermatura delle macromolecole, i nuclei di idrogeno non appartenenti all'acqua subiscono un campo magnetico diverso da quello esterno. Si assistono, dunque, a più moti di precessione con frequenza di Larmor diverse.

\begin{figure}
\centering
\includegraphics[width=3.67692in,height=3.00926in,alt={P2800\#yIS1}]{media/6_IntroMRI/image61.pdf}\caption{Figura .: Moti di precessione con frequenze diverse}
\end{figure}

L'applicazione dello stimolo non ruoterà tutti i momenti magnetici allo stesso modo. Dunque, il ritorno all'equilibrio produce dei campi magnetici variabili nel tempo, i quali, a loro volta, inducono sulle antenne riceventi delle fem. con differenti contenuti frequenziali.

Il segnale prelevato è proporzionale alla somma dei veri momenti magnetici che procedono introno all'asse \(z\), con diversa frequenza di Larmor, poiché è diverso il campo principale percepito:

\[s \propto \sum_{k}^{}{M(k)\omega(k)\exp\left( - j\omega_{0}kt \right)}\]

Si ha, ovvero, una somma di \(N\) segnali con proprie frequenze, in base alla molecola a cui il nucleo di idrogeno è legato.

Ogni tessuto biologico è caratterizzato da una certa composizione chimica, quindi, determinando la contrazione dei costituenti, come le proteine, è possibile valutare lo stato di salute di un tessuto e la sua attività metabolica.

Nei tessuti, tuttavia, sono presenti un gran numero di molecole, oltre alle proteine o sostanze di interesse come acqua, acidi nucleici ed ecc. Per cui, al fine di aumentare il numero di molecole di cui si vuole effettuare l'imaging è necessario aumentare le dimensioni del volumetto elementare \(dV\).

Nel caso delle immagini funzionali, di conseguenza, l'aumento del SNR comporta un peggioramento della risoluzione spaziale. In questo modo, è possibile distinguere il contenuto metabolico del tessuto dalla restante parte di acqua e altri costituenti.

Per ottenere le immagini, infine, oltre al campo magnetico costante di grande intensità, si applica un campo stazionario variabile lungo una direzione \(x\) o \(y\). Si instaura così un gradiente di campo magnetico che determina la variazione, per ogni punto \(\overset{\underline{}}{r}\) del paziente, la frequenza di precessione con cui si muovono gli spin varia con la posizione:

\[\omega\left( \overset{\underline{}}{r} \right) = \gamma\overset{\underline{}}{B}\left( \overset{\underline{}}{r} \right)\]

Se il gradiente è posizionato lungo \(z\), il campo magnetico è del tipo:

\[B = B_{0} + G_{z}z\]

Con questa soluzione, la frequenza di precessione è data da:

\[f(z) = \frac{\omega(z)}{2\pi} = \frac{\gamma}{2\pi}\left( B_{0} + G_{z}z \right)\]

Si definisce \(\overline{\gamma} = \gamma/2\pi\). Questa quantità, per il nucleo di idrogeno è:

\[\overline{\gamma} = \frac{\gamma}{2\pi} = \frac{2.68 \cdot 10^{8}\ \frac{rad}{Ts}}{2\pi} = 42.6\frac{MHz}{T}\]

Con un campo statico principale di \(1.5\ T\) si ha una frequenza di precessione data da:

\[f_{0} = \overline{\gamma}B_{0} = 42.6\frac{MHz}{T}1.5\ T \simeq 64\ MHz\]

Per un campo di \(3\ T\), risulta invece:

\[f_{0} = \overline{\gamma}B_{0} = 42.6\frac{MHz}{T}3\ T \simeq 128\ MHz\]

Queste frequenze rientrano nello spettro delle onde radio, in particolare, nella banda di frequenze normalmente utilizzate nella trasmissione FM.

L'uso dei gradienti i campo permette di variare la frequenza di precessione con la posizione. In questo modo, è possibile selezionare una sola fetta del corpo del paziente di cui si vuole eseguire l'imaging. Nello specifico, una volta raggiunto l'equilibrio termodinamico, si applica uno stimolo che ribalta il vettore di magnetizzazione che procede a una determinata frequenza.

\subsection{Momento di precessione}\label{momento-di-precessione}

Si considera un singolo spin immerso in un campo magnetico diretto lungo l'asse \(z\). All'equilibrio termico questo spin si allinea lungo l'asse individuato dal campo magnetico.

Dal punto di vista classico, lo spin immerso nel campo magnetico subisce una torsione meccanica, data da:

\[\overset{\underline{}}{\tau} = \overset{\underline{}}{\mu} \times \overset{\underline{}}{B}\]

Dove:

\[\overset{\underline{}}{\tau} = \frac{d\overset{\underline{}}{L}}{dt}\]

\(\overset{\underline{}}{L}\) è il momento angolare, legato al momento magnetico \(\overset{\underline{}}{\mu}\) dal fattore giromagnetico \(\gamma\):

\[\overset{\underline{}}{\mu} = \gamma\overset{\underline{}}{L} \Leftrightarrow \overset{\underline{}}{L} = \frac{1}{\gamma}\overset{\underline{}}{\mu}\]

Sostituendo i due risultati nell'equazione differenziale si ha:

\[\overset{\underline{}}{\mu} \times \overset{\underline{}}{B} = \frac{d}{dt}\left( \frac{1}{\gamma}\overset{\underline{}}{\mu} \right) \Leftrightarrow \frac{d\overset{\underline{}}{\mu}}{dt} = \gamma\overset{\underline{}}{\mu} \times \overset{\underline{}}{B}\]

La descrizione classica, rappresentata dall'equazione differenziale appena individuata, non è precisa, tuttavia, presenta gli stessi risultati della meccanica quantistica. Quest'ultima teoria si basa sui livelli energetici \(\left| + \right\rangle\) e \(\left| - \right\rangle\). Si dimostra che la proiezione dello spin lungo l'asse \(x\) è del tipo:

\[\mu_{x} \propto \cos\left( \omega_{0}t \right)\]

Dove \(\omega_{0} = \gamma B_{0}\) è la frequenza di precessione di Larmor.

L'equazione differenziale:

\[\frac{d\overset{\underline{}}{\mu}}{dt} = \gamma\overset{\underline{}}{\mu} \times \overset{\underline{}}{B}\]

Presenta la stessa soluzione della meccanica quantistica ma con una descrizione semplificata; per tale motivo, si ricorre alla descrizione classica.

Si considera il prodotto scalare tra \(\overset{\underline{}}{\mu}\) e la sua derivata:

\[\overset{\underline{}}{\mu} \cdot \frac{d\overset{\underline{}}{\mu}}{dt} = \gamma\overset{\underline{}}{\mu} \cdot \left( \overset{\underline{}}{\mu} \times \overset{\underline{}}{B} \right)\]

Il vettore \(\overset{\underline{}}{\mu} \times \overset{\underline{}}{B}\) è ortogonale sia al vettore \(\overset{\underline{}}{\mu}\) che \(\overset{\underline{}}{B}\), dunque, il prodotto scalare è nullo:

\[\overset{\underline{}}{\mu} \cdot \frac{d\overset{\underline{}}{\mu}}{dt} = \gamma\overset{\underline{}}{\mu} \cdot \left( \overset{\underline{}}{\mu} \times \overset{\underline{}}{B} \right) = 0\]

Il prodotto scalare tra \(\overset{\underline{}}{\mu}\) e la sua derivata può essere scritto come:

\[\overset{\underline{}}{\mu} \cdot \frac{d\overset{\underline{}}{\mu}}{dt} = \frac{1}{2}\frac{d}{dt}\left( \overset{\underline{}}{\mu} \cdot \overset{\underline{}}{\mu} \right) = \frac{1}{2}\frac{d}{dt}\left| \overset{\underline{}}{\mu} \right|^{2} = 0\]

La derivata del modulo quadro è nulla, dunque, il modulo del momento magnetico intrinseco è costante nel tempo:

\[\left| \overset{\underline{}}{\mu} \right| = const\]

Nonostante il suo modulo sia costante, la fase del momento magnetico decresce costantemente nel tempo, infatti, risulta:

\[\frac{d\varphi}{dt} = - \omega\]

Il momento magnetico precede nel piano \(xy\) in senso orario. La fase può essere scritta come:

\[\varphi(t) = \varphi_{0} - \omega t\]

Dove \(\varphi_{0}\) è la fase iniziale.

\begin{figure}
\centering
\includegraphics[width=2.36664in,height=1.64815in,alt={P2852\#yIS1}]{media/6_IntroMRI/image62.pdf}\caption{Figura .: Verso di rotazione del moto di precessione}
\end{figure}

Si risolve l'equazione differenziale per il momento magnetico. A tale scopo si scompone il prodotto vettoriale \(\overset{\underline{}}{\mu} \times \overset{\underline{}}{B}\) lungo gli assi:

\[\overset{\underline{}}{\mu} \times \overset{\underline{}}{B} = \left| \begin{matrix}
{\widehat{i}}_{x} & {\widehat{i}}_{y} & {\widehat{i}}_{z} \\
\mu_{x} & \mu_{y} & \mu_{z} \\
0 & 0 & B_{0}
\end{matrix} \right| = B_{0}\left( \mu_{y}{\widehat{i}}_{x} - \mu_{x}{\widehat{i}}_{y} \right)\]

L'equazione differenziale, si scrive come:

\[\frac{d\overset{\underline{}}{\mu}}{dt} = \gamma B_{0}\left( \mu_{y}{\widehat{i}}_{x} - \mu_{x}{\widehat{i}}_{y} \right)\]

Scomponendo lungo gli assi, si ha:

\[\left\{ \begin{matrix}
\frac{d\mu_{x}}{dt} = \gamma B_{0}\mu_{y} \\
\frac{d\mu_{y}}{dt} = - \gamma B_{0}\mu_{x} \\
\frac{d\mu_{z}}{dt} = 0
\end{matrix} \right.\ \]

Si deriva la seconda equazione rispetto al tempo:

\[\frac{d^{2}\mu_{y}}{dt^{2}} = - \gamma B_{0}\frac{d\mu_{x}}{dt}\]

Sostituendo la prima equazione, si ha:

\[\frac{d^{2}\mu_{y}}{dt^{2}} = - \gamma^{2}B_{0}^{2}\mu_{y} \Leftrightarrow \frac{d^{2}\mu_{y}}{dt^{2}} + \gamma^{2}B_{0}^{2}\mu_{y} = 0\]

Passando al polinomio associato si ha:

\[\lambda^{2} + \gamma^{2}B_{0}^{2} = 0 \Leftrightarrow \lambda = \pm j\gamma B_{0}\]

Ponendo, \(\omega_{0} = \gamma B_{0}\), la soluzione \(\mu_{y}\) è del tipo:

\[\mu_{x}(t) = A\cos\left( \omega_{0}t \right) + B\sin\left( \omega_{0}t \right)\]

Dove \(A\) e \(B\) sono due costanti di integrazione, ottenute applicando le condizioni iniziali.

Nota l'espressione per \(\mu_{x}(t)\), è possibile ottenere quella di \(\mu_{y}(t)\); dalla prima equazione differenziale, infatti, risulta:

\[\frac{d\mu_{x}}{dt} = \gamma B_{0}\mu_{y} \Leftrightarrow \mu_{y} = \frac{1}{\omega_{0}}\frac{d\mu_{x}}{dt} = \frac{1}{\omega_{0}}\frac{d}{dt}\left\lbrack A\cos\left( \omega_{0}t \right) + B\sin\left( \omega_{0}t \right) \right\rbrack\]

Svolgendo la derivata si ha:

\[\mu_{y}(t) = - A\cos\left( \omega_{0}t \right) + B\sin\left( \omega_{0}t \right)\]

Lungo \(z\), la derivata del momento magnetico è nulla, dunque, \(\mu_{z}\) è costante. Si suppone di conoscere lo stato iniziale del momento:

\[\left\{ \begin{matrix}
\left. \ \mu_{x}(t) \right|_{t = 0} = \mu_{x}(0) \\
\left. \ \mu_{y}(t) \right|_{t = 0} = \mu_{y}(0) \\
\left. \ \mu_{z}(t) \right|_{t = 0} = \mu_{z}(0)
\end{matrix} \right.\ \]

Al fine di ricavare le due costanti di integrazione \(A\) e \(B\), si usano le prime due equazioni:

\[\left\{ \begin{matrix}
\left\lbrack A\cos\left( \omega_{0}t \right) + B\sin\left( \omega_{0}t \right) \right\rbrack_{t = 0} = \mu_{x}(0) \\
\left\lbrack - A\cos\left( \omega_{0}t \right) + B\cos\left( \omega_{0}t \right) \right\rbrack_{t = 0} = \mu_{y}(0)
\end{matrix} \right.\  \Leftrightarrow \left\{ \begin{matrix}
A = \mu_{x}(0) \\
B = \mu_{y}(0)
\end{matrix} \right.\ \]

La soluzione dell'equazione differenziale è, dunque:

\[\left\{ \begin{matrix}
\mu_{x}(t) = \mu_{x}(0)\cos\left( \omega_{0}t \right) + \mu_{y}(0)\sin\left( \omega_{0}t \right) \\
\mu_{y}(t) = - \mu_{x}(0)\cos\left( \omega_{0}t \right) + \mu_{y}(0)\sin\left( \omega_{0}t \right) \\
\mu_{z}(t) = \mu_{z}(0)
\end{matrix} \right.\ \]

È possibile scrivere la soluzione dell'equazione differenziale:

\[\frac{d\overset{\underline{}}{\mu}}{dt} = \gamma\overset{\underline{}}{\mu} \times \overset{\underline{}}{B}\]

in forma compatta introducendo la matrice di rotazione intorno all'asse \(z\):

\[{\overset{\underline{}}{\overset{\underline{}}{R}}}_{z}\left( \omega_{0}t \right) = \begin{pmatrix}
\cos\left( \omega_{0}t \right) & \sin\left( \omega_{0}t \right) & 0 \\
 - \sin\left( \omega_{0}t \right) & \cos\left( \omega_{0}t \right) & 0 \\
0 & 0 & 1
\end{pmatrix}\]

Con questa posizione, il momento magnetico \(\overset{\underline{}}{\mu}\) in funzione del tempo è dato da:

\[\overset{\underline{}}{\mu}(t) = {\overset{\underline{}}{\overset{\underline{}}{R}}}_{z}\left( \omega_{0}t \right)\overset{\underline{}}{\mu}(0)\]

\subsubsection{Rappresentazione complessa}\label{rappresentazione-complessa}

Dato che gli spin dei protoni eseguono un moto di precessione intorno all'asse \(z\) individuato dal campo magnetico, è utile introdurre una notazione utilizzante il piano complesso, in modo da considerare anche il movimento degli spin nel piano trasverso al campo magnetico. Si definisce la grandezza fasoriale \(\mu_{\bot}(t)\) come:

\[\mu_{\bot}(t) = \mu_{x}(t) + j\mu_{y}(t)\]

Il fasore permette di descrivere il movimento nel piano trasverso del momento magnetico nel tempo.

La derivata temporale di un fasore si scrive come:

\[\frac{d\mu_{\bot}}{dt} = - j\omega_{o}\mu_{\bot}\]

Dove \(\omega_{o} = \gamma B_{0}\) è la frequenza di precessione di Larmor. Per definizione di \(\mu_{\bot}(t)\), si ha:

\[\frac{d\mu_{\bot}}{dt} = - j\omega_{o}\mu_{\bot} = - j\omega_{o}\mu_{x}(t) - j\omega_{o}\left\lbrack j\mu_{y}(t) \right\rbrack = - j\omega_{o}\mu_{x}(t) - j\omega_{o}\mu_{y}(t)\]

Inoltre, deve risultare:

\[\frac{d\mu_{\bot}}{dt} = \frac{d\mu_{x}}{dt} + j\frac{d\mu_{y}}{dt}\]

Confrontando le due espressioni risulta che:

\[\frac{d\mu_{x}}{dt} = - j\omega_{o}\mu_{x}(t),\ \ \frac{d\mu_{y}}{dt} = - j\omega_{o}\mu_{y}(t)\]

La notazione fasoriale è più semplice rispetto al sistema di equazioni differenziali. Si integra per variabili separabili l'equazione:

\[\frac{d\mu_{\bot}}{dt} = - j\omega_{o}\mu_{\bot} \Leftrightarrow \frac{1}{\mu_{\bot}}d\mu_{\bot} = - j\omega_{o}\ dt\]

Da cui si ottiene:

\[\mu_{\bot}(t) = \mu_{\bot}(0)\exp\left( - j\omega_{0}t \right)\]

Questa soluzione coincide con quella nel dominio del tempo del vettore momento magnetico, ristretto al piano \(xy\). Il termine esponenziale, \(\exp\left( - j\omega_{0}t \right)\), rappresenta una rotazione in senso orario nel piano complesso o, equivalentemente, intorno all'asse \(z\) individuato dal campo principale. La rotazione nel piano \(xy\) è dovuto alla rotazione del momento magnetico intrinseco dello spin sul piano trasverso al campo applicato.

La fase del fasore è strettamente correlata con la pulsazione con cui il vettore momento magnetico intrinseco ruota nel piano trasverso. La conoscenza della posizione, ovvero della fase, è fondamentale per ricostruire l'immagine di risonanza magnetica. Per cui, data l'importanza della fase, è molto utile introdurre una notazione complessa per tale quantità, in modo da esplicitarla.

In generale, un numero complesso può essere espresso esplicitando modulo e fase:

\[\mu_{\bot}(t) = \left| \mu_{\bot}(t) \right|\exp\left\lbrack - j\varphi(t) \right\rbrack\]

Il modulo del fasore è costante nel tempo, infatti, nel dominio del tempo si è dimostrato che:

\[\overset{\underline{}}{\mu} \cdot \frac{d\overset{\underline{}}{\mu}}{dt} = 0\]

La fase, invece, è data da:

\[\angle\mu_{\bot} = \varphi(t) + \angle\mu(0) = - j\omega_{0}t + \varphi_{0}\]

Dove \(\varphi_{0} = \angle\mu(0)\).

L'evoluzione del fasore \(\mu_{\bot}\) può essere scritto come:

\[\mu_{\bot}(t) = \left| \mu_{\bot}(0) \right|\exp\left\lbrack - j\left( \omega_{0}t + \varphi_{0} \right) \right\rbrack\]

A partire dalla fase \(\varphi_{0}\) è possibile determinare l'evoluzione della posizione dello spin, nota la pulsazione di Larmor \(\omega_{0} = \gamma B_{0}\).

\subsubsection{\texorpdfstring{Sistema di riferimento rotante con velocità angolare \(\mathbf{\omega}\)}{Sistema di riferimento rotante con velocità angolare \textbackslash mathbf\{\textbackslash omega\}}}\label{sistema-di-riferimento-rotante-con-velocituxe0-angolare-mathbfomega}

Applicando un campo magnetico lungo un asse di un sistema di riferimento fisso, gli spin si orientano lungo l'asse individuato dal campo principale, \(z\), compiendo un moto di precessione all'equilibrio termodinamico.

Al fine di ottenere un segnale misurabile è necessario perturbare l'equilibrio termodinamica, mediante un campo elettromagnetico esterno con opportuna frequenza.

Per descrivere al meglio la perturbazione, si introduce un sistema di riferimento rotante con pulsazione \(\omega\) in senso orario interno all'asse \(z\).

Sia \(\left( x',y',z' \right)\) il sistema di riferimento rotante. Nel sistema di riferimento fisso del laboratorio, il sistema rotante presenta una velocità angolare:

\[\overset{\underline{}}{\Omega} = - \omega\widehat{z}\]

Dove il segno meno è dovuto alla rotazione oraria. La velocità di rotazione può variare nel tempo.

Si vuole determinare una relazione che leghi un vettore nel sistema fisso alle componenti del vettore rotante. Ogni vettore \(\overset{\underline{}}{c}\), istantaneamente ruotato con velocità angolare \(\overset{\underline{}}{\Omega}\), presenta una variazione temporale rispetto al sistema di riferimento fisso del laboratorio, descritta dalla relazione:

\[\frac{d\overset{\underline{}}{c}}{dt} = \overset{\underline{}}{\Omega} \times \overset{\underline{}}{c}\]

Il vettore \(\overset{\underline{}}{c}\) può essere espresso come somma della componente parallela all'asse \(z\) e della componente trasversale:

\[\overset{\underline{}}{c} = {\overset{\underline{}}{c}}_{\|} + {\overset{\underline{}}{c}}_{\bot}\]

Il vettore \(\overset{\underline{}}{\Omega}\) ha solo componente lungo \(z\), per cui il prodotto vettorare tra \(\overset{\underline{}}{\Omega}\) e la componente parallela all'asse \(z\) di \(\overset{\underline{}}{c}\), \({\overset{\underline{}}{c}}_{\|}\), è nullo:

\[\overset{\underline{}}{\Omega} \times {\overset{\underline{}}{c}}_{\|} = \overset{\underline{}}{0}\]

Se il vettore \(\overset{\underline{}}{c}\) è fermo nel sistema di riferimento fisso, la sua derivata temporale è nulla:

\[\frac{d\overset{\underline{}}{c}}{dt} = \overset{\underline{}}{0}\]

Nel sistema di riferimento rotante, il vettore \(\overset{\underline{}}{c}\) non è fermo; infatti, questo sistema vede il vettore ruotare, per cui risulta:

\[\left( \frac{d\overset{\underline{}}{c}}{dt} \right)' = \overset{\underline{}}{\Omega} \times {\overset{\underline{}}{c}}' \neq \overset{\underline{}}{0}\]

Dove \(\left( d\overset{\underline{}}{c}/dt \right)'\) indica la derivata del vettore \(\overset{\underline{}}{c}\) nel sistema di riferimento rotante \(x'y'\).

Nel sistema di riferimento fisso, il vettore \(\overset{\underline{}}{c}\) ha delle componenti:

\[\overset{\underline{}}{c}(t) = c_{x}(t){\widehat{i}}_{x} + c_{y}(t){\widehat{i}}_{y} + c_{z}(t){\widehat{i}}_{z}\]

Il tempo è considerato invariante nei due sistemi di riferimento poiché le velocità considerate sono molto minore della velocità della luce. Nel sistema rotante, invece, il vettore \(\overset{\underline{}}{c}\) possiede delle componenti diversi:

\[{\overset{\underline{}}{c}}'(t) = c_{x'}(t){\widehat{i}}_{x'} + c_{y'}(t){\widehat{i}}_{y'} + c_{z'}(t){\widehat{i}}_{z'}\]

Dato che i due sistemi possiedono l'asse \(z\) in comune, deve risultare \({\widehat{i}}_{z} \equiv {\widehat{i}}_{z'}\), per cui:

\[{\overset{\underline{}}{c}}'(t) = c_{x'}(t){\widehat{i}}_{x'} + c_{y'}(t){\widehat{i}}_{y'} + c_{z}(t){\widehat{i}}_{z}\]

Si osserva che non è necessario distinguere \(\overset{\underline{}}{c}(t)\) nel sistema fisso e \({\overset{\underline{}}{c}}'(t)\) nel sistema rotante in quanto il vettore è lo stesso per entrambi i sistemi di riferimento.

Le componenti del vettore nel sistema fisso dipendono da quelle nel sistema rotante e viceversa. Si calcola la derivata rispetto al tempo del vettore \(\overset{\underline{}}{c}(t)\) nel sistema fisso di riferimento:

\[\frac{d\overset{\underline{}}{c}}{dt} = \frac{d}{dt}\left\lbrack c_{x}(t){\widehat{i}}_{x} + c_{y}(t){\widehat{i}}_{y} + c_{z}(t){\widehat{i}}_{z} \right\rbrack = \frac{dc_{x}}{dt}{\widehat{i}}_{x} + c_{x}\frac{d{\widehat{i}}_{x}}{dt} + \frac{dc_{y}}{dt}{\widehat{i}}_{y} + c_{y}\frac{d{\widehat{i}}_{y}}{dt} + \frac{dc_{z}}{dt}{\widehat{i}}_{z} + c_{z}\frac{d{\widehat{i}}_{z}}{dt}\]

Siccome il sistema di riferimento è fisso, le derivate dei suoi versori sono nulle:

\[\frac{d{\widehat{i}}_{x}}{dt} = 0,\ \ \frac{d{\widehat{i}}_{y}}{dt} = 0,\ \ \frac{d{\widehat{i}}_{z}}{dt} = 0\]

Si ottiene:

\[\frac{d\overset{\underline{}}{c}}{dt} = \frac{dc_{x}}{dt}{\widehat{i}}_{x} + \frac{dc_{y}}{dt}{\widehat{i}}_{y} + \frac{dc_{z}}{dt}{\widehat{i}}_{z}\]

La derivata del sistema rotante è, invece:

\[\frac{d\overset{\underline{}}{c}}{dt} = \frac{dc_{x'}}{dt}{\widehat{i}}_{x'} + c_{x'}\frac{d{\widehat{i}}_{x'}}{dt} + \frac{dc_{y'}}{dt}{\widehat{i}}_{y'} + c_{y'}\frac{d{\widehat{i}}_{y'}}{dt} + \frac{dc_{z'}}{dt}{\widehat{i}}_{z'} + c_{z'}\frac{d{\widehat{i}}_{z'}}{dt}\]

La relazione \(d\overset{\underline{}}{c}/dt = \overset{\underline{}}{\Omega} \times \overset{\underline{}}{c}\) è valida per un generico vettore \(\overset{\underline{}}{c}\), dunque, è valida anche per i versori degli assi coordinati:

\[\frac{d{\widehat{i}}_{x'}}{dt} = \overset{\underline{}}{\Omega} \times {\widehat{i}}_{x'},\ \ \frac{d{\widehat{i}}_{y'}}{dt} = \overset{\underline{}}{\Omega} \times {\widehat{i}}_{y'},\ \ \frac{d{\widehat{i}}_{z'}}{dt} = \overset{\underline{}}{\Omega} \times {\widehat{i}}_{z'}\]

Per cui la derivata nel sistema rotante può essere scritta sostituendo queste relazioni:

\[\frac{d\overset{\underline{}}{c}}{dt} = \frac{dc_{x'}}{dt}{\widehat{i}}_{x'} + c_{x'}\overset{\underline{}}{\Omega} \times {\widehat{i}}_{x'} + \frac{dc_{y'}}{dt}{\widehat{i}}_{y'} + c_{y'}\overset{\underline{}}{\Omega} \times {\widehat{i}}_{y'} + \frac{dc_{z'}}{dt}{\widehat{i}}_{z'} + c_{z'}\overset{\underline{}}{\Omega} \times {\widehat{i}}_{z'} = \frac{dc_{x'}}{dt}{\widehat{i}}_{x'} + \frac{dc_{y'}}{dt}{\widehat{i}}_{y'} + \frac{dc_{z'}}{dt}{\widehat{i}}_{z'} + \overset{\underline{}}{\Omega} \times \left( c_{x'}{\widehat{i}}_{x'} + c_{y'}{\widehat{i}}_{y'} + c_{z'}{\widehat{i}}_{z'} \right)\]

Dove \({\overset{\underline{}}{c}}' = c_{x'}{\widehat{i}}_{x'} + c_{y'}{\widehat{i}}_{y'} + c_{z'}{\widehat{i}}_{z'}\) è il vettore nel sistema di riferimento fisso. Si indica con:

\[\left( \frac{d\overset{\underline{}}{c}}{dt} \right)' = \frac{dc_{x'}}{dt}{\widehat{i}}_{x'} + \frac{dc_{y'}}{dt}{\widehat{i}}_{y'} + \frac{dc_{z'}}{dt}{\widehat{i}}_{z'}\]

La derivata del vettore \(\overset{\underline{}}{c}\) nel sistema di riferimento rotante, i cui versori sono fissi. La derivata nel sistema di riferimento rotante può essere scritta come:

\[\frac{d\overset{\underline{}}{c}}{dt} = \left( \frac{d\overset{\underline{}}{c}}{dt} \right)' + \overset{\underline{}}{\Omega} \times {\overset{\underline{}}{c}}'\]

La derivata del vettore \(\overset{\underline{}}{c}\) nel sistema rotante è data dalla variazione nel tempo del vettore \(\overset{\underline{}}{c}\) nel sistema rotante a cui si aggiunge un termine che tiene conto della rotazione del sistema.

Si considera, come vettore \(\overset{\underline{}}{c}\), il momento magnetico di uno spin immerso in un campo magnetico \(\overset{\underline{}}{B}\) uniforme e diretto lungo \(z\):

\[\frac{d\overset{\underline{}}{\mu}}{dt} = \left( \frac{d\overset{\underline{}}{\mu}}{dt} \right)' + \overset{\underline{}}{\Omega} \times \overset{\underline{}}{\mu}\]

La variazione del momento magnetico è legata al campo magnetico dalla relazione:

\[\frac{d\overset{\underline{}}{\mu}}{dt} = \gamma\overset{\underline{}}{\mu} \times \overset{\underline{}}{B}\]

Sostituendo questo risultato nella relazione del sistema rotante, si ha:

\[\gamma\overset{\underline{}}{\mu} \times \overset{\underline{}}{B} = \left( \frac{d\overset{\underline{}}{\mu}}{dt} \right)' + \overset{\underline{}}{\Omega} \times \overset{\underline{}}{\mu}\]

Ricavando la derivata nel sistema rotante si ha:

\[\left( \frac{d\overset{\underline{}}{\mu}}{dt} \right)' = \gamma\overset{\underline{}}{\mu} \times \overset{\underline{}}{B} - \overset{\underline{}}{\Omega} \times \overset{\underline{}}{\mu}\]

È valido il risultato \(- \overset{\underline{}}{\Omega} \times \overset{\underline{}}{\mu} = \overset{\underline{}}{\mu} \times \overset{\underline{}}{\Omega}\). Sostituendo si ha:

\[\left( \frac{d\overset{\underline{}}{\mu}}{dt} \right)' = \gamma\overset{\underline{}}{\mu} \times \overset{\underline{}}{B} + \overset{\underline{}}{\mu} \times \overset{\underline{}}{\Omega} = \gamma\overset{\underline{}}{\mu} \times \left( \overset{\underline{}}{B} + \frac{1}{\gamma}\overset{\underline{}}{\Omega} \right)\]

Si definisce campo magnetico effettivo o efficace come:

\[{\overset{\underline{}}{B}}_{eff} = \overset{\underline{}}{B} + \frac{1}{\gamma}\overset{\underline{}}{\Omega}\]

L'equazione per il momento magnetico nel sistema rotante assume la forma:

\[\left( \frac{d\overset{\underline{}}{\mu}}{dt} \right)' = \gamma\overset{\underline{}}{\mu} \times {\overset{\underline{}}{B}}_{eff}\]

Si ottiene la stessa forma dell'equazione ricavata per il sistema fisso, a patto di sostituire il campo magnetico principale \(\overset{\underline{}}{B}\) applicato, il campo effettivamente visto nel sistema rotante \({\overset{\underline{}}{B}}_{eff}\), il quale prevede che, al campo magnetico applicato nel sistema rotante, si aggiunge un campo fittizio, dovuta alla rotazione \(\overset{\underline{}}{\Omega}\) del sistema. Si suppone che il sistema ruoti intorno all'asse \(z\) in senso orario con velocità \(\omega\):

\[\overset{\underline{}}{\Omega} = - \omega{\widehat{i}}_{z}\]

In queste condizioni, il campo efficace è:

\[{\overset{\underline{}}{B}}_{eff} = B_{0}{\widehat{i}}_{z} - \frac{1}{\gamma}\omega{\widehat{i}}_{z}\]

Se la frequenza con cui ruota il sistema coincide con quella di Larmor, ovvero \(\omega = \omega_{0} = \gamma B_{0}\), allora:

\[{\overset{\underline{}}{B}}_{eff} = B_{0}{\widehat{i}}_{z} - \frac{1}{\gamma}\omega_{0}{\widehat{i}}_{z} = B_{0}{\widehat{i}}_{z} - \frac{1}{\gamma}\gamma B_{0}{\widehat{i}}_{z} = \overset{\underline{}}{0}\]

Se la velocità di rotazione è uguale alla frequenza di Larmor, il momento magnetico \(\overset{\underline{}}{\mu}\) appare fisso nel sistema di riferimento rotante. Ne discende che questo sistema di riferimento è solidale con gli spin.

L'introduzione del sistema rotante permette di descrivere in modo semplice il segnale registrato. Nella pratica non è semplice produrre un campo con la stessa frequenza con cui risuonano i protoni, quindi, il campo non è mai nullo.

\subsubsection{Rotazione del sistema di riferimento per un campo a radiofrequenza}\label{rotazione-del-sistema-di-riferimento-per-un-campo-a-radiofrequenza}

Si considera un protone allineato al campo magnetico esterno, diretto lungo l'asse \(z\). Il protone risuona o precede alla frequenza di Larmor nel piano trasverso. Al fine di eccitare lo spin per eseguire la misura, bisogna stimarlo mediante un campo elettromagnetico alla frequenza di Larmor. Questo campo magnetico, essendo oscillante, possiede una polarizzazione.

Si suppone che la polarizzazione del campo eccitante sia lineare, ovvero che il campo sia diretto solamente lungo un asse, ad esempio \(x\). Il campo possiede, dunque, un andamento del tipo:

\[{\overset{\underline{}}{B}}_{x}(t) = b_{x}(t)\cos\left( \omega_{0}t \right){\widehat{i}}_{x}\]

Nell'espressione vi è la dipendenza addizionale dell'addizionale dell'ampiezza delle oscillazioni \(b_{x}(t)\), poiché il segnale emanato è, solitamente, un pacchetto di sinusoidi.

L'uso del sistema rotante rende la descrizione del campo magnetico a cui è soggetto lo spin più semplice. Infatti, in questo sistema il campo può essere descritto semplicemente osservando che i versori del sistema rotante sono legati a quelli del sistema fisso dalle relazioni:

\[\left\{ \begin{matrix}
{\widehat{i}}_{x'} = {\widehat{i}}_{x}\cos\left( \omega_{0}t \right) - {\widehat{i}}_{y}\sin\left( \omega_{0}t \right) \\
{\widehat{i}}_{y'} = {\widehat{i}}_{x}\sin\left( \omega_{0}t \right) + {\widehat{i}}_{y}\cos\left( \omega_{0}t \right)
\end{matrix} \right.\ \]

Si scrive il sistema in forma matriciale, in modo poter ricavare i versori del sistema fisso in funzione di quelli del sistema rotante.

\[\begin{pmatrix}
{\widehat{i}}_{x'} \\
{\widehat{i}}_{y'}
\end{pmatrix} = \begin{pmatrix}
\cos\left( \omega_{0}t \right) & - \sin\left( \omega_{0}t \right) \\
\sin\left( \omega_{0}t \right) & \cos\left( \omega_{0}t \right)
\end{pmatrix}\begin{pmatrix}
{\widehat{i}}_{x} \\
{\widehat{i}}_{y}
\end{pmatrix}\]

Dove:

\[R_{z}(\omega t) = \begin{pmatrix}
\cos\left( \omega_{0}t \right) & - \sin\left( \omega_{0}t \right) \\
\sin\left( \omega_{0}t \right) & \cos\left( \omega_{0}t \right)
\end{pmatrix}\]

È la matrice di rotazione intorno all'asse \(z\). Il suo determinante è unitario:

\[\det{R_{z}\left( \omega_{0}t \right)} = \left| \begin{matrix}
\cos\left( \omega_{0}t \right) & - \sin\left( \omega_{0}t \right) \\
\sin\left( \omega_{0}t \right) & \cos\left( \omega_{0}t \right)
\end{matrix} \right| = \cos^{2}\left( \omega_{0}t \right) + \sin^{2}\left( \omega_{0}t \right) = 1\]

\(R_{z}\left( \omega_{0}t \right)\) è invertibile:

\[R_{z}^{- 1}\left( \omega_{0}t \right) = \begin{pmatrix}
\cos\left( \omega_{0}t \right) & \sin\left( \omega_{0}t \right) \\
 - \sin\left( \omega_{0}t \right) & \cos\left( \omega_{0}t \right)
\end{pmatrix}\]

Per cui la relazione matriciale invertite è:

\[\begin{pmatrix}
{\widehat{i}}_{x} \\
{\widehat{i}}_{y}
\end{pmatrix} = \begin{pmatrix}
\cos\left( \omega_{0}t \right) & \sin\left( \omega_{0}t \right) \\
 - \sin\left( \omega_{0}t \right) & \cos\left( \omega_{0}t \right)
\end{pmatrix}\begin{pmatrix}
{\widehat{i}}_{x'} \\
{\widehat{i}}_{y'}
\end{pmatrix}\]

Il versore relativo all'asse \(x\) del sistema fisso, in funzione dei versori del sistema rotante, è:

\[{\widehat{i}}_{x} = {\widehat{i}}_{x'}\cos\left( \omega_{0}t \right) + {\widehat{i}}_{y'}\sin\left( \omega_{0}t \right)\]

Il campo magnetico polarizzato linearmente nel sistema rotante può essere scritto come:

\[{\overset{\underline{}}{B}}_{x}(t) = b_{x}(t)\cos\left( \omega_{0}t \right){\widehat{i}}_{x} = b_{x}(t)\cos\left( \omega_{0}t \right)\left\lbrack {\widehat{i}}_{x'}\cos\left( \omega_{0}t \right) + {\widehat{i}}_{y'}\sin\left( \omega_{0}t \right) \right\rbrack = b_{x}(t)\cos^{2}\left( \omega_{0}t \right){\widehat{i}}_{x'} + b_{x}(t)\cos\left( \omega_{0}t \right)\sin\left( \omega_{0}t \right){\widehat{i}}_{y'}\]

Per le formule di duplicazione e addizione, si ottiene:

\[{\overset{\underline{}}{B}}_{x}(t) = b_{x}(t)\left\lbrack \frac{1}{2}\cos\left( 2\omega_{0}t \right) + \frac{1}{2} \right\rbrack{\widehat{i}}_{x'} + \frac{1}{2}b_{x}(t)\sin\left( 2\omega_{0}t \right){\widehat{i}}_{y'}\]

Nel sistema rotante il campo magnetico lineare può essere scritto come somma di una costante con due campi a frequenza doppia di quella impostata:

\[{\overset{\underline{}}{B}}_{x}(t) = \frac{1}{2}b_{x}(t){\widehat{i}}_{x'} + b_{x}(t)\left\lbrack \frac{1}{2}\cos\left( 2\omega_{0}t \right){\widehat{i}}_{x'} + \frac{1}{2}\sin\left( 2\omega_{0}t \right){\widehat{i}}_{y'} \right\rbrack\]

Si calcola il valor medio del campo magnetico su un intervallo di tempo sufficientemente lungo, come un periodo dell'onda a radiofrequenza \(T\):

\[\left\langle {\overset{\underline{}}{B}}_{x}(t) \right\rangle = \frac{1}{T}\int_{T}^{}{{\overset{\underline{}}{B}}_{x}(t)dt} = \frac{1}{2T}\int_{T}^{}{\left| \left\{ b_{x}(t){\widehat{i}}_{x'} + b_{x}(t)\left\lbrack \cos\left( 2\omega_{0}t \right){\widehat{i}}_{x'} + \sin\left( 2\omega_{0}t \right){\widehat{i}}_{y'} \right\rbrack \right\} \right|dt} = \frac{1}{2}\left\langle b_{x}(t) \right\rangle\]

Gli altri termini sono nulli poiché termini sinusoidali integrati su un intervallo temporale uguale al doppio del periodo di oscillazione. L'applicazione del valor medio implica che solo metà della polarizzazione lineare è utilizzata per ruotare gli spin intorno all'asse \(x\). Per un tempo sufficientemente lungo il campo può essere considerato a media costante.

In molte applicazioni, per rendere più precisa la ricostruzione delle immagini, si instaura un campo magnetico con polarizzazione circolare, ottenuto sovrapponendo due campi lineari di ugual intensità, in quadratura e diretti lungo due assi ortogonali.

\begin{figure}
\centering
\includegraphics[width=2.96296in,height=1.78624in,alt={P3006\#yIS1}]{media/6_IntroMRI/image63.pdf}\caption{Figura .: Sovrapposizione di onde ortogonali}
\end{figure}

A tale scopo si posizionano due antenne ortogonali tra loro, ognuna delle quali eroga un campo magnetico lineare. Affinché la polarizzazione sia circolare è necessario che i due campi siano in quadratura tra loro. Il campo totale, polarizzato circolarmente, è espresso, nel sistema di riferimento fisso, come:

\[\overset{\underline{}}{B}(t) = B_{1}\left\lbrack \cos\left( \omega_{0}t \right){\widehat{i}}_{x} - \sin\left( \omega_{0}t \right){\widehat{i}}_{x} \right\rbrack\]

Si ricava l'espressione del campo magnetico polarizzato circolarmente nel sistema di riferimento rotante alla frequenza di Larmor. A tale scopo, si considerano le relazioni tra i versori dei due sistemi di riferimento:

\[\begin{pmatrix}
{\widehat{i}}_{x} \\
{\widehat{i}}_{y}
\end{pmatrix} = \begin{pmatrix}
\cos\left( \omega_{0}t \right) & \sin\left( \omega_{0}t \right) \\
 - \sin\left( \omega_{0}t \right) & \cos\left( \omega_{0}t \right)
\end{pmatrix}\begin{pmatrix}
{\widehat{i}}_{x'} \\
{\widehat{i}}_{y'}
\end{pmatrix} \Leftrightarrow \left\{ \begin{matrix}
{\widehat{i}}_{x} = \cos\left( \omega_{0}t \right){\widehat{i}}_{x'} + \sin\left( \omega_{0}t \right){\widehat{i}}_{y'} \\
{\widehat{i}}_{y} = - \sin\left( \omega_{0}t \right){\widehat{i}}_{x'} + \cos\left( \omega_{0}t \right){\widehat{i}}_{y'}
\end{matrix} \right.\ \]

Sostituendo queste relazioni nell'espressione del campo magnetico circolare si ha:

\[\overset{\underline{}}{B}(t) = B_{1}\left\lbrack \cos\left( \omega_{0}t \right){\widehat{i}}_{x} - \sin\left( \omega_{0}t \right){\widehat{i}}_{x} \right\rbrack = B_{1}\left\{ \cos\left( \omega_{0}t \right)\left\lbrack \cos\left( \omega_{0}t \right){\widehat{i}}_{x'} + \sin\left( \omega_{0}t \right){\widehat{i}}_{y'} \right\rbrack - \sin\left( \omega_{0}t \right)\left\lbrack - \sin\left( \omega_{0}t \right){\widehat{i}}_{x'} + \cos\left( \omega_{0}t \right){\widehat{i}}_{y'} \right\rbrack \right\} = B_{1}\left\{ \cos^{2}\left( \omega_{0}t \right){\widehat{i}}_{x'} + \cos\left( \omega_{0}t \right)\sin\left( \omega_{0}t \right){\widehat{i}}_{y'} + \sin^{2}\left( \omega_{0}t \right){\widehat{i}}_{x'} - \sin\left( \omega_{0}t \right)\cos\left( \omega_{0}t \right){\widehat{i}}_{y'} \right\} = B_{1}\left\lbrack \cos^{2}\left( \omega_{0}t \right){\widehat{i}}_{x'} + \sin^{2}\left( \omega_{0}t \right){\widehat{i}}_{x'} \right\rbrack = B_{1}{\widehat{i}}_{x'}\]

Per ogni istante temporale, il campo magnetico a polarizzazione circolare con frequenza uguale a quella del sistema rotante è orientato lungo l'asse \(x'\); inoltre, a differenza del caso lineare, la relazione è valida per ogni istante di tempo poiché non è ottenuta mediante operazione di media.

Riassumendo, quando si applica l'impulso a radiofrequenza con polarizzazione circolare, gli spin ruotano intorno all'asse \(x'\), lungo cui iniziano un moto di precessione.

È possibile dimostrare che si ha un minor dispendio energetico per generare un campo a polarizzazione circolare. A causa di ciò, corredati ad altri fattori riguardanti il rapporto segnale/rumore e la necessità di omogeneizzare il campo a radiofrequenza, i campi rotanti a polarizzazione circolare sono molto usati in risonanza magnetica.

\paragraph{Condizione di risonanza}\label{condizione-di-risonanza}

Si applica un campo a radiofrequenza con polarizzazione lineare o circolare; nel primo caso bisogna considerare quantità medie, mentre nel secondo si hanno relazioni esatte dal punto di vista teorico. Nel sistema rotante il campo magnetico subito da uno spin è detto campo effettivo.

Prima dell'applicazione dell'impulso, l'equazione che descrive il comportamento dello spin, dal punto di vista classico, nel sistema rotante è:

\[\left( \frac{d\overset{\underline{}}{\mu}}{dt} \right)' = \gamma\overset{\underline{}}{\mu} \times {\overset{\underline{}}{B}}_{eff}\]

Dove:

\[{\overset{\underline{}}{B}}_{eff} = B_{0}{\widehat{i}}_{z} - \frac{1}{\gamma}\omega{\widehat{i}}_{z}\]

\(B_{0}{\widehat{i}}_{z}\) è il campo statico esterno o principale applicato.

Si applica, ora, un campo a radiofrequenza. Se il sistema di riferimento ruota con la stessa pulsazione \(\omega_{0}\) del campo magnetico applicato, al campo effettivo, \({\overset{\underline{}}{B}}_{eff}\), va aggiunto un termine costante, diretto lungo \({\widehat{i}}_{x'}\):

\[\left( \frac{d\overset{\underline{}}{\mu}}{dt} \right)' = \gamma\overset{\underline{}}{\mu} \times \left( B_{0}{\widehat{i}}_{z} - \frac{1}{\gamma}\omega{\widehat{i}}_{z} + B_{1}{\widehat{i}}_{x'} \right)\]

I due sistemi di riferimento possiedono gli assi \(z\) paralleli, per cui \({\widehat{i}}_{z} = {\widehat{i}}_{z'}\).

Si dice condizione di risonanza se la frequenza dell'impulso a radiofrequenza è uguale alla frequenza di Larmor degli spin contenuti in uno strato di tessuto:

\[\omega = \omega_{0} = \gamma B_{0}\]

In questo caso, risulta:

\[\left( \frac{d\overset{\underline{}}{\mu}}{dt} \right)' = \gamma B_{1}\overset{\underline{}}{\mu} \times {\widehat{i}}_{x'}\]

Si assiste, dunque, a una processione degli spin intorno all'asse \({\widehat{i}}_{x}'\)

\begin{figure}
\centering
\includegraphics[width=4.4166in,height=3.61111in,alt={P3032\#yIS1}]{media/6_IntroMRI/image64.pdf}\caption{Figura .: Precessione nel sistema rotante in condizione di risonanza}
\end{figure}

Il termine risonanza, in questa tecnica di imaging, è legato al fatto che, per ottenere l'immagine, il campo magnetico applicato è sintonizzato con la frequenza degli spin in moto di precessione intorno al campo magnetico applicato.

L'introduzione del sistema rotante consente di descrivere in modo semplice il comportamento degli spin, mediante un moto di precessione intorno all'asse \(x'\). Nel sistema di riferimento fisso del laboratorio, invece, il contributo del campo a radiofrequenza si somma a quello stato e omogeneo, diretto lungo \({\widehat{i}}_{z}\). Si dimostra che la combinazione dei due campi produce un modo elicoidale, il cui raggio aumento all'avvicinarsi del piano \(xy\).

\begin{figure}
\centering
\includegraphics[width=4.13528in,height=2.63095in,alt={P3036\#yIS1}]{media/6_IntroMRI/image65.pdf}\caption{Figura .: Moto elicoidale nel sistema fisso}
\end{figure}

Nel sistema rotante con pulsazione \(\omega\), è possibile scrivere il campo efficace come:

\[{\overset{\underline{}}{B}}_{eff} = B_{0}{\widehat{i}}_{z} - \frac{1}{\gamma}\omega{\widehat{i}}_{z} + B_{1}{\widehat{i}}_{x'}\]

Con pulsazione di Larmor:

\[\omega = \gamma B_{0} \Leftrightarrow B_{0} = \frac{1}{\gamma}\omega_{0}\]

La pulsazione del campo a radiofrequenza applicato è data da:

\[\omega_{1} = \gamma B_{1} \Leftrightarrow B_{1} = \frac{1}{\gamma}\omega_{1}\]

Per cui è possibile scrivere:

\[{\overset{\underline{}}{B}}_{eff} = \frac{1}{\gamma}\left\lbrack \left( \omega_{0} - \omega \right){\widehat{i}}_{z} + \omega_{1}{\widehat{i}}_{x'} \right\rbrack\]

In generale, avere un campo a radiofrequenza con stessa frequenza di Larmor, con cui precedono gli spin, è abbastanza complesso per la schermatura che alcune molecole compiono nei confronti dei campi magnetici. Inoltre, in linea di principio la frequenza con cui ruota il sistema rotante, in senso orario, è scelta arbitrariamente rispetto sia al campo a radiofrequenza che alla frequenza di precessione degli spin. Nel caso generale, si ha la stessa complessità della descrizione del sistema fisso. I vantaggi del sistema rotante si mostrano quando \(\omega_{1} = \omega_{0} = \omega\), poiché il vettore momento magnetico ruota intorno all'asse \(x'\).

Generalmente gli impulsi a radiofrequenza hanno ampiezza di qualche \(mT\) o \(\mu T\). Si suppone, infatti, di applicare un impulso a radiofrequenza per un tempo \(\tau = 1\ ms\), così da far ruotare di \(\pi/2\) il campo momento magnetico. L'ampiezza del campo a radiofrequenza necessaria a tale scopo è data da:

\[\mathrm{\Delta}\vartheta = \gamma B_{1}\tau\]

Dove \(\gamma = 42.6\ MHz/T\); si ottiene:

\[B_{1} = \frac{\mathrm{\Delta}\vartheta}{\gamma\tau} = \frac{\frac{\pi}{2}}{42.6\ \frac{MHz}{T}1\ ms} \simeq 5.9\ \mu T\]

Il ribaltamento degli spin a opera del campo magnetico a radiofrequenza è detto in gergo to flip.

La maggior difficoltà pratica e tecnica della risonanza magnetica consiste nella produzione del campo magnetico principale, costante nel tempo ed omogeno nello spazio.

I campi a radiofrequenza con frequenza di Larmor riescono a flippare gli spin anche avendo un intensità molto minore del campo principale; tuttavia, più i campo a radiofrequenza si allontana dalla frequenza di precessione di Larmor e più il campo effettivo nel sistema rotante tende a quello stato, ovvero, è minore il numero degli spin ruotati di \(\mathrm{\Delta}\vartheta\). Le frequenze degli impulsi devono essere compatibili con i tempi di rilassamento del corpo in esame.

\paragraph{Calcolo del campo magnetico a radiofrequenza}\label{calcolo-del-campo-magnetico-a-radiofrequenza}

L'uso del sistema rotante intorno all'asse \({\widehat{i}}_{z}\) permette di descrivere un campo a polarizzazione circolare in modo molto semplice. Inoltre, se la frequenza del campo a radiofrequenza e quella del sistema rotante sono uguali a quella di Larmor, è possibile scrivere la relazione:

\[\left( \frac{d\overset{\underline{}}{\mu}}{dt} \right)' = \gamma B_{1}\overset{\underline{}}{\mu} \times {\widehat{i}}_{x'}\]

Dove \({\overset{\underline{}}{B}}_{1} = B_{1}{\widehat{i}}_{x'}\) è il campo magnetico polarizzato circolarmente visto nel sistema rotante. La rotazione intorno all'asse \({\widehat{i}}_{z}\), nel sistema fisso, si scrive come:

\[\overset{\underline{}}{\mu}(t) = {\overset{\underline{}}{\overset{\underline{}}{R}}}_{z}(\omega t)\overset{\underline{}}{\mu}(0)\]

Dove \({\overset{\underline{}}{\overset{\underline{}}{R}}}_{z}\) è la matrice di rotazione intorno all'asse \({\widehat{i}}_{z}\):

\[{\overset{\underline{}}{\overset{\underline{}}{R}}}_{z} = \begin{pmatrix}
\cos\left( \omega_{0}t \right) & - \sin\left( \omega_{0}t \right) & 0 \\
\sin\left( \omega_{0}t \right) & \cos\left( \omega_{0}t \right) & 0 \\
0 & 0 & 1
\end{pmatrix}\]

Invece di ruotare il sistema lungo l'asse \({\widehat{i}}_{z}\), si suppone che la rotazione avvenga intorno all'asse \({\widehat{i}}_{y}\). In questo caso, la soluzione si scrive come:

\[\overset{\underline{}}{\mu}(t) = {\overset{\underline{}}{\overset{\underline{}}{R}}}_{y}(\omega t)\overset{\underline{}}{\mu}(0)\]

Dove:

\[{\overset{\underline{}}{\overset{\underline{}}{R}}}_{y} = \begin{pmatrix}
\cos\left( \omega_{0}t \right) & 0 & - \sin\left( \omega_{0}t \right) \\
0 & 1 & 0 \\
\sin\left( \omega_{0}t \right) & 0 & \cos\left( \omega_{0}t \right)
\end{pmatrix}\]

Con rotazione del sistema attorno all'asse \({\widehat{i}}_{x}\), la soluzione è:

\[\overset{\underline{}}{\mu}(t) = {\overset{\underline{}}{\overset{\underline{}}{R}}}_{x}(\omega t)\overset{\underline{}}{\mu}(0)\]

Con:

\[{\overset{\underline{}}{\overset{\underline{}}{R}}}_{z} = \begin{pmatrix}
1 & 0 & 0 \\
0 & \cos\left( \omega_{0}t \right) & - \sin\left( \omega_{0}t \right) \\
0 & \sin\left( \omega_{0}t \right) & \cos\left( \omega_{0}t \right)
\end{pmatrix}\]

L'angolo di cui ruota il sistema di riferimento rotante rispetto a quello fisso, fissato un istante temporale \(\tau\), è dato da:

\[\phi = \omega_{1}\tau\]

\(\omega_{1}\) è legato al campo a radiofrequenza applicato dalla relazione:

\[\omega_{1} = \gamma B_{1}\]

Con \(B_{1}\) costante. L'angolo di cui ruota il sistema può essere scritto come:

\[\phi = \gamma B_{1}\tau\]

Nel caso generale, in cui l'ampiezza del campo a radiofrequenza cambia nel tempo, ovvero \(B_{1} = B_{1}(t)\), la fase all'istante fissato \(\tau\), è ottenuto come integrale temporale, infatti:

\[\frac{d\phi}{dt} = \omega_{1} \Leftrightarrow \phi = \int_{t_{0}}^{\tau}{\omega_{1}dt}\]

Per il legame tra pulsazione del sistema rotante e campo a radiofrequenza applicato, si ha:

\[\phi = \int_{t_{0}}^{\tau}{\gamma B_{1}dt}\]

Questa generalizzazione è necessaria nel caso in cui il campo polarizzato circolarmente sia applicato per un intervallo temporale finito e abbia ampiezza variabile nel tempo. L'uso dell'integrale, quindi, permette di ottenere una descrizione più generale, applicabile anche a campi di ampiezza variabile nel tempo.

Si suppone di applicare un campo a radiofrequenza \(B_{1}\), diretto lungo \({\widehat{i}}_{x'}\), tale da far ruotare gli spin di un angolo \(\vartheta\) rispetto all'asse \({\widehat{i}}_{x'}\). Si applica, non appena viene interrotta la trasmissione del primo campo, un secondo impulso a radiofrequenza, diretto lungo l'asse \({\widehat{i}}_{y'}\). La descrizione del moto di precessione, nel sistema di riferimento fisso, in questo scenario, è dato da:

\[\overset{\underline{}}{\mu}(t) = {\overset{\underline{}}{\overset{\underline{}}{R}}}_{y}{\overset{\underline{}}{\overset{\underline{}}{R}}}_{x}(\omega t)\overset{\underline{}}{\mu}(0)\]

\subsection{Vettore di magnetizzazione}\label{vettore-di-magnetizzazione}

Per ottenere un'immagine di una sezione del corpo umano mediante il fenomeno della risonanza magnetica, si suddivide il corpo del paziente in volumetti elementi contenenti un numero di Avogadro di protoni che precedono alla frequenza di Larmor.

Dato l'elevato numero di spin si introduce il vettore di magnetizzazione su unità di volume \(\overset{\underline{}}{M}\).

Si posiziona un volumetto elementare contenente un numero di Avogadro \(N_{A}\) di protoni, in un campo magnetico diretto lungo \({\widehat{i}}_{z}\). Gli spin nel volumetto si allineano rispetto all'asse del campo magnetico esterno.

\begin{figure}
\centering
\includegraphics[width=2.89286in,height=2.63294in,alt={P3086\#yIS1}]{media/6_IntroMRI/image66.pdf}\caption{Figura .: Volumetto elementare di \(N_{A}\) di protoni in un campo magnetico diretto lungo \({\widehat{i}}_{z}\)}
\end{figure}

Secondo la meccanica quantistica, in realtà, gli spin saltano su due livelli energetici: \(\left\langle - \right|\), di energia \(+ \hslash/2\) e \(\left| + \right\rangle\), di energia \(- \hslash/2\). Dal punto di vista macroscopico, ogni volumetto elementare presenta un vettore di magnetizzazione \(\overset{\underline{}}{M}\), dato dalla somma dei singoli momenti magnetici associati ai protoni, normalizzato il volume elementare:

\[\overset{\underline{}}{M} = \frac{1}{V}\sum_{i = 1}^{N_{A}}{\overset{\underline{}}{\mu}}_{i}\]

L'insieme degli spin nel volumetto elementare, dato che risuonano alla stessa frequenza, è chiamato spin \emph{isochromat} poiché può essere considerato come un insieme o un dominio di spin con stessa frequenza.

Trascurando le iterazioni dei protoni con il loro ambiente, è possibile supporre che ogni spin precede intorno al campo applicato secondo l'equazione:

\[\frac{d{\overset{\underline{}}{\mu}}_{i}}{dt} = \gamma{\overset{\underline{}}{\mu}}_{i} \times {\overset{\underline{}}{B}}_{0}\]

Sommando i vari contributi di ogni spin del volumetto e dividendo per il volume elementare si ha:

\[\frac{1}{V}\sum_{i = 1}^{N_{A}}\frac{d{\overset{\underline{}}{\mu}}_{i}}{dt} = \frac{1}{V}\sum_{i = 1}^{N_{A}}{\gamma{\overset{\underline{}}{\mu}}_{i} \times {\overset{\underline{}}{B}}_{0}}\]

Per la linearità dell'operatore somma e derivata è possibile scrivere:

\[\sum_{i = 1}^{N_{A}}{\frac{d}{dt}\left( \frac{1}{V}{\overset{\underline{}}{\mu}}_{i} \right)} = \sum_{i = 1}^{N_{A}}{\gamma\frac{1}{V}{\overset{\underline{}}{\mu}}_{i} \times {\overset{\underline{}}{B}}_{0}} \Leftrightarrow \frac{d}{dt}\left( \sum_{i = 1}^{N_{A}}{\frac{1}{V}{\overset{\underline{}}{\mu}}_{i}} \right) = \gamma\left( \sum_{i = 1}^{N_{A}}{\frac{1}{V}{\overset{\underline{}}{\mu}}_{i}} \right) \times {\overset{\underline{}}{B}}_{0}\]

Per definizione del vettore di magnetizzazione per unità di volume è possibile scrivere:

\[\frac{d\overset{\underline{}}{M}}{dt} = \gamma\overset{\underline{}}{M} \times {\overset{\underline{}}{B}}_{0}\]

È molto conveniente analizzare la magnetizzazione del volumetto e, quindi, l'equazione differenziale, in termini del vettore di magnetizzazione parallelo all'asse individuato dal campo magnetico principale e la componente trasversale. In altre parole, si indica:

\[\overset{\underline{}}{M} = {\overset{\underline{}}{M}}_{\|} + {\overset{\underline{}}{M}}_{\bot}\]

Dove, in coordinate cartesiane e con un campo magnetico esterno diretto lungo \({\widehat{i}}_{z}\):

\[{\overset{\underline{}}{M}}_{\|} = M_{z}{\widehat{i}}_{z}\]

\[{\overset{\underline{}}{M}}_{\bot} = M_{x}{\widehat{i}}_{x} + = M_{y}{\widehat{i}}_{y}\]

L'equazione differenziale può essere scritta scomponendo il vettore di magnetizzazione:

\[\frac{d}{dt}\left( {\overset{\underline{}}{M}}_{\|} + {\overset{\underline{}}{M}}_{\bot} \right) = \gamma\left( {\overset{\underline{}}{M}}_{\|} + {\overset{\underline{}}{M}}_{\bot} \right) \times B_{o}{\widehat{i}}_{Z}\]

Per linearità:

\[\frac{d{\overset{\underline{}}{M}}_{\|}}{dt} + \frac{d{\overset{\underline{}}{M}}_{\bot}}{dt} = \gamma B_{o}{\overset{\underline{}}{M}}_{\|} \times {\widehat{i}}_{Z} + \gamma B_{o}{\overset{\underline{}}{M}}_{\bot} \times {\widehat{i}}_{Z}\]

Per definizione di componente parallela del vettore di magnetizzazione, il termine \({\overset{\underline{}}{M}}_{\|} \times {\widehat{i}}_{Z}\) si annulla:

\[{\overset{\underline{}}{M}}_{\|} \times {\widehat{i}}_{Z} = M_{z}{\widehat{i}}_{Z} \times {\widehat{i}}_{Z} = 0\]

Per cui:

\[\frac{d{\overset{\underline{}}{M}}_{\|}}{dt} + \frac{d{\overset{\underline{}}{M}}_{\bot}}{dt} = \gamma B_{o}{\overset{\underline{}}{M}}_{\bot} \times {\widehat{i}}_{Z}\]

Scrivendo le due equazioni per le proiezioni si ha:

\[\left\{ \begin{matrix}
\frac{dM_{\|}}{dt} = 0 \\
\frac{d{\overset{\underline{}}{M}}_{\bot}}{dt} = \gamma B_{o}{\overset{\underline{}}{M}}_{\bot} \times {\widehat{i}}_{Z}
\end{matrix} \right.\ \]

Questa modellazione non considera i fenomeni di interazione tra gli spin e il reticolo o tra i veri spin del volumetto. In particolare, delle equazioni risulta che la dinamica longitudinale, evolve in maniera indipendente da quella trasversale. La prima è descritta dal tempo di rilassamento longitudinale \(T_{1}\) mentre la seconda dal tempo di rilassamento trasversale \(T_{2}\). Come risultato delle due evoluzioni, il modulo del vettore magnetizzazione non è costante nel tempo.

Per comprendere tale concetto si suppone di porre un volumetto elementare in un campo magnetico diretto lungo l'asse \({\widehat{i}}_{z}\). Raggiunto l'equilibrio termodinamico, il vettore di magnetizzazione si raggiunge il regime, con modulo \(M_{0}\) e diretto come il campo magnetico esterno. Si suppone di applicare un impulso elettromagnetico tale da far ruotare il vettore di magnetizzazione. Durante il ritorno alle condizioni di equilibrio termodinamico, la componente longitudinali del vettore di magnetizzazione è più lenta nel raggiungere il valore di regime \(M_{0}\), rispetto alla componente trasversale, che deve raggiungere il valore iniziale, ovvero nullo.

Questo comportamento è descritto dalle equazioni di Bloch, basate su una descrizione classica della materia. La complessità del moto del vettore di magnetizzazione è dovuta al fatto che tale vettore è composto da circa \(10^{23}\) momenti magnetici, ognuno dei quali con una proprio andamento, influenzato dall'ambiente.

Si osservi che a temperature ambiente, il vettore di magnetizzazione è dato dalla legge di Curie:

\[M_{0} \simeq \frac{\gamma^{2}\hslash^{2}}{4k_{B}T}B_{0}\frac{N_{A}}{V}\]

Il rapporto tra numero di spin, uguale al numero di Avogadro per un volumetto elementare, e il volume del corpo è detto densità protonica ed è indicato con \(\rho\):

\[\rho = \frac{N_{A}}{V}\]

Per cui:

\[M_{0} \simeq \frac{\gamma^{2}\hslash^{2}}{4k_{B}T}B_{0}\rho\]

Le iterazioni degli spin con l'ambiente dipendono dal tessuto analizzato e dal suo stato di salute. Note le tempistiche con cui evolve il vettore di magnetizzazione macroscopico, nel ritorno alla condizione di regime, è possibile ricostruire l'immagine del tessuto.

\subsubsection{Iterazione spin-reticolo}\label{iterazione-spin-reticolo}

In normali condizioni, gli spin non sono indipendenti tra loro ma interagiscono sia con il materiale in cui sono contenuti sia, nello stesso istante, con gli altri spin del reticolo. Ciò determina che l'equazione per la componente longitudinale:

\[\frac{dM_{\|}}{dt} = 0\]

è errata poiché non tiene conto di tali fenomeni; infatti, momenti magnetici dei protoni tendono ad allinearsi col campo esterno, tuttavia, le iterazioni con l'ambiente circostante non consentono al vettore magnetizzazione di essere diretto esattamente lungo l'asse del campo principale, \({\widehat{i}}_{z}\).

Ogni spin è inclinato rispetto l'asse verticale di un angolo \(\vartheta\), inoltre, ogni spin interagisce con gli altri prossimi. Nello specifico, l'angolo \(\vartheta\) è l'inclinazione del vettore che congiunge il centro del sistema di riferimento con lo spin rispetto l'asse individuato dal campo magnetico esterno.

Nell'analisi della componente longitudinale è possibile trascurare gli effetti di repulsione dei protoni dovute alle interazioni coulombiane. In altre parole, si ritiene che la distanza tra due protoni, detta distanza internucleare, sia molto maggiore del raggio gi azione delle forze elettriche.

\begin{figure}
\centering
\includegraphics[width=5.08796in,height=4.76042in,alt={P3130\#yIS1}]{media/6_IntroMRI/image67.pdf}\caption{Figura .: Spin con diverse inclinazioni}
\end{figure}

Ogni singolo spin può essere descritto come un dipolo magnetico elementare, il cui potenziale vettore si dimostra essere:

\[\overset{\underline{}}{A} = \frac{1}{\left| \overset{\underline{}}{r} \right|}\overset{\underline{}}{\mu} \times \overset{\underline{}}{r}\]

Dove \(\overset{\underline{}}{r}\) è il vettore che congiunge due spin.

\begin{figure}
\centering
\includegraphics[width=3.92856in,height=2.17593in,alt={P3135\#yIS1}]{media/6_IntroMRI/image68.pdf}\caption{Figura .: Vettore distanza tra due spin}
\end{figure}

Dal punto di vista classico, il campo prodotto da uno spin è dato da:

\[\overset{\underline{}}{B} = \overset{\underline{}}{\nabla} \times \overset{\underline{}}{A} = \frac{1}{\left| \overset{\underline{}}{r} \right|}\overset{\underline{}}{\nabla} \times \left( \overset{\underline{}}{\mu} \times \overset{\underline{}}{r} \right)\]

Si dimostra che la soluzione a tale equazione è data da:

\[\left\{ \begin{matrix}
B_{x} = 3\mu\frac{\sin\vartheta\cos\vartheta}{r^{3}} \\
B_{y} = 0 \\
B_{z} = \mu\frac{3\cos^{2}\vartheta - 1}{r^{3}}
\end{matrix} \right.\ \]

La componente lungo \({\widehat{i}}_{y}\) è nulla poiché l'andamento delle linee di campo magnetico di uno spin hanno simmetria cilindrica.

\begin{figure}
\centering
\includegraphics[width=2.20513in,height=2.44444in,alt={P3142\#yIS1}]{media/6_IntroMRI/image69.pdf}\caption{Figura .: Simmetria cilindrica del campo prodotto da uno spin}
\end{figure}

Per il principio di sovrapposizione è possibile che uno spin influenzi quelli vicini e viceversa, uno spin è influenzato da quelli vicini.

La componente lungo \({\widehat{i}}_{z}\) è legata all'energia del sistema, infatti, in meccanica classica, si prova che l'energia di un momento magnetico \(\overset{\underline{}}{\mu}\) immerso in un campo magnetico \({\overset{\underline{}}{B}}_{0}\) è data da:

\[U = - \overset{\underline{}}{\mu} \cdot {\overset{\underline{}}{B}}_{0}\]

Con un campo con solo componente verticale ovvero \({\overset{\underline{}}{B}}_{0} = B_{0}{\widehat{i}}_{z}\), si ha:

\[U = - \mu_{z}B_{0}\]

Ne discende che la componente verticale, lungo \({\widehat{i}}_{z}\), è coinvolta negli scambi energetici tra dipoli elementari.

L'interpretazione della componente lungo \({\widehat{i}}_{x}\) del campo prodotto da uno spin \(B_{x}\), è più complessa in quanto bisogna tener conto che i dipoli non sono fermi nello spazio ma precedono intorno all'asse individuato dal campo esterno, uno ente a influenzare questa componente.

Nello specifico, a causa dell'agitazione termica, i dipoli non sono fissi nello spazio ma in moto casuale. L'agitazione termica è molto forte nei liquidi, di cui è prevalentemente composto il corpo umano, e nei gas.

A causa dell'agitazione termica, i dipoli si muovono reciprocamente tra loro in modo continuo. Le iterazioni reciproche dei campo magnetici degli spin sono molto complesse da descrivere; infatti, la teoria del rilassamento è molto sofistica e può essere affrontata solamente adoperando la meccanica quantistica. In letteratura, esistono delle trattazioni riguardati solamente le sostanze pure. A oggi, non esistono analisi basate sulla meccanica quantistica in grado di descrivere il fenomeno del rilassamento nel corpo umano per l'elevato numero di molecole di cui è composto, come acqua, sali disciolti, proteine, lipidi ed ecc.

Nell'analisi della risonanza magnetica, si forniscono delle leggi che descrivono in modo approssimativo, ma allo stesso modo abbastanza fedelmente, i meccanismi di rilassamento all'interno dei tessi umani,

A causa delle iterazioni tra i vari spin, un protone non è soggetto a un campo magnetico uguale a quello statico applicato, ma prossimo a esso. Ogni spin, per le influenze degli altri, ruoterà attorno all'asse individuato dal campo magnetico principale con propria frequenza di Larmor.

Quando è applicato un campo magnetico a frequenza \(\omega_{0} = \gamma B_{0}\), è probabile che avvengano delle transizioni degli spin dallo stato \(\left| + \right\rangle\) allo stato \(\left| - \right\rangle\) e viceversa. Dunque, un dipolo magnetico, soggetto a una radiazione elettromagnetica a frequenza \(\omega_{0}\), prossima a quella con cui precede, può essere indotto a una transizione verso l'alto o verso il basso:

\begin{itemize}
\item
  Nella transizione dal livello energetico \(\left| + \right\rangle\) a \(\left| - \right\rangle\), il protone emette un fotone di energia \(\hslash\gamma B_{0} = \hslash\omega_{0}\). Tale fenomeno è noto come emissione stimolata;
\item
  Nella transizione dal livello energetico \(\left| - \right\rangle\) a \(\left| + \right\rangle\), protone assorbe un foto di energia \(\hslash\omega_{0}\).
\end{itemize}

Il passaggio dallo stato \(\left| + \right\rangle\) a \(\left| - \right\rangle\) produce l'emissione di un fotone che può stimolare un altro spin vicino, il quale transita a sua volta da \(\left| - \right\rangle\) a \(\left| + \right\rangle\). Le iterazioni energetiche a livello microscopico sono, quindi, molto complesse a causa dell'elevato numero di spin contenuti in un volumetto elementare.

Per analizzare il comportamento di un singolo spin con l'intera massa che lo circonda, è possibile ricorrere alla statistica di Boltzmann, considerando lo spin a contatto con il reticolo, o \emph{lattice}, come un piccolo sistema a contatto con un altro avente un numero di spin molto maggiore.

Sia \(N^{+}\) il numero di protoni nel livello energetico \(E^{+} = \hslash\gamma B_{0}/2\) e \(N^{-}\) il numero di protoni nel livello energetico \(E^{+} = - \hslash\gamma B_{0}/2\). Sia \(W_{+ \rightarrow -}\) la probabilità di transizione di uno spin dallo stato \(\left| + \right\rangle\) a \(\left| - \right\rangle\), mentre \(W_{- \rightarrow +}\) la probabilità opposta, ovvero che uno spin transiti da uno stato \(\left| - \right\rangle\) a \(\left| + \right\rangle\).

\begin{figure}
\centering
\includegraphics[width=4.84649in,height=1.12963in,alt={P3161\#yIS1}]{media/6_IntroMRI/image70.pdf}\caption{Figura .: Probabilità di transizioni}
\end{figure}

In meccanica quantistica, le due probabilità si esprimono come:

\[W_{+ \rightarrow -} = \left\langle + \right|\widehat{\mu}\left| - \right\rangle\]

\[W_{- \rightarrow +} = \left\langle - \right|\widehat{\mu}\left| + \right\rangle\]

Queste quantità possono essere stimate a patto di conoscere l'hamiltoniano \(\widehat{H}\) del sistema. Dato il gran numero di particelle presente nel volumetto elementare, la trattazione con la meccanica quantistica, attraverso l'operatore hamiltoniano, è molto complessa. Per tale ragione si ricorre alla statistica di Boltzmann.

All'equilibrio termodinamico, il numero delle transizioni dallo stato \(\left| + \right\rangle\) allo stato \(\left| - \right\rangle\) deve essere uguale al numero delle transizioni da \(\left| - \right\rangle\) a \(\left| + \right\rangle\), in quanto l'energia del sistema si conserva. Si instaura così un equilibrio dinamico nel sistema. Per tale motivo è possibile scrivere l'uguaglianza:

\[N^{+}W_{+ \rightarrow -} = N^{-}W_{- \rightarrow +}\]

Dove \(N^{+}W_{+ \rightarrow -}\) è il numero dei protoni che dallo stato \(\left| + \right\rangle\) transitano allo stato \(\left| - \right\rangle\) e \(N^{-}W_{- \rightarrow +}\) è il numero dei protoni che eseguono la transizione opposta. Le transizioni devono essere tali da mantenere la popolazione degli spin costante.

Si scrive l'equazione all'equilibrio termodinamico come:

\[N^{+}W_{+ \rightarrow -} = N^{-}W_{- \rightarrow +} \Leftrightarrow \frac{W_{+ \rightarrow -}}{W_{- \rightarrow +}} = \frac{N^{-}}{N^{+}}\]

Per la statistica di Boltzmann, il rapporto tra le due probabilità è dato dal rapporto degli esponenziali:

\[\frac{W_{+ \rightarrow -}}{W_{- \rightarrow +}} = \frac{\exp\left( \frac{E^{+}}{k_{B}T} \right)}{\exp\left( \frac{E^{-}}{k_{B}T} \right)} = \exp\left( \frac{E^{+} - E^{-}}{k_{B}T} \right)\]

Sebbene non sia possibile valutare numericamente le due probabilità, è possibile valutarne il rapporto mediante la meccanica statistica.

Si scrive l'equazione differenziale che collega l'evoluzione temporale del numero di protoni che si trovano nello stato \(\left| + \right\rangle\). La variazione nel tempo del numero di spin nello stato energetico \(\left| + \right\rangle\) dipende dal numero di protoni che dallo stato \(\left| - \right\rangle\) passano allo stato \(\left| + \right\rangle\), a cui va sottratto il numero di protoni che transitano dallo stato \(\left| + \right\rangle\) a quello \(\left| - \right\rangle\) nell'intervallo temporale infinitesimo \(dt\):

\[\frac{dN^{+}}{dt} = N^{-}W_{- \rightarrow +} - N^{+}W_{+ \rightarrow -}\]

Da questa espressione si evince che le quantità \(W_{- \rightarrow +}\) e \(W_{+ \rightarrow -}\) sono dimensionalmente omogenee con l'inverso di un tempo:

\[\left\lbrack W_{- \rightarrow +} \right\rbrack = \left\lbrack \frac{1}{s} \right\rbrack\]

In altre parole, \(W_{- \rightarrow +}\) e \(W_{+ \rightarrow -}\) sono delle probabilità per unità di tempo.

Analogamente per la popolazione di spin nello stato \(\left| - \right\rangle\): la variazione del numero di spin nell'unità di tempo è data dalla popolazione di spin nello stato \(\left| + \right\rangle\) che transita nello stato \(\left| - \right\rangle\) a cui va sottratto il numero di spin che esegue la transizione opposta:

\[\frac{dN^{-}}{dt} = N^{+}W_{+ \rightarrow -} - N^{-}W_{- \rightarrow +}\]

Si scrivono le equazioni in termini di differenza di popolazioni \(\mathrm{\Delta}N\):

\[\mathrm{\Delta}N = N^{+} - N^{-}\]

Il parametro \(\mathrm{\Delta}N\) è fondamentale poiché la magnetizzazione macroscopica del volumetto elementare \(\overset{\underline{}}{M}\) è proporzionale alla differenza di popolazione di spin nello stato \(\left| + \right\rangle\) rispetto a quelle nello stato \(\left| - \right\rangle\):

\[\left| \overset{\underline{}}{M} \right| \propto \mathrm{\Delta}N = N^{+} - N^{-}\]

La magnetizzazione è, in ultima analisi, legata al netto di spin paralleli al campo magnetico principale, rispetto a quelli antiparalleli. Da ciò si evince la convenienza nel ragionare in termini di evoluzione della differenza di popolazione, piuttosto che mediante la variazione delle singole popolazioni.

Dalle due equazioni differenziali, che descrivono l'evoluzione temporale di una sola popolazione, sottraendo membro a membro si ha:

\[\left\{ \begin{matrix}
\frac{dN^{+}}{dt} = N^{-}W_{- \rightarrow +} - N^{+}W_{+ \rightarrow -} \\
\frac{dN^{-}}{dt} = N^{+}W_{+ \rightarrow -} - N^{-}W_{- \rightarrow +}
\end{matrix} \right.\ \]

\[\frac{d}{dt}\left( N^{+} - N^{-} \right) = N^{-}W_{- \rightarrow +} - N^{+}W_{+ \rightarrow -} - N^{+}W_{+ \rightarrow -} + N^{-}W_{- \rightarrow +}\]

Da cui:

\[\frac{d\mathrm{\Delta}N}{dt} = 2N^{-}W_{- \rightarrow +} - 2N^{+}W_{+ \rightarrow -}\]

Sia \(N\) il numero totale degli spin contenuto nel volumetto elementare. Siccome il volume non scambia materia con l'esterno, il numero degli spin è costante:

\[N = N^{+} + N^{-} = const\]

Per definizione:

\[\mathrm{\Delta}N = N^{+} - N^{-}\]

Sommando membro a membro sia ha:

\[N + \mathrm{\Delta}N = 2N^{+} \Leftrightarrow N^{+} = \frac{1}{2}(N + \mathrm{\Delta}N)\]

Invece, sottraendo membro a membro, si ha:

\[N - \mathrm{\Delta}N = 2N^{+} \Leftrightarrow N^{-} = \frac{1}{2}(N - \mathrm{\Delta}N)\]

L'equazione differenziale:

\[\frac{d}{dt}\left( N^{+} - N^{-} \right) = 2N^{-}W_{- \rightarrow +} - 2N^{+}W_{+ \rightarrow -}\]

Può essere scritta come:

\[\frac{d\mathrm{\Delta}N}{dt} = 2\frac{1}{2}(N - \mathrm{\Delta}N)W_{- \rightarrow +} - 2\frac{1}{2}(N + \mathrm{\Delta}N)W_{+ \rightarrow -} = (N - \mathrm{\Delta}N)W_{- \rightarrow +} - (N + \mathrm{\Delta}N)W_{+ \rightarrow -}\]

Raccogliendo \(N\) e \(\mathrm{\Delta}N\) al secondo membro si ha:

\[\frac{d\mathrm{\Delta}N}{dt} = N\left( W_{- \rightarrow +} - W_{+ \rightarrow -} \right) - \mathrm{\Delta}N\left( W_{- \rightarrow +} + W_{+ \rightarrow -} \right)\]

Queste equazioni devono essere sempre valide poiché non sono state proposte ipotesi particolari semplificative. Se le equazioni sono sempre valide, lo sono anche all'equilibrio termodinamico, condizione in cui non vi è nessuna variazione di \(\mathrm{\Delta}N\) dato che le due popolazioni presentano lo stesso numero di spin, nonostante la variazione della configurazione del sistema.

Dal punto di vista matematico, all'equilibrio termodinamico, risulta:

\[\frac{d\mathrm{\Delta}N}{dt} = 0\]

Dunque, \(\mathrm{\Delta}N = \mathrm{\Delta}N_{0} = const\). Siano \(N_{0}^{+}\) il numero di spin nello stato \(\left| + \right\rangle\) e \(N_{0}^{-}\) il numero di spin nello stato \(\left| - \right\rangle\) all'equilibrio termodinamico, risulta:

\[\mathrm{\Delta}N_{0} = N_{0}^{+} - N_{o}^{-}\]

\(\mathrm{\Delta}N_{0}\) è la differenza di spin nello stato \(\left| + \right\rangle\) rispetto a quelli nello stato \(\left| - \right\rangle\) all'equilibrio termodinamico; ciò produce la magnetizzazione macroscopica \(\overset{\underline{}}{M}\).

All'equilibrio termodinamico, l'equazione:

\[\frac{d\mathrm{\Delta}N}{dt} = N\left( W_{- \rightarrow +} - W_{+ \rightarrow -} \right) - \mathrm{\Delta}N\left( W_{- \rightarrow +} + W_{+ \rightarrow -} \right)\]

Diventa:

\[N\left( W_{- \rightarrow +} - W_{+ \rightarrow -} \right) - \mathrm{\Delta}N_{0}\left( W_{- \rightarrow +} + W_{+ \rightarrow -} \right) = 0\]

Si ricava \(\mathrm{\Delta}N_{0}\):

\[\mathrm{\Delta}N_{0} = N\frac{W_{- \rightarrow +} - W_{+ \rightarrow -}}{W_{- \rightarrow +} + W_{+ \rightarrow -}}\]

Sapendo che il valore all'equilibrio termodinamico del netto di spin è \(\mathrm{\Delta}N_{0}\), è possibile ricavare l'andamento temporale di \(\mathrm{\Delta}N\). Infatti, siccome \(\mathrm{\Delta}N_{0}\) è costante, è possibile sottrarre la sua derivata al primo membro dell'equazione:

\[\frac{d\mathrm{\Delta}N}{dt} = N\left( W_{- \rightarrow +} - W_{+ \rightarrow -} \right) - \mathrm{\Delta}N\left( W_{- \rightarrow +} + W_{+ \rightarrow -} \right)\]

Ottenendo:

\[\frac{d}{dt}\left( \mathrm{\Delta}N - \mathrm{\Delta}N_{0} \right) = N\left( W_{- \rightarrow +} - W_{+ \rightarrow -} \right) - \mathrm{\Delta}N\left( W_{- \rightarrow +} + W_{+ \rightarrow -} \right)\]

Al secondo membro si raccoglie il termine \(\left( W_{- \rightarrow +} + W_{+ \rightarrow -} \right)\), ottenendo:

\[\frac{d}{dt}\left( \mathrm{\Delta}N - \mathrm{\Delta}N_{0} \right) = \left( N\frac{W_{- \rightarrow +} - W_{+ \rightarrow -}}{W_{- \rightarrow +} + W_{+ \rightarrow -}} - \mathrm{\Delta}N \right)\left( W_{- \rightarrow +} + W_{+ \rightarrow -} \right)\]

All'equilibrio termodinamico, si ha:

\[\mathrm{\Delta}N_{0} = N\frac{W_{- \rightarrow +} - W_{+ \rightarrow -}}{W_{- \rightarrow +} + W_{+ \rightarrow -}}\]

Per cui, l'equazione differenziale può essere scritta come:

\[\frac{d}{dt}\left( \mathrm{\Delta}N - \mathrm{\Delta}N_{0} \right) = \left( \mathrm{\Delta}N_{0} - \mathrm{\Delta}N \right)\left( W_{- \rightarrow +} + W_{+ \rightarrow -} \right)\]

Dato che \(W_{- \rightarrow +}\) e \(W_{+ \rightarrow -}\) sono omogenee con l'inverso di un tempo, si pone:

\[\frac{1}{T_{1}} = W_{- \rightarrow +} + W_{+ \rightarrow -}\]

Con questa posizione, l'equazione differenziale si può scrivere:

\[\frac{d}{dt}\left( \mathrm{\Delta}N - \mathrm{\Delta}N_{0} \right) = \frac{1}{T_{1}}\left( \mathrm{\Delta}N_{0} - \mathrm{\Delta}N \right)\]

Al fine di avere la stessa quantità, si raccoglie un segno meno al secondo membro:

\[\frac{d}{dt}\left( \mathrm{\Delta}N - \mathrm{\Delta}N_{0} \right) = - \frac{1}{T_{1}}\left( \mathrm{\Delta}N - \mathrm{\Delta}N_{0} \right)\]

L'evoluzione temporale di \(\mathrm{\Delta}N\) dipende dalla costante di tempo \(T_{1}\), data dalla somma delle probabilità di transizione da uno stato all'altro.

La relazione individuata è molto approssimata, poiché le probabilità non sono note a priori, ma permette di descrivere l'evoluzione temporale del netto degli spin nello stato parallelo nel tempo. Si osserva che la soluzione dell'equazione è di tipo esponenziale crescente, con costante di tempo \(T_{1}\), il valore di regime è \(\mathrm{\Delta}N_{0}\).

Generalizzando, in un sistema di spin immerso in un campo magnetico, la componente longitudinale del vettore di magnetizzazione tende a raggiungere il valore di regime, dato dalla legge di Curie, con una costante di tempo \(\tau = T_{1}\).

\begin{figure}
\centering
\includegraphics[width=2.2315in,height=1.68519in,alt={P3237\#yIS1}]{media/6_IntroMRI/image71.pdf}\caption{Figura .: Evoluzione temporale del netto di vettore di magnetizzazione per effetto del campo principale}
\end{figure}

Disattivando il campo magnetico principale dopo che il sistema ha raggiuto l'equilibrio termodinamico, la componente longitudinale del vettore di magnetizzazione si annulla. Il decadimento è di tipo esponenziale con costante di tempo sempre uguale a \(T_{1}\).

\begin{figure}
\centering
\includegraphics[width=3.31667in,height=2.56742in,alt={P3240\#yIS1}]{media/6_IntroMRI/image72.pdf}\caption{Figura .: Evoluzione del vettore di magnetizzazione alla rimozione del campo principale}
\end{figure}

Il tempo \(T_{1}\) interessa gli scambi tra i vari spin contenuti nel materiale ed è legato alla componente longitudinale, lungo \({\widehat{i}}_{z}\). L'equazione di Bloch \(dM_{z}/dt = 0\) viene corretta introducendo l'andamento temporale, dettato dalla costante di tempo \(T_{1}\):

\[\frac{dM_{z}}{dt} = \frac{1}{T_{1}}\left( M_{0} - M_{z}\  \right)\]

Tale equazione è detta I equazione di Bloch. Inoltre, il tempo \(T_{1}\) è detto tempo di rilassamento longitudinale o spin-reticolo o, in letteratura anglosassone, \emph{spin-lattice}. Dal nome si evince che il tempo \(T_{1}\) tiene conto delle iterazioni di uno spin a contatto con l'intero reticolo, visto come un serbatoio termico, in cui è inserito. Generalmente, \(T_{1}\) è maggiore nei liquidi rispetto ai solidi.

La soluzione della prima equazione di Bloch è del tipo:

\[M_{z}(t) = M_{z}(0)\exp\left( - \frac{t}{T_{1}} \right) + M_{0}\left\lbrack 1 - \exp\left( - \frac{t}{T_{1}} \right) \right\rbrack\]

L'equazione di Bloch è fondamentale per studiare come il vettore di magnetizzazione raggiunga nuovamente l'equilibrio termodinamico, a valle di una perturbazione.

La costante di tempo \(T_{1}\) è un parametro essenziale per la caratterizzazione dei tessuti, così da eseguire l'\emph{imaging}. Tessuti diversi sono caratterizzati da un tempo di rilassamento longitudinale specifico.

\subsubsection{Iterazioni spin-spin}\label{iterazioni-spin-spin}

Un importante meccanismo, che determina il decadimento delle componenti trasversali della magnetizzazione, è la generazioni di campi locali prodotti dagli spin. Questi campi si combinano col campo principale applicando, modificando localmente il campo in cui sono immersi i campi vicini.

Si considera un singolo spin, il campo prodotto da questo spin è indicato con \({\overset{\underline{}}{B}}_{loc}\). Gli altri spin genereranno dei campi diversi. Tutte le componenti verticali possiedono intensità diverse, non predicibili in quanto non è possibile ricostruire il moto di ogni singolo spin del volumetto considerato.

Per studiare il campo locale visto da uno spin si utilizza il valor quadratico medio:

\[B_{RMS} = \sqrt{\left\langle \left| {\overset{\underline{}}{B}}_{loc} \right| \right\rangle^{2}}\]

Dunque, vi è una componente magnetica del campo valor quadratico medio lungo \({\widehat{i}}_{z}\). Questa componente si sovrappone al campo magnetico principale \(B_{0}{\widehat{i}}_{z}\), applicato dall'esterno; ciò determina una variazione della frequenza con cui i singoli spin precedono intorno all'asse \({\widehat{i}}_{z}\).

Data l'agitazione termina, gli spin sono in moto relativo, quindi, la loro posizione non è statica ma dinamica. Ne consegue che il campo locale prodotto da uno spin varia nel tempo, poiché, appunto, varia la distribuzione locale degli spin. Per tale motivo il campo valor quadratico medio è una funzione del tempo:

\[B_{RMS}(t) = \sqrt{\left\langle \left| {\overset{\underline{}}{B}}_{loc}(t) \right| \right\rangle^{2}}\]

Sia \(t_{1}\) un primo istante di osservazione; il campo locale di uno spin vale \({\overset{\underline{}}{B}}_{loc}\left( t_{1} \right)\). In un secondo istante temporale \(t_{2}\), successivo al primo, il campo locale sarà diverso, a causa del modi termici, per cui:

\[{\overset{\underline{}}{B}}_{loc}\left( t_{1} \right) \neq {\overset{\underline{}}{B}}_{loc}\left( t_{2} \right)\]

Questo risultato è dovuto alla diversa configurazione degli spin tra gli istanti \(t_{1}\) e \(t_{2}\).

Per ogni spin, la frequenza di precessione intorno all'asse \({\widehat{i}}_{z}\) è diversa dagli altri ed è data da:

\[\mathrm{\Delta}\omega(t) = \gamma{\overset{\underline{}}{B}}_{loc}(t)\]

Dove \(\mathrm{\Delta}\omega = \omega_{0} - \omega_{loc}(t)\). La differenza tra le frequenze di precessione è, dunque, una funzione del tempo. Dati due istanti temporale, risulta:

\[\mathrm{\Delta}\omega\left( t_{1} \right) = \gamma{\overset{\underline{}}{B}}_{loc}\left( t_{1} \right) \neq \mathrm{\Delta}\omega\left( t_{2} \right) = \gamma{\overset{\underline{}}{B}}_{loc}\left( t_{2} \right)\]

Ovviamente \(\mathrm{\Delta}\omega\left( t_{1} \right) \neq \mathrm{\Delta}\omega\left( t_{2} \right)\) poiché i campi locali nei due istanti temporali sono diversi.

Si vuole studiare il fenomeno del campo locale nel sistema di riferimento locale. Tralasciando la frequenza principale \(\omega_{0}\), restano solamente le variazioni \(\mathrm{\Delta}\omega\) dovute alle iterazioni tra uno spin e i protoni locali. Queste variazioni, nel sistema rotante danno luogo a delle variazioni di fase:

\[\vartheta = \int_{t_{0}}^{t_{1}}{\mathrm{\Delta}\omega\left( t' \right)dt'}\]

Può capitare che in un istante la fase cresca più rapidamente, altre volte meno rapidamente. In ogni caso la rotazione degli spin avviene sempre, complessivamente, in senso orario. Nel sistema rotante, in particolare, le variazioni di fase possono essere sia positive che negative; di conseguenza, la proiezione dello spin sul piano \(x' - y'\) non è fissa ma varia nel tempo, secondo la sua variazione di fase legate al campo locale.

\begin{figure}
\centering
\includegraphics[width=4.22917in,height=1.49311in,alt={P3268\#yIS1}]{media/6_IntroMRI/image73.pdf}\caption{Figura .: Variazione della fase a causa del campo locale}
\end{figure}

La proiezione di tutti gli spine nel piano \(x' - y'\) è del tutto casuale, poiché casuale è il campo locale che ogni spin percepisce. È possibile, quindi, concludere che le proiezioni degli spin sono tutte distribuite uniformemente nel piano \(x' - y'\). Di conseguenza, l'effetto medio, percepito a livello macroscopico nel volumetto contenente un numero di Avogadro di particelle, è l'annullamento delle componenti trasverse del vettore di magnetizzazione. Infatti, le componente trasversa, essendo casuali, si elidono a vicenda; ovvero, la risultate media della componente trasversa del vettore di magnetizzazione è nulla, dato l'elevato numero di spin considerato.

\begin{figure}
\centering
\includegraphics[width=4.40341in,height=3.58333in,alt={P3271\#yIS1}]{media/6_IntroMRI/image74.pdf}\caption{Figura .: Eliminazione delle componenti trasverse degli spin causa distribuzione uniforme}
\end{figure}

Le componenti longitudinali del campo magnetico prodotto da uno spin induce uno sfasamento e defasamento degli spin vicini, poiché procedono con frequenze diverse e casuali. Ne discende che la somma delle componenti trasversali degli spin mediamente è nulla. Da notare che lo sfasamento è legato alla fase diversa per ogni singolo spin.

Il tempo con cui la componente trasversale media va zero può essere stimato, studiando le iterazioni tra due spin. Si suppone che a un certo instante di tempo le fasi dei due spin siano le stesse.

In generale, la differenza di fase è data da:

\[\mathrm{\Delta}\phi = \phi_{1} - \phi_{2} = \left( \phi_{0}^{1} - \mathrm{\Delta}\omega_{{loc}_{1}}t \right) - \left( \phi_{0}^{2} - \mathrm{\Delta}\omega_{{loc}_{2}}t \right)\]

Dove \(\mathrm{\Delta}\omega_{{loc}_{1}}\) e \(\mathrm{\Delta}\omega_{{loc}_{2}}\) sono le frequenze di precessione relative rispetto a quelle del campo magnetico principale, legati ai campi locali, rispettivamente del primo spin, \(B_{{loc}_{1}}\) e del secondo spin \(B_{{loc}_{2}}\).

Se le fasi inziali dei due spin sono le stesse, ovvero \(\phi_{0}^{1} = \phi_{0}^{2}\), la differenza di fase si scrive come:

\[\mathrm{\Delta}\phi = - \left( \mathrm{\Delta}\omega_{{loc}_{1}} - \mathrm{\Delta}\omega_{{loc}_{2}} \right)t\]

Le frequenze di precessione relative rispetto a quelle del campo magnetico principale possono essere scritte in funzione del campo locale:

\[\mathrm{\Delta}\omega_{{loc}_{i}} = \gamma B_{{loc}_{i}},\ \ i = 1,2\]

Per cui la differenza di fase può essere scritta come:

\[\mathrm{\Delta}\phi = - \left( \gamma B_{{loc}_{1}} - \gamma B_{{loc}_{2}} \right)t\]

Si indica la differenza di campi locali come:

\[\mathrm{\Delta}B_{loc} = B_{{loc}_{1}} - B_{{loc}_{2}}\]

Si ottiene:

\[\mathrm{\Delta}\phi = - \gamma\mathrm{\Delta}B_{loc}t\]

Per avere una risultate netta nulla nel piano trasverso, le due proiezioni sul piano \(x' - y'\) devono essere uguali e oppure, dunque, la differenza di fase deve essere \(\mathrm{\Delta}\phi = \pi\). È, ora, possibile ottenere il tempo con cui due spin annullano le proprie componenti trasversali. Si indica tale tempo con \(\tau\):

\[\mathrm{\Delta}\phi = \pi = - \gamma\mathrm{\Delta}B_{loc}\tau\]

Da cui:

\[\tau = - \frac{\pi}{\gamma\mathrm{\Delta}B_{loc}}\]

Le differenze dei campi indotti hanno un'intensità proporzionale al fattore \(3\mu/r^{3}\):

\[B_{loc} \propto \frac{3\mu}{r^{3}}\]

Il tempo di defasamento, al limite (ignorando le costanti), è dato da:

\[\tau\sim\frac{\pi}{\gamma\frac{\mu}{r^{3}}} = \frac{\pi r^{3}}{\mu\gamma}\]

Da questa relazione, noti i valori medi del momento magnetico e della distanza interatomica, è possibile ottenere una stima del tempo di defasamento più o meno concorde ai dati sperimentali. La stima di basa su una descrizione non esatta del fenomeno, in quanto fondata della meccanica classica; tuttavia, è usata poiché permette di ottenere una buona stima dei risultati sperimentali.

Il tempo di defasamento \(\tau\) è indicato con \(T_{2}\) ed è detto tempo di rilassamento traversarsele. \(T_{2}\) rappresenta la costante di tempo con cui la componente trasversale del vettore di magnetizzazione, \({\overset{\underline{}}{M}}_{\bot}\), tende a zero:

\[M_{x} \rightarrow 0,\ \ M_{y} \rightarrow 0\]

\(T_{2}\) è generalmente minore nei solidi rispetto ai liquidi.

\subsection{Equazioni di Bloch}\label{equazioni-di-bloch}

All'equazione che descrive la magnetizzazione macroscopica di un volumetto elementare:

\[\frac{d\overset{\underline{}}{M}}{dt} = \gamma\overset{\underline{}}{M} \times {\overset{\underline{}}{B}}_{0}\]

vanno aggiunti due termini che tengono conto dei fenomeni di rilassamento. Il primo termine è legato agli scambi energetici tra uno spin e il reticolo, visto come un serbatoio termico, il secondo legato alle iterazioni locali tra spin:

\[\frac{d\overset{\underline{}}{M}}{dt} = \gamma\overset{\underline{}}{M} \times {\overset{\underline{}}{B}}_{0} + \frac{1}{T_{1}}\left( M_{0} - M_{z}\  \right){\widehat{i}}_{z} - \frac{1}{T_{2}}{\overset{\underline{}}{M}}_{\bot}\]

La precedente equazione è detta di Bloch. Il segno meno nel termine \({\overset{\underline{}}{M}}_{\bot}/T_{2}\) è dovuto al tendere a zero della componente traversale del vettore di magnetizzazione con un tempo \(T_{2}\).

Le equazioni di Bloch forniscono una descrizione fenomenologica del vettore di magnetizzazione, poiché non descrive i fenomeni fisici ma forniscono dei risultati concordi alle osservazioni sperimentali. Inoltre, le equazioni di Bloch non possono essere dedotte utilizzando la meccanica quantistica.

In letteratura anglosassone si definisce rilassività o relaxivity longitudinale e trasversale come:

\[R_{1} = \frac{1}{T_{1}},\ \ R_{2} = \frac{1}{T_{2}}\]

L'equazione vettoriale di Bloch può essere anche scritta in termini di relaxivity

\[\frac{d\overset{\underline{}}{M}}{dt} = \gamma\overset{\underline{}}{M} \times {\overset{\underline{}}{B}}_{0} + R_{1}\left( M_{0} - M_{z}\  \right){\widehat{i}}_{z} - R_{2}{\overset{\underline{}}{M}}_{\bot}\]

I tempi di rilassamento permettono di caratterizzare tessuti diversi e di identificare lo stato di salute di un tessuto stesso.

I tempi di rilassamento sono abbastanza variabili a seconda del tessuto; in particolare, tra i vari tessuti molli vi è una grande variabilità dei tempi di rilassamento. Ciò consente di discriminare con gradi di grigio diversi i tempi di rilassamento i diversi tessuti.

Le differenze di tempo di rilassamento longitudinale \(T_{1}\) non possono essere determinate sulla base della teoria quantistica, a causa dell'elevata complessità biochimiche dei tessuti umani.

In media, la composizione dei vari tessuti è simile tra i diversi individui, quindi, in un paziente i tempo di rilassamento sono grossomodo simili a quelli medi. Mediante delle opportune sequenze di acquisizione è possibile ottenere informazioni sui tempi di rilassamento, caratterizzando così il tessuto.

\begin{longtable}[]{@{}
  >{\centering\arraybackslash}p{(\linewidth - 4\tabcolsep) * \real{0.3333}}
  >{\centering\arraybackslash}p{(\linewidth - 4\tabcolsep) * \real{0.3333}}
  >{\centering\arraybackslash}p{(\linewidth - 4\tabcolsep) * \real{0.3334}}@{}}
\caption{Tabella 6.1: Tempi di rilassamento di vari tessuti molli con campo principale di \(1.5\ T\) e temperatura di \(37\ {^\circ}C\)}\tabularnewline
\toprule\noalign{}
\begin{minipage}[b]{\linewidth}\centering
Tessuto
\end{minipage} & \begin{minipage}[b]{\linewidth}\centering
Tempo di rilassamento longitudinale \(T_{1}\)
\end{minipage} & \begin{minipage}[b]{\linewidth}\centering
Tempo di rilassamento trasverale \(T_{2}\)
\end{minipage} \\
\midrule\noalign{}
\endfirsthead
\toprule\noalign{}
\begin{minipage}[b]{\linewidth}\centering
Tessuto
\end{minipage} & \begin{minipage}[b]{\linewidth}\centering
Tempo di rilassamento longitudinale \(T_{1}\)
\end{minipage} & \begin{minipage}[b]{\linewidth}\centering
Tempo di rilassamento trasverale \(T_{2}\)
\end{minipage} \\
\midrule\noalign{}
\endhead
\bottomrule\noalign{}
\endlastfoot
Materia grigia & \(\sim 950\ ms\) & \(\sim 100\ ms\) \\
Tessuto muscolare & \(\sim 900\ ms\) & \(\sim 50\ ms\) \\
Grasso & \(\sim 250\ ms\) & \(\sim 60\ ms\) \\
Sangue & \(\sim 1200\ ms\) & \(\sim 100\ ms\) \\
Materia bianca & \(\sim 600\ ms\) & \(\sim 80ms\) \\
Fluido cerebrospinale (CSF) & \(\sim 4500\ ms\) & \(\sim 2200\ ms\) \\
\end{longtable}

Generalmente risulta che il tempo di rilassamento longitudinale è maggiore di quello trasversale, \(T_{1} > T_{2}\), quindi, l'evoluzione longitudinale evolve più lentamente di quella trasversale. Analogamente per le relaxivity risulta \(R_{2} > R_{1}\).

\subsubsection{Risoluzione dell'equazione di Bloch}\label{risoluzione-dellequazione-di-bloch}

Si è visto che l'equazione vettoriale di Bloch, comprendente i tempi di rilassamento longitudinale e traversale è:

\[\frac{d\overset{\underline{}}{M}}{dt} = \gamma\overset{\underline{}}{M} \times {\overset{\underline{}}{B}}_{0} + \frac{1}{T_{1}}\left( M_{0} - M_{z}\  \right){\widehat{i}}_{z} - \frac{1}{T_{2}}{\overset{\underline{}}{M}}_{\bot}\]

Questa equazione descrive da un punto di vista classico l'andamento del vettore di magnetizzazione \(\overset{\underline{}}{M}\) nel tempo.

Si suppone di forzare il sistema di spin mediante un campo magnetico principale \(B_{0}\) diretto lungo \({\widehat{i}}_{z}\). Si scrive il prodotto vettorale tra il vettore di magnetizzazione e il campo esterno applicato:

\[\overset{\underline{}}{M} \times {\overset{\underline{}}{B}}_{0} = \left| \begin{matrix}
{\widehat{i}}_{x} & {\widehat{i}}_{y} & {\widehat{i}}_{z} \\
M_{x} & M_{y} & M_{z} \\
0 & 0 & B_{0}
\end{matrix} \right| = B_{0}\left( M_{y}{\widehat{i}}_{x} - M_{x}{\widehat{i}}_{y} \right)\]

Si scrive l'equazione di Bloch esplicitando le componenti dei vettori coinvolti:

\[\frac{d}{dt}\left( M_{x}{\widehat{i}}_{x} + M_{y}{\widehat{i}}_{y} + M_{z}{\widehat{i}}_{z} \right) = \gamma B_{0}\left( M_{y}{\widehat{i}}_{x} - M_{x}{\widehat{i}}_{y} \right) + \frac{1}{T_{1}}\left( M_{0} - M_{z}\  \right){\widehat{i}}_{z} - \frac{1}{T_{2}}\left( M_{x}{\widehat{i}}_{x} + M_{y}{\widehat{i}}_{y} \right)\]

Per effetto del campo magnetico principale, il vettore di magnetizzazione si sposta da una configurazione iniziale \(\overset{\underline{}}{M}\left( t_{0} \right)\) a quella finale, descritta dall'equazione di Bloch.

Si proietta quest'ultima lungo gli assi:

\[\left\{ \begin{matrix}
\frac{dM_{x}}{dt} = \gamma B_{0}M_{y} - \frac{1}{T_{2}}M_{x} \\
\frac{dM_{y}}{dt} = - \gamma B_{0}M_{x} - \frac{1}{T_{2}}M_{y} \\
\frac{dM_{z}}{dt} = \frac{1}{T_{1}}\left( M_{0} - M_{z}\  \right)
\end{matrix} \right.\ \]

Com'è facile osservare, l'evoluzione della componente longitudinale, \(M_{z}\), è indipendente da quelle trasversali \(M_{x}\) e \(M_{y}\); infatti, le relative equazioni non sono accoppiate dopo la proiezione sugli assi. Le componenti trasversa sono, invece, legate tra loro.

Si risolve l'equazione relativa alla componente longitudinale del vettore \(\overset{\underline{}}{M}\):

\[\frac{dM_{z}}{dt} = \frac{1}{T_{1}}\left( M_{0} - M_{z}\  \right)\]

Si passa all'equazione omogenea associata:

\[\frac{dM_{z}}{dt} = \frac{1}{T_{1}}M_{z}\]

La soluzione di questa equazione è del tipo:

\[M_{z}(t) = k\exp\left( - \frac{t}{T_{1}} \right)\]

Dove \(k\) è una costante dipendente dalle condizioni iniziali del sistema.

Alla soluzione dell'omogenea, va associata una soluzione particolare. Siccome il forzamento è costante una possibile soluzione particolare è del tipo:

\[M_{z}(t) = c = const\]

Si sostituisce tale soluzione nell'equazione differenziale:

\[\left. \ \frac{dM_{z}}{dt} \right|_{M_{z} = c} = \frac{1}{T_{1}}\left. \ \left( M_{0} - M_{z}\  \right) \right|_{M_{z} = c} \Leftrightarrow \frac{dc}{dt} = \frac{1}{T_{1}}\left( M_{0} - c \right)\]

La derivata di una costante è nulla, per cui:

\[\frac{1}{T_{1}}\left( M_{0} - c \right) = 0 \Leftrightarrow c = M_{0}\]

L'integrale generale dell'equazione differenziale per la componente longitudinale è:

\[M_{z}(t) = k\exp\left( - \frac{t}{T_{1}} \right) + M_{0}\]

Si suppone che il vettore di magnetizzazione sia noto all'istante \(t = t_{0}\). Applicando la condizione iniziale è possibile determinare il valore di \(k\):

\[M_{z}\left( t_{0} \right) = k\exp\left( - \frac{t_{0}}{T_{1}} \right) + M_{0} \Leftrightarrow k = \left\lbrack M_{z}\left( t_{0} \right) - M_{0} \right\rbrack\exp\left( \frac{t_{0}}{T_{1}} \right)\]

La soluzione è, dunque:

\[M_{z}(t) = \left\lbrack M_{z}\left( t_{0} \right) - M_{0} \right\rbrack\exp\left( \frac{t_{0}}{T_{1}} \right)\exp\left( - \frac{t}{T_{1}} \right) + M_{0}\]

Esplicitando la componente transitoria e quella di regime, si ha:

\[M_{z}(t) = M_{0}\left\lbrack 1 - \exp\left( \frac{t_{0} - t}{T_{1}} \right) \right\rbrack + M_{z}\left( t_{0} \right)\exp\left( \frac{t_{0} - t}{T_{1}} \right)\]

Ovviamente, per tempi sufficientemente lunghi:

\[M_{z}\left( t_{0} \right)\exp\left( \frac{t_{0} - t}{T_{1}} \right) \rightarrow 0,\ \ t \rightarrow \infty\]

Se \(t_{0} = 0\), risulta:

\[M_{z}(t) = M_{0}\left\lbrack 1 - \exp\left( - \frac{t}{T_{1}} \right) \right\rbrack + M_{z}(0)\exp\left( - \frac{t}{T_{1}} \right)\]

Supponendo che il valore iniziale della magnetizzazione sia nullo, la soluzione diventa:

\[M_{z}(t) = M_{0}\left\lbrack 1 - \exp\left( - \frac{t}{T_{1}} \right) \right\rbrack\]

Applicando il campo magnetico principale, da una condizione di equilibrio termodinamico, la componente longitudinale tende al valore di regime con andamento esponenziale e costante di tempo \(T_{1}\). Il valore di regime è dato dall'equazione di Curie:

\[M_{0} \simeq \ \frac{N}{V}\frac{\gamma^{2}\hslash^{2}}{4k_{B}T}B_{0}\]

\begin{figure}
\centering
\includegraphics[width=4.32769in,height=3.10307in,alt={P3386\#yIS1}]{media/6_IntroMRI/image75.pdf}\caption{Tabella 6.2: Andamento della soluzione longitudinale dell'equazione di Bloch}
\end{figure}

È possibile ritenere che la componente longitudinale raggiunga il regime dopo un tempo uguale a \(4 \div 5\) volte la costante di tempo \(T_{1}\). Generalmente il tempo di rilassamento longitudinale è dell'ordine del secondo, per cui si ritiene che il transitorio abbia una durata di circa \(5\ s\).

Nella pratica, una volta immesso il paziente nel gantry della risonanza magnetica è necessario aspettare un giusto tempo prima di eseguire l'imaging, affinché gli spin dei protoni contenuti nel corpo del paziente raggiungano l'equilibrio termodinamico. L'esame, quindi, non può iniziale istantaneamente ma è necessario che il vettore di magnetizzazione del paziente raggiunga l'equilibrio termodinamico col campo. In caso contrario, si ottengono delle immagini con risultati falsati. In un tempo di \(5 \div 10\ s\) il tecnico radiologo entra nella sala di comando.

Per quanto riguarda le componenti trasversali, le equazioni che descrivono la loro evoluzione sono:

\[\left\{ \begin{matrix}
\frac{dM_{x}}{dt} = \gamma B_{0}M_{y} - \frac{1}{T_{2}}M_{x} \\
\frac{dM_{y}}{dt} = - \gamma B_{0}M_{x} - \frac{1}{T_{2}}M_{y}
\end{matrix} \right.\ \]

Dove \(\omega_{0} = \ \gamma B_{0}\); per cui il sistema può essere scritto come:

\[\left\{ \begin{matrix}
\frac{dM_{x}}{dt} = \omega_{0}M_{y} - \frac{1}{T_{2}}M_{x} \\
\frac{dM_{y}}{dt} = - \omega_{0}M_{x} - \frac{1}{T_{2}}M_{y}
\end{matrix} \right.\ \]

Si pone il sistema in forma matriciale:

\[\frac{d}{dt}\left( \begin{array}{r}
M_{x} \\
M_{y}
\end{array} \right) = \begin{pmatrix}
 - \frac{1}{T_{2}} & \omega_{0} \\
 - \omega_{0} & - \frac{1}{T_{2}}
\end{pmatrix}\left( \begin{array}{r}
M_{x} \\
M_{y}
\end{array} \right)\]

La soluzione di questa equazione è del tipo:

\[{\overset{\underline{}}{M}}_{\bot} = \overset{\underline{}}{k}\exp\left( \lambda\overset{\underline{}}{\overset{\underline{}}{I}}t \right)\]

Sostituendo tale equazione nell'equazione differenziale si ha:

\[\frac{d}{dt}\left\lbrack \overset{\underline{}}{k}\exp\left( \lambda\overset{\underline{}}{\overset{\underline{}}{I}}t \right) \right\rbrack = \begin{pmatrix}
 - \frac{1}{T_{2}} & \omega_{0} \\
 - \omega_{0} & - \frac{1}{T_{2}}
\end{pmatrix}\overset{\underline{}}{k}\exp\left( \lambda\overset{\underline{}}{\overset{\underline{}}{I}}t \right) \Leftrightarrow \lambda\overset{\underline{}}{\overset{\underline{}}{I}}\overset{\underline{}}{k}\exp\left( \lambda\overset{\underline{}}{\overset{\underline{}}{I}}t \right) = \begin{pmatrix}
 - \frac{1}{T_{2}} & \omega_{0} \\
 - \omega_{0} & - \frac{1}{T_{2}}
\end{pmatrix}\overset{\underline{}}{k}\exp\left( \lambda\overset{\underline{}}{\overset{\underline{}}{I}}t \right)\]

Moltiplicando per la matrice inversa a \(\overset{\underline{}}{k}\exp\left( \lambda\overset{\underline{}}{\overset{\underline{}}{I}}t \right)\) si ottiene:

\[\lambda\overset{\underline{}}{\overset{\underline{}}{I}} = \begin{pmatrix}
 - \frac{1}{T_{2}} & \omega_{0} \\
 - \omega_{0} & - \frac{1}{T_{2}}
\end{pmatrix} \Leftrightarrow \begin{pmatrix}
 - \frac{1}{T_{2}} - \lambda & \omega_{0} \\
 - \omega_{0} & - \frac{1}{T_{2}} - \lambda
\end{pmatrix} = \overset{\underline{}}{\overset{\underline{}}{0}}\]

Si pone il determinate della matrice individuata a zero, al fine di identificare i suoi autovettori. Conoscendo gli autovalori è possibile individuare anche il tipo di moto a cui sono soggette le componenti trasverse:

\[\det\begin{pmatrix}
 - \frac{1}{T_{2}} - \lambda & \omega_{0} \\
 - \omega_{0} & - \frac{1}{T_{2}} - \lambda
\end{pmatrix} = 0 \Leftrightarrow \left( - \frac{1}{T_{2}} - \lambda \right)\left( - \frac{1}{T_{2}} - \lambda \right) + \omega_{0}^{2} = 0 \Leftrightarrow \left( \frac{1}{T_{2}} + \lambda \right)^{2} + \omega_{0}^{2} = 0\]

Svolgendo i prodotti si ha:

\[\frac{1}{T_{2}^{2}} + \frac{2\lambda}{T_{2}} + \lambda^{2} + \omega_{0}^{2} = 0 \Leftrightarrow \lambda^{2} + \frac{2\lambda}{T_{2}} + \omega_{0}^{2} + \frac{1}{T_{2}^{2}} = 0\]

Si valuta il delta dell'equazione:

\[\frac{\Delta}{4} = \frac{1}{T_{2}^{2}} - \left( \omega_{0}^{2} + \frac{1}{T_{2}^{2}} \right) = \frac{1}{T_{2}^{2}} - \omega_{0}^{2} - \frac{1}{T_{2}^{2}} = - \omega_{0}^{2}\]

Il determinate è negativo per cui l'equazione non ammette soluzioni reali. Ne consegue che l'evoluzione delle componenti trasverse sono di tipo oscillanti smorzate.

Gli autovalori della matrice dei coefficienti sono:

\[\lambda_{1,2} = - \frac{1}{T_{2}} \pm j\omega_{0} \Leftrightarrow \lambda_{1} = - \frac{1}{T_{2}} - j\omega_{0},\ \ \lambda_{2} = - \frac{1}{T_{2}} + j\omega_{0}\]

Le soluzioni delle componenti trasverse sono del tipo:

\[\left\{ \begin{matrix}
M_{x}(t) = k_{1,x}\exp\left\lbrack \left( - \frac{1}{T_{2}} - j\omega_{0} \right)t \right\rbrack + k_{2,x}\exp\left\lbrack \left( - \frac{1}{T_{2}} + j\omega_{0} \right)t \right\rbrack \\
M_{y}(t) = k_{1,y}\exp\left\lbrack \left( - \frac{1}{T_{2}} - j\omega_{0} \right)t \right\rbrack + k_{2,y}\exp\left\lbrack \left( - \frac{1}{T_{2}} + j\omega_{0} \right)t \right\rbrack
\end{matrix} \right.\ \]

Per le proprietà degli esponenziali, è possibile scrivere:

\[\left\{ \begin{matrix}
M_{x}(t) = k_{1,x}\exp\left( - \frac{t}{T_{2}} \right)\exp\left( - j\omega_{0}t \right) + k_{2,x}\exp\left( - \frac{t}{T_{2}} \right)\exp\left( j\omega_{0}t \right) \\
M_{y}(t) = k_{1,y}\exp\left( - \frac{t}{T_{2}} \right)\exp\left( - j\omega_{0}t \right) + k_{2,y}\exp\left( - \frac{t}{T_{2}} \right)\exp\left( j\omega_{0}t \right)
\end{matrix} \right.\ \]

Raccogliendo il termine dipendente da \(T_{2}\), si ottiene:

\[\left\{ \begin{matrix}
M_{x}(t) = \left\lbrack k_{1,x}\exp\left( - j\omega_{0}t \right) + k_{2,x}\exp\left( j\omega_{0}t \right) \right\rbrack\exp\left( - \frac{t}{T_{2}} \right) \\
M_{y}(t) = \left\lbrack k_{1,y}\exp\left( - j\omega_{0}t \right) + k_{2,y}\exp\left( j\omega_{0}t \right) \right\rbrack\exp\left( - \frac{t}{T_{2}} \right)
\end{matrix} \right.\ \]

Per individuare le costanti di integrazione bisognerebbe sostituire le equazioni individuate nel sistema di equazioni differenziali per le componenti trasverse e applicare le condizioni al contorno.

Per semplificare la trattazione si nota che la dipendenza da \(T_{2}\) è espressa mediante un fattore esponenziale moltiplicativo, quindi, è possibile scrivere le soluzioni delle componenti trasversali come:

\[M_{x}(t) = m_{x}(t)\exp\left( - \frac{t}{T_{2}} \right),\ \ M_{y}(t) = m_{y}(t)\exp\left( - \frac{t}{T_{2}} \right)\]

Le due equazioni differenziali si scrivono come:

\[\left\{ \begin{matrix}
\frac{dM_{x}}{dt} = \omega_{0}M_{y} - \frac{1}{T_{2}}M_{x} \\
\frac{dM_{y}}{dt} = - \omega_{0}M_{x} - \frac{1}{T_{2}}M_{y}
\end{matrix} \right.\  \Leftrightarrow \left\{ \begin{matrix}
\frac{d}{dt}\left\lbrack m_{x}\exp\left( - \frac{t}{T_{2}} \right) \right\rbrack = \omega_{0}m_{y}\exp\left( - \frac{t}{T_{2}} \right) - \frac{1}{T_{2}}m_{x}\exp\left( - \frac{t}{T_{2}} \right) \\
\frac{d}{dt}\left\lbrack m_{y}\exp\left( - \frac{t}{T_{2}} \right) \right\rbrack = - \omega_{0}m_{x}\exp\left( - \frac{t}{T_{2}} \right) - \frac{1}{T_{2}}m_{y}\exp\left( - \frac{t}{T_{2}} \right)
\end{matrix} \right.\ \]

Svolgendo le derivate, si ha:

\[\left\{ \begin{matrix}
\frac{dm_{x}}{dt}\exp\left( - \frac{t}{T_{2}} \right) - \frac{1}{T_{2}}m_{x}\exp\left( - \frac{t}{T_{2}} \right) = \omega_{0}m_{y}\exp\left( - \frac{t}{T_{2}} \right) - \frac{1}{T_{2}}m_{x}\exp\left( - \frac{t}{T_{2}} \right) \\
\frac{dm_{y}}{dt}\exp\left( - \frac{t}{T_{2}} \right) - \frac{1}{T_{2}}m_{y}\exp\left( - \frac{t}{T_{2}} \right) = - \omega_{0}m_{x}\exp\left( - \frac{t}{T_{2}} \right) - \frac{1}{T_{2}}m_{y}\exp\left( - \frac{t}{T_{2}} \right)
\end{matrix} \right.\ \]

Semplificando i termini comuni al primo e al secondo membro e il termine esponenziale dipendente da \(T_{2}\), si ottiene un semplice sistema di equazioni differenziali, la cui soluzione è nota:

\[\left\{ \begin{matrix}
\frac{dm_{x}}{dt} = \omega_{0}m_{y} \\
\frac{dm_{y}}{dt} = - \omega_{0}m_{x}
\end{matrix} \right.\ \]

Si deriva la prima equazione rispetto al tempo:

\[\frac{d^{2}m_{x}}{dt} = \omega_{0}\frac{dm_{y}}{dt}\]

Sostituendo la seconda equazione, si ottiene:

\[\frac{d^{2}m_{x}}{dt} = - \omega_{0}^{2}m_{x}\]

L'equazione differenziale ha come soluzione:

\[m_{x} = C_{1}\exp\left( \lambda_{1}t \right) + C_{2}\exp\left( \lambda_{2}t \right) = C_{1}\exp\left( - j\omega_{0}t \right) + C_{2}\exp\left( j\omega_{0}t \right)\]

Nota \(m_{x}\) è possibile ricavare l'equazione per \(m_{y}\) dalla prima relazione del sistema:

\[\frac{dm_{x}}{dt} = \omega_{0}m_{y} \Leftrightarrow m_{y} = \frac{1}{\omega_{0}}\frac{dm_{x}}{dt} = \frac{1}{\omega_{0}}\frac{d}{dt}\left\lbrack C_{1}\exp\left( - j\omega_{0}t \right) + C_{2}\exp\left( j\omega_{0}t \right) \right\rbrack\]

Svolgendo l'operazione di derivata si ottiene:

\[m_{y} = \frac{1}{\omega_{0}}\left\lbrack - j\omega_{0}C_{1}\exp\left( - j\omega_{0}t \right) + j\omega_{0}C_{2}\exp\left( j\omega_{0}t \right) \right\rbrack = - jC_{1}\exp\left( - j\omega_{0}t \right) + jC_{2}\exp\left( j\omega_{0}t \right)\]

In definitiva, si è ottenuto:

\[\left\{ \begin{matrix}
m_{x} = C_{1}\exp\left( - j\omega_{0}t \right) + C_{2}\exp\left( j\omega_{0}t \right) \\
m_{y} = - jC_{1}\exp\left( - j\omega_{0}t \right) + jC_{2}\exp\left( j\omega_{0}t \right)
\end{matrix} \right.\ \]

Tornando alle componenti trasverse si ha:

\[\left\{ \begin{matrix}
M_{x}(t) = \left\lbrack C_{1}\exp\left( - j\omega_{0}t \right) + C_{2}\exp\left( j\omega_{0}t \right) \right\rbrack\exp\left( - \frac{t}{T_{2}} \right) \\
M_{y}(t) = \left\lbrack - jC_{1}\exp\left( - j\omega_{0}t \right) + jC_{2}\exp\left( j\omega_{0}t \right) \right\rbrack\exp\left( - \frac{t}{T_{2}} \right)
\end{matrix} \right.\ \]

Si suppone che il vettore di magnetizzazione sia noto all'istante \(t_{0}\). In questo modo è possibile individuare le costanti di integrazione:

\[\left\{ \begin{matrix}
M_{x}\left( t_{0} \right) = \left\lbrack C_{1}\exp\left( - j\omega_{0}t_{0} \right) + C_{2}\exp\left( j\omega_{0}t_{0} \right) \right\rbrack\exp\left( - \frac{t_{0}}{T_{2}} \right) \\
M_{y}\left( t_{0} \right) = \left\lbrack - jC_{1}\exp\left( - j\omega_{0}t_{0} \right) + jC_{2}\exp\left( j\omega_{0}t_{0} \right) \right\rbrack\exp\left( - \frac{t_{0}}{T_{2}} \right)
\end{matrix} \right.\ \]

Si divide per il termine esponenziale per entrambe le equazioni:

\[\left\{ \begin{matrix}
M_{x}\left( t_{0} \right)\exp\left( \frac{t_{0}}{T_{2}} \right) = C_{1}\exp\left( - j\omega_{0}t_{0} \right) + C_{2}\exp\left( j\omega_{0}t_{0} \right) \\
M_{y}\left( t_{0} \right)\exp\left( \frac{t_{0}}{T_{2}} \right) = - jC_{1}\exp\left( - j\omega_{0}t_{0} \right) + jC_{2}\exp\left( j\omega_{0}t_{0} \right)
\end{matrix} \right.\ \]

Si divide per l'unità immaginaria nella seconda equazione:

\[\left\{ \begin{matrix}
M_{x}\left( t_{0} \right)\exp\left( \frac{t_{0}}{T_{2}} \right) = C_{1}\exp\left( - j\omega_{0}t_{0} \right) + C_{2}\exp\left( j\omega_{0}t_{0} \right) \\
 - jM_{y}\left( t_{0} \right)\exp\left( \frac{t_{0}}{T_{2}} \right) = - C_{1}\exp\left( - j\omega_{0}t_{0} \right) + C_{2}\exp\left( j\omega_{0}t_{0} \right)
\end{matrix} \right.\ \]

Sommando membro a membro si ottiene:

\[M_{x}\left( t_{0} \right)\exp\left( \frac{t_{0}}{T_{2}} \right) - jM_{y}\left( t_{0} \right)\exp\left( \frac{t_{0}}{T_{2}} \right) = 2C_{2}\exp\left( j\omega_{0}t_{0} \right)\]

Da cui si ricava \(C_{2}\)

\[C_{2} = \frac{1}{2}\left\lbrack M_{x}\left( t_{0} \right)\exp\left( \frac{t_{0}}{T_{2}} \right) - jM_{y}\left( t_{0} \right)\exp\left( \frac{t_{0}}{T_{2}} \right) \right\rbrack\exp\left( - j\omega_{0}t_{0} \right)\]

Sottraendo membro a membro si ricava \(C_{1}\):

\[M_{x}\left( t_{0} \right)\exp\left( \frac{t_{0}}{T_{2}} \right) + jM_{y}\left( t_{0} \right)\exp\left( \frac{t_{0}}{T_{2}} \right) = 2C_{1}\exp\left( - j\omega_{0}t_{0} \right)\]

\[C_{1} = \frac{1}{2}\left\lbrack M_{x}\left( t_{0} \right)\exp\left( \frac{t_{0}}{T_{2}} \right) + jM_{y}\left( t_{0} \right)\exp\left( \frac{t_{0}}{T_{2}} \right) \right\rbrack\exp\left( j\omega_{0}t_{0} \right)\]

La soluzione lungo \({\widehat{i}}_{x}\) è:

\[M_{x}(t) = \left\lbrack C_{1}\exp\left( - j\omega_{0}t \right) + C_{2}\exp\left( j\omega_{0}t \right) \right\rbrack\exp\left( - \frac{t}{T_{2}} \right)\]

Dove:

\[C_{1}\exp\left( - j\omega_{0}t \right) = \frac{1}{2}\left\lbrack M_{x}\left( t_{0} \right)\exp\left( \frac{t_{0}}{T_{2}} \right) + jM_{y}\left( t_{0} \right)\exp\left( \frac{t_{0}}{T_{2}} \right) \right\rbrack\exp\left( j\omega_{0}t_{0} \right)\exp\left( - j\omega_{0}t \right) =\]

\[= \left\lbrack \frac{1}{2}M_{x}\left( t_{0} \right) + j\frac{1}{2}M_{y}\left( t_{0} \right) \right\rbrack\exp\left( \frac{t_{0}}{T_{2}} \right)\exp\left\lbrack - j\omega_{0}\left( t - t_{0} \right) \right\rbrack = \left\lbrack \frac{1}{2}M_{x}\left( t_{0} \right) - \frac{1}{2j}M_{y}\left( t_{0} \right) \right\rbrack\exp\left( \frac{t_{0}}{T_{2}} \right)\exp\left\lbrack - j\omega_{0}\left( t - t_{0} \right) \right\rbrack\]

\[C_{2}\exp\left( j\omega_{0}t \right) = \frac{1}{2}\left\lbrack M_{x}\left( t_{0} \right)\exp\left( \frac{t_{0}}{T_{2}} \right) - jM_{y}\left( t_{0} \right)\exp\left( \frac{t_{0}}{T_{2}} \right) \right\rbrack\exp\left( - j\omega_{0}t_{0} \right)\exp\left( j\omega_{0}t \right) = \left\lbrack \frac{1}{2}M_{x}\left( t_{0} \right) - j\frac{1}{2}M_{y}\left( t_{0} \right) \right\rbrack\exp\left( - j\omega_{0}t_{0} \right)\exp\left\lbrack j\omega_{0}\left( t - t_{0} \right) \right\rbrack = \left\lbrack \frac{1}{2}M_{x}\left( t_{0} \right) + \frac{1}{2j}M_{y}\left( t_{0} \right) \right\rbrack\exp\left( - j\omega_{0}t_{0} \right)\exp\left\lbrack j\omega_{0}\left( t - t_{0} \right) \right\rbrack\]

Quindi, la soluzione lungo \({\widehat{i}}_{x}\) si scrive come:

\[M_{x}(t) = \left\{ \left\lbrack \frac{1}{2}M_{x}\left( t_{0} \right) - \frac{1}{2j}M_{y}\left( t_{0} \right) \right\rbrack\exp\left( \frac{t_{0}}{T_{2}} \right)\exp\left\lbrack - j\omega_{0}\left( t - t_{0} \right) \right\rbrack + \left\lbrack \frac{1}{2}M_{x}\left( t_{0} \right) - j\frac{1}{2}M_{y}\left( t_{0} \right) \right\rbrack\exp\left( - j\omega_{0}t_{0} \right)\exp\left\lbrack j\omega_{0}\left( t - t_{0} \right) \right\rbrack \right\}\exp\left( - \frac{t}{T_{2}} \right)\]

Raccogliendo opportunamente si ha:

\[M_{x}(t) = \left\{ M_{x}\left( t_{0} \right)\frac{\exp\left\lbrack j\omega_{0}\left( t - t_{0} \right) \right\rbrack + \exp\left\lbrack - j\omega_{0}\left( t - t_{0} \right) \right\rbrack}{2} + M_{y}\left( t_{0} \right)\frac{\exp\left\lbrack j\omega_{0}\left( t - t_{0} \right) \right\rbrack - \exp\left\lbrack - j\omega_{0}\left( t - t_{0} \right) \right\rbrack}{2j} \right\}\exp\left( - \frac{t - t_{0}}{T_{2}} \right)\]

Per le relazioni di Eulero, risulta:

\[M_{x}(t) = \left\{ M_{x}\left( t_{0} \right)\cos\left\lbrack \omega_{0}\left( t - t_{0} \right) \right\rbrack + M_{y}\left( t_{0} \right)\sin{\cos\left\lbrack \omega_{0}\left( t - t_{0} \right) \right\rbrack} \right\}\exp\left( - \frac{t - t_{0}}{T_{2}} \right)\]

Analogamente, per la componente lungo \({\widehat{i}}_{y}\) si ha:

\[M_{y}(t) = \left\lbrack - jC_{1}\exp\left( - j\omega_{0}t \right) + jC_{2}\exp\left( j\omega_{0}t \right) \right\rbrack\exp\left( - \frac{t}{T_{2}} \right)\]

Dove:

\[- jC_{1}\exp\left( - j\omega_{0}t \right) = - j\frac{1}{2}\left\lbrack M_{x}\left( t_{0} \right)\exp\left( \frac{t_{0}}{T_{2}} \right) + jM_{y}\left( t_{0} \right)\exp\left( \frac{t_{0}}{T_{2}} \right) \right\rbrack\exp\left( j\omega_{0}t_{0} \right)\exp\left( - j\omega_{0}t \right) = \left\lbrack - j\frac{1}{2}M_{x}\left( t_{0} \right) + \frac{1}{2}M_{y}\left( t_{0} \right) \right\rbrack\exp\left( \frac{t_{0}}{T_{2}} \right)\exp\left\lbrack - j\omega_{0}\left( t - t_{0} \right) \right\rbrack = \left\lbrack \frac{1}{2j}M_{x}\left( t_{0} \right) + \frac{1}{2}M_{y}\left( t_{0} \right) \right\rbrack\exp\left( \frac{t_{0}}{T_{2}} \right)\exp\left\lbrack - j\omega_{0}\left( t - t_{0} \right) \right\rbrack\]

\[jC_{2}\exp\left( j\omega_{0}t \right) = j\frac{1}{2}\left\lbrack M_{x}\left( t_{0} \right)\exp\left( \frac{t_{0}}{T_{2}} \right) - jM_{y}\left( t_{0} \right)\exp\left( \frac{t_{0}}{T_{2}} \right) \right\rbrack\exp\left( - j\omega_{0}t_{0} \right)\exp\left( j\omega_{0}t \right) = = \left\lbrack j\frac{1}{2}M_{x}\left( t_{0} \right) + \frac{1}{2}M_{y}\left( t_{0} \right) \right\rbrack\exp\left( \frac{t_{0}}{T_{2}} \right)\exp\left\lbrack j\omega_{0}\left( t - t_{0} \right) \right\rbrack =\]

\[= \left\lbrack - \frac{1}{2j}M_{x}\left( t_{0} \right) + \frac{1}{2}M_{y}\left( t_{0} \right) \right\rbrack\exp\left( \frac{t_{0}}{T_{2}} \right)\exp\left\lbrack j\omega_{0}\left( t - t_{0} \right) \right\rbrack\]

Quindi, la soluzione lungo \({\widehat{i}}_{y}\) si scrive come:

\[M_{y}(t) = \left\{ \left\lbrack \frac{1}{2j}M_{x}\left( t_{0} \right) + \frac{1}{2}M_{y}\left( t_{0} \right) \right\rbrack\exp\left( \frac{t_{0}}{T_{2}} \right)\exp\left\lbrack - j\omega_{0}\left( t - t_{0} \right) \right\rbrack + \left\lbrack - \frac{1}{2j}M_{x}\left( t_{0} \right) + \frac{1}{2}M_{y}\left( t_{0} \right) \right\rbrack\exp\left( \frac{t_{0}}{T_{2}} \right)\exp\left\lbrack j\omega_{0}\left( t - t_{0} \right) \right\rbrack \right\}\exp\left( - \frac{t}{T_{2}} \right)\]

Raccogliendo opportunamente si ha:

\[M_{y}(t) = \left\{ - M_{x}\left( t_{0} \right)\frac{\exp\left\lbrack j\omega_{0}\left( t - t_{0} \right) \right\rbrack - \exp\left\lbrack - j\omega_{0}\left( t - t_{0} \right) \right\rbrack}{2j} + M_{y}\left( t_{0} \right)\frac{\exp\left\lbrack j\omega_{0}\left( t - t_{0} \right) \right\rbrack + \exp\left\lbrack - j\omega_{0}\left( t - t_{0} \right) \right\rbrack}{2} \right\}\exp\left( - \frac{t - t_{0}}{T_{2}} \right)\]

Per l'equivalenza di Eulero risulta:

\[M_{y}(t) = \left\{ - M_{x}\left( t_{0} \right)\sin\left\lbrack \omega_{0}\left( t - t_{0} \right) \right\rbrack + M_{y}\left( t_{0} \right)\cos\left\lbrack \omega_{0}\left( t - t_{0} \right) \right\rbrack \right\}\exp\left( - \frac{t - t_{0}}{T_{2}} \right)\]

In definitiva, le componenti trasverse evolvono secondo le seguenti equazioni:

\[\left\{ \begin{matrix}
M_{x}(t) = \left\{ M_{x}\left( t_{0} \right)\cos\left\lbrack \omega_{0}\left( t - t_{0} \right) \right\rbrack + M_{y}\left( t_{0} \right)\sin\left\lbrack \omega_{0}\left( t - t_{0} \right) \right\rbrack \right\}\exp\left( - \frac{t - t_{0}}{T_{2}} \right) \\
M_{y}(t) = \left\{ - M_{x}\left( t_{0} \right)\sin\left\lbrack \omega_{0}\left( t - t_{0} \right) \right\rbrack + M_{y}\left( t_{0} \right)\cos\left\lbrack \omega_{0}\left( t - t_{0} \right) \right\rbrack \right\}\exp\left( - \frac{t - t_{0}}{T_{2}} \right)
\end{matrix} \right.\ \]

Se l'istante iniziale coincide con l'origine dei tempi, \(t_{0} = 0\), risulta:

\[\left\{ \begin{matrix}
M_{x}(t) = \left\lbrack M_{x}(0)\cos\left( \omega_{0}t \right) + M_{y}(0)\sin\left( \omega_{0}t \right) \right\rbrack\exp\left( - \frac{t}{T_{2}} \right) \\
M_{y}(t) = \left\lbrack - M_{x}(0)\sin\left( \omega_{0}t \right) + M_{y}(0)\cos\left( \omega_{0}t \right) \right\rbrack\exp\left( - \frac{t - t_{0}}{T_{2}} \right)
\end{matrix} \right.\ \]

Per un tempo sufficientemente lungo, circa \(4 \div 5\) volte \(T_{2}\) è possibile ritenere la risposta transitoria esaurita, per cui le componenti trasversali sono nulle:

\[M_{x}(t) \rightarrow 0,\ \ t \rightarrow \infty\]

\[M_{y}(t) \rightarrow 0,\ \ t \rightarrow \infty\]

Siccome il tempo di rilassamento trasversale è dell'ordine di \(100\ ms\), il tempo necessario affinché le componenti trasversali del vettore di magnetizzazione si annullino è dell'ordine di \(500\ ms\), ovvero un ordine di grandezza inferiore rispetto al tempo che la componente longitudinale impiega per raggiungere il regime.

È possibile scrivere la soluzione delle componenti trasversali in forma complessa, introducendo il fasore \(M_{+}\), definito come:

\[M_{+}(t) = M_{x}(t) + jM_{y}(t) = \left\lbrack M_{x}(0)\cos\left( \omega_{0}t \right) + M_{y}(0)\sin\left( \omega_{0}t \right) - jM_{x}(0)\sin\left( \omega_{0}t \right) + jM_{y}(0)\cos\left( \omega_{0}t \right) \right\rbrack\exp\left( - \frac{t}{T_{2}} \right)\]

Raccogliendo si ottiene:

\[M_{+}(t) = \left\{ M_{x}(0)\left\lbrack \cos\left( \omega_{0}t \right) - j\sin\left( \omega_{0}t \right) \right\rbrack + jM_{y}(0)\left\lbrack \cos\left( \omega_{0}t \right) - j\sin\left( \omega_{0}t \right) \right\rbrack \right\}\exp\left( - \frac{t}{T_{2}} \right) = \left\{ \left\lbrack \cos\left( \omega_{0}t \right) - j\sin\left( \omega_{0}t \right) \right\rbrack\left\lbrack M_{x}(0) + jM_{y}(0) \right\rbrack \right\}\exp\left( - \frac{t}{T_{2}} \right)\]

Dove:

\[M_{+}(0) = M_{x}(0) + jM_{y}(0)\]

Inoltre, per l'identità di Eulero e le proprietà delle funzioni trigonometriche:

\[\cos\left( \omega_{0}t \right) - j\sin\left( \omega_{0}t \right) = j\sin\left( - \omega_{0}t \right) + \cos\left( - \omega_{0}t \right) = \exp\left( - j\omega_{0}t \right)\]

Per cui il fasore si scrive come:

\[M_{+}(t) = M_{+}(0)\exp\left( - j\omega_{0}t \right)\exp\left( - \frac{t}{T_{2}} \right) = M_{+}(0)\exp\left\lbrack - \left( j\omega_{0}t + \frac{t}{T_{2}} \right) \right\rbrack\ \]

Tale relazione fornisce l'evoluzione temporale del vettore di magnetizzazione nel piano complesso. Il moto nel piano trasverso avviene con pulsazione \(\omega_{0}\) e con ampiezza che decade esponenzialmente con constante ti tempo \(T_{2}\). In circa \(500\ ms\) la componente trasversa del vettore di magnetizzazione si annulla.

\begin{figure}
\centering
\includegraphics[width=2.7037in,height=2.1196in,alt={P3496\#yIS1}]{media/6_IntroMRI/image76.pdf}\caption{Figura .: Andamento del vettore di magnetizzazione nel piano trasverso}
\end{figure}

Il vettore di magnetizzazione \(\overset{\underline{}}{M}\), in definitiva, evolve secondo un movimento rotatorio decrescente nel piano trasverso con constante di tempo \(T_{2}\) e con andamento esponenziale crescente, e costante di tempo \(T_{1}\), lungo la direzione longitudinale. Le componenti del vettore di magnetizzazione sono:

\[\left\{ \begin{matrix}
M_{x}(t) = \left\lbrack M_{x}(0)\cos\left( \omega_{0}t \right) + M_{y}(0)\sin\left( \omega_{0}t \right) \right\rbrack\exp\left( - \frac{t}{T_{2}} \right) \\
M_{y}(t) = \left\lbrack - M_{x}(0)\sin\left( \omega_{0}t \right) + M_{y}(0)\cos\left( \omega_{0}t \right) \right\rbrack\exp\left( - \frac{t - t_{0}}{T_{2}} \right) \\
M_{z}(t) = M_{0}\left\lbrack 1 - \exp\left( - \frac{t}{T_{1}} \right) \right\rbrack
\end{matrix} \right.\ \]

La composizione dei due moti determina che il vettore di magnetizzazione ha un modulo variabile nel tempo.

Si considera come condizione iniziale il vettore di magnetizzazione a valle di un ribaltamento nel piano trasverso a opera di un impulso a radiofrequenza. Al tempo \(t = 0\), il vettore di magnetizzazione è:

\[\overset{\underline{}}{M}(0) = M_{x}(0){\widehat{i}}_{x} + M_{y}(0){\widehat{i}}_{y}\]

La componente lungo \({\widehat{i}}_{z}\) parte dal valore nullo e si porta al valore di regime in un tempo di \(5 \div 10\ s\), mentre quelle trasversali vanno a zero in un tempo di \(0.5 \div 1\ s\).

La curva descritta dal vettore di magnetizzazione tende a raggiungere il valore di regime \(M_{0}\) sull'asse delle \({\widehat{i}}_{z}\) mediante un andamento a spirale, convergente sull'asse del campo principale. Il modo elicoidale a raggio variabile lo si ritrova dopo una perturbazione ed è il moto in cui il vettore \(\overset{\underline{}}{M}\) torna all'equilibrio termodinamico per effetto del campo principale \(B_{0}\). I tempi di rilassamento, quindi, interessano il ritorno all'equilibrio.

\begin{figure}
\centering
\includegraphics[width=3.1141in,height=2.12733in,alt={P3505\#yIS1}]{media/6_IntroMRI/image77.pdf}\caption{Figura .: Evoluzione temporale del vettore di magnetizzazione a valle di un ribaltamento sul piano trasverso}
\end{figure}

\subsubsection{Vettore magnetizzazione durante una perturbazione}\label{vettore-magnetizzazione-durante-una-perturbazione}

Si vuole analizzare l'evoluzione temporale del vettore di magnetizzazione durante l'applicazione di un impulso a radiofrequenza, che perturba l'equilibrio del sistema. La trattazione viene eseguita nel sistema rotante in modo da poter essere facilmente comprensibile.

Il campo efficace, visto da uno spin nel sistema di riferimento rotante al quale è applicato un impulso a radiofrequenza diretto lungo l'asse \({\widehat{i}}_{x'}\), è:

\[{\overset{\underline{}}{B}}_{eff} = \left( B_{0} - \frac{\omega}{\gamma} \right){\widehat{i}}_{z} + B_{1}{\widehat{i}}_{x'}\]

Dove \(\omega\) è la velocità di rotazione del sistema.

L'equazione di Bloch, scritte nel sistema di riferimento rotante, è analoga a quella scritta nel sistema fisso del laboratorio:

\[\left( \frac{d\overset{\underline{}}{M}}{dt} \right)' = \gamma\overset{\underline{}}{M} \times {\overset{\underline{}}{B}}_{eff} + \frac{1}{T_{1}}\left( M_{0} - M_{z}\  \right){\widehat{i}}_{z} - \frac{1}{T_{2}}{\overset{\underline{}}{M}}_{\bot}\]

Questa equazione è stata ottenuta senza applicare nessuna ipotesi sul sistema di riferimento, quindi, è valida sia in sistemi inerziali che non inerziali. In particolare, la componente longitudinale del vettore di magnetizzazione evolve in maniera indipendente da quelle trasversali, quindi, il modo lungo l'asse \({\widehat{i}}_{z}\) si conserva.

Si proietta l'equazione vettoriale di Bloch lungo gli assi, svolgendo il prodotto vettoriale:

\[\overset{\underline{}}{M} \times {\overset{\underline{}}{B}}_{eff} = \left| \begin{matrix}
{\widehat{i}}_{x'} & {\widehat{i}}_{y'} & {\widehat{i}}_{z}\  \\
M_{x'} & M_{y'} & M_{z} \\
B_{1} & 0 & B_{0} - \frac{\omega}{\gamma}
\end{matrix} \right| = M_{y'}\left( B_{0} - \frac{\omega}{\gamma} \right){\widehat{i}}_{x'} + M_{z}B_{1}{\widehat{i}}_{y'} - M_{y'}B_{1}{\widehat{i}}_{z} - M_{x'}\left( B_{0} - \frac{\omega}{\gamma} \right){\widehat{i}}_{y'}\]

Raccogliendo i termini, si ottiene:

\[\overset{\underline{}}{M} \times {\overset{\underline{}}{B}}_{eff} = M_{y'}\left( B_{0} - \frac{\omega}{\gamma} \right){\widehat{i}}_{x'} + M_{z}B_{1}{\widehat{i}}_{y'} - M_{y'}B_{1}{\widehat{i}}_{z} - M_{x'}\left( B_{0} - \frac{\omega}{\gamma} \right){\widehat{i}}_{y'} = M_{y'}\left( B_{0} - \frac{\omega}{\gamma} \right){\widehat{i}}_{x'} - \left\lbrack M_{x'}\left( B_{0} - \frac{\omega}{\gamma} \right) - M_{z}B_{1} \right\rbrack{\widehat{i}}_{y'} - M_{y'}B_{1}{\widehat{i}}_{z}\]

L'equazione vettoriale di Bloch si scrive:

\[\left( \frac{d\overset{\underline{}}{M}}{dt} \right)' = \gamma\overset{\underline{}}{M} \times {\overset{\underline{}}{B}}_{eff} + \frac{1}{T_{1}}\left( M_{0} - M_{z}\  \right){\widehat{i}}_{z} - \frac{1}{T_{2}}{\overset{\underline{}}{M}}_{\bot} = \gamma\left\{ M_{y'}\left( B_{0} - \frac{\omega}{\gamma} \right){\widehat{i}}_{x'} - \left\lbrack M_{x'}\left( B_{0} - \frac{\omega}{\gamma} \right) - M_{z}B_{1} \right\rbrack{\widehat{i}}_{y'} - M_{y'}B_{1}{\widehat{i}}_{z} \right\} + \frac{1}{T_{1}}\left( M_{0} - M_{z}\  \right){\widehat{i}}_{z} - \frac{1}{T_{2}}\left( M_{x'}{\widehat{i}}_{x'} + M_{y'}{\widehat{i}}_{y'} \right)\]

Scomponendo lungo gli assi del sistema rotante si ottiene:

\[\left\{ \begin{matrix}
\left( \frac{dM_{x'}}{dt} \right)' = \gamma M_{y'}\left( B_{0} - \frac{\omega}{\gamma} \right) - \frac{1}{T_{2}}M_{x'} \\
\left( \frac{dM_{y'}}{dt} \right)' = \gamma M_{z}B_{1} - \gamma M_{x'}\left( B_{0} - \frac{\omega}{\gamma} \right) - \frac{1}{T_{2}}M_{y'} \\
\left( \frac{dM_{z}}{dt} \right)' = \frac{1}{T_{1}}\left( M_{0} - M_{z}\  \right) - \gamma M_{y'}B_{1}
\end{matrix} \right.\ \]

Dove \(\omega_{1} = \gamma B_{1}\) e \(\omega_{0} = \gamma B_{0}\):

\[\left\{ \begin{matrix}
\left( \frac{dM_{x'}}{dt} \right)' = \left( \omega_{0} - \omega \right)M_{y'} - \frac{1}{T_{2}}M_{x'} \\
\left( \frac{dM_{y'}}{dt} \right)' = \omega_{1}M_{z} - \left( \omega_{0} - \omega \right)M_{x'} - \frac{1}{T_{2}}M_{y'} \\
\left( \frac{dM_{z}}{dt} \right)' = \frac{1}{T_{1}}\left( M_{0} - M_{z}\  \right) - \omega_{1}M_{y'}
\end{matrix} \right.\ \]

Nelle equazioni \(\omega_{0}\) è la frequenza di Larmor, \(\omega_{1}\) è la frequenza del campo a radiofrequenza, mentre \(\omega\) è la frequenza con cui ruota il sistema di riferimento.

Si definisce \(\mathrm{\Delta}\omega = \omega_{0} - \omega\) e rappresenta la deviazione dalla condizione ideale. Questa deviazione è legata alle disomogeneità del campo o alla variazioni delle frequenze dell'impulso utilizzato. Affinché il vettore di magnetizzazione ruoti dell'angolo desiderato, è necessario che l'impulso a radiofrequenza abbia una durata di qualche \(ms\). Nel dettaglio, il periodo di applicazione del campo a radiofrequenza, che genera la processione intorno a \({\widehat{i}}_{x'}\), è di qualche millisecondo ed è legato alla pulsazione \(\omega_{1}\) dalla relazione:

\[\omega_{1} = \frac{2\pi}{T}\]

Dato che il tempo di rilassamento longitudinale \(T_{1}\) è dell'ordine dei secondi, risulta che:

\[T \ll T_{1} \Leftrightarrow \frac{1}{T} \gg \frac{1}{T_{1}}\]

A meno di un fattore \(2\pi\), risulta:

\[\frac{2\pi}{T} = \omega_{1} \gg \frac{1}{T_{1}}\]

Analogo discorso vale per il tempo di rilassamento trasversale \(T_{2}\), dell'ordine dei \(500\ ms\). Rispetto alla pulsazione \(\omega_{1}\), i termini che evolvono con costanti di tempo \(T_{1}\) e \(T_{2}\) possono essere trascurati, in quanto molto più lenti. L'evoluzione del vettore di magnetizzazione, in ultima analisi, non dipende dai tempi di rilassamento:

\[\left\{ \begin{matrix}
\left( \frac{dM_{x'}}{dt} \right)' = \left( \omega_{0} - \omega \right)M_{y'} \\
\left( \frac{dM_{y'}}{dt} \right)' = \omega_{1}M_{z} - \left( \omega_{0} - \omega \right)M_{x'} \\
\left( \frac{dM_{z}}{dt} \right)' = - \omega_{1}M_{y'}
\end{matrix} \right.\ \]

Se il sistema ruota con una pulsazione angolare molto prossima a quella di Larmor risulta:

\[\omega_{0} \simeq \omega \Leftrightarrow \mathrm{\Delta}\omega = \omega_{0} - \omega \simeq 0\]

È possibile, quindi, trascurare i termini contenenti \(\mathrm{\Delta}\omega\):

\[\left\{ \begin{matrix}
\left( \frac{dM_{x'}}{dt} \right)' = 0 \\
\left( \frac{dM_{y'}}{dt} \right)' = \omega_{1}M_{z} \\
\left( \frac{dM_{z}}{dt} \right)' = - \omega_{1}M_{y'}
\end{matrix} \right.\ \]

Le equazioni individuate suggeriscono un moto di precessione intorno l'asse \({\widehat{i}}_{x'}\).

Per le sequenze applicate normalmente nella pratica, in definitiva, si ritiene che la rotazione del vettore di magnetizzazione avvenga senza l'influenza dei tempi di rilassamento, poiché l'evoluzione temporale legata all'impulso a radiofrequenza è molto più veloce di quella legata ai fenomeni di rilassamento. Si approssima, inoltre, la frequenza di risonanza con quella del campo a radiofrequenza. Anche in presenza di tali approssimazioni, i risultati ottenuti sono attendibili, nel senso che concordi ai risultati sperimentali. Il ritorno all'equilibrio è, invece, caratterizzato dai tempi di rilassamento del tessuto.

L'evoluzione del vettore di magnetizzazione, legato all'applicazione dell'impulso a radiofrequenza, si compone di due fasi:

\begin{itemize}
\item
  Raggiunto l'equilibrio termodinamico, il vettore di magnetizzazione è ribaltato lungo uno degli assi \(x'\) o \(y'\), utilizzando un impulso a radiofrequenza a polarizzazione lineare o circolare. Come detto precedentemente, il ribaltamento è descritto trascurando gli effetti del rilassamento, poiché i tempi \(T_{1}\) e \(T_{2}\) sono molto più lunghi della durata dell'impulso a radiofrequenza;
\end{itemize}

\begin{longtable}[]{@{}
  >{\raggedright\arraybackslash}p{(\linewidth - 2\tabcolsep) * \real{0.5000}}
  >{\raggedright\arraybackslash}p{(\linewidth - 2\tabcolsep) * \real{0.5000}}@{}}
\caption{Figura .: L'applicazione dell'impulso RF porta a una precessione intorno all'asse \(y'\)}\tabularnewline
\toprule\noalign{}
\begin{minipage}[b]{\linewidth}\centering
\includegraphics[width=3.17495in,height=3.19514in,alt={P3542C1T5\#yIS1}]{media/6_IntroMRI/image78.pdf}\end{minipage} & \begin{minipage}[b]{\linewidth}\centering
\includegraphics[width=2.90572in,height=3.19514in,alt={P3543C2T5\#yIS1}]{media/6_IntroMRI/image79.pdf}\end{minipage} \\
\midrule\noalign{}
\endfirsthead
\toprule\noalign{}
\begin{minipage}[b]{\linewidth}\centering
\includegraphics[width=3.17495in,height=3.19514in,alt={P3542C1T5\#yIS1}]{media/6_IntroMRI/image78.pdf}\end{minipage} & \begin{minipage}[b]{\linewidth}\centering
\includegraphics[width=2.90572in,height=3.19514in,alt={P3543C2T5\#yIS1}]{media/6_IntroMRI/image79.pdf}\end{minipage} \\
\midrule\noalign{}
\endhead
\bottomrule\noalign{}
\endlastfoot
\end{longtable}

\begin{itemize}
\item
  Il fenomeno di recupero della magnetizzazione è caratterizzato dal ritorno all'equilibrio della magnetizzazione e avviene con costante di tempo \(T_{1}\), tempo di rilassamento per la componente longitudinale e costante di tempo \(T_{2}\) per la componente trasversale. I tempi di evoluzione del vettore di magnetizzazione dono paragonabili a quelli di rilassamento per cui non possono essere trascurati.
\end{itemize}

\subsubsection{Sequenza FID}\label{sequenza-fid}

Per ottenere una misura del vettore di magnetizzazione, è necessario perturbare l'equilibrio raggiunto dagli spin contenuti nei tessuti del paziente. Dal segnale registrato è possibile, in seguito, ricavare le informazioni sui tempi di rilassamento \emph{spin-lattice}, \(T_{1}\), e spin-spin \(T_{2}\), così da caratterizzare completamente il tessuto.

La sequenza più semplice da applicare per perturbare il sistema è detta FID (\emph{Free Induction Decay}) in cui si applica una radiazione a radiofrequenza, indicata nei diagrammi con una \(rect\) o con un pacchetto di onde sinusoidali, a frequenza \(\omega \simeq \omega_{0}\).

Successivamente, si registra il segnale \(s\) dovuto al ritorno all'equilibrio del vettore magnetizzazione. Le antenne in ricezione iniziano ad acquisire il segnale subito dopo la fine dell'impulso.

Il segnale registrato è proporzionale alla componente trasversa del campo, dunque, le antenne ricevono una sinusoide, a frequenza \(\omega_{0}\) smorzata con un decadimento di tipo esponenziale e costante di tempo \(T_{2}\). Adoperando la notazione complessa, il segnale registrato dall'antenna è proporzionale al fasore \(M_{+}(t)\):

\[M_{+}(t) = M_{+}(0)\exp\left( - \frac{t}{T_{2}} \right)\exp\left( - j\omega_{0}t \right)\]

Dove \(M_{+}(0)\) è proporzionale al vettore di magnetizzazione all'equilibrio termodinamico \(M_{0}\), che a sua volta dipende dalla densità protonica \(\rho\), indicante il numero di protoni nel volumetto \(V\).

Siccome non vi è nessun gradiente di campo, ogni spin all'interno del volume del paziente precede alla frequenza di Larmor \(\omega_{0}\), dunque, l'applicazione del campo rotante a frequenza prossima a quella di Larmor eccita tutti gli spin nel corpo del paziente. In questo caso, il volume-paziente non è considerato in tanti volumetti elementari ma è assunto essere un unico volume.

La sequenza FID non fornisce informazioni sulla composizione chimica dei tessuti, ovvero non fornisce i tempi di rilassamento di ogni singolo volumetto. In altre parole, la sequenza FID non fornire informazioni utili all'\emph{imaging} ma permette di correggere degli errori introdotti dalle disomogeneità del campo magnetico principale. Infatti, le misure eseguite con la sequenza FID possono fornire un'idea su quanto il campo magnetico principale si discosta dall'andamento ideale, ovvero uniforme in tutto lo spazio contenente il paziente.

Il tempo di rilassamento trasversale \(T_{2}\) è determinato dall'iterazioni spin-spin, che ha l'effetto di alterare localmente il campo magnetico visto da uno spin. Se il campo magnetico esterno è disomogeneo, il campo locale visto dagli spin varia con la posizione, dunque, si aggiunge un ulteriore fonte di disturbo.

In generale, i fornitori di risonanze magnetiche garantiscono che l'omogeneità del campo nel gantry abbia una valore nominale variabile di una parte per milione (ppm), all'interno di una sfera centrata nel gantry e raggio dell'ordine dei \(20\ cm\).

\begin{figure}
\centering
\includegraphics[width=3.73125in,height=2.04747in,alt={P3558\#yIS1}]{media/6_IntroMRI/image80.pdf}\caption{Figura .: Regione di spazio del gantry in cui il campo è omogeneo}
\end{figure}

Se il valore nominale del campo è di \(1.5\ T\), la variazione di campo all'interno della sfera è dell'ordine di \(10^{- 6}\ \), ovvero una variazione massima di \(1.5\ \mu T\). Queste disomogeneità di campo si aggiungono ai campi locali visti dai singoli spin. Ciò porta a un'ulteriore riduzione del tempo di rilassamento trasversale \(T_{2}\).

Si introduce il tempo \(T_{2}^{*}\) legato sia alla disomogeneità del campo principale, sia all'interazione spin-spin. Il primo fenomeno è quantificato da un tempo \(T_{2}'\), legato alla tecnologia costruttiva con cui si produce il campo magnetico, che, fondamentalmente, definisce le disomogeneità di \(B_{0}\).

Il tempo \(T_{2}^{*}\) è definito come:

\[\frac{1}{T_{2}^{*}} = \frac{1}{T_{2}} + \frac{1}{T_{2}'}\]

Con la sequenza FID è possibile ottenere delle informazioni su \(T_{2}^{*}\) e, di conseguenza, su quanto il campo magnetico principale varia nello spazio. Noto il tempo \(T_{2}^{*}\), mediante appositi algoritmi, è possibile correggere le immagini, le quali mostrano una maggiore affidabilità nella stima dei tempi di rilassamento \(T_{1}\) e \(T_{2}\).

\begin{center}
\vfill
    \chapter{Segnale della risonanza magnetica}
    \label{blx:refsection\therefsection}
\vfill

\minitoc
\newpage
\end{center}
\justify


\section{Segnale registrato in RMI}\label{segnale-registrato-in-rmi}

Per ottenere una misura del vettore di magnetizzazione, è necessario perturbare l'equilibrio raggiunto dagli spin contenuti nei tessuti del paziente. Una volta perturbato il sistema mediante un impulso a radiofrequenze, il vettore di magnetizzazione ritorna al valore di equilibrio, secondo un'evoluzione dettata dai tempi di rilassamento spin-reticolo, \(T_{1}\), e spin-spin, \(T_{2}\).

All'interno del gantry vi sono delle antenne (\emph{RF coil}) poste ortogonalmente tra loro in modo da produrre un campo a polarizzazione circolare. Nello specifico, le antenne possono essere pensate, in linea di principio, come due spire percorse da corrente poste una nel piano verticale e l'altra nel piano orizzontale rispetto al corpo del paziente.

\begin{figure}
\centering
\includegraphics[width=6.27171in,height=4.22976in,alt={P3570\#yIS1}]{media/7_MRISignal/image81.pdf}\caption{Figura .: Schema strutturale di gantry per risonanza magnetica}
\end{figure}

Nel gantry sono presenti altre bobine utilizzate per generare un gradiente di campo, utile per variare la frequenza di Larmor lungo la direzione \({\widehat{i}}_{z}\) con un andamento noto. In questo modo è possibile selezionare una singola \emph{slice} del corpo umano.

Una volta terminata l'erogazione del campo a radiofrequenza, la magnetizzazione si rilassa, ovvero torna all'equilibrio termodinamico con il campo principale. Durante la fase di ritorno all'equilibrio è possibile captare i segnali con apposite antenne, spesso le stesse utilizzate per la trasmissione del campo a radiofrequenza ma possono essere anche atre antenne.

Le antenne posso essere pensare come spire percorse da corrente su cui il rilassamento della magnetizzazione induce una fem. (o emf.) per la legge di Faraday-Neumann-Lentz. Tramite la fem. indotta è possibile ricostruire immagini di sezioni del corpo umano.

\subsection{Valutazione fem. indotta sulle antenne riceventi}\label{valutazione-fem.-indotta-sulle-antenne-riceventi}

Al fine di ricostruire immagini di sezioni del paziente è necessario capire come la fem. indotta sull'antenna ricevente sia legata alle variazioni di magnetizzazione, durante il ritorno all'equilibrio termodinamico del vettore di magnetizzazione.

Per la legge di Faraday-Neumann-Lent, la fem. indotta sull'antenna ricevente è:

\[emf = - \dfrac{d}{dt}\Phi_{S}(B)\]

Dove \(\Phi_{S}(B)\) è il flusso del campo magnetico concatenato con l'antenna di area \(S\):

\[\Phi_{S}(B) = \int_{S}^{}{\overset{\underline{}}{B} \cdot d\overset{\underline{}}{S}}\]

Dato un volumetto \(V'\) di spin, il vettore di magnetizzazione di tale volumetto si concatena con la spira di area \(S\). Le variazioni del vettore di magnetizzazione inducono una fem. sulla spira.

In generale, il campo magnetico è legato al potenziale vettore \(\overset{\underline{}}{A}\) dalla definizione:

\[\overset{\underline{}}{B} = \overset{\underline{}}{\nabla} \times \overset{\underline{}}{A}\ \]

Da cui la fem.:

\[emf = - \dfrac{d}{dt}\Phi_{S}(B) = - \dfrac{d}{dt}\int_{S}^{}{\overset{\underline{}}{B} \cdot d\overset{\underline{}}{S}} = - \dfrac{d}{dt}\int_{S}^{}{\overset{\underline{}}{\nabla} \times \overset{\underline{}}{A} \cdot d\overset{\underline{}}{S}}\]

Si dimostra che il potenziale vettore in un certo punto \(\overset{\underline{}}{r}\), esterno alla regione contenente le sorgenti, è dato da:

\[\overset{\underline{}}{A}\left( \overset{\underline{}}{r} \right) = \dfrac{\mu_{0}}{4\pi}\int_{V'}^{}{\dfrac{1}{\left| \overset{\underline{}}{r} - {\overset{\underline{}}{r}}' \right|}\overset{\underline{}}{J}\left( {\overset{\underline{}}{r}}' \right)dV'}\]

Dove \({\overset{\underline{}}{r}}'\) è un punto appartenente al volumetto elementare di spin.

\begin{figure}
\centering
\includegraphics[width=3.32351in,height=3.05in,alt={P3589\#yIS1}]{media/7_MRISignal/image82.pdf}\caption{Figura .: Distanze tra volumetto elementare e antenna}
\end{figure}

Il sistema di riferimento è scelto in modo tale che in \({\overset{\underline{}}{r}}'\) vi sia una certa densità di corrente \(\overset{\underline{}}{J}\left( {\overset{\underline{}}{r}}' \right)\), sorgente del campo infinitesimo \(d\overset{\underline{}}{B}\) indotto nella posizione \(\overset{\underline{}}{r}\) sull'antenna.

L'integrale che permette di calcolare il potenziale vettore è valutato su \({\overset{\underline{}}{r}}'\) ed è risolvibile note le sorgenti \(\overset{\underline{}}{J}\), ovvero la densità di corrente nel volumetto. La quantità \(\left| \overset{\underline{}}{r} - {\overset{\underline{}}{r}}' \right|\) rappresenta la distanza tra il punto di osservazione \(\overset{\underline{}}{r}\) rispetto i punti del volumetti \({\overset{\underline{}}{r}}'\), soggetti alla densità di corrente \(\overset{\underline{}}{J}\).

È noto che, per effetto dei campi magnetici applicati, si generano delle densità di correnti vincolate date dalla relazione:

\[{\overset{\underline{}}{J}}_{vinc}\left( {\overset{\underline{}}{r}}' \right) = {\overset{\underline{}}{\nabla}}' \times \overset{\underline{}}{M}\left( {\overset{\underline{}}{r}}' \right)\]

Dove la notazione \({\overset{\underline{}}{\nabla}}' \times\) indica che il rotore è valutato rispetto la coordinata \({\overset{\underline{}}{r}}'\).

Sull'espressione della fem. indotta è possibile applicare il teorema di Stokes, con il quale il flusso del rotore può essere scritto come la circuitazione del potenziale vettore \(\overset{\underline{}}{A}\) sulla linea che rappresenta il contorno \(\partial S\), della superficie \(S\) della spira:

\[emf = - \dfrac{d}{dt}\int_{S}^{}{\overset{\underline{}}{\nabla} \times \overset{\underline{}}{A} \cdot d\overset{\underline{}}{S}} = - \dfrac{d}{dt}\oint_{l}^{}{\overset{\underline{}}{A} \cdot d\overset{\underline{}}{l}}\]

Dove si pone \(l = \partial S\). Si sostituisce l'espressione per il potenziale vettore in funzione delle densità di correnti:

\[emf = - \dfrac{d}{dt}\oint_{l}^{}{\overset{\underline{}}{A} \cdot d\overset{\underline{}}{l}} = - \dfrac{d}{dt}\oint_{l}^{}{\left\lbrack \dfrac{\mu_{0}}{4\pi}\int_{V'}^{}{\dfrac{1}{\left| \overset{\underline{}}{r} - {\overset{\underline{}}{r}}' \right|}\overset{\underline{}}{J}\left( {\overset{\underline{}}{r}}' \right)dV'} \right\rbrack \cdot d\overset{\underline{}}{l}}\]

Inoltre, le densità di corrente nel volumetto sono di tipo vincolato, dunque, l'espressione per la fem. si può scrive come:

\[emf = - \dfrac{\mu_{0}}{4\pi}\dfrac{d}{dt}\oint_{l}^{}{\left\lbrack \int_{V'}^{}{\dfrac{1}{\left| \overset{\underline{}}{r} - {\overset{\underline{}}{r}}' \right|}\overset{\underline{}}{J}\left( {\overset{\underline{}}{r}}' \right)dV'} \right\rbrack \cdot d\overset{\underline{}}{l}} = - \dfrac{\mu_{0}}{4\pi}\dfrac{d}{dt}\oint_{l}^{}{\left\lbrack \int_{V'}^{}{\dfrac{{\overset{\underline{}}{\nabla}}' \times \overset{\underline{}}{M}\left( {\overset{\underline{}}{r}}' \right)}{\left| \overset{\underline{}}{r} - {\overset{\underline{}}{r}}' \right|}dV'} \right\rbrack \cdot d\overset{\underline{}}{l}}\]

È possibile applicare la relazione:

\[\overset{\underline{}}{\nabla} \times \left\lbrack f\left( \overset{\underline{}}{r} \right)\overset{\underline{}}{a}\left( \overset{\underline{}}{r} \right) \right\rbrack = \nabla f\left( \overset{\underline{}}{r} \right) \times \overset{\underline{}}{a}\left( \overset{\underline{}}{r} \right) + f\left( \overset{\underline{}}{r} \right)\nabla \times \overset{\underline{}}{a}\left( \overset{\underline{}}{r} \right)\]

Dove \(f\left( \overset{\underline{}}{r} \right)\) è una qualsiasi funzione scalare e \(\overset{\underline{}}{a}\left( \overset{\underline{}}{r} \right)\) una funzione vettoriale. Da questa relazione si ricava \(f\left( \overset{\underline{}}{r} \right)\nabla \times \overset{\underline{}}{a}\left( \overset{\underline{}}{r} \right)\):

\[f\left( \overset{\underline{}}{r} \right)\overset{\underline{}}{\nabla} \times \overset{\underline{}}{a}\left( \overset{\underline{}}{r} \right) = \overset{\underline{}}{\nabla} \times \left\lbrack f\left( \overset{\underline{}}{r} \right)\overset{\underline{}}{a}\left( \overset{\underline{}}{r} \right) \right\rbrack - \overset{\underline{}}{\nabla}f\left( \overset{\underline{}}{r} \right) \times \overset{\underline{}}{a}\left( \overset{\underline{}}{r} \right)\]

Si pone:

\[f\left( \overset{\underline{}}{r} \right) = \dfrac{1}{\left| \overset{\underline{}}{r} - {\overset{\underline{}}{r}}' \right|},\ \ \overset{\underline{}}{a}\left( \overset{\underline{}}{r} \right) = {\overset{\underline{}}{\nabla}}' \times \overset{\underline{}}{M}\left( {\overset{\underline{}}{r}}' \right)\]

Con questa posizione l'identità può essere scritta come:

\[\dfrac{1}{\left| \overset{\underline{}}{r} - {\overset{\underline{}}{r}}' \right|}\overset{\underline{}}{\nabla} \times \overset{\underline{}}{M}\left( {\overset{\underline{}}{r}}' \right) = \overset{\underline{}}{\nabla} \times \left\lbrack \dfrac{1}{\left| \overset{\underline{}}{r} - {\overset{\underline{}}{r}}' \right|}\overset{\underline{}}{M}\left( {\overset{\underline{}}{r}}' \right) \right\rbrack - \overset{\underline{}}{\nabla}\left( \dfrac{1}{\left| \overset{\underline{}}{r} - {\overset{\underline{}}{r}}' \right|} \right) \times \overset{\underline{}}{M}\left( {\overset{\underline{}}{r}}' \right)\]

L'espressione per l'\emph{electromotive force} può essere scritta come:

\[emf = - \dfrac{\mu_{0}}{4\pi}\dfrac{d}{dt}\oint_{l}^{}{\left\lbrack \int_{V'}^{}{\left\{ {\overset{\underline{}}{\nabla}}' \times \left\lbrack \dfrac{1}{\left| \overset{\underline{}}{r} - {\overset{\underline{}}{r}}' \right|}\overset{\underline{}}{M}\left( {\overset{\underline{}}{r}}' \right) \right\rbrack - \overset{\underline{}}{\nabla'}\left( \dfrac{1}{\left| \overset{\underline{}}{r} - {\overset{\underline{}}{r}}' \right|} \right) \times \overset{\underline{}}{M}\left( {\overset{\underline{}}{r}}' \right) \right\} dV'} \right\rbrack \cdot d\overset{\underline{}}{l}} = - \dfrac{\mu_{0}}{4\pi}\dfrac{d}{dt}\oint_{l}^{}{\left\lbrack \int_{V'}^{}{{\overset{\underline{}}{\nabla}}' \times \left\lbrack \dfrac{1}{\left| \overset{\underline{}}{r} - {\overset{\underline{}}{r}}' \right|}\overset{\underline{}}{M}\left( {\overset{\underline{}}{r}}' \right) \right\rbrack dV'} - \int_{V'}^{}{\overset{\underline{}}{\nabla'}\left( \dfrac{1}{\left| \overset{\underline{}}{r} - {\overset{\underline{}}{r}}' \right|} \right) \times \overset{\underline{}}{M}\left( {\overset{\underline{}}{r}}' \right)}dV' \right\rbrack \cdot d\overset{\underline{}}{l}}\]

Si considera l'integrale:

\[\int_{V'}^{}{{\overset{\underline{}}{\nabla}}' \times \left\lbrack \dfrac{1}{\left| \overset{\underline{}}{r} - {\overset{\underline{}}{r}}' \right|}\overset{\underline{}}{M}\left( {\overset{\underline{}}{r}}' \right) \right\rbrack dV'}\]

Si può dimostrare che, per un teorema simile a quello di Stokes, l'integrale considerato è uguale all'integrale calcolato sulla frontiera del volumetto della funzione di cui si applica il rotore, vettor la normale della frontiera. In altre parola, è valida la relazione:

\[\int_{V'}^{}{{\overset{\underline{}}{\nabla}}' \times \left\lbrack \dfrac{1}{\left| \overset{\underline{}}{r} - {\overset{\underline{}}{r}}' \right|}\overset{\underline{}}{M}\left( {\overset{\underline{}}{r}}' \right) \right\rbrack dV'} = \int_{\partial V'}^{}{\widehat{n} \times \left\lbrack \dfrac{1}{\left| \overset{\underline{}}{r} - {\overset{\underline{}}{r}}' \right|}\overset{\underline{}}{M}\left( {\overset{\underline{}}{r}}' \right) \right\rbrack dS'}\]

Si ottiene così, un integrale superficiale. La relazione è sempre valida, quindi, vale anche per il volumetto \(V\), su cui effettuare l'integrale, leggermente più grande del volumetto \(V'\), contenente le sorgenti del campo. All'esterno del volumetto la magnetizzazione è nulla, di conseguenza l'integrale di flusso è nullo, in quanto il vettore di magnetizzazione è nullo sulla superficie.

\begin{figure}
\centering
\includegraphics[width=1.46875in,height=1.26687in,alt={P3617\#yIS1}]{media/7_MRISignal/image83.pdf}\caption{Figura .: Volume \(V\) su cui calcolare l'integrale, leggermente più grande di \(V'\) contenente gli spin}
\end{figure}

La forza elettromotrice si riduce a:

\[emf = - \dfrac{\mu_{0}}{4\pi}\dfrac{d}{dt}\oint_{l}^{}{\left\lbrack - \int_{V'}^{}{\overset{\underline{}}{\nabla'}\left( \dfrac{1}{\left| \overset{\underline{}}{r} - {\overset{\underline{}}{r}}' \right|} \right) \times \overset{\underline{}}{M}\left( {\overset{\underline{}}{r}}' \right)dV'} \right\rbrack \cdot d\overset{\underline{}}{l}} = \dfrac{\mu_{0}}{4\pi}\dfrac{d}{dt}\oint_{l}^{}{\int_{V'}^{}{\overset{\underline{}}{\nabla'}\left( \dfrac{1}{\left| \overset{\underline{}}{r} - {\overset{\underline{}}{r}}' \right|} \right) \times \overset{\underline{}}{M}\left( {\overset{\underline{}}{r}}' \right)dV'} \cdot d\overset{\underline{}}{l}}\]

Per la linearità è possibile invertire gli integrali su \(d\overset{\underline{}}{l}\) e \(dV'\):

\[emf = \dfrac{\mu_{0}}{4\pi}\dfrac{d}{dt}\int_{V'}^{}{\oint_{l}^{}{\overset{\underline{}}{\nabla'}\left( \dfrac{1}{\left| \overset{\underline{}}{r} - {\overset{\underline{}}{r}}' \right|} \right) \times \overset{\underline{}}{M}\left( {\overset{\underline{}}{r}}' \right) \cdot d\overset{\underline{}}{l}}dV'}\]

Nell'integrale della fem. compare un prodotto misto, per cui valgono le identità:

\[\overset{\underline{}}{\nabla'}\left( \dfrac{1}{\left| \overset{\underline{}}{r} - {\overset{\underline{}}{r}}' \right|} \right) \times \overset{\underline{}}{M}\left( {\overset{\underline{}}{r}}' \right) \cdot d\overset{\underline{}}{l} = \overset{\underline{}}{M}\left( {\overset{\underline{}}{r}}' \right) \times d\overset{\underline{}}{l} \cdot \overset{\underline{}}{\nabla'}\left( \dfrac{1}{\left| \overset{\underline{}}{r} - {\overset{\underline{}}{r}}' \right|} \right) = d\overset{\underline{}}{l} \times \overset{\underline{}}{\nabla'}\left( \dfrac{1}{\left| \overset{\underline{}}{r} - {\overset{\underline{}}{r}}' \right|} \right) \cdot \overset{\underline{}}{M}\left( {\overset{\underline{}}{r}}' \right)\]

L'ultimo termine può essere scritto come:

\[\overset{\underline{}}{\nabla'}\left( \dfrac{1}{\left| \overset{\underline{}}{r} - {\overset{\underline{}}{r}}' \right|} \right) \times \overset{\underline{}}{M}\left( {\overset{\underline{}}{r}}' \right) \cdot d\overset{\underline{}}{l} = - \overset{\underline{}}{M}\left( {\overset{\underline{}}{r}}' \right) \cdot \overset{\underline{}}{\nabla'}\left( \dfrac{1}{\left| \overset{\underline{}}{r} - {\overset{\underline{}}{r}}' \right|} \right) \times d\overset{\underline{}}{l}\]

Dalla relazione \(\overset{\underline{}}{\nabla}f\left( \overset{\underline{}}{r} \right) \times \overset{\underline{}}{a}\left( \overset{\underline{}}{r} \right) = \overset{\underline{}}{\nabla} \times \left\lbrack f\left( \overset{\underline{}}{r} \right)\overset{\underline{}}{a}\left( \overset{\underline{}}{r} \right) \right\rbrack - f\left( \overset{\underline{}}{r} \right)\overset{\underline{}}{\nabla} \times \overset{\underline{}}{a}\left( \overset{\underline{}}{r} \right)\) è possibile scrivere:

\[\overset{\underline{}}{\nabla'}\left( \dfrac{1}{\left| \overset{\underline{}}{r} - {\overset{\underline{}}{r}}' \right|} \right) \times d\overset{\underline{}}{l} = \overset{\underline{}}{\nabla} \times \left( \dfrac{1}{\left| \overset{\underline{}}{r} - {\overset{\underline{}}{r}}' \right|}d\overset{\underline{}}{l} \right) - \dfrac{1}{\left| \overset{\underline{}}{r} - {\overset{\underline{}}{r}}' \right|}\overset{\underline{}}{\nabla} \times \ d\overset{\underline{}}{l}\]

La fem. può essere scritta come:

\[emf = \dfrac{\mu_{0}}{4\pi}\dfrac{d}{dt}\int_{V'}^{}{\oint_{l}^{}\left\lbrack - \overset{\underline{}}{M}\left( {\overset{\underline{}}{r}}' \right) \cdot {\overset{\underline{}}{\nabla}}' \times \left( \dfrac{1}{\left| \overset{\underline{}}{r} - {\overset{\underline{}}{r}}' \right|}d\overset{\underline{}}{l} \right) + \overset{\underline{}}{M}\left( {\overset{\underline{}}{r}}' \right) \cdot \dfrac{1}{\left| \overset{\underline{}}{r} - {\overset{\underline{}}{r}}' \right|}{\overset{\underline{}}{\nabla}}' \times \ d\overset{\underline{}}{l} \right\rbrack}dV'\]

\(d\overset{\underline{}}{l}\) è esterno al volumetto elementare, poiché versore che agisce sulla spira, dunque, \({\overset{\underline{}}{\nabla}}' \times \ d\overset{\underline{}}{l} = \overset{\underline{}}{0}\):

\[emf = - \dfrac{\mu_{0}}{4\pi}\dfrac{d}{dt}\int_{V'}^{}{\overset{\underline{}}{M}\left( {\overset{\underline{}}{r}}' \right) \cdot {\overset{\underline{}}{\nabla}}' \times \oint_{l}^{}{\dfrac{1}{\left| \overset{\underline{}}{r} - {\overset{\underline{}}{r}}' \right|}d\overset{\underline{}}{l}}}dV'\]

La circuitazione non opera all'interno del volumetto, dunque, su \({\overset{\underline{}}{r}}'\) ma solo sulla spira, quindi su \(\overset{\underline{}}{r}\). Di conseguenza, è possibile portare \(\overset{\underline{}}{M}\left( {\overset{\underline{}}{r}}' \right)\) all'esterno del simbolo di circuitazione.

Nell'espressione della fem. compare il termine:

\[\dfrac{\mu_{0}}{4\pi}\oint_{l}^{}{\dfrac{1}{\left| \overset{\underline{}}{r} - {\overset{\underline{}}{r}}' \right|}d\overset{\underline{}}{l}}\]

È noto che il potenziale vettore generato da una corrente filiforme \(I\), che scorre nell'elemento infinitesimo di antenna \(d\overset{\underline{}}{l}\), è dato da:

\[\overset{\underline{}}{A}\left( {\overset{\underline{}}{r}}' \right) = \dfrac{\mu_{0}}{4\pi}\oint_{l}^{}{\dfrac{I}{\left| \overset{\underline{}}{r} - {\overset{\underline{}}{r}}' \right|}d\overset{\underline{}}{l}}\]

L'integrale nel calcolo della fem. è, quindi, il potenziale vettore nel punto \({\overset{\underline{}}{r}}'\) della spira usata come antenna quando in essa scorre una corrente unitaria. L'espressione della fem. si scrive come:

\[emf = - \dfrac{\mu_{0}}{4\pi}\dfrac{d}{dt}\int_{V'}^{}{\overset{\underline{}}{M}\left( {\overset{\underline{}}{r}}' \right) \cdot {\overset{\underline{}}{\nabla}}' \times \oint_{l}^{}{\dfrac{1}{\left| \overset{\underline{}}{r} - {\overset{\underline{}}{r}}' \right|}d\overset{\underline{}}{l}}dV'} = - \dfrac{\mu_{0}}{4\pi}\dfrac{d}{dt}\int_{V'}^{}{\overset{\underline{}}{M}\left( {\overset{\underline{}}{r}}' \right) \cdot {\overset{\underline{}}{\nabla}}' \times \overset{\underline{}}{A}\left( {\overset{\underline{}}{r}}' \right)dV'}\]

Il campo \({\overset{\underline{}}{B}}_{ric}\left( {\overset{\underline{}}{r}}' \right) = {\overset{\underline{}}{\nabla}}' \times \overset{\underline{}}{A}\left( {\overset{\underline{}}{r}}' \right)\) è il campo magnetico che sarebbe prodotto dalla spira ricevente in \({\overset{\underline{}}{r}}'\) se in essa scorresse una corrente unitaria. Il campo \({\overset{\underline{}}{B}}_{ric}\) è detto ricevente e l'equazione per la fem. si scrive come:

\[emf = - \dfrac{\mu_{0}}{4\pi}\dfrac{d}{dt}\int_{V'}^{}{\overset{\underline{}}{M}\left( {\overset{\underline{}}{r}}' \right) \cdot {\overset{\underline{}}{B}}_{ric}\left( {\overset{\underline{}}{r}}' \right)dV'}\]

La relazione ricava lega la forza elettromotrice alla magnetizzazione tramite la derivata dell'integrale di volume del prodotto scalare della magnetizzazione stessa col campo che sarebbe prodotto dalla spira in \({\overset{\underline{}}{r}}'\) con una corrente unitaria.

Il costruttore e/o il progettista del macchinario realizzano l'antenna ricevente in modo che produca un campo noto in \({\overset{\underline{}}{r}}'\).

Misurata la fem. indotta sulla spira, l'unica incognita della relazione per la fem. è la magnetizzazione.

\[emf = - \dfrac{\mu_{0}}{4\pi}\dfrac{d}{dt}\int_{V'}^{}{\overset{\underline{}}{M}\left( {\overset{\underline{}}{r}}' \right) \cdot {\overset{\underline{}}{B}}_{ric}\left( {\overset{\underline{}}{r}}' \right)dV'}\]

Nell'ipotesi in cui il volume non vari nel tempo, è possibile scambiare il simbolo di derivata con quello di integrale:

\[emf = - \dfrac{\mu_{0}}{4\pi}\int_{V'}^{}{\dfrac{\partial\overset{\underline{}}{M}\left( {\overset{\underline{}}{r}}',t \right)}{\partial t} \cdot {\overset{\underline{}}{B}}_{ric}\left( {\overset{\underline{}}{r}}' \right)dV'}\]

Il campo generato della spira è fissato e non dipende dal tempo.

Per effettuare l'\emph{imaging} si applica un impulso a radiofrequenza che ribalta la magnetizzazione. Durante il ritorno all'equilibrio, il vettore di magnetizzazione varia nel tempo, inducendo una fem. sulla spira ricevente, proporzionale alla velocità di variazione della magnetizzazione.

Esplicitando il prodotto scalare tra magnetizzazione e campo ricevente si ottiene:

\[emf = - \dfrac{\mu_{0}}{4\pi}\int_{V'}^{}{\left( \dfrac{\partial M_{x}}{\partial t}B_{ric,x} + \dfrac{\partial M_{y}}{\partial t}B_{ric,y} + \dfrac{\partial M_{z}}{\partial t}B_{ric,z} \right)dV'}\]

Dato che le componenti trasverse \(M_{x}\) e \(M_{y}\) evolvono nel sistema fisso secondo un andamento oscillatorio smorzato, le derivate di queste componenti, a meno di costanti di fase, avranno dei termini del tipo:

\[\omega_{0}M_{0}\exp\left( j\omega_{0}t \right) + \dfrac{\partial M_{z}}{\partial t}B_{ric,z}\]

La componente longitudinale \(M_{z}\) evolve esponenzialmente con costante di tempo \(T_{1}\), quindi la derivata di tale componente rispetto al tempo sarà del tipo:

\[\dfrac{1}{T_{1}}\exp\left( - \dfrac{t}{T_{1}} \right)\]

Dato che \(\omega_{0}\) è dell'ordine di \(64 \cdot 10^{6}\ rad/s\) mentre il tempo di rilassamento \emph{spin-lattice} è di \(1\ s\), risulta che:

\[\omega_{0} \gg \dfrac{1}{T_{1}}\]

Nella fem. indotta, il contributo delle componenti trasversale è molto maggiore del contributo longitudinale, che può essere trascurato. In altre parole, la fem. è proporzionale approssimativamente solo alle derivate delle componenti trasverse:

\[emf \propto \dfrac{\partial M_{x}}{\partial t}B_{ric,x} + \dfrac{\partial M_{y}}{\partial t}B_{ric,y}\]

La componente longitudinale può essere trascurata.

Si è dimostrato che il segnale ricevuto dall'antenna, indotto dalle variazioni della magnetizzazione macroscopica di un volumetto elementare del paziente è proporzionale a:

\[emf \propto \int_{V'}^{}{\left( \dfrac{\partial M_{x}}{\partial t}B_{ric,x} + \dfrac{\partial M_{y}}{\partial t}B_{ric,y} + \dfrac{\partial M_{z}}{\partial t}B_{ric,z} \right)dV'}\]

Dove \({\overset{\underline{}}{B}}_{ric}\) è il campo magnetico che sarebbe generato dall'antenna ricevente nella posizione \({\overset{\underline{}}{r}}'\) del volumetto paziente, quando in essa scorre una corrente unitaria. Questo campo non dipendente dal tempo, poiché la corrente è costantemente uguale a \(1\ A\). Il volumetto \(V'\) deve avere dimensioni opportune per far sì che vi sia un numero sufficiente di spin e, allo stesso tempo, avere una buona risoluzione dell'immagine.

\begin{figure}
\centering
\includegraphics[width=3.21667in,height=2.45459in,alt={P3664\#yIS1}]{media/7_MRISignal/image84.pdf}\caption{Figura .: Rappresentazione del vettore di magnetizzazione e del campo ricevuto}
\end{figure}

L'evoluzione temporale nel piano trasversale al campo principale del vettore di magnetizzazione può essere studiata mediante il fasore \(M_{+}\), definito come:

\[M_{+}\left( \overset{\underline{}}{r},t \right) = M_{x}\left( \overset{\underline{}}{r},t \right) + jM_{y}\left( \overset{\underline{}}{r},t \right)\]

Ne risulta che \(M_{x}\left( \overset{\underline{}}{r},t \right) = Re\left\{ M_{+}\left( \overset{\underline{}}{r},t \right) \right\}\) e \(M_{x}\left( \overset{\underline{}}{r},t \right) = Im\left\{ M_{+}\left( \overset{\underline{}}{r},t \right) \right\}\). Inoltre, si è visto che il fasore \(M_{+}\) può essere scritto come:

\[M_{+}\left( \overset{\underline{}}{r},t \right) = M_{+}\left( \overset{\underline{}}{r},\ 0 \right)\exp\left( - j\omega_{0}t \right)\exp\left( - \dfrac{t}{T_{2}} \right) = \left| M_{+}\left( \overset{\underline{}}{r},\ 0 \right) \right|\exp\left\lbrack j\phi_{0}\left( \overset{\underline{}}{r} \right) \right\rbrack\exp\left( - j\omega_{0}t \right)\exp\left( - \dfrac{t}{T_{2}} \right)\]

Il tempo di rilassamento \(T_{2}\) dipende dal particolare volumetto considerato, poiché tra i diversi volumetti cambiano le proprietà chimico-fisiche dei tessuti. Analogo discorso vale per la fase iniziale, infatti, ogni vettore di magnetizzazione presenta una fase iniziale nel piano \(x - y\) diverse in base alla posizione del volumetto. Le quantità \(\left| M_{+}\left( \overset{\underline{}}{r},\ 0 \right) \right|\) e \(\phi_{0}\left( \overset{\underline{}}{r},\ 0 \right)\) dipendono dalle condizioni iniziali del volumetto.

Siccome \(\omega_{0} \gg T_{1}^{- 1}\) è possibile trascurare la componente longitudinale nella valutazione della fem. Si suppone di applicare una perturbazione a opera di un campo magnetico a radiofrequenza; il vettore di magnetizzazione viene ribaltato lungo il piano trasverso lungo uno degli assi. Si osservi che è possibile avere anche delle perturbazioni che portano la magnetizzazione a giacere sul piano trasverso e non su uno degli assi.

Interrotta la perturbazione, il vettore di magnetizzazione evolve secondo un andamento elicoidale con raggio variabile, descritto delle equazioni di Bloch. In questo contesto non è possibile trascurare i tempi di rilassamento, in quanto le evoluzioni delle componenti del vettore di magnetizzazione sono dettate dai tempi \(T_{1}\) e \(T_{2}\):

\[\dfrac{d\overset{\underline{}}{M}}{dt} = \gamma\overset{\underline{}}{M} \times {\overset{\underline{}}{B}}_{0} + \dfrac{1}{T_{1}}\left( M_{0} - M_{z}\  \right){\widehat{i}}_{z} - \dfrac{1}{T_{2}}{\overset{\underline{}}{M}}_{\bot}\]

Il movimento del vettore di magnetizzazione induce sull'antenna un segnale fem. dipendente prevalentemente dalle componenti trasverse del vettore \(\overset{\underline{}}{M}\), descritte dal fasore \(M_{+}\left( \overset{\underline{}}{r},t \right)\). Questa quantità tiene conto della dipendenza dallo spazio, dal tempo di rilassamento trasversale, dalla fase e dalla frequenza di Larmor alla quale risuonano gli spin nel volumetto elementare. La frequenza di Larmor dei diversi volumetti può essere diversa a causa delle disomogeneità di campo o dei gradienti di campo applicato.

Trascurando l'evoluzione longitudinale, la fem. indotta è proporzionale a:

\[emf \propto \int_{V'}^{}{\left( \dfrac{\partial M_{x}}{\partial t}B_{ric,x} + \dfrac{\partial M_{y}}{\partial t}B_{ric,y} + \dfrac{\partial M_{z}}{\partial t}B_{ric,z} \right)dV'} \simeq \int_{V'}^{}{\left( \dfrac{\partial M_{x}}{\partial t}B_{ric,x} + \dfrac{\partial M_{y}}{\partial t}B_{ric,y} \right)dV'}\]

Portando il simbolo di derivata all'esterno dell'integrale, nell'ipotesi che il volumetto elementare non vari nel tempo, è possibile scrivere:

\[emf \propto \dfrac{d}{dt}\int_{V'}^{}{\left( M_{x}B_{ric,x} + M_{y}B_{ric,y} \right)dV'}\]

La funzione integranda può essere espressa in espressa in termini del fasore:

\[M_{x}B_{ric,x} + M_{y}B_{ric,y} = {\overset{\underline{}}{M}}_{\bot} \cdot {\overset{\underline{}}{B}}_{ric} = \left( Re\left\{ M_{+}\left( \overset{\underline{}}{r},t \right) \right\}{\widehat{i}}_{x} + Im\left\{ M_{+}\left( \overset{\underline{}}{r},t \right) \right\}{\widehat{i}}_{y} \right) \cdot {\overset{\underline{}}{B}}_{ric}\left( \overset{\underline{}}{r} \right)\]

Dove:

\[Re\left\{ M_{+}\left( \overset{\underline{}}{r},t \right) \right\} = Re\left\{ \left| M_{+}\left( \overset{\underline{}}{r},\ 0 \right) \right|\exp\left\lbrack - j\left( \phi_{0}\left( \overset{\underline{}}{r} \right) - \omega_{0}\left( \overset{\underline{}}{r} \right)t \right) \right\rbrack\exp\left( - \dfrac{t}{T_{2}\left( \overset{\underline{}}{r} \right)} \right) \right\} = \left| M_{+}\left( \overset{\underline{}}{r},\ 0 \right) \right|\exp\left( - \dfrac{t}{T_{2}\left( \overset{\underline{}}{r} \right)} \right)\cos\left\lbrack \phi_{0}\left( \overset{\underline{}}{r} \right) - \omega_{0}\left( \overset{\underline{}}{r} \right)t \right\rbrack = \left| M_{+}\left( \overset{\underline{}}{r},\ 0 \right) \right|\exp\left( - \dfrac{t}{T_{2}\left( \overset{\underline{}}{r} \right)} \right)\cos\left\lbrack \omega_{0}\left( \overset{\underline{}}{r} \right)t - \phi_{0}\left( \overset{\underline{}}{r} \right) \right\rbrack\]

\[Im\left\{ M_{+}\left( \overset{\underline{}}{r},t \right) \right\} = Im\left\{ \left| M_{+}\left( \overset{\underline{}}{r},\ 0 \right) \right|\exp\left\lbrack - j\left( \phi_{0}\left( \overset{\underline{}}{r} \right) - \omega_{0}\left( \overset{\underline{}}{r} \right)t \right) \right\rbrack\exp\left( - \dfrac{t}{T_{2}\left( \overset{\underline{}}{r} \right)} \right) \right\} = \left| M_{+}\left( \overset{\underline{}}{r},\ 0 \right) \right|\exp\left( - \dfrac{t}{T_{2}\left( \overset{\underline{}}{r} \right)} \right)\sin\left\lbrack \phi_{0}\left( \overset{\underline{}}{r} \right) - \omega_{0}\left( \overset{\underline{}}{r} \right)t \right\rbrack = - \left| M_{+}\left( \overset{\underline{}}{r},\ 0 \right) \right|\exp\left( - \dfrac{t}{T_{2}\left( \overset{\underline{}}{r} \right)} \right)\sin\left\lbrack \omega_{0}\left( \overset{\underline{}}{r} \right)t - \phi_{0}\left( \overset{\underline{}}{r} \right) \right\rbrack\]

Per cui la funzione integranda è data da:

\[{\overset{\underline{}}{M}}_{\bot} \cdot {\overset{\underline{}}{B}}_{ric}\left( \overset{\underline{}}{r} \right) = \left( Re\left\{ M_{+}\left( \overset{\underline{}}{r},t \right) \right\}{\widehat{i}}_{x} + Im\left\{ M_{+}\left( \overset{\underline{}}{r},t \right) \right\}{\widehat{i}}_{y} \right) \cdot {\overset{\underline{}}{B}}_{ric}\left( \overset{\underline{}}{r} \right) = \left| M_{+}\left( \overset{\underline{}}{r},\ 0 \right) \right|\exp\left( - \dfrac{t}{T_{2}\left( \overset{\underline{}}{r} \right)} \right)\left\lbrack B_{ric,x}\left( \overset{\underline{}}{r} \right)\cos\left\lbrack \omega_{0}\left( \overset{\underline{}}{r} \right)t - \phi_{0}\left( \overset{\underline{}}{r} \right) \right\rbrack - B_{ric,y}\left( \overset{\underline{}}{r} \right)\sin\left\lbrack \omega_{0}\left( \overset{\underline{}}{r} \right)t - \phi_{0}\left( \overset{\underline{}}{r} \right) \right\rbrack \right\rbrack\]

La fem. indotta sull'antenna è data dalla derivata della quantità appena individuata:

\[emf \propto \dfrac{d}{dt}\int_{V'}^{}{\left| M_{+}\left( \overset{\underline{}}{r},\ 0 \right) \right|\exp\left( - \dfrac{t}{T_{2}\left( \overset{\underline{}}{r} \right)} \right)\left\lbrack B_{ric,x}\left( \overset{\underline{}}{r} \right)\cos\left\lbrack \omega_{0}\left( \overset{\underline{}}{r} \right)t - \phi_{0}\left( \overset{\underline{}}{r} \right) \right\rbrack - B_{ric,y}\left( \overset{\underline{}}{r} \right)\sin\left\lbrack \omega_{0}\left( \overset{\underline{}}{r} \right)t - \phi_{0}\left( \overset{\underline{}}{r} \right) \right\rbrack \right\rbrack dV'} =\]

In ipotesi di volumetto costante nel tempo, è possibile scrivere:

\[= \int_{V'}^{}{\dfrac{\partial}{\partial t}\left\{ \left| M_{+}\left( \overset{\underline{}}{r},\ 0 \right) \right|\exp\left( - \dfrac{t}{T_{2}\left( \overset{\underline{}}{r} \right)} \right)\left\lbrack B_{ric,x}\left( \overset{\underline{}}{r} \right)\cos\left\lbrack \omega_{0}\left( \overset{\underline{}}{r} \right)t - \phi_{0}\left( \overset{\underline{}}{r} \right) \right\rbrack - B_{ric,y}\left( \overset{\underline{}}{r} \right)\sin\left\lbrack \omega_{0}\left( \overset{\underline{}}{r} \right)t - \phi_{0}\left( \overset{\underline{}}{r} \right) \right\rbrack \right\rbrack \right\} dV'} =\]

Svolgendo la derivata temporale, si ha:

\[= \int_{V'}^{}{\left\{ - \dfrac{1}{T_{2}\left( {\overset{\underline{}}{r}}' \right)}\left| M_{+}\left( \overset{\underline{}}{r},\ 0 \right) \right|\exp\left( - \dfrac{t}{T_{2}\left( \overset{\underline{}}{r} \right)} \right)\left\lbrack B_{ric,x}\left( \overset{\underline{}}{r} \right)\cos\left\lbrack \omega_{0}\left( \overset{\underline{}}{r} \right)t - \phi_{0}\left( \overset{\underline{}}{r} \right) \right\rbrack - B_{ric,y}\left( \overset{\underline{}}{r} \right)\sin\left\lbrack \omega_{0}\left( \overset{\underline{}}{r} \right)t - \phi_{0}\left( \overset{\underline{}}{r} \right) \right\rbrack \right\rbrack + \left| M_{+}\left( \overset{\underline{}}{r},\ 0 \right) \right|\exp\left( - \dfrac{t}{T_{2}\left( \overset{\underline{}}{r} \right)} \right)\left\lbrack - \omega_{0}\left( \overset{\underline{}}{r} \right)B_{ric,x}\left( \overset{\underline{}}{r} \right)\sin\left\lbrack \omega_{0}\left( \overset{\underline{}}{r} \right)t - \phi_{0}\left( \overset{\underline{}}{r} \right) \right\rbrack - \omega_{0}\left( \overset{\underline{}}{r} \right)B_{ric,y}\left( \overset{\underline{}}{r} \right)\cos\left\lbrack \omega_{0}\left( \overset{\underline{}}{r} \right)t - \phi_{0}\left( \overset{\underline{}}{r} \right) \right\rbrack \right\rbrack \right\} dV'} \simeq\]

Siccome \(\omega_{0} \gg T_{2}^{- 1}\), è possibile trascurare il termine dipendente solamente dal tempo di rilassamento trasversale:

\[\simeq \int_{V'}^{}{\left\{ \omega_{0}\left( \overset{\underline{}}{r} \right)\left| M_{+}\left( \overset{\underline{}}{r},\ 0 \right) \right|\exp\left( - \dfrac{t}{T_{2}\left( \overset{\underline{}}{r} \right)} \right)\left\{ - B_{ric,x}\left( \overset{\underline{}}{r} \right)\sin\left\lbrack \omega_{0}\left( \overset{\underline{}}{r} \right)t - \phi_{0}\left( \overset{\underline{}}{r} \right) \right\rbrack - B_{ric,y}\left( \overset{\underline{}}{r} \right)\cos\left\lbrack \omega_{0}\left( \overset{\underline{}}{r} \right)t - \phi_{0}\left( \overset{\underline{}}{r} \right) \right\rbrack \right\} \right\} dV'} =\]

All'interno dell'integrale sono presenti solo quantità dipendenti dalla posizione, come la fase iniziale, la frequenza di Larmon, tempo di rilassamento e magnetizzazione iniziale.

Se il campo ricevente è del tipo:

\[B_{ric,x}\left( \overset{\underline{}}{r} \right) = B_{\bot}\left( \overset{\underline{}}{r} \right)\cos\left\lbrack \vartheta\left( \overset{\underline{}}{r} \right) \right\rbrack,\ \ B_{ric,x}\left( \overset{\underline{}}{r} \right) = B_{\bot}\left( \overset{\underline{}}{r} \right)\sin\left\lbrack \vartheta\left( \overset{\underline{}}{r} \right) \right\rbrack\]

Il segnale ricevuto può essere scritto come:

\[= \int_{V'}^{}{\left\{ \omega_{0}\left( \overset{\underline{}}{r} \right)\left| M_{+}\left( \overset{\underline{}}{r},\ 0 \right) \right|\exp\left( - \dfrac{t}{T_{2}\left( \overset{\underline{}}{r} \right)} \right)\left\{ - B_{\bot}\left( \overset{\underline{}}{r} \right)\cos\left\lbrack \vartheta\left( \overset{\underline{}}{r} \right) \right\rbrack\sin\left\lbrack \omega_{0}\left( \overset{\underline{}}{r} \right)t - \phi_{0}\left( \overset{\underline{}}{r} \right) \right\rbrack - B_{\bot}\left( \overset{\underline{}}{r} \right)\sin\left\lbrack \vartheta\left( \overset{\underline{}}{r} \right) \right\rbrack\cos\left\lbrack \omega_{0}\left( \overset{\underline{}}{r} \right)t - \phi_{0}\left( \overset{\underline{}}{r} \right) \right\rbrack \right\} \right\} dV'} = \int_{V'}^{}{\left\{ \omega_{0}\left( \overset{\underline{}}{r} \right)\left| M_{+}\left( \overset{\underline{}}{r},\ 0 \right) \right|B_{\bot}\left( \overset{\underline{}}{r} \right)\exp\left( - \dfrac{t}{T_{2}\left( \overset{\underline{}}{r} \right)} \right)\left\{ - \cos\left\lbrack \vartheta\left( \overset{\underline{}}{r} \right) \right\rbrack\sin\left\lbrack \omega_{0}\left( \overset{\underline{}}{r} \right)t - \phi_{0}\left( \overset{\underline{}}{r} \right) \right\rbrack - \sin\left\lbrack \vartheta\left( \overset{\underline{}}{r} \right) \right\rbrack\cos\left\lbrack \omega_{0}\left( \overset{\underline{}}{r} \right)t - \phi_{0}\left( \overset{\underline{}}{r} \right) \right\rbrack \right\} \right\} dV'}\]

Per le formula di addizione del seno:

\[\cos{\vartheta\left( \overset{\underline{}}{r} \right)}\sin\left\lbrack \omega_{0}\left( \overset{\underline{}}{r} \right)t - \phi_{0}\left( \overset{\underline{}}{r} \right) \right\rbrack + \sin{\vartheta\left( \overset{\underline{}}{r} \right)}\cos\left\lbrack \omega_{0}\left( \overset{\underline{}}{r} \right)t - \phi_{0}\left( \overset{\underline{}}{r} \right) \right\rbrack = \sin\left\lbrack \omega_{0}\left( \overset{\underline{}}{r} \right)t - \phi_{0}\left( \overset{\underline{}}{r} \right) + \vartheta\left( \overset{\underline{}}{r} \right) \right\rbrack\]

Dunque, trascurando il segno meno, si ottiene:

\[emf \propto \int_{V'}^{}{\left\{ \omega_{0}\left( \overset{\underline{}}{r} \right)\left| M_{+}\left( \overset{\underline{}}{r},\ 0 \right) \right|B_{\bot}\left( \overset{\underline{}}{r} \right)\exp\left( - \dfrac{t}{T_{2}\left( \overset{\underline{}}{r} \right)} \right)\sin\left\lbrack \omega_{0}\left( \overset{\underline{}}{r} \right)t - \phi_{0}\left( \overset{\underline{}}{r} \right) + \vartheta\left( \overset{\underline{}}{r} \right) \right\rbrack \right\} dV'}\]

Le variazione della posizione della frequenza di precessione di Larmor, \(\omega_{0}\left( \overset{\underline{}}{r} \right)\), può essere omessa poiché le sue variazioni sono molto ridotte. È, infatti, possibile esprimere la frequenza di precessione di Larmor come:

\[\omega_{0}\left( \overset{\underline{}}{r} \right) = \omega_{0} + \mathrm{\Delta}\omega_{0}\left( \overset{\underline{}}{r} \right)\]

Le variazioni tipiche della frequenza \(\mathrm{\Delta}f = \mathrm{\Delta}\omega_{0}/2\pi\), sono molto piccole se confrontate con la quantità \(\gamma B_{0}/2\pi\). Infatti, tale quantità è dell'ordine di qualche \(kHz\), mentre la frequenza di risonanza è dell'ordine di \(60\ MHz\), per cui:

\[\omega_{0} \gg \mathrm{\Delta}\omega_{0}\left( \overset{\underline{}}{r} \right)\]

Per cui è possibile portare all'esterno dell'integrale \(\omega_{0}\):

\[emf \propto \omega_{0}\int_{V'}^{}{\left\{ \left| M_{+}\left( \overset{\underline{}}{r},\ 0 \right) \right|B_{\bot}\left( \overset{\underline{}}{r} \right)\exp\left( - \dfrac{t}{T_{2}\left( \overset{\underline{}}{r} \right)} \right)\sin\left\lbrack \omega_{0}\left( \overset{\underline{}}{r} \right)t - \phi_{0}\left( \overset{\underline{}}{r} \right) + \vartheta\left( \overset{\underline{}}{r} \right) \right\rbrack \right\} dV'}\]

All'esterno dell'integrale le variazioni della pulsazione di Larmor possono essere trascurate, mentre all'interno della funzione trigonometrica è importante, in quanto tiene conto che la funzione sinusoidale può variare significativamente anche per piccole variazioni di \(\omega_{0}\).

Il segnale misurato dipende essenzialmente dal tempo di rilassamento traversale, la fase iniziale, magnetizzazione iniziale e in parte anche dal campo generato dall'antenna che, comunque, è una quantità nota poiché opportunamente progettata.

La dipendenza spaziale può essere trascurata se la magnetizzazione proviene da un piccolo campione omogeneo, come, ad esempio, come un bicchiere d'acqua. In questo caso l'integrale è di semplice risoluzione poiché nessuna quantità, come \(T_{2}\) e \(\omega_{0}\), non dipendono dalla posizione spaziale del cubetto elementare \(\overset{\underline{}}{r}\). Sia \(V_{s}\) il volume del campione, il segnale ottenuto è proporzionale a:

\[emf \propto \omega_{0}\int_{V'}^{}{\left\{ \left| M_{+} \right|B_{\bot}\exp\left( - \dfrac{t}{T_{2}} \right)\sin\left\lbrack \omega_{0}t - \phi_{0} + \vartheta \right\rbrack \right\} dV'} = \omega_{0}\left| M_{+} \right|V_{s}B_{\bot}\exp\left( - \dfrac{t}{T_{2}} \right)\sin\left\lbrack \omega_{0}t - \phi_{0} + \vartheta \right\rbrack\]

Affinché la frequenza di precessione sia costante è necessario che il campo principale sia uniforme in tutto lo spazio del campione. Questa ipotesi è verificata se il campione di materiale omogeneo ha dimensioni ridotte.

Le costante di fase sono estremamente importi per il confronto di segnali provenienti da sorgenti di magnetizzazione diverse, in cui è necessario che si verificano degli abbattimenti tra le varie fase dei segnali. Le frequenze \(\omega_{0}\), \(M_{+}\) e \(B_{\bot}\) sono imposte dall'esterno dall'operatore.

Per un campione omogeneo è semplice ottenere informazioni sulla fase. In particolare, avendo un segnale sinusoidale a frequenza \(\omega_{0}\) nota, il processo più semplice da utilizzare è la demodulazione coerente, ovvero una demodulazione in cui si estraggono le informazioni di fase della sinusoide.

\subsection{Demodulazione del segnale registrato}\label{demodulazione-del-segnale-registrato}

Le rapide oscillazioni alla frequenza \(\omega_{0}\), contenuti nel segnale registrato dalle antenne nella risonanza magnetica sono rimosse mediante una circuiteria elettronica che si occupa della demodulazione. Questo processo equivale ad analizzare il segnale proveniente dal sistema di riferimento rotante a frequenza di Larmor.

Il processo di demodulazione corrisponde alla moltiplicazione del segnale estratto dalle antenne, \(s(t)\), per un'oscillazione a frequenza prossima a quella di Larmor, in fase col segnale da demodulare. In seguito, un filtro passa-basso (LPF) estrae le componenti di interesse del segnale in uscita dal moltiplicatore.

\begin{figure}
\centering
\includegraphics[width=6.67083in,height=2.07625in,alt={Immagine che contiene nero, oscurità}]{media/7_MRISignal/image85.pdf}\caption{Figura .: Demodulazione del segnale acquisito dalle antenne}
\end{figure}

Il segnale \(s(t)\) presente ha uno spettro centrato sulla frequenza \(\omega_{0}\). Si ha, quindi:

\[s(t) \propto V_{S}M_{\bot}B_{\bot}\sin\left( \omega_{0}t - \phi_{0} + \vartheta \right)\]

Nel caso in cui il segnale provenga da un piccolo campione omogeneo, lo spettro coincide con due impulsi centrati su \(\omega_{0}\).

\begin{figure}
\centering
\includegraphics[width=3.87452in,height=3.425in]{media/7_MRISignal/image86.pdf}\caption{Figura .: Spettro di materiale omogeneo}
\end{figure}

In questo caso, la frequenza a cui risuonano gli spin del campione, in generale, può risulta diversa dall'oscillazione che produce il sistema di demodulazione. La frequenza del segnale \(s(t)\) registrato può essere espressa come:

\[\omega = \omega_{0} + \delta\omega_{0}\]

Dal punto di vista dello spettro, gli impulsi non sono centrati a \(\omega_{0}\), ma piuttosto dove a:

\[\omega = \omega_{0} + \delta\omega_{0}\]

Dove \(\delta\omega_{0}\) è nota come frequenza di offset rispetto la frequenza di Larmor \(\omega_{0}\).

Trascurando la dipendenza da \(\exp\left( - t/T_{2} \right)\), il seganale registrato \(s(t)\) dipende solo dalla sinusoide:

\[s(t) \propto \sin\left( \omega_{0}t + \delta\omega_{0}t + \vartheta - \phi_{0} \right)\]

In uscita al moltiplicatore, si ottiene un segnale proporzionale a:

\[s(t)\sin\left( \omega_{0}t \right) \propto \sin\left( \omega_{0}t + \delta\omega_{0}t + \vartheta - \phi_{0} \right)\sin\left( \omega_{0}t \right)\]

Per le formule di prostaferesi, il segnale in uscita è proporzionale, a meno di un fattore \(1/2\), a:

\[s(t)\sin\left( \omega_{0}t \right) \propto \cos\left( 2\omega_{o}t + \delta\omega_{0}t + \vartheta - \phi_{0} \right) - \cos\left( \delta\omega_{0}t + \vartheta - \phi_{0} \right)\]

Ovvero il segnale in uscita dal moltiplicatore è dato dalla somma di un termine a frequenza \(2\omega_{o} + \delta\omega_{0}\) e un termine a frequenza \(\delta\omega_{0}\), differenza tra la frequenza di risonanza di Larmor e dell'oscillatore.

Generalmente, risulta:

\[2f_{o} + \delta f_{0} \gg \ \delta f_{0}\]

Infatti, \(f_{0} \simeq 64\ MHz\), mentre \(\delta f_{0}\) è dell'ordine di qualche \(kHz\).

È possibile eseguire un processo di filtraggio, attraverso il quale il termine a frequenza \(2f_{o} + \delta f_{0}\) è completamente rimosso, mentre il termine a bassa frequenza è lasciato inalterato.

\begin{figure}
\centering
\includegraphics[width=4.225in,height=1.9202in]{media/7_MRISignal/image87.pdf}\caption{Figura .: Filtraggio a valle del moltiplicatore}
\end{figure}

Nel momento in cui si considera il tempo di rilassamento \(T_{2}\), lo spettro non è più di tipo impulsivo ma è slargato e centrato sulle frequenze \(\delta f_{0}\) e \(2f_{0} + \delta f_{0}\). Mediante il processo di filtraggio, si conserva solamente il termine in bassa frequenza, centrato su \(\delta f_{0}\).

Nel dominio del tempo, il segnale registrato dalle antenne è un'oscillazione ad alta frequenza, con ampiezza modulata dal termine \(\exp\left( - t/T_{2} \right)\).

\begin{figure}
\centering
\includegraphics[width=6.69306in,height=3.69861in]{media/7_MRISignal/image88.pdf}\caption{Figura .: Segnale registrato dalle antenne}
\end{figure}

Dopo la demodulazione del segnale registrato, la frequenza della sinusoide modulata da \(\exp\left( - t/T_{2} \right)\) è data dalla bassa frequenza \(\delta f_{0}\), sovrapposta a un'eventuale differenza tra oscillatore locale e segnale registrato.

\begin{figure}
\centering
\includegraphics[width=6.69306in,height=3.69861in,alt={Immagine che contiene diagramma, linea, Diagramma Descrizione generata automaticamente}]{media/7_MRISignal/image89.pdf}\caption{Figura .: Segnale a valle della modulazione}
\end{figure}

Nella pratica, si realizza una demodulazione su due canali:

\begin{itemize}
\item
  Su un canale il segnale registrato dalle antenne \(s(t)\) viene moltiplicato per un'oscillazione sinusoidale a frequenza \(\omega_{0}\);
\item
  Sul secondo canale, lo stesso segnale è moltiplicato per un'oscillazione cosinusoidale, in quadratura col seno, alla stessa frequenza \(\omega_{0}\).
\end{itemize}

Questo tipo di modulazione è detta coerente.

\begin{figure}
\centering
\includegraphics[width=5.79167in,height=3.14174in]{media/7_MRISignal/image90.pdf}\caption{Figura .: Demodulazione a due canali}
\end{figure}

Il segnale in uscita dal primo canale è proporzionale al \(\cos\left( \delta\omega_{0} + \vartheta - \phi_{0} \right)\), ovvero alla parte reale della quantità \(M_{+} = \exp\left( j\omega_{0}t \right)\exp\left\lbrack j\left( \vartheta - \phi_{0} \right) \right\rbrack\). Questo canale è detto reale e, studiando l'inviluppo del segnale ottenuto a frequenza \(\delta\omega_{0}\), è possibile ottenere informazioni sulla quantità \(T_{2}\), ovvero la costante di tempo con cui decade l'ampiezza dell'oscillazione.

Il secondo canale è detto immaginario ed è ottenuto moltiplicando il segnale registrato dalle antenne per una cosinusoide a frequenza \(\omega_{0}\). Il segnale a valle del moltiplicatore è, quindi, proporzionale a:

\[s(t)\cos\left( \omega_{o}t \right) \propto \sin\left( \omega_{0}t + \delta\omega_{0}t + \vartheta - \phi_{0} \right)\cos\left( \omega_{0}t \right)\]

Per le formule di prostaferesi, a meno di un fattore moltiplicativo \(1/2\), si ha:

\[s(t)\cos\left( \omega_{o}t \right) \propto \sin\left( \delta\omega_{0}t + \vartheta - \phi_{0} \right) - \sin\left( 2\omega_{0}t + \delta\omega_{0}t + \vartheta - \phi_{0} \right)\]

Il segnale in uscita dal canale immaginario è la parte immaginaria della quantità \(M_{+} = \exp\left( j\omega_{0}t \right)\exp\left\lbrack j\left( \vartheta - \phi_{0} \right) \right\rbrack\).

Mediante la demodulazione a due canali si riesce a ricostruire l'evoluzione sia della parte immaginaria che della parte immaginaria del segnale \(M_{+}\), ovvero le componenti longitudinali e trasversali del vettore di magnetizzazione \(\overset{\underline{}}{M}\).

Siccome il segnale in ingresso ai due canali è lo stesso e poiché i due canali sono posti in parallelo, il rumore sovrapposto al segnale parte reale e parte immaginaria è correlato. In generale, si sfruttano altri schemi che sfruttano delle antenne poste in quadratura, ovvero orientate in maniera ortogonale tra loro, rispetto al campione di cui si vuole eseguire l'imaging.

\begin{figure}
\centering
\includegraphics[width=3.87533in,height=4.13095in]{media/7_MRISignal/image91.pdf}\caption{Figura .: Antenne in quadratura}
\end{figure}

Il segnale registrato da una delle due antenne è in quadratura col segnale registrato dalla seconda, quindi, è possibile ricavare la parte reale e immaginaria del fasore \(M_{+}\) ma con rumore incorrelato. Con questa soluzione è possibile aumentare il rapporto segnale/rumore.

\subsection{Acquisizione con sequenza Free Induction Dacay}\label{acquisizione-con-sequenza-free-induction-dacay}

Il segnale ricevuto dall'antenna, usata come detettore, \(s(t)\) non dipende semplicemente dai parametri del tessuto ma anche da come sono applicati i campo magnetici al corpo in esame, secondo una precisa sequenza di applicazione.

La sequenza più semplice da applicare consiste nell'esperimento \emph{Free Induction Dacay} (decadimento libero dell'induzione) o FID, in cui si irradia il campione con un singolo impulso a radiofrequenza.

Si suppone di avere un campione di un materiale omogeneo, contenente un numero di Avogadro di spin. Immediatamente dopo l'impulso a radiofrequenza \({\overset{\underline{}}{B}}_{1}\), la magnetizzazione, secondo l'equazione di Bloch, si porta dall'asse \({\widehat{i}}_{x'}\) o \({\widehat{i}}_{y'}\) al valore di equilibrio mediante una traiettoria ellittica.

Il segnale registrato dalle antenne non è altro che il decadimento delle componenti trasversali del vettore di magnetizzazione durante la fase di ritorno all'equilibrio. Il segnale registrato dalle antenne, quindi, decade con costante di tempo \(T_{2}\).

\begin{figure}
\centering
\includegraphics[width=4.83512in,height=2.48333in]{media/7_MRISignal/image92.pdf}\caption{Figura .: Sequenza FID nel sistema fisso del laboratorio}
\end{figure}

Nel sistema di riferimento rotante, il campo \({\overset{\underline{}}{B}}_{1}\) appare come un impulso costante sull'asse \({\widehat{i}}_{x'}\) o \({\widehat{i}}_{y'}\). Subito dopo l'esaurimento dell'impulso a radiofrequenze la magnetizzazione trasversale si riduce, in modulo, con costante di tempo \(T_{2}\).

\begin{figure}
\centering
\includegraphics[width=4.125in,height=2.07423in]{media/7_MRISignal/image93.pdf}\caption{Figura .: Sequenza FID nel sistema rotante}
\end{figure}

Il segnale visto nel sistema rotante equivale al segnale acquisito nel sistema fisso dopo la demodulazione. Se la velocità di rotazione del sistema rotante, \(\omega\), è diversa dalla frequenza di Larmor \(\omega_{0}\), nel sistema rotante si osserva un'oscillazione a frequenza \(\delta\omega = \omega_{0} - \omega\).

\begin{figure}
\centering
\includegraphics[width=4.02924in,height=2.45833in]{media/7_MRISignal/image94.pdf}\caption{Figura .: Sequenza FID nel sistema rotante con \(\delta\omega = \omega_{0} - \omega \neq 0\)}
\end{figure}

Si vuole studiare il comportamento della fase del segnale registrato. Il segnale di tensione indotto sull'antenna ricevente, in ipotesi di campione omogeneo, è proporzionale a:

\[s(t) \propto \omega_{0}\exp\left( - \dfrac{t}{T_{2}} \right)\exp\left\{ - j\left\lbrack \left( \omega_{0} - \omega \right)t + \phi_{0} - \vartheta_{B} \right\rbrack \right\}\int_{V}^{}{{\overset{\underline{}}{B}}_{\bot}\left( \overset{\underline{}}{r} \right) \cdot {\overset{\underline{}}{M}}_{\bot}\left( \overset{\underline{}}{r} \right)dV}\]

Per un campione omogeneo, è possibile trascurare il prodotto scalare tra le componenti traverse del vettore di magnetizzazione e del campo prodotto dall'antenna ricevente nel punto \(\overset{\underline{}}{r} \in V\) se fosse percorsa da una corrente unitaria, in quanto si vuole studiare l'evoluzione della fase del segnale registrato. In altre parole, solamente l'argomento dell'esponenziale complesso è di interesse. Con un abuso di notazione si pone:

\[\phi_{0} = \phi_{0} - \vartheta_{B}\]

Dove \(\vartheta_{B}\) è la \emph{field angle}, ovvero l'angolo oltre il quale l'intensità della sorgente si riduce del \(10\%\).

Con questa posizione la fase del segnale registrato è:

\[\phi(t) = \left( \omega_{0} - \omega \right)t + \phi_{0}\]

Una volta che il segnale è stato demodulato, le componenti frequenziali a frequenza maggiore di \(\delta\omega_{0}\) sono eliminate dal filtraggio passabasso. A valle della demodulazione, la fase è:

\[\phi(t) = \delta\omega_{0}t + \phi_{0}\]

Si osservi che l'attenuazione del segnale avviene con constante di tempo \(T_{2}\), la quale varia anche in base alle disomogeneità di campo magnetico. Il tempo \(T_{2}\) è legato allo sfasamento degli spin nel campo trasverso al campo principale. Questo sfasamento dipende dalle iterazioni spin-spin le quali alterano localmente il campo magnetico percepito da uno spin.

Il campo magnetico principale non è omogeneo in tutto lo spazio occupato dal paziente, ma presenta una certa variabilità dell'ordine di grandezza di \(1\ ppa\). Il campo magnetico avvertito da uno spin può, quindi, essere espresso come somma del campo magnetico principale e di un certo termine \(\mathrm{\Delta}B\) legato alle disomogeneità:

\[B\left( \overset{\underline{}}{r} \right) = B_{0} + \mathrm{\Delta}B\left( \overset{\underline{}}{r} \right)\]

Le differenze locali dei campi magnetici sono percepite dai vari spin e si traducono in differenti frequenze di precessione. Dati due spin, che precedono rispettivamente alle frequenze \(\omega_{1}\) e \(\omega_{2}\), le frequenze con cui gli spin precedono possono essere espresse come:

\[\omega_{1} = \omega_{0} + \mathrm{\Delta}\omega_{1},\ \ \omega_{2} = \omega_{0} + \mathrm{\Delta}\omega_{2}\]

Dunque, oltre alle iterazioni spin-spin, si ha un ulteriore effetto legato ai limiti costruttivi dei meccanismi di generazione del campo. All'interno del volumetto elementare di circa \(1\ mm\) di lato e contente un numero di Avogadro di protoni, gli effetti elle impurità del campo si traducono in una riduzione significativa del tempo di rilassamento traversale, poiché le fasi di distribuiscono ancor più casualmente. Mentre il tempo di rilassamento traversale \(T_{2}\) è legato alle interazioni spin-spin, il tempo di rilassamento \(T_{2}'\) è dovuto alle disomogeneità del campo principale.

Il tempo complessivo con cui la componente trasversale \({\overset{\underline{}}{M}}_{\bot}\) si porta a zero è indicato con \(T_{2}^{*}\) ed è legato ai due tempi \(T_{2}\) e \(T_{2}'\) dalla relazione:

\[\dfrac{1}{T_{2}^{*}} = \dfrac{1}{T_{2}} + \dfrac{1}{T_{2}'}\]

La fase del segnale registrato dalle antenne in ricezione è:

\[\phi(t) = \left( \omega_{0} - \omega \right)t + \phi_{0}\]

Dove \(\omega = \gamma B_{0} + \gamma\mathrm{\Delta}B\). In definitiva, la fase del segnale è proporzionale a:

\[\phi(t) \propto \gamma\mathrm{\Delta}Bt\]

\(\phi(t)\) tende rapidamente a zero, quando si considera un volumetto elementare del paziente. Ciò determina un decadimento molto più rapido del segnale registrato dalle antenne.

\begin{figure}
\centering
\includegraphics[width=4.85833in,height=2.57017in]{media/7_MRISignal/image95.pdf}\caption{Figura .: Andamento dello sfasamento al variare del tempo di rilassamento}
\end{figure}

L'esperimento FID permette di effettuare lo \emph{shimming} del capo magnetico, ovvero la compensazione delle disomogeneità del campo principale mediante una serie di operazioni meccaniche ed elettriche, come la regolazione della corrente di alimentazione di apposite bobine dette, appunto, di \emph{shimming}. L'esperimento FID è effettuato a monte di altri al fine di omogeneizzare il campo principale.

Un campo principale non omogeneo porta a una serie di problematiche nella valutazione dei tempi di rilassamento, ciò determina una forte incertezza durante l'imaging sulla localizzazione dei volumetti. Il posizionamento dei volumetti è legato ai gradienti di campo.

\subsubsection[Origine del tempo T2*]{Origine del tempo $\mathbf{T}_{\mathbf{2}}^{\mathbf{*}}$}
\label{origine-del-tempo-mathbft_mathbf2mathbf}

Si vuole determinare l'origine della relazione tra il tempo di rilassamento traversale \(T_{2}\) e quello \(T_{2}^{*}\) legato alle disomogeneità di campo.

Il segnale captato dalle antenne è del tipo:

\[s(t) \propto \omega_{0}\int_{V}^{}{\exp\left\lbrack - \dfrac{t}{T_{2}\left( \overset{\underline{}}{r} \right)} \right\rbrack{\overset{\underline{}}{B}}_{\bot}\left( \overset{\underline{}}{r} \right) \cdot {\overset{\underline{}}{M}}_{\bot}\left( \overset{\underline{}}{r} \right)\sin\left\lbrack \omega_{0}\left( \overset{\underline{}}{r} \right) \right\rbrack dV}\]

In linea di principio, per un materiale generico, il tempo di rilassamento traversale, così come la frequenza di precessione di Larmor, dipende dalla posizione \(\overset{\underline{}}{r}\) del volumetto elementare nel corpo, anche a causa delle disomogeneità del campo \(\mathrm{\Delta}\overset{\underline{}}{B}\).

Il processo di demodulazione coerente avviene a frequenza \(\omega\) generata dalla strumentazione di demodulazione e uguale alla velocità angolare del sistema rotante.

\begin{figure}
\centering
\includegraphics[width=4.14306in,height=3.19048in]{media/7_MRISignal/image96.pdf}\caption{Figura .: Demodulazione coerente a frequenza \(\omega\)}
\end{figure}

Il segnale demodulato può essere scritto in forma complessa:

\[s(t) = Re\left\{ s(t) \right\} + jIm\left\{ s(t) \right\}\]

Di pone \(s_{R}(t) = Re\left\{ s(t) \right\}\) e \(s_{I}(t) = Im\left\{ s(t) \right\}\), per cui:

\[s(t) = s_{R}(t) + js_{I}(t)\]

Si è visto che, a valle del processo di demodulazione, il segnale registrato è dato da:

\[s(t) \propto \omega_{0}\int_{V}^{}{\exp\left\lbrack - \dfrac{t}{T_{2}\left( \overset{\underline{}}{r} \right)} \right\rbrack{\overset{\underline{}}{B}}_{\bot}\left( \overset{\underline{}}{r} \right) \cdot {\overset{\underline{}}{M}}_{\bot}\left( \overset{\underline{}}{r},0 \right)\exp\left\{ - j\left\lbrack \left( \omega - \omega_{0} \right)t + \phi_{o}\left( \overset{\underline{}}{r} \right) \right\rbrack \right\} dV}\]

Il segnale \(s(t)\) è proporzionale a un'oscillazione complessa \(\exp(j\mathrm{\Delta}\omega t)\). Si suppone che la fase iniziale dell'oscillazione sia nulla, ovvero \(\phi_{o}\left( \overset{\underline{}}{r} \right) = 0\). Si considera, inoltre, un campione di materiale omogeneo come un bicchiere d'acqua. In questo caso, il tempo di rilassamento trasversale \(T_{2}\) non dipende dalla posizione. Si ritiene, infine, che l'antenna sia ideale, ovvero che il campo irradiato non dipende dalla posizione \(\overset{\underline{}}{r}\). Il segnale registrato, sotto queste ipotesi, è proporzionala a:

\[s(t) \propto \omega_{0}\exp\left( - \dfrac{t}{T_{2}} \right)B_{\bot}\int_{V}^{}{M_{\bot}\left( \overset{\underline{}}{r},0 \right)\exp(j\mathrm{\Delta}\omega t)dV}\]

Le disomogeneità di campo principale \(\mathrm{\Delta}B\) introducono delle variazioni della frequenza di precessione dei vari isocromi, dell'ordine della decina di \(kHz\). La variazione della differenze tra le frequenze di oscillazione del sistema e di precessione degli spin può essere descritta mediante una distribuzione gaussiana o lorentziana.

La distribuzione guassiana che meglio descrive le variazioni di frequenza \(\mathrm{\Delta}\omega\) presenta una media nulla e una varianza dipendente dalla disomogeneità del campo magnetico. Gli spin che presentano una differenza di frequenza positiva, ovvero \(\mathrm{\Delta}\omega > 0 \Leftrightarrow \omega_{0} - \omega > 0\), presentano una velocità maggiore rispetto al sistema di riferimento; mentre quelli per cui la differenza di frequenza è negativa, ovvero \(\mathrm{\Delta}\omega < 0 \Leftrightarrow \omega_{0} - \omega < 0\), presentano una velocità inferiore dell'oscillatore.

\begin{figure}
\centering
\includegraphics[width=3.22727in,height=2.60327in]{media/7_MRISignal/image97.pdf}\caption{Figura .: Distribuzione gaussiana}
\end{figure}

Un'altra possibile distribuzione che può essere attribuita alla variazione della frequenza di precessione è la lorentziana, descritta dalla \(pdf\):

\[p(\mathrm{\Delta}\omega) = \dfrac{2T_{2}'}{1 + (2\pi\mathrm{\Delta}f)^{2}{T_{2}'}^{2}}\]

Dove \(T_{2}'\) è un parametro della lorentziana che, in questo caso, descrive le disomogeneità di campo magnetico, sorgenti della distribuzione stessa.

\begin{figure}
\centering
\includegraphics[width=4.12847in,height=2.5514in]{media/7_MRISignal/image98.pdf}\caption{Figura .: Distribuzione lorentziana}
\end{figure}

Prove sperimentali hanno dimostrato che la reale distribuzione delle frequenze di processione presenta valori intermedi tra le due distribuzioni citate. Per semplicità, si ritiene che la distribuzione delle frequenze di processione sia di tipo lorentziana.

Il segnale prelevato, a seguito della demodulazione, ricorrendo alla \(pdf\) lorentziana, può essere espresso come:

\[s(t) \propto \omega_{0}\exp\left( - \dfrac{t}{T_{2}} \right)B_{\bot}\int_{- \infty}^{+ \infty}{p(\mathrm{\Delta}\omega)\exp(j\mathrm{\Delta}\omega t)d\omega}\]

La percentuale degli isocromati è espressa dalla \(pdf\) lorentziana, ovvero:

\[s(t) \propto \omega_{0}\exp\left( - \dfrac{t}{T_{2}} \right)B_{\bot}\int_{- \infty}^{+ \infty}{\dfrac{2T_{2}'}{1 + (2\pi\mathrm{\Delta}f)^{2}{T_{2}'}^{2}}\exp(j2\pi\mathrm{\Delta}ft)df}\]

La percentuale degli isocromati è massima per \(\mathrm{\Delta}f = 0\), ed è uguale a \(2T_{2}'\); mentre tende a zero per \(\mathrm{\Delta}f \rightarrow \pm \infty\).

L'integrale rappresenta l'antitrasformata di Fourier della \(pdf\) della lorentziana. È noto che:

\[\int_{- \infty}^{+ \infty}{\dfrac{2T_{2}'}{1 + (2\pi\mathrm{\Delta}f)^{2}{T_{2}'}^{2}}\exp(j2\pi\mathrm{\Delta}ft)df} = \exp\left( - \dfrac{|t|}{T_{2}'} \right)\]

Siccome si considerano tempi positivi, \(t > 0\), il segnale registrato è proporzionale a:

\[s(t) \propto \omega_{0}\exp\left( - \dfrac{t}{T_{2}} \right)\exp\left( - \dfrac{t}{T_{2}'} \right) = \omega_{0}\exp\left\lbrack - t\left( \dfrac{1}{T_{2}} + \dfrac{1}{T_{2}'} \right) \right\rbrack\]

Si definisce:

\[\dfrac{1}{T_{2}^{*}} = \dfrac{1}{T_{2}} + \dfrac{1}{T_{2}'}\]

Il tempo \(T_{2}^{*}\) dipende dalla distribuzione della frequenza di precessione dei singoli isocromi, dovuta alle disomogeneità di campo quest'ultimo parametro quantificato da \(T_{2}'\) nella distribuzione di lorentziana, e dal tempo \(T_{2}\), legato alle caratteristiche del materiale in esame.

L'uso della distribuzione di Lorentz rende l'analisi del tempo di rilassamento \(T_{2}^{*}\) molto semplice dal punto di vista analitico, ma non descrive in modo molto accurato i fenomeni reali. Al contrario, la distribuzione gaussiana rende l'analisi più complessa, poiché non esiste una primitiva della \(pdf\). Nei modelli simulati si preferisce utilizzare la distribuzione gaussiana in quanto di più semplice implementazione.

\subsection{Sequenza spin-echo}\label{sequenza-spin-echo}

Anche in presenza di opportune compensazioni, il campo magnetico principale presente una disomogeneità dell'ordine di \(1\ ppm\), ovvero:

\[\mathrm{\Delta}B \simeq 10^{- 6}B_{0}\]

Se, \(B_{0} = 1.5\ T\), la disomogeneità di campo è dell'ordine di \(1.5\ \mu T\). Sebbene tale variazione sia molto bassa, la disomogeneità di campo non può essere trascurata, poiché confrontabile con le variazioni locali del campo, a opera degli spin.

Il tempo di rilassamento \(T_{2}^{*}\) è strettamente legato alle disomogeneità di campo, infatti, per un \(\mathrm{\Delta}B = 1\ ppm\), il tempo \(T_{2}\) di un tessuto passa da \(100\ ms\) a \(5\ ms\) o minore. Inoltre, le disomogeneità di campo possono ridurre fortemente il segnale ottenuto. Per evitare ciò sono state proposte delle sequenze di impulsi per attenuare gli effetti della disomogeneità di campo. La prima proposta è la sequenza spin-echo, composta da due impulsi: il primo su un asse trasversale come \({\widehat{i}}_{x'}\) che fa ruotare la magnetizzazione sul piano trasverso, allineandolo all'asse lungo cui è diretto l'impulso. Se, ad esempio, l'asse di rotazione è \(x'\), l'impulso è indicato con \((\pi/2)_{x'}\), in quanto ruota il vettore di magnetizzazione di un angolo pari a \(\pi/2\), così da precedere intorno all'asse \(x'\).

Il secondo impulso può essere diretto sia lungo \({\widehat{i}}_{x'}\) sia lungo \({\widehat{i}}_{y'}\) ed è di tipo \(\pi\), ovvero ruota la magnetizzazione nel piano trasverso di un angolo pari a \(\pi\)

\begin{figure}
\centering
\includegraphics[width=5.43021in,height=1.3332in,alt={Immagine che contiene linea, diagramma, bianco Descrizione generata automaticamente}]{media/7_MRISignal/image99.pdf}\caption{Figura .: Sequenza spin-echo}
\end{figure}

Al termine del primo impulso a \(\pi/2\) lungo l'asse \(x'\), si registra il segnale \(s(t)\), dato da un'oscillazione a elevata frequenza, nello spettro delle radio-onde, con un'ampiezza che decade con costante di tempo \(T_{2}^{*}\).

Nel sistema di riferimento rotante, i vari spin del volumetto elementare si sfasano tra loro molto rapidamente; per cui, in poco tempo, si orientano in modo casuale, ottenendo così una risultate nulla nel piano trasverso. Il tempo necessario per l'azzeramento della componente trasversa del vettore di magnetizzazione è di circa \(5\ ms\).

Si applica un secondo impulso di \(\pi\) sull'asse \(x'\). Si considera uno spin generico. Esso si trova nel piano trasverso al campo principale \(B_{0}\) e precede interno all'asse \(x'\). Applicato l'impulso a \(\pi\), lo spin percorre metà circonferenza intorno all'asse \(x'\), trovandosi così nel lato opposto, rispetto all'asse \(x'\), alla posizione che occupava prima dell'applicazione dell'impulso a \(\pi\). Generalizzando, dopo l'applicazione dell'impulso a \(\pi\), ogni spin si trova in una posizione speculare alla precedente, rispetto all'asse su cui è applicato l'impulso.

Si osservi che ogni spin possiede una propria frequenza di precessione a causa delle disomogeneità di campo, esprimibile come \(\omega = \omega_{0} + \mathrm{\Delta}\omega\). Gli spin che presentano una frequenza di precessione maggiore ovvero \(\mathrm{\Delta}\omega < 0\), si trovano a destra dell'asse \(x'\), ovvero per \(y' > 0\). Questi spin si allontanano dall'asse \(x'\) poiché la loro fase è negativa nel sistema di riferimento rotante e il loro moto avviene in senso antiorario. Viceversa, gli spin con una frequenza di precessione maggiore della velocità di rotazione del sistema di riferimento rotante, si trovano alla sinistra dell'asse \(x'\), ovvero nella regione di spazio \(y' < 0\). Questi spin, avendo \(\mathrm{\Delta}\omega > 0\), si allontanano dall'asse \(x'\) in senso orario. Infine, gli spin che presentano una \(\mathrm{\Delta}\omega = 0\) precedono alla frequenza \(\omega_{0}\), quindi, sono fermi nel sistema di riferimento rotante.

Dopo l'applicazione dell'impulso a \(\pi\) intorno all'asse \(x'\), gli spin sono ruotati in modo a occupare una posizione speculare rispetto all'asse lungo cui è applicato l'impulso. Di conseguenza, gli spin che precedono con frequenza minore (\(\mathrm{\Delta}\omega > 0\)) si trovano nella regione di spazio che era occupata dagli spin con frequenza di precessione maggiore del sistema di riferimento rotante (\(y' < 0\)) e viceversa.

La frequenza con cui gli spin precedono non cambia con l'applicazione degli impulsi a radiofrequenza, poiché \(\mathrm{\Delta}\omega\) dipende solo dalla disomogeneità del campo principale, non modificata dagli impulsi.

Gli spin lenti, cioè con \(\mathrm{\Delta}\omega < 0\), tendono ad avvicinarsi all'asse \(x'\) in senso antiorario, mentre gli spin veloci, con \(\mathrm{\Delta}\omega > 0\), si muovono verso l'asse \(x'\) in senso orario. Si osservi, tuttavia, che nel sistema fisso del laboratorio tutti gli spin si muovono in senso orario.

Dopo l'applicazione dell'impulso a \(\pi\), gli spin tendono a rioccupare le posizioni che occupavano prima della radioonda. In questo caso si parla di rifocalizzazione, in quanto, dopo un certo tempo, tutti gli spin sanno allineati nuovamente lungo l'asse \(x'\).

\begin{figure}
\centering
\includegraphics[width=4.62912in,height=4.42403in,alt={Immagine che contiene diagramma, disegno, schizzo, Disegno tecnico Descrizione generata automaticamente}]{media/7_MRISignal/image100.pdf}\caption{Figura .: Movimento degli spin nella sequenza spin-echo}
\end{figure}

Nel sistema di riferimento fisso, le varie componenti trasversali degli spin si sommano, producendo un progressivo aumento della componente trasversale, poiché i vari spin si stanno rifocalizzando, portando la somma delle loro componenti trasversali a un valore abbastanza significativo. Dunque, la magnetizzazione trasversale diventa sempre più intensa. Dopo che la magnetizzazione raggiunge il massimo, si ha lo stesso comportamento dopo l'applicazione dell'impulso a \(\pi/2\). Infatti, spin lenti e veloci nel sistema di riferimento rotante tendono a raggiungere il comportamento all'equilibrio termodinamico, mediante un defasamento degli spin, il quale tende ad annullare la componente trasversale della magnetizzazione.

\begin{figure}
\centering
\includegraphics[width=3.00295in,height=0.95921in,alt={Immagine che contiene schizzo, disegno, linea, arte Descrizione generata automaticamente}]{media/7_MRISignal/image101.pdf}\caption{Figura .: Segnale registrato dopo l'applicazione dell'impulso a \(\pi\)}
\end{figure}

Questo processo avviene poiché gli spin conservano la loro frequenza di precessione, dunque, ci sarà un istante in cui gli spin attraversano l'asse \(x'\); dopodiché la magnetizzazione si rilasserà con una costante di tempo \(T_{2}^{*}\). Si noti, inoltre, che anche la rifocalizzazione avviene con costante di tempo \(T_{2}^{*}\), poiché dipendente dalle disomogeneità di campo.

La sequenza spin-echo, composta dai due impulsi, sfrutta la rifocalizzazione della magnetizzazione per stimare il tempo \(T_{2}\).

Analiticamente, è possibile determinare l'instante di tempo in cui la magnetizzazione si rifocalizzata sull'asse \(x'\). La fase dei vari spin è legata alla disomogeneità di campo principale, nel sistema di riferimento rotante, dalla relazione:

\[\phi(t) = - \gamma\mathrm{\Delta}Bt\]

Nel tempo, la fase decresce finché non si applica il secondo impulso. In altre parole, dopo l'applicazione del primo impulso a \(\pi\), per un certo spin, la fase decresce con legge lineare nel tempo.

Sia \(t = 0\ s\) il tempo di fine dell'impulso a \(\pi/2\); dopo un tempo \(\tau\) si applica l'impulso a \(\pi\). La fase dello spin dopo quest'ultimo sarà opposta a quella che lo spin possedeva all'istante di tempo appena precedente a \(\tau\). Ciò è dovuto al fatto che gli spin sono ribaltati rispetto all'asse su cui è applicativo l'impulso a \(\pi\).

Analiticamente, sia \(\phi\left( \tau^{-} \right)\) la fase un istante prima dell'applicazione dell'impulso a \(\pi\):

\[\phi\left( \tau^{-} \right) = - \gamma\mathrm{\Delta}B\tau\]

La fase subito dopo l'interruzione del secondo impulso nel sistema rotante è:

\[.\phi\left( \tau^{+} \right) = - \ \phi\left( \tau^{-} \right) = \ \gamma\mathrm{\Delta}B\tau\]

Esaurito l'impulso a \(\pi\), l'andamento della fase continua a decrescere linearmente, con una fase iniziale \(\phi_{0} = \ \gamma\mathrm{\Delta}B\tau\):

\[\phi(t) = - \gamma\mathrm{\Delta}B(t - \tau) + \ \gamma\mathrm{\Delta}B\tau = \gamma\mathrm{\Delta}B(2\tau - t)\]

Ciò accade poiché l'applicazione degli impulsi non provoca una variazione nella disomogeneità del campo principale, dunque, la frequenza di precessione degli spin non varia a causa degli impulsi a radiofrequenze.

Nel diagramma delle fasi si hanno due tratti paralleli: il primo parte da una condizione di equilibrio, quindi, evolve dallo zero fino a raggiungere la fase \(\phi\left( \tau^{-} \right) = - \gamma\mathrm{\Delta}B\tau\) a causa dell'applicazione del primo impulso. Il secondo tratto ha la stessa pendenza, poiché \(\mathrm{\Delta}B\) è costante, ed ha come condizione iniziale \(\phi_{0} = \ \gamma\mathrm{\Delta}B\tau\).

\begin{figure}
\centering
\includegraphics[width=5.13386in,height=3.25197in]{media/7_MRISignal/image102.pdf}\caption{Figura .: Diagrammi della fase dopo l'applicazione dell'impulso a \(\pi/2\)}
\end{figure}

Ogni spin vede una disomogeneità di campo principale \(\mathrm{\Delta}B\) diversa, dunque, la rispettiva fase evolve con una pendenza diversa. In ogni caso, tutte le fasi degli spin convergono alla fase nulla nello stesso istante, per il parallelismo tra l'andamento dell'andamento della fase prima e dopo l'applicazione dell'impulso a \(\pi\).

\begin{figure}
\centering
\includegraphics[width=5.12992in,height=3.25197in]{media/7_MRISignal/image103.pdf}\caption{Figura .: Per tutte gli spin le fasi si annullano nello stesso istante}
\end{figure}

Il tempo in cui la fase di tutti gli spin è nulla è data dall'equazione:

\[\phi(t) = 0 \Leftrightarrow \mathrm{\Delta}B(2\tau - t) = 0 \Leftrightarrow t = 2\tau\]

Il tempo necessario affinché la fase si annulla, dopo l'applicazione dell'impulso a \(\pi\), è esattamente \(2\tau\), dove \(\tau\) è l'intervallo di tempo tra i due impulsi. Nell'istante \(t = 2\tau\), il segnale di magnetizzazione è il più alto possibile.

Applicando un impulso a \(\pi/2\) si ha un primo sfasamento. Dopo un tempo \(\tau\) si applica il secondo impulso a \(\pi\). La magnetizzazione presenta, quindi, una prima fase di rifasamento e una seconda di defasamento. Per entrambi gli impulsi, per ogni step, gli inviluppo presentano un andamento con costante di tempo \(T_{2}^{*}\).

\begin{figure}
\centering
\includegraphics[width=4.81835in,height=3.11733in,alt={Immagine che contiene schizzo, diagramma, disegno, linea Descrizione generata automaticamente}]{media/7_MRISignal/image104.pdf}\caption{Figura .: Sequenza spin-echo e segnale registrato}
\end{figure}

% lista delle immagini (senza virgole!)
\def\imagelist{
image105, image106, image107, image108, image109, image110, image111, image112,
image113, image114, image115, image116, image117, image118, image119, image120,
image121, image122, image123, image124, image125}

% --- corpo principale ---
\setcounter{imgcount}{0}

\begin{center}
\foreach \image in \imagelist {%
    \includegraphics[width=\imgwidth]{media/7_MRISignal/\image.pdf}%
    \stepcounter{imgcount}%
    \ifnum\value{imgcount}<\imagesperrow
        \hspace{0.02\textwidth}% piccolo spazio tra immagini
    \else
        \par\vspace{0cm}% fine riga
        \setcounter{imgcount}{0}% reset del contatore
    \fi
}
\captionof{figure}{Andamento degli spin nella sequenza spin-echo}
\end{center}

Il punto in cui si ha il recupero della magnetizzazione, ovvero dove il segnale misurato è massimo, si trova in corrispondenza del tempo di echo, \(T_{E} = 2\tau\).

Durante il processo, gli inviluppo vanno come \(T_{2}^{*}\), poiché legati alle disomogeneità di campo ma, tuttavia, questo è un tempo fittizio, poiché l'unico vero tempo è quello di rilassamento trasversale \(T_{2}\). L'iterazione spin-spin, quantificata appunto dal tempo \(T_{2}\), per com'è stata modellata non dipende dalle variazioni del campo principale, quindi, il suo effetto continua ad agire durante tutta la sequenza di applicazione degli impulsi.

L'ampiezza degli impulsi registrati dalle antenne è modulata anche dalla costante di tempo \(T_{2}\). In altre parole, il primo decadimento e il successivo rifasamento-defasamento presentano delle ampiezze pesate da un termine \(\exp\left( - t/T_{2} \right)\), con cui la magnetizzazione si rilassa nel piano trasversale:

\[M_{\bot} \propto \exp\left( - \dfrac{t}{T_{2}} \right)\]

In definitiva, la magnetizzazione al tempo di echo è ridotta di \(\exp\left( - T_{E}/T_{2} \right)\) rispetto alla magnetizzazione misurata in corrispondenza del primo impulso a \(\pi/2\).

Un primo metodo per ottenere una stima del tempo \(T_{2}\) da una sequenza di impulsi spin-echo consiste nell'attivare una finestra di acquisizione dopo un tempo \(\tau\) dal primo impulso. Il segnale ricevuto riguarda la rifocalizzazione degli spin e il conseguente defasamento.

Il segnale ricevuto è registrato dalle antenne, demodulato e, in seguito, campionato così da poter eseguire delle elaborazioni digitali su di esso.

\begin{figure}
\centering
\includegraphics[width=2.68333in,height=2.01894in]{media/7_MRISignal/image126.pdf}\caption{Figura .: Elaborazione del segnale registrato}
\end{figure}

È possibile misurare il picco dell'echo ricevuto, ricavando informazioni sulla magnetizzazione trasversale, che a sua volta dipende dal tempo \(T_{2}\):

\[M_{\bot} \propto \exp\left( - \dfrac{T_{E}}{T_{2}} \right)\]

In questo modo si ottengono informazioni su \(T_{2}\), annullando gli effetti del tempo \(T_{2}^{*}\). Tuttavia, la misura di un solo segnale non è sufficiente a valutare il tempo di rilassamento trasversale con buona precessione.

La modulazione del segnale registrato come \(\exp\left( - t/T_{2} \right)\) è dovuto al fatto che il vettore di magnetizzazione \(\overset{\underline{}}{M}\) evolve secondo le leggi di Bloch, mentre i vari spin con stessa frequenza (iscocromati) hanno una fase casuale. L'effetto risultante è un segnale trasversale che decresce come \(\exp\left( - t/T_{2}^{*} \right)\) per effetto delle disomogeneità del campo; ciò porta gli spin isocromati a sfasarsi rapidamente rispetti ad altri isocromati, quindi, la magnetizzazione trasversa decade velocemente a zero.

Il vettore di magnetizzazione, per le leggi di Bloch, ha un'ampiezza che decresce come \(\exp\left( - t/T_{2} \right)\). Grazie a questo fenomeno è possibile misurare il tempo di rilassamento trasversale.

\subsection{Multiple spin-echo}\label{multiple-spin-echo}

Esistono due principali soluzioni per stimare di rilassamento trasversale \(T_{2}\) mediante la sequenza spin-echo, entrambe basata sull'acquisizione di segnali con tempi di echo diversi.

\subsubsection{Applicazione multipla della sequenza spin-echo}\label{applicazione-multipla-della-sequenza-spin-echo}

Una prima strategia per ottenere una misura del tempo di rilassamento trasversale \(T_{2}\) consiste nell'applicazione di due sequenze spin-echo con impulsi a \(\pi\) distanziati a tempi di echo diversi. In questa soluzione, la sequenza consiste nell'applicazione di due sequenze spin-echo, separate da un certo intervallo di tempo. La prima sequenza presenta un tempo di echo indicato con \(T_{E_{1}}\), mentre la seconda di \(T_{E_{2}}\). In altre parole, subito dopo il decadimento della componente trasversa a valle della prima applicazione della spin-echo, il segnale acquisito nella seconda applicazione della spin-echo, parte della stessa condizione iniziale della prima ed evolve con stessa dinamica del tipo \(\exp\left( - t/T_{2} \right)\).

\begin{figure}
\centering
\includegraphics[width=5.52083in,height=2.62445in]{media/7_MRISignal/image127.pdf}\caption{Figura .: Sequenze spin-echo con tempi di eco diversi}
\end{figure}

Successivamente, nelle finestre di acquisizione si registrano due echi, il primo con tempo \(T_{E_{1}}\) e il secondo con un tempo di echo \(T_{E_{2}}\) diversi tra loro. Dato il decadimento uguale per entrambe le sequenze spin-echo, acquisendo il segnale nella prima finestra si registra un segnale \(s\left( T_{E_{1}} \right)\) proporzionale al decadimento esponenziale delle componenti trasverse del vettore di magnetizzazione:

\[s\left( T_{E_{1}} \right) \propto M_{\bot} \propto \exp\left( - \dfrac{T_{E_{1}}}{T_{2}} \right)\]

Nella seconda finestra di acquisizione si registra un segnale al tempo di echo \(T_{E_{2}}\), proporzionale alla magnetizzazione trasversa:

\[s\left( T_{E_{2}} \right) \propto M_{\bot} \propto \exp\left( - \dfrac{T_{E_{2}}}{T_{2}} \right)\]

I due segnali ottenuti sono correlati, poiché la magnetizzazione è forzata a evolvere con lo stesso valore iniziale, ovvero \(M_{\bot}\left( \overset{\underline{}}{r},0 \right)\) è comune a entrambe le sequenze. Calcolando il rapporto tra i segnali ottenuti, i fattori di proporzionalità si semplificano, poiché uguali nei due termini, quindi si ottiene:

\[\dfrac{s\left( T_{E_{1}} \right)}{s\left( T_{E_{2}} \right)} = \dfrac{\exp\left( - \dfrac{T_{E_{1}}}{T_{2}} \right)}{\exp\left( - \dfrac{T_{E_{2}}}{T_{2}} \right)}\]

Per semplicità di notazione, si pone:

\[s_{1} = s\left( T_{E_{1}} \right),\ \ s_{2} = s\left( T_{E_{2}} \right)\]

Per cui è possibile scrivere

\[\dfrac{s_{1}}{s_{2}} = \dfrac{\exp\left( - \dfrac{T_{E_{1}}}{T_{2}} \right)}{\exp\left( - \dfrac{T_{E_{2}}}{T_{2}} \right)} = \exp\left( - \dfrac{T_{E_{1}} - T_{E_{2}}}{T_{2}} \right)\]

Invertendo quest'ultima relazione, è possibile ottenere un'espressione per la valutazione di \(T_{2}\):

\[T_{2} = \dfrac{T_{E_{2}} - T_{E_{1}}}{\log\left( \dfrac{s_{1}}{s_{2}} \right)} = \dfrac{T_{E_{2}} - T_{E_{1}}}{\log\left( s_{1} \right) - \log\left( s_{2} \right)}\]

In questo modo si è ottenuta una misura del tempo \(T_{2}\).

Storicamente le prime misure biologiche del tempo \(T_{2}\) sui tessuti umani sfruttavano la metodica appena descritta. Tuttavia, le misure del tempo \(T_{2}\) eseguite sfruttano il segnale registrato, ai tempi di echo, dalle antenne e demodulato è affetta da un errore; infatti, i valori di \(s_{1}\) e \(s_{2}\) sono misurati con una certa incertezza, dunque, per la propagazione dell'errore, anche la valutazione dell'errore \(T_{2}\) è affetta da errore.

La stima del tempo \(T_{2}\) presenta un errore che dipende dall'errore commesso sulla valutazione dei segnali ai tempi di echo \(T_{E_{1}}\) e \(T_{E_{2}}\). Tale errore può essere anche molto importante. Inoltre, l'ipotesi fondamentale, su cui si basa la valutazione di \(T_{2}\) con questa tecnica è che la magnetizzazione iniziale delle due sequenze spin-echo sia la stessa. Affinché questa ipotesi sia verificata, tra una sequenza spin-echo e la successiva deve passare un tempo almeno pari a \(5T_{1}\), in modo che la magnetizzazione raggiunga l'equilibrio.

\subsubsection[Applicazione multipla dei gradienti a pi]{Applicazione multipla dei gradienti a $\mathbf{\pi}$}
\label{applicazione-gradienti-pi}

Una soluzione per stimare il tempo di rilassamento trasversale \(T_{2}\) con maggiore precisione prevede l'acquisizione di più echi, mediante una sequenza nota come multiple echo. Si applica, cioè, un primo impulso a \(\pi/2\), il quale perturba l'equilibrio del campione. In seguito, si applica una serie di impulsi a \(\pi\) distanziati da un tempo \(\tau = T_{E}/2\).

\begin{figure}
\centering
\includegraphics[width=4.64394in,height=2.92077in]{media/7_MRISignal/image128.pdf}\caption{Figura .: Sequenza multiple spin-echo}
\end{figure}

Con questa soluzione, mediante un'unica stimolazione inziale a \(\pi/2\), si acquisiscono più misure della magnetizzazione, così da valutare al meglio il tempo \(T_{2}\).

Dal punto di vista del sistema rotante, il primo impulso a \(\pi/2\) ribalta la magnetizzazione sul piano trasverso, in modo che gli spin precedano su uno degli assi \(x'\) o \(y'\). Se l'impulso è applicato lungo l'asse \(y'\), ad esempio, la magnetizzazione si focalizza su questo asse. Gli spin si allontanano in senso orario dall'asse \(y'\) se precedono con una velocità maggiore della rotazione del sistema di riferimento, altrimenti in senso antiorario.

Si applica l'impulso a \(\pi\), ad esempio, diretto lungo l'asse \(x'\). Gli spin sono così ribaltati in modo da occupare una posizione speculare rispetto all'asse su cui è applicato l'impulso, \(x'\), rispetto a quella che possedeva prima dell'impulso.

Gli spin si rifocalizzano su \(- y'\), emettendo l'echo, e, in seguito, si defocalizzano a causa delle diverse velocità di precessione.

Appena gli spin sono abbastanza defocalizzati, si applica un secondo impulso a \(\pi\) intorno all'asse \(x'\). Questa volta gli spin si focalizzano su \(y'\). Ripetendo la sequenza, gli spin si focalizzano alternativamente su \(y'\) e \(- y'\).

Analogamente, se gli impulsi a \(\pi\) sono applicati lungo \(y'\), gli spin si focalizzano alternativamente su \(x'\) e \(- x'\).

% lista delle immagini (senza virgole!)
\def\imagelist{
image129, image130, image131, image132, image133, image134, image135, image136,
image137, image138, image139, image140, image141, image142, image143, image144,
image145, image146, image147, image148, image149, image150, image151, image152,
image153, image154, image155, image156, image157, image158, image159, image160,
image161, image162, image163, image164, image165, image166, image167, image168,
image169, image170, image171, image172, image173, image174, image175, image176,
image177, image178, image179, image180, image181, image182, image183, image184,
image185, image186, image187, image188, image189, image190, image191, image192,
image193, image194, image195, image196, image197, image198, image199, image200,
image201, image202, image203, image204, image205, image208}

% --- corpo principale ---
\setcounter{imgcount}{0}

\begin{center}
\foreach \image in \imagelist {%
    \includegraphics[width=\imgwidth]{media/7_MRISignal/\image.pdf}%
    \stepcounter{imgcount}%
    \ifnum\value{imgcount}<\imagesperrow
        \hspace{0.02\textwidth}% piccolo spazio tra immagini
    \else
        \par\vspace{0cm}% fine riga
        \setcounter{imgcount}{0}% reset del contatore
    \fi
}
\captionof{figure}{Movimento degli spin nella sequenza multiple spin-echo}
\end{center}

Gli impulsi di focalizzazione permettono di misurare un valore del segnale proporzionale alla magnetizzazione, in corrispondenza dei tempi di echo. Tutte le misure condividono lo stesso valore iniziale.

In definitiva, il processo di misura equivale a un campionamento del segnale \(M_{\bot} \propto \exp\left( - t/T_{2} \right)\), con periodo di campionamento \(T_{E}\). Ciò permette una valutazione più accurata del tempo di rilassamento trasversale \(T_{2}\).

In particolare, il tempo di rilassamento \(T_{2}\) è dell'ordine delle centinaia di \(ms\), per cui, affinché il segnale \(\exp\left( - t/T_{2} \right)\) possa considerarsi estinto, è necessario aspettare un intervallo temporale di \(5T_{2}\sim 500\ ms\). Scegliendo un tempo di echo di \(10\ ms\), è possibile eseguire circa \(50\) misurazioni per ricostruire l'evoluzione temporale della magnetizzazione trasversale \(M_{\bot} \propto \exp\left( - t/T_{2} \right)\).

\paragraph[Stima del tempo T2 da n misurazioni]{Stima del tempo $\mathbf{T}_{\mathbf{2}}$ da $\mathbf{n}$ misurazioni}
\label{stima-tempo-T2-n-misurazioni}

Si suppone di applicare una sequenza multiple spin-echo. Il segnale misurato, a causa delle disomogeneità di campo, decresce con costante di tempo \(T_{2}^{*}\). Per l'applicazione degli impulsi a \(\pi\) ripetuti, l'ampiezza del picco massimo dell'echo si riduce come \(\exp\left( - t/T_{2} \right)\). Al tempo \(t_{n} = nT_{E}\) gli isocromi sono focalizzati sull'asse \(x'\) o su \(y'\). In questa condizione il segnale registrato ha un massimo, legati a \(\exp\left( - nT_{E}/T_{2} \right)\). Valutando i segnali registrati ai tempi di echo, si ottengono dei campioni del segnale \(\exp\left( - t/T_{2} \right)\), da cui è possibile ricavare il tempo di rilassamento trasversale \(T_{2}\).

Il segnale \(s(t)\) misurato è uguale al modello scelto per descrivere il comportamento del vettore di magnetizzazione, a cui si somma un termine di errore \(\varepsilon\), supposto essere additivo:

\[s(t) = \exp\left( - \dfrac{t}{T_{2}} \right) + \varepsilon\]

Il modello non corrisponde esattamente alle misure sperimentali.

Si valutano i campioni nei tempi di echi \(t_{n} = nT_{E}\):

\[s\left( nT_{E} \right) = s_{n} = \exp\left( - \dfrac{nT_{E}}{T_{2}} \right) + \varepsilon_{n}\]

Il termine di errore \(\varepsilon_{n}\) dipende dalla misura, in quanto ogni misura è affetta da un errore diverso dagli altri e statisticamente indipendenti.

Per valutare il tempo \(T_{2}\) dalle \(n\) misurazioni si applica il metodo dei minimi quadrati o \emph{Least Squares} (LS) o \emph{Ordinary Least Squares} (OLS) ideato da Gauss. A tale scopo, per rendere la relazione tra la misura e la quantità da valutare lineare si applica il logaritmo a ambo i membri dell'equazione per \(s_{n}\):

\[\log\left( s_{n} \right) = - \dfrac{nT_{E}}{T_{2}} + \varepsilon_{n}'\]

Dove \(\varepsilon_{n}'\) è un termine di errore additivo dipendente dal logaritmo dell'errore \(\varepsilon_{n}\). La relazione così scritta può essere scritta nella forma:

\[y = mx + q + \varepsilon\]

Dove \(y = \log\left( s_{n} \right)\), \(m = - nT_{E}/T_{2}\) e \(q \neq 0\) in generale.

Mediante \(n\) misurazioni si ottiene una popolazione di \(n\) coppie \(\left( y_{i},x_{i} \right)\), legati dalla relazione:

\[y_{i} = mx_{i} + q + \varepsilon_{i},\ \ i = 1,\ldots,n\]

Per ogni campione si ottiene una relazione lineare tra \(y\) e i coefficienti \(m\) e \(q\) della regressione. Il sistema di equazioni può essere scritto in forma matriciale:

\[\left\{ \begin{matrix}
y_{1} = mx_{1} + q + \varepsilon_{1} \\
\ldots \\
y_{n} = mx_{n} + q + \varepsilon_{n}
\end{matrix} \right.\ \]

Si introduce il vettore delle misure:

\[\overset{\underline{}}{y} = \left( \begin{array}{r}
y_{1} \\
y_{2} \\
\ldots \\
y_{n}
\end{array} \right)\]

Si introduce la matrice dei coefficienti o di design:

\[\overset{\underline{}}{\overset{\underline{}}{X}} = \begin{pmatrix}
x_{1} & 1 \\
x_{2} & 1 \\
\ldots & \ldots \\
x_{n} & 1
\end{pmatrix}\]

Infine, si definisce il vettore delle incognite, spesso indicato con il simbolo \(\overset{\underline{}}{\vartheta}\), con due sole componenti:

\[\overset{\underline{}}{\vartheta} = \left( \begin{array}{r}
m \\
q
\end{array} \right)\]

La relazione di regressione può essere scritta come:

\[\overset{\underline{}}{y} = \overset{\underline{}}{\overset{\underline{}}{X}}\overset{\underline{}}{\vartheta}\]

In generale, questa equazione è valida per ogni tipologia di regressione lineare e, dunque, anche il metodo proposto per valutare \(T_{2}\) è valido.

Si osservi che \(\overset{\underline{}}{\overset{\underline{}}{X}}\) è una matrice \(2 \times n\), quindi, non può essere invertita per ottenere la soluzione. In generale, la matrice dei coefficienti nel metodo dei minimi quadrati è di \(m \times n\), dove \(m\) è il numero delle incognite.

Sia \({\overset{\underline{}}{\vartheta}}^{*}\) il valore vero dei parametri incogniti; la seguente equazione:

\[{\overset{\underline{}}{y}}^{*} = \overset{\underline{}}{\overset{\underline{}}{X}}{\overset{\underline{}}{\vartheta}}^{*}\]

rappresenta il valore vero delle osservazioni.

I vari elementi del vettore delle misure \(\overset{\underline{}}{y}\) sono affetti da rumore approssimabile come variabili aleatori \(\varepsilon_{i}\). Sia:

\[\overset{\underline{}}{\varepsilon} = \left( \begin{array}{r}
\varepsilon_{1} \\
\varepsilon_{2} \\
\ldots \\
\varepsilon_{n}
\end{array} \right)\]

Si vuole trovare il vettore \(\overset{\underline{}}{\vartheta}\) tale da minimizzare l'errore quadratico medio, ovvero:

\[\widehat{\overset{\underline{}}{\vartheta}} = {\arg{\min_{\overset{\underline{}}{\vartheta}}\left\| \overset{\underline{}}{\varepsilon} \right\|}}^{2} = \arg{\min_{\overset{\underline{}}{\vartheta}}\left\| \overset{\underline{}}{y} - \overset{\underline{}}{\overset{\underline{}}{X}}\overset{\underline{}}{\vartheta} \right\|^{2}}\]

I parametri \(\overset{\underline{}}{\vartheta}\) devono rendere minima la distanza tra le misure \(\overset{\underline{}}{y}\) e le previsioni del modello \(\overset{\underline{}}{\overset{\underline{}}{X}}\overset{\underline{}}{\vartheta}\).

La quantità \(\left\| \overset{\underline{}}{y} - \overset{\underline{}}{\overset{\underline{}}{X}}\overset{\underline{}}{\vartheta} \right\|^{2}\) può essere scritta in forma matriciale come:

\[\left\| \overset{\underline{}}{y} - \overset{\underline{}}{\overset{\underline{}}{X}}\overset{\underline{}}{\vartheta} \right\|^{2} = \left( \overset{\underline{}}{y} - \overset{\underline{}}{\overset{\underline{}}{X}}\overset{\underline{}}{\vartheta} \right)^{T}\left( \overset{\underline{}}{y} - \overset{\underline{}}{\overset{\underline{}}{X}}\overset{\underline{}}{\vartheta} \right)\]

Si indica con \(S\left( \overset{\underline{}}{\vartheta} \right) = \left\| \overset{\underline{}}{y} - \overset{\underline{}}{\overset{\underline{}}{X}}\overset{\underline{}}{\vartheta} \right\|^{2}\), l'ultima relazione si scrive come:

\[S\left( \overset{\underline{}}{\vartheta} \right) = \left( \overset{\underline{}}{y} - \overset{\underline{}}{\overset{\underline{}}{X}}\overset{\underline{}}{\vartheta} \right)^{T}\left( \overset{\underline{}}{y} - \overset{\underline{}}{\overset{\underline{}}{X}}\overset{\underline{}}{\vartheta} \right)\]

Svolgendo il trasposto e i prodotti si ha:

\[S\left( \overset{\underline{}}{\vartheta} \right) = \left( \overset{\underline{}}{y} - \overset{\underline{}}{\overset{\underline{}}{X}}\overset{\underline{}}{\vartheta} \right)^{T}\left( \overset{\underline{}}{y} - \overset{\underline{}}{\overset{\underline{}}{X}}\overset{\underline{}}{\vartheta} \right) = \left( {\overset{\underline{}}{y}}^{T} - {\overset{\underline{}}{\vartheta}}^{T}{\overset{\underline{}}{\overset{\underline{}}{X}}}^{T} \right)\left( \overset{\underline{}}{y} - \overset{\underline{}}{\overset{\underline{}}{X}}\overset{\underline{}}{\vartheta} \right) = {\overset{\underline{}}{y}}^{T}\overset{\underline{}}{y} - {\overset{\underline{}}{y}}^{T}\overset{\underline{}}{\overset{\underline{}}{X}}\overset{\underline{}}{\vartheta} - {\overset{\underline{}}{\vartheta}}^{T}{\overset{\underline{}}{\overset{\underline{}}{X}}}^{T}\overset{\underline{}}{y} + {\overset{\underline{}}{\vartheta}}^{T}{\overset{\underline{}}{\overset{\underline{}}{X}}}^{T}\overset{\underline{}}{\overset{\underline{}}{X}}\overset{\underline{}}{\vartheta}\]

Le quantità \({\overset{\underline{}}{y}}^{T}\overset{\underline{}}{\overset{\underline{}}{X}}\overset{\underline{}}{\vartheta}\) e \({\overset{\underline{}}{\vartheta}}^{T}{\overset{\underline{}}{\overset{\underline{}}{X}}}^{T}\overset{\underline{}}{y}\) sono degli scalari, quindi, possono essere sommati tra loro:

\[S\left( \overset{\underline{}}{\vartheta} \right) = {\overset{\underline{}}{y}}^{T}\overset{\underline{}}{y} + {\overset{\underline{}}{\overset{\underline{}}{X}}}^{T}\overset{\underline{}}{\overset{\underline{}}{X}}{\overset{\underline{}}{\vartheta}}^{2} - 2{\overset{\underline{}}{y}}^{T}\overset{\underline{}}{\overset{\underline{}}{X}}\overset{\underline{}}{\vartheta}\]

Minimizzare il valore quadratico medio equivale a uguagliare a zero la derivata della quantità \(S\left( \overset{\underline{}}{\vartheta} \right)\):

\[\dfrac{\partial S\left( \overset{\underline{}}{\vartheta} \right)}{\partial\overset{\underline{}}{\vartheta}} = 0 \Leftrightarrow 2{\overset{\underline{}}{\overset{\underline{}}{X}}}^{T}\overset{\underline{}}{\overset{\underline{}}{X}}\overset{\underline{}}{\vartheta} - 2{\overset{\underline{}}{y}}^{T}\overset{\underline{}}{\overset{\underline{}}{X}} = 0 \Leftrightarrow {\overset{\underline{}}{\overset{\underline{}}{X}}}^{T}\overset{\underline{}}{\overset{\underline{}}{X}}\overset{\underline{}}{\vartheta} = {\overset{\underline{}}{y}}^{T}\overset{\underline{}}{\overset{\underline{}}{X}}\]

Risulta che:

\[{\overset{\underline{}}{y}}^{T}\overset{\underline{}}{\overset{\underline{}}{X}} = {\overset{\underline{}}{\overset{\underline{}}{X}}}^{T}\overset{\underline{}}{y}\]

Per cui si ottiene:

\[{\overset{\underline{}}{\overset{\underline{}}{X}}}^{T}\overset{\underline{}}{\overset{\underline{}}{X}}\overset{\underline{}}{\vartheta} = {\overset{\underline{}}{\overset{\underline{}}{X}}}^{T}\overset{\underline{}}{y}\]

Moltiplicando a destra e a sinistra per \(\left( {\overset{\underline{}}{\overset{\underline{}}{X}}}^{T}\overset{\underline{}}{\overset{\underline{}}{X}} \right)^{- 1}\), si ottiene l'equazione per la soluzione \emph{ordinary least squares}:

\[\widehat{\overset{\underline{}}{\vartheta}} = \left( {\overset{\underline{}}{\overset{\underline{}}{X}}}^{T}\overset{\underline{}}{\overset{\underline{}}{X}} \right)^{- 1}{\overset{\underline{}}{\overset{\underline{}}{X}}}^{T}\overset{\underline{}}{y}\]

Si suppone che il rumore sia a media nulla, ovvero:

\[E\left\lbrack \overset{\underline{}}{\varepsilon} \right\rbrack = \overset{\underline{}}{0}\]

La matrice della covarianza è data da:

\[E\left\lbrack \overset{\underline{}}{\varepsilon}{\overset{\underline{}}{\varepsilon}}^{T} \right\rbrack = \sigma^{2}\overset{\underline{}}{\overset{\underline{}}{I}}\]

Si dimostra che la media della soluzione OLS tende al valore vero dei parametri incogniti \({\overset{\underline{}}{\vartheta}}^{*}\), infatti la media di \(\widehat{\overset{\underline{}}{\vartheta}}\) è data da:

\[E\left\lbrack \widehat{\overset{\underline{}}{\vartheta}} \right\rbrack = E\left\lbrack \left( {\overset{\underline{}}{\overset{\underline{}}{X}}}^{T}\overset{\underline{}}{\overset{\underline{}}{X}} \right)^{- 1}{\overset{\underline{}}{\overset{\underline{}}{X}}}^{T}\overset{\underline{}}{y} \right\rbrack\]

Ma \(\overset{\underline{}}{y} = \overset{\underline{}}{\overset{\underline{}}{X}}{\overset{\underline{}}{\vartheta}}^{*} + \overset{\underline{}}{\varepsilon}\); inoltre la matrice di design non contiene variabili aleatorie, quindi, può essere portato fuori dall'operazione di media statistica:

\[E\left\lbrack \widehat{\overset{\underline{}}{\vartheta}} \right\rbrack = E\left\lbrack \left( {\overset{\underline{}}{\overset{\underline{}}{X}}}^{T}\overset{\underline{}}{\overset{\underline{}}{X}} \right)^{- 1}{\overset{\underline{}}{\overset{\underline{}}{X}}}^{T}\overset{\underline{}}{y} \right\rbrack = \left( {\overset{\underline{}}{\overset{\underline{}}{X}}}^{T}\overset{\underline{}}{\overset{\underline{}}{X}} \right)^{- 1}{\overset{\underline{}}{\overset{\underline{}}{X}}}^{T}E\left\lbrack \overset{\underline{}}{y} \right\rbrack = \left( {\overset{\underline{}}{\overset{\underline{}}{X}}}^{T}\overset{\underline{}}{\overset{\underline{}}{X}} \right)^{- 1}{\overset{\underline{}}{\overset{\underline{}}{X}}}^{T}E\left\lbrack \overset{\underline{}}{\overset{\underline{}}{X}}{\overset{\underline{}}{\vartheta}}^{*} + \overset{\underline{}}{\varepsilon} \right\rbrack =\]

Per la linearità dell'operatore valor medio si ha:

\[= \left( {\overset{\underline{}}{\overset{\underline{}}{X}}}^{T}\overset{\underline{}}{\overset{\underline{}}{X}} \right)^{- 1}{\overset{\underline{}}{\overset{\underline{}}{X}}}^{T}\left( \overset{\underline{}}{\overset{\underline{}}{X}}E\left\lbrack {\overset{\underline{}}{\vartheta}}^{*} \right\rbrack + E\left\lbrack \overset{\underline{}}{\varepsilon} \right\rbrack \right)\]

Per ipotesi il rumore è a media nulla, per cui:

\[E\left\lbrack \widehat{\overset{\underline{}}{\vartheta}} \right\rbrack = \left( {\overset{\underline{}}{\overset{\underline{}}{X}}}^{T}\overset{\underline{}}{\overset{\underline{}}{X}} \right)^{- 1}{\overset{\underline{}}{\overset{\underline{}}{X}}}^{T}\overset{\underline{}}{\overset{\underline{}}{X}}E\left\lbrack {\overset{\underline{}}{\vartheta}}^{*} \right\rbrack\]

Per definizione di matrice inversa risulta:

\[E\left\lbrack \widehat{\overset{\underline{}}{\vartheta}} \right\rbrack = E\left\lbrack {\overset{\underline{}}{\vartheta}}^{*} \right\rbrack\]

La stima di \(T_{2}\) con questo metodo è detto non polarizzata o \emph{unbiased}.

Si calcola, ora, la matrice di covarianza; secondo la definizione si ha:

\[E\left\lbrack \left( \widehat{\overset{\underline{}}{\vartheta}} - {\overset{\underline{}}{\vartheta}}^{*} \right)\left( \widehat{\overset{\underline{}}{\vartheta}} - {\overset{\underline{}}{\vartheta}}^{*} \right)^{T} \right\rbrack = E\left\lbrack \left( \left( {\overset{\underline{}}{\overset{\underline{}}{X}}}^{T}\overset{\underline{}}{\overset{\underline{}}{X}} \right)^{- 1}{\overset{\underline{}}{\overset{\underline{}}{X}}}^{T}\left( \overset{\underline{}}{\overset{\underline{}}{X}}{\overset{\underline{}}{\vartheta}}^{*} + \overset{\underline{}}{\varepsilon} \right) - {\overset{\underline{}}{\vartheta}}^{*} \right)\left( \left( {\overset{\underline{}}{\overset{\underline{}}{X}}}^{T}\overset{\underline{}}{\overset{\underline{}}{X}} \right)^{- 1}{\overset{\underline{}}{\overset{\underline{}}{X}}}^{T}\left( \overset{\underline{}}{\overset{\underline{}}{X}}{\overset{\underline{}}{\vartheta}}^{*} + \overset{\underline{}}{\varepsilon} \right) - {\overset{\underline{}}{\vartheta}}^{*} \right)^{T} \right\rbrack =\]

Svolgendo i prodotti si ha:

\[= E\left\lbrack \left( \left( {\overset{\underline{}}{\overset{\underline{}}{X}}}^{T}\overset{\underline{}}{\overset{\underline{}}{X}} \right)^{- 1}{\overset{\underline{}}{\overset{\underline{}}{X}}}^{T}\overset{\underline{}}{\overset{\underline{}}{X}}{\overset{\underline{}}{\vartheta}}^{*} + \left( {\overset{\underline{}}{\overset{\underline{}}{X}}}^{T}\overset{\underline{}}{\overset{\underline{}}{X}} \right)^{- 1}{\overset{\underline{}}{\overset{\underline{}}{X}}}^{T}\overset{\underline{}}{\varepsilon} - {\overset{\underline{}}{\vartheta}}^{*} \right)\left( \left( {\overset{\underline{}}{\overset{\underline{}}{X}}}^{T}\overset{\underline{}}{\overset{\underline{}}{X}} \right)^{- 1}{\overset{\underline{}}{\overset{\underline{}}{X}}}^{T}\overset{\underline{}}{\overset{\underline{}}{X}}{\overset{\underline{}}{\vartheta}}^{*} + \left( {\overset{\underline{}}{\overset{\underline{}}{X}}}^{T}\overset{\underline{}}{\overset{\underline{}}{X}} \right)^{- 1}{\overset{\underline{}}{\overset{\underline{}}{X}}}^{T}\overset{\underline{}}{\varepsilon} - {\overset{\underline{}}{\vartheta}}^{*} \right)^{T} \right\rbrack =\]

Per la proprietà della matrice inversa, si ha:

\[= E\left\lbrack \left( {\overset{\underline{}}{\vartheta}}^{*} + \left( {\overset{\underline{}}{\overset{\underline{}}{X}}}^{T}\overset{\underline{}}{\overset{\underline{}}{X}} \right)^{- 1}{\overset{\underline{}}{\overset{\underline{}}{X}}}^{T}\overset{\underline{}}{\varepsilon} - {\overset{\underline{}}{\vartheta}}^{*} \right)\left( {\overset{\underline{}}{\vartheta}}^{*} + \left( {\overset{\underline{}}{\overset{\underline{}}{X}}}^{T}\overset{\underline{}}{\overset{\underline{}}{X}} \right)^{- 1}{\overset{\underline{}}{\overset{\underline{}}{X}}}^{T}\overset{\underline{}}{\varepsilon} - {\overset{\underline{}}{\vartheta}}^{*} \right)^{T} \right\rbrack\]

Semplificando e svolgendo i prodotti si ha:

\[= E\left\lbrack \left( \left( {\overset{\underline{}}{\overset{\underline{}}{X}}}^{T}\overset{\underline{}}{\overset{\underline{}}{X}} \right)^{- 1}{\overset{\underline{}}{\overset{\underline{}}{X}}}^{T}\overset{\underline{}}{\varepsilon} \right)\left( \left( {\overset{\underline{}}{\overset{\underline{}}{X}}}^{T}\overset{\underline{}}{\overset{\underline{}}{X}} \right)^{- 1}{\overset{\underline{}}{\overset{\underline{}}{X}}}^{T}\overset{\underline{}}{\varepsilon} \right)^{T} \right\rbrack = E\left\lbrack \left( \left( {\overset{\underline{}}{\overset{\underline{}}{X}}}^{T}\overset{\underline{}}{\overset{\underline{}}{X}} \right)^{- 1}{\overset{\underline{}}{\overset{\underline{}}{X}}}^{T}\overset{\underline{}}{\varepsilon} \right)\left( {\overset{\underline{}}{\varepsilon}}^{T}\overset{\underline{}}{\overset{\underline{}}{X}}\left( {\overset{\underline{}}{\overset{\underline{}}{X}}}^{T}\overset{\underline{}}{\overset{\underline{}}{X}} \right)^{- T} \right) \right\rbrack = \left( {\overset{\underline{}}{\overset{\underline{}}{X}}}^{T}\overset{\underline{}}{\overset{\underline{}}{X}} \right)^{- 1}{\overset{\underline{}}{\overset{\underline{}}{X}}}^{T}E\left\lbrack \overset{\underline{}}{\varepsilon}{\overset{\underline{}}{\varepsilon}}^{T} \right\rbrack\overset{\underline{}}{\overset{\underline{}}{X}}\left( {\overset{\underline{}}{\overset{\underline{}}{X}}}^{T}\overset{\underline{}}{\overset{\underline{}}{X}} \right)^{- T}\]

Ma, per ipotesi \(E\left\lbrack \overset{\underline{}}{\varepsilon}{\overset{\underline{}}{\varepsilon}}^{T} \right\rbrack\overset{\underline{}}{\overset{\underline{}}{X}} = \sigma^{2}\overset{\underline{}}{\overset{\underline{}}{I}}\), per cui:

\[E\left\lbrack \left( \widehat{\overset{\underline{}}{\vartheta}} - {\overset{\underline{}}{\vartheta}}^{*} \right)\left( \widehat{\overset{\underline{}}{\vartheta}} - {\overset{\underline{}}{\vartheta}}^{*} \right)^{T} \right\rbrack = \left( {\overset{\underline{}}{\overset{\underline{}}{X}}}^{T}\overset{\underline{}}{\overset{\underline{}}{X}} \right)^{- 1}{\overset{\underline{}}{\overset{\underline{}}{X}}}^{T}\sigma^{2}\overset{\underline{}}{\overset{\underline{}}{I}}\overset{\underline{}}{\overset{\underline{}}{X}}\left( {\overset{\underline{}}{\overset{\underline{}}{X}}}^{T}\overset{\underline{}}{\overset{\underline{}}{X}} \right)^{- T} = \sigma^{2}\left( {\overset{\underline{}}{\overset{\underline{}}{X}}}^{T}\overset{\underline{}}{\overset{\underline{}}{X}} \right)^{- 1}{\overset{\underline{}}{\overset{\underline{}}{X}}}^{T}\overset{\underline{}}{\overset{\underline{}}{X}}\left( {\overset{\underline{}}{\overset{\underline{}}{X}}}^{T}\overset{\underline{}}{\overset{\underline{}}{X}} \right)^{- T}\]

Dove \({\overset{\underline{}}{\overset{\underline{}}{X}}}^{T}\overset{\underline{}}{\overset{\underline{}}{X}}\) è una matrice simmetrica, per cui l'inversa è anch'essa simmetrica, ovvero:

\[\left\lbrack \left( {\overset{\underline{}}{\overset{\underline{}}{X}}}^{T}\overset{\underline{}}{\overset{\underline{}}{X}} \right)^{- 1} \right\rbrack^{T} = \left\lbrack \left( {\overset{\underline{}}{\overset{\underline{}}{X}}}^{T}\overset{\underline{}}{\overset{\underline{}}{X}} \right)^{T} \right\rbrack^{- 1} = \left( {\overset{\underline{}}{\overset{\underline{}}{X}}}^{T}\overset{\underline{}}{\overset{\underline{}}{X}} \right)^{- 1}\]

Per cui:

\[\left( {\overset{\underline{}}{\overset{\underline{}}{X}}}^{T}\overset{\underline{}}{\overset{\underline{}}{X}} \right)^{- 1}{\overset{\underline{}}{\overset{\underline{}}{X}}}^{T}\overset{\underline{}}{\overset{\underline{}}{X}} = \overset{\underline{}}{\overset{\underline{}}{I}}\]

Di conseguenza:

\[E\left\lbrack \left( \widehat{\overset{\underline{}}{\vartheta}} - {\overset{\underline{}}{\vartheta}}^{*} \right)\left( \widehat{\overset{\underline{}}{\vartheta}} - {\overset{\underline{}}{\vartheta}}^{*} \right)^{T} \right\rbrack = \sigma^{2}\left( {\overset{\underline{}}{\overset{\underline{}}{X}}}^{T}\overset{\underline{}}{\overset{\underline{}}{X}} \right)^{- 1}\]

La curva stimata per ricostruire le misure può differire da quella teorica sia per eccesso che per difetto; tuttavia, non vi è modo di prevedere né di conoscere esattamente il parametro stimato. Ovviamente, minore è l'errore additivo e più precisa sarà la stima di \(T_{2}\) mediante il metodo dei minimi quadrati.

\begin{figure}
\centering
\includegraphics[width=3.78269in,height=3.17361in,alt={Exponential Fitting Using OriginLab 2021 \textbar{} \textbar{} Drawing/Graphing-27}]{media/7_MRISignal/image212.pdf}\caption{Figura .: Fitting della curva esponenziale}
\end{figure}

\subsection{Inversion recovery}\label{inversion-recovery}

Le sequenze FID e spin-echo sono molto utili per determinare il tempo di rilassamento trasversale \(T_{2}\) e il tempo \(T_{2}^{*}\). Questi due esperimenti, tuttavia, non permettono di valutare il tempo di rilassamento longitudinale \(T_{1}\).

Esiste un esperimento noto come \emph{inversion recovery} che permette di ottenere il tempo \(T_{1}\). Tale metodo può essere impiegato per ottenere una valutazione precisa del tempo di rilassamento longitudinale mediante una sola misura. Nello specifico, la sequenza \emph{inversion recovery} è simile alla spin-echo in cui l'impulso a \(\pi/2\) segue, dopo un opportuno intervallo temporale, un impulso a \(\pi\).

Nella misura di \(T_{1}\) attraverso l'\emph{inversion recovery} è necessario analizzare l'evoluzione temporale della componente longitudinale del vettore di magnetizzazione, regolata dall'equazione di Bloch:

\[\dfrac{dM_{z}}{dt} = \dfrac{1}{T_{1}}\left( M_{0} - M_{z} \right)\]

Dopo l'impulso a \(\pi\) la magnetizzazione è ruotata in modo da essere orientata lungo la direzione negativa dell'asse \(z\).

\begin{figure}
\centering
\includegraphics[width=3.95833in,height=4.19583in]{media/7_MRISignal/image213.pdf}\caption{Figura .: Inversione della magnetizzazione dopo l'impulso a \(\pi\)}
\end{figure}

Sia \(t = 0\) l'istante in cui l'impulso a \(\pi\) è interrotto. La magnetizzazione, in questo istante, è ribaltata di \(\pi\), per cui:

\[M(0) = - M_{0}\]

Dove \(M_{0}\) è il valore della magnetizzazione all'equilibrio termodinamico.

Subito dopo l'applicazione dell'impulso a \(\pi\), il vettore di magnetizzazione evolve mediante un andamento esponenziale del tipo:

\[M_{z}(t) = M_{0}\left\lbrack 1 - \exp\left( - \dfrac{t}{T_{1}} \right) \right\rbrack - M_{0}\exp\left( - \dfrac{t}{T_{1}} \right) = M_{0}\left\lbrack 1 - 2\exp\left( - \dfrac{t}{T_{1}} \right) \right\rbrack\]

Al tempo \(t = T_{I}\) si applica l'impulso a \(\pi/2\) lungo uno degli assi trasversali, \(x'\) o \(y'\). A causa di ciò la magnetizzazione longitudinale è rovesciata nel piano trasversale tramite l'impulso a \(\pi/2\).

\begin{figure}
\centering
\includegraphics[width=5.65428in,height=3.825in]{media/7_MRISignal/image214.pdf}\caption{Figura .: Andamento della componente longitudinale nel tempo}
\end{figure}

Dall'equazione per \(M_{z}(t)\) si nota che esiste un certo istante temporale in cui la magnetizzazione longitudinale si annulla. Se si applica in questo istante l'impulso a \(\pi/2\) non si registrerebbe nessun segnale, in quanto il vettore di magnetizzazione è nullo, dunque, non può essere ribaltato.

Ogni tessuto presenta un tempo di rilassamento \(T_{1}\), quindi, applicando un impulso a \(\pi/2\) nel momento opportuno è possibile ottenere informazioni solamente da alcuni tessuti, mentre altri sono rimossi, in quanto non danno nessun contributo alla magnetizzazione longitudinale. In altre parole, i tessuti che presentano una magnetizzazione longitudinale nulla al tempo \(T_{I}\), \(M_{z}\left( T_{I} \right) = 0\), non contribuiscono all'immagine.

Questa tecnica è utilizzata per rimuovere il grasso, processo noto come \emph{fat suppression}. Noto il tempo \(T_{1}\) del grasso, si applica un impulso a \(\pi\) per ribaltare la magnetizzazione, mentre il secondo impulso a \(\pi/2\) è applicato quando la magnetizzazione del grasso è nulla, così che questo tessuto non contribuisce all'immagine.

Per ottenere la misura di \(T_{1}\) attraverso la sequenza \emph{inversion recovery}, si osservi che al tempo \(t = 0^{+}\), a fine applicazione dell'impulso a \(\pi\), si ha una magnetizzazione orientata lungo \(- {\widehat{i}}_{z}\):

\[M\left( 0^{+} \right) = - M_{0}\]

Prima del tempo \(T_{I}\) di applicazione dell'impulso a \(\pi/2\), il vettore di magnetizzazione longitudinale, ritorna all'equilibrio mediante la l'equazione:

\[M_{z}(t) = M_{0}\left\lbrack 1 - 2\exp\left( - \dfrac{t}{T_{1}} \right) \right\rbrack,\ \ 0 < t < T_{I}\]

Al tempo di inversione, si applica l'impulso a \(\pi/2\) che ribalta la magnetizzazione nel piano trasverso. Al tempo \(T_{I}\), la magnetizzazione longitudinale ha ampiezza:

\[M_{z}\left( T_{I} \right) = M_{0}\left\lbrack 1 - 2\exp\left( - \dfrac{T_{I}}{T_{1}} \right) \right\rbrack\]

La componente trasversale evolve, di conseguenza, mediante la legge:

\[M_{\bot}(t) = \left| M_{0}\left\lbrack 1 - 2\exp\left( - \dfrac{T_{I}}{T_{1}} \right) \right\rbrack \right|\exp\left( - \dfrac{t - T_{I}}{T_{2}^{*}} \right)\]

L'ampiezza del segnale misurato è modulata da un fattore dipendente dal tempo di rilassamento longitudinale \(T_{1}\) tramite \(\left| M_{0}\left\lbrack 1 - 2\exp\left( - T_{I}/T_{1} \right) \right\rbrack \right|\). Il segnale registrato dalle antenne, posizionate in modo da acquisire la magnetizzazione trasversale, si annulla quando:

\[\left| M_{0}\left\lbrack 1 - 2\exp\left( - \dfrac{T_{I}}{T_{1}} \right) \right\rbrack \right| = 0 \Leftrightarrow \exp\left( - \dfrac{T_{I}}{T_{1}} \right) = \dfrac{1}{2}\]

Questa relazione permette di ricavare il tempo di applicazione dell'impulso a \(\pi/2\) in funzione del tempo di rilassamento longitudinale di un tessuto:

\[- \dfrac{T_{I}}{T_{1}} = \log\left( \dfrac{1}{2} \right) \Leftrightarrow \dfrac{T_{I}}{T_{1}} = - \log\left( \dfrac{1}{2} \right) \Leftrightarrow \dfrac{T_{I}}{T_{1}} = \log(2)\]

Da cui:

\[T_{I} = T_{1}\log(2)\]

\begin{figure}
\centering
\includegraphics[width=4.41074in,height=3.03691in,alt={Immagine che contiene diagramma, linea, Diagramma Descrizione generata automaticamente}]{media/7_MRISignal/image215.pdf}\caption{Figura .: Andamento della magnetizzazione trasversale nel tempo nell'inversion recovery}
\end{figure}

Il segnale registrato \(s(t)\) è una funzione del tempo di rilassamento longitudinale \(T_{1}\) la cui valutazione richiede una scelta oculata del tempo di \emph{inversion recovery} \(T_{I}\), in quanto la magnetizzazione non deve essere nulla per il tessuto di interesse.

Noto il tessuto è possibile ricavare il valore \(T_{I}\) per annullare i suoi contributi alla magnetizzazione trasversale. Questo azzeramento permette di cancellare i contributi di un tessuti di interesse. Ad esempio, eccitando i tessuti di un paziente con la sequenza spin-echo, a valle di un \emph{inversion recovery}, è possibile scegliere il tempo di echo tale che la magnetizzazione trasversa del tessuto da sopprimere sia nulla.

\begin{center}
\vfill
    \chapter{Uso dei gradienti in MRI}
    \label{blx:Grad\therefsection}
\vfill

\minitoc
\newpage
\end{center}
\justify

\section{Gradienti di campo magnetico}\label{gradienti-di-campo-magnetico}

Un gradiente di campo è un campo magnetico che viene aggiunto a \(B_{0}\), la cui intensità varia linearmente con la posizione lungo un asse scelto.

\subsection{Segnale a valle della demodulazione complessa}\label{segnale-a-valle-della-demodulazione-complessa}

Sia \(s(t)\) il segnale ottenuto a valle della demodulazione, dato da:

\[s(t) \propto \omega_{0}B_{\bot}\int_{V}^{}{M_{\bot}\left( \overset{\underline{}}{r},0 \right)\exp\left\{ j\left\lbrack \Omega t + \phi\left( \overset{\underline{}}{r},t \right) \right\rbrack \right\} dV}\]

Si introduce la costante \(\Lambda\) che include i fattori di proporzionalità del sistema di detezione elettronica e di altri fattori di proporzionalità tra l'integrale di volume e il segnale:

\[s(t) = \Lambda\omega_{0}B_{\bot}\int_{V}^{}{M_{\bot}\left( \overset{\underline{}}{r},0 \right)\exp\left\{ j\left\lbrack \Omega t + \phi\left( \overset{\underline{}}{r},t \right) \right\rbrack \right\} dV}\]

Questa relazione è valida se il campo \(B_{\bot}\), irradiato dall'antenna, nel tempo sia sufficientemente uniforme nel volume irradiato. In questa ipotesi, sia il campo \(B_{\bot}\) sia la fase iniziale del campo irradiato non dipendono dalla posizione.

La quantità \(\Omega\) è la frequenza di rotazione del sistema di riferimento rotante, ovvero la frequenza alla quale si demodula. In assenza di disomogeneità di campo principale \(\Omega\) coincide cona la frequenza di precessione di Larmor:

\[\Omega = \omega_{0}\]

Il segnale registrato \(s(t)\) dipende essenzialmente dalla variazione di fase \(\phi\left( \overset{\underline{}}{r},t \right)\), legata alla frequenza di precessione degli isocromati \(\omega\), dalla relazione:

\[\phi\left( \overset{\underline{}}{r},t \right) = - \int_{0}^{t}{\omega\left( \overset{\underline{}}{r},\tau \right)d\tau}\]

Dove \(\omega = \omega_{0}\) solamente nel caso in cui il campo magnetico principale sia uniforme in tutto lo spazio. Il segno meno nella relazione tra fase e velocità angolare è dovuta alla rotazione in senso orario degli isocromati.

\subsection{Densità protonica}\label{densituxe0-protonica}

Si suppone che il campo magnetico principale sia omogeneo, ovvero uguale a \(B_{0}\) sull'asse \(z\) in tutto lo spazio. Si ritiene, inoltre, che la finestra di acquisizione sia molto più piccola dei tempi di rilassamento longitudinale \(T_{1}\) e trasversale \(T_{2}\). Data che \(T_{1}\) è dell'ordine di \(1\ s\) e \(T_{2}\) di \(100\ ms\), la finestra di acquisizione deve avere un'ampizza di \(3 \div 4\ ms\). All'equilibrio la magnetizzazione può essere espressa in termini di densità protonica \(\rho\), definita come il numero di spin per unità di volume, mediante legge di Curie:

\[M_{0} \simeq \rho\dfrac{\gamma^{2}\hslash^{2}}{4k_{B}T}B_{0},\ \ \rho = \dfrac{N}{V}\]

Si applica un impulso a radiofrequenza all'istante \(t = 0\ s\), dopodiché il vettore di magnetizzazione è lasciato in evoluzione libera. Dopo il ribaltamento per l'effetto di un campo a radiofrequenza, la componente trasversale è:

\[M_{\bot}\left( \overset{\underline{}}{r},0 \right) = M_{0}\left( \overset{\underline{}}{r} \right) = \dfrac{1}{4}\rho\left( \overset{\underline{}}{r} \right)\dfrac{\gamma^{2}\hslash^{2}}{k_{B}T}B_{0}\]

Il segnale registrato e demodulato:

\[s(t) = \Lambda\omega_{0}B_{\bot}\int_{V}^{}{M_{\bot}\left( \overset{\underline{}}{r},0 \right)\exp\left( j\left( \Omega t + \phi\left( \overset{\underline{}}{r},t \right) \right) \right)dV}\]

può essere scritto come:

\[s(t) = \Lambda\omega_{0}B_{\bot}\int_{V}^{}{\dfrac{1}{4}\rho\left( \overset{\underline{}}{r} \right)\dfrac{\gamma^{2}\hslash^{2}}{k_{B}T}B_{0}\exp\left( j\left( \Omega t + \phi\left( \overset{\underline{}}{r},t \right) \right) \right)dV}\]

Si definisce densità protonica efficace \(\widehat{\rho}\left( \overset{\underline{}}{r} \right)\) come:

\[\widehat{\rho}\left( \overset{\underline{}}{r} \right) = \dfrac{1}{4}\Lambda\omega_{0}B_{\bot}\dfrac{\gamma^{2}\hslash^{2}}{k_{B}T}B_{0}\rho\left( \overset{\underline{}}{r} \right)\]

Questa relazione risulta valida anche nel caso in cui il campo irradiato dall'antenna ricevente, quando in essa scorre una corrente unitaria, non è omogeneo ma dipendente dalla posizione con un andamento noto:

\[B_{\bot} = B_{\bot}\left( \overset{\underline{}}{r} \right)\]

Con questa posizione il segnale demodulato può essere scritto come:

\[s(t) = \int_{V}^{}{\widehat{\rho}\left( \overset{\underline{}}{r} \right)\exp\left( j\left( \Omega t + \phi\left( \overset{\underline{}}{r},t \right) \right) \right)dV}\]

La densità di spin efficace è legata, in modo proporzionale, alla densità protonica \(\rho\left( \overset{\underline{}}{r} \right)\) del materiale sotto analisi, tramite delle costanti note o applicate dall'esterno. In particolare, la proporzionalità interessa costanti quali la temperatura, la frequenza di Larmor e il campo principale. Tutte queste grandezze sono regolabili dall'esterno.

Nel caso in cui gli effetti del rilassamento non siano trascurabili, ovvero la finestra di acquisizione ha una durata paragonabile con i tempi \(T_{1}\) e \(T_{2}\), la densità protonica efficace, oltre a dipendere dalla posizione, dipende anche dal rilassamento:

\[\widehat{\rho} = \widehat{\rho}\left( \overset{\underline{}}{r},\ T_{1}{,T}_{2} \right)\]

Per eseguire l'imaging è necessario valutare la densità protonica, con tecniche opportune.

\subsection{Applicazione dei gradienti}\label{applicazione-dei-gradienti}

È possibile variare la frequenza di precessione di Larmor degli isocromati, in base alla loro posizione lungo l'asse \(z\), variando il campo magnetico principale lungo questa direzione. Si introducono delle disomogeneità di campo, di solito molto minore dell'ampiezza del campo principale. Nelle applicazioni pratiche si introducono delle disomogeneità, in modo da far variare il campo principale linearmente lungo l'asse \(z\):

\[B_{z}(z,t) = B_{0} + G_{z}(t)z\]

Dove \(G_{z}\) è detto gradiente del campo magnetico principale ed è dato dalla relazione:

\[G_{z}(t) = \dfrac{\partial B_{z}}{\partial z}\]

La scelta di un gradiente del campo principale linearmente variabile con la posizione \(z\) permette di semplificare la logica di controllo e la trattazione analitica del comportamento degli spin.

Il gradiente di campo \(G_{z}\) può dipendere dal tempo, poiché si potrebbe voler cambiare la sua ampiezza nel corso dell'esperimento al fine di ottenere le informazioni necessarie.

L'applicazione del gradiente determina una variazione della frequenza di precessione dei vari spin lungo la coordinata \(z\). Nel sistema di riferimento fisso del laboratorio, la frequenza di precessione è:

\[\omega(z,t) = \gamma B_{z}(z,t) = \gamma B_{0} + \gamma G_{z}(t)z\]

A seguito della demodulazione, la componente a frequenza \(\gamma B_{0}\) è rimossa. In altre parole, nel sistema di riferimento rotante la frequenza di precessione dei vari isocromati è:

\[\omega(z,t) = \gamma G_{z}(t)z\]

L'uso di un gradiente di campo permette di stabile una relazione biunivoca tra la posizione \(z\) degli spin e la loro frequenza di precessione. La \(\omega(z,t) = \gamma G_{z}(t)z\) è detta codifica di frequenza o \emph{frequency encoding}.

Lo scopo principale della risonanza magnetica per l'imaging del copro umano è quello di estrarre informazioni sulla distribuzione dei protoni degli atomi di idrogeno, di un volumetto elementare. A tale scopo, si rende molto importante la codifica di frequenza.

Dal punto di vista operativo, i costruttori garantiscono un gradiente di campo lineare in una sfera con diametro di \(50\ cm\) circa e centrato al centro del gantry. Tale diametro è detto \emph{Diameter of Spherical Volume} o DSV. Al di fuori della sfera, il campo magnetico si discosta dal valore atteso di una ppm rispetto al gradiente applicato.

\begin{figure}
\centering
\includegraphics[width=3.15in,height=1.95384in]{media/8_Grad/image216.pdf}\caption{Figura .: DSV}
\end{figure}

Per ottenere immagini ben definite, la porzione di corpo che si vuole visualizzare deve essere posizionata al centro della DSV, in cui sia il campo magnetico principale sia i gradienti sono costanti e regolari nello spazio.

Se la frequenza di precessione varia lungo \(z\), allora anche la fase degli isocromati risulta essere una funzione di questa coordinata. È noto, infatti, che la variazione di fase, in un intervallo temporale \(\lbrack 0;t\rbrack\) è:

\[\phi(z,t) = - \int_{0}^{t}{\omega(z,\tau)d\tau}\]

Nel sistema di riferimento rotante, quindi a valle della demodulazione, la frequenza di Larmor è \(\omega(z,t) = \gamma G_{z}(t)z\), per cui:

\[\phi(z,t) = - \int_{0}^{t}{\omega(z,\tau)d\tau} = - \int_{0}^{t}{\gamma G_{z}(t)zd\tau}\]

Le quantità \(\gamma\) e \(z\) non dipendono dalla variabile temporale, quindi, possono essere portate fuori dal simbolo di integrale:

\[\phi(z,t) = - z\gamma\int_{0}^{t}{G_{z}(t)d\tau}\]

Si scrive questa relazione in termini di \(\overline{\gamma} = \gamma\backslash 2\pi \Leftrightarrow \gamma = 2\pi\overline{\gamma}\):

\[\phi(z,t) = - 2\pi\overline{\gamma}z\int_{0}^{t}{G_{z}(t)d\tau}\]

Si definisce la variabile \(k\) o frequenza spaziale, come:

\[k(t) = \overline{\gamma}\int_{0}^{t}{G_{z}(t)d\tau}\]

Questa variabile definisce un dominio definito \(k\)-spazio. Se il gradiente \(G_{z}\) è costante nel tempo, la variabile \(k\) del \(k\)-spazio può essere scritta come:

\[k(t) = \overline{\gamma}G_{z}t\]

Esiste, in definitiva, una corrispondenza tra la variabile \(k\) e il tempo di tipo lineare.

Mediante l'introduzione della variabile \(k\), la fase può essere scritta come:

\[\phi(z,t) = - 2\pi k(t)z\]

In gergo tecnico, si dice che questa relazione descrive la fase nel \(k\)-spazio.

Il segnale \(s(t)\), registrato dall'antenna, a seguito della demodulazione può essere espresso come funzione delle variabili \(z\) e \(t\) come:

\[s(t) = \int_{}^{}{\widehat{\rho}(z)\exp\left\lbrack j\phi(z,t) \right\rbrack dz}\]

In termini della variabile \(k\), il segnale registrato e demodulato può essere riscritto come segue:

\[s(k) = \int_{}^{}{\widehat{\rho}(z)\exp( - j2\pi kz)dz}\]

Questa espressione mostra che, quando si applica un gradiente di campo lineare, il segnale \(s(k)\) rappresenta la trasformata di Fourier della densità protonica efficace del volumetto elementare. Viceversa, la funzione \(\widehat{\rho}(z)\) è l'anti-trasformata di Fourier del segnale \(s(k)\) registrato dalle antenne e demodulato:

\[\widehat{\rho}(z) = \int_{}^{}{s(k)\exp(j2\pi kz)dk}\]

Si usa dire che la densità protonica è la codifica di Fourier o \emph{Fourier encoding} lungo \(z\) di un gradiente lineare. In definitiva, il segnale \(s(k)\) e l'immagine \(\widehat{\rho}(z)\) sono una coppia di trasformate di Fourier, per cui, nota una trasformata è possibile ricavare l'altra.

I campi principali, generalmente usati nella pratica clinica, sono di \(1.5\ T\) o \(3\ T\), mentre valori tipici dei gradienti di campo sono \(10\ mT/m\), \(20\ mT/m\) o \(40\ mT/m\).

Si suppone che il diametro del volumetto sferico in cui il campo principale è uniforme sia di \(50\ cm\). Si vuole determinare la differenza tra la pulsazione di precessione di Larmor ai due estremi della DSV:

\[\mathrm{\Delta}\omega = \omega_{1}\left( z_{1} \right) - \omega_{2}\left( z_{2} \right)\]

Dove, nel sistema fisso del laboratorio, \(\omega_{1}\left( z_{1} \right) = \gamma B_{0} - \gamma G_{z}z_{1}\) e \(\omega_{2}\left( z_{2} \right) = \gamma B_{0} - \gamma G_{z}z_{2}\) per cui:

\[\mathrm{\Delta}\omega = \omega_{1}\left( z_{1} \right) - \omega_{2}\left( z_{2} \right) = \gamma B_{0} - \gamma G_{z}z_{1}\  - \gamma B_{0} + \gamma G_{z}z_{2} = \gamma G_{z}\left( z_{2} - z_{1} \right)\]

Dove \(z_{2} - z_{1}\) è il diametro \(d\) della sfera.

\begin{figure}
\centering
\includegraphics[width=2.70833in,height=1.46098in]{media/8_Grad/image217.pdf}\caption{Figura .: Andamento del gradiente nel DSV}
\end{figure}

Per cui:

\[\mathrm{\Delta}\omega = \gamma G_{z}d\]

Si scrive tale equazione in termini di frequenza, moltiplicando per \(2\pi\) ambo i membri, si ha:

\[\mathrm{\Delta}f = \overline{\gamma}G_{z}d\]

Per l'idrogeno \(\overline{\gamma} = 42.5\ MH_{z}/\), per cui la differenza di frequenze ai due estremi del DSV è:

\[\mathrm{\Delta}f = 42.5\ \dfrac{MHz}{T} \cdot 10\dfrac{mT}{m} \cdot 0.5\ m \simeq 200\ kHz\]

Le frequenze di risonanze degli isocromati sono centrate intorno alla frequenza princi pale \(\omega_{0} = \gamma B_{0}\) con una banda di circa \(200\ kHz\). Nel momento in cui i segnale viene de modulato che resta del contenuto frequenziale del segnale è il solo spettro di \(200\ kHz\) centrato in banda base.

\begin{figure}
\centering
\includegraphics[width=5.39823in,height=3.0463in]{media/8_Grad/image218.pdf}\caption{Figura .: Banda degli isocromati}
\end{figure}

Cambiando il gradiente applicato, l'ampiezza della banda, dalla relazione \(\mathrm{\Delta}f = \overline{\gamma}G_{z}d\), varia linearmente.

Si suppone di avere solamente due isocromati in posizione \(- z_{0}\) e \(z_{0}\), a cui corrisponde una certa densità protonica efficace, rispettivamente \({\widehat{\rho}}_{0}\) e \({\widehat{\rho}}_{1}\).

\begin{figure}
\centering
\includegraphics[width=3.925in,height=0.99815in]{media/8_Grad/image219.pdf}\caption{Figura .: Distribuzione di due isocromati in posizione speculare}
\end{figure}

l segnale trasmesso e demodulato è dato da:

\[s(t) = \int_{- \infty}^{\infty}{\widehat{\rho}(z)\exp\left\lbrack j\phi(z,t) \right\rbrack dz}\]

La fase, per l'applicazione del gradiente lineare, ha un andamento anch'essa lineare:

\[\phi\left( z_{0},t \right) = - \gamma G_{z}z_{0}t,\ \ \phi\left( z_{0},t \right) = \gamma G_{z}z_{0}t\]

Siccome gli isocromati si trovano su due punti ben definiti, la distribuzione di densità protonica efficace che lì descrive è data da una somma di impulsi di Dirac, centrati su \(z_{0}\) e \(- z_{0}\):

\[s(t) = \int_{- \infty}^{\infty}{\widehat{\rho}(z)\left\lbrack \delta\left( z - z_{0} \right) + \delta\left( z + z_{0} \right) \right\rbrack\exp\left\lbrack j\phi(z,t) \right\rbrack dz}\]

Integrando si ha:

\[s(t) = {\widehat{\rho}}_{0}\exp\left\lbrack j\phi\left( z_{0},t \right) \right\rbrack + {\widehat{\rho}}_{1}\exp\left\lbrack j\phi\left( z_{1},t \right) \right\rbrack\]

Supponendo che \({\widehat{\rho}}_{1} = {\widehat{\rho}}_{0}\), si ha:

\[s(t) = {\widehat{\rho}}_{0}\left\{ \exp\left\lbrack j\phi\left( z_{0},t \right) \right\rbrack + \exp\left\lbrack j\phi\left( z_{1},t \right) \right\rbrack \right\}\]

Sostituendo il valore delle fasi si ha:

\[s(t) = {\widehat{\rho}}_{0}\left\{ \exp\left\lbrack j\gamma G_{z}z_{0}t \right\rbrack + \exp\left\lbrack - j\gamma G_{z}z_{0}t \right\rbrack \right\}\]

Per le formule di Eulero è possibile scrivere:

\[s(t) = 2{\widehat{\rho}}_{0}\cos\left( \gamma G_{z}z_{0}t \right)\]

In termini del \(k\)-spazio, il segnale registrato è dato da:

\[s(k) = 2{\widehat{\rho}}_{0}\cos\left( 2\pi z_{0}k \right)\]

Campionando il segnale \(s(k)\) è possibile ricostruire la funzione densità protonica mediante la trasformata inversa di Fourier del segnale stesso. Nel caso in esame, infatti:

\[\widehat{\rho}(z) = \int_{- \infty}^{+ \infty}{s(k)\exp(j2\pi kz)dk} = \int_{- \infty}^{+ \infty}{2{\widehat{\rho}}_{0}\cos\left( 2\pi z_{0}k \right)\exp(j2\pi kz)dk} = {\widehat{\rho}}_{0}\left\lbrack \delta\left( z - z_{0} \right) + \delta\left( z + z_{0} \right) \right\rbrack\]

Mediante la anti-trasformata di Fourier, si ottengono i punti su cui sono centrati gli isocromati.

Dal punto di vista pratico e operativo, la trasformata di Fourier inversa non può essere eseguita per problemi legati alla memoria finita dei calcolatori utilizzati. Sul segnale acquisito nel \(k\)- spazio, non è possibile applicare un campionamento infinitesimo ma, bensì, finito. Inoltre, il segnale deve avere un supporto limitato, in quanto acquisito in una finestra di opportuna ampiezza. L'elaborazione numerica prevede di ricostruire la densità protonica approssimandola con la trasformata discreta di Fourier:

\[\widehat{\rho}(z) = \sum_{n}^{}{s\left( k_{n} \right)\exp\left( jn\pi fk_{n}t \right)}\]

Gli algoritmi di ricostruzione introducono degli errori nell'ottenere l'immagine, dovuti sia al campionamento sia all'interruzione del segnale, registrato in una finestra di acquisizione con ampiezza finita. Il campionamento nel tempo è completamente equivalente al campionamento nel \(k\)-spazio, perché le due entità sono correlate. Sia \(\mathrm{\Delta}t\) l'intervallo di campionamento temporale, l'intervallo di campionamento nel \(k\)-spazio è :

\[\mathrm{\Delta}k = \overline{\gamma}G_{z}\mathrm{\Delta}t\]

\subsection{Sequenza gradient-echo}\label{sequenza-gradient-echo}

Note le relazioni tra il tempo \(t\) e il \(k\)-spazio, ci si chiede come debba essere effettuato il campionamento nel \(k\)-spazio, operazione che in gergo viene detta tracciamento delle traiettorie in questo spazio. Si suppone di avere un campione di materiale omogeneo, contenente un numero elevato di spin.

La posizione degli isocromati può essere decodificata mediante l'applicazione di un gradiente di campo. Dal segnale registrato, poi, si campiona il \(k\)-spazio e, mediante una trasformata inversa di Fourier, si ricava l'immagine indicante la posizione degli spin per unità di volume nel campione.

\begin{figure}
\centering
\includegraphics[width=1.31559in,height=2.92708in]{media/8_Grad/image220.pdf}\caption{Figura .: Campione}
\end{figure}

Per ottenere un buon campionamento del \(k\)-spazio, una tecnica è offerta dalla sequenza gradient-echo, in cui è applicato un singolo impulso a \(\pi\backslash 2\) lungo un asse trasverso del sistema di riferimento, e un gradiente lungo l'asse \(z\) per un intervallo di tempo prefissato.

L'impulso a \(\pi/2\) porta la magnetizzazione sul piano trasverso. Nel sistema di riferimento fisso del laboratorio, il recupero della magnetizzazione è visto come una sequenza FID; tuttavia, l'applicazione del gradiente \(G_{z}\) determina un rifasamento più rapido, legato alle diverse distribuzioni degli isocromati nello spazio. Si suppone che l'andamento del gradiente sia di tipo:

\[G_{z}(t) = \left\{ \begin{matrix}
 - G_{z} & t_{1} < t < t_{2} \\
0 & t < t_{1},t > t_{2}
\end{matrix} \right.\ \]

\begin{figure}
\centering
\includegraphics[width=5.08584in,height=2.34649in]{media/8_Grad/image221.pdf}\caption{Figura .: Applicazione del gradiente lungo \(z\) e impulso RF}
\end{figure}

Il gradiente ha polarità negativa ed è applicato nell'intervallo di tempo \(\left\lbrack t_{1},t_{2} \right\rbrack\). Nel dominio del \(k\)-spazio, il gradiente costante si traduce in un andamento lineare della variabile \(k\):

\[k = \overline{\gamma}\int_{t_{1}}^{t_{2}}{G_{z}(\tau)d\tau} = - \overline{\gamma}G_{z}\left( t_{2} - t_{1} \right)\]

Dove il tempo \(t_{0} = 0\) è posizionato al centro del primo impulso a radiofrequenza.

La variabile \(k\) evolve, durante l'applicazione del gradiente con legge lineare e pendenza \(- \overset{\underline{}}{\gamma}G_{z}\). Esaurito l'impulso la variabile \(k\) resta costante.

\begin{figure}
\centering
\includegraphics[width=4.95563in,height=3.325in]{media/8_Grad/image222.pdf}\caption{Figura .: Andamento di \(k\) nella sequenza gradient-echo}
\end{figure}

È noto che la variabile \(k\) è legata alla fase dalla relazione:

\[\phi = - 2\pi kz = 2\pi\overset{\underline{}}{\gamma}G_{z}\left( t_{2} - t_{1} \right)z\]

La fase dipende anche dalla variabile \(z\), quindi, in un diagramma della fase in funzione del tempo, la pendenza della fase da \(z\), infatti, il coefficiente angolare è:

\[m = 2\pi\overline{\gamma}G_{z}z\]

Al variare di \(z\) si ottiene un coefficiente angolare diverso per la fase.

\begin{figure}
\centering
\includegraphics[width=5.10264in,height=4.25417in]{media/8_Grad/image223.pdf}\caption{Figura .: Andamento della fase nella sequenza gradient-echo}
\end{figure}

La pendenza della fase, quindi, dipende dalla posizione degli isocromati. Questo comportamento è in contrasto con quello di \(k\), uguale per tutti i punti dello spazio poiché non dipende dalla posizione. In altre parole, \(k\) è unico per tutti gli isocromati del volume.

L'applicazione di un solo gradient-echo mediante l'applicazione di un solo gradiente non è ottima per riempire il \(k\)-spazio, poiché permette di ottenere valori negativi di questa variabile, rendendo complicata la ricostruzione dell'immagine se non impossibile.

Si costruisce una sequenza gradient-eco in cui si applicano due gradienti di polarità opposta, in modo che l'area sottesa dal secondo impulso sia maggiore (o doppia) di quella sottesa dal primo. Il tempo necessario affinché il secondo impulso sottenda la stessa area del primo corrisponde al tempo di echo, \(T_{E}\), poiché il comportamento della magnetizzazione è simile alla sequenza spin-echo. Il segnale indotto sulle bobine è raccolto durante il secondo gradiente.

\begin{figure}
\centering
\includegraphics[width=6.42639in,height=2.4385in]{media/8_Grad/image224.pdf}\caption{Figura .: Sequenza gradient-echo con due gradienti di polarità opposta}
\end{figure}

Si vuole analizzare il comportamento della fase nei vari instanti di tempo della sequenza. Il gradiente può essere scritto come:

\[G_{z}(t) = \left\{ \begin{matrix}
 - G_{z}, & t_{1} < t < t_{2} \\
G_{z}, & t_{3} < t < t_{4} \\
0, & altrove
\end{matrix} \right.\ \]

Il comportamento nel \(k\)-spazio è ottenuto integrando il gradiente nel tempo, da cui:

\[k(t) = \left\{ \begin{matrix}
0, & t < t_{1} \\
 - \overline{\gamma}G_{z}\left( t - t_{1} \right), & t_{1} < t < t_{2} \\
 - \overline{\gamma}G_{z}\left( t_{2} - t_{1} \right), & t_{2} < t <_{t3} \\
\overline{\gamma}G_{z}\left( t - t_{3} \right) - \overline{\gamma}G_{z}\left( t_{2} - t_{1} \right), & t_{3} < t < t_{4} \\
\overline{\gamma}G_{z}\left( t_{4} - t_{3} \right) - \overline{\gamma}G_{z}\left( t_{2} - t_{1} \right), & t > t_{4}
\end{matrix} \right.\ \]

\begin{figure}
\centering
\includegraphics[width=6.4in,height=2.98818in]{media/8_Grad/image225.pdf}\caption{Figura .: Sequenza gradient-echo con due gradienti di polarità opposta}
\end{figure}

In definitiva, la variabile \(k\) nel tempo presenta un andamento lineare a tratti, in cui, durante il primo gradiente ha una pendenza negativa; nell'intervallo di tempo tra il primo e il secondo gradiente resta costante per poi procedere con pendenza positiva durante l'applicazione del gradiente con polarità opposta al primo. Il tempo in cui la variabile \(k\) si annulla è dato da:

\[\overline{\gamma}G_{z}\left( t - t_{3} \right) - \overline{\gamma}G_{z}\left( t_{2} - t_{1} \right) = 0 \Leftrightarrow t = t_{3} + t_{2} - t_{1}\]

In questo istante temporale, l'area del secondo gradiente è uguale a quella sottesa dal primo; ovvero l'istante \(t\) appena determinato coincide con la definizione del tempo di echo, per cui:

\[T_{E} = t_{3} + t_{2} - t_{1}\]

Nella finestra di acquisizione nell'intervallo \(\left\lbrack t_{3};t_{4} \right\rbrack\) la variabile \(k\) assume valori sia positivi che negativi, quindi, è possibile tracciare una traiettoria nel \(k\)-spazio che permetta la ricostruzione dell'immagine.

La fase degli isocromati, invece, ha un andamento opposto a quello della variabile \(k\), con pendenza legata alla posizione \(z\):

\[\phi(t) = - 2\pi k(t)z\]

In particolare, la pendenza della fase \(\phi\) aumenta con la posizione \(z\). Nel momento in cui si applica il primo gradiente la fase cresce con andamento lineare. Dopodiché resta costante all'esaurirsi del primo gradiente. Non appena è applicato il secondo gradiente, la fase decresce con pendenza lineare e, per come sono costruiti i gradienti, si annulla al tempo di echo \(T_{E}\), indipendentemente dalla posizione \(z\) degli isocromati.

\begin{figure}
\centering
\includegraphics[width=6.375in,height=4.42482in]{media/8_Grad/image226.pdf}\caption{Figura .: Andamento della fase nella sequenza gradient-echo con due gradienti di polarità opposta}
\end{figure}

In questo caso si ha un fenomeno del tutto simile allo spin-echo, in cui al tempo di echo si recupera la fase mediante il rifasamento dei vari isocromati. Nella sequenza spin-echo, infatti, si è dimostrato che la fase, appena dopo l'impulso RF a \(\pi/2\), decresce con pendenza costante (\textbf{Errore. L'origine riferimento non è stata trovata.}). In seguito all'impulso a \(\pi\), la fase è ribaltata, mantenendo invariata la pendenza. La fase si annulla al tempo d'echo, in quanto si verifica il fenomeno della rifocalizzazione lungo un asse del sistema rotante.

Il comportamento della gradient-echo è del tutto analogo alla spin-echo: in entrambe la fase si annulla al tempo d'echo a cui corrisponde il massimo del segnale percepito.

L'echo è generato dai gradienti, i quali possono essere inseriti secondo un ordine arbitrario, nel senso che il risultato non cambia se si applica prima l'impulso positivo e poi quello negativo, a patto che le aree degli impulsi mantengano i rapporti di almeno \(1:2\). Il primo gradiente applicato è detto gradiente di defasamento o defase, mentre il secondo è detto di refocalizzazione, rifasamento o refase.

Gli isocromati presentano una velocità di precessione dipendente dalla posizione nel sistema rotante. Per l'applicazione del gradiente di defasamento, gli spin si sfasano, mentre con l'applicazione del gradiente di rifocalizzazione si rifasano al tempo di echo \(T_{E}\).

Per rendere la notazione grafica più sempre il \(k\)-spazio non è descritto mediante l'andamento temporale della variabile \(k\), ma si considera un solo asse e la componente del \(k\)-spazio legato a ad esso mediante il gradiente. Dal punto \(z = 0\), l'andamento lineare con pendenza negativa della fase è rappresentato mediante una freccia che si muove verso sinistra. In gergo tecnico si dice che la fase è rappresentata mediante un vettore che si sposta verso sinistra, se il gradiente è negativo, verso sinistra se il gradiente è positivo. Di conseguenza, la fase crescente è descritta da un vettore con direzione opposta al primo, applicato nel punto in cui è arrivata la punta, ovvero \(- k_{\min}\). Il vettore da sinistra a destra raggiunge il valore massimo \(k_{\max}\).

\begin{figure}
\centering
\includegraphics[width=5.53069in,height=1.58333in]{media/8_Grad/image227.pdf}\caption{Figura .: Traiettoria nel \(k\)-spazio}
\end{figure}

Nel dominio del tempo, il segnale registrato è dato dalla defocalizzazione e rifocalizzazione degli isocromati, quest'ultimo punto il cui raggiunge il massimo valore al tempo di echo. L'inviluppo del segnale registrato decade come \(T_{2}^{*}\).

\begin{figure}
\centering
\includegraphics[width=5.70833in,height=3.95417in]{media/8_Grad/image228.pdf}\caption{Figura .:Segnale prodotto durante una sequenza gradient-echo con due impulsi di polarità opposta}
\end{figure}

\begin{figure}
\centering
\includegraphics[width=4.25585in,height=5.18138in,alt={Immagine che contiene diagramma, schizzo, disegno, Disegno tecnico Il contenuto generato dall\textquotesingle IA potrebbe non essere corretto.}]{media/8_Grad/image229.pdf}\caption{Figura .: Sequenza gradient-echo nelle varie fasi}
\end{figure}

\subsection{Gradient-echo e spin-echo}\label{gradient-echo-e-spin-echo}

La sequenza gradient-echo con i due impulsi di polarità opposta non è ancora quella effettivamente utilizzata nella pratica; infatti, la procedura applicata permette di focalizzare gli spin che procedono a frequenza di Larmour diverse, quindi gli sfasamenti, legati al campo gradiente applicato \(G_{z}z\) imposta dall'esterno. In altre parole, la refocalizzazione al tempo di echo è dovuta solamente agli sfasamenti indotti dal gradiente di campo magnetico.

Nella pratica, il campo magnetico principale \(B_{0}\), in assenza di gradiente, presente delle disomogeneità, indicate con \(\Delta B\). Nel momento in cui si applicano i due gradienti di polarità inversa, le differenze di frequenze legate alla disomogeneità di campo principale non sono rifocalizzate dalla sequenza gradient-echo.

Per rifocalizzare anche le disomogeneità di campo legati al tempo \(T_{2}^{*}\), si applica una sequenza ottenuta da una combinazione della sequenza spin-echo con la gradient-echo. Nel dettaglio, si applica un impulso a \(\pi/2\), seguito da un secondo impulso a \(\pi\) lungo uno degli assi \(x'\) o \(y'\). Tra i due impulsi si applica un gradiente lungo \(z\) di una certa polarità. A seguito dell'impulso a \(\pi\), si applica il secondo gradiente lungo \(z\) con stessa polarità del primo e area almeno doppia. Il segnale viene registrato durante l'applicazione del secondo gradiente.

\begin{figure}
\centering
\includegraphics[width=5.89167in,height=2.45875in]{media/8_Grad/image230.pdf}\caption{Figura .: Sequenza gradient-echo con spin-echo}
\end{figure}

Si vuole determinare la fase degli isocromati applicando questa nuova sequenza. Nel dettaglio, ogni isocromato avverte due disomogeneità di campo: una legata all'applicazione del gradiente \(G_{z}(t)\), e l'altra intrinseca del campo magnetico principale, il quale non è perfettamente omogeneo nella regione di interesse.

Si considera una certa coordinata \(z\). La fase degli isocromati è legata sia ai gradienti applicati sia dagli impulsi a radiofrequenza. È possibile applicare la sovrapposizione degli effetti per studiare il comportamento quando solamente una delle due cause, \(G_{z}(t)\) o \(\Delta B\), è applicata.

A causa dell'applicazione del gradiente \(G_{z}(t)\), la fase varia secondo una legge lineare con pendenza negativa. Dal tempo \(t_{2}\) in cui il gradiente si annulla, la fase resta costante, finché non è applicato l'impulso a \(\pi\) che la ribalta.

L'applicazione del secondo produce un rifasamento e defasamento, che annulla la fase, \(\phi(t)\), al tempo di echo, \(T_{E}\). Durante l'applicazione del secondo gradiente la pendenza con cui decresce la fase è negativa ed è uguale al tratto precedente.

\begin{figure}
\centering
\includegraphics[width=6.65998in,height=3.61319in]{media/8_Grad/image231.pdf}\caption{Figura .: Andamento della fase dovuto gli impulsi a RF e ai gradienti nella sequenza gradient-echo con spin-echo}
\end{figure}

La fase degli isocromati, in contemporanea, è soggetta alla disomogeneità del campo principale; di conseguenza, lo sfasamento degli isocromati non inizia col gradiente \(G_{z}\) ma dall'applicazione del primo impulso a \(\pi/2\) lungo uno degli assi del sistema rotante, al tempo \(t_{0} = 0\). Dopo il primo impulso a radiofrequenza, la fase evolve linearmente con pendenza negativa fino all'applicazione dell'impulso a radiofrequenza che ribalta la magnetizzazione di \(\pi\). Questo impulso ribalta la fase in modo speculare rispetto all'istante precedente la sua applicazione. In seguito, la fase procede con la stessa pendenza fino ad annullarsi al tempo di echo, diventando successivamente negativa. La pendenza uguale della fase è legata al fatto che gli impulsi a radiofrequenza non modificano le disomogeneità di campo principale \(\Delta B\).

Sia la fase legata alla disomogeneità di campo principale \(B_{0}\), sia legata al gradiente lungo \(z\) si rifasano al tempo di echo, \(T_{E}\). Ovviamente, per il gradiente lungo \(z\), l'area del secondo impulso dal tempo \(t_{3}\) al tempo di echo \(T_{E}\) deve essere la stessa del primo gradiente applicato, così da garantire, appunto, l'annullamento della fase degli isocromati.

La nuova sequenza nel \(k\)-spazio è rappresentata tenendo conto che la fase subisce un salto a causa dell'impulso a \(\pi\):

\[\phi(z,t) = - \overline{\gamma}\int_{0}^{t}{G_{z}d\tau} = 2\pi kz\]

Se la fase è ribaltata anche la variabile \(k\) è ribaltata. Dal punto di vista grafico, la situazione si rappresenta come:

\begin{enumerate}
\def\labelenumi{\arabic{enumi})}
\item
  Un vettore che porta \(k\) verso destra, fino ad arrivare a \(k_{\max}\);
\item
  A seguito del ribaltamento, il \(k\) si sposta da \(k_{\max}\) a \(- k_{\max}\) senza compiere un'effettiva traiettoria nel \(k\)-spazio. La situazione è descritta graficamente mediante una curva che conuce la variabile \(k\) dal valore che assumeva prima del ribaltamento al suo valore speculare, in modo istantaneo;
\item
  Una volta raggiunto il valore \(- k_{\max}\), il secondo gradiente di rifocalizzazione, dopo l'impulso a radiofrequenza, porta la frequenza spaziale \(k\) a crescere con andamento lineare, rappresentato come un vettore da \(- k_{\max}\) a \(k_{\max}\).
\end{enumerate}

\begin{figure}
\centering
\includegraphics[width=5.82917in,height=2.7516in]{media/8_Grad/image232.pdf}\caption{Figura .: Andamento nel \(k\)-spazio con sequenza gradient-echo con spin-echo}
\end{figure}

La finestra di acquisizione è attivata durante l'applicazione del secondo gradiente, quindi, il segnale \(s(t)\) è registrato dalle antenne e campionato nel tempo, per poter essere elaborato da una circuiteria digitale.

Il campionamento nel dominio del tempo può essere legato al campionamento nel dominio della \(k\). Se, infatti \(\Delta t\) è l'intervallo di tempo campionamento nel dominio del tempo, allora, in ipotesi di gradiente costante, il campionamento lungo \(k\) è dato da:

\[\Delta k = \overline{\gamma}G_{z}\Delta t\]

Da cui si passa da \(s\left( t_{n} \right)\) a \(s\left( k_{n} \right)\).

L'applicazione della sequenza spin-echo mista alla gradient-echo permette di campionare il \(k\)-spazio per tutti i valori di \(k\) contenuti nell'intervallo \(\left\lbrack - k_{\max};k_{\max} \right\rbrack\), nella direzione delle \(k\) crescenti, ovvero da \(k\) negative a positive.

\begin{figure}
\centering
\includegraphics[width=4.98687in,height=1.10833in]{media/8_Grad/image233.pdf}\caption{Figura .: Campionamento nel \(k\)-spazio}
\end{figure}

Campionando il segnale \(s(k)\) nel \(k\)-spazio, durante la finestra in cui si forma l'echo è possibile ricostruire la densità protonica del campione mediante trasformata inversa di Fourier:

\[\widehat{\rho}(z) = \sum_{n}^{}{s\left( k_{n} \right)\exp\left( j2\pi nk_{n}z \right)}\]

Questa sequenza permette la sola ricostruzione di un'immagine della densità protonica efficace, legata alla densità protonica lungo la direzione \(z\).

\subsubsection{Segnale registrato della gradient-echo}\label{segnale-registrato-della-gradient-echo}

Si considera un campione di materiale omogeneo di forma cilindrica, disposto lungo la coordinata \(z\) e dimensione longitudinale \(2z_{0}\)

\begin{figure}
\centering
\includegraphics[width=4.93525in,height=3.45833in]{media/8_Grad/image234.pdf}\caption{Figura .: Cilindro di materiale omogeneo di lunghezza \(2z_{0}\)}
\end{figure}

Si applica una sequenza data da un impulso a \(\pi/2\) lungo \(x'\) e un gradiente lungo \(z\) con una durata di tempo \(T = t_{2} - t_{1}\). Si vuole determinare il segnale ricevuto dalle antenne durante l'applicazione del gradiente.

\begin{figure}
\centering
\includegraphics[width=5.6333in,height=2.60347in]{media/8_Grad/image235.pdf}\caption{Figura .: Applicazione di un impulso a RF e un gradiente lungo \(z\)}
\end{figure}

Il segnale nel \(k\)-spazio è ottenuto dalla trasformata di Fourier della densità protonica efficace \(\widehat{\rho}\):

\[s(k) = \int_{- z_{0}}^{z_{0}}{\widehat{\rho}(z)\exp( - j2\pi kz)dz}\]

La densità protonica efficace è costante, poiché il materiale è uniforme, quindi:

\[\widehat{\rho} = {\widehat{\rho}}_{0}\]

\({\widehat{\rho}}_{0}\) può essere portata fuori dal simbolo di integrale:

\[s(k) = {\widehat{\rho}}_{0}\int_{- z_{0}}^{z_{0}}{\exp( - j2\pi kz)dz}\]

Per risolvere l'integrale si moltiplica e divide per il fattore moltiplicativo \(z\) nell'argomento dell'esponenziale:

\[s(k) = {\widehat{\rho}}_{0}\int_{- z_{0}}^{z_{0}}{\exp( - j2\pi kz)dz} = - \dfrac{1}{j2\pi k}{\widehat{\rho}}_{0}\int_{- z_{0}}^{z_{0}}{\left\lbrack j2\pi kz\exp( - j2\pi kz) \right\rbrack dz} =\]

La cui soluzione è:

\[= - \dfrac{1}{j2\pi k}{\widehat{\rho}}_{0}\left. \ \exp( - j2\pi kz) \right|_{- z_{0}}^{z_{0}} = - \dfrac{1}{j2\pi k}{\widehat{\rho}}_{0}\left\lbrack \exp\left( - j2\pi kz_{0} \right) - \exp\left( j2\pi kz_{0} \right) \right\rbrack\]

Per le formule di Eulero, risulta che:

\[\exp\left( - j2\pi kz_{0} \right) - \exp\left( j2\pi kz_{0} \right) = - 2j\sin\left( 2\pi kz_{0} \right)\]

Il segnale registrato si scrive come:

\[s(k) = {\widehat{\rho}}_{0}\left( - \dfrac{1}{j2\pi k} \right)\left\lbrack - 2j\sin\left( 2\pi kz_{0} \right) \right\rbrack = {\widehat{\rho}}_{0}\dfrac{\sin\left( 2\pi kz_{0} \right)}{\pi k}\]

Per scrivere il segnale registrato nel \(k\)-spazio più compatto, si moltiplica e divide per \(2z_{0}\):

\[s(k) = {\widehat{\rho}}_{0}\dfrac{\sin\left( 2\pi kz_{0} \right)}{\pi k}\dfrac{2z_{0}}{2z_{0}} = 2z_{0}{\widehat{\rho}}_{0}\dfrac{\sin\left( 2\pi kz_{0} \right)}{2\pi kz_{0}}\]

Per la definizione di \(sinc\), si ha:

\[s(k) = 2z_{0}{\widehat{\rho}}_{0}{sinc}\left( kz_{0} \right)\]

A una distribuzione uniforme lungo \(z\), ovvero rettangolare, corrisponde un segnale ricevuto \(s(k)\) di tipo \(sinc\) e ampiezza \(2{\widehat{\rho}}_{0}z_{0}\).

Nel dominio del tempo, il segnale registrato è ottenuto ricorrendo al legame tra \(k\) e \(t\):

\[k = - \overline{\gamma}G_{z}\left( t - t_{1} \right)\]

Dato che la \(sinc\) è una funzione pari, il segnale registrato nel dominio del tempo può essere scritto come:

\[s(t) = 2z_{0}{\widehat{\rho}}_{0}{sinc}\left\lbrack \overline{\gamma}G_{z}\left( t - t_{1} \right)z_{0} \right\rbrack\]

Il segnale appena determinato è un decadimento di tipo FID.

Dal punto di vista della finestra di acquisizione, il segnale registrato \(s(t)\), essendo centrato al tempo \(t_{1}\), è campionata soltanto per metà onda. Si ha, dunque, una perdita di informazioni in quanto solo metà del segnale emanato viene memorizzato; equivalentemente, la variabile \(k\) assume solamente valori positivi.

L'uso del gradiente di defasamento congiunto con quello di refasamento permette di ottenere l'intero segnale \(sinc\), ottenendo così maggiori informazioni sul materiale considerato. Equivalentemente, la variabile \(k\) assume valori sia positivi che negativi.

La registrazione di una \(sinc\) troncata non permette la ricostruzione ottimale della densità protonica, poiché manchevole di importanti informazioni; viceversa, registrando il segnale durante il gradiente di rifasamento si ottiene l'intera \(sinc\), ricostruendo l'informazione in modo ottimale, poiché il contenuto informativo è massimo.

Il segnale registrato nel secondo gradiente è ottenuto dalla trasformata di Fourier della densità protonica:

\[s(k) = \int_{- z_{0}}^{z_{0}}{\widehat{\rho}(z)\exp( - j2\pi kz)dz}\]

In ipotesi di mezzo omogeneo, \(\widehat{\rho}(z) = {\widehat{\rho}}_{0}\), per cui:

\[s(k) = \int_{- z_{0}}^{z_{0}}{\widehat{\rho}(z)\exp( - j2\pi kz)dz} = {\widehat{\rho}}_{0}\int_{- z_{0}}^{z_{0}}{\exp( - j2\pi kz)dz}\]

La cui soluzione, come nel caso precedente, è data da:

\[s(k) = 2z_{0}{\widehat{\rho}}_{0}{sinc}\left( kz_{0} \right)\]

In questo caso, la variabile \(k\) è data da:

\[k(t) = \overline{\gamma}G_{z}\left( t - t_{3} \right) - \overline{\gamma}G_{z}\left( t_{2} - t_{1} \right) = \overline{\gamma}G_{z}\left( t - t_{3} - t_{2} + t_{1} \right)\]

Per cui, il segnale registrato nel dominio del tempo è:

\[s(t) = 2z_{0}{\widehat{\rho}}_{0}{sinc}\left\lbrack \overline{\gamma}G_{z}\left( t - t_{3} - t_{2} + t_{1} \right)z_{0} \right\rbrack\]

In questo caso, la \(sinc\) è centrata al tempo di echo:

\[T_{E} = t_{3} + t_{2} - t_{1}\]

È possibile scrivere il segnale registrato come:

\[s(t) = 2z_{0}{\widehat{\rho}}_{0}{sinc}\left\lbrack \overline{\gamma}G_{z}\left( t - T_{E} \right)z_{0} \right\rbrack\]

In questo caso, il contenuto informativo è massimo.

\begin{figure}
\centering
\includegraphics[width=6.90754in,height=4.82847in]{media/8_Grad/image236.pdf}\caption{Figura .: Segnale registrato per sequenza gradient-echo con mezzo omogeneo}
\end{figure}

Si osservi che la finestra di registrazione del segnale deve essere tale da poter trascurare gli effetti legati ai tempi di rilassamento \(T_{1}\) e \(T_{2}\). La finestra di acquisizione, per tale ragione, deve essere dell'ordine dei \(ms\).

\begin{center}
\vfill
    \chapter{Ricostruzione bidimensionale in MRI}
    \label{blx:Ric2D\therefsection}
\vfill

\minitoc
\newpage
\end{center}
\justify


\section{Ricostruzione tridimensionale in RMI}\label{ricostruzione-tridimensionale-in-rmi}

L'imagining monodimensionale permette la ricostruzione della densità protonica solamente lunga una direzione, decodificando la posizione con la frequenza di precessione degli isocromati, a opera del gradiente.

Per ottenere una ricostruzione tridimensionale della densità protonica è possibile applicare più gradienti ortogonali tra loro. Infatti, se un gradiente lungo \(z\) permette di campionare il \(k\)-spazio lungo la direzione \(k_{z} = \overline{\gamma}G_{z}z\), l'applicazione di altri due gradienti lungo le altre due dimensioni \(x\) e \(y\) consente di campionare lo spazio \(k\) anche nelle direzioni \(k_{x}\) e \(k_{y}\).

Dal punto di vista analitico, la variabile \(\overset{\underline{}}{k}\) può essere espressa come una terna del tipo:

\[\overset{\underline{}}{k} = \left( \begin{array}{r}
k_{x} \\
k_{y} \\
k_{z}
\end{array} \right)\]

Il segnale registrato durante la finestra di acquisizione è ottenuto come trasformata tridimensionale della densità protonica efficace:

\[s\left( \overset{\underline{}}{k} \right) = \int_{V}^{}{\widehat{\rho}\left( \overset{\underline{}}{r} \right)\exp\left( - j2\pi\overset{\underline{}}{k} \cdot \overset{\underline{}}{r} \right)d^{(3)}\overset{\underline{}}{r}}\]

Dove \(\overset{\underline{}}{r}\) è il vettore posizione nello spazio:

\[\overset{\underline{}}{r} = \left( \begin{array}{r}
x \\
y \\
z
\end{array} \right)\]

Registrando il segnale in una finestra di acquisizione abbastanza breve da poter trascurare gli effetti dei tempi di rilassamento, è possibile ricostruire la densità protonica mediante trasformata inversa di Fourier:

\[\widehat{\rho}\left( \overset{\underline{}}{r} \right) = \int_{V}^{}{s\left( \overset{\underline{}}{k} \right)\exp\left( j2\pi\overset{\underline{}}{k} \cdot \overset{\underline{}}{r} \right)d^{(3)}\overset{\underline{}}{k}}\]

Dal punto di vista realizzativo, il processo di ricostruzione è ottenuti un impulso a radiofrequenza che ribalta la magnetizzazione di \(\pi/2\) su uno degli assi del sistema rotante e tre gradienti di campo principale, nelle tre direzioni dello spazio.

\[G_{z} = \frac{\partial}{\partial z}B_{z}\left( \overset{\underline{}}{r} \right),G_{x} = \frac{\partial}{\partial x}B_{z}\left( \overset{\underline{}}{r} \right),G_{y} = \frac{\partial}{\partial y}B_{z}\left( \overset{\underline{}}{r} \right)\]

In forma compatta è possibile scrivere:

\[\overset{\underline{}}{G} = \overset{\underline{}}{\nabla}B_{z}\]

Questa relazione è valida nell'ipotesi in cui la componente lungo \(z\) del campo magnetico principale varia in base alla posizione nello spazio, mentre le altre sono nulle:

\[{\overset{\underline{}}{B}}_{0} = \left( \begin{array}{r}
0 \\
0 \\
B_{z}(x,y,z)
\end{array} \right)\]

Ciò è dovuto al fatto che la frequenza di Larmour \(\omega_{0}\) è legata solamente alla coordinata \(z\).

\begin{figure}
\centering
\includegraphics[width=3.97247in,height=2.5in]{media/9_Ric2D/image237.pdf}\caption{Figura .: Campo vettoriale diretto lungo l'asse z, con intensità che varia linearmente con la posizione tridimensionale}
\end{figure}

La sequenza utilizzata è composta da tre gradienti lungo le tre dimensioni del sistema di riferimento fisso. I gradienti lungo \(z\) e \(y\) hanno durata, rispettivamente di \(\tau_{z}\) e \(t_{y}\) e possono anche essere sovrapposti tra loro. Il gradiente lungo \(x\) è detto di lettura e deve essere applicato separatamente. Durante l'applicazione di questo gradiente è attivata la finestra temporale per la registrazione del segnale nel \(k\)-spazio, \(s\left( \overset{\underline{}}{k} \right).\)

I gradienti lungo \(y\) e \(z\) sono detti gradienti di codifica di fase poiché consentono di alterare la fase in base alla posizione nello spazio degli isocromati.

\begin{figure}
\centering
\includegraphics[width=4.13333in,height=2.3864in]{media/9_Ric2D/image238.pdf}\caption{Figura .: Sequenza per l'imagining tridimensionale}
\end{figure}

Data la presenza di più gradienti, è possibile associare a ogni posizione \(\overset{\underline{}}{r}\) una certa frequenza lungo le tre dimensioni spaziali.

Il segnale acquisito nella finestra di acquisizione è del tipo:

\[s\left( \overset{\underline{}}{k} \right) = \int_{V}^{}{\widehat{\rho}\left( \overset{\underline{}}{r} \right)\exp\left\lbrack j\phi\left( \overset{\underline{}}{r} \right) \right\rbrack d^{(3)}\overset{\underline{}}{r}}\]

Dove la fase è legata alla variabile \(\overset{\underline{}}{k}\) nel \(k\)-spazio dalla relazione:

\[\phi\left( \overset{\underline{}}{r} \right) = 2\pi\overset{\underline{}}{k} \cdot \overset{\underline{}}{r}\ \]

La fase varia sia per l'applicazione dei gradienti lungo \(z\), per un tempo \(\tau_{z}\), e lungo \(y\), per un tempo \(\tau_{y}\), sia per l'applicazione del gradiente lungo \(x\). Le componenti del \(k\)-spazio sono definite come:

\[k_{i} = \overline{\gamma}\int_{t_{1}}^{t}{G_{i}(\tau)d\tau},i = x,y,z\]

Per gradienti costanti nel tempo, si ha:

\[\left\{ \begin{matrix}
k_{z} = \overline{\gamma}G_{z}\tau_{z} \\
k_{y} = \overline{\gamma}G_{y}\tau_{y} \\
k_{x} = \overline{\gamma}G_{x}\left( t - t_{1} \right)
\end{matrix} \right.\ \]

La fase può essere scritta come sovrapposizione delle varie fasi generate dai gradienti, principio noto come sovrapposizione delle fasi. A valle dalla demodulazione si ha:

\[\phi\left( \overset{\underline{}}{r},t \right) = 2\pi\overline{\gamma}G_{z}\tau_{z}z + 2\pi\overline{\gamma}G_{y}\tau_{y}y + 2\pi\overline{\gamma}G_{x}\left( t - t_{1} \right)x\]

Per definizione \(2\pi\overline{\gamma} = \gamma\), quindi:

\[\phi\left( \overset{\underline{}}{r},t \right) = \gamma G_{z}\tau_{z}z + \gamma G_{y}\tau_{y}y + \gamma G_{x}\left( t - t_{1} \right)x\]

La quantità \(G_{z}z\) rappresenta la disomogeneità di campo magnetico applicata lungo \(z\), ovvero la disomogeneità di campo vista dagli isocromati lungo \(z\). La quantità \(\gamma G_{z}z\) è la frequenza di precessione degli isocromati a valle della demodulazione. Infine, \(\gamma G_{z}\tau_{z}z\) fornisce la variazione di fase degli isocromati legati alla posizione \(z\).

Analogamente, la quantità \(\gamma G_{y}\tau_{y}y\) è la variazione di fase degli isocromati in base alla loro posizione lungo \(y\).

La fase legata alla precessione lungo \(x\) dipende esplicitamente dal tempo poiché la finestra di acquisizione è aperta solo su questo gradiente. In definitiva, all'applicazione del gradiente di lettura si è accumulata una fase su \(z\), data da \(\gamma G_{z}\tau_{z}z\), e su \(y\), data da \(\gamma G_{y}\tau_{y}y\); mentre con il gradiente di lettura si registra la variazione di fase o equivalentemente si campiona il \(k\)-spazio.

Il segnale registrato nella finestra di acquisizione è dato dalla trasformata tridimensionale della densità protonica:

\[s(t) = \iiint_{V}^{}{\widehat{\rho}(x,y,z)\exp\left( - j\gamma G_{z}\tau_{z}z - j\gamma G_{y}\tau_{y}y - j\gamma G_{x}tx \right)dxdydz}\]

Ponendo:

\[\left\{ \begin{matrix}
k_{y} = \overline{\gamma}\int_{0}^{t}{G_{y}(\tau)d\tau} \\
k_{z} = \overline{\gamma}\int_{0}^{t}{G_{z}(\tau)d\tau} \\
k_{x} = \overline{\gamma}\int_{0}^{t}{G_{x}(\tau)d\tau}
\end{matrix} \right.\  \Leftrightarrow \left\{ \begin{matrix}
k_{y} = \overline{\gamma}G_{y}\tau_{y} \\
k_{z} = \overline{\gamma}G_{z}\tau_{z} \\
k_{x} = \overline{\gamma}G_{x}t
\end{matrix} \right.\ \]

Il segnale registrato può essere espresso come:

\[s\left( \overset{\underline{}}{k} \right) = \iiint_{V}^{}{\widehat{\rho}(x,y,z)\exp\left\lbrack - j\left( 2\pi k_{z}z + 2\pi k_{y}y + 2\pi k_{x}x \right) \right\rbrack dxdydz}\]

Mentre \(k_{z}\) e \(k_{y}\) sono fissati, \(k_{x} = k_{x}(t)\) dipende dal tempo, per cui il passaggio tra la componente temporale col \(k\)-spazio avviene mediante \(k_{x}\).

Il segnale nel \(k\)-spazio, \(s\left( \overset{\underline{}}{k} \right)\) è una trasformata di Fourier tridimensionale della densità protonica efficace.

Il \(k\)-spazio, quindi, è una struttura tridimensionale, dove per ricostruire l'immagine \(k_{y}\) e \(k_{z}\) sono fissi mentre \(k_{x}\) è campionato nel \(k\)-spazio. In altre parole, la sequenza introdotta permette di campionare una retta nel \(k\)-spazio.

\begin{figure}
\centering
\includegraphics[width=2.69774in,height=2.6in]{media/9_Ric2D/image239.pdf}\caption{Figura .: Campionamento nel k-spazio della retta}
\end{figure}

\subsection[Utilizzo dei gradienti per campionare il k-spazio]{Utilizzo dei gradienti per campionare il $\mathbf{k}$-spazio}
\label{utilizzo-gradienti-campionamento-k-spazio}

Si pone, ora, il problema di come disporre i vari gradienti tra loro. Si suppone, a titolo d'esempio, di applicare il gradiente lungo \(x\) per un tempo predefinito \(\tau_{x}\), quello lungo \(y\) per un tempo \(\tau_{y}\) e quello lungo \(x\) per \(\tau_{x}\); inoltre, tutti i gradienti sono applicati contemporaneamente.

\begin{figure}
\centering
\includegraphics[width=4.00178in,height=2.48333in]{media/9_Ric2D/image240.pdf}\caption{Figura .: Gradienti lungo gli assi applicati contemporaneamente}
\end{figure}

In questo caso la relazione che lega la densità protonica efficace \(\widehat{\rho}\left( \overset{\underline{}}{r} \right)\) col segnale acquisito nel \(k\)-spazio è ottenuto mediante trasformata tridimensionale di Fourier:

\[s\left( \overset{\underline{}}{k} \right) = \iiint_{V}^{}{\widehat{\rho}(x,y,z)\exp\left\lbrack - j\left( 2\pi k_{z}z + 2\pi k_{y}y + 2\pi k_{x}x \right) \right\rbrack dxdydz}\]

Per ricostruire la densità protonica efficace \(\widehat{\rho}\left( \overset{\underline{}}{r} \right)\) è necessario campionare il \(k\)-spazio, acquisendo una matrice \(3 \times 3\), così che, mediante algoritmi digitali, è possibile ricavare la trasformata inversa di Fourier.

Aver scelto di scelto di sovrapporre gli impulsi, determina l'acquisizione di un unico punto nello spazio \(k\) di coordinate:

\[\overset{\underline{}}{k} = \left( \begin{array}{r}
\overline{\gamma}G_{x}\tau_{x} \\
\overline{\gamma}G_{y}\tau_{y} \\
\overline{\gamma}G_{z}\tau_{z}
\end{array} \right)\]

Avendo acquisito il segnale a fine dei tre gradienti, la fase degli isocromati è costante e, dunque, non si ottiene una matrice tridimensionale ma un singolo punto.

\begin{figure}
\centering
\includegraphics[width=1.88483in,height=1.63333in,alt={Generazione immagine completata}]{media/9_Ric2D/image241.pdf}\caption{Figura .: Campionamento nel k-spazio con gradienti applicati contemporaneamente}
\end{figure}

Per ottenere un buon campionamento nel \(k\)-spazio è necessario ripetere la tecnica di applicazione dei gradienti lungo le tre dimensioni spaziali, diverse volte, modificano la durata o l'ampiezza del gradiente in modo da ottenere il giusto numero di campioni per invertire la trasformata di Fourier. Tipicamente, si mantiene costante la durata temporale e si varia l'ampiezza del gradiente. Se, ad esempio, si mantiene costante l'ampiezza di \(G_{x}\) e \(G_{y}\) ma si varia \(G_{z}\), si ottiene un secondo punto con stesse coordinate \(k_{x}\) e \(k_{y}\) ma diverse \(k_{z}\) e così via.

Tra l'applicazione di una sequenza e l'altra è necessario che il vettore di magnetizzazione ritorni all'equilibrio, dopo essere stati ribaltato sul piano trasverso per l'applicazione dell'impulso a \(\pi/2\).

Affinché il vettore di magnetizzazione raggiunga l'equilibrio è necessario aspettare \(3 \div 5\ T_{1}\), tempo di rilassamento longitudinale, che governa l'evoluzione della componente lungo \(z\) del vettore di magnetizzazione nel tempo. Non aspettando un tempo almeno uguale a \(3T_{1}\) tra l'applicazione dei gradienti, allora si acquisirebbero dei punti con diverse ampiezze della magnetizzazione, portando ad artefatti nella ricostruzione dell'immagine. Infatti, la densità protonica sarebbe diversa tra punti di rette diverse nel \(k\)-spazio. Se, invece, il vettore di magnetizzazione raggiunge l'equilibrio la densità protonica è uguale per ogni punto.

Per ottenere un campionamento soddisfacente nel \(k\)-spazio tridimensionale è buona norma dividere il volume in voxel da \(256 \times 256 \times 80\) o \(256 \times 256 \times 100\). Ognuno di questi elementi di volume occupa una certa posizione nello spazio.

\begin{figure}
\centering
\includegraphics[width=1.64348in,height=1.65in]{media/9_Ric2D/image242.pdf}\caption{Figura .: Voxel elementare}
\end{figure}

La matrice per la ricostruzione dell'immagine ha una dimensione di \(2^{8} \times 2^{8} \times 100\); per cui il numero di punti nel \(k\)-spazio è dato all'incirca da:

\[2^{8} \times 2^{8} \times 100 = 2^{16} \times 100 \simeq 6.5 \cdot 10^{6}\]

Se, per acquisire due punti successivi è necessario aspettare almeno un tempo di \(3\ s\), per campionare completamente il volume è necessario un tempo apprensivamente di:

\[3\ s \cdot 6.5 \cdot 10^{6} \simeq 19 \cdot 10^{6}\ s \simeq 220\ day\]

All'atto pratico non è possibile ricostruire un intero volume del paziente sia perché il tempo di acquisizione è estremamente lungo, sia perché il paziente potrebbe muoversi causando artefatti nella ricostruzione. I movimenti del paziente, infatti, variano la distribuzione della densità protonica nello spazio e, quindi, i punti acquisiti per lo stesso volume potrebbero presentare una densità protonica diversa nel tempo. La ricostruzione porta a un errore di visualizzazione per la variazione nel tempo della funzione \(\widehat{\rho}\). In conclusione, questo approccio non può essere realizzato all'atto pratico.

\begin{figure}
\centering
\includegraphics[width=3.53728in,height=2.80208in]{media/9_Ric2D/image243.pdf}\caption{Figura .: Andamento della magnetizzazione nel tempo}
\end{figure}

La scelta di mantenere costante l'ampiezza costante del gradiente e variare il tempo di applicazione risulta ancora più impraticabile poiché dilatare sempre di più il tempo d'esame.

La soluzione adottata nella pratica sfrutta i gradienti di codifica di fase e il gradiente di lettura, applicati separatamente. Durante l'ultimo gradiente, si registra il segnale da campionare per riempire il \(k\)-spazio.

I gradienti di codifica di fase hanno una durata prefissate così da ottenere un campionamento nel \(k\)-spazio dipendente dall'intensità del gradiente. I passi di campionamento sono dati da:

\[\left\{ \begin{matrix}
\Delta k_{z} = \overline{\gamma}\Delta G_{z}\tau_{z} \\
\Delta k_{z} = \overline{\gamma}\Delta G_{z}\tau_{z}
\end{matrix} \right.\ \]

Il gradiente di lettura permette, invece, di acquisire il segnale \(s\left( \overset{\underline{}}{k} \right)\) nel tempo, ovvero il campionamento lungo \(k_{x}\) non avviene per variazione del gradiente ma mediante campionamento nel tempo del segnale registrato. L'intervallo di campionamento lungo \(k_{x}\) è legato al campionamento nel tempo \(\Delta t\):

\[\Delta k_{x} = \overline{\gamma}G_{x}\Delta t\]

Si considera una sequenza composta da due gradienti di codifica di fase \(G_{z}\) e \(G_{y}\) con durata, rispettivamente \(\tau_{z}\) e \(\tau_{y}\) applicati contemporaneamente. Ciò equivale a fissare le due coordinate nel \(k\)-spazio:

\[\left\{ \begin{matrix}
k_{z} = \overline{\gamma}G_{z}\tau_{z} \\
k_{y} = \overline{\gamma}G_{y}\tau_{y}
\end{matrix} \right.\ \]

\begin{figure}
\centering
\includegraphics[width=4.50081in,height=2.63333in]{media/9_Ric2D/image244.pdf}\caption{Figura .: Sequenza con gradienti di codifica di fase applicati contemporaneamente e di lettura separato}
\end{figure}

Iniziando la finestra di acquisizione duante il gradiente di lettura, applicato dopo la fine dei primi due gradienti, si ottiene una coordinata \(k_{x}\) variabile nel tempo secondo la relazione:

\[k_{x}(t) = \overline{\gamma}G_{z}t\]

Questo processo equivale a selezionare una riga, di coordinate \(\left( k_{x}(t),k_{y},k_{z} \right)\), nel \(k\)-spazio. È così possibile campionare con un passo \(\Delta t\), il segnale acquisito; il che equivale a campionare il \(k\)-spazio in una direzione, mentre le altre due sono fisse. In gergo, si parla di acquisire una riga del \(k\)-spazio con valori di \(k\) sia positiiv che nevegativi.

Variando l'ampiezza dei gradienti di codifica di fase in modo opportuno è possibile campionare un'ulteriore riga del \(k\)-spazio nella direzione delle \(k_{x}\).

\begin{figure}
\centering
\includegraphics[width=2.42441in,height=2.26667in,alt={Immagine che contiene linea, schizzo, diagramma, disegno Il contenuto generato dall\textquotesingle IA potrebbe non essere corretto.}]{media/9_Ric2D/image245.pdf}\caption{Figura .: Due righe del \(k\)-spazio}
\end{figure}

Per ottenere \(256\) campioni in questa direzione è necessario scegliere opportunemente il periodo di camponamento \(\Delta t\), così da acquisire il giusto numero di punti nella finestra di acquisizione. Al fine di ottenere \(256\) punti lungo l'asse \(z\) è necessario variare l'ampiezza del gradiente di altrettanti valori, così da fissare la coordinata \(k_{z}\). Infine, per i \(100\) campioni lungo la coordinata \(y\) di varia l'ampiezza del gradiente in tale direizone lo stesso numero di volte. Il piano \(y - z\) è, in definitiva, campionato puntualmente.

I \(256\) campioni lungo \(x\) sono acquisiti durante la finestra di acquisizione, dell'ordine dei \(ms\); di conseguenza, il tempo per campionare i punti \(k_{x}\) è trascurabile.

Il tempo che intercorre tra due senquenze con valori dei gradienti diversi deve essere almeno di \(3T_{1} \simeq 3\ s\). Il tempo di acquisizione del volumetto elementare si riduce a:

\[2^{8} \times 100 \times 3\ s = 76800\ s \simeq 21\ h\]

Questo tempo, sebbene molto minore di quello necessario ad acquisire i vari punti nel \(k\)-spazio singolarmente, è ancora estremamente lungo; per cui si rende necessario l'uso di altre strategie per velocizzare i tempi di acquisizione.

Dalle strateggie applicate per campionare il \(k\)-spazio discende la potenza della tecnica di eccitazione con campi elettromagnetici da appilicare in risonanza.

Una volta campionato il \(k\)-sèazio, si ricostruisce l'immagine della densità protonica mediante trasformata inversa di Fourier dei camponi acquisi:

\[\widehat{\rho} = \sum_{i}^{}{\sum_{k}^{}{\sum_{j}^{}{s\left( k_{i}.k_{k},k_{j} \right)\exp\left\lbrack - j2\pi\left( k_{x}x + k_{y}y + k_{z}z \right) \right\rbrack}}}\]

Nell'anlisi della sequenza nel tempo, per indicare che l'ampiezza del gradiente varia tra una sequenza e l'altra si utilizza il simbolo composto da righe parallele, rappresentanti l'ampiezza dei gradienti e una freccia indicante il verso di variazione dell'ampiezza del gradiente. Quest'ultima può anche non essere presente.

\begin{figure}
\centering
\includegraphics[width=1.51456in,height=1.28958in]{media/9_Ric2D/image246.pdf}\caption{Figura .: Simbolo per indicare che l'ampiezza del gradiente aumenta tra una sequenza e la successiva}
\end{figure}

Con questa notazione, la sequenza introdotta per il riempimento del \(k\)-spazio con gradienti di codifica di fase applicati contemporaneamente è la sequente:

\begin{figure}
\centering
\includegraphics[width=5.97772in,height=3.8151in]{media/9_Ric2D/image247.pdf}\caption{Figura .: Sequenza con gradienti variabili tra un'applicazione e la successiva}
\end{figure}

Per indicare che la sequenza è ripetuta si utilizza una freccia ricurva.

In definitiva, mediante questa tecnica è possibile ottenere un campionamento tridimensionale del \(k\)-spazio. La configurazione può essere implementata mediante della sequenza spin-echo o gradient-echo.

\subsection{Imagining bidimensionale}\label{imagining-bidimensionale}

Si restringe l'analisi al solo piano del \(k\)-spazio, \(k_{x} - k_{y}\). Il riempimento del \(k\)-spazio deve avvenire per linee in cui, il valore del gradiente lungo \(y\), quindi \(k_{y}\), è fissato. Il segnale registrato viene campionato nel \(k\)-spazio lungo la direzione \(k_{x}\), ottenendo una riga.

\begin{figure}
\centering
\includegraphics[width=4.12121in,height=4.15909in]{media/9_Ric2D/image248.pdf}\caption{Figura .: Linee acquisite nel k-spazio bidimensionale}
\end{figure}

La sequenza necessaria per lo scopo è ottenuta mediante l'applicazione di un impulso a radiofrequenza, per eccitare gli isocromati, e due impulsi di gradiente, uno lungo \(x\) e l'altro lungo \(y\). Il gradiente di codifica di fase, \(G_{y}\), è applicato per un tempo \(\tau_{y}\), mentre il gradiente di lettura, \(G_{x}\), sfrutta una sequenza gradiente-echo.

\begin{figure}
\centering
\includegraphics[width=6.33214in,height=3.0463in]{media/9_Ric2D/image249.pdf}\caption{Figura .: Sequenza applicata per campionare il \(k\)-spazio bidimensionale}
\end{figure}

L'intensità del gradiente \(G_{y}\) varia ogni volta che la sequenza è ripetuta; in questo modo, a ogni ciclo viene fissata un'ordinata nel \(k\)-spazio, \(k_{y}\) data dalla relazione:

\[k_{y} = \overline{\gamma}G_{y}\tau_{y}\]

Il campionamento lungo la coordinata \(k_{y}\) è ottenuta variando l'ampiezza del gradiente di una quantità \(\Delta G_{y}\), dunque:

\[\Delta k_{y} = \overline{\gamma}\Delta G_{y}\tau_{y}\]

Dove \(\tau_{y}\) è fissato.

Scelto un valore del gradiente \(G_{y}\), ovvero una coordinata \(k_{y}'\), l'applicazione del primo impulso di defasamento sposta la coordinata \(k_{x}\) a valori negativi, dato dalla quantità:

\[k_{x} = - \overline{\gamma}G_{x}\tau_{x}\]

Dove il segno è dovuto alla polarità negativa del gradiente. Il secondo impulso di gradiente di rifasamento, sposta la coordinata \(k_{x}\) dal valore \(- \overline{\gamma}G_{x}\tau_{x}\) al valore finale:

\[k_{x} = \overline{\gamma}G_{x}\tau_{x}'\]

La coordinata \(k_{x}\) è detta anche codifica di frequenza.

Solitamente si sceglie \(\tau_{x}' = 2\tau_{x}\) in questo modo \(k_{x}\) percorre una riga nel \(k\)-spazio di coordinata \(k_{y}'\) da \(- k_{\max}\) \(k_{\max}\).

Il passo di campionamento, ovvero l'intervallo tra un valore acquisito di \(k_{x}\) e il successivo, dipende dal periodo di campionamento \(\Delta t\) nel dominio del tempo, con cui è registrato il segnale durante la finestra di acquisizione:

\[\Delta k_{x} = \overline{\gamma}G_{x}\Delta t\]

Alla seconda applicazione della sequenza si varia il gradiente lungo \(y\) di una quantità \(\Delta G_{y}\). La coordinata della riga cambia, passando da \(k_{y}'\) a \(k_{y}^{''}\). La sequenza gradiente-echo si ripete allo stesso modo, quindi, si acquisiscono valori sia positivi che negativi di \(k_{x}\), con un passo di campionamento legato a \(\Delta t\) da un valore minimo a uno massimo.

Ripetendo la procedura per \(100\) volte si ottiene una matrice bidimensionale di \(256 \times 100\) da cui è possibile ricostruire una fetta della densità protonica \(\widehat{\rho}(x,y)\). Con questa metodica si realizza una scansione ordinata del \(k\)-spazo che parte da uno dei valori estremi fino all'altro estremo, di polarità opposta, mantenendo costante \(k_{y}\).

Il tempo di ripetizione tra l'applicazione di un impulso e il successivo deve essere tale da garantire il recupero della magnetizzazione. Se il tempo di ripetizione è di \(3\ s\), per ottenere \(100\) righe lungo \(k_{y}\) è necessario un tempo di \(300\ s \simeq 5\ min\) per ottenere una singla fetta. Si osservi che il campionamento di \(k_{x}\) avviene in una finestra temporale di ampiezza molto minore dei tempi di rilassamento \(T_{1}\) e \(T_{2}\), quindi, può essere trascurata nell'analisi dei tempi di acquisizione del \(k\)-spazio. Più nel dettaglio \(\Delta t\) è dell'ordine dei \(ms\).

Avere due componenti diverse lungo gli assi non introduce una quota di rumore nella ricostruzione dell'immagine poiché, nel sistema di riferimento del laboratorio, l'asse \(z\) è disposto parallelamente al gantry, quindi diretto nella direzione verticale del paziente; l'asse \(x\) è diretto verso la sinistra del paziente mentre l'asse \(y\) è entrante nella schiena del paziente.

\begin{figure}
\centering
\includegraphics[width=6.69306in,height=2.7755in,alt={Coordinate sytems.png}]{media/9_Ric2D/image250.pdf}\caption{Figura .: Sistema di riferimento in risonanza magnetica}
\end{figure}

Questa organizzazione, sebbene non seguita da tutti i costruttori di apparecchiature, è seguito dallo standard DICOM.

Con il sistema di riferimento scelto, l'asse delle \(x\) è diretto nel verso della direzione maggiore della sezione del corpo umano, con dimensione dell'ordine di \(50 \div 60\ cm\). L'asse \(y\), invece, è diretto verso la dimensione minore della sezione del corpo, dell'ordine di \(25 \div 30\ cm\). In questo contesto, per ottenere una buona ricostruzione, non è fondamentale campionare le due dimensioni del piano con lo stesso passo di campionamento, poiché le dimensioni non sono uguali: alla dimensione minore corrisponde il minor numero di punti, mentre alla codifica di frequenza, dato che avviene lungo la direzione maggiore, corrisponde un numero di punti maggiore.

\begin{figure}
\centering
\includegraphics[width=3.63889in,height=2.15854in]{media/9_Ric2D/image251.pdf}\caption{Figura .: Schematizzazione della sezione del corpo umano e sistema di riferimento}
\end{figure}

L'acquisizione sequenziale della codifica di fase, mediante la variazione del gradiente lungo \(y\) da un valore massimo al minimo o viceversa, può portare a degli errori di ricostruzione a causa dei movimenti del paziente, che rendono la densità protonica efficace non statica ma variabile nel tempo. Di conseguenza, tra una sequenza di acquisizione e l'altra, durante il tempo di ripetizione, possono presentarsi degli artefatti da movimento, Ad esempio, tra la prima e l'ultima riga del \(k\)-spazio, il paziente può essersi mossi, variando la disposizione spaziale della densità protonica. In questa evenienza, il \(k\)-spazio contiene dati non consistenti che introducono degli errori nella ricostruzione.

\begin{figure}
\centering
\includegraphics[width=6.06335in,height=3.58383in,alt={Immagine che contiene testo, Carattere, schermata, linea Il contenuto generato dall\textquotesingle IA potrebbe non essere corretto.}]{media/9_Ric2D/image252.pdf}\caption{Figura .: Acquisizione del k-spazio con densità protonica efficace dipendente dal tempo}
\end{figure}

La densità protonica efficace è legata al segnale registrato tramite la trasformata di Fourier, dunque, il segnale registrato rappresenta il contenuto frequenziale dell'immagine \(\widehat{\rho}(x,y)\). È noto che le basse frequenze corrispondono a zone dell'immagine omogenee, mentre le alte frequenze a zone che subiscono rapide variazioni. Acquisendo il segnale in modo sequenziale, da un estremo all'altro, le componenti frequenziali intorno allo zero possono essere già disturbate da artefatti da movimento, nel senso che si riferiscono a una diversa distribuzione protonica rispetto all'inizio dell'acquisizione. La ricostruzione dell'immagine porta a degli errori sia nelle componenti omogenee dell'immagine, legate al parenchima di un organo, sia dei contorni sfumati, legati ai bordi dei vari organi.

Per preservare il contenuto a basse frequenze si riempie il \(k\)-spazio tracciando dapprima le righe che sono più vicine all'asse \(k_{x}\) e acquisendo, alternativamente, una riga per \(k_{y} > 0\) e l'altra per \(k_{y} < 0\). Questa alternanza permette di acquisire le prime righe al centro del \(k\)-spazio in cui risiedono le informazioni relative al parenchima di un organo.

\begin{figure}
\centering
\includegraphics[width=4.15972in,height=2.65556in]{media/9_Ric2D/image253.pdf}\caption{Figura .: Acquisizione alternata del \(k\)-spazio}
\end{figure}

Il vantaggio di questa tecnica risiede nel fatto che, statisticamente, il paziente resta fermo nei primi minuti della scansione della slice. Gli artefatti da movimento, dunque, saranno visibili solamente alle alte frequenze, determinando di conseguenza una sfumatura dei bordi.

L'acquisizione delle righe spettrali, partendo dal centro in maniera alternata, è detta centrica e permette di avere le righe centrali acquisite con maggiore affidabilità. Gli artefatti a basse frequenze si presentano se il paziente si muove molto durante l'imaging, negli altri casi almeno le righe associate a regioni omogenee sono acquisite con fedeltà.

A differenza del RMI, la CT riesce ad acquisire immagini tomografiche total body anche in breath hold in pochi secondi, per cui non presenta il fenomeno dello smussamento dei bordi. Lo svantaggio risiede nella necessità di fornire radiazioni ionizzanti al paziente, acquisendo così la densità elettronica per discriminare i tessuti.

\subsection{Multislice selection}\label{multislice-selection}

La risonanza magnetica può essere utilizzata per acquisire contemporaneamente più fette mediante l'applicazione di impulsi a radiofrequenza che selezionano ascisse predefinite lungo \(z\), con un determinato spessore. Mediante l'applicazione del gradiente di campo magnetico lungo \(z\), si instaura una relazione biunivoca tra la posizione degli isocromati e la frequenza di precessione di Larmor, secondo la relazione:

\[\omega(z) = \gamma B_{0} + \gamma G_{z}z \Leftrightarrow f(z) = f_{0} + \overline{\gamma}G_{z}z\]

Si definisce asse di selezione della slice o \emph{slice selection axis} come la direzione perpendicolare al piano in cui giace la slice di interesse. Generalmente, la slice giace nel piano \(xy\), dunque, il \emph{slice selection axis} coincide con l'asse \(z\). In questo modo si ottiene una slice detta trasversale del corpo umano.

La scelta dell'asse \(y\) come slice selection asxis porta a ottenrere delle slice del corpo dette radiali o frontali; mentre la scelta dell'asse \(x\) porta a ottenere delle slice definite sagittali.

\begin{figure}
\centering
\includegraphics[width=5.80587in,height=5.46875in]{media/9_Ric2D/image254.pdf}\caption{Figura .: Piani in cui viene diviso il corpo}
\end{figure}

\begin{longtable}[]{@{}
  >{\centering\arraybackslash}p{(\linewidth - 4\tabcolsep) * \real{0.4566}}
  >{\centering\arraybackslash}p{(\linewidth - 4\tabcolsep) * \real{0.1700}}
  >{\centering\arraybackslash}p{(\linewidth - 4\tabcolsep) * \real{0.3734}}@{}}
\caption{Tabella 9.1: Piani in cui è sezionato il corpo umano}\tabularnewline
\toprule\noalign{}
\begin{minipage}[b]{\linewidth}\centering
\textbf{Applied slice select gradient}
\end{minipage} & \begin{minipage}[b]{\linewidth}\centering
\textbf{Name}
\end{minipage} & \begin{minipage}[b]{\linewidth}\centering
\textbf{Slice plane orientation}
\end{minipage} \\
\midrule\noalign{}
\endfirsthead
\toprule\noalign{}
\begin{minipage}[b]{\linewidth}\centering
\textbf{Applied slice select gradient}
\end{minipage} & \begin{minipage}[b]{\linewidth}\centering
\textbf{Name}
\end{minipage} & \begin{minipage}[b]{\linewidth}\centering
\textbf{Slice plane orientation}
\end{minipage} \\
\midrule\noalign{}
\endhead
\bottomrule\noalign{}
\endlastfoot
\(G_{x}\) & sagittal & parallel to \(y - z\) plane \\
\(G_{y}\) & coronal & parallel to \(x - z\) plane \\
\(G_{z}\) & transverse & parallel to \(x - y\) plane \\
\end{longtable}

Nello specifico, per la slice trasversale, la frequenza di precessione è una funzione lineare della posizione \(z\) degli isocromati.

Si vuole eccitare una sottile fetta del corpo umano, considerato uniforme, mediante l'eccitazione degli isocromati a una data frequenza di precessione. L'eccitazione dell'impulso a radiofrequenza eccita gli isocromati della slice allo stesso modo, nel sendo che tutti gli isocromati possiedono la stessa fase e flippano allo stesso modo dopo la slice selection. Dal punto di vista analitico, il paziente più essere diviso in infinite fette sottilissime, posizionate lungo la coordinata \(z\).

\begin{figure}
\centering
\includegraphics[width=4.55406in,height=3.03588in,alt={Generazione immagine completata}]{media/9_Ric2D/image255.pdf}\caption{Figura .: Paziente diviso in slice}
\end{figure}

Per selezionare una singola fetta alla coordinata \(z_{0}\) all'interno del volume del paziente, è necessario che l'impulso a radiofrequenza abbia una frequenza esattamente uguale a \(f\left( z_{0} \right) = f_{0} + \overline{\gamma}G_{z}z\), frequenza di precessione degli spin a quella data ascissa. In ipotesi di campo principale omogeno, il termine \(f_{0}\) viene rimosso a valle della demodulazione.

Per selezionare solamente la frequenza \(f\left( z_{0} \right)\) l'impulso a radiofrequenza deve essere una sinusoide infinita, cosicché il suo spettro sia una delta di Dirac centrata alla frequenza \(f\left( z_{0} \right)\). Nel sistema di riferimento fisso del laboratorio il campo a radiofrequenza deve essere:

\[B_{1}(t) = A\cos\left( 2\pi f\left( z_{0} \right)t \right)\]

Nel sistema di riferimento rotante o a valle della demodulazione, l'impulsi RF si scrive come:

\[B_{1}(t) = A\cos\left( 2\pi\overline{\gamma}G_{z}z_{o}t \right)\]

In questo modo gli isocromati a frequenza \(z_{0}\) flippano sul piano trasverso.

Nella pratica non è possibile ottenere un impulso infinitamente lungo e, per motivi di rapidità di esecuzione, tempi lungi portano a maggiori artefatti da movimento.

Lo spettro di frequenza di un impulso reale non è impulsivo, come desiderato, ma è finito. Di conseguenza, non viene selezionata una fetta infinitesima del corpo umano ma con un certo spessore \(\Delta z\).

Nel dominio del tempo, l'impulso a radiofrequenza può essere pensato come un segnale \(sinc\) troncato nello spazio:

\[B_{1} = A{sinc}\left( 2\pi f\left( z_{0} \right)t \right){rect}\left( \frac{z}{\Delta z} \right)\]

La trasformata di Fourier dell'impulso è una \(rect\) smussata, così da avere una migliore selezione. Nel sistema di riferimento fisso, l'impulso a radiofrequenza è ottenuto come modulazione della \(sinc\) di una portante a frequenza \(f\left( z_{0} \right) = f_{0} + \overline{\gamma}G_{z}z_{0}\). Nel sistema rotante, invece, l'impulso è centrato nell'origine del sistema di riferimento, in quanto è stata eseguita l'operazione di demodulazione.

La banda della \(sinc\) può essere considerata coincidente con il lobo principale, che si estende da \(- \overline{\gamma}G_{z}\Delta z/z\) a \(\overline{\gamma}G_{z}\Delta z/z\).

\begin{figure}
\centering
\includegraphics[width=6.68958in,height=2.34861in,alt={Immagine che contiene testo, linea, Diagramma, diagramma Il contenuto generato dall\textquotesingle IA potrebbe non essere corretto.}]{media/9_Ric2D/image256.pdf}\caption{Figura .: Impulso a RF e sua trasformata di Fourier}
\end{figure}

La banda dell'impulso a radiofrequenza è data dal:

\[BW_{RF} \equiv \Delta f = \overline{\gamma}G_{z}\Delta z\]

In altre parole, grazie all'applicazione del gradiente di campo \(G_{z}\), vi è un intervallo frequenziale, dipendente dallo spessore \(\Delta z\) della slice, che viene eccitato dall'impulso a radiofrequenza. Gli isocromati contenenti nella fetta di spessore \(\Delta z\) sono ribaltati, dando origine a una magnetizzazione trasversa diversa da zero. Il ritorno all'equilibrio produce il segnale registrato dalle antenne.

Si introduce lo spessore di fetta o \emph{slice} \emph{thickness} come:

\[TH = \Delta z\]

Lo spessore della slice eccitata dall'impulso a radiofrequenza, con banda \(BW_{RF}\) è data da:

\[TH = \Delta z = \frac{BW_{RF}}{\overline{\gamma}G_{z}}\]

Progettando opportunamente l'impulso a radiofrequenza, in modo da avere una banda \(BW_{RF}\) quanto più piccola possibile, si riesce a ridurre lo spessore della slice del corpo umano. Tipici spessori in risonanza magnetica sono di \(1\ mm\) fino a \(3 \div 4\ mm\) in alcune applicazioni.

A causa di problemi pratici, non è possibile ottenere fette con uno spessore minore di \(1\ mm\), poiché, preso un cubetto elementare con spessore inferiore a tale dimensione, il numero di protoni contenuti potrebbe essere anche molto minore del numero di Avogadro. Il segnale dovuto al ritorno all'equilibrio della fetta, di conseguenza, potrebbe non essere registrato poiché confuso dal rumore. Per garantire un certo rapporto segnale/rumore il minimo spessore della slice del corpo umano è fissato a \(1\ mm\).

Data la finitezza dello spessore della slice, gli spin all'estremo \(- z_{0}\) saranno più lenti rispetto agli spin posizionati all'estremo \(z_{0}\). Alla fine dell'impulso a radiofrequenza, gli spin non avranno la stessa fase ma ci sarà un certo defasamento, dovuto proprio alla dimensione finita della slice.

\begin{figure}
\centering
\includegraphics[width=6.68958in,height=5.34097in]{media/9_Ric2D/image257.pdf}\caption{Figura .: Andamento della fase a causa dello spessore della slice}
\end{figure}

Per evitare la dispersione degli isocromati nel piano trasverso, bisogna applicare, subito dopo il gradiente di eccitazione, un gradiente di rifocalizzazione con polarità opposta al primo. Questo gradiente inverte l'andamento degli spin in modo da avere una rifocalizzazione sull'asse del ribaltamento. Tipicamente si progetta il gradiente di eccitazione in modo che abbia area doppia rispetto a quello di focalizzazione successivo; in modo che, all'istante \(t = 0\ s\) di fine del gradiente di focalizzazione, la magnetizzazione è tutta focalizzata sull'asse del ribaltamento. Gli spin nella fetta selezionata, in altre parole, possiedono tutti la stessa fase.

\begin{figure}
\centering
\includegraphics[width=5.15909in,height=4.11903in]{media/9_Ric2D/image258.pdf}\caption{Figura .: Sequenza per focalizzare tutti gli isocromati}
\end{figure}

Per campionare il \(k\)-spazio è necessario applicare un gradiente nella direzione \(y\), con ampiezza variabile tra una sequenza e l'altra, mentre sull'asse \(x\) è applicata una sequenza gradient-echo. Il lungo \(y\) si ha il gradiente di codifica di fase o \emph{phase encoding} \(G_{y}\), mentre lungo \(x\) quello di lettura.

\begin{figure}
\centering
\includegraphics[width=4.01515in,height=4.01515in,alt={Immagine che contiene testo, diagramma, linea, Carattere Il contenuto generato dall\textquotesingle IA potrebbe non essere corretto.}]{media/9_Ric2D/image259.pdf}\caption{Figura .: Sequenza per l'acquisizione del segnale proveniente da una singola slice}
\end{figure}

Il segnale è acquisito durante l'impulso di rifocalizzazione del gradiente lungo \(x\), così da campionare l'asse \(k_{x}\) con un passo di campionamento di:

\[\Delta k_{x} = \gamma G_{z}\Delta t\]

Il riempimento del \(k\)-spazio avviene per righe, ovvero, si fissa l'ampiezza \(G_{y}\) del gradiente lungo \(y\) e si campiona la riga. Alla successiva applicazione della sequenza si varia il valore del gradiente di codifica di fase così da selezionare una seconda ordinata del \(k\)-spazio, acquisendo una seconda riga, e così via. Tra due sequenze consecutive è necessario aspettare un tempo di ripetizione \(T_{R}\) di almeno \(3\ s\) per garantire che il vettore di magnetizzazione ritorni all'equilibrio.

Per ridurre i tempi di acquisizione dell'intero volume del paziente mediante la tecnica della slice selection, è possibile progettare gli impulso in modo da eccitare più fette contemporaneamente. Infatti, valendo il principio di sovrapposizione delle fasi, è possibile applicare il gradiente di selezione, di fase e di lettura, prestando attenzione a non sovrapporre temporalmente il gradiente di defasamento della gradient-echo con gli altri due.

Per un imaging della slice bidimensionale, la sequenza più rapida possibile prevede:

\begin{enumerate}
\def\labelenumi{\arabic{enumi}.}
\item
  Un impulso a radiofrequenza;
\item
  Un gradiente di selezione della slice, seguito da un gradiente di focalizzazione;
\item
  Un gradiente di codifica di fase sovrapposto al gradiente di focalizzazione;
\item
  Un gradiente di defasamento e di lettura tramite la sequenza gradient-echo.
\end{enumerate}

\begin{figure}
\centering
\includegraphics[width=6.69306in,height=7.16875in,alt={Immagine che contiene testo, diagramma, linea, Carattere Il contenuto generato dall\textquotesingle IA potrebbe non essere corretto.}]{media/9_Ric2D/image260.pdf}\caption{Figura .: Sequenza tipicamente adottata nella slice selection}
\end{figure}

I gradienti di focalizzazione, di codifica, di fase e di defasamento sono sovrapposti, mentre i gradienti di selezione della fetta e di lettura sono applicati, rispettivamente, subito prima e subito dopo i tre gradienti.

Lo schema proposto, oltre a essere ripetuto nel tempo, può essere ripetuto in modo da selezionare delle slice ad ascisse \(z\) diverse tra loro, permettendo di limitare le interferenze tra i segnali emanati dal ritorno all'equilibrio del vettore di magnetizzazione della singola slice.

Applicando la stessa sequenza di eccitazione, ma con un impulso RF a frequenza:

\[f\left( z_{1} \right) = \overline{\gamma}B_{0} + \overline{\gamma}G_{z}z_{1}\]

Si eccita contemporaneamente sia la slice a \(z_{0}\) sia a \(z_{1}\). Per evitare fenomeni di interferenza tra le due fette, acquisite contemporaneamente, deve intercorrere una distanza di \(3 \div 4\ mm\), dove:

\[d = \frac{\Delta\omega_{slice} - 2\pi BW}{\gamma G_{z}}\]

Nel caso in cui non sia rispettato il giusto gap possono verificarsi problemi di sfocamento o blander.

La selezione di più fette, separate da un giusto gap, risulta essere molto conveniente per ridurre l'intero processo di acquisizione dei dati volumetrici. Infatti, in un tempo \(T_{R}\) si ottengono informazioni per \(n\) fette contemporaneamente, riducendo i tempi di acquisizione. Con questa soluzione, in \(5\ min\) è possibile ottenere più fette.

L'unione delle fette, mediante appositi algoritmi di interpolazione, permette di ottenere un'immagine tridimensionale del corpo umano. Questa metodica è nota come 3D-multislice.

La metodica appena descritta differisce dall'eccitazione 3D poiché, in quest'ultimo caso, l'impulso a radiofrequenza eccita l'intero volume, dunque, non vi è necessità del gradiente di selezione della fetta.

In alcune metodiche, come l'oncologia o nello studio della diffusione in vivo, si applica la ricostruzione 3D multislice, selezionando opportunamente le slice da eccitare per la ricostruzione; viceversa, in alcuni casi si preferisce la ricostruzione tridimensionale come l'analisi spettroscopica.

\subsection{Spettroscopia con risonanza magnetica}\label{spettroscopia-con-risonanza-magnetica}

La risonanza magnetica spettroscopica o MRSI (\emph{Magnetic Resonance Spettroscopic Imaging}) è una tecnica di risonanza magnetica grazie alla quale è possibile rilevare la composizione metabolica del volume sotto analisi.

Si divide il volume-paziente in tanti volumetti elementari, ognuno dei quali contiene un certo numero di macromolecole, in cui sono contenuti gli atomi di idrogeno di cui si vuole determinare la distribuzione.

Le molecole che contengono l'idrogeno schermano i protoni di idrogeno dal campo magnetico principale, quindi, il campo visto dal protone è diverso da quello applicato esternamente. Questo fenomeno è dovuto alla rotazione degli elettroni orbitanti intorno al protone in modo piuttosto complesso.

\begin{figure}
\centering
\includegraphics[width=3.37378in,height=3.17708in]{media/9_Ric2D/image261.pdf}\caption{Figura .: Elettroni orbitanti in macromolecola}
\end{figure}

Il campo percepito dal protone è dato da:

\[B_{0} + \Delta B\]

Le variazioni del campo principale \(\Delta B\) sono caratteristiche strettamente alle molecole a cui è legato l'atomo di idrogeno. Di conseguenza, ogni sostanza presenta protoni dell'idrogeno che risuonano a una determinata frequenza di precessione, dipendente dalla molecola stessa.

Il segnale prelevato è dato dalla somma di tutti i segnali, a frequenze diverse, dei nuclei legati alle diverse macromolecole

\begin{figure}
\centering
\includegraphics[width=5.15139in,height=3.85312in,alt={Risonanza magnetica per spettroscopia encefalo convenzionata Roma}]{media/9_Ric2D/image262.pdf}\caption{Figura .: Spettroscopia RM in neuroradiologia}
\end{figure}

In basi allo spettro ricevuto possibile determinare la composizione chimica del tessuto e la sua attività metabolica. Ogni spettro del segnale è caratteristico di una specifica molecola.

Per ottenere questo risultato è necessario applicare una sequenza 3D non multislice, applicando un gradiente di codificaa di fase lungo \(y\) e \(z\), e un gradiente di lettura, mediante gradient-echo, lungo \(x\).

\begin{figure}
\centering
\includegraphics[width=6.69306in,height=6.69306in,alt={Immagine che contiene testo, diagramma, Disegno tecnico, linea Il contenuto generato dall\textquotesingle IA potrebbe non essere corretto.}]{media/9_Ric2D/image263.pdf}\caption{Figura .: Sequenza di imagining 3D}
\end{figure}

Il segnale letto, una volta antitrasformato, fornisce lo spettro del segnale ricevuto da ogni singolo volumetto in cui è possibile scomporre il paziente.

Questo tipo di imaging ha una durata maggiore di un 3D multislice. Mediante delle sequenze, dette veloci, è possibile ottenere un volume dell'ordine di \(256 \times 256 \times 128\) in circa \(4 \div 5\ min\) mediante un 2D multislice.

Per effettuare un'analisi oncologica in grado di determinare il tipo e lo stadio del tumore, in base al suo metabolismo, è necessario applicare la 3D tradizionale, più lunga.

È possibile eseguire un 3D multislice per analizzare il volume da studiare dal punto di vista fisiologico, e un secondo tipo di analisi funzionale per evidenziare i comportamenti metabolici di una massa o un tessuto.

È possibile eseguire un imaging di diffusione con la tecnica del DWI (\emph{Diffusion Weighted Imaging}) in cui si evidenzia la mobilità dei protoni dell'acqua di un tessuto o di un farmaco. Questa tecnica si presta molto bene all'individuazione di edemi, con risoluzione più spinta della CT.

\begin{center}
\vfill
    \chapter{Ricostruzione tridimensionale in MRI}
    \label{blx:Ric3D\therefsection}
\vfill

\minitoc
\newpage
\end{center}
\justify

\section{Ricostruzione dell'immagine bidimensionale}\label{ricostruzione-dellimmagine-bidimensionale}

Il segnale \(s(k)\), acquisito nel \(k\)-spazio è legato alla densità protonica efficace \(\widehat{\rho}\left( \overset{\underline{}}{r} \right)\) nello spazio-immagine tramite la trasformata di Fourier:

\[s\left( \overset{\underline{}}{k} \right) = \int_{\mathbb{R}^{3}}^{}{\widehat{\rho}\left( \overset{\underline{}}{r} \right)\exp\left( - j2\pi\overset{\underline{}}{k} \cdot \overset{\underline{}}{r} \right)d^{(3)}\overset{\underline{}}{r}}\]

\(s\left( \overset{\underline{}}{k} \right)\) è la trasformata di Fourier della densità protonica efficace:

\[\widehat{\rho}\left( \overset{\underline{}}{r} \right) = \int_{\mathbb{R}^{3}}^{}{s\left( \overset{\underline{}}{k} \right)\exp\left( j2\pi\overset{\underline{}}{k} \cdot \overset{\underline{}}{r} \right)d^{(3)}\overset{\underline{}}{k}}\]

Per semplicità si considera il caso monodimensionale. Per ottenere l'immagine della densità protonica nello spazio-immagine, dal punto di vista analitico, è necessario invertire semplicemente la trasformata di Fourier, al fine di ottenere la densità protonica; tuttavia, nella pratica, tale processo non è così semplice da svolgere sia per la finitezza delle memorie degli elaboratori digitali, sia per le altre problematiche associate alla strumentazione stessa. Possono esserci dei casi in cui la densità protonica ricostruita \(\widehat{\rho}(x)\) sia complessa, ovvero abbia sia una parte reale che immaginaria.

Una delle prime problematiche riguardanti la ricostruzione del segnale è dovuto all'acquisizione non illimitata del \(k\)-spazio; infatti, il segnale registrato è campionato nel dominio delle \(k\) in un'apposita finestra di acquisizione dalla durata limitata. Ad esempio, in una sequenza gradient-echo bidimensionale, in cui si applicano il gradiente di selezione della fetta \(G_{ss}\), il gradiente di codifica di fase (o \emph{phase econding}) \(G_{PE}\), e il gradiente di lettura \(G_{R}\), modellato, appunto, come una sequenza gradient-echo.

Il segnale è acquisito in un intorno del tempo di echo \(T_{E}\), in una finestra temporale con durata molto minore dei tempi di rilassamento.

Successivamente, il segnale acquisito sarà demodulato mediante un demodulatore in quadratura o complesso, in cui il segnale è inviato a due moltiplicatori in quadratura, ovvero, un ramo effettua la moltiplicazione per un segnale sinusoidale, mentre il ramo parallelo per un segnale sfatato di \(\pi/2\), quindi in quadratura.

Il ramo che effettua la moltiplicazione per il segnale sinusoidale è detto canale reale poiché restituisce un segnale in fase con quelo in ingresso; mentre il secondo restituisce un segnale in quadratura e, per tale motivo, è noto come canale immaginario.

\begin{figure}
\centering
\includegraphics[width=6.69306in,height=3.28333in,alt={Immagine che contiene testo, schermata, diagramma, Carattere Il contenuto generato dall\textquotesingle IA potrebbe non essere corretto.}]{media/10_Ric3D/image264.pdf}\caption{Figura .: Demodulatore complesso}
\end{figure}

Il segnale in uscita dal demodulatore può, quindi, essere scritto come:

\[s(k) = s_{R}(k) + js_{I}(k)\]

Il campionamento e la durata finita della finestra di acquisizione possono indurre degli errori nella ricostruzione e, allo stesso tempo, il ricevitore può introdurre del rumore durante la ricezione del segnale, a causa della non perfetta coincidenza della frequenza dell'oscillatore armonico all'interno del demodulatore e la frequenza del segnale registrato \(s(k)\). In questo scenario, le oscillazioni con cui si moltiplica il segnale \(s(k)\) possono essere scritte come:

\[\cos\left\lbrack (\omega + \Delta\omega)t + \phi \right\rbrack,\ \sin\left\lbrack (\omega + \Delta\omega)t + \phi \right\rbrack\]

In questa condizione, la densità protonica efficace \(\widehat{\rho}(x)\) non è puramente reale, come condizione teorica, ma è complessa:

\[\widehat{\rho}(x) = \left| \widehat{\rho}(x) \right|\exp\left( j\angle\widehat{\rho}(x) \right)\]

Intrinsecamente \(\widehat{\rho}(x)\) indica il numero di spin contenuti in un volumetto reale, quindi, corrisponde a un numero reale; tuttavia, nel caso in cui il demodulatore presenta uno sfasamento o una differenza di frequenza rispetto l'oscillatore armonico, il segnale ricostruito potrebbe essere complesso.

Si suppone che tra l'oscillatore del demodulatore e il segnale \(s(k)\), registrato dalle antenne, vi sia uno sfasamento \(\phi\). Ciò porta i due canali a non produrre parte reale e parte immaginaria del segnale in ingresso, ma dei valori misti tra loro. Nello specifico, si suppone che il segnale registrato \(s(k)\) abbia un andamento del tipo:

\[s(k) \propto \sin\left( \omega_{0}t + \xi \right)\]

In ipotesi di frequenza perfettamente uguale tra segnale e oscillatore, la differenza di frequenza è nulla \(\Delta\omega = 0\). Il segnale \(s(k)\) è moltiplicato, nel canale reale da \(\sin\left( \omega_{0}t + \phi \right)\) e nel canale immaginario da \(\cos\left( \omega_{0}t + \phi \right)\).

Per il canale reale, risulta:

\[\sin\left( \omega_{0}t + \xi \right)\sin\left( \omega_{0}t + \phi \right) =\]

Applicando le formule di Werner, si ha:

\[= \dfrac{1}{2}\cos(\xi - \phi) - \dfrac{1}{2}\cos\left( 2\omega_{0}t + \phi + \xi \right)\]

Analogamente per il canale immaginario si ha:

\[\sin\left( \omega_{0}t + \xi \right)\cos\left( \omega_{0}t + \phi \right) =\]

Per Wagner si ha:

\[= \dfrac{1}{2}\sin(\xi - \phi) + \dfrac{1}{2}\sin\left( 2\omega_{0}t + \phi + \xi \right)\]

Mediante filtraggi passa-basso è possibile rimuovere le componenti a elevata frequenza, centrate su \(2\omega_{0}\). In uscita al filtro, il segnale è dato da:

\[s(k) \propto \dfrac{1}{2}\cos(\xi - \phi) + j\dfrac{1}{2}\sin(\xi - \phi) =\]

Applicando le relazioni trigonometriche si ottiene:

\[= \dfrac{1}{2}\left\lbrack \left( \cos\xi\cos\phi + \sin\xi\sin\phi \right) + j\left( \sin\xi\cos\phi - \cos\xi\sin\phi \right) \right\rbrack =\]

Raccogliendo \(\cos\xi\) e \(\sin\xi\), si ottiene:

\[= \dfrac{1}{2}\left\lbrack \cos\xi\left( \cos\phi - j\sin\phi \right) + \sin\xi\left( \sin\phi + j\cos\phi \right) \right\rbrack =\]

Nel termine \(\sin\phi + j\cos\phi\) si mette in evidenza \(j\):

\[= \dfrac{1}{2}\left\lbrack \cos\xi\left( \cos\phi - j\sin\phi \right) + j\sin\xi\left( \cos\phi - j\sin\phi \right) \right\rbrack =\]

Per le formule di Eulero, si ottiene:

\[= \dfrac{1}{2}\left\lbrack \cos\xi\exp( - j\phi) + j\sin\xi\exp( - j\phi) \right\rbrack =\]

Raccogliendo \(\exp( - j\phi)\) e applicando le formule di Eulero, si ottiene:

\[= \dfrac{1}{2}\exp( - j\phi)\exp(j\xi)\]

In questa espressione \(\exp(j\xi)\) rappresenta il segnale registrato dalle antenne, demodulato, mentre \(\exp( - j\phi)\) un termine di fase dovuto allo sfasamento tra segnale e oscillatore. Di conseguenza, il segnale complessivo può essere espresso come:

\[s(k) = s(k)\exp( - j\phi)\]

dove \(s(k)\) è il segnale ideale demodulato, privo di sfasamenti.

In conclusione, quando la frequenza del segnale è perfettamente allineata a quella dell'oscillatore locale, il segnale demodulato differisce da quello ideale solo per un termine di fase \(\phi\), legato allo sfasamento tra le due portanti.

La densità protonica \(\widehat{\rho}(x)\) è ottenuta mediante antitrasformata di Fourier del segnale \(s(k)\) nel \(k\)-spazio. In ipotesi di acquisizione continua risulta:

\[\widehat{p}(x) = \int_{}^{}{s(k)\exp( - j\phi)\exp(j2\pi kx)dk} = \exp( - j\phi)\int_{}^{}{s(k)\exp(j2\pi kx)dk} = \widehat{p}(x)\exp( - j\phi)\]

La densità protonica ottenuta è uguale a quella teorica, moltiplicato un fattore di sfasamento \(\exp( - j\phi)\). Il semplice errore di demodulazione legato allo sfasamento tra l'oscillatore di ricezione e il segnale registrato conduce a una densità protonica ricostruita complessa.

Siccome non è possibile rappresentare nel piano o nello spazio una funzione complessa, per ottenere la raffigurazione della slice o del volume anatomico di interesse, si effettua un'operazione di modulo:

\[\left| \widehat{\rho}(x) \right| = \left| \widehat{p}(x)\exp( - j\phi) \right| = \left| \widehat{p}(x) \right|\]

In queto modo, dal punto di vista analitico, l'immagine ricostruita coincide con l'immagine originale senza tener conto dello sfasamento.

Si suppone, ora, che i canali reale e immaginario siano invertiti. Tale effetto si verifica nel momento in cui le antenne di ricezione in quadratura sono invertite. In questa evenienza, il segnale ottenuto a fine demodulazione risulta avere parte reale e parte immaginaria invertite:

\[s(k) = s_{I}(k) + js_{R}(k)\]

La densità protonica ricostruita è ottenuta mediante trasformata di Fourier inversa di \(s(k)\), per cui risulta:

\[\widehat{p}(x) = \int_{}^{}{s(k)\exp(j2\pi kx)dk}\]

Siccome \(\rho(x)\) è una funzione reale, il segnale \(s(k)\), nel \(k\)-spazio, è hermitiana:

\[s^{*}\left( \overset{\underline{}}{k} \right) = s\left( - \overset{\underline{}}{k} \right)\]

\(\overset{\underline{}}{k}\) è una variabile reale, per cui \({\overset{\underline{}}{k}}^{*} = \overset{\underline{}}{k}\). Moltiplicando per \(j\):

\[js^{*}\left( \overset{\underline{}}{k} \right) = j\left( s_{R} + js_{I} \right)^{*} = j\left( s_{R} - js_{I} \right) = s_{I} + js_{R}\]

In questo modo si ottiene il segnale fisicamente registrato dalle antenne sui canali reale e immaginario rispettivamente. La densità protonica si può scrivere come:

\[\widehat{p}(x) = \int_{}^{}{js^{*}\left( \overset{\underline{}}{k} \right)\exp(j2\pi kx)dk} =\]

Si scrive \(j\) in notazione esponenziale, ovvero \(j = \exp(j\pi/2)\):

\[= \int_{}^{}{s^{*}\left( \overset{\underline{}}{k} \right)\exp\left( j\dfrac{\pi}{2} \right)\exp(j2\pi kx)dk} =\]

Per le proprietà del complesso coniugato è possibile scrivere:

\[= \left\lbrack \int_{}^{}{s\left( \overset{\underline{}}{k} \right)\exp\left( - j\dfrac{\pi}{2} \right)\exp( - j2\pi kx)dk} \right\rbrack^{*}\]

Per definizione di antitrasformata di Fourier, si ha:

\[\widehat{\rho}(x) = \widehat{\rho}( - x)\exp\left( - j\dfrac{\pi}{2} \right)\]

L'immagine finale ricostruita è ottenuta moltiplicando l'immagine originale, ribaltate rispetto all'asse di lettura, moltiplicando per un fattore di fase noto \(\exp( - j\pi/2)\). Considerando il modulo dell'immagine ricostruite, il termine di fase può essere ignorato poiché con modulo unitario:

\[\left| \widehat{\rho}(x) \right| = \left| \widehat{\rho}( - x)\exp\left( - j\dfrac{\pi}{2} \right) \right| = \left| \widehat{\rho}( - x) \right|\]

L'immagine ottenuta e mostrata a video è semplicemente ribaltata rispetto all'origine, quindi, è sempre possibile eseguire la diagnosi.

È possibile avere un errore di ricostruzione dovuto a una diversa frequenza tra l'oscillazione del segnale registrato dalle antenne e dall'oscillatore del demodulatore. Esiste, quindi, una differenza tra la frequenza di precessione \(\omega_{0}\) e la frequenza \(\omega\) del sistema rotante. In questo caso, le oscillazioni con cui si demodula il segnale possono essere scritte come:

\[\cos(\omega t),\ \ \sin(\omega t)\]

Dove \(\omega = \omega_{0} - \Delta\omega\). Il segnale demodulato è del tipo \(s(k) = s_{R}(k) + js_{I}(k)\) con:

\[s_{R} \propto \sin\left( \omega_{0}t + \xi \right)\cos(\omega t),\ \ s_{I} \propto \sin\left( \omega_{0}t + \xi \right)\sin(\omega t)\]

Per il canale reale, si ottiene:

\[s_{R} \propto \sin\left( \omega_{0}t + \xi \right)\cos(\omega t) = \dfrac{1}{2}\sin\left\lbrack \left( \omega_{0} - \omega \right)t + \xi \right\rbrack + \dfrac{1}{2}\sin\left\lbrack \left( \omega_{0} + \omega \right)t + \xi \right\rbrack\]

Per il canale immaginario, invece, risulta:

\[s_{I} \propto \sin\left( \omega_{0}t + \xi \right)\sin(\omega t) = \dfrac{1}{2}\cos\left\lbrack \left( \omega_{0} - \omega \right)t + \xi \right\rbrack - \dfrac{1}{2}\cos\left\lbrack \left( \omega_{0} + \omega \right)t + \xi \right\rbrack\]

Il segnale \(s(k)\) nel \(k\)-spazio, dopo la moltiplicazione con le oscillazioni, può essere scritto come:

\[s(k) \propto \dfrac{1}{2}\sin\left\lbrack \left( \omega_{0} - \omega \right)t + \xi \right\rbrack + \dfrac{1}{2}\sin\left\lbrack \left( \omega_{0} + \omega \right)t + \xi \right\rbrack + j\dfrac{1}{2}\cos\left\lbrack \left( \omega_{0} - \omega \right)t + \xi \right\rbrack - j\dfrac{1}{2}\cos\left\lbrack \left( \omega_{0} + \omega \right)t + \xi \right\rbrack\]

A valle del filtro passa-basso, le frequenze a \(\omega_{0} + \omega\) sono rimosse, per cui si ottiene:

\[s(k) \propto \dfrac{1}{2}\sin\left\lbrack \left( \omega_{0} - \omega \right)t + \xi \right\rbrack + j\dfrac{1}{2}\cos\left\lbrack \left( \omega_{0} - \omega \right)t + \xi \right\rbrack\]

Dove \(\omega_{0} - \omega = \Delta\omega\):

\[s(k) \propto \dfrac{1}{2}\sin(\Delta\omega t + \xi) + j\dfrac{1}{2}\cos(\Delta\omega t + \xi)\]

Si raccoglie \(j\) per applicare le formule di Eulero:

\[\dfrac{1}{2}\sin(\Delta\omega t + \xi) + j\dfrac{1}{2}\cos(\Delta\omega t + \xi) = j\dfrac{1}{2}\left\lbrack \cos(\Delta\omega t + \xi) - j\sin(\Delta\omega t + \xi) \right\rbrack\]

Per le formule di Eulero, si può scrivere:

\[s(k) \propto \dfrac{1}{2}\exp( - j\Delta\omega t)\exp\left( - j\dfrac{\pi}{2} - j\xi \right)\]

È possibile scrivere:

\[s(k) = s(k)\exp( - j\Delta\omega t)\]

Il risultato finale mostra che il segnale nel \(k\)-spazio è moltiplicato per un termine rotante. La fase del segnale demodulato varia nel tempo con frequenza \(2\pi\Delta\omega\). In termini del \(k\)-spazio, lo sfasamento introdotto dalla quantità \(\Delta\omega t = \gamma\Delta Bt\), dove \(\Delta B\) è la differenza di campo magnetico che determina le due diverse frequenze; infatti, a causa delle disomogeneità di campo, gli isocromati non risuonano alla stessa frequenza di eccitazione.

Il campo \(\Delta B\) può essere valutato come un gradiente \(G\) lungo la direzione di lettura, ovvvero \(\Delta B = Gx_{0}\), dove \(x_{0}\) è tale che:

\[\Delta\omega t = \gamma\Delta Bt = \gamma Gx_{0}t\]

È possibile scrivere \(\gamma = 2\pi\overline{\gamma}\), ottenendo:

\[\Delta\omega t = \gamma\Delta Bt = \gamma Gx_{0}t = 2\pi\overline{\gamma}Gx_{0}t\]

Per definizione, \(k = \overline{\gamma}Gt\), da cui si ottiene:

\[\Delta\omega t = 2\pi kx_{0}\]

Risolvendo rispetto a \(x_{0}\), si ottiene:

\[x_{0} = \dfrac{\Delta\omega t}{2\pi k}\]

Il segnale a valle della demodulazione può essere scritto come:

\[s(k) = s(k)\exp( - j\Delta\omega t) = \ s(k)\exp\left( 2\pi kx_{0} \right)\]

Lo sfasamento dovuti a una differenza di frequenza tra l'oscillatore armonico e il segnale ricevuto si traduce in uno spostamento lungo l'asse \(x\) di una quantità \(x_{0}\). Se ssicalcola la trasformata inversa di Fourier per il segnale demodulato \(s(k)\), si ottiene il segnale densità protonica ricostruita:

\[\widehat{p}(x) = \int_{}^{}{s(k)\exp(j2\pi kx)dk} = \int_{}^{}{s(k)\exp\left( 2\pi kx_{0} \right)\exp(j2\pi kx)dk}\]

Ovvero:

\[\widehat{p}(x) = \int_{}^{}{s(k)\exp\left\lbrack j2\pi k\left( x - x_{0} \right) \right\rbrack dk}\]

La soluzione è, quindi uno shift sull'asse di lettura:

\[\widehat{p}(x) = \int_{}^{}{s(k)\exp\left\lbrack j2\pi k\left( x - x_{0} \right) \right\rbrack dk} = \widehat{p}\left( x - x_{0} \right) = \widehat{p}\left( x - \dfrac{\Delta\omega t}{2\pi k} \right)\]

Dalla differenza di frequenza \(\Delta\omega\) discende che l'immagine ricostruita è una versione traslata dell'immagine originale di una quantità \(x_{0}\), dipendente dal tempo della finestra di acquisizione e dalla differenza di fase. Nonostante non sia strettamente necessario è comodo valutare il modulo dell'immagine ricostruita:

\[\left| \widehat{p}(x) \right| = \left| \widehat{p}\left( x - x_{0} \right) \right|\]

\subsection{Problema di ricostruzione legato alle disomogeneità di campo}\label{problema-di-ricostruzione-legato-alle-disomogeneituxe0-di-campo}

Si vuole determinare l'effetto delle disomogeneità di campo sull'immagine finale ricostruita. A tale scopo si considera una sequenza gradient-echo. Quest'ultima è progettata in modo che, sul gradiente di lettura, l'area dell'impulso di rifocalizzazione uguagli l'area del gradiente di focalizzazione al tempo di echo, \(T_{E}\). Nell'intorno di questo istante temporale si attiva la finestra di acquisizione per prelevare il segnale emesso dal tessuto.

\begin{figure}
\centering
\includegraphics[width=3.74167in,height=3.05569in,alt={Pulse diagram of a gradient echo sequence. \textbar{} Download Scientific Diagram}]{media/10_Ric3D/image265.pdf}\caption{Figura .: Sequenza gradient-echo con segnale}
\end{figure}

Per effetto delle disomogeneità di campo si introduce un gradiente aggiuntivo a quello applicato. A causa di ciò l'uguaglianza tra le aree citate non si verifica al tempo di echo \(T_{E}\) previsto ma a un tempo \(T_{E}'\) che può essere maggiore o minore del tempo di echo previso:

\begin{itemize}
\item
  \(T_{E}' < T_{E}\), se la disomogeneità di campo si somma a \(G_{R}\);
\item
  \(T_{E}' > T_{E}\), se la disomogeneità di campo si sottrae a \(G_{R}\).
\end{itemize}

Siccome la finestra di acquisizione è centrata intorno al tempo di echo previsto, il gradiente aggiuntivo determina la registrazione di un segnale \(s(k)\) traslato rispetto al tempo \(T_{E}\) di una quantità \(k_{0}\) data da:

\[k_{0} = \dfrac{\Delta\phi}{2\pi x}\]

Dove \(\Delta\phi\) è lo sfasamento indotto da \(\Delta B\).

\begin{figure}
\centering
\includegraphics[width=6.4in,height=3.28333in,alt={Immagine che contiene linea, Diagramma, diagramma Il contenuto generato dall\textquotesingle IA potrebbe non essere corretto.}]{media/10_Ric3D/image266.pdf}\caption{Figura .: Segnale ricevuto teorico e sfasato a causa delle disomogeneità di campo}
\end{figure}

Si suppone che la finestra riesca comunque ad acquisire l'intero segnale con un tempo di campionamento infinitesimo; allora l'immagine ricostruita è data da:

\[\widehat{\rho}(x) = \int_{}^{}{s\left( k - k_{0} \right)\exp(j2\pi kx)dk}\]

Si aggiunge e sottrae \(j2\pi k_{0}x\) all'esponente:

\[\widehat{\rho}(x) = \int_{}^{}{s\left( k - k_{0} \right)\exp( - j2\pi kx)dk} = \int_{}^{}{s\left( k - k_{0} \right)\exp\left( j2\pi kx + \ j2\pi k_{0}x - \ j2\pi k_{0}x \right)dk}\]

Per le proprietà degli esponenziali:

\[= \int_{}^{}{s\left( k - k_{0} \right)\exp\left( j2\pi kx + \ j2\pi k_{0}x - \ j2\pi k_{0}x \right)dk} = \int_{}^{}{s\left( k - k_{0} \right)\exp\left\lbrack j2\pi\left( k - k_{0} \right)x \right\rbrack\exp\left( \ j2\pi k_{0}x \right)dk}\]

Il termine \(\exp\left( \ j2\pi k_{0}x \right)\) non dipende da \(k\), per cui può essere portato all'esterno del simbolo di integrale:

\[\widehat{\rho}(x) = \exp\left( \ j2\pi k_{0}x \right)\int_{}^{}{s\left( k - k_{0} \right)\exp\left\lbrack j2\pi\left( k - k_{0} \right)x \right\rbrack dk}\]

Siccome \(k_{0}\) è costante, l'integrale si risolve nell'immagine \(\widehat{\rho}(x)\):

\[\widehat{\rho}(x) = \widehat{\rho}(x)\exp\left( \ j2\pi k_{0}x \right)\]

A causa delle disomogeneità di campo, la densità protonica ricostruita è data dalla distribuzione degli spin nel volume, pesata per dei parametri del sistema \(\widehat{\rho}(x)\), a cui va moltiplicato un termine di fase dipendente dalla variabile \(x\), direzione di lettura. La differenza di fase, in particolare, è data da:

\[\phi(x) = \arctan\left( \dfrac{Im\left\{ \widehat{\rho}(x) \right\}}{Re\{\widehat{\rho}(x)\}} \right)\]

Siccome la funzione tangente è invertibile solamente nell'intervallo \(( - \pi;\pi)\) la fase \(\phi(x)\) può variare solamente in questo intervallo spaziale. Visualizzando nel piano-immagine, l'andamento della fase l'andamento nella fase \(\phi(x)\) si osserva che i punti distanziati da un valore di \(x\), tale che \(\phi = 2\pi\), hanno la stessa intensità.

Se l'omogeneità di campo interessa solamente una direzione, si verifica un effetto noto come a strisce di zebra per l'alternanza di righe chiare con righe scure, molto simile a un'immagine affetta da aliasing.

\begin{figure}
\centering
\includegraphics[width=3.275in,height=3.275in,alt={Generazione immagine completata}]{media/10_Ric3D/image267.pdf}\caption{Figura .: Effetto a strisce di zembra su un'immagine di risonanza magnetica}
\end{figure}

La disomogeneità di campo può essere presente sia verso il gradiente di lettura sia di codifica di fase. Le disomogeneità locali del campo producono una fase nella densità protonica ricostruita \(\widehat{\rho}\) dipendente sia dall'asse di lettura che dalla codifica di fase.

Sebbene l'operazione di modulo permetta di ottenere la densità protonica reale, in questo caso l'immagine di fase contiene informazioni molto importante sulla disomogeneità di campo. Osservano l'immagine di fase è possibile determinare \(k_{0}\) come pendenza della gradazione di grigio da bianco a nero. Nota la quantità \(k_{0}\) si determina l'intensità della disomogeneità di campo risalendo anche al tempo di echo.

L'immagine di fase può essere sfruttata per compensare il gradiente indesiderato; infatti, nota la sua intensità e il suo verso, si applica un gradiente di campo opposto, tale da minimizzare gli effetti sul tempo di echo.

Nella pratica clinica si utilizza la sola immagine di modulo, mentre per compensare gli effetti delle limitazioni tecnologiche, si utilizzano le immagini di fase, col fine di migliorare le ricostruzioni delle successive immagini.

\subsection{Guadagno diverso dei due ricevitori}\label{guadagno-diverso-dei-due-ricevitori}

Durante la demodulazione è possibile che i due filtri passa-basso, a valle del moltiplicatore per le sinusoidi, abbiano un guadagno anche leggermente diverso.

Se i filtri del ricevitore non presentano lo stesso guadagno, uno dei due segnali è posto in uscita così come previsto mentre l'altro risulta essere amplificato di una quantità \(\delta\), come il \(90\%\) del valore atteso. Ciò determina una diversa ampiezza tra il canale reale e quello immaginario.

\begin{figure}
\centering
\includegraphics[width=7.075in,height=3.96666in,alt={Immagine che contiene testo, diagramma, Carattere, schermata Il contenuto generato dall\textquotesingle IA potrebbe non essere corretto.}]{media/10_Ric3D/image268.pdf}\caption{Figura .: Demodulatore che presenta un guadagno inferiore sul canale immaginario}
\end{figure}

Si suppone che il canale immaginario abbia problemi di guadagno. Il segnale in uscita è dato da:

\[s(k) = s_{R} + j\delta s_{I}\]

Si aggiunge e sottrae \(js_{I}\), segnale che sarebbe stato posto in uscita dal canale immaginario se non ci fosse il difetto di guadagno:

\[s(k) = s_{R} + j\delta s_{I} + js_{I} - js_{I}\]

Raccogliendo è possibile scrivere:

\[s(k) = s_{R} + js_{I} + j(\delta - 1)s_{I}\]

La densità protonica è ottenuta dalla trasformata inversa di Fourier di \(s(k)\):

\[\widehat{\rho}(x) = \int_{}^{}{s(k)\exp(j2\pi kx)dk} = \int_{}^{}{\left\lbrack s_{R} + js_{I} + j(\delta - 1)s_{I} \right\rbrack\exp(j2\pi kx)dk}\]

Se il segnale densità protonica efficace è reale, allora il segnale nel \(k\)-spazio è harmitiono. In particolare, risulta:

\[s(k) + s( - k) = s(k) + s^{*}(k) = 2Re\left\{ s(k) \right\} = 2s_{R} \Leftrightarrow s_{R} = \dfrac{s(k) + s( - k)}{2}\]

\[s(k) - s( - k) = s(k) - s^{*}(k) = 2jIm\left\{ s(k) \right\} = 2s_{I} \Leftrightarrow s_{I} = \dfrac{s(k) - s( - k)}{2j}\]

Applicando la relazione per \(s_{I}\) nella relazione per \(\widehat{\rho}(x)\) e ricordando che \(s_{R} + js_{I} = s(k)\), è possibile scrivere:

\[\widehat{\rho}(x) = \int_{}^{}{\left\lbrack s_{R} + js_{I} + j(\delta - 1)s_{I} \right\rbrack\exp(j2\pi kx)dk} = \int_{}^{}{\left\lbrack s(k) + j(\delta - 1)\dfrac{s(k) - s( - k)}{2j} \right\rbrack\exp(j2\pi kx)dk} = \int_{}^{}{\left\lbrack s(k) + (\delta - 1)\dfrac{s(k) - s( - k)}{2} \right\rbrack\exp(j2\pi kx)dk}\]

Svolgendo i prodotti si ottiene:

\[= \int_{}^{}{\left\lbrack s(k) + \dfrac{1}{2}(\delta - 1)s(k) - \dfrac{1}{2}(\delta - 1)s( - k) \right\rbrack\exp(j2\pi kx)dk}\]

Da cui:

\[\widehat{\rho}(x) = \int_{}^{}{\left\lbrack \left( 1 + \dfrac{\delta - 1}{2} \right)s(k) - \dfrac{1}{2}(\delta - 1)s( - k) \right\rbrack\exp(j2\pi kx)dk} = \widehat{\rho}(x) =\]

Da cui:

\[\widehat{\rho}(x) = \int_{}^{}{\left\lbrack \dfrac{1}{2}(\delta + 1)s(k) - \dfrac{1}{2}(\delta - 1)s( - k) \right\rbrack\exp(j2\pi kx)dk}\]

Il segnale ricostruito, per la linearità dell'operatore integrale e della trasformata inversa di Fourier, può essere scritto come:

\[\widehat{\rho}(x) = \dfrac{1}{2}(1 + \delta)\widehat{\rho}(x) - \dfrac{1}{2}(1 - \delta)\widehat{\rho}( - x)\]

Nel caso opposto, in cui il canale reale presenta un guadagno \(\delta\), si ottiene una relazione uguale; infatti, il segnale può essere scritto come \(s(k) = \delta s_{R} + js_{I}\), per cui la densità protonica è:

\[\widehat{\rho}(x) = \int_{}^{}{\left( \delta s_{R} + js_{I} \right)\exp(j2\pi kx)dk}\]

Per la proprietà di hermitianità si ha:

\[\widehat{\rho}(x) = \int_{}^{}{\left\lbrack \delta\left( \dfrac{s(k) + s( - k)}{2} \right) + j\dfrac{s(k) - s( - k)}{2j} \right\rbrack\exp(j2\pi kx)dk} = \int_{}^{}{\left\lbrack \delta\left( \dfrac{s(k) + s( - k)}{2} \right) + \dfrac{s(k) - s( - k)}{2} \right\rbrack\exp(j2\pi kx)dk} = \int_{}^{}{\left\lbrack \dfrac{\delta}{2}s(k) + \dfrac{\delta}{2}s( - k) + \dfrac{1}{2}s(k) - \dfrac{1}{2}s( - k) \right\rbrack\exp(j2\pi kx)dk} = \dfrac{\delta}{2}\widehat{\rho}(x) + \dfrac{\delta}{2}\widehat{\rho}( - x) + \dfrac{1}{2}\widehat{\rho}(x) - \dfrac{1}{2}\widehat{\rho}( - x) = \dfrac{1}{2}(1 + \delta)\widehat{\rho}(x) + \dfrac{1}{2}(\delta - 1)\widehat{\rho}( - x)\]

L'immagine ricostruita contiene una componente \((1 + \delta)\widehat{\rho}(x)\), ovvero l'immagine attesa, pesata per una quantità dipendente dal canale con valore diverso di un \(\delta\) e una componente \(\widehat{\rho}( - x)\), ottenuta dal ribaltamento della densità protonica reale e scalata di un fattore \(\delta - 1\). L'effetto finale è la presenza di una ripetizione dell'immagine, ribaltata rispetto l'asse di lettura, di minore ampiezza, sovrapposta all'immagine attesa. Questo fenomeno è noto come effetto ghost e consiste nell'apparizione di porzioni dell'immagini ribaltata e posizionate in luoghi non attesi.

\begin{figure}
\centering
\includegraphics[width=2.66667in,height=2.66667in]{media/10_Ric3D/image269.pdf}\caption{Figura .:Ghosting in un'immagine}
\end{figure}

Questi artefatti sono presenti nell'immagine anche nel caso in cui il paziente si muova durante l'esame. Gli artefatti da movimento si mostrano come ghost sull'immagine.

\begin{figure}
\centering
\includegraphics[width=6.15364in,height=2.13333in]{media/10_Ric3D/image270.pdf}\caption{Figura .: Immagine attesa, artefatto da sbilanciamento dei guadagni e artefatto da movimento}
\end{figure}

La presenza dei ghost non può essere rimossa una volta ricostruita l'immagine, tuttavia, osservando l'artefatto, è possibile discriminare tra un movimento del paziente e lo sbilanciamento del guadagno nei due canali. In quest'ultimo caso è possibile rimuovere la causa scatenante l'artefatto regolando i guadagni dei due canali, ovvero eseguendo interventi di manutenzione o riparazione.

Anche con questo artefatto si preferisce visualizzare l'immagine del modulo. Nella maggior parte dei casi clinici l'immagine di fase non è riportata.

Esistono dei casi, come la spettroscopia o imaging di fase, in cui l'immagine dell'andamento spaziale della fase risulta essere importante. L'immagine di fase permette di verificare la presenza di disomogeneità di campo o problematiche relative alla strumentazione.

\subsection{Esaltazione dei bordi e dei contenuti omogenei}\label{esaltazione-dei-bordi-e-dei-contenuti-omogenei}

Dall'analisi spettrale, è noto che la derivata nel dominio dello spazio-immagine si traduce nel dominio del \(k\)-spazio in una moltiplicazione per \(j2\pi k\), dato che i due domini sono legati tra loro dalla trasformata di Fourier:

\[\dfrac{d}{dx}\rho(x) \rightarrow j2\pi k\ s(k)\]

Questa operazione corrisponde a una riduzione delle basse frequenze e un'esaltazione delle alte. In particolare, alle alte frequenze corrispondono aree di transizione tra i vari distretti anatomici, ovvero, i bordi; mentre le basse frequenze si riferiscono ad aree omogenee dell'immagine, ovvero il parenchima di un organo o tessuto.

L'operazione di derivata, in altre parole, corrisponde a un filtraggio di tipo passa-alto. L'operazione di derivata può essere applicata sia lungo l'asse di lettura sia di codifica di fase in base a ciò che si vuole osservare.

\begin{figure}
\centering
\includegraphics[width=2.81667in,height=2.81667in]{media/10_Ric3D/image271.pdf}\caption{Figura .: Esaltazione dei bordi mediante l'operazione di derivata}
\end{figure}

Per esaltare le componenti continue, ovvero il parenchima di un organo, è necessario esaltare le basse frequenze e sopprimere le componenti ad alte frequenze. Questa operazione può essere realizzata mediante un filtraggio di tipo smoothing a valor medio:

\[\int_{}^{}{\rho(x)dx} \rightarrow \dfrac{1}{j2\pi k}s(k)\]

\begin{figure}
\centering
\includegraphics[width=3.68472in,height=3.68472in,alt={Generazione immagine completata}]{media/10_Ric3D/image272.pdf}\caption{Figura .: Esaltazione delle componenti continue mediante un filtraggio di tipo smoothing a valor medio}
\end{figure}

\subsection{Effetto del campionamento e troncamento}\label{effetto-del-campionamento-e-troncamento}

Le antenne restituiscono un segnale \(s(k)\) continuo, acquisito in una finestra di registrazione con durata finita che, per effetti dei gradienti, deve essere molto minore dei tempi di rilassamento \(T_{1}\), \(T_{2}\) e \(T_{2}^{*}\). Il segnale acquisito, dunque, non coincide con tutto il segnale emesso dal volume-campione ma una sua versione troncata.

Per essere elaborato mediante elaboratori digitali, inoltre, il segnale \(s(k)\) deve essere campionato con un periodo di campionamento \(\Delta t\); ciò equivale a ottenere una serie di punti nel \(k\)-spazio.

I due processi sono estremamente importanti per ottenere una ricostruzione soddisfacente dell'immagine della densità protonica contenuta all'interno del tessuto sotto esame.

Per la sovrapposizione degli effetti è possibile considerare applicato un singolo effetto alla volta; ovvero si analizza il caso in cui il segnale viene campionato ma non troncato e il caso in cui il segnale viene troncato ma non campionato. L'effetto risultante è una combinazione di campionamento e troncamento.

\subsubsection{Effetto del campionamento}\label{effetto-del-campionamento}

Si suppone, dapprima, di campionare il segnale \(s(k)\) in tutto lo spazio \(k\), ovvero che il campionamento nel tempo avvenga da \(( - \infty; + \infty)\). Anche se questo processo non è fisicamente realizzabile, può essere assunto possibile per comodità dimostrativa.

Il segnale, in risonanza magnetica, è spesso acquisito mediante un passo di campionamento uniforme \(\Delta t\), a cui corrisponde un passo di campionamento costante \(\Delta k\), nel \(k\)-spazio. Infatti, fissato un valore del gradiente di lettura \(G_{R}\), il campionamento nella direzione \(k_{R}\) avviene con passo:

\[\Delta k_{R} = \Delta k = \gamma G_{R}\Delta t\]

Il segnale \(s(k)\) nel \(k\)-spazio, campionato con passo \(\Delta k_{R}\), può essere espresso come il segnale stesso, moltiplicato per una funzione di campionamento o pettina di Dirac, data da un treno di impulsi applicati a istanti \(p\), multipli di \(\Delta k_{R}\):

\[u(k) = \Delta k_{R}\sum_{p = - \infty}^{+ \infty}{\delta\left( k - p\Delta k_{R} \right)}\]

Con \(\Delta k_{R}\) costante di spaziamento. Il segnale campionato e con supporto infinito \(s_{\infty}(k)\), può essere scritto come:

\[s_{\infty}(k) = s(k)u(k) = s(k)\ \Delta k_{R}\sum_{p = - \infty}^{+ \infty}{\delta\left( k - p\Delta k_{R} \right)}\]

Per la proprietà di linearità della sommatoria e di campionamento della delta, è possibile scrivere il segnale campionato come:

\[s_{\infty}(k) = \Delta k_{R}\sum_{p = - \infty}^{+ \infty}{s(k)\delta\left( k - p\Delta k_{R} \right)} = \Delta k_{R}\sum_{p = - \infty}^{+ \infty}{s\left( p\Delta k_{R} \right)\delta\left( k - p\Delta k_{R} \right)}\]

Il segnale campionato \(s_{\infty}(k)\) è dato da un treno di impulsi applicati nei multipli interi di \(\Delta k_{R}\) e di area \(\Delta k_{R}s\left( p\Delta k_{R} \right)\).

La trasformata inversa di Fourier del segnale \(s_{\infty}(k)\) permette di ottenere la densità protonica ricostruita \({\widehat{\rho}}_{\infty}(x)\). Per le proprietà della \(\mathfrak{F}\)-trasformata, alla moltiplicazione delle trasformate nel \(k\)-spazio, corrisponde la convoluzione delle due antitrasformate nello spazio-immagine. Si applica la definizione di antitasformata di Fourier:

\[{\widehat{\rho}}_{\infty}(x) = \int_{- \infty}^{+ \infty}{s_{\infty}(k)\exp(j2\pi kx)dk} = \int_{- \infty}^{+ \infty}{\left\lbrack \Delta k_{R}\sum_{p = - \infty}^{+ \infty}{s\left( p\Delta k_{R} \right)\delta\left( k - p\Delta k_{R} \right)} \right\rbrack\exp(j2\pi kx)dk}\]

Per linearità è possibile invertire il simbolo di sommatoria con quello di integrale, si ottiene così:

\[{\widehat{\rho}}_{\infty}(x) = \Delta k_{R}\sum_{p = - \infty}^{+ \infty}\left\{ \int_{- \infty}^{+ \infty}{\left\lbrack s\left( p\Delta k_{R} \right)\delta\left( k - p\Delta k_{R} \right) \right\rbrack\exp(j2\pi kx)dk} \right\} = \Delta k_{R}\sum_{p = - \infty}^{+ \infty}{s\left( p\Delta k_{R} \right)\left\{ \int_{- \infty}^{+ \infty}{\delta\left( k - p\Delta k_{R} \right)\exp(j2\pi kx)dk} \right\}}\]

Per le proprietà della delta, l'integrale si riduce all'esponenziale valutato in \(k = p\Delta k_{R}\):

\[{\widehat{\rho}}_{\infty}(x) = \Delta k_{R}\sum_{p = - \infty}^{+ \infty}{s\left( p\Delta k_{R} \right)\exp\left( j2\pi p\Delta k_{R}x \right)}\]

Questa espressione coincide con una serie infinita di Fourier che rappresenta una prima approssimazione della trasformata inversa di Fourier continua.

Nel caso in cui la densità protonica abbia un supporto finito, la serie produce una serie di infinite repliche della densità protonica, a patto che il periodo di campionamento \(\Delta k_{R}\) sia maggiore della dimensione frequenziale delle repliche. Un periodo di campionamento troppo grande determina la sovrapposizione delle repliche, producendo un fenomeno noto come aliasing.

La densità protonica ricostruita campionando il segnale \(s(k)\) su tutto l'asse \(k_{R}\) può essere espressa anche come convoluzione tra le antitrasformate di \(s_{\infty}(k)\) e \(u(k)\):

\[{\widehat{\rho}}_{\infty}(x) = \mathfrak{F}^{- 1}\left\lbrack s_{\infty}(k) \right\rbrack*\mathfrak{F}^{- 1}\left\lbrack u(k) \right\rbrack\]

Dove \(\mathfrak{F}^{- 1}\left\lbrack s_{\infty}(k) \right\rbrack = \rho(x)\) è la densità protonica effettiva. Per la seconda formula di somma di Poisson, risulta:

\[U(x) = \mathfrak{F}^{- 1}\left\lbrack u(k) \right\rbrack = \sum_{q = - \infty}^{+ \infty}{\delta\left( x - \dfrac{q}{\Delta k_{R}} \right)}\]

La densità protonica ricostruita si può scrivere come:

\[{\widehat{\rho}}_{\infty}(x) = \mathfrak{F}^{- 1}\left\lbrack s_{\infty}(k) \right\rbrack*\mathfrak{F}^{- 1}\left\lbrack u(k) \right\rbrack = \rho(x)*U(x) = \ \rho(x)*\sum_{q = - \infty}^{+ \infty}{\delta\left( x - \dfrac{q}{\Delta k_{R}} \right)}\]

Per l proprietà di campionamento della \(\delta\), è possibile scrivere:

\[{\widehat{\rho}}_{\infty}(x) = \sum_{q = - \infty}^{+ \infty}{\rho(x)*\delta\left( x - \dfrac{q}{\Delta k_{R}} \right)} = \sum_{q = - \infty}^{+ \infty}{\rho\left( \dfrac{q}{\Delta k_{R}} \right)\delta\left( x - \dfrac{q}{\Delta k_{R}} \right)}\]

\begin{figure}
\centering
\includegraphics[width=6.69306in,height=2.04028in,alt={Immagine che contiene diagramma, linea, testo, bianco Il contenuto generato dall\textquotesingle IA potrebbe non essere corretto.}]{media/10_Ric3D/image273.pdf}\caption{Figura .: Somma di Poisson}
\end{figure}

Il campionamento spaziale della densità protonica \(\rho(x)\) avviene con passo \(\Delta k_{R}^{- 1}\). Si definisce:

\[L_{R} = \dfrac{1}{\Delta k_{R}}\]

Il \emph{field of view} o FOV e rappresenta l'intervallo spaziale oltre al quale si presentano le repliche dell'immagine. Il pedice \(R\) indica che il FOV è relativo all'asse di lettura, che nel caso in esame coincide con l'asse \(x\).

Per ricostruire l'immagine si utilizza un apposito filtro interpolatore che estrae la sola replica in banda base. Ovviamente, per poter ricostruire il segnale è necessario che le repliche non si sovrappongono tra loro, dando origine al fenomeno dell'aliasing.

\begin{figure}
\centering
\includegraphics[width=4.10833in,height=4.10833in]{media/10_Ric3D/image274.pdf}\caption{Figura .: Filtro interpolatore per l'estrazione della banda base}
\end{figure}

Sia \(A\) la dimensione dell'oggetto contenuto nell'immagine. Se il FOV è minore delle dimensioni dell'oggetto \(L_{R} = FOV_{R} < A_{R}\), allora parte delle repliche frequenziali si sovrappongono, producendo l'errore di aliasing, rendendo la ricostruzione dell'immagine affetta da errori, soprattutto alle alte frequenze. I contorni sono sfumati a causa della sovrapposizione.

\begin{figure}
\centering
\includegraphics[width=3.49375in,height=3.2284in]{media/10_Ric3D/image275.pdf}\caption{Figura .: Effetto dell'aliasing sull'immagine e sullo spettro}
\end{figure}

Per evitare gli errori da aliasing, è necessario che il FOV sia maggiore della dimensione \(A_{R}\) dell'oggetto, in un discorso monodimensionale, deve risultati, quindi, che:

\[FOV_{R} \geq \dfrac{1}{\Delta k_{R}} \geq A_{R} \Leftrightarrow \Delta k_{R} \leq \dfrac{1}{A_{R}}\]

Questa condizione è nota come criterio di Nyquist per l'asse di lettura. In altre parole, l'intervallo di campionamento nel \(k\)-spazio deve essere minore del reciproco della dimensione dell'immagine.

È noto che il campionamento lungo l'asse di lettura avviene con passo \(\Delta k_{R} = \overline{\gamma}G_{R}\Delta t\), per cui la condizione di Nyquist è:

\[FOV_{R} \geq \dfrac{1}{\Delta k_{R}} \geq A_{R} \Leftrightarrow \Delta k_{R} \leq \dfrac{1}{A_{R}} \Leftrightarrow \overline{\gamma}G_{R}\Delta t \leq \dfrac{1}{A_{R}}\]

Passando ai reciproci:

\[\dfrac{1}{\overline{\gamma}G_{R}\Delta t} \geq A_{R}\]

Si isola \(1/\Delta t\):

\[\dfrac{1}{\Delta t} \geq \overline{\gamma}G_{R}A_{R}\]

Si definisce frequenza di campionamento lungo l'asse di lettura come:

\[f_{R} = \dfrac{1}{\Delta t}\]

Da questa relazione si evince un vincolo sulla frequenza di campionamento:

\[f_{R} \geq \overline{\gamma}G_{R}A_{R}\]

Ovviamente la massima frequenza di campionamento corrisponde alla massima banda che può contenere il segnale di ingresso per non causare aliasing. Si definisce banda di ricezione lungo l'asse di lettura \(BW_{R}\):

\[{BW}_{R} \equiv f_{R} = \overline{\gamma}G_{R}A_{R}\]

La banda di ricezione è nota, poiché dipendente dai parametri della strumentazione di risonanza imposti dall'esterno:

\[\Delta k_{R} = \overline{\gamma}G_{R}\Delta t\]

Per definizione di FOV e frequenza di campionamento\_

\[\dfrac{1}{L_{R}} = \overline{\gamma}G_{R}\dfrac{1}{{BW}_{R}}\ \]

Isolando \({BW}_{R}\), si ottiene:

\[{BW}_{R} = \overline{\gamma}G_{R}L_{R}\]

Dalla relazione \(FOV_{R} \geq A_{R}\), risulta:

\[{BW}_{R} = \overline{\gamma}G_{R}L_{R} \geq \overline{\gamma}G_{R}A_{R}\]

Banda di recezione, FOV e dimensione dell'oggetto sono legate tra loro mediante uguaglianze e disuguaglianze. Se il FOV è minore dell'oggetto-immagine compaiono delle repliche dell'oggetto in posizioni diverse da quelle attese, ovvero dei ghost legati all'aliasing.

Applicando una sequenza di acquisizione per ricostruire un'immagine, non è sufficiente campionare solamente lungo il gradiente di lettura, ma anche verso i gradienti di codifica di fase, \(G_{PE}\), selezione della slice, \(G_{SS}\). Il tempo di questi gradienti è fissato, quindi, il campionamento lungo queste dipende solamente dall'incremento dei gradienti. Per la codifica di fase, l'incremento tra l'applicazione di una sequenza e la successiva è indicato con \(\Delta G_{PE}\). In questo caso, il passo di campionamento nel \(k\)-spazio nella direzione \(PE\) si scrive come:

\[\Delta k_{PE} = \overline{\gamma}\Delta G_{PE}\tau_{PE}\]

Siccome \(G_{PE}\) è fissato dalla strumentazione, è possibile agire su \(\tau_{PE}\).

In questo caso di definisce FOV lungo la direzione di codifica di fase come:

\[{FOV}_{PE} = L_{PE} = \dfrac{1}{\Delta k_{PE}}\]

Anche in questo caso è valida la relazione di Nyquist, quindi il \({FOV}_{PE}\) deve essere maggiore della dimensione dell'oggetto-immagine da visualizzare. Sia \(A_{PE}\)0 la dimensione lineare dell'oggetto, il criterio di Nyquist nella direzione di phase enoding si esprime come:

\[{FOV}_{PE} \geq A_{PE}\]

Il \({FOV}_{PE}\) nella direzione di codifica di fase è legato al tempo di applicazione del gradiente \(\tau_{PE}\), fissato, e all'incremento dell'ampiezza del gradiente \(\Delta G_{PE}\) dalla relazione:

\[{FOV}_{PE} = \dfrac{1}{\overline{\gamma}\Delta G_{PE}\tau_{PE}} \geq A_{PE}\]

Il tempo \(\tau_{PE}\) non può essere troppo lungo per non aumentare il tempo di esecuzione; inoltre, l'ampiezza della variazione del gradiente dovrebbe diminuire conseguentemente.

La frequenza di campionamento, dunque, la massima banda dell'oggetto, in ricezione è:

\[{BW}_{PE} = \dfrac{1}{\Delta G_{PE}} \geq A_{PE}\]

Tipicamente il tecnico radiologo agisce sulla dimensione dei \(FOV\), tramite un apposito software, per risolvere gli oggetti necessari all'indagine clinica, contenuti nell'immagine.

\subsubsection{Effetto del troncamento}\label{effetto-del-troncamento}

La trattazione della ricostruzione dell'immagine in risonanza magnetica si complica introducendo la finestra di acquisizione. Il segnale acquisito non è campionato in un intervallo di tempo illimitato ma finito, generalmente dell'ordine dei \(ms\).

Il troncamento o \emph{windowing} del segnale è modellato matematicamente moltiplicando la funzione campionata in un intervallo di tempo illimitato per una finestra rettangolare di ampiezza \(W\) e centrata in \(\Delta k/2\):

\[s_{m}(k) = s_{\infty}(k)rect\left( \dfrac{k + \dfrac{\Delta k}{2}}{W} \right)\]

\(s_{\infty}(k)\) è dato dalla moltiplicazione del segnale originale per il treno di impulsi di Dirac:

\[s_{m}(k) = s(k)u(k)rect\left( \dfrac{k + \dfrac{\Delta k}{2}}{W} \right)\]

Sia \(N\) il numero degli impulsi contenuti nella finestra di acquisizione, ovvero il numero dei campioni che ricadono nell'intervallo di ampiezza \(W\). Il parametro \(N\) è scelto in modo da essere un numero pari, generalmente multiplo di \(2\), così da rendere pi semplice l'applicazione di algoritmi di ricostruzione mediante trasformata discreta di Fourier (DFT) e fast-fourier transform (FFT), quest'ultimo opera proprio su un numero pari di campioni ed è ancora più efficiente se il numero \(N\) è una potenza di \(2\).

L'ampiezza della finestra rettangolare, fissato il numero di punti, è univocamente determinata dalla spaziatura nel \(k\)-spazio lungo l'asse di lettura, \(\Delta k_{R}\):

\[W = N\Delta k_{R}\]

Se il numero dei punti deve essere pari, allora:

\[W = 2n\Delta k_{R}\]

Siccome i punti sono pari, la finestra di acquisizione non è simmetrica rispetto l'origine, ma lievemente asimmetrica con \(n\) punto prima dell'origine e \(n - 1\) punti dopo.

\begin{figure}
\centering
\includegraphics[width=4.4277in,height=2.08362in,alt={Immagine che contiene linea Il contenuto generato dall\textquotesingle IA potrebbe non essere corretto.}]{media/10_Ric3D/image276.pdf}\caption{Figura .: Finestra rettangolare discreta}
\end{figure}

Per il troncamento, il segnale misurato, \(s_{m}(k)\), è dato da:

\[s_{m}(k) = \Delta k_{R}\left\lbrack \sum_{p = - \infty}^{+ \infty}{s\left( p\Delta k_{R} \right)\delta\left( k - p\Delta k_{R} \right)} \right\rbrack rect\left( \dfrac{k + \dfrac{\Delta k}{2}}{W} \right) = \Delta k_{R}\sum_{p = - n}^{n - 1}{s\left( p\Delta k_{R} \right)\delta\left( k - p\Delta k_{R} \right)}\]

Il segnale misurato è dato da una somma discreta e finita di impulsi \(\delta\) applicati sui multipli interi di \(\Delta k_{R}\) e di area \(s\left( p\Delta k_{R} \right)\ \Delta k_{R}\).

Se la finestra fosse stata simmetrica, allora sarebbero stati acquisiti \(2n + 1\) campioni, rendendo difficile l'applicazione di algoritmi FFT. Oltre a ciò, la definizione della finestra simmetrica porta a uno spostamento dei campioni acquisiti rispetto alla situazione originale in cui è campionato lo zero; ciò porta a non acquisire il valore di picco del segnale trasmesso a \(k = 0\), perdendo così informazioni importanti.

\begin{figure}
\centering
\includegraphics[width=6.69306in,height=2.17222in]{media/10_Ric3D/image277.pdf}\caption{Figura .: Campioni del segnale acquisito con finestra simmetrica e asimmetrica}
\end{figure}

La ricostruzione della densità protonica per un campionamento in una finestra limitata è ottenuta mediante trasformata inversa di Fourier:

\[\widehat{\rho}(x) = \int_{- \infty}^{+ \infty}{s_{m}(k)\exp(j2\pi kx)dk} = \int_{- \infty}^{+ \infty}{\left\lbrack \Delta k_{R}\sum_{p = - n}^{n - 1}{s\left( p\Delta k_{R} \right)\delta\left( k - p\Delta k_{R} \right)} \right\rbrack\exp(j2\pi kx)dk} =\]

Per linearità è possibile invertire il simbolo di integrale con quello di sommatoria:

\[= \ \Delta k_{R}\sum_{p = - n}^{n - 1}\left\lbrack s\left( p\Delta k_{R} \right)\int_{- \infty}^{+ \infty}{\delta\left( k - p\Delta k_{R} \right)\exp(j2\pi kx)dk} \right\rbrack\]

Per la proprietà di campionamento della delta di Dirac, si ha:

\[\widehat{\rho}(x) = \Delta k_{R}\sum_{p = - n}^{n - 1}\left\lbrack s\left( p\Delta k_{R} \right)\exp\left( j2\pi p\Delta k_{R}x \right) \right\rbrack\]

La densità protonica ricostruita \(\widehat{\rho}(x)\) è periodica di periodo \(x = \Delta k_{R}^{- 1}\). Se il FOV è scelto opportunamente, il criterio di Nyquist è rispettato per cui le repliche non si sovrappongono.

Il campionamento finito porta a una stima della densità protonica, ricostruita mediante una serie finita di Fourier. L'effetto del campionamento e del troncamento si manifesta con una riduzione delle alte frequenze, dunque, uno sfocamento dei bordi.

Il campionamento finito nel \(k\)-spazio si traduce nel prodotto di convoluzione tra le antitrasformate del segnale registrato dalle antenne, \(s_{\infty}(k)\), la funzione di campionamento data dai treni di impulsi \(u(k)\) e la finestra rettangolare di troncamento:

\[\widehat{\rho}(x) = \mathfrak{F}^{- 1}\left\lbrack s_{\infty}(k) \right\rbrack*\mathfrak{F}^{- 1}\left\lbrack u(k) \right\rbrack*\mathfrak{F}^{- 1}\left\lbrack rect\left( \dfrac{k + \dfrac{\Delta k}{2}}{W} \right) \right\rbrack\]

Dove l'antitrasformata dalla finestra di troncamento corrisponde a una \(sinc\):

\[\mathfrak{F}^{- 1}\left\lbrack rect\left( \dfrac{k + \dfrac{\Delta k}{2}}{W} \right) \right\rbrack = Wsinc(\pi Wx)\exp{\left( - j2\pi\dfrac{\Delta k}{2}x\  \right)\ }\]

La \(sinc\) presenta il primo zero in \(x = \pm W^{- 1}\).

Nel dominio dello spazio-immagine, la densità protonica ricostruita è data da:

\[\widehat{\rho}(x) = \rho(x)*\sum_{q = - \infty}^{+ \infty}{\delta\left( x - \dfrac{q}{\Delta k_{R}} \right)}*Wsinc(\pi Wx)\exp{\left( - j2\pi\dfrac{\Delta k}{2}x\  \right)\ }\]

La convoluzione della densità protonica reale \(\rho(x)\) col treno di impulsi restituisce la densità protonica ricostruita nel caso in cui il campionamento non sia troncato \({\widehat{\rho}}_{\infty}(x)\):

\[\widehat{\rho}(x) = {\widehat{\rho}}_{\infty}(x)*Wsinc(\pi Wx)\exp{\left( - j2\pi\dfrac{\Delta k}{2}x\  \right)\ }\]

La convoluzione del segnale campionato in una finestra illimitata per la funzione \(sinc\) sfoca i contorni dell'immagine. Infatti, poiché la \(sinc\) non è limitata, le varie repliche interagiscono tra loro, portando a delle distorsioni nei confini dell'oggetto.

\begin{figure}
\centering
\includegraphics[width=5.9908in,height=7.20351in]{media/10_Ric3D/image278.pdf}\caption{Figura .: Sfocamento dovuto al campionamento e al troncamento}
\end{figure}

Tutte le immagini ricostruite presentano questa limitazione dovuta al campionamento finito, tuttavia, lo sfocamento tende a essere trascurabile per un troncamento opportuno; ovvero per una finestra rettangolare di ampiezza \(W\) opportuna. Infatti, a una finestra rettangolare sufficientemente ampia, corrisponde, nel dominio-immagine, a una \(sinc\) caratterizzata da un lobo principale di durata molto limitata, che ben approssima il comportamento di un impulso. Al limite per \(W \rightarrow \infty\), il primo zero della \(sinc\) si sposta verso \(1/W \rightarrow 0\).

\subsection{Valutazione della banda di ricezione}\label{valutazione-della-banda-di-ricezione}

Si suppone che la dimensione dell'oggetto lungo l'asse di lettura sia di \(0.5\ m\), gradiente di lettura \(G_{R} = 10\ mT/m\), con un massimo, nella pratica di \(50\ mT/m\). Si vuole determinare la banda di ricezione minima del segnale registrato \(s(k)\).

È noto che la banda minima in ricezione è data da:

\[BW_{R} = f_{R} = \overline{\gamma}G_{R}L_{R}\]

Dove \(L_{R} \geq A_{R}\) dimensione dell'oggetto di cui si vuole ricostruire l'immagine. Risulta:

\[f_{R} \geq \overline{\gamma}G_{R}A_{R}\]

Dove \(\overline{\gamma} = 42.6\ MHz/T\). Facendo valere il simbolo di uguaglianza, risulta:

\[f_{R} = \overline{\gamma}G_{R}A_{R} = 42.6\dfrac{MHz}{T}10\dfrac{mT}{m}0.5\ m = 42.6 \cdot 10^{6}\dfrac{Hz}{T}10 \cdot 10^{- 3}\dfrac{T}{m}0.5\ m = 213\ kHz\]

Se il gradiente di lettura raddoppia, anche la banda minima del senale raddoppia:

\[f_{R} = \overline{\gamma}2G_{R}A_{R} = 42.6 \cdot 10^{6}\dfrac{Hz}{T}2 \cdot 10 \cdot 10^{- 3}\dfrac{T}{m}0.5\ m = 426\ kHz\]

Nella pratica le frequenze utilizzate per la banda di ricezione hanno ordine di grandezza del centinaio di \(kHz\), contro la frequenza di una decina di \(MHz\) del segnale trasmesso.

La massima banda di ricezione dipende essenzialmente dal gradiente di lettura, poiché il rapporto giromagnetico \(\overline{\gamma}\) è costante, mentre la dimensione dell'oggetto da visualizzare, \(A_{R}\), è una quantità fissata dal tecnico radiologo in base all'esame da effettuare, dunque, in base al distretto anatomico da visualizzare.

\subsection{Descrete Fourier Transform}\label{descrete-fourier-transform}

Date le due sequenze segnale acquisito \(s(p\Delta k)\) e densità protonica ricostruita \(\widehat{\rho}(q\Delta x)\), esse sono collegate da una coppia di antitrasformate di Fourier inverse o indirette, se il numero di punti con cui sono campionate è lo stesso. Inoltre, detta \(\widetilde{U}(x)\) la funzione campionatrice nel dominio dello spazio-immagine:

\[\widetilde{U}(x) = \Delta x\sum_{q = - \infty}^{+ \infty}{\delta(x - q\Delta x)}\]

Questa è limitata dalla finestra rettangolare di durata \(L\), centrata in \(\Delta x/2\):

\[rect\left( \dfrac{x + \dfrac{1}{2}\Delta x}{L} \right)\]

La \(rect\) è la funzione finestratura nella quale si campiona la densità protonica spazio-immagine, la quale può essere scritta come prodotto tra la funzione finestra e la funzione campionatrice:

\[{\widehat{\rho}}_{m}(x) = \widehat{\rho}(x)\widetilde{U}(x)\ rect\left( \dfrac{x + \dfrac{1}{2}\Delta x}{L} \right)\]

Si esplicita il termine di campionamento:

\[{\widehat{\rho}}_{m}(x) = \widehat{\rho}(x)\Delta x\sum_{q = - \infty}^{+ \infty}{\delta(x - q\Delta x)}rect\left( \dfrac{x + \dfrac{1}{2}\Delta x}{L} \right)\]

Si suppone che nella finestra di acquisizione, entrino \(N = 2n\) campioni del segnale densità protonica. Il segnale misurato si può scrivere come:

\[{\widehat{\rho}}_{m}(x) = \widehat{\rho}(x)\Delta x\sum_{q = - n}^{n + 1}{\delta(x - q\Delta x)}\]

Per la linearità dell'operatore sommatoria, è possibile scrivere:

\[{\widehat{\rho}}_{m}(x) = \Delta x\sum_{q = - n}^{n + 1}{\widehat{\rho}(x)\delta(x - q\Delta x)}\]

Per la proprietà di campionamento della delta di Dirac, è possibile scrivere:

\[{\widehat{\rho}}_{m}(x) = \Delta x\sum_{q = - n}^{n + 1}{\widehat{\rho}(q\Delta x)\delta(x - q\Delta x)}\]

La densità protonica \({\widehat{\rho}}_{m}(x)\) è l'antitrasformata del segnale \(\widehat{s}(k)\) registrato nel \(k\)-spazio, ricostruito nota, appunto, la serie \({\widehat{\rho}}_{m}(x)\):

\[\widehat{s}(k) = \int_{}^{}{{\widehat{\rho}}_{m}(x)\exp( - j2\pi kx)dx}\]

Si sostituisce l'espressione per \({\widehat{\rho}}_{m}(x)\):

\[\widehat{s}(k) = \int_{}^{}{\left\lbrack \Delta x\sum_{q = - n}^{n + 1}{\widehat{\rho}(q\Delta x)\delta(x - q\Delta x)} \right\rbrack\exp( - j2\pi kx)dx}\]

Per la linearità degli operatori sommatoria e integrale è possibile scrivere:

\[\widehat{s}(k) = \Delta x\sum_{q = - n}^{n + 1}{\widehat{\rho}(q\Delta x)\int_{}^{}{\delta(x - q\Delta x)\exp( - j2\pi kx)dx}}\]

Per la proprietà di campionamento dell'impulso di Dirac, si ha:

\[\widehat{s}(k) = \Delta x\sum_{q = - n}^{n + 1}{\widehat{\rho}(q\Delta x)\exp( - j2\pi kq\Delta x)}\]

Il segnale nel \(k\)-spazio è campionato con passo \(\Delta k\), per cui \(\widehat{s}(k)\) è noto solamente sui multipli interi di \(k = r\Delta k\):

\[\widehat{s}(r\Delta k) = \Delta x\sum_{q = - n}^{n + 1}{\widehat{\rho}(q\Delta x)\exp( - j2\pi r\Delta kq\Delta x)}\]

La densità protonica ricostruita \(\widehat{\rho}(q\Delta x)\) è legata al segnale effettivamente registrato \(s(k)\) dall'antitrasformata di Fourier:

\[\widehat{\rho}(q\Delta x) = \Delta k\sum_{p = - n}^{n - 1}\left\lbrack s(p\Delta k)\exp(j2\pi p\Delta kq\Delta x) \right\rbrack\]

Da cui si ottiene:

\[\widehat{s}(r\Delta k) = \Delta x\sum_{q = - n}^{n + 1}{\left\{ \Delta k\sum_{p = - n}^{n - 1}\left\lbrack s(p\Delta k)\exp(j2\pi p\Delta kq\Delta x) \right\rbrack \right\}\exp( - j2\pi r\Delta kq\Delta x)}\]

Per la linearità della sommatoria si ha:

\[\widehat{s}(r\Delta k) = \Delta x\Delta k\sum_{q = - n}^{n + 1}\left\{ \sum_{p = - n}^{n - 1}\left\lbrack s(p\Delta k)\exp(j2\pi p\Delta kq\Delta x)\exp( - j2\pi r\Delta kq\Delta x) \right\rbrack \right\}\]

Da cui:

\[\widehat{s}(r\Delta k) = \Delta x\Delta k\sum_{q = - n}^{n + 1}\left\{ \sum_{p = - n}^{n - 1}\left\lbrack s(p\Delta k)\exp\left( j2\pi q(p - r)\Delta k\Delta x \right) \right\rbrack \right\}\]

Per definizione il FOV è:

\[FOV \equiv L = \dfrac{1}{\Delta k}\]

Da cui:

\[\Delta k = \dfrac{1}{L}\]

Il FOV coincide con l'estensione della finestra lungo la direzione di lettura:

\[L = 2n\Delta x \Leftrightarrow \Delta x = \dfrac{L}{2n}\]

Il prodotto \(\Delta k\Delta x\) può, quindi, essere scritto come:

\[\Delta k\Delta x = \dfrac{1}{L}\dfrac{L}{2n} = \dfrac{1}{2n}\]

Questa relazione, in una sorta di analogia col principio di indeterminazione di Heisenberg, determina una relazione di inversa proporzionalità tra il campionamento nel \(k\)-spazio e nel dominio dello spazio-immagine. Nel dettaglio, maggiore è il FOV e minore risoluzione, e viceversa.

Il segnale ricostruito, sostituendo \(\Delta x = L^{- 1}\) e \(\Delta k\Delta x = (2n)^{- 1}\) è:

\[\widehat{s}\left( \dfrac{r}{L} \right) = \dfrac{1}{2n}\sum_{q = - n}^{n + 1}\left\{ \sum_{p = - n}^{n - 1}\left\lbrack s\left( \dfrac{p}{L} \right)\exp\left( j2\pi q\dfrac{p - r}{2n} \right) \right\rbrack \right\}\]

Per la linearità, si ha:

\[\widehat{s}\left( \dfrac{r}{L} \right) = \dfrac{1}{2n}\sum_{q = - n}^{n + 1}\left\{ s\left( \dfrac{p}{L} \right)\sum_{p = - n}^{n - 1}\left\lbrack \exp\left( j2\pi q\dfrac{p - r}{2n} \right) \right\rbrack \right\}\]

È noto che

\[\sum_{p = - n}^{n - 1}\left\lbrack \exp\left( j2\pi q\dfrac{p - r}{2n} \right) \right\rbrack = 2n\delta_{pr}\]

Dove \(\delta_{pr}\) è la delta di Kronecker, definita come:

\[\delta_{pr} = \left\{ \begin{aligned}
1,\ \  & p = r \\
0,\ \  & p \neq r
\end{aligned} \right.\ \]

Il segnale ricostruito nel \(k\)-spazio è:

\[\widehat{s}\left( \dfrac{r}{L} \right) = \dfrac{1}{2n}2n\sum_{q = - n}^{n + 1}\left\lbrack s\left( \dfrac{p}{L} \right)\delta_{pr} \right\rbrack = \sum_{q = - n}^{n + 1}\left\lbrack s\left( \dfrac{p}{L} \right)\delta_{pr} \right\rbrack\]

Il segnale ricostruito a partire dai campioni dello spazio-immagine è uguale, punto per punto, al segnale originale campionato nel \(k\)-spazio. Dalla relazione appena individuata si evince che il segnale registrato \(s(pk)\) e densità protonica ricostruita \(\widehat{\rho}(qL/2n)\) sono legate da una coppia di trasformate di Fourier discrete:

\[\left\{ \begin{matrix}
s\left( \dfrac{p}{L} \right) = \Delta x\sum_{q = - n}^{n - 1}{\widehat{\rho}\left( q\dfrac{L}{2n} \right)\exp\left( - j2\pi\dfrac{pq}{2n} \right)} \\
s\left( q\dfrac{L}{2n} \right) = \Delta k\sum_{q = - n}^{n - 1}{s\left( \dfrac{p}{L} \right)\exp\left( j2\pi\dfrac{pq}{2n} \right)}
\end{matrix} \right.\ \]

La trasformata discreta di Fourier del segnale registrato \(s(p/L)\) è proporzionale alla densità protonica fisicamente presente in un voxel dell'immagine.

Dalla relazione \(\Delta x\Delta k = (2n)^{- 1}\) discende che le due quantità non possono essere piccole a piacere. In particolare, \(\Delta k\) regola il FOV, ovvero la dimensione dell'immagine, mentre \(\Delta x\) l'ampiezza del campionamento nel dominio dell'immagine.

Il passo di campionamento nello spazio-immagine \(\Delta x\) è detto Fourier Pixel Size e rappresenta la migliore risoluzione possibile senza utilizzare particolari algoritmi o filtri.

Il campionamento e il troncamento, in definitiva, introducono degli artefatti che riducono la risoluzione. La relazione:

\[\Delta x = \dfrac{1}{2n\Delta k}\]

Rappresenta, dunque, la risoluzione nel caso migliore possibile ed è legata al campionamento nel \(k\)-spazio e al numero di punti acquisiti nella finestra rettangolare.

\subsection{Valutazione dei parametri frequenziali tecnici}\label{valutazione-dei-parametri-frequenziali-tecnici}

Si suppone che l'imaging bidimensionale sia caratterizzato dai parametri \(L_{R} = L_{PE} = 256\ mm\), \(N_{R} = N_{PE} = 256\), \(TH = 5\ mm\) e \(T_{R} = 600\ ms\). Si assume che \(\widehat{x}\), \(\widehat{y}\) e \(\widehat{z}\) sono, rispettivamente, l'asse di lettura, di codifica di fase e di \emph{slice selction}; la finestra di acquisizione abbia durata \(T_{S} = 5.12\ ms\) e \(\tau_{PE} = 2.56\ ms\) e una banda di eccitazione a radiofrequenza di \(2\ kHz\).

Si vuole determinare:

\begin{enumerate}
\def\labelenumi{\alph{enumi})}
\item
  La banda di ricezione \(BW_{R}\), l'intervallo di campionamento \(\Delta t_{R}\) di Nyquist nella direzione di lettura, l'intervallo di campionamento di Nyquist nella direzione \(k_{R}\) e il valore del gradiente di lettura;
\item
  L'intervallo di campionamento di Nyquist nella direzione di codifica di fase \(\Delta k_{PE}\), le variazioni del gradiente in step successivi, \(\Delta G_{PE}\) e il suo valore massimo nella direzione di codifica di fase;
\item
  Il gradiente di \emph{slice selection} usato;
\item
  Cosa accade se il numero di punti lungo la direzione di lettura \(N_{R}\) e di codifica di fase \(N_{PE}\) raddoppiano mentre lo spessore della slice viene dimezzato, mantenendo costante gli altri parametri;
\item
  Come cambia il gradiente di lettura \(G_{R}\), quando \(L_{R}\) è dimezzato, mentre gli altri parametri sono mantenuti invariati. Se \(T_{S} = 2.56\ ms\). \(L_{R} = 256\ mm\) e \(N_{R} = 256\) cisa accade a \(G_{R}\).
\end{enumerate}

Supponendo che l'immagine sia acquisita con una metodologia tridimensionale con \(L_{SS} = 32\ mm\) e \(N_{ss} = 16\), si vuole determinare:

\begin{enumerate}
\def\labelenumi{\roman{enumi})}
\item
  \(G_{SS}\);
\item
  \(\Delta G_{z}\) e il valore massimo di tale gradiente;
\item
  Il tempo totale di acquisizione se \(T_{R} = 600\ ms\) e \(T_{R} = 60\ ms\)
\item
  Confrontare \(G_{SS}\) nei due casi.
\end{enumerate}

La banda di ricezione minima \(BW_{R}\)è data da:

\[BW_{R} = \overline{\gamma}G_{R}L_{R}\]

Inoltre, per definizione, la banda passante è legata al passo di campionamento lungo la direzione di lettura dalla relazione:

\[BW_{R} = \dfrac{1}{\Delta t_{R}}\]

Il tempo di campionamento può essere ottenuto semplicemente dividendo l'ampiezza della finestra di acquisizione per il numero di punti memorizzati in tale intervallo temporale, lungo l'asse di lettura:

\[\Delta t_{R} = \dfrac{T_{S}}{N_{R}}\]

La banda di ricezione minima è, dunque:

\[BW_{R} = \dfrac{1}{\Delta t_{R}} = \dfrac{N_{R}}{T_{S}} = \dfrac{256}{5.12 \cdot 10^{- 3}\ s} = 50\ kHz\]

L'intervallo di campionamento è, invece:

\[\Delta t_{R} = \dfrac{T_{S}}{N_{R}} = \dfrac{10^{- 3}s}{256} = 0.02\ ms\]

Il gradiente lungo la direzione di lettura necessario per ottenere la banda di ricezione minima individuata si ottiene dalla relazione:

\[BW_{R} = \overline{\gamma}G_{R}L_{R} \Leftrightarrow G_{R} = \dfrac{BW_{R}}{\overline{\gamma}L_{R}}\]

Numericamente, si scrive:

\[G_{R} = \dfrac{BW_{R}}{\overline{\gamma}L_{R}} = \dfrac{50 \cdot 10^{3}Hz}{42.6 \cdot 10^{6}\dfrac{Hz}{T} \cdot 256 \cdot 10^{- 3}m} = 4.58\dfrac{mT}{m}\]

L'intervallo di campionamento lungo l'asse di lettura è:

\[\Delta k_{R} = \dfrac{1}{L_{R}} = \dfrac{1}{256\ mm} = 3.9\ m^{- 1}\]

Analogamente, per l'asse di codifica di fase, il campionamento avviene con passo:

\[\Delta k_{PE} = \dfrac{1}{L_{PE}} = \dfrac{1}{256\ mm} = 3.9\ m^{- 1}\]

La variazione del gradiente lungo l'asse di codifica di fase, tra una sequenza e l'altra, è dato da:

\[\Delta k_{PE} = \overline{\gamma}\Delta G_{PE}\tau_{PE} \Leftrightarrow \ \Delta G_{PE} = \dfrac{\Delta k_{PE}}{\overline{\gamma}\tau_{PE}}\]

Numericamente:

\[\Delta G_{PE} = \dfrac{\Delta k_{PE}}{\overline{\gamma}\tau_{PE}} = \dfrac{3.9\dfrac{1}{m}}{42.6 \cdot 10^{6}\dfrac{Hz}{T} \cdot 256 \cdot 10^{- 3}m} = 0.036\dfrac{mT}{m}\]

Siccome la sequenza va ripetuta \(N_{PE}\) volte, il gradiente è applicato tra un valore \(- \Delta G_{\min}\) a \(\Delta G_{\max}\), passando per lo \(0\). Nell'ipotesi che \(G_{PE}\) assuma \(N/2\) valori prima dello zero e dopo, è possibile scrivere:

\[\Delta G_{\max} = \Delta G_{PE}\dfrac{N_{PE}}{2}\]

Numericamente, si scrive:

\[\Delta G_{\max} = \Delta G_{PE}\dfrac{N_{PE}}{2} = 0.036 \cdot 10^{- 3}\dfrac{T}{m} \cdot \dfrac{256}{2} = 4.57\dfrac{mT}{m}\]

Il gradiente di selezione della fetta è legato allo spessore della fetta, \(TH\), tramite la relazione:

\[TH = \dfrac{BW_{rf}}{\overline{\gamma}G_{SS}} \Leftrightarrow G_{SS} = \dfrac{BW_{rf}}{\overline{\gamma}TH}\]

Numericamente, si ha:

\[G_{SS} = \dfrac{BW_{rf}}{\overline{\gamma}TH} = \dfrac{2\ kHz}{42.6 \cdot 10^{6}\dfrac{Hz}{T} \cdot 3 \cdot 10^{- 3}\ m} = 9.38\dfrac{mT}{m}\]

Il tempo necessario per acquisire una singola fetta è dato dall'intervallo temporale in cui è attivata la finestra di ricezione \(T_{S}\) e il tempo tra due ripetizioni \(T_{R}\) da una sequenza e la successiva. Numericamente, \(T_{R} = 600\ ms\) mentre \(T_{S} = 5.12\ ms\). È chiaro che \(T_{R} \gg T_{S}\), quindi, l'intervallo in cui si preleva il segnale può essere trascurato. In altre parole, si soppone che l'asse di lettura sia acquisito istantaneamente rispetto all'asse di codifica di fase.

Per acquisire \(N_{PE}\) punti sull'asse di codifica di fase è dato da:

\[T_{acq} = N_{PE}T_{R} = 256 \cdot 600 \cdot 10^{- 3}\ s = 154.6\ s\]

L'acquisizione di una singola avviene in un tempo di circa \(2.6\ min\).

Se si raddoppiano \(N_{R}\) e \(N_{PE}\) e si dimezza \(TH\), a parità di tutti gli altri parametri, dalla relazione:

\[G_{R} = \dfrac{BW_{R}}{\overline{\gamma}L_{R}}\]

Per definizione di \(BW_{R} = N_{R}/T_{S}\), si ha:

\[G_{R} = \dfrac{N_{R}}{\overline{\gamma}L_{R}T_{S}}\]

Se il numero di punti campionato lungo la direzione di lettura, \(N_{R}\), raddoppia, anche il gradiente \(G_{R}\) raddoppia.

Il massimo gradiente lungo l'asse di codifica di fase è dato da:

\[G_{PE,max} = \Delta G_{PE}\dfrac{N_{PE}}{2}\]

Se raddoppia il numero di punti \(N_{PE}\) anche il massimo gradiente applicato lungo la direzione di codifica di fase è raddoppia.

Dimezzando \(TH\) invece, \(G_{SS}\) raddoppia poiché i due parametri sono legati da una legge di inversa proporzionalità:

\[G_{SS} = \left. \ \dfrac{BW_{rf}}{\overline{\gamma}TH} \right|_{TH = \dfrac{TH}{2}} = 2\dfrac{BW_{rf}}{\overline{\gamma}TH}\]

Se si dimezza la dimensione dell'oggetto \(L_{R}\), il gradiente di lettura raddoppia, infatti le due quantità sono inversamente proporzionate:

\[G_{R} = \left. \ \dfrac{BW_{R}}{\overline{\gamma}L_{R}} \right|_{L_{R} = \dfrac{L_{R}}{2}} = 2\dfrac{BW_{R}}{\overline{\gamma}L_{R}}\]

Il tempo di acquisizione della finestra \(T_{S}\) è legato agli altri parametri lungo la direzione di lettura dalla relazione:

\[G_{R} = \dfrac{N_{R}}{\overline{\gamma}L_{R}T_{S}}\]

Aumentando \(T_{S}\), il gradiente lungo \(G_{R}\) si riduce per inversa proporzionalità. Numericamente:

\[G_{R} = \dfrac{N_{R}}{\overline{\gamma}L_{R}T_{S}} = \dfrac{256}{42.6 \cdot 10^{6}\dfrac{Hz}{T} \cdot 256 \cdot 10^{- 3}m \cdot 2.56 \cdot 10^{- 3}s} = 9.2\dfrac{mT}{m}\]

Il valore del gradiente di selezione della fetta è dato da:

\[G_{SS} = \dfrac{BW_{rf}}{\overline{\gamma}TH} = \dfrac{2\ kHz}{42.6 \cdot 10^{6}\dfrac{Hz}{T}36 \cdot 10^{- 3}m} = 1.3\dfrac{mT}{m}\]

Il passo di campionamento lungo l'asse di \emph{slice selection} è dato da:

\[\Delta k_{SS} = \overline{\gamma}\Delta G_{SS}\tau_{SS} \Leftrightarrow \ \Delta G_{SS} = \dfrac{\Delta k_{SS}}{\overline{\gamma}\tau_{SS}}\]

Dove \(\Delta k_{SS} = 1/L_{SS}\), ovvero:

\[\Delta k_{SS} = \dfrac{1}{L_{SS}} = \dfrac{1}{32 \cdot 10^{- 3}m} = 31.3\ m^{- 1}\]

Noto questo parametro è possibile ricavare l'incremento del gradiente di selezione della fetta tra una sequenza e la successiva:

\[\ \Delta G_{SS} = \dfrac{\Delta k_{SS}}{\overline{\gamma}\tau_{SS}} = \dfrac{31.3\ m^{- 1}}{42.6 \cdot 10^{6}\dfrac{Hz}{T}256 \cdot 10^{- 3}m} = 2.87\dfrac{\mu T}{m}\]

Il gradiente massimo applicato lungo la direzione di selezione della fetta è dato da:

\[G_{SS,max} = \Delta G_{SS}N_{SS} = 16 \cdot 2.87\dfrac{\mu T}{m} = 45.9\dfrac{\mu T}{m}\]

Il tempo di acquisizione deve tener conto, oltre ai \(N_{PE}\) punti lungo l'asse di codifica di fase, anche gli \(N_{SS}\) punti lungo l'asse di selezione della fetta. Per un tempo di ripetizione di \(T_{R} = 600\ ms\), si ha un tempo:

\[T_{acq} = T_{R}N_{SS}N_{PE} = 600\ ms \cdot 16 \cdot 256 \simeq 2458\ s \simeq 41\ min\]

Il tempo di acquisizione in caso di imaging 3D è molto maggiore del tempo necessario per ottenere una singola fetta. In questo caso l'esame di risonanza magnetica, essendo molto più lungo, ha una maggiore predisposizione alla presenza di artefatti da movumento.

Se il tempo di ripetizione fosse di \(60\ ms\) potrebbe non essere raggiunto l'equilibrio termodinamico con il campo magnetico principale. Di conseguenza non si ottiene il massimo segnale possibile, riducendo di molto il rapporto segnale/rumore. Il tempo di acquisizione, nel caso di \(T_{R} = 60\ ms\), è:

\[T_{acq} = 60\ ms \cdot 256 \cdot 16 \simeq 246 \simeq 4\ min\]

Grazie a questo esempio è possibile avere una maggiore confidenza con gli ordini di grandezza utilizzati in risonanza magnetica.

\subsubsection{Corretto campionamento}\label{corretto-campionamento}

Il campionamento del \(k\)-spazio pone dei limiti sulla collezione dei dati relativi alla direzione di lettura, codifica di fase e selezione della fetta. Ciò determina solo una parziale convergenza del \(k\)-spazio comportando errori nella valutazione e ricostruzione della densità protonica tramite la trasformata di Fourier. Tale problematica è detto limite dell'inversa di Fourier.

Nel caso monodimensionale, l'immagine ricostruita è ottenuta come:

\[{\widehat{\rho}}_{m}(x) = \Delta x\sum_{p = - n}^{n + 1}{s(p\Delta k)\exp(j2\pi kp\Delta kx)}\]

La trasformata inversa di Fourier è ottenuta come somma di \(2n\) termini a causa del troncamento. Il passo di campionamento \(\Delta k\) è legato all'ampiezza della finestra \(W\) e al numero di punti \(N\) dalla relazione:

\[W = N\Delta k\]

Siccome, per il teorema di Nyquist applicato lungo l'asse di lettura, il FOV deve essere maggiore della dimensione degli oggetti dell'immagine:

\[L \geq A,\Delta k = \dfrac{1}{L}\]

Risulta:

\[\Delta k \leq \dfrac{1}{A}\]

Da cui discende che la finestra di acquisizione deve essere più piccola dell'inverso della dimensione dell'oggetto, moltiplicata per \(N\):

\[W = N\Delta k \Leftrightarrow W \leq \dfrac{N}{A}\]

\subsubsection{Banda di ricezione dell'imaging}\label{banda-di-ricezione-dellimaging}

L'imaging della densità protonica può essere sia bidimensionale, mediante l'applicazione del gradiente di selezione della fetta, oppure tridimensionale, eccitando l'intero volume paziente. Nell'imaging bidimensionale, la banda di ricezione può influenzare lo spessore della fetta selezionata; il FOV, infatti, è legato alla banda minima sull'asse di lettura dalla relazione:

\[BW_{R} = \overline{\gamma}G_{R}L_{R}\]

È possibile controllare la porzione di oggetto da cui si vuole ricavare il segnale di risonanza magnetica, proveniente dal corpo del paziente. Dato un oggetto di dimensione \(L_{R}\) lungo la direzione di lettura e applicati dei gradienti, il campo magnetico varia linearmente lungo la direzione di applicazione del gradiente:

\[B_{z}(z) = B_{0} + G_{R}x\]

Si instaura una corrispondenza biunivoca tra la frequenza di precessione degli isocromati e la loro posizione. Nel sistema di riferimento rotante, infatti, si ha che la differenza di frequenza di precessione \(\Delta\omega\) è legata al gradiente applicato dalla relazione:

\[\Delta\omega = \gamma\Delta B = \gamma G_{R}x\]

Agendo sulla banda di frequenza ricevuta è possibile controllare la porzione di volume-paziente da cui ricevere il segnale. In particolare, dato che la banda minima di ricezione \(BW_{R}\) è legato al \(FOV = L_{R}\), regolando la banda di ricezione sul gradiente di lettura si determina automaticamente il \(FOV\). Scelta una certa banda, \(BW_{R}\), si riceve solamente il segnale da un determinato distretto anatomico, trascurando le altre regioni con frequenze maggiori o minori.

La banda di ricezione è correlata con la banda di eccitazione dell'impulso a radiofrequenza; infatti, se la banda di eccitazione \(BW_{RF}\) è, ad esempio, tale da stimolare l'interno volume, anche le regioni esterne alla banda di ricezione trasmetteranno un certo segnale. Nella finestra di acquisizione oltre al segnale di interesse, relativo a posizioni specifica del corpo imano, riceve anche delle interferenze provenienti da frequenze positive, per isormati che si trovano nella direzione del gradiente di lettura positivi, e negative per isocromati posizionati nel verso opposto.

\begin{figure}
\centering
\includegraphics[width=5.38701in,height=3.21667in]{media/10_Ric3D/image279.pdf}\caption{Figura .: Segnale ricevuto comprendente la zona sovra-campionata e sotto-campionata}
\end{figure}

Se la banda di ricezione è leggermente slargata in frequenza, si selezionano porzioni più ampie del volume. Se il passo di campionamento \(\Delta t\) è fissato per quella determinata banda di ricezione, i segnali che provengono da regioni esterne saranno sottocampionati se provengono da regioni a frequenza maggiore di \(BW_{R}\) o sovracampionati se provengono da regioni a frequenza minore.

I campioni prelevati appartengono sia ai segnali fuori banda sovra e sotto campionati, sia al segnale proveniente dalla sezione del volume desiderato. A causa del sottocampionamento, si verifica il fenomeno dell'aliasing.

Non basta agire sulla frequenza di campionamento ma è necessario adoperare un opportuno filtraggio nel momento in cui si sceglie di eccitare l'interno volume-paziente. In questo modo si evita la comparsa del ghost dovuto all'aliasing.

La soluzione più semplice a tale problematica è quella di eccitare con un impulso a radiofrequenza solamente le porzioni del volume-paziente dalla quale si preleva il segnale.

Per evitare l'aliasing, l'eccitazione e la lettura devono essere coordinati dal punto di vista frequenziale. In particolare, la frequenza dell'impulso di eccitazione deve essere tale da avere la stessa banda del sistema di ricezione, così da non eccitare regioni esterne alla porzione di volume di cui si vuole eseguire l'imaging.

Questa soluzione non è adottabile in tutte le circostanze, ad esempio, nelle metodiche spettroscopiche è necessario eccitare completamente l'interno volume-paziente. In questi casi, senza un opportuno filtraggio è possibile avere effetti di aliasing per interferenze tra fette vicine, che determinano la presenza di ghost.

Dov'è possibile, nell'imaging bidimensionale si fa in modo di avere la banda di eccitazione quanto più vicina alla banda di ricezione alla banda di ricezione così da ridurre al minimo le interferenze dovute alle slice vicine.

Nella maggior parte dei casi pratici la banda di eccitazione non è perfettamente rettangolare ma slargata in frequenza; quindi, sono possibili piccole interferenze tra slice vicine.

\subsection{Smoothing in risonanza magnetica}\label{smoothing-in-risonanza-magnetica}

Il troncamento e il campionamento, dal punto di vista analitico, possono essere espressi come la convoluzione tra la finestra rettangolare e il treno di impulso di campionamento:

\[{\widehat{\rho}}_{m}(x) = {\widehat{\rho}}_{\infty}(x)*U(x)*Wsinc(\pi Wx)\exp{\left( - j2\pi\dfrac{\Delta k}{2}x\  \right)\ }\]

Nel \(k\)-spazio la relazione può essere espressa come:

\[s_{m}(k) = s_{\infty}(k)u(k)rect\left( \dfrac{k + \dfrac{\Delta k}{2}}{W} \right)\]

Dove \(W = 2n\Delta k = \Delta x^{- 1}\). Nel \(k\)-spazio questa operazione corrisponde a un filtraggio a opera della funzione di trasferimento \(H(k)\), definita come:

\[H(k) = u(k)rect\left( \dfrac{k + \dfrac{\Delta k}{2}}{W} \right)\]

Nello spazio-immagine la funzione \(H(k)\) è identificata con la \emph{point spread function} o PSF. La funzione rettangolare corrisponde a un filtraggio detto con \emph{windowed} e indicato con \(H_{W}\):

\[H_{W}(k) = rect\left( \dfrac{k + \dfrac{\Delta k}{2}}{W} \right)\]

Il treno di impulsi corrisponde a un filtraggio detto sampling e indicato con \(H_{S}\):

\[H_{S}(k) = u(k) = \Delta k\sum_{p = - \infty}^{+ \infty}{\delta(k - p\Delta k)}\]

I due contributi offrono insieme un comportamento del tipo \emph{windowed} e \emph{sampling}:

\[H_{WS}(k) = \Delta k\sum_{p = - \infty}^{+ \infty}{\delta(k - p\Delta k)}rect\left( \dfrac{k + \dfrac{\Delta k}{2}}{W} \right)\]

Nel \(k\)-spazio il segnale misurato può essere espresso come:

\[s_{m}(k) = s_{\infty}(k)H_{WS}(k)\]

Antitasformando si ottiene la densità protonica nello spazio immagine:

\[\widehat{\rho}(x) = \int_{- \infty}^{+ \infty}{s_{m}(k)\exp(j2\pi kx)dk}\]

Il segnale nel \(k\)-spazio è dato da:

\[\widehat{\rho}(x) = \int_{- \infty}^{+ \infty}{s_{\infty}(k)H_{WS}(k)\exp(j2\pi kx)dk}\]

Sostituendo l'espressione del filtro \emph{windowed} e \emph{sampling}, si ottiene:

\[\widehat{\rho}(x) = \int_{- \infty}^{+ \infty}{s_{\infty}(k)\left\lbrack \Delta k\sum_{p = - \infty}^{+ \infty}{\delta(k - p\Delta k)}rect\left( \dfrac{k + \dfrac{\Delta k}{2}}{W} \right) \right\rbrack\exp(j2\pi kx)dk}\]

Per effetto della finestra rettangolare, la sommatoria non si estende per tutti i numeri interi ma solamente sui \(2n\) punti che ricadono nella finestra stessa:

\[\widehat{\rho}(x) = \int_{- \infty}^{+ \infty}{s_{\infty}(k)\left\lbrack \Delta k\sum_{p = - n}^{n - 1}{\delta(k - p\Delta k)} \right\rbrack\exp(j2\pi kx)dk}\]

Invertendo il simbolo di sommatoria con quello di integrale si scrive:

\[\widehat{\rho}(x) = \Delta k\sum_{p = - n}^{n - 1}{\int_{- \infty}^{+ \infty}{s_{\infty}(k)\delta(k - p\Delta k)\exp(j2\pi kx)dk}}\]

Per la proprietà di campionamento della delta, si ha:

\[\widehat{\rho}(x) = \Delta k\sum_{p = - n}^{n - 1}{s_{\infty}(p\Delta k)\exp(j2\pi p\Delta kx)}\]

Questa relazione, se confrontata con l'antitrasformata di \({\widehat{s}}_{m}(k)\) tenendo conto della proprietà del prodotto di convoluzione, si ha:

\[\widehat{\rho}(x) = \mathfrak{F}^{- 1}\left\lbrack s_{\infty}(k) \right\rbrack(x)*h_{WS}(x)\]

Dal confronto è evidente che l'antitrasformata della funzione di filtraggio o PSF è data da:

\[h_{WS}(x) = \ \Delta k\sum_{p = - n}^{n - 1}{\exp(j2\pi p\Delta kx)}\]

La sommatoria che definisce \(h_{WS}(x)\) ha somma nota:

\[h_{WS}(x) = \ \Delta k\sum_{p = - n}^{n - 1}{\exp(j2\pi p\Delta kx)} = \Delta k\exp( - j2\pi n\Delta kx)\dfrac{1 - \exp(j4\pi n\Delta kx)}{1 - \exp(j2\pi\Delta kx)}\]

Si considera le formule di Eulero per \(\exp(j\vartheta)\) e \(\exp( - j\vartheta)\):

\[\exp(j\vartheta) = \cos\vartheta + j\sin\vartheta\]

\[\exp( - j\vartheta) = \cos\vartheta - j\sin\vartheta\]

Si considera la quantità \(\exp(j\vartheta) - \exp( - j\vartheta)\):

\[\exp(j\vartheta) - \exp( - j\vartheta) = \cos\vartheta + j\sin\vartheta - \cos\vartheta + j\sin\vartheta = 2j\sin\vartheta\]

Si ottiene, in definitiva:

\[\exp(j\vartheta) - \exp( - j\vartheta) = 2j\sin\vartheta\]

Si mette in evidenza \(\exp( - j\vartheta)\), da cui:

\[\left( \exp(j2\vartheta) - 1 \right)\exp( - j\vartheta) = 2j\sin\vartheta\]

Dividendo ambo i membri per \(\exp( - j\vartheta)\), si ha:

\[\exp(j2\vartheta) - 1 = 2j\sin\vartheta\exp(j\vartheta)\]

Applicando questo risultato all'espressione per la PSF nello spazio-immagine, \(h_{WS}(x)\), si ottiene:

\[1 - \exp(j4\pi n\Delta kx) = - 2j\sin(2\pi n\Delta kx)\exp(j2\pi n\Delta kx)\]

\[1 - \exp(j2\pi\Delta kx) = - 2j\sin(\pi\Delta kx)\exp(j\pi n\Delta kx)\]

Da cui:

\[h_{WS}(x) = \Delta k\dfrac{1 - \exp(j4\pi n\Delta kx)}{1 - \exp(j2n\Delta kx)}\exp( - j2\pi n\Delta kx) = \Delta k\dfrac{- 2j\sin(2\pi n\Delta kx)\exp(j2\pi n\Delta kx)}{- 2j\sin(\pi\Delta kx)\exp(j\pi n\Delta kx)}\exp( - j2\pi n\Delta kx) =\]

È noto che \(W = 2n\Delta k\), per cui:

\[= \Delta k\dfrac{\sin(\pi Wx)}{\sin(\pi\Delta kx)}\exp(j\pi n\Delta kx)\exp( - j2\pi n\Delta kx) =\]

Si moltiplica e divide il secondo membro per \(\pi Wx\):

\[= \Delta k\dfrac{\dfrac{\sin(\pi Wx)}{\pi Wx}}{\dfrac{\sin(\pi\Delta kx)}{\pi Wx}}\exp( - j\pi n\Delta kx) =\]

Per definizione:

\[\dfrac{\sin(\pi Wx)}{\pi Wx} = sinc(\pi Wx)\]

Inoltre, al denominatore si ricorda che \(W = 2n\Delta k\):

\[\dfrac{\sin(\pi\Delta kx)}{2n\pi\Delta kx} = \dfrac{1}{2n}\dfrac{\sin(\pi\Delta kx)}{\pi\Delta kx}\dfrac{1}{2n}{sinc}(\pi\Delta kx)\]

Sostituendo questo risultato appena ottenuto in \(h_{WS}(x)\), si ha:

\[h_{WS}(x) = 2n\Delta k\dfrac{{sinc}(\pi Wx)}{{sinc}(\pi\Delta kx)}\exp( - j\pi n\Delta kx)\]

Ovvero:

\[h_{WS}(x) = W\dfrac{{sinc}(\pi Wx)}{{sinc}(\pi\Delta kx)}\exp( - j\pi n\Delta kx)\]

Come nel caso continuo, la PSF, \(h_{WS}(x)\), determina degli sfocamenti nelle regioni ad alta frequenza dell'immagine, ovvero dei bordi dell'immagine. La \(sinc\) presenta delle oscillazioni che non portano a transizioni nette tra i vari organi ma a delle sfocature che determinano una cattiva visione dei confini tra i vari parenchima degli organi.

Il troncamento netto nel dominio delle \(k\) fornisce delle oscillazioni nel dominio dello spazio-immagine. Questo fenomeno è dovuto all'effetto Gibbs, ovvero degli errori di ricostruzione dovuti alla banda finita della banda di ricostruzione, quando si hanno funzioni con brutte discontinuità.

\begin{figure}
\centering
\includegraphics[width=5.49167in,height=2.69955in,alt={Immagine che contiene schizzo, linea, diagramma, disegno Il contenuto generato dall\textquotesingle IA potrebbe non essere corretto.}]{media/10_Ric3D/image280.pdf}\caption{Figura .: Fenomeno di Gibbs}
\end{figure}

È possibile dimostrare che le oscillazioni introdotte per la convoluzione con la PSF, causate dal fenomeno di Gibbs, si estende fino al \(10\%\) della discontinuità sia per le sottoelongazioni che sovra-elongazioni.

È possibile, inoltre, che le oscillazioni introdotte dal fenomeno di Gibbs sono correlata alle porzioni di \(k\)-spazio considerate: maggiore è la porzione del \(k\)-spazio, maggiore è la frequenza delle oscillazioni e maggiori sono le sotto-elongazioni e sovra-elongazioni.

\begin{figure}
\centering
\includegraphics[width=4.27715in,height=3.96206in,alt={Immagine che contiene Imaging medicale, radiologia, Radiografia medica, radiografia Il contenuto generato dall\textquotesingle IA potrebbe non essere corretto.}]{media/10_Ric3D/image281.pdf}
\caption{Figura .: Fenomeno di GIbbs nella degradazione dell'immagine}
\end{figure}

Le oscillazioni di Gibbs possono degradare l'immagine in modo anche critico. Generalmente, aumentando il numero dei campioni per un campo di vista (o FOV) si riduce la degradazione poiché si riduce le regioni dello spazio-immagine interessate dalle oscillazioni. In altre parole, aumentare il numero di campioni migliora la ricostruzione. Per tale motivo, nella pratica si esegue una scansione con un numero di campioni doppio rispetto al minimo richiesto, così da ottenere un oggetto con confini meno oscillanti. Non sempre è possibile aumentare il numero di punti per questioni temporali.

\begin{figure}
\centering
\includegraphics[width=4.44379in,height=4.80389in,alt={Immagine che contiene cerchio, colino, stoviglie, bianco e nero Il contenuto generato dall\textquotesingle IA potrebbe non essere corretto.}]{media/10_Ric3D/image282.pdf}\caption{Figura .: Riduzione delle oscillazioni di Gibbs all'aumentare del numero di punti}
\end{figure}

Un rimedio alternativo consiste nell'utilizzare un ulteriore filtro di smoothing sul segnale registrato. Infatti, se il set di dati ottenuto è moltiplicato per una funzione che si annulla nel \(k\)-spazio intorno ai valori limite \(- k_{\min}\) e \(+ k_{\max}\) si ottiene un filtraggio che presenta una transizione più dolce dal valore massimo al valore minimo, riducendo così le oscillazioni che si hanno nel dominio dello spazio-immagine. Questo processo è noto in gergo come apodizzazione.

\begin{figure}
\centering
\includegraphics[width=6.68958in,height=3.97708in]{media/10_Ric3D/image283.pdf}\caption{Figura .: Finestra di Hamming nel dominio del \(k\)-spazio e dello spazio-immagine}
\end{figure}

Un semplice funzione filtrante è offerta dalla finestra di Hamming o coseno rialzato, data dall'equazione nel \(k\)-spazio:

\[H_{ham}(k) = \dfrac{1}{2} + \dfrac{1}{2}\cos\left( \dfrac{2\pi k}{W} \right) = \cos^{2}\left( \dfrac{\pi k}{W} \right)\]

Dal punto di vista numerico \(H_{ham}(k)\) è un vettore detto convolution window.

Si analizza il caso in cui \(k \rightarrow + k_{\max}\) e \(k \rightarrow - k_{\min}\). In questo contesto, l'ampiezza della finestra \(W\) è data dalla differenza tra \(- k_{\min}\) e \(k_{\max}\). Nell'ipotesi che \(\left| k_{\min} \right| = k_{\max}\), risulta:

\[W = k_{\max} - \left( - k_{\min} \right) = 2k_{\max}\]

Se \(k \rightarrow \pm k_{\max}\), si ha:

\[\dfrac{2\pi k}{W} = \dfrac{2\pi k}{k_{\max} + k_{\min}} = \dfrac{2\pi k}{2k_{\max}} = \dfrac{\pi k}{k_{\max}} \rightarrow \pm \pi,k \rightarrow \pm k_{\max}\ \]

In questa condizione:

\[H_{ham}(k) = \cos^{2}\left( \dfrac{\pi k}{W} \right) \rightarrow 0,k \rightarrow \pm k_{\max}\ \]

Nel dominio dello spazio-immagine la finestra di Hamming si esprime come somma di impulsi di opportuna area e punti di applicazione:

\[h_{ham}(x) = \dfrac{1}{2}\delta(x) + \dfrac{1}{4}\left\lbrack \delta\left( x - \dfrac{1}{W} \right) + \delta\left( x + \dfrac{1}{W} \right) \right\rbrack\]

Dove \(W\) è legato alla risoluzione spaziale \(\Delta x\) dalla relazione:

\[W = \dfrac{1}{\Delta x} \Leftrightarrow \Delta x = \dfrac{1}{W}\]

Per cui, è possibile la finestra nello spazio-immagine:

\[h_{ham}(x) = \dfrac{1}{2}\delta(x) + \dfrac{1}{4}\left\lbrack \delta(x - \Delta x) + \delta(x + \Delta x) \right\rbrack\]

Se agisce solamente la finestra di Hamming, la densità protonica ricostruita è data dalla convoluzione dall'antitrasformata della finestra di smoothing con la densità protonica fisicamente presente nel volume:

\[\widehat{\rho}(x) = \rho(x)*h_{ham}(x)\]

Sostituendo l'espressione della finestra di Hamming:

\[\widehat{\rho}(x) = \rho(x)*\left\{ \dfrac{1}{2}\delta(x) + \dfrac{1}{4}\left\lbrack \delta(x - \Delta x) + \delta(x + \Delta x) \right\rbrack \right\}\]

Per la proprietà di campionamento della delta, si ha:

\[\widehat{\rho}(x) = \dfrac{1}{2}\rho(x) + \dfrac{1}{4}\rho(x - \Delta x) + \dfrac{1}{4}\rho(x + \Delta x)\]

Il filtraggio con la finestra di Hamming nel dominio dell'immagne corrisponde a un'operazione di media di \(\rho(x)\) e una sua versione scalata e traslata di \(\pm \Delta x\). Con questa soluzione si riduce il contenuto oscillatorio dei bordi. La finestratura non rimuove comunque lo sfocamento dei bordi, rendendo difficile il riconoscimento dei vari organi. Si perde, quindi, in risoluzione a favore dell'abbattimento delle oscillazioni o ringing sia all'interno che all'esterno dell'immagine.

Se si tronca il segnale con la finestra rettangolare e si campione, la densità protonica ricostruita è del tipo:

\[\widehat{\rho}(x) = \ \rho(x)*h_{WS}(x)*h_{ham}(x)\]

Sostituendo le espressioni per le funzioni di trasferimento, si ha:

\[\widehat{\rho}(x) = \ \rho(x)*\left\lbrack W\dfrac{{sinc}(\pi Wx)}{{sinc}(\pi\Delta kx)}\exp( - j\pi n\Delta kx) \right\rbrack*\left\lbrack \dfrac{1}{2}\delta(x) + \dfrac{1}{4}\delta(x - \Delta x) + \dfrac{1}{4}\delta(x + \Delta x) \right\rbrack\]

Applicando le proprietà della delta di Dirac tra le due funzioni di trasferimento, si ha:

\[\widehat{\rho}(x) = \ \rho(x)*\left\lbrack \dfrac{1}{2}W\dfrac{{sinc}(\pi Wx)}{{sinc}(\pi\Delta kx)}\exp( - j\pi n\Delta kx) + \dfrac{1}{4}W\dfrac{{sinc}\left( \pi W(x - \Delta x) \right)}{{sinc}\left( \pi\Delta k(x - \Delta x) \right)}\exp\left( - j\pi n\Delta k(x - \Delta x) \right)\dfrac{1}{4}W\dfrac{{sinc}\left( \pi W(x + \Delta x) \right)}{{sinc}\left( \pi\Delta k(x + \Delta x) \right)}\exp\left( - j\pi n\Delta k(x + \Delta x) \right) \right\rbrack\]

A questo segnale ricostruito corrisponde una media spaziale della funzione filtratane \(h_{WS}(x)\) che riduce l'ampiezza delle oscillazioni ma allo stesso tempo sfoca i contorni dell'immagine.

\subsection{Risoluzione spaziale in risonanza magnetica}\label{risoluzione-spaziale-in-risonanza-magnetica}

La risoluzione spaziale di un'immagine si riferisce alla più piccola distanza interposta tra due oggetti diversi o due differenti righe nell'immagine affinché siano identificabili come due oggetti distinti. La risoluzione dipende dalla scelta degli oggetti da distinguere, dai limiti della strumentazione di misura, dalle caratteristiche fisiologiche degli organi o altri tessuti e altro ancora.

La \emph{point spread function} permette di quantificare e stimare la risoluzione o la sfocatura introdotta dal metodo di misura.

Dal punto di vista teorico si vorrebbe una PSF quanto più stretta possibile, al limite impulsiva, nello spazio-immagine in modo che il prodotto di convoluzione restituisca il valore vero della densità protonica per ogni punto \(x\). Nel limite in cui \(h(x) = \delta(x)\) si ha infatti:

\[\widehat{\rho}(x) = \ \rho(x)*h(x) = \rho(x)*\delta(x) = \rho(x)\]

Nella pratica, la PSF presenta una certa estensione finita nello spazio-immagine. La convoluzione con la densità protonica provoca uno smussamento dei bordi dell'oggetto, causando una sfocatura complessiva tra i confini dei vari organi.

\begin{figure}
\centering
\includegraphics[width=4.11364in,height=3.93037in,alt={Immagine che contiene testo, schermata, diagramma, Diagramma Il contenuto generato dall\textquotesingle IA potrebbe non essere corretto.}]{media/10_Ric3D/image284.pdf}\caption{Figura .: Effetto della PSF sull'output}
\end{figure}

Una prima definizione di risoluzione spaziale è data dall'area sottesa in tutto il dominio dello spazzi-immagine dalla PSF, normalizzata al valore che essa assume nell'origine:

\[\Delta x_{MRI} = \dfrac{1}{h(0)}\int_{- \infty}^{+ \infty}{h(x)dx}\]

È possibile scrivere l'integrale al secondo membro in termini di trasformata di Fourier:

\[\int_{- \infty}^{+ \infty}{h(x)dx} = \int_{- \infty}^{+ \infty}{h(x)\exp(j2\pi k0)dx} = H(0)\]

L'area sottesa dalla PSF in tutto il dominio dello spazio-immagine coincide con il valore dell'origine della sua trasformata di Fourier, nota come \emph{Modulation Transfert Function} o MTF. Da questo risultato si evince che \(H(0)\) è una misura dell'energia della PSF.

La definizione di \(\Delta x_{MRI}\) fornisce una stima della più piccola distanza tra due oggetti che può insistere tra i due affinché siano ancora distinguibili, mediante una ricostruzione standard dell'immagine in risonanza magnetica con filtri addizionali. In particolare, la definizione di \(\Delta x_{MRI}\) permette la costruzione di un rettangolo centrato sullo stesso punto della PSF con la stessa area.

\begin{figure}
\centering
\includegraphics[width=5.61994in,height=4.35556in,alt={Immagine che contiene testo, Diagramma, diagramma, linea Il contenuto generato dall\textquotesingle IA potrebbe non essere corretto.}]{media/10_Ric3D/image285.pdf}\caption{Figura .: Rappresentazione grafica della risoluzione come energia della PSF normalizzata}
\end{figure}

Per comprendere a pieno il significato della definizione si suppone che la densità protonica sia di tipo impulsiva, centrata in due punti \(x_{1}\) e \(x_{2}\):

\[\rho(x) = \delta\left( x - x_{1} \right) + \delta\left( x - x_{2} \right)\]

La densità protonica ricostruita è ottenuta dalla convoluzione della densità protonica reale \(\rho\) e la PSF:

\[\widehat{\rho}(x) = \rho(x)*h(x) = \left\lbrack \delta\left( x - x_{1} \right) + \delta\left( x - x_{2} \right) \right\rbrack*h(x)\]

Per la proprietà della delta di Dirac, risulta:

\[\widehat{\rho}(x) = h\left( x - x_{1} \right) + h\left( x - x_{2} \right)\]

A causa della PSF, i due impulsi sono trasformati in due campane con durata finita, rappresentate proprio da due PSF centrate su \(x_{1}\) e \(x_{2}\).

\begin{figure}
\centering
\includegraphics[width=6.68958in,height=3.92424in]{media/10_Ric3D/image286.pdf}\caption{Figura .: Densità protonica ricostruita a partire dalla convoluzione tra due impulsi e la PSF}
\end{figure}

La risoluzione \(\Delta x_{MRI}\) rappresenta il grado con cui i due impulsi reali si sovrappongono. Se la distanza tra \(x_{1}\) e \(x_{2}\) è inferiore a \(\Delta x_{RMI}\) si osserva una forte sovrapposizione tre le due PSF traslate e, quindi, non è più possibile distinguere i due oggetti a causa della sfocatura introdotta dal filtraggio intrinseco del processo di acquisizione e ricostruzione.

Si osservi che la risoluzione così definita non coincide necessariamente con Fourier pixel size \(\Delta x = W^{- 1} = (2n\Delta k)^{- 1}\). Nel caso migliore possibile, in presenza del solo troncamento e campionamento, la risoluzione appena definita \(\Delta x_{RMI}\) è data da:

\[h_{ws}(x) = \sum_{p = - n}^{n - 1}{\exp(j2\pi p\Delta kx)}\]

Valutando in \(0\) la PSF si ha:

\[h_{ws}(x) = \sum_{p = - n}^{n - 1}{\exp(0)} = \sum_{p = - n}^{n - 1}1 = 2n\]

Si valuta l'energia della PSF legato al solo troncamento e campionamento, \(h_{ws}(k)\):

\[H_{WS}(0) = \int_{- \infty}^{+ \infty}{\left\lbrack \sum_{p = - n}^{n - 1}{\exp(j2\pi p\Delta kx)} \right\rbrack dx}\]

Sia \(\left\lbrack - \dfrac{L}{2};\dfrac{L}{2} \right\rbrack\) l'intervallo in cui la PSF è non nulla, dove \(L\) è l'estensione della finestra di acquisizione, ovvero il FOV. L'integrale si riduce a:

\[H_{WS}(0) = \int_{- \dfrac{L}{2}}^{+ \dfrac{L}{2}}{\left\lbrack \sum_{p = - n}^{n - 1}{\exp(j2\pi p\Delta kx)} \right\rbrack dx}\]

Si inverte il simbolo di integrale con quello di sommatoria, ottenendo:

\[H_{WS}(0) = \sum_{p = - n}^{n - 1}\left\lbrack \int_{- \dfrac{L}{2}}^{+ \dfrac{L}{2}}{\exp(j2\pi p\Delta kx)dx} \right\rbrack\]

Si risolve l'integrale:

\[\int_{- \dfrac{L}{2}}^{+ \dfrac{L}{2}}{\exp(j2\pi p\Delta kx)dx} = \dfrac{1}{j2\pi p\Delta k}\left\lbrack \exp(j2\pi p\Delta kx) \right\rbrack_{- \dfrac{L}{2}}^{+ \dfrac{L}{2}} = \ \dfrac{1}{j2\pi p\Delta k}\left\lbrack \exp\left( j2\pi p\Delta k\dfrac{L}{2} \right) - \exp\left( - j2\pi p\Delta k\dfrac{L}{2} \right) \right\rbrack\]

Per le formule di Eulero, risulta:

\[\exp\left( j2\pi p\Delta k\dfrac{L}{2} \right) - \exp\left( - j2\pi p\Delta k\dfrac{L}{2} \right) = 2j\sin\left( 2\pi p\Delta k\dfrac{L}{2} \right)\]

Quindi la soluzione dell'integrale è:

\[\int_{- \dfrac{L}{2}}^{+ \dfrac{L}{2}}{\exp(j2\pi p\Delta kx)dx} = \dfrac{1}{j2\pi p\Delta k}2j\sin\left( 2\pi p\Delta k\dfrac{L}{2} \right) = \dfrac{1}{\pi p\Delta k}\sin(\pi p\Delta kL) =\]

Moltiplicando e dividendo per \(L\) il secondo membro è possibile esprimere l'integrale in termini di \(sinc\):

\[= L\dfrac{\sin(\pi p\Delta kL)}{\pi p\Delta kL} = L{sinc}(\pi p\Delta kL) =\]

Il campionamento nel \(k\)-spazio, \(\Delta k\), e il FOV sono legati dalla relazione \(\Delta k = L^{- 1}\), per cui:

\[= L{sinc}\left( \pi p\dfrac{1}{L}L \right) = L{sinc}(\pi p)\]

Al variare di \(p \in \mathbb{N}_{0}\), la \({sinc}(\pi p)\) è nulla, a eccezione del caso \(p = 0\), nella quale è unitaria. Formalmente, è possibile scrivere che:

\[L{sinc}(\pi p) = L\delta_{p0}\]

Con:

\[\delta_{p0} = \left\{ \begin{aligned}
1,\ \ p = 0 \\
x,\ \  & p \neq 0
\end{aligned} \right.\ \]

Sostituendo questo risultato nell'espressione per \(H_{WS}(0)\), si ottiene:

\[H_{WS}(0) = \sum_{p = - n}^{n - 1}\left\lbrack L\delta_{p0} \right\rbrack = L\sum_{p = - n}^{n - 1}\delta_{p0}\]

Di tutta la sommatoria esiste un unico \(p = 0\), gli altri sono diversi da tale valore, per cui, tutta la sommatoria si riduce a \(1\):

\[H_{WS}(0) = L\sum_{p = - n}^{n - 1}\delta_{p0} = L\]

La risoluzione spaziale nel caso di solo campionamento e troncamento, \(\Delta x_{RMI}\), per definizione è data da:

\[\Delta x_{MRI} = \dfrac{1}{h_{WS}(0)}\int_{- \infty}^{+ \infty}{h_{WS}(x)dx} = \dfrac{H_{WS}(0)}{h_{WS}(0)}\]

Sostituendo i risultati ottenuti si ottiene:

\[\Delta x_{MRI} = \dfrac{L}{2n} = \dfrac{1}{2n\Delta k} = \Delta x\]

Nel caso migliore, la risoluzione \(\Delta x_{MRI}\) coincide con il Fourier pixel size \(\Delta x\).

Quando è presente un ulteriore filtraggio, applicato al segnale, la PSF è ottenuta dalla convoluzione della finestra di troncamento e campionamento, \(h_{WS}\), e dalla PSF del filtro applicato, \(h_{filtro}\):

\[h(x) = h_{WS}(x)*h_{filtro}(x) = \sum_{p = - n}^{n - 1}{H_{filro}(p\Delta k)\exp(j2\pi\Delta kk)}\]

La PSF complessiva cambia e la risoluzione spaziale non coincide più on il Fourier pixel size, ma presenta un valore maggiore di questa quantità;

\[\Delta x_{RMI} > \Delta x\]

Si ha, quindi, un peggioramento della risoluzione spaziale.

\subsubsection{Full width half maximum}\label{full-width-half-maximum}

La risoluzione di risoluzione spaziale come integrale della PSF può non essere sempre calcolata e misurata. In questi casi si utilizza una definizione alternativa di risoluzione spaziale, basata sul profilo della PSF, simmetrica perché reale.

La \emph{full width half maximum} (FWHM) è definita come il doppio della distanza del centro della PSF simmetrica e un punto in cui la sua ampiezza decade di un fattore \(k\) rispetto al valore massimo, ottenuto nell'origine:

\[h\left( \dfrac{\Delta x_{RMI}}{2} \right) = kh(0)\]

Solitamente, come riferimento si utilizza metà dell'ampiezza massima, ovvero \(k = 1/2\). In questo caso si ottiene:

\[h\left( \dfrac{\Delta x_{RMI}}{2} \right) = \dfrac{1}{2}h(0)\]

\begin{figure}
\centering
\includegraphics[width=5.53788in,height=3.20307in,alt={Immagine che contiene testo, linea, Diagramma, diagramma Il contenuto generato dall\textquotesingle IA potrebbe non essere corretto.}]{media/10_Ric3D/image287.pdf}\caption{Figura .:Esempio d i FWHM}
\end{figure}

La sovrapposizione tra due oggetti si verifica quando questi ultimi sono separati da una distanza minore di \(\Delta x_{RMI} \equiv FWHM\).

Non è detto che la definizione secondo la \emph{full width half maximum} coincida con quella basata sull'energia normalizzata della PSF.

La risoluzione spaziale, nel caso in cui siano applicate contemperamento due filtri con la stessa forma, si ottiene sommando la risoluzione spaziale delle due finestre:

\[\Delta x_{tot} = \Delta x_{filtro1} + \Delta x_{filtro2}\]

\subsection{Filtraggio introdotto dal rilassamento nella gradient echo}\label{filtraggio-introdotto-dal-rilassamento-nella-gradient-echo}

L'effetto del campionamento e del filtraggio può essere modellato come un filtraggio passa-basso sull'immagine di risonanza magnetica; tuttavia, non sono gli unici filtraggi presenti in un normale esperimento di questa tecnologia. Anche i tempi di rilassamento \(T_{1}\), \(T_{2}\) e \(T_{2}^{*}\), che alterano l'ampiezza del segnale registrato nel \(k\)-spazio, possono essere modellati come filtri.

Gli effetti del rilassamento non possono essere trascurati nel caso in cui la finestra di acquisizione \(T_{S}\) è paragonabile col tempo \(T_{2}^{*}\) con cui decade il segnale registrato. In questa condizione il segnale campionato presenta un'ampiezza dipendente da un decadimento esponenziale del tipo:

\[s_{m}(k) = s(k)\exp\left( - \dfrac{t}{T_{2}^{*}} \right)\]

Dove l'origine dei tempi \(t = 0\ s\) è scelto, generalmente, al centro dell'impulso a radiofrequenza. Si considera, ad esempio, una sequenza gradient-echo. Si adopera la sostituzione \(t' = t - T_{E}\) in modo da riportare l'origine dei tempi al tempo di echo \(T_{E}\); in altre parole, si sceglie \(T_{E}\) come riferimento temporale.

\begin{figure}
\centering
\includegraphics[width=6.69306in,height=5.34306in]{media/10_Ric3D/image288.pdf}\caption{Figura .: Gradient-echo con segnale registrato il cui inviluppo va come \(\exp\left( t/T_{2}^{*} \right)\)}
\end{figure}

Nel dominio del \(k\)-spazio, è possibile scrivere che:

\[k = \overline{\gamma}Gt' \Leftrightarrow t' = \dfrac{k}{\overline{\gamma}G}\]

Il tempo \(t\), legato a \(t'\), può essere espresso come:

\[t' = t - T_{E} \Leftrightarrow t = t' + T_{E} = \dfrac{k}{\overline{\gamma}G} + T_{E}\]

Con questa posizione, il decadimento esponenziale del segnale registrato nel \(k\)-spazio si scrive come:

\[\exp\left( - \dfrac{t}{T_{2}^{*}} \right) = \exp\left( - \dfrac{k}{\overline{\gamma}GT_{2}^{*}} \right)\exp\left( - \dfrac{T_{E}}{T_{2}^{*}} \right)\]

Applicando anche il troncamento al segnale registrato nel \(k\)-spazio, si ottiene:

\[s_{m}(k) = s(k)\exp\left( - \dfrac{k}{\overline{\gamma}GT_{2}^{*}} \right)\exp\left( - \dfrac{T_{E}}{T_{2}^{*}} \right)H_{w}(k)\]

Una volta fissato il tempo di echo e le disomogeneità di campo, il termine \(\exp\left( - T_{E}/T_{2}^{*} \right)\) è costante.

Si esplicita la finestra di troncamento, \(H_{w}(k)\):

\[s_{m}(k) = s(k)\exp\left( - \dfrac{k}{\overline{\gamma}GT_{2}^{*}} \right){rect}\left( \dfrac{k + \dfrac{\Delta k}{2}}{W} \right)\exp\left( - \dfrac{T_{E}}{T_{2}^{*}} \right)\]

Il segnale registrato non è simmetrico intorno all'origine, dunque, la sua antitrasformata restituisce una densità protonica ricostruita complessa.

All'atto pratico, dominio dello spazio-immagine, oltre alla convoluzione con la finestra di campionamento e troncamento, bisogna aggiungere anche la convoluzione con la finestra \(h_{T_{2}^{*}}\), ottenuta antitrasformato i termini esponenziali che dipendono da \(T_{2}^{*}\):

\[\mathfrak{F}^{-}\left\lbrack \exp\left( - \dfrac{k}{\overline{\gamma}GT_{2}^{*}} \right)\exp\left( - \dfrac{T_{E}}{T_{2}^{*}} \right) \right\rbrack(x) = \exp\left( - \dfrac{T_{E}}{T_{2}^{*}} \right)\mathfrak{F}^{-}\left\lbrack \exp\left( - \dfrac{k}{\overline{\gamma}GT_{2}^{*}} \right) \right\rbrack(x)\]

Si applica la definizione di trasformata inversa di Fourier:

\[\mathfrak{F}^{-}\left\lbrack \exp\left( - \dfrac{k}{\overline{\gamma}GT_{2}^{*}} \right) \right\rbrack(x) = \int_{- \infty}^{+ \infty}{\exp\left( - \dfrac{k}{\overline{\gamma}GT_{2}^{*}} \right)\exp(j2\pi kx)dk} = \int_{- \infty}^{+ \infty}{\exp\left( - \dfrac{k}{\overline{\gamma}GT_{2}^{*}} + j2\pi kx \right)dk} =\]

Il termine esponenziale è un esponenziale decrescente, non simmetrico rispetto all'origine, definito su un intervallo finito di acquisizione \(\left\lbrack - k_{\max}:k_{\max} \right\rbrack\) a causa del troncamento. La trasformata inversa di Fourier si riduce a:

\[\mathfrak{F}^{-}\left\lbrack \exp\left( - \dfrac{k}{\overline{\gamma}GT_{2}^{*}} \right) \right\rbrack(x) = \int_{- k_{\max}}^{+ k_{\max}}{\exp\left( - \dfrac{k}{\overline{\gamma}GT_{2}^{*}} + j2\pi kx \right)dk} = \int_{- k_{\max}}^{+ k_{\max}}{\exp\left( \left( j2\pi x - \dfrac{1}{\overline{\gamma}GT_{2}^{*}} \right)k \right)dk} =\]

Si risolve l'integrale al secondo membro, moltiplicando e dividendo per l'argomento di \(k\):

\[= \int_{- k_{\max}}^{+ k_{\max}}{\exp\left( \left( j2\pi x - \dfrac{1}{\overline{\gamma}GT_{2}^{*}} \right)k \right)dk} = \dfrac{1}{j2\pi x - \dfrac{1}{\overline{\gamma}GT_{2}^{*}}}\left\lbrack \exp\left( \left( j2\pi x - \dfrac{1}{\overline{\gamma}GT_{2}^{*}} \right)k \right) \right\rbrack_{- k_{\max}}^{k_{\max}} = \dfrac{\exp\left( \left( j2\pi x - \dfrac{1}{\overline{\gamma}GT_{2}^{*}} \right)k_{\max} \right) - \exp\left( - \left( j2\pi x - \dfrac{1}{\overline{\gamma}GT_{2}^{*}} \right)k_{\max} \right)}{j2\pi x - \dfrac{1}{\overline{\gamma}GT_{2}^{*}}} =\]

Il gradiente \(G\) è legato alla finestra di lettura dalla relazione:

\[T_{S} = \dfrac{W}{\overline{\gamma}G} \Leftrightarrow \ G = \dfrac{W}{\overline{\gamma}T_{S}}\]

La finestra di acquisizione si estende da \(- k_{\max}\) a \(k_{\max}\), quindi, l'estensione della finestra di acquisizione è due volte \(k_{\max}\):

\[W = 2k_{\max}\]

Da cui:

\[G = \dfrac{2k_{\max}}{\overline{\gamma}T_{S}}\]

È possibile scrivere:

\[= \dfrac{\exp\left( \left( j2\pi x - \dfrac{1}{\overline{\gamma}\dfrac{2k_{\max}}{\overline{\gamma}T_{S}}T_{2}^{*}} \right)k_{\max} \right) - \exp\left( - \left( j2\pi x - \dfrac{1}{\overline{\gamma}\dfrac{2k_{\max}}{\overline{\gamma}T_{S}}T_{2}^{*}} \right)k_{\max} \right)}{j2\pi x - \dfrac{1}{\overline{\gamma}\dfrac{2k_{\max}}{\overline{\gamma}T_{S}}T_{2}^{*}}} =\]

Si svolgono i prodotti:

\[= \dfrac{\exp\left( j2\pi xk_{\max} - \dfrac{T_{S}}{2T_{2}^{*}} \right) - \exp\left( - j2\pi xk_{\max} + \dfrac{T_{S}}{2T_{2}^{*}} \right)}{j2\pi x - \dfrac{T_{S}}{2k_{\max}T_{2}^{*}}} =\]

La finestra di acquisizione si estende da \(- k_{\max}\) a \(k_{\max}\), quindi, l'estensione della finestra di acquisizione è due volte \(k_{\max}\), \(W = 2k_{\max}\). L'ampiezza della finestra \(W\) è legata all'inverso del campionamento \(\Delta x\):

\[W = 2k_{\max} \Leftrightarrow 2k_{\max} = \dfrac{1}{\Delta x} \Leftrightarrow k_{\max} = \dfrac{1}{2\Delta x}\]

Sostituendo tale risultato si ottiene:

\[= \dfrac{\exp\left( j2\pi x\dfrac{1}{2\Delta x} - \dfrac{T_{S}}{2T_{2}^{*}} \right) - \exp\left( - j2\pi x\dfrac{1}{2\Delta x} + \dfrac{T_{S}}{2T_{2}^{*}} \right)}{j2\pi x - \dfrac{T_{S}}{\dfrac{2}{2\Delta x}T_{2}^{*}}}\]

Semplificando, si ottiene:

\[\mathfrak{F}^{-}\left\lbrack \exp\left( - \dfrac{k}{\overline{\gamma}GT_{2}^{*}} \right) \right\rbrack(x) = \dfrac{\exp\left( j\pi\dfrac{x}{\Delta x} - \dfrac{T_{S}}{2T_{2}^{*}} \right) - \exp\left( - j\pi\dfrac{x}{\Delta x} + \dfrac{T_{S}}{2T_{2}^{*}} \right)}{j2\pi x - \Delta x\dfrac{T_{S}}{T_{2}^{*}}}\]

Mettendo un segno negativo in evidenza al numerato e al denominatore, si ottiene:

\[\mathfrak{F}^{-}\left\lbrack \exp\left( - \dfrac{k}{\overline{\gamma}GT_{2}^{*}} \right) \right\rbrack(x) = \dfrac{\exp\left( \dfrac{T_{S}}{2T_{2}^{*}} - j\pi\dfrac{x}{\Delta x} \right) - \exp\left( - \dfrac{T_{S}}{2T_{2}^{*}} + j\pi\dfrac{x}{\Delta x} \right)}{\Delta x\dfrac{T_{S}}{T_{2}^{*}} - j2\pi x}\]

Considerando anche il termine esponenziale costante, la PSF introdotta dal tempo di rilassamento \(T_{2}^{*}\) è data da:

\[h_{T_{2}^{*}}(x) = \exp\left( - \dfrac{T_{E}}{T_{2}^{*}} \right)\mathfrak{F}^{-}\left\lbrack \exp\left( - \dfrac{k}{\overline{\gamma}GT_{2}^{*}} \right) \right\rbrack(x) = \dfrac{\exp\left( \dfrac{T_{S}}{2T_{2}^{*}} - j\pi\dfrac{x}{\Delta x} \right) - \exp\left( - \dfrac{T_{S}}{2T_{2}^{*}} + j\pi\dfrac{x}{\Delta x} \right)}{\Delta x\dfrac{T_{S}}{T_{2}^{*}} - j2\pi x}\exp\left( - \dfrac{T_{E}}{T_{2}^{*}} \right)\]

La risoluzione \(\Delta x_{RMI}\) introdotta da questo filtraggio può essere ottenuta applicando la definizione:

\[\Delta x_{MRI} = \dfrac{1}{h_{T_{2}^{*}}(0)}\int_{\infty}^{\infty}{h_{T_{2}^{*}}(x)dx} = \dfrac{H_{T_{2}^{*}}(0)}{h_{T_{2}^{*}}(0)}\]

La trasformata di Fourier della PSF è nota ed è:

\[H_{T_{2}^{*}}(k) = \exp\left( - \dfrac{k}{\overline{\gamma}GT_{2}^{*}} \right)\exp\left( - \dfrac{T_{E}}{T_{2}^{*}} \right)\]

La quale, valutata per \(k = 0\), si riduce a:

\[H_{T_{2}^{*}}(0) = \exp\left( - \dfrac{T_{E}}{T_{2}^{*}} \right)\]

La PSF valutata in \(x = 0\) è data, invece, da:

\[h_{T_{2}^{*}}(x) = \left. \ \dfrac{\exp\left( \dfrac{T_{S}}{2T_{2}^{*}} - j\pi\dfrac{x}{\Delta x} \right) - \exp\left( - \dfrac{T_{S}}{2T_{2}^{*}} + j\pi\dfrac{x}{\Delta x} \right)}{\Delta x\dfrac{T_{S}}{T_{2}^{*}} - j2\pi x}\exp\left( - \dfrac{T_{E}}{T_{2}^{*}} \right) \right|_{x = 0} = \dfrac{\exp\left( \dfrac{T_{S}}{2T_{2}^{*}} \right) - \exp\left( - \dfrac{T_{S}}{2T_{2}^{*}} \right)}{\Delta x\dfrac{T_{S}}{T_{2}^{*}}}\exp\left( - \dfrac{T_{E}}{T_{2}^{*}} \right)\]

La risoluzione della PSF è, quindi:

\[\Delta x_{MRI} = \dfrac{H_{T_{2}^{*}}(0)}{h_{T_{2}^{*}}(0)} = \dfrac{\exp\left( - \dfrac{T_{E}}{T_{2}^{*}} \right)}{\dfrac{\exp\left( \dfrac{T_{S}}{2T_{2}^{*}} \right) - \exp\left( - \dfrac{T_{S}}{2T_{2}^{*}} \right)}{\Delta x\dfrac{T_{S}}{T_{2}^{*}}}\exp\left( - \dfrac{T_{E}}{T_{2}^{*}} \right)} = \Delta x\dfrac{T_{S}}{T_{2}^{*}}\dfrac{1}{\exp\left( \dfrac{T_{S}}{2T_{2}^{*}} \right) - \exp\left( - \dfrac{T_{S}}{2T_{2}^{*}} \right)}\]

Si definisce \(sinc\) iperbolico, \(sinch\), come:

\[{sinch}(x) = \dfrac{\sinh(x)}{x} = \dfrac{e^{x} - e^{- x}}{2x}\]

È possibile scrivere, dunque:

\[{sinch}\left( \dfrac{T_{S}}{2T_{2}^{*}} \right) = \dfrac{\exp\left( \dfrac{T_{S}}{2T_{2}^{*}} \right) - \exp\left( - \dfrac{T_{S}}{2T_{2}^{*}} \right)}{2\dfrac{T_{S}}{2T_{2}^{*}}} = \dfrac{\exp\left( \dfrac{T_{S}}{2T_{2}^{*}} \right) - \exp\left( - \dfrac{T_{S}}{2T_{2}^{*}} \right)}{\dfrac{T_{S}}{T_{2}^{*}}}\]

La risoluzione \(\Delta x_{MRI}\) relativo alla PSF dovuta al tempo di rilassamento \(T_{2}^{*}\) si può scrivere come:

\[\Delta x_{MRI} = \dfrac{\Delta x}{{sinch}\left( \dfrac{T_{S}}{2T_{2}^{*}} \right)}\]

Se risulta:

\[\dfrac{T_{S}}{2T_{2}^{*}} \ll 1\]

È possibile approssimare la funzione \(sinc\) iperbolico come:

\[{sinch}\left( \dfrac{T_{S}}{2T_{2}^{*}} \right) \simeq 1 - \dfrac{x^{2}}{6}\]

In questa condizione, si ha:

\[\Delta x_{MRI} = \dfrac{\Delta x}{{sinch}\left( \dfrac{T_{S}}{2T_{2}^{*}} \right)} \simeq \dfrac{\Delta x}{\left( 1 - \dfrac{1}{6}\left( \dfrac{T_{S}}{2T_{2}^{*}} \right)^{2} \right)}\]

Se risulta che \(T_{S} = T_{2}^{*}\), la risoluzione è:

\[\left. \ \Delta x_{MRI} \right|_{T_{S} = T_{2}^{*}} = \left. \ \dfrac{\Delta x}{{sinch}\left( \dfrac{T_{S}}{2T_{2}^{*}} \right)} \right|_{T_{S} = T_{2}^{*}} = \left. \ \dfrac{\Delta x}{\dfrac{\exp\left( \dfrac{T_{S}}{2T_{2}^{*}} \right) - \exp\left( - \dfrac{T_{S}}{2T_{2}^{*}} \right)}{\dfrac{T_{S}}{T_{2}^{*}}}} \right|_{T_{S} = T_{2}^{*}} = \dfrac{\Delta x}{\exp\left( \dfrac{1}{2} \right) - \exp\left( - \dfrac{1}{2} \right)} \simeq 1.04\Delta x\]

In questa condizione la risoluzione della PSF differisce dal Fourier pixel size, \(\Delta x\), del \(4\%\). Inoltre, la risoluzione introdotta dal decadimento esponenziale per i tempi di rilassamento.

In caso, anche se il tempo di campionamento è dello stesso ordine di grandezza di \(T_{2}^{*}\), la risoluzione differisce dal caso ideale per qualche punto percentuale, provocando un allargamento di circa il \(5\%\). Questa degradazione della PSF non permette di visualizzare oggetti estremamente piccoli, ma, in molti casi pratici, può essere considerata trascurabile.

Si valuta la risoluzione spaziale con la tecnica del \emph{full width half maximum}¸ al fine di ottenere una misura alternativa della sfocatura introdotta dal filtro legato alle disomogeneità di campo. Si suppone che:

\[\exp\left( - \dfrac{T_{S}}{2T_{2}^{*}} \right) \ll 1\]

Ciò equivale a fissare la condizione:

\[T_{S} \gg 2T_{2}^{*}\]

La PSF introdotta dal decadimento \(T_{2}^{*}\) è data da:

\[h_{T_{2}^{*}}(x) = \dfrac{\exp\left( \dfrac{T_{S}}{2T_{2}^{*}} - j\pi\dfrac{x}{\Delta x} \right) - \exp\left( - \dfrac{T_{S}}{2T_{2}^{*}} + j\pi\dfrac{x}{\Delta x} \right)}{\Delta x\dfrac{T_{S}}{T_{2}^{*}} - j2\pi x}\exp\left( - \dfrac{T_{E}}{T_{2}^{*}} \right)\]

Si considera il solo numeratore:

\[\exp\left( \dfrac{T_{S}}{2T_{2}^{*}} - j\pi\dfrac{x}{\Delta x} \right) - \exp\left( - \dfrac{T_{S}}{2T_{2}^{*}} + j\pi\dfrac{x}{\Delta x} \right)\]

Per le proprietà degli esponenziali, si ha:

\[\exp\left( \dfrac{T_{S}}{2T_{2}^{*}} - j\pi\dfrac{x}{\Delta x} \right) - \exp\left( - \dfrac{T_{S}}{2T_{2}^{*}} + j\pi\dfrac{x}{\Delta x} \right) = \exp\left( \dfrac{T_{S}}{2T_{2}^{*}} \right)\exp\left( - j\pi\dfrac{x}{\Delta x} \right) - \exp\left( - \dfrac{T_{S}}{2T_{2}^{*}} \right)\exp\left( j\pi\dfrac{x}{\Delta x} \right)\]

Raccogliendo, è possibile scrivere:

\[\exp\left( - j\pi\dfrac{x}{\Delta x} \right)\left( \exp\left( \dfrac{T_{S}}{2T_{2}^{*}} \right) - \exp\left( - \dfrac{T_{S}}{2T_{2}^{*}} \right)\exp\left( j2\pi\dfrac{x}{\Delta x} \right) \right)\]

Nell'ipotesi che:

\[\exp\left( - \dfrac{T_{S}}{2T_{2}^{*}} \right) \ll 1\]

Il secondo termine può essere trascurato rispetto al primo, infatti:

\[\exp\left( \dfrac{T_{S}}{2T_{2}^{*}} \right) \gg \exp\left( - \dfrac{T_{S}}{2T_{2}^{*}} \right)\]

Per cui si ottiene:

\[\exp\left( \dfrac{T_{S}}{2T_{2}^{*}} - j\pi\dfrac{x}{\Delta x} \right) - \exp\left( - \dfrac{T_{S}}{2T_{2}^{*}} + j\pi\dfrac{x}{\Delta x} \right) \simeq \exp\left( - j\pi\dfrac{x}{\Delta x} \right)\exp\left( \dfrac{T_{S}}{2T_{2}^{*}} \right)\]

La PSF può essere scritta come:

\[h_{T_{2}^{*}}(x) \simeq \dfrac{\exp\left( - j\pi\dfrac{x}{\Delta x} \right)\exp\left( \dfrac{T_{S}}{2T_{2}^{*}} \right)}{\Delta x\dfrac{T_{S}}{T_{2}^{*}} - j2\pi x}\exp\left( - \dfrac{T_{E}}{T_{2}^{*}} \right)\]

Per valutare l'ascissa in cui la finestra è uguale a metà dell'ampiezza massima si valuta il modulo di \(h_{T_{2}^{*}}(x)\):

\[\left| h_{T_{2}^{*}}(x) \right| \simeq \dfrac{\exp\left( \dfrac{T_{S}}{2T_{2}^{*}} \right)}{\sqrt{\left( \dfrac{T_{S}\Delta x}{T_{2}^{*}} \right)^{2} + (2\pi x)^{2}}}\exp\left( - \dfrac{T_{E}}{T_{2}^{*}} \right)\]

Che può essere riscritta eseguendo il minimo comune multiplo al denominatore:

\[\left| h_{T_{2}^{*}}(x) \right| \simeq \dfrac{T_{2}^{*}\exp\left( \dfrac{T_{S}}{2T_{2}^{*}} \right)}{\sqrt{\left( T_{S}\Delta x \right)^{2} + \left( 2\pi xT_{2}^{*} \right)^{2}}}\exp\left( - \dfrac{T_{E}}{T_{2}^{*}} \right)\]

Si applica la definizione di FWHM alla finestra \(\left| h_{T_{2}^{*}}(x) \right|\):

\[h_{T_{2}^{*}}\left( \dfrac{\Delta x_{RMI}}{2} \right) = \dfrac{1}{2}h_{T_{2}^{*}}(0)\]

Da cui:

\[\left. \ \dfrac{T_{2}^{*}\exp\left( \dfrac{T_{S}}{2T_{2}^{*}} \right)}{\sqrt{\left( T_{S}\Delta x \right)^{2} + \left( 2\pi xT_{2}^{*} \right)^{2}}}\exp\left( - \dfrac{T_{E}}{T_{2}^{*}} \right) \right|_{x = \dfrac{\Delta x_{RMI}}{2}} = \dfrac{1}{2}\left. \ \dfrac{T_{2}^{*}\exp\left( \dfrac{T_{S}}{2T_{2}^{*}} \right)}{\sqrt{\left( T_{S}\Delta x \right)^{2} + \left( 2\pi xT_{2}^{*} \right)^{2}}}\exp\left( - \dfrac{T_{E}}{T_{2}^{*}} \right) \right|_{x = 0}\]

\[\dfrac{T_{2}^{*}\exp\left( \dfrac{T_{S}}{2T_{2}^{*}} \right)}{\sqrt{\left( T_{S}\Delta x \right)^{2} + \left( 2\pi\dfrac{\Delta x_{RMI}}{2}T_{2}^{*} \right)^{2}}}\exp\left( - \dfrac{T_{E}}{T_{2}^{*}} \right) = \dfrac{1}{2}\dfrac{T_{2}^{*}\exp\left( \dfrac{T_{S}}{2T_{2}^{*}} \right)}{\sqrt{\left( T_{S}\Delta x \right)^{2}}}\exp\left( - \dfrac{T_{E}}{T_{2}^{*}} \right)\]

Semplificando i termini comuni tra i due membri, si ottiene:

\[\dfrac{1}{\sqrt{\left( T_{S}\Delta x \right)^{2} + \left( \pi\Delta x_{RMI}T_{2}^{*} \right)^{2}}} = \dfrac{1}{2}\dfrac{1}{\sqrt{\left( T_{S}\Delta x \right)^{2}}} = \dfrac{1}{2}\dfrac{1}{T_{S}\Delta x}\]

Si eleva al quadrato entrambi i membri per poter ricavare \(\Delta x_{RMI}\):

\[\dfrac{1}{\left( T_{S}\Delta x \right)^{2} + \left( \pi\Delta x_{RMI}T_{2}^{*} \right)^{2}} = \dfrac{1}{4}\dfrac{1}{\left( T_{S}\Delta x \right)^{2}}\]

Passando ai reciproci, si ottiene:

\[\left( T_{S}\Delta x \right)^{2} + \left( \pi\Delta x_{RMI}T_{2}^{*} \right)^{2} = 4\left( T_{S}\Delta x \right)^{2}\]

Risolvendo rispetto a \(x_{RMI}\):

\[\left( \pi\Delta x_{RMI}T_{2}^{*} \right)^{2} = 3\left( T_{S}\Delta x \right)^{2}\]

Si applica la radice quadrata ambo i membri:

\[\pi\Delta x_{RMI}T_{2}^{*} = \sqrt{3}T_{S}\Delta x\]

\[\Delta x_{RMI} = \dfrac{\sqrt{3}}{\pi}\dfrac{T_{S}}{T_{2}^{*}}\Delta x\]

Questa relazione rappresenta uno sfocamento addizionale maggiore dovuto all'usuale campionamento e troncamento.

\subsection{Filtraggio introdotto dal rilassamento nella spin-echo}\label{filtraggio-introdotto-dal-rilassamento-nella-spin-echo}

L'effetto del filtraggio causato dal rilassamento della componente trasversa nell'esperimento spin-echo può essere meglio compreso separando i due contributi: \(T_{2}\), legato ai meccanismi intrinseci di rilassamento spin-spin, e \(T_{2}'\), dovuto alle disomogeneità statiche del campo magnetico.

Il decadimento associato a \(T_{2}'\) è simmetrico rispetto al tempo d'eco \(T_{E}\), poiché le disomogeneità di campo agiscono in modo reversibile e vengono compensate dall'impulso \(\pi\). Nel \(k\)-spazio, questo decadimento si esprime come:

\[s(k) \propto \exp\left( - \dfrac{|k|}{\overline{\gamma}GT_{2}'} \right)\]

Il decadimento \(T_{2}\), invece, è asimmetrico rispetto a \(T_{E}\), poiché inizia subito dopo l'impulso \(\pi/2\) e non viene rifocalizzato. Nel \(k\)-spazio, il suo effetto può essere approssimato da:

\[s(k) \propto \exp\left( - \dfrac{T_{E}}{T_{2}} \right)\exp\left( - \dfrac{k}{\overline{\gamma}GT_{2}} \right)\]

Il filtro equivalente, che descrive il decadimento con \(T_{2}\) e \(T_{2}'\), è ottenuto moltiplicando i due contributi:

\[H_{SE}(k) = \exp\left( - \dfrac{T_{E}}{T_{2}} \right)\exp\left( - \dfrac{k}{\overline{\gamma}GT_{2}} \right)\exp\left( - \dfrac{|k|}{\overline{\gamma}GT_{2}'} \right)\]

La sequenza spin-echo è composta da due impulsi RF, uno a \(\pi/2\) e l'altro \(\pi\). Il decadimento \(T_{2}\) inizia subito dopo il primo impulso e non è simmetrico rispetto a \(T_{E}\). Al contrario, l'effetto di \(T_{2}'\) viene rifocalizzato e si manifesta solo nella finestra di acquisizione.

\begin{figure}
\centering
\includegraphics[width=6.69306in,height=4.16944in,alt={Immagine che contiene testo, schermata, linea, diagramma Il contenuto generato dall\textquotesingle IA potrebbe non essere corretto.}]{media/10_Ric3D/image289.pdf}\caption{Figura .: Sequenza spin-echo con modulazioni dovute a \(T_{2}\) e \(T_{2}'\)}
\end{figure}

Il filtro complessivo, nel dominio dello spazio immagine, o PSF è dato dall'antitrasformata di Fourier della finestra nel \(k\)-spazio:

\[h_{SE}(x) = \mathfrak{F}^{-}\left\lbrack H_{SE}(k) \right\rbrack(x) = \mathfrak{F}^{-}\left\lbrack \exp\left( - \dfrac{k}{\overline{\gamma}GT_{2}} \right)\exp\left( - \dfrac{|k|}{\overline{\gamma}GT_{2}'} \right) \right\rbrack(x) =\]

Il termine \(\exp\left( - T_{E}/T_{2} \right)\) non dipende da \(k\), per cui è un fattore costante:

\[h_{SE}(x) = \exp\left( - \dfrac{T_{E}}{T_{2}} \right)\mathfrak{F}^{-}\left\lbrack \exp\left( - \dfrac{k}{\overline{\gamma}GT_{2}} \right)\exp\left( - \dfrac{|k|}{\overline{\gamma}GT_{2}'} \right) \right\rbrack(x)\]

Si risolve l'antitrasformata:

\[\mathfrak{F}^{-}\left\lbrack \exp\left( - \dfrac{k}{\overline{\gamma}GT_{2}} \right)\exp\left( - \dfrac{|k|}{\overline{\gamma}GT_{2}'} \right) \right\rbrack(x) = \int_{- \infty}^{+ \infty}{\exp\left( - \dfrac{k}{\overline{\gamma}GT_{2}} \right)\exp\left( - \dfrac{|k|}{\overline{\gamma}GT_{2}'} \right)\exp(j2\pi kx)dk}\]

Al fine di risolvere l'integrale, per la presenza del modulo, è necessario distinguere il caso di \(k\) negative e positive. L'integrale, in altre parole, viene scisso nella somma di due integrali:

\[\int_{- \infty}^{+ \infty}{\exp\left( - \dfrac{k}{\overline{\gamma}GT_{2}} \right)\exp\left( - \dfrac{|k|}{\overline{\gamma}GT_{2}'} \right)\exp(j2\pi kx)dk} = \int_{- \infty}^{0}{\exp\left( - \dfrac{k}{\overline{\gamma}GT_{2}} \right)\exp\left( \dfrac{k}{\overline{\gamma}GT_{2}'} \right)\exp(j2\pi kx)dk} + \int_{0}^{+ \infty}{\exp\left( - \dfrac{k}{\overline{\gamma}GT_{2}} \right)\exp\left( - \dfrac{k}{\overline{\gamma}GT_{2}'} \right)\exp(j2\pi kx)dk}\]

Il gradiente \(G\) è legato alla finestra di lettura dalla relazione:

\[T_{S} = \dfrac{W}{\overline{\gamma}G} \Leftrightarrow \ G = \dfrac{W}{\overline{\gamma}T_{S}}\]

La finestra di acquisizione si estende da \(- k_{\max}\) a \(k_{\max}\), quindi, l'estensione della finestra di acquisizione è due volte \(k_{\max}\):

\[W = 2k_{\max}\]

Da cui:

\[G = \dfrac{2k_{\max}}{\overline{\gamma}T_{S}} \Leftrightarrow k_{\max} = \dfrac{1}{2}\overline{\gamma}T_{S}G\]

Se \(k_{\max} \rightarrow \infty\) anche la finestra di acquisizione tende all'infinito. In altre parole, nella trattazione analitica per individuare la PSF dovuta ai fenomeni di rilassamento si ignora il troncamento.

Per le proprietà dell'esponenziale è possibile scrivere:

\[\int_{- \infty}^{+ \infty}{\exp\left( - \dfrac{k}{\overline{\gamma}GT_{2}} \right)\exp\left( - \dfrac{|k|}{\overline{\gamma}GT_{2}'} \right)\exp(j2\pi kx)dk} = \int_{- \infty}^{0}{\exp{\left( \dfrac{k}{\overline{\gamma}G}\left( \dfrac{1}{T_{2}'} - \dfrac{1}{T_{2}} \right) + j2\pi kx \right)dk}} + \int_{0}^{+ \infty}{\exp\left( - \dfrac{k}{\overline{\gamma}G}\left( \dfrac{1}{T_{2}} + \dfrac{1}{T_{2}'} \right) + j2\pi kx \right)dk}\]

Si risolve il primo integrale:

\[\int_{- \infty}^{0}{\exp\left( \dfrac{k}{\overline{\gamma}G}\left( \dfrac{1}{T_{2}'} - \dfrac{1}{T_{2}} \right) + j2\pi kx \right)dk} = \dfrac{1}{\dfrac{1}{\overline{\gamma}G}\left( \dfrac{1}{T_{2}'} - \dfrac{1}{T_{2}} \right) + j2\pi x}\left\lbrack \exp\left( \dfrac{k}{\overline{\gamma}G}\left( \dfrac{1}{T_{2}'} - \dfrac{1}{T_{2}} \right) + j2\pi kx \right) \right\rbrack_{- \infty}^{0}\]

In tutte le applicazioni pratiche, \(T_{2} > T_{2}'\), per cui il fattore moltiplicativo \(k\) è positivo:

\[\left( \dfrac{1}{T_{2}'} - \dfrac{1}{T_{2}} \right) > 0\]

Di conseguenza, l'esponenziale, nel limite per \(k \rightarrow - \infty\) tende a \(0\). Inoltre, \(\exp( - j2\pi kx)\) è un termine oscillante, il cui modulo è unitario. Per il teorema dei carabinieri è valida la relazione:

\[\exp\left( \dfrac{k}{\overline{\gamma}G}\left( \dfrac{1}{T_{2}'} - \dfrac{1}{T_{2}} \right) + j2\pi kx \right) = \exp\left( \dfrac{k}{\overline{\gamma}G}\left( \dfrac{1}{T_{2}'} - \dfrac{1}{T_{2}} \right) \right)\exp(j2\pi kx) \rightarrow 0,k \rightarrow - \infty\]

L'integrale ha quindi soluzione:

\[\int_{- \infty}^{0}{\exp\left( \dfrac{k}{\overline{\gamma}G}\left( \dfrac{1}{T_{2}'} - \dfrac{1}{T_{2}} \right) + j2\pi kx \right)dk} = \dfrac{1}{\dfrac{1}{\overline{\gamma}G}\left( \dfrac{1}{T_{2}'} - \dfrac{1}{T_{2}} \right) + j2\pi x}\]

È possibile ripetere un discorso analogo per il secondo integrale:

\[\int_{0}^{+ \infty}{\exp\left( - \dfrac{k}{\overline{\gamma}G}\left( \dfrac{1}{T_{2}} + \dfrac{1}{T_{2}'} \right) + j2\pi kx \right)dk} = \dfrac{1}{- \dfrac{1}{\overline{\gamma}G}\left( \dfrac{1}{T_{2}} + \dfrac{1}{T_{2}'} \right) + j2\pi x}\left\lbrack \exp\left( - \dfrac{k}{\overline{\gamma}G}\left( \dfrac{1}{T_{2}} + \dfrac{1}{T_{2}'} \right) + j2\pi kx \right) \right\rbrack_{0}^{+ \infty}\]

Il fattore moltiplicativo la variabile \(k\) è positivo e il termine \(\exp( - j2\pi kx)\) è oscillante, per cui:

\[\exp\left( - \dfrac{k}{\overline{\gamma}G}\left( \dfrac{1}{T_{2}} + \dfrac{1}{T_{2}'} \right) \right)\exp(j2\pi kx) \rightarrow 0,k \rightarrow + \infty\]

La soluzione del secondo integrale è, quindi:

\[\int_{0}^{+ \infty}{\exp\left( - \dfrac{k}{\overline{\gamma}G}\left( \dfrac{1}{T_{2}} + \dfrac{1}{T_{2}'} \right) + j2\pi kx \right)dk} = - \dfrac{1}{- \dfrac{1}{\overline{\gamma}G}\left( \dfrac{1}{T_{2}} + \dfrac{1}{T_{2}'} \right) + j2\pi x}\]

La PSF complessiva è data dalla somma dei due contributi appena valutati:

\[h_{SE}(x) = \mathfrak{F}^{-}\left\lbrack H_{SE}(k) \right\rbrack(x) = \dfrac{1}{\dfrac{1}{\overline{\gamma}G}\left( \dfrac{1}{T_{2}'} - \dfrac{1}{T_{2}} \right) + j2\pi x} - \dfrac{1}{- \dfrac{1}{\overline{\gamma}G}\left( \dfrac{1}{T_{2}} + \dfrac{1}{T_{2}'} \right) + j2\pi x} =\]

Si svolge il minimo comune multiplo:

\[\dfrac{- \dfrac{1}{\overline{\gamma}G}\left( \dfrac{1}{T_{2}} + \dfrac{1}{T_{2}'} \right) + j2\pi x - \dfrac{1}{\overline{\gamma}G}\left( \dfrac{1}{T_{2}'} - \dfrac{1}{T_{2}} \right) - j2\pi x}{\left\lbrack \dfrac{1}{\overline{\gamma}G}\left( \dfrac{1}{T_{2}'} - \dfrac{1}{T_{2}} \right) + j2\pi x \right\rbrack\left\lbrack - \dfrac{1}{\overline{\gamma}G}\left( \dfrac{1}{T_{2}} + \dfrac{1}{T_{2}'} \right) + j2\pi x \right\rbrack}\]

Si considera solamente il numeratore. I termini \(j2\pi x\) si elidono mentre i restanti possono essere sommati:

\[- \dfrac{1}{\overline{\gamma}G}\left( \dfrac{1}{T_{2}} + \dfrac{1}{T_{2}'} \right) + j2\pi x - \dfrac{1}{\overline{\gamma}G}\left( \dfrac{1}{T_{2}'} - \dfrac{1}{T_{2}} \right) - j2\pi x = - \dfrac{1}{\overline{\gamma}G}\left( \dfrac{1}{T_{2}} + \dfrac{1}{T_{2}'} \right) - \dfrac{1}{\overline{\gamma}G}\left( \dfrac{1}{T_{2}'} - \dfrac{1}{T_{2}} \right) = - \dfrac{k}{\overline{\gamma}G}\left( \dfrac{1}{T_{2}} + \dfrac{1}{T_{2}'} + \dfrac{1}{T_{2}'} - \dfrac{1}{T_{2}} \right) = - \dfrac{2}{\overline{\gamma}GT_{2}'}\]

Per il denominatore, invece, si eseguono i prodotti:

\[\left\lbrack j2\pi x + \dfrac{1}{\overline{\gamma}G}\left( \dfrac{1}{T_{2}'} - \dfrac{1}{T_{2}} \right) \right\rbrack\left\lbrack j2\pi x - \dfrac{1}{\overline{\gamma}G}\left( \dfrac{1}{T_{2}'} + \dfrac{1}{T_{2}} \right) \right\rbrack = (j2\pi x)^{2} - \dfrac{1}{{\overline{\gamma}}^{2}G^{2}}\left( \dfrac{1}{T_{2}'} - \dfrac{1}{T_{2}} \right)\left( \dfrac{1}{T_{2}'} + \dfrac{1}{T_{2}} \right) + \ j2\pi x\left\lbrack - \dfrac{1}{\overline{\gamma}G}\left( \dfrac{1}{T_{2}'} + \dfrac{1}{T_{2}} \right) + \dfrac{1}{\overline{\gamma}G}\left( \dfrac{1}{T_{2}'} - \dfrac{1}{T_{2}} \right) \right\rbrack = - 4\pi^{2}x^{2} - \dfrac{1}{{\overline{\gamma}}^{2}G^{2}}\left( \dfrac{1}{{T_{2}'}^{2}} - \dfrac{1}{T_{2}^{2}} \right) + \ j2\pi x\left\lbrack - \dfrac{1}{\overline{\gamma}G}\left( \dfrac{1}{T_{2}'} + \dfrac{1}{T_{2}} \right) + \dfrac{1}{\overline{\gamma}G}\left( \dfrac{1}{T_{2}'} - \dfrac{1}{T_{2}} \right) \right\rbrack = - 4\pi^{2}x^{2} - \dfrac{1}{{\overline{\gamma}}^{2}G^{2}}\left( \dfrac{1}{{T_{2}'}^{2}} - \dfrac{1}{T_{2}^{2}} \right) + \ j2\pi x\left\lbrack \dfrac{1}{\overline{\gamma}G}\left( - \dfrac{1}{T_{2}'} - \dfrac{1}{T_{2}} + \dfrac{1}{T_{2}'} - \dfrac{1}{T_{2}} \right) \right\rbrack = - 4\pi^{2}x^{2} - \dfrac{1}{{\overline{\gamma}}^{2}G^{2}}\left( \dfrac{1}{{T_{2}'}^{2}} - \dfrac{1}{T_{2}^{2}} \right) - \dfrac{4\pi x}{\overline{\gamma}GT_{2}}j\]

La PSF è, in definitiva, data da:

\[h_{SE}(x) = \dfrac{- \dfrac{2}{\overline{\gamma}GT_{2}'}}{- 4\pi^{2}x^{2} - \dfrac{1}{{\overline{\gamma}}^{2}G^{2}}\left( \dfrac{1}{{T_{2}'}^{2}} - \dfrac{1}{T_{2}^{2}} \right) - \dfrac{4\pi x}{\overline{\gamma}GT_{2}}j} = \dfrac{\dfrac{2}{\overline{\gamma}GT_{2}'}}{4\pi^{2}x^{2} + \dfrac{1}{{\overline{\gamma}}^{2}G^{2}}\left( \dfrac{1}{{T_{2}'}^{2}} - \dfrac{1}{T_{2}^{2}} \right) + \dfrac{4\pi x}{\overline{\gamma}GT_{2}}j}\]

In molti casi pratici, è valida l'approssimazione \(T_{2}' \ll T_{2}\); questa condizione significa che il decadimento della magnetizzazione trasversale è dominato dagli effetti \(T_{2}'\), ovvero dalle disomogeneità del campo magnetico e dagli effetti di suscettibilità, piuttosto che dalle interazioni spin-spin intrinseche (\(T_{2}\)). Di conseguenza è possibile approssimare:

\[\dfrac{1}{T_{2}'} - \dfrac{1}{T_{2}} \simeq \dfrac{1}{T_{2}'}\]

Inoltre, la parte immaginaria al denominatore dipende da \(T_{2}^{- 1}\) per cui può essere trascurata. In quest3 ipotesi la PSF, si scrive come:

\[h_{SE}(x) \simeq \dfrac{\dfrac{2}{\overline{\gamma}GT_{2}'}}{4\pi^{2}x^{2} + \dfrac{1}{{\overline{\gamma}}^{2}G^{2}{T_{2}'}^{2}}}\]

Riarrangiando i termini si scrive:

\[h_{SE}(x) \simeq \dfrac{2\overline{\gamma}GT_{2}'}{1 + 4\pi^{2}x^{2}{\overline{\gamma}}^{2}G^{2}{T_{2}'}^{2}}\]

La PSF così ottenuta è una distribuzione lorentziana centrata nell'origine.

Si valuta la risoluzione secondo l'approccio FWHM, per cui \(\Delta x_{RMI}\) deve essere tale da:

\[h_{SE}\left( \dfrac{\Delta x_{RMI}}{2} \right) = \dfrac{1}{2}h_{SE}(0)\]

Sostituendo l'espressione della PSF, si ottiene:

\[\dfrac{2\overline{\gamma}GT_{2}'}{1 + 4\pi^{2}\left( \dfrac{\Delta x_{RMI}}{2} \right)^{2}{\overline{\gamma}}^{2}G^{2}{T_{2}'}^{2}} = \dfrac{1}{2}2\overline{\gamma}GT_{2}'\]

I numeratori sono uguali, a meno del \(2\), quindi possono essere semplificati.

\[\dfrac{2}{1 + 4\pi^{2}\left( \dfrac{\Delta x_{RMI}}{2} \right)^{2}{\overline{\gamma}}^{2}G^{2}{T_{2}'}^{2}} = 1\]

Si passa ai reciproci:

\[1 + 4\pi^{2}\dfrac{\Delta x_{RMI}^{2}}{4}{\overline{\gamma}}^{2}G^{2}{T_{2}'}^{2} = 2\]

Semplificando:

\[\pi^{2}\Delta x_{RMI}^{2}{\overline{\gamma}}^{2}G^{2}{T_{2}'}^{2} = 1\]

Si isola \(\Delta x_{RMI}^{2}\):

\[\Delta x_{RMI}^{2} = \dfrac{1}{{\overline{\gamma}}^{2}G^{2}{T_{2}'}^{2}\pi^{2}}\]

Applicando la radice a entrambi i membri si ottiene la risoluzione secondo il metodo FWHM:

\[\Delta x_{RMI} = \dfrac{1}{\overline{\gamma}GT_{2}'\pi}\]

Il Fourier pixel size, per definizione è:

\[\Delta x = \dfrac{1}{W}\]

Dove \(\overline{\gamma}GT_{S} = W\), per cui:

\[\Delta x = \dfrac{1}{W} = \dfrac{1}{\overline{\gamma}GT_{S}}\]

La relazione individuata per \(\Delta x_{RMI}\) può essere riscritta in modo tale che la risoluzione spaziale per una sequenza gradient-echo, secondo l'approccio FWHM, sia funzione del Fourier pixel size \(\Delta x\). A tale scopo si divide e moltiplica per la durata della finestra di acquisizione \(T_{S}\) il secondo membro:

\[\Delta x_{RMI} = \dfrac{1}{\overline{\gamma}GT_{2}'\pi}\dfrac{T_{S}}{T_{S}}\]

Da cui:

\[\Delta x_{RMI} = \dfrac{\Delta x}{\pi}\dfrac{T_{S}}{T_{2}'}\]

La perdita di risoluzione, a cui consegue uno sfocamento aggiuntivo a quello introdotto dal campionamento e dal troncamento, in una sequenza spin-echo può essere approssimata sommando i contributi dovuti al tempo \(T_{2}\) e al tempo \(T_{2}'\):

\[FWHM_{SE} = \dfrac{\sqrt{3}}{\pi}\dfrac{T_{S}}{T_{2}}\Delta x + \dfrac{\Delta x}{\pi}\dfrac{T_{S}}{T_{2}'}\]

La perdita di risoluzione introdotta è di circa il \(5\%\) del valore ideale del Fourier pixel size, \(\Delta x\). Questo effetto, seppur di poco, limita la visione di oggetti estremamente piccoli. Al fine di limitare gli effetti dei tempi di rilassamento, si tende a strutturare l'acquisizione in una finestra dalla durata temporale limitata.

La sequenza gradient-echo, sebbene sia più veloce, non permette il recupero della disomogeneità di campo così come non consente la stima del tempo \(T_{2}\). Nonostante ciò, la sequenza gradient-echo è più precisa poiché introduce una differenza tra la risoluzione nel caso migliore, \(\Delta x\), e quella legata ai fenomeni di rilassamento minore della spin-echo.

La sequenza spin-echo, recuperando e compensando la disomogeneità di campo, permette di ottenere delle immagini meno rumorose, sebbene con risoluzione leggermente minore e con un tempo di acquisizione più lungo.

\subsubsection{Spin-echo con troncamento}\label{spin-echo-con-troncamento}

Si vuole stimare la PSF nel caso di sequenza spin-echo considerando anche gli effetti introdotti da una finestra di acquisizione limitata.

A causa del fenomeno del troncamento, il segnale acquisito varia nell'intervallo \(\left\lbrack - k_{\max};k_{\max} \right\rbrack\), dunque, l'integrale si riduce a:

\[= \int_{- k_{\max}}^{+ k_{\max}}{\exp\left( - \dfrac{k}{\overline{\gamma}GT_{2}} \right)\exp\left( - \dfrac{|k|}{\overline{\gamma}GT_{2}'} \right)\exp(j2\pi kx)dk} =\]

Al fine di risolvere l'integrale, per la presenza del modulo, è necessario distinguere il caso di \(k\) negative e positive. L'integrale, in altre parole, viene scisso nella somma di due integrali:

\[= \int_{- k_{\max}}^{0}{\exp\left( - \dfrac{k}{\overline{\gamma}GT_{2}} \right)\exp\left( \dfrac{k}{\overline{\gamma}GT_{2}'} \right)\exp(j2\pi kx)dk} + \int_{0}^{+ k_{\max}}{\exp\left( - \dfrac{k}{\overline{\gamma}GT_{2}} \right)\exp\left( - \dfrac{k}{\overline{\gamma}GT_{2}'} \right)\exp(j2\pi kx)dk} =\]

Per le proprietà degli esponenziali è possibile scrivere:

\[= \int_{- k_{\max}}^{0}{\exp\left( \left( \dfrac{1}{\overline{\gamma}GT_{2}'} - \dfrac{1}{\overline{\gamma}GT_{2}} + j2\pi x \right)k \right)dk} + \int_{0}^{+ k_{\max}}{\exp\left( \left( - \dfrac{1}{\overline{\gamma}GT_{2}'} - \dfrac{1}{\overline{\gamma}GT_{2}} + j2\pi x \right)k \right)dk} =\]

Si risolve il primo integrale; a tal fine si moltiplica e divide per il fattore moltiplicativo \(k\) nell'argomento dell'esponenziale:

\[\int_{- k_{\max}}^{0}{\exp\left( \left( \dfrac{1}{\overline{\gamma}GT_{2}'} - \dfrac{1}{\overline{\gamma}GT_{2}} + j2\pi x \right)k \right)dk} = \dfrac{1}{\dfrac{1}{\overline{\gamma}GT_{2}'} - \dfrac{1}{\overline{\gamma}GT_{2}} + j2\pi x}\left\lbrack \exp\left( \left( \dfrac{1}{\overline{\gamma}GT_{2}'} - \dfrac{1}{\overline{\gamma}GT_{2}} + j2\pi x \right)k \right) \right\rbrack_{- k_{\max}}^{0} = \dfrac{1}{\dfrac{1}{\overline{\gamma}GT_{2}'} - \dfrac{1}{\overline{\gamma}GT_{2}} + j2\pi x}\left\lbrack 1 - \exp\left( - \left( \dfrac{1}{\overline{\gamma}GT_{2}'} - \dfrac{1}{\overline{\gamma}GT_{2}} + j2\pi x \right)k_{\max} \right) \right\rbrack =\]

Il gradiente \(G\) è legato alla finestra di lettura dalla relazione:

\[T_{S} = \dfrac{W}{\overline{\gamma}G} \Leftrightarrow \ G = \dfrac{W}{\overline{\gamma}T_{S}}\]

La finestra di acquisizione si estende da \(- k_{\max}\) a \(k_{\max}\), quindi, l'estensione della finestra di acquisizione è due volte \(k_{\max}\):

\[W = 2k_{\max}\]

Da cui:

\[G = \dfrac{2k_{\max}}{\overline{\gamma}T_{S}} \Leftrightarrow k_{\max} = \dfrac{1}{2}\overline{\gamma}T_{S}G\]

Sostituendo tale risultato si ha:

\[= \dfrac{1}{\dfrac{1}{\overline{\gamma}GT_{2}'} - \dfrac{1}{\overline{\gamma}GT_{2}} + j2\pi x}\left\lbrack 1 - \exp\left( - \left( \dfrac{1}{\overline{\gamma}GT_{2}'} - \dfrac{1}{\overline{\gamma}GT_{2}} + j2\pi x \right)\dfrac{1}{2}\overline{\gamma}T_{S}G \right) \right\rbrack =\]

Svolgendo i prodotti nel termine esponenziale e raccogliendo, si ha:

\[= \dfrac{1}{\dfrac{1}{\overline{\gamma}G}\left( \dfrac{1}{T_{2}'} - \dfrac{1}{T_{2}} \right) + j2\pi x}\left\lbrack 1 - \exp\left( - \dfrac{1}{2}\left( \dfrac{T_{S}}{T_{2}'} - \dfrac{T_{S}}{T_{2}} + j2\pi\overline{\gamma}T_{S}Gx \right) \right) \right\rbrack = \dfrac{1}{\dfrac{1}{\overline{\gamma}G}\left( \dfrac{1}{T_{2}'} - \dfrac{1}{T_{2}} \right) + j2\pi x}\left\lbrack 1 - \exp\left( - \dfrac{T_{S}}{2}\left( \dfrac{1}{T_{2}'} - \dfrac{1}{T_{2}} \right) - j\pi\overline{\gamma}T_{S}Gx \right) \right\rbrack\]

Analogamente, per il secondo integrale:

\[\int_{0}^{+ k_{\max}}{\exp\left( \left( - \dfrac{1}{\overline{\gamma}GT_{2}'} - \dfrac{1}{\overline{\gamma}GT_{2}} + j2\pi x \right)k \right)dk} = \dfrac{1}{- \dfrac{1}{\overline{\gamma}GT_{2}'} - \dfrac{1}{\overline{\gamma}GT_{2}} + j2\pi x}\left\lbrack \exp\left( \left( - \dfrac{1}{\overline{\gamma}GT_{2}'} - \dfrac{1}{\overline{\gamma}GT_{2}} + j2\pi x \right)k \right) \right\rbrack_{0}^{k_{\max}} = \dfrac{1}{- \dfrac{1}{\overline{\gamma}GT_{2}'} - \dfrac{1}{\overline{\gamma}GT_{2}} + j2\pi x}\left\lbrack \exp\left( \left( - \dfrac{1}{\overline{\gamma}GT_{2}'} - \dfrac{1}{\overline{\gamma}GT_{2}} + j2\pi x \right)k_{\max} \right) - 1 \right\rbrack =\]

Dalla relazione tra \(G\) e \(k_{\max}\), si ha:

\[= \dfrac{1}{- \dfrac{1}{\overline{\gamma}G}\left( \dfrac{1}{T_{2}'} + \dfrac{1}{T_{2}} \right) + j2\pi x}\left\lbrack \exp\left( \left( - \dfrac{1}{\overline{\gamma}GT_{2}'} - \dfrac{1}{\overline{\gamma}GT_{2}} + j2\pi x \right)\dfrac{1}{2}\overline{\gamma}T_{S}G \right) - 1 \right\rbrack =\]

\[= \dfrac{1}{- \dfrac{1}{\overline{\gamma}G}\left( \dfrac{1}{T_{2}'} + \dfrac{1}{T_{2}} \right) + j2\pi x}\left\lbrack \exp\left( \dfrac{1}{2}\left( - \dfrac{T_{S}}{T_{2}'} - \dfrac{T_{S}}{T_{2}} + j2\pi\overline{\gamma}T_{S}Gx \right) \right) - 1 \right\rbrack =\]

Raccogliendo:

\[= \dfrac{1}{- \dfrac{1}{\overline{\gamma}G}\left( \dfrac{1}{T_{2}'} + \dfrac{1}{T_{2}} \right) + j2\pi x}\left\lbrack \exp\left( - \dfrac{T_{S}}{2}\left( \dfrac{1}{T_{2}'} + \dfrac{1}{T_{2}} \right) + j\pi\overline{\gamma}T_{S}Gx \right) - 1 \right\rbrack\]

Utilizzando i risultati ottenuti, l'antitrasformata si scrive:

\[\mathfrak{F}^{-}\left\lbrack \exp\left( - \dfrac{k}{\overline{\gamma}GT_{2}} \right)\exp\left( - \dfrac{|k|}{\overline{\gamma}GT_{2}'} \right) \right\rbrack(x) = \dfrac{1}{\dfrac{1}{\overline{\gamma}G}\left( \dfrac{1}{T_{2}'} - \dfrac{1}{T_{2}} \right) + j2\pi x}\left\lbrack 1 - \exp\left( - \dfrac{T_{S}}{2}\left( \dfrac{1}{T_{2}'} - \dfrac{1}{T_{2}} \right) - j\pi\overline{\gamma}T_{S}Gx \right) \right\rbrack + \dfrac{1}{- \dfrac{1}{\overline{\gamma}G}\left( \dfrac{1}{T_{2}'} + \dfrac{1}{T_{2}} \right) + j2\pi x}\left\lbrack \exp\left( - \dfrac{T_{S}}{2}\left( \dfrac{1}{T_{2}'} + \dfrac{1}{T_{2}} \right) + j\pi\overline{\gamma}T_{S}Gx \right) - 1 \right\rbrack\]

Si svolge il minimo comune multiplo:

\[\mathfrak{F}^{-}\left\lbrack \exp\left( - \dfrac{k}{\overline{\gamma}GT_{2}} \right)\exp\left( - \dfrac{|k|}{\overline{\gamma}GT_{2}'} \right) \right\rbrack(x) = \dfrac{\left\lbrack 1 - \exp\left( - \dfrac{T_{S}}{2}\left( \dfrac{1}{T_{2}'} - \dfrac{1}{T_{2}} \right) - j\pi\overline{\gamma}T_{S}Gx \right) \right\rbrack\left\lbrack j2\pi x - \dfrac{1}{\overline{\gamma}G}\left( \dfrac{1}{T_{2}'} + \dfrac{1}{T_{2}} \right) \right\rbrack}{\left\{ \left\lbrack j2\pi x + \dfrac{1}{\overline{\gamma}G}\left( \dfrac{1}{T_{2}'} - \dfrac{1}{T_{2}} \right) \right\rbrack\left\lbrack j2\pi x - \dfrac{1}{\overline{\gamma}G}\left( \dfrac{1}{T_{2}'} + \dfrac{1}{T_{2}} \right) \right\rbrack \right\}} + \dfrac{\left\lbrack \exp\left( - \dfrac{T_{S}}{2}\left( \dfrac{1}{T_{2}'} + \dfrac{1}{T_{2}} \right) + j\pi\overline{\gamma}T_{S}Gx \right) - 1 \right\rbrack\left\lbrack j2\pi x + \dfrac{1}{\overline{\gamma}G}\left( \dfrac{1}{T_{2}'} - \dfrac{1}{T_{2}} \right) \right\rbrack}{\left\{ \left\lbrack j2\pi x + \dfrac{1}{\overline{\gamma}G}\left( \dfrac{1}{T_{2}'} - \dfrac{1}{T_{2}} \right) \right\rbrack\left\lbrack j2\pi x - \dfrac{1}{\overline{\gamma}G}\left( \dfrac{1}{T_{2}'} + \dfrac{1}{T_{2}} \right) \right\rbrack \right\}}\]

Si considerano i due termini del numeratore:

\[\left\lbrack 1 - \exp\left( - \dfrac{T_{S}}{2}\left( \dfrac{1}{T_{2}'} - \dfrac{1}{T_{2}} \right) - j\pi\overline{\gamma}T_{S}Gx \right) \right\rbrack\left\lbrack j2\pi x - \dfrac{1}{\overline{\gamma}G}\left( \dfrac{1}{T_{2}'} + \dfrac{1}{T_{2}} \right) \right\rbrack = \ j2\pi x - \dfrac{1}{\overline{\gamma}G}\left( \dfrac{1}{T_{2}'} + \dfrac{1}{T_{2}} \right) - \left\lbrack j2\pi x - \dfrac{1}{\overline{\gamma}G}\left( \dfrac{1}{T_{2}'} + \dfrac{1}{T_{2}} \right) \right\rbrack\exp\left( - \dfrac{T_{S}}{2}\left( \dfrac{1}{T_{2}'} - \dfrac{1}{T_{2}} \right) - j\pi\overline{\gamma}T_{S}Gx \right)\]

\[\left\lbrack \exp\left( - \dfrac{T_{S}}{2}\left( \dfrac{1}{T_{2}'} + \dfrac{1}{T_{2}} \right) + j\pi\overline{\gamma}T_{S}Gx \right) - 1 \right\rbrack\left\lbrack j2\pi x + \dfrac{1}{\overline{\gamma}G}\left( \dfrac{1}{T_{2}'} - \dfrac{1}{T_{2}} \right) \right\rbrack = \left\lbrack j2\pi x + \dfrac{1}{\overline{\gamma}G}\left( \dfrac{1}{T_{2}'} - \dfrac{1}{T_{2}} \right) \right\rbrack\exp\left( - \dfrac{T_{S}}{2}\left( \dfrac{1}{T_{2}'} + \dfrac{1}{T_{2}} \right) + j\pi\overline{\gamma}T_{S}Gx \right) - \ j2\pi x - \dfrac{1}{\overline{\gamma}G}\left( \dfrac{1}{T_{2}'} - \dfrac{1}{T_{2}} \right)\]

Sommando i risultati, si ha:

\[j2\pi x - \dfrac{1}{\overline{\gamma}G}\left( \dfrac{1}{T_{2}'} + \dfrac{1}{T_{2}} \right) - \left\lbrack j2\pi x - \dfrac{1}{\overline{\gamma}G}\left( \dfrac{1}{T_{2}'} + \dfrac{1}{T_{2}} \right) \right\rbrack\exp\left( - \dfrac{T_{S}}{2}\left( \dfrac{1}{T_{2}'} - \dfrac{1}{T_{2}} \right) - j\pi\overline{\gamma}T_{S}Gx \right) + \left\lbrack j2\pi x + \dfrac{1}{\overline{\gamma}G}\left( \dfrac{1}{T_{2}'} - \dfrac{1}{T_{2}} \right) \right\rbrack\exp\left( - \dfrac{T_{S}}{2}\left( \dfrac{1}{T_{2}'} + \dfrac{1}{T_{2}} \right) + j\pi\overline{\gamma}T_{S}Gx \right) - \ j2\pi x - \dfrac{1}{\overline{\gamma}G}\left( \dfrac{1}{T_{2}'} - \dfrac{1}{T_{2}} \right)\]

I termini \(j2\pi x\) si elidono, inoltre:

\[- \dfrac{1}{\overline{\gamma}G}\left( \dfrac{1}{T_{2}'} + \dfrac{1}{T_{2}} \right) - \dfrac{1}{\overline{\gamma}G}\left( \dfrac{1}{T_{2}'} - \dfrac{1}{T_{2}} \right) = \dfrac{1}{\overline{\gamma}G}\left( - \dfrac{1}{T_{2}'} - \dfrac{1}{T_{2}} - \dfrac{1}{T_{2}'} + \dfrac{1}{T_{2}} \right) = - \dfrac{2}{\overline{\gamma}GT_{2}'}\]

Componendo il denominatore, si ottiene:

\[- \dfrac{2}{\overline{\gamma}GT_{2}'} - \left\lbrack j2\pi x - \dfrac{1}{\overline{\gamma}G}\left( \dfrac{1}{T_{2}'} + \dfrac{1}{T_{2}} \right) \right\rbrack\exp\left( - \dfrac{T_{S}}{2}\left( \dfrac{1}{T_{2}'} - \dfrac{1}{T_{2}} \right) - j\pi\overline{\gamma}T_{S}Gx \right) + \left\lbrack j2\pi x + \dfrac{1}{\overline{\gamma}G}\left( \dfrac{1}{T_{2}'} - \dfrac{1}{T_{2}} \right) \right\rbrack\exp\left( - \dfrac{T_{S}}{2}\left( \dfrac{1}{T_{2}'} + \dfrac{1}{T_{2}} \right) + j\pi\overline{\gamma}T_{S}Gx \right)\]

Per il denominatore, invece, si ha:

\[\left\{ \left\lbrack j2\pi x + \dfrac{1}{\overline{\gamma}G}\left( \dfrac{1}{T_{2}'} - \dfrac{1}{T_{2}} \right) \right\rbrack\left\lbrack j2\pi x - \dfrac{1}{\overline{\gamma}G}\left( \dfrac{1}{T_{2}'} + \dfrac{1}{T_{2}} \right) \right\rbrack \right\} = (j2\pi x)^{2} - \dfrac{1}{{\overline{\gamma}}^{2}G^{2}}\left( \dfrac{1}{T_{2}'} - \dfrac{1}{T_{2}} \right)\left( \dfrac{1}{T_{2}'} + \dfrac{1}{T_{2}} \right) + \ j2\pi x\left\lbrack - \dfrac{1}{\overline{\gamma}G}\left( \dfrac{1}{T_{2}'} + \dfrac{1}{T_{2}} \right) + \dfrac{1}{\overline{\gamma}G}\left( \dfrac{1}{T_{2}'} - \dfrac{1}{T_{2}} \right) \right\rbrack = - 4\pi^{2}x^{2} - \dfrac{1}{{\overline{\gamma}}^{2}G^{2}}\left( \dfrac{1}{{T_{2}'}^{2}} - \dfrac{1}{T_{2}^{2}} \right) + \ j2\pi x\left\lbrack - \dfrac{1}{\overline{\gamma}G}\left( \dfrac{1}{T_{2}'} + \dfrac{1}{T_{2}} \right) + \dfrac{1}{\overline{\gamma}G}\left( \dfrac{1}{T_{2}'} - \dfrac{1}{T_{2}} \right) \right\rbrack = - 4\pi^{2}x^{2} - \dfrac{1}{{\overline{\gamma}}^{2}G^{2}}\left( \dfrac{1}{{T_{2}'}^{2}} - \dfrac{1}{T_{2}^{2}} \right) + \ j2\pi x\left\lbrack \dfrac{1}{\overline{\gamma}G}\left( - \dfrac{1}{T_{2}'} - \dfrac{1}{T_{2}} + \dfrac{1}{T_{2}'} - \dfrac{1}{T_{2}} \right) \right\rbrack = - 4\pi^{2}x^{2} - \dfrac{1}{{\overline{\gamma}}^{2}G^{2}}\left( \dfrac{1}{{T_{2}'}^{2}} - \dfrac{1}{T_{2}^{2}} \right) - \dfrac{4\pi x}{\overline{\gamma}GT_{2}}j\]

La PSF nel dominio dello spazio-immagine è:

\[h_{SE}(x) = \dfrac{- \dfrac{2}{\overline{\gamma}GT_{2}'} - \left\lbrack j2\pi x - \dfrac{1}{\overline{\gamma}G}\left( \dfrac{1}{T_{2}'} + \dfrac{1}{T_{2}} \right) \right\rbrack\exp\left( - \dfrac{T_{S}}{2}\left( \dfrac{1}{T_{2}'} - \dfrac{1}{T_{2}} \right) - j\pi\overline{\gamma}T_{S}Gx \right)}{- 4\pi^{2}x^{2} - \dfrac{1}{{\overline{\gamma}}^{2}G^{2}}\left( \dfrac{1}{{T_{2}'}^{2}} - \dfrac{1}{T_{2}^{2}} \right) - \dfrac{4\pi x}{\overline{\gamma}GT_{2}}j} + \dfrac{\left\lbrack j2\pi x + \dfrac{1}{\overline{\gamma}G}\left( \dfrac{1}{T_{2}'} - \dfrac{1}{T_{2}} \right) \right\rbrack\exp\left( - \dfrac{T_{S}}{2}\left( \dfrac{1}{T_{2}'} + \dfrac{1}{T_{2}} \right) + j\pi\overline{\gamma}T_{S}Gx \right)}{- 4\pi^{2}x^{2} - \dfrac{1}{{\overline{\gamma}}^{2}G^{2}}\left( \dfrac{1}{{T_{2}'}^{2}} - \dfrac{1}{T_{2}^{2}} \right) - \dfrac{4\pi x}{\overline{\gamma}GT_{2}}j}\]

Al denominatore la parte immaginaria è pesata per \(T_{2}\), per cui può essere trascurata:

\[- 4\pi^{2}x^{2} - \dfrac{1}{{\overline{\gamma}}^{2}G^{2}}\left( \dfrac{1}{{T_{2}'}^{2}} - \dfrac{1}{T_{2}^{2}} \right) - \dfrac{4\pi x}{\overline{\gamma}GT_{2}} \simeq - 4\pi^{2}x^{2} - \dfrac{1}{{\overline{\gamma}}^{2}G^{2}{T_{2}'}^{2}}\]

Per quanto riguarda il numeratore, si considera il termine:

\[\left\lbrack j2\pi x - \dfrac{1}{\overline{\gamma}G}\left( \dfrac{1}{T_{2}'} + \dfrac{1}{T_{2}} \right) \right\rbrack\exp\left( - \dfrac{T_{S}}{2}\left( \dfrac{1}{T_{2}'} - \dfrac{1}{T_{2}} \right) - j\pi\overline{\gamma}T_{S}Gx \right) \simeq \left( j2\pi x - \dfrac{1}{\overline{\gamma}GT_{2}'} \right)\exp\left( - \dfrac{1}{2}\dfrac{T_{S}}{T_{2}'} - j\pi\overline{\gamma}T_{S}Gx \right)\]

Per il secondo termine, si ottiene:

\[\left\lbrack j2\pi x + \dfrac{1}{\overline{\gamma}G}\left( \dfrac{1}{T_{2}'} - \dfrac{1}{T_{2}} \right) \right\rbrack\exp\left( - \dfrac{T_{S}}{2}\left( \dfrac{1}{T_{2}'} + \dfrac{1}{T_{2}} \right) + j\pi\overline{\gamma}T_{S}Gx \right) \simeq \left( \ j2\pi x + \dfrac{1}{\overline{\gamma}GT_{2}'} \right)\exp\left( - \dfrac{T_{S}}{2T_{2}'} + j\pi\overline{\gamma}T_{S}Gx \right)\]

La PSF, in questa ipotesi, è data da:

\[h_{SE}(x) \simeq \dfrac{- \dfrac{2}{\overline{\gamma}GT_{2}'} + \left( j2\pi x - \dfrac{1}{\overline{\gamma}GT_{2}'} \right)\exp\left( - \dfrac{1}{2}\dfrac{T_{S}}{T_{2}'} - j\pi\overline{\gamma}T_{S}Gx \right) + \left( \ j2\pi x + \dfrac{1}{\overline{\gamma}GT_{2}'} \right)\exp\left( - \dfrac{T_{S}}{2T_{2}'} + j\pi\overline{\gamma}T_{S}Gx \right)}{- \dfrac{1}{{\overline{\gamma}}^{2}G^{2}{T_{2}'}^{2}} - 4\pi^{2}x^{2}}\]

Si suppone che la finestra di acquisizione \(T_{S}\) sia molto maggiore del tempo di rilassamento \(T_{2}'\); ciò equivale ad affermare che il segnale è stato acquisito per un tempo così lungo che può essere considerato decaduto. In questa condizione

\[\exp\left( - \dfrac{1}{2}\dfrac{T_{S}}{T_{2}'} - j\pi\overline{\gamma}T_{S}Gx \right) = \exp\left( - \dfrac{1}{2}\dfrac{T_{S}}{T_{2}'} \right)\exp\left( - j\pi\overline{\gamma}T_{S}Gx \right)\]

Per valutare cosa accade nel limite \(T_{S} \rightarrow \infty\) si ricorda che \(\exp\left( - j\pi\overline{\gamma}T_{S}Gx \right)\) è un termine oscillante, il cui modulo è unitario. Per il teorema dei carabinieri è valida la relazione:

\[\exp\left( - \dfrac{1}{2}\dfrac{T_{S}}{T_{2}'} \right)\exp\left( - j\pi\overline{\gamma}T_{S}Gx \right) \rightarrow 0,T_{S} \gg T_{2}'\]

Analogamente:

\[\exp\left( - \dfrac{1}{2}\dfrac{T_{S}}{T_{2}'} \right)\exp\left( j\pi\overline{\gamma}T_{S}Gx \right) \rightarrow 0,T_{S} \gg T_{2}'\]

La PSF, aggiungendo questa ulteriore ipotesi, si riduce a:

\[h_{SE}(x) = \dfrac{- \dfrac{2}{\overline{\gamma}GT_{2}'}}{- \dfrac{1}{{\overline{\gamma}}^{2}G^{2}{T_{2}'}^{2}} - 4\pi^{2}x^{2}}\]

Riarragiando i termini si può scrivere:

\[h_{SE}(x) = \dfrac{2\overline{\gamma}GT_{2}'}{1 - 4\pi^{2}x^{2}{\overline{\gamma}}^{2}G^{2}{T_{2}'}^{2}}\]

Si è ottenuto lo stesso risultato del caso precedente.

\subsection{Zero filled}\label{zero-filled}

Un metodo di interpolazione molto utilizzato nella pratica, al fine di ricostruire un segnale di cui si conosce un numero di finito di campioni mediante la trasformata di Fourier, consiste nell'aggiunta di zeri così da ottenere un maggior numero sui quali ricostruire l'immagine.

Si suppone di aver campionato nel \(k\)-spazio, da un valore \(- k_{\min}\) a \(+ k_{\max}\) con un passo di campionamento \(\Delta k\). Il campionamento del segnale, inoltre, avviene in una finestra temporale di lunghezza \(W\). Mediante la trasformata inversa di Fourier è possibile ricostruire la densità protonica \(\widehat{\rho}(x)\) su \(N\) punti pari mediante la relazione:

\[\widehat{\rho}(x) = \Delta k\sum_{p = - \dfrac{N}{2}}^{\dfrac{N}{2} - 1}{s(p\Delta k)\exp\left( \dfrac{j2\pi pq}{N} \right)}\]

Dove \(\Delta x = W^{- 1}\) e il \(FOV = L = \Delta k^{- 1}\). L'ampiezza dell'oggetto da ricostruire dipende, in ultima analisi, da \(\Delta k\).

Il Fourier pixel size, \(\Delta x\), è legato alla risoluzione spaziale tramite la PSF ed è dovuto essenzialmente al troncamento, al campionamento e l'applicazione di eventuali filtri per ridurre gli errori di rinning. La risoluzione spaziale \(\Delta x_{RMI}\) fornisce un'informazione sulla minima dimensione che devono possedere due entità affinché siano distinti nell'immagine.

La risoluzione spaziale \(\Delta x\) è legata anche al numero di campioni acquisiti nel \(k\)-spazio, tramite la relazione:

\[\Delta x = \dfrac{1}{N\Delta k}\]

Per migliorare la risoluzione è necessario aumentare il numero di punti o ridurre l'intervallo di campionamento nel \(k\)-spazio. Tuttavia, la durata del campionamento è legata a vari fattori della sequenza, ad esempio, nella gradient-echo, la durata complessiva dell'acquisizione \(W\) deve essere tale da non sentire gli effetti del rilassamento.

Una metodica utilizzata per migliorare la risoluzione consiste nell'aumentare, mediante appositi algoritmi digitali, la finestra di acquisizione \(W\) un certo numero di zeri del segnale \(s(k)\). Tale algoritmo è detto zero filling.

Nella pratica, generalmente, non si estende la finestra di acquisizione \(W\) oltre un raddoppio di quella in cui sono acquisiti i campioni. Se la finestra \(W\) raddoppia, la risoluzione spaziale si dimezza:

\[\Delta x = \left. \ \dfrac{1}{NW} \right|_{W = 2W} = \dfrac{1}{2NW} = \dfrac{\Delta x}{2}\]

Il set di dati acquisiti, a valle dell'introduzione dello zero filled, si scrive come:

\[{\widehat{s}}_{0}(p\Delta k) = \left\{ \begin{aligned}
0,\ \  & - N \leq p \leq - \dfrac{N}{2} - 1 \\
s(p\Delta k),\ \  & - \dfrac{N}{2} \leq p \leq \dfrac{N}{2} - 1 \\
0,\ \  & \dfrac{N}{2} \leq p \leq N - 1
\end{aligned} \right.\ \]

In ogni caso, il numero \(N\) dei campioni deve essere scelto in modo da essere un numero pari, se possibile potenza di \(2\).

Il segnale \({\widehat{s}}_{0}(p\Delta k)0\), nell'intervallo di campionamento fisicamente realizzato, corrisponde al segnale realmente acquisito; mentre all'esterno di questo intervallo si aggiungono tanti zeri, centrati si \(p\Delta k\), sia alla sinistra che alla destra dell'intervallo \(\lbrack - N/2;N/2 - 1\rbrack\) così da avere una finestra di ampiezza doppia.

\begin{figure}
\centering
\includegraphics[width=6.69182in,height=2.34091in,alt={Immagine che contiene testo, linea, Diagramma, schermata Il contenuto generato dall\textquotesingle IA potrebbe non essere corretto.}]{media/10_Ric3D/image290.pdf}\caption{Figura .: Segnale acquisito ed estensione con zero padding con \(N = 32\)}
\end{figure}

Si applica ora la trasformata discreta di Fourier inversa al segnale \({\widehat{s}}_{0}(p\Delta k)\), a valle dello zero filling. In questo modo si ricostruisce un segnale proporzionale alla densità protonica, valutata in \(q\Delta x/2\):

\[{\widehat{\rho}}_{0}\left( q\dfrac{\Delta x}{2} \right) = \dfrac{1}{2N}\sum_{p = - N}^{N - 1}{{\widehat{s}}_{0}(p\Delta k)\exp\left( j2\pi\dfrac{pq}{2N} \right)}\]

La normalizzazione per \(2N\) è dovuta al fatto che il numero di punti su cui si calcola la trasformata è raddoppiato. Il segnale \({\widehat{s}}_{0}(p\Delta k)\) è null' all'esterno dell'intervallo \(\lbrack - N/2;N/2 - 1\rbrack\) per cui è possibile ridurre gli indici su cui calcolare la sommatoria:

\[{\widehat{\rho}}_{0}\left( q\dfrac{\Delta x}{2} \right) = \dfrac{1}{2N}\sum_{p = - \dfrac{N}{2}}^{\dfrac{N}{2} - 1}{{\widehat{s}}_{0}(p\Delta k)\exp\left( j2\pi\dfrac{pq}{2N} \right)}\]

Dalla sequenza di dati acquisiti nel \(k\)-spazio, ottenuta mediante lo zero filling, è possibile ricostruire due immagini, una per \(q\) pari e l'altra per \(q\) dispari.

Si considerano i multipli peri della quantità \(q = 2r\), dove \(r \in \lbrack - N/2;N/2 - 1\rbrack\). L'immagine ricostruita su questi punti è data da:

\[\left. \ {\widehat{\rho}}_{0}\left( q\dfrac{\Delta x}{2} \right) \right|_{q = 2r} = {\widehat{\rho}}_{0}(r\Delta x) = \dfrac{1}{2N}\sum_{p = - \dfrac{N}{2}}^{\dfrac{N}{2} - 1}{{\widehat{s}}_{0}(p\Delta k)\exp\left( j2\pi\dfrac{rq}{N} \right)}\]

L'immagine ricostruita coincide con l'immagine che sarebbe stata ricostruita in assenza di zero filled, scalata di fattore \(1/2\):

\[{\widehat{\rho}}_{0}(q\Delta x) = \dfrac{1}{2}\widehat{\rho}(r\Delta x)\]

L'immagine ottenuta coincide con la densità protonica che sarebbe stata ricostruita in assenza di zero padding, \({\widehat{\rho}}_{0}(q\Delta x)\), e contiene solamente \(N\) punti. In altre parole, la trasformata inversa di Fourier a vale dello zero padding coincide, nei punti multipli pari di \(\Delta x/2\), con la trasformata che si avrebbe senza l'applicazione dello zero filling, a meno di un fattore di scala \(1/2\).

Si analizza, ora, il comportamento dell'antitrasformata della sequenza con zero filling nei punti dispari di \(\Delta x/2\). A tale scopo si scrive \(p = 2r + 1\); sottiene:

\[\left. \ {\widehat{\rho}}_{0}\left( q\dfrac{\Delta x}{2} \right) \right|_{q = 2r} = {\widehat{\rho}}_{0}\left( (2r + 1)\dfrac{\Delta x}{2} \right) = \dfrac{1}{2N}\sum_{p = - \dfrac{N}{2}}^{\dfrac{N}{2} - 1}{{\widehat{s}}_{0}(p\Delta k)\exp\left( j2\pi\dfrac{(2r + 1)q}{2N} \right)} = \dfrac{1}{2N}\sum_{p = - \dfrac{N}{2}}^{\dfrac{N}{2} - 1}{{\widehat{s}}_{0}(p\Delta k)\exp\left( j\pi\dfrac{(2r + 1)q}{N} \right)}\]

Per le proprietà degli esponenziali, si ha:

\[= \dfrac{1}{2N}\sum_{p = - \dfrac{N}{2}}^{\dfrac{N}{2} - 1}{\left( {\widehat{s}}_{0}(p\Delta k)\exp\left( j\pi\dfrac{p}{N} \right) \right)\exp\left( j2\pi\dfrac{rp}{N} \right)}\]

Per il teorema della traslazione, è possibile scrivere:

\[{\widehat{\rho}}_{0}\left( (2r + 1)\dfrac{\Delta x}{2} \right) = \dfrac{1}{2}\widehat{\rho}\left( r\Delta x + \dfrac{\Delta x}{2} \right)\]

La densità protonica ricostruita con lo zero filling, per i multipli dispari di \(\Delta x/2\), coincide con la densità protonica ricostruita su \(N\) punti e traslata di \(\Delta x/2\) rispetto al punto \(r\), e scalta di un fattore \(1/2\).

Mediante la tecnica dello zero filling, la densità protonica è sia nota nei punti \(r\Delta x\) sia nei punti \((2r + 1)\Delta x/2\), permettendo così una ricostruzione migliore dell'immagine.

Si osservi che l'interpolazione con zero filling non varia il FOV e il passo di campionamento effettivo nel \(k\)-spazio, \(\Delta k\), ma varia il campionamento nello spazio-immagine nell'intervallo \(\lbrack - L/2;L/2\rbrack\). In altre parole, lo zero padding non aggiunge \emph{nuova informazione indipendente} sul segnale, ovvero non migliora la risoluzione intrinseca data dalla larghezza di banda campionata, ma agisce come un'\textbf{interpolazione} nel dominio spaziale. Aumentando il numero di punti nella Trasformata Inversa di Fourier, otteniamo un'immagine con più pixel, che appare più dettagliata e meno "a blocchi" o "scalettata", perché il segnale è valutato anche nei punti intermedi.

\begin{figure}
\centering
\includegraphics[width=6.68958in,height=5.33333in,alt={Immagine che contiene testo, diagramma, linea, Diagramma Il contenuto generato dall\textquotesingle IA potrebbe non essere corretto.}]{media/10_Ric3D/image291.pdf}\caption{Figura .: Ricostruzione del segnale tramite la trasformata inversa di Fourier a valle dello zero filled}
\end{figure}

Si considera un voxel con dimensione \(\Delta x\) e un oggetto, contenuto nel voxel, prossimo a metà dell'intervallo \(\lbrack 0;\Delta x\rbrack\) molto piccolo.

\begin{figure}
\centering
\includegraphics[width=2.31508in,height=2.78731in,alt={Immagine che contiene testo, Rettangolo, cerchio, schermata Il contenuto generato dall\textquotesingle IA potrebbe non essere corretto.}]{media/10_Ric3D/image292.pdf}\caption{Figura .: Voxel con piccolo oggetto}
\end{figure}

A causa del campionamento e del troncamento nel \(k\)-spazio, la densità protonica dell'oggetto di cui si vuole eseguire l'imaging, \(\rho(x)\), è filtrata dalla PSF dovuta al troncamento e al campionamento, \(h_{WS}(x)\):

\[\widehat{\rho}(x) = \rho(x)*h_{WS}(x)\]

La densità protonica ricostruita è una versione filtrata dell'oggetto a causa della PSF. Nel caso migliore, la PSF ha un'estensione uguale al Fourier pixel size, \(\Delta x\).

Si applica questo concetto all'oggetto prima considerato, posizione nei pressi di \(\Delta x/2\). Durante la ricostruzione, a causa del filtraggio e del campionamento, l'oggetto è slargato per la convoluzione con la PSF nel dominio dello spazio-immagine. Inoltre, in assenza di zero padding, il campione ricostruito è posizione nei multipli interi di \(\Delta x\). Ne consegue che i campioni più vicini all'oggetto considerato si trovano per \(x = 0\) e \(x = \Delta x\); ciò determina una sottostima della densità protonica del voxel posizionato tra due campioni.

\begin{figure}
\centering
\includegraphics[width=2.36364in,height=2.7741in,alt={Immagine che contiene testo, Rettangolo, schermata, design Il contenuto generato dall\textquotesingle IA potrebbe non essere corretto.}]{media/10_Ric3D/image293.pdf}\caption{Figura .: Effetto della ricostruzione in assenza di zero filled}
\end{figure}

Mediante l'introduzione dell'interpolazione zero filled l'immagine viene ricostruita anche nei punti multipli di \(\Delta x/2\), migliorando la stima della densità protonica del voxel.

\begin{figure}
\centering
\includegraphics[width=6.69208in,height=3.82576in,alt={Immagine che contiene testo, diagramma, Diagramma, linea Il contenuto generato dall\textquotesingle IA potrebbe non essere corretto.}]{media/10_Ric3D/image294.pdf}\caption{Figura .: Sottostima dell'intensità dell'oggetto a causa dell'assenza di interpolazione}
\end{figure}

Si può dimostrare che in assenza di zero filled e con PSF di tipo \(sinc\) la perdita di segnale può essere anche del \(30\% \div 40\%\); di conseguenza, gli oggetti di piccole dimensioni sono attenutati, in termini di intensità cromatica nell'immagine, anche del \(40\%\)m rendendo questi ultimi poco visibili.

Lo zero filled permette di osservare anche oggetti più piccoli, riducendo gli effetti dello sfocamento sull'immagine ricostruita.

\begin{figure}
\centering
\includegraphics[width=6.69306in,height=3.67361in,alt={Immagine che contiene testo, linea, diagramma, Diagramma Il contenuto generato dall\textquotesingle IA potrebbe non essere corretto.}]{media/10_Ric3D/image295.pdf}\caption{Figura .: Ricostruzione con zero padding}
\end{figure}

Si osservi che la PSF non cambia a valle dell'applicazione della tecnica dello zero filled poiché quest'ultima non introduce nessun effetto filtrante. La PSF legata al campionamento e al troncamento è un limite intrinseco della risonanza magnetica e della strumentazione di elaborazione.

\subsection{Partial Fourier imaging}\label{partial-fourier-imaging}

In alcuni contesti pratici, come quando si vuole, ad esempio, minimizzare i tempi di imaging di una sequenza (o in altre occasioni) non si campiona il \(k\)-spazio simmetricamente. In questi casi si acquisisce un \(k\)-spazio asimmetrico nella direzione del phase encoding, così da, appunto, ridurre i tempi di acquisizione, avendo un minor numero di ripetizioni.

Le tecniche di ricostruzione con \(k\)-spazio acquisito in modo asimmetrico prevedono che il campionamento lungo la direzione \(k_{PE}\) rispetti il criterio di Nyquist solamente per metà del \(k\)-spazio, spesso la metà positiva con \(k_{R} > 0\), mentre il campionamento nell'altro semipiano del \(k\)-spazio, spesso per le direzioni di codifica di fase negative \(k_{PE} < 0\), è solo parziale. Ciò permette di ridurre il numero di incrementi del gradiente di codifica di fase.

\begin{figure}
\centering
\includegraphics[width=6.69306in,height=1.06061in,alt={Immagine che contiene linea, Diagramma, schermata, diagramma Il contenuto generato dall\textquotesingle IA potrebbe non essere corretto.}]{media/10_Ric3D/image296.pdf}\caption{Figura .: Distribuzione dei campioni asimmetrici lungo la direzione di codifica di fase}
\end{figure}

L'operazione di acquisizione di un solo semipiano del \(k\)-spazio nella direzione di codifica di fase è possibile, infatti, dal punto di vista teorico il segnale registrato gode della proprietà di hermitianità:

\[s( - k) = s^{*}(k)\]

Questa proprietà discende dal fatto che il segnale \(s(k)\) è la trasformata di Fourier della densità protonica \(\rho(x)\) la quale è una funzione reale.

I campioni nella direzione della codifica di fase positiva non sono indipendente da quelli negativi, quindi, noto l'andamento per \(k_{PE} > 0\) è possibile ricavare, per simmetria, le componenti negative.

In linea teorica è sia possibile dimezzare i tempi di acquisizione, sia acquisire il doppio delle informazioni per la codifica di fase se vengono campionate \(k_{PE}\) positive e negative. Con quest'ultima soluzione si ottiene un FOV più largo lungo l'asse di codifica di fase.

L'approssimazione di funzione hermitiana per \(s(k)\) non è verificata, infatti, a causa delle disomogeneità di campo magnetico, o sfasamento dei ricettori, la densità protonica ricostruita, antitrasformando il segnale \(s(k)\) nel \(k\)-spazio, non è reale ma immaginaria, la quale è utile per analizzare il tipo di problematica presente nel sistema di ricezione e le disomogeneità di campo.

Si suppone che la relazione di hermitianità sia valida, ovvero \(s( - k) = s^{*}(k)\). In questa condizione si ricostruisce una densità protonica reale, che non coincide con quella effettivamente misurata. Ciò porta a errori di ricostruzione.

Non potendo campionare solamente il semipiano positivo del \(k\)-spazio, nella direzione di codifica di fase, si adotta una strategia di ricostruzione parziale in cui si acquisisce metà del \(k\)-spazio nella direzione positiva del phase econding e alcune righe nella direzione negativa.

\begin{figure}
\centering
\includegraphics[width=5.85625in,height=5.85625in,alt={Immagine che contiene testo, schermata, linea, Parallelo Il contenuto generato dall\textquotesingle IA potrebbe non essere corretto.}]{media/10_Ric3D/image297.pdf}\caption{Figura .: Righe del k-spazio con campionamento parziale lungo l'asse di cofica di fase}
\end{figure}

Tipicamente si indica con \(n_{+}\) il numero delle righe positive e con \(m_{-}\) il numero delle righe negative. Di solito i due parametri sono scelti in modo che \(n_{+} \gg n_{-}\). Ad esempio, se si acquisiscono \(128\) righe positive, quelle negative potrebbero essere \(16\), così di collezionare anche alcune informazioni sul versante negativo.

Un possibile algoritmo per la ricostruzione della densità protonica è composto da una serie di istruzioni iterative con cui si arriva alla migliore stima della densità protonica, mediante approssimazioni successive.

L'algoritmo si basa sull'ipotesi che, la regione centrale del \(k\)-spazio ovvero nell'intervallo \(\left\lbrack - n_{-};n_{-} - 1 \right\rbrack\), sia una buona approssimazione della fase dell'immagine ricostruita. Dai punti centrali è, di conseguenza, possibile ricavare la fase \(\phi\) dell'immagine ricostruita mediante una DFT inversa. Per incrementare la precisione della risoluzione è possibile utilizzare anche l'interpolazione zero filled.

\begin{figure}
\centering
\includegraphics[width=2.99097in,height=5.25in,alt={Immagine che contiene testo, schermata, design Il contenuto generato dall\textquotesingle IA potrebbe non essere corretto.}]{media/10_Ric3D/image298.pdf}\caption{Figura .: Sezione del k-spazio utilizzato per la ricostruzione della fase}
\end{figure}

Il primo passo dell'algoritmo prevede il troncamento nella finestra \(\left\lbrack - n_{-};n_{-} - 1 \right\rbrack\) e lo zero padding del set di dati misurato nell'intervallo simmetrico \(\left\lbrack - n_{-};n_{+} - 1 \right\rbrack\), ottenendo il segnale:

\[{\widehat{s}}_{\phi}(p\Delta k) = \left\{ \begin{aligned}
0,\ \  & - n_{-} \leq p \leq - n_{-} - 1 \\
s(p\Delta k),\ \  & - n_{-} \leq p \leq n_{-} - 1 \\
0,\ \  & n_{-} \leq p \leq n_{+} - 1
\end{aligned} \right.\ \]

Mediante lo zero padding si costruisce una finestra di acquisizione simmetrica, quindi, è prevista l'aggiunta di un numero maggiore di zeri nella porzione con \(k_{PE}\) negativi.

Dai campioni del segnale \({\widehat{s}}_{0}\) si ricostruisce la densità protonica, esplicitando modulo e fase, mediante la IDFT:

\[{\widehat{\rho}}_{\phi}(q\Delta x) = IDFT\left\{ {\widehat{s}}_{\phi}(p\Delta k) \right\}(q\Delta x)\]

Indicando la \(IDFT\) con \(\mathfrak{D}^{- 1}\), la fase della densità protonica ricostruita è data da:

\[\phi(q\Delta x) = \angle{\widehat{\rho}}_{\phi}(q\Delta x) = \angle\mathfrak{D}^{- 1}\left\{ {\widehat{s}}_{\phi}(p\Delta k) \right\}\]

L'immagine di partenza \({\widehat{\rho}}_{\phi}\) può essere scritta in termini di modulo e fase:

\[{\widehat{\rho}}_{\phi}(q\Delta x) = \left| \mathfrak{D}^{- 1}\left\{ {\widehat{s}}_{\phi}(p\Delta k) \right\} \right|\exp\left( j\angle\mathfrak{D}^{- 1}\left\{ {\widehat{s}}_{\phi}(p\Delta k) \right\} \right)\]

Inizializzata una variabile di conteggio \(j = 0\), un'immagine iniziale \(\rho(x)\) è ottenuta da una versione zero padded dei dati misurati, estesi a \(2n_{+}\). Detta \(s_{0}(p\Delta k)\) il segnale nel \(k\)-spazio alla prima iterazione, dato dal segnale misurato \(s_{m}\) in \(\left\lbrack - n_{-};n_{+} - 1 \right\rbrack\) e da zero nell'intervallo \(\left\lbrack - n_{+}; - n_{-} - 1 \right\rbrack\) per lo zero padding:

\[s_{0}(p\Delta x) = \left\{ \begin{aligned}
s_{m}(p\Delta k),\ \  & - n_{-} \leq p \leq n_{+} - 1 \\
0,\ \  & - n_{+} \leq p \leq - n_{-} - 1
\end{aligned} \right.\ \]

Gli zeri sono posizionati in modo da ottenere una finestra simmetrica \(\left\lbrack - n_{+};n_{+} \right\rbrack\), ovvero una finestra simmetrica rispetto l'origine a meno di un campione.

Con il segnale \(s_{0}(p\Delta x)\) del \(k\)-spazio, si ricostruisce l'immagine mediante IDFT. Sia \({\widehat{\rho}}_{0}\) l'immagine ricostruita alla prima iterazione:

\[{\widehat{\rho}}_{0}(q\Delta x) = \mathfrak{D}^{- 1}\left\{ s_{0}(p\Delta x) \right\}\]

Da questo passo inizia l'algoritmo iterativo. In particolare, l'immagine all'iterazione \(j + 1\)-esima è ottenuta a partire dall'immagine \(j\)-esima ricostruita nell'iterazione precedente e la fase iniziale calcolata nel primo passo.

Nello specifico, dall'immagine \({\widehat{\rho}}_{j}\) del \(j\)-esimo passo, si considera solamente il modulo, poiché la fase \(\phi(q\Delta x)\), calcolata nel primo passo, è supposta essere quella che meglio approssima la fase reale. All'iterazione \(j + 1\)-esima, l'immagine ricostruita è data da:

\[{\widetilde{\rho}}_{j + 1}(q\Delta x) = \left| {\widehat{p}}_{j}(q\Delta x) \right|\exp\left( j\angle\mathfrak{D}^{- 1}\left\{ {\widehat{s}}_{\phi}(p\Delta k) \right\} \right) = \left| {\widehat{p}}_{j}(q\Delta x) \right|\exp\left( j\phi\left( (p\Delta k) \right) \right)\]

Questa immagine intermedia è trasformata secondo Fourier al fine di creare un set di dati intermedi del \(k\)-spazio, ricostruendo i campioni del segnale \(s(k)\). Applicando la DFT all'immagine nello step \(j + 1\)-esimo, si ottiene il segnale \({\widehat{s}}_{j + 1}(p\Delta k)\) nel \(k\)-spazio:

\[{\widetilde{s}}_{j + 1}(p\Delta k)\mathfrak{= D}\left\{ {\widetilde{\rho}}_{j + 1}(q\Delta x) \right\}(p\Delta k)\]

Il segnale ricostruito nello step \(j + 1\)-esimo, \({\widetilde{s}}_{j + 1}(p\Delta k)\), a differenza del segnale misurato \(s_{m}(p\Delta x)\) e con zero padding, \(s_{0}(p\Delta x)\), non ha zeri nell'intervallo \(\left\lbrack - n_{+};n_{+} - 1 \right\rbrack\) poiché l'immagine ricostruita \({\widehat{p}}_{j + 1}(q\Delta x)\) non coincide con l'immagine iniziale \({\widehat{\rho}}_{0}(q\Delta x)\).

I dati complessi dell'iterazione precedente,\(s_{j}(p\Delta k)\) contenuti nell'intervallo di padding \(\left\lbrack - n_{+}\Delta k;n_{-}\Delta k \right\rbrack\) sono sostituiti con quelli dell'iterazione corrente \(j + 1\)-esima \({\widetilde{s}}_{j + 1}\); mentre i campioni nell'intervallo di misura \(\left\lbrack - n_{-};n_{+} - 1 \right\rbrack\), dov'è contenuto il segnale acquisito, non sono modificati. In altre parole, il segnale ricostruito allo step \(j + 1\)-esimo è:

\[s_{j + 1}(p\Delta x) = \left\{ \begin{aligned}
s_{m}(p\Delta k),\ \  & - n_{-} \leq p \leq n_{+} - 1 \\
{\widetilde{s}}_{j + 1}(p\Delta k),\ \  & - n_{+} \leq p \leq - n_{-} - 1
\end{aligned} \right.\ \]

\begin{figure}
\centering
\includegraphics[width=6.69306in,height=1.58403in,alt={Immagine che contiene schermata, testo, linea, Diagramma Il contenuto generato dall\textquotesingle IA potrebbe non essere corretto.}]{media/10_Ric3D/image299.pdf}\caption{Figura .: Aggiornamento della pozione negativa di \(k_{PE}\)}
\end{figure}

In questo modo i campioni del \(k\)-spazio nella direzione di codifica di fase negativa, \(K_{PE} < 0\), sono corrette e completate a ogni iterazione.

Antitrasformato l'ultimo segnale campionato nel \(k\)-spazio ottenuto, \(s_{j + 1}(p\Delta x)\), si ottiene la densità protonica dell'iterazione \(j + 1\)-esima:

\[{\widehat{\rho}}_{j + 1}(q\Delta x) = \mathfrak{D}^{- 1}\left\{ s_{j + 1}(p\Delta x) \right\}\]

Se le immagini ricostruite al passo \(j\)-esimo e \(j + 1\)-esimo sono tali che il modula della differenza sia sufficientemente piccola, ovvero al di sotto di una certa soglia:

\[\left| {\widehat{\rho}}_{j + 1}(q\Delta x) - {\widehat{\rho}}_{j}(q\Delta x) \right| < \varepsilon\]

L'algoritmo ha termine, altrimenti si procede col passo \(j + 2\)-esimo, definendo la nuova immagine come:

\[{\widetilde{\rho}}_{j + 2}(q\Delta x) = \left| {\widehat{p}}_{j + 1}(q\Delta x) \right|\exp\left( j\phi\left( (p\Delta k) \right) \right)\]

Questo algoritmo si basa, in definitva, sulla convergenza della sequenza di immagini \({\widetilde{\rho}}_{j}\) all'immagine reale per \(j \rightarrow \infty\).

Se la condizione \(\left| {\widehat{\rho}}_{j + 1}(q\Delta x) - {\widehat{\rho}}_{j}(q\Delta x) \right| < \varepsilon\) è soddisfatta \({\widehat{\rho}}_{j + 1}(q\Delta x)\) è l'immagine ricostruita finale.

La convergenza di \({\widehat{\rho}}_{j}(q\Delta x) \rightarrow \rho(q\Delta x),j \rightarrow \infty\) non è dimostrata analiticamente, tuttavia, nella pratica, si è visto che l'uso di questo algoritmo permette di ricostruire in modo sufficientemente affidabile l'immagine reale.

Per questo algoritmo, in definitiva, è importante la sostituzione dei campioni di \(s_{j + 1}\) per ogni iterazione, con i campioni negati nel \(k\)-spazio di \({\widetilde{s}}_{j + 1}\), al fine di convergere all'immagine reale.

\begin{figure}
\centering
\includegraphics[width=6.69306in,height=2.28333in,alt={Immagine che contiene schermata, Rettangolo Il contenuto generato dall\textquotesingle IA potrebbe non essere corretto.}]{media/10_Ric3D/image300.pdf}\caption{Figura .: Esempio di ricostruzione mediante metodo iterativo}
\end{figure}

Anche in questo caso gli artefatti legati al troncamento e al campionamento sono sempre presenti, dunque, può essere aggiungere un ulteriore filtraggio dell'immagine ricostruita con la finestra di Hamming, al fine di ridurre gli artefatti da rinning per l'effetti Gibbs. Ciò può essere eseguito nell'ultimo passo dell'algoritmo, quando la relazione \(\left| {\widehat{\rho}}_{j + 1}(q\Delta x) - {\widehat{\rho}}_{j}(q\Delta x) \right| < \varepsilon\) è soddisfatta.

Si dimostra che nel \(k\)-spazio, il segnale, ottenuto a valle delle iterazioni e dopo la finestratura con Hamming, è dato da:

\[s_{j + 1}(p\Delta x) = \left\{ \begin{matrix}
s_{m}(p\Delta k) & - n_{-} \leq p \leq n_{+} - 1 \\
{\widetilde{s}}_{j + 1}(p\Delta k) & - n_{+} \leq p \leq - n_{-} - 1 \\
\dfrac{1}{2}\left\{ \begin{array}{r}
s_{m}(p\Delta k)\left\lbrack 1 + \cos\left( \pi + \dfrac{\pi\left( p + n_{-} \right)}{u} \right) \right\rbrack + \\
 + {\widetilde{s}}_{j + 1}(p\Delta k)\left\lbrack 1 + \cos\left( 1 + \dfrac{\pi\left( p + n_{-} \right)}{u} \right) \right\rbrack
\end{array} \right\} & - n_{-} \leq p \leq - n_{-} + o - 1
\end{matrix} \right.\ \]

\subsection{Immagini DICOM}\label{immagini-dicom}

Il DICOM (\emph{Digital Imaging and Communications in Medicine}) è lo standard internazionale utilizzato per memorizzare e trasmettere dati di bioimmagini, come quelli derivanti da risonanza magnetica (MRI), tomografia computerizzata (CT), tomografia a emissione di positroni (PET) o ultrasuoni. Generalmente, ogni immagine è salvata in un file separato con estensione .dcm, anche se esistono file privi di estensione.

Il DICOM è anche un'architettura di rete che permette la trasmissione delle informazioni radiologiche tra diverse apparecchiature. Questa componente è fondamentale perché consente di integrare i dati provenienti da sistemi diagnostici differenti.

Lo standard DICOM è nato da una collaborazione tra produttori di dispositivi per imaging medico per definire criteri comuni per comunicazione, visualizzazione, archiviazione e stampa delle immagini biomediche. Il DICOM non definisce un algoritmo di compressione: nella maggior parte dei casi, le immagini sono salvate non compresse, secondo la codifica originale della strumentazione.

Un file DICOM contiene due sezioni principali:

\begin{itemize}
\item
  Header (o intestazione) contenenti informazioni sul paziente, tipo di scansione, dimensioni e parametri dell'immagine;
\item
  Immagine ovvero i dati grezzi che compongono visivamente l'immagine diagnostica.
\end{itemize}

La dimensione dell'header varia in base alla quantità di informazioni presenti.

I dati sono organizzati in una struttura gerarchica composta da:

\begin{enumerate}
\def\labelenumi{\arabic{enumi}.}
\item
  Paziente, cioè il soggetto a cui è riferita l'acquisizione diagnostica;
\item
  Study (o esame), una sessione clinica in cui il paziente si sottopone a uno o più esami;
\item
  Series (of serie), un insieme di immagini ottenute con la stessa tecnica/modalità;
\item
  Immagini, cioè i singoli file che compongono l'acquisizione diagnostica.
\end{enumerate}

Ogni study può comprendere più series, ciascuna ottenuta con una modalità di acquisizione (modality), come MRI, CT o X-Ray. Il campo Modality, presente nell'header DICOM, identifica la modalità di acquisizione impiegata (es. CT, MRI, US, CR). Ogni modality corrisponde a una tecnica diagnostica diversa, e può produrre immagini singole o interi set tridimensionali, con caratteristiche specifiche di contrasto e risoluzione.

\begin{figure}
\centering
\includegraphics[width=4.10107in,height=5.3125in,alt={Immagine che contiene testo, schermata, Carattere, numero Il contenuto generato dall\textquotesingle IA potrebbe non essere corretto.}]{media/10_Ric3D/image301.pdf}\caption{Figura .: Struttura gerarchica del DICOM}
\end{figure}

Modalità che generano immagini 3D sono:

\begin{itemize}
\item
  MRI, che permette la ricostruzione di volumi cerebrali, articolari, ecc.
\item
  CT, per, ad esempio, scansioni toraciche, addominali;
\item
  PET per immagini di carsttere metabolico.
\end{itemize}

Per la ricostruzione volumetrica, il DICOM utilizza campi come:

\begin{itemize}
\item
  SliceLocation;
\item
  ImagePositionPatient;
\item
  PixelSpacing;
\item
  SliceThickness.
\end{itemize}

Questi valori indicano la posizione e l'orientamento nello spazio delle immagini, e permettono ai software di ricostruire e visualizzare correttamente il volume 3D. Tali immagini possono essere utilizzate per ricostruzioni multiplanari, rendering volumetrici, e pianificazione preoperatoria.

Nell'header Sono contenute delle meta-informazioni che non necessariamente sono legate all'immagine, tra cui:

\begin{itemize}
\item
  Nome file, identificatore del file immagine salvato;
\item
  Versione Dicom, specifica la versione dello standard seguita. Attualmente la V3. 3 è in vigore;
\item
  OP Class UID, identificatore unico globale dell'oggetto memorizzato;
\item
  Dati del paziente come nome, ID, sesso, data di nascita;
\item
  Spessore delle slice, indica lo spessore delle sezioni anatomiche acquisite (in \(mm\)).
\item
  Dimensione dei pixel, descrive l'altezza e larghezza dei singoli pixel;
\item
  Profondità in bit, numero di bit usati per rappresentare l'intensità di ciascun pixel. Nel DICOM, il numero di bit utilizzato per rappresentare un pixel (BitsStored) può differire dal numero di bit effettivamente utilizzati per rappresentare la scala dei toni (BitsAllocated). Questo consente di usare solo una parte del range disponibile, utile ad esempio per limitare il rumore o adattarsi a vincoli di visualizzazione;
\item
  Tempo di ripetizione (\(T_{R}\)), intervallo tra due impulsi di eccitazione (MRI);
\item
  Tempo di inversione (\(T_{I}\)), intervallo tra impulso di inversione e impulso di eccitazione (MRI);
\item
  Tempo di echo (\(T_{E}\)), tempo tra eccitazione e ricezione del segnale (MRI);
\item
  Tipo di pesatura, che può essere di tipo \(T_{1}\), \(T_{2}\) o densità protonica, in base ai parametri di sequenza;
\item
  Gap tra le slice, spazio vuoto (in \(mm\)) tra due sezioni consecutive;
\item
  Intensità del campo magnetico, indica la potenza del magnete in Tesla (es. \(1.5\ T\), \(3\ T\));
\item
  Numero di ripetizioni, quante volte è ripetuta la sequenza per aumentare SNR. Quando una stessa sequenza viene ripetuta più volte con parametri invariati, i segnali ottenuti vengono mediati. La media riduce il rumore casuale, migliorando il rapporto segnale/rumore (SNR), un fattore critico nella qualità diagnostica delle immagini;
\item
  Percentuale di \(k\)-spazio acquisito, frazione dei dati raccolti per la ricostruzione mediante algoritmi iterativi;
\item
  Banda di ricezione, intervallo di frequenze su cui si acquisisce il segnale. Una banda più ampia consente una migliore risoluzione temporale, ma può aumentare il rumore;
\item
  FOV (\emph{Field of View}), area visibile nell'immagine (espressa in \(mm\)) regolabile in base all'esame clinico;
\item
  Localizzazione della slice, posizione specifica della sezione acquisita lungo l'asse;
\item
  Pixel spacing, distanza in \(mm\) tra pixel adiacenti in ciascuna direzione;
\end{itemize}

In tecniche come la MRI, può capitare che alcuni pixel abbiano valori negativi. Questo è dovuto alla \textbf{codifica di fase} nel segnale ricevuto: il segnale può avere un'oscillazione positiva o negativa in funzione della direzione del gradiente applicato. La presenza di pixel negativi può influenzare la visualizzazione, e spesso viene compensata nei software di post-processing.

In passato, prima della diffusione dei sistemi di acquisizione digitali, le immagini radiologiche venivano prodotte su \textbf{film fotografici} (pellicole) che rappresentavano visivamente le strutture anatomiche grazie alla trasparenza variabile del supporto.

Con l'introduzione del formato DICOM e dei \textbf{sistemi PACS} (Picture Archiving and Communication System), è emersa l'esigenza di \textbf{digitalizzare anche le immagini analogiche} per integrarli in archivi elettronici e garantire continuità nelle cartelle cliniche digitali.

Questa operazione di digitalizzazione avviene attraverso dispositivi detti:

\begin{itemize}
\item
  \textbf{Scanner di film radiologici}: macchine apposite che convertono le pellicole in immagini digitali DICOM.
\item
  Il formato risultante mantiene una qualità sufficientemente alta per la \textbf{visualizzazione diagnostica}, ma la risoluzione dipende dalla tecnologia dello scanner.
\end{itemize}

Nel DICOM, le immagini ottenute da film scannerizzati sono etichettate spesso con:

\begin{itemize}
\item
  \textbf{Imaging type: "SECONDARY"} -- indica che l'immagine non è originaria (non prodotta direttamente dal dispositivo di acquisizione), ma ottenuta per conversione.
\item
  \textbf{Origine dello studio}: è documentato nell'header attraverso tag specifici che tracciano la provenienza analogica dell'immagine.
\end{itemize}

Questa distinzione tra immagini \textbf{primary} (direttamente digitali) e \textbf{secondary} (digitalizzate) è fondamentale, poiché può influenzare:

\begin{itemize}
\item
  L'affidabilità della diagnosi
\item
  La compatibilità con alcuni algoritmi di elaborazione
\item
  L'uso legale delle immagini in ambito forense
\end{itemize}

Tra le meta-informazioni vi sono anche degli attributi riservati al produttore, che possono contenere parametri personalizzati o proprietari, leggibili solo con software dedicati del produttore dell'apparecchiatura.

All'interno delle meta-informazioni sono presenti daate importanti come:

\begin{itemize}
\item
  Data accesso paziente, quando il paziente entra nel sistema ospedaliero;
\item
  Data acquisizione immagine, quando l'immagine viene effettivamente registrata;
\item
  Data disponibilità immagine, quando il file immagine è pronto per la visualizzazione dopo l'elaborazione.
\end{itemize}

Lo standard DICOM è gestito anche da MatLab mediante i comandi:

\begin{itemize}
\item
  dicominfo() che estrae tutte le meta-informazioni dell'immagine;
\item
  dicomread(), permette di leggere e visualizzare l'immagine contenuta nel file DICOM.
\end{itemize}

In conclusione, DICOM è il formato standard per l'imaging medico. Tutti i software e le apparecchiature moderne devono supportarlo. Esso consente una gestione integrata e strutturata dei dati clinici, tecnici e diagnostici, facilitando la condivisione in ambito sanitario.

\subsection{Istogramma per la visualizzazione dell'immagine}\label{istogramma-per-la-visualizzazione-dellimmagine}

Per descrivere il meccanismo di visualizzazione delle immagini si considera, a titolo Per descrivere il meccanismo di visualizzazione delle immagini, si consideri, a titolo di esempio, una risonanza magnetica alle mammelle finalizzata a evidenziare la presenza di eventuali lesioni maligne.

In questo tipo di esame, la paziente viene posizionata in \textbf{decubito prono} (cioè a pancia in giù), con la schiena rivolta verso l'alto e le mammelle inserite all'interno di \textbf{antenne riceventi} dedicate, solitamente costituite da due avvolgimenti.

Mediante algoritmi digitali è possibile selezionare e visualizzare un determinato intervallo di livelli di grigio presenti nell'immagine, in modo da ottimizzare la visibilità della lesione.

Si può, ad esempio, costruire l'\textbf{istogramma} dell'immagine contando il numero di pixel corrispondenti a ciascuna gradazione di grigio. Le tecniche di manipolazione dell'immagine agiscono spesso modificando la distribuzione dei livelli di grigio rappresentati nell'istogramma.

\begin{figure}
\centering
\includegraphics[width=6.69306in,height=3.125in,alt={L\textquotesingle istogramma in fotografia: cos\textquotesingle è e come usarlo al meglio}]{media/10_Ric3D/image302.pdf}\caption{Figura .: Istogramma di un'immagine radiologica}
\end{figure}

Nell'immagine radiologica possono mancare alcune gradazioni di grigio e, inoltre, l'occhio umano non è in grado di distinguere tutte le gradazioni presenti in un'immagine digitale.

Per migliorare la percezione visiva, è possibile \textbf{selezionare una finestra} di livelli di grigio (windowing). La scelta di una porzione dell'istogramma e la sua espansione all'intero intervallo di visualizzazione produce un'immagine con \textbf{migliore contrasto apparente}: non perché il contrasto intrinseco tra i tessuti sia aumentato, ma perché, concentrando l'intervallo su piccole variazioni di densità protonica, il sistema assegna a tali differenze un maggior numero di gradazioni visibili. Il risultato è un'evidenziazione più netta delle differenze di intensità tra i tessuti.

Se la finestra selezionata è molto stretta, gran parte dell'immagine apparirà tendente al \textbf{bianco}; se invece è molto ampia, prevarranno toni più scuri fino al \textbf{nero}.

\begin{figure}
\centering
\includegraphics[width=6.68333in,height=5.55833in]{media/10_Ric3D/image303.pdf}\caption{Figura .: Elaborazione basata sull'istogramma}
\end{figure}

\begin{center}
\vfill
    \chapter{Rapporto segnale/rumore in MRI}
    \label{blx:SNR\therefsection}
\vfill

\minitoc
\newpage
\end{center}
\justify


\section{SNR in RMI}\label{snr-in-rmi}

Tutte le misure fisicamente realizzate sono caratterizzate da una certa quota di rumore, ovvero un segnale di disturbo con andamento casuale. Il rumore può rendere difficoltoso o, alle volte, impossibile la lettura e l'interpretazione dei dati misurati.

Il parametro con cui si valuta la corruzione della misura a causa del rumore è dato dal rapporto segnale/rumore o SNR (Signal-Noise Ratio), che per definizione è dato da:

\[SNR = \dfrac{s}{\sigma}\]

Dove \(s\) è il valore del segnale mentre \(\sigma\) è la deviazione standard del rumore nel voxel.

In risonanza magnetica il rapporto segnale/rumore è un parametro fondamentale per valutare l'efficacia di una sequenza o un esperimento, al fine di ottenere un'immagine. Ovviamente, se il rapporto segnale/rumore non è sufficientemente alto, diventa complesso distinguere i vari tessuti o un tessuto dallo sfondo a causa del rumore sovrapposto.

\subsection{Valutazione del segnale del voxel in RMI}\label{valutazione-del-segnale-del-voxel-in-rmi}

Il segnale ricevuto dalle antenne, dovuto al ritorno all'equilibrio del vettore magnetizzazione nel voxel, è legato alla densità protonica efficace mediante trasformata di Fourier:

\[s\left( \overset{\underline{}}{k} \right) = \int_{V}^{}{\rho\left( \overset{\underline{}}{r} \right)\exp\left( - j2\pi\overset{\underline{}}{k} \cdot \overset{\underline{}}{r} \right)d^{(3)}\overset{\underline{}}{r}}\]

Il segnale registrato non è continuo nel tempo ma campionato lungo i tre assi ortogonali del \(k\)-spazio con passi di campionamento, rispettivamente, \(\Delta k_{x}\), \(\Delta k_{y}\) e \(\Delta k_{z}\). L'immagine ricostruita nel dominio dello spazio è, quindi, legata al segnale nel \(k\)-spaio da una trasformata inversa di Fourier discreta:

\[{\widehat{\rho}}_{m}(p\Delta x,q\Delta y,r\Delta z) = \dfrac{1}{N_{x}N_{y}N_{z}}\sum_{p' = - N_{x}}^{N_{x}}{\sum_{q' = - N_{y}}^{N_{y}}{\sum_{r' = - N_{z}}^{N_{z}}{s\left( p'\Delta k_{x},q'\Delta k_{y},r'\Delta k_{z} \right)\exp\left( j2\pi\left( \dfrac{pp'}{N_{x}} + \dfrac{qq'}{N_{y}} + \dfrac{rr'}{N_{z}} \right) \right)}}}\]

È possibile compattare la notazione unendo le tre sommatorie:

\[{\widehat{\rho}}_{m}(p\Delta x,q\Delta y,r\Delta z) = \dfrac{1}{N_{x}N_{y}N_{z}}\sum_{p',q',r'}^{}{s\left( p'\Delta k_{x},q'\Delta k_{y},r'\Delta k_{z} \right)\exp\left( j2\pi\left( \dfrac{pp'}{N_{x}} + \dfrac{qq'}{N_{y}} + \dfrac{rr'}{N_{z}} \right) \right)}\]

La densità protonica effettiva misurata \({\widehat{\rho}}_{m}\) è anche detta voxel signal poiché è il segnale rappresentato nel volume elementare \(\Delta x\Delta y\Delta z\) posizionato nel punto \((p\Delta x,q\Delta y,r\Delta z)\) dell'immagine ricostruita.

Il segnale nel voxel spesso non dipende solamente dalla magnetizzazione trasversale all'interno del voxel ma contiene informazioni, anche blande, sui tempi di rilassamento.

Dalla risoluzione spaziale è noto che la dimensione del voxel è rappresentata dall'area sottesa alla PSF normalizzata rispetto al suo valore nell'origine. In particolare, se si riducono le dimensioni del voxel, \(\Delta x\Delta y\Delta z\), la risoluzione spaziale migliora, tuttavia, nello stesso tempo, il segnale del voxel si riduce proporzionalmente, per la diretta proporzionalità tra densità protonica e volume:

\[{\widehat{\rho}}_{m}(p\Delta x,q\Delta y,r\Delta z) \propto \Delta x\Delta y\Delta z\]

Dato che il voxel contiene un certo numero di protoni, la densità protonica \(\rho\) è legata anche alla magnetizzazione all'equilibrio.

Scegliendo un voxel piccolo a sufficienza, tale fa poter considerare il suo contenuto omogeneo e, in ipotesi di comportamento ideale delle antenne riceventi, il segnale può essere scritto come:

\[\widehat{\rho}(p\Delta x,q\Delta y,r\Delta z) \propto \dfrac{{\overline{\gamma}}^{3}\hslash^{2}}{4k_{B}T}B_{0}^{2}B_{\bot}(p\Delta x,q\Delta y,r\Delta z)\Delta x\Delta y\Delta z\]

Dove \(B_{0}\) è il campo magnetico principale applicato. Da questa relazione si evince che il voxel signal dipende dalla componente trasversa del campo magnetico.

\subsection{Valutazione del rumore del voxel in RMI}\label{valutazione-del-rumore-del-voxel-in-rmi}

Noto il segnale è necessario valutare il rumore sovrapposto, al fine di ottenere un'espressione per il rapporto segnale/rumore. Generalmente, il rumore deriva dalle fluttuazioni casuali degli elettroni contenuti nel metallo dell'antenna e dal campionamento del segnale. In linea teorica sono presenti altre fonti di rumore come quello di quantizzazione, legato alla digitalizzazione del segnale, e i ghosting, legati al movimento degli spin, tuttavia, i loro effetti sono minimi almeno in un esperimento ideale.

La varianza delle fluttuazioni elettroniche o di Nyquist-Johnson è data dalla relazione:

\[\sigma^{2} = 4k_{B}TRBW\]

Dove \(R\) è la resistenza offerta dall'antenna in ricezione, posizionata sul corpo del paziente e \(BW\) la lunghezza della banda del rumore sovrapposto allo spettro del segnale utile. Questo rumore è legato, essenzialmente, alla temperatura \(T\) diversa da quella assoluta dei portatori di carica. Gli elettroni si muovono nel materiale conduttore in modo casuale per agitazione termica e ciò determina la presenza del rumore.

Il rumore di Nyquist-Johnson, avendo un'ampiezza costante nella banda del segnale utile, può essere considerato come un rumore gaussiano bianco a media nulla e varianza \(\sigma^{2}\). La gaussianità del rumore deriva dal teorema del limite centrale, il quale afferma che la somma di aventi indipendenti tende a una gaussiana.

Nel \(k\)-spazio il segnale misurato può essere modellato come la somma del segnale ideale e del rumore \(\varepsilon\) incorrelato col segnale utile:

\[s_{m}(k) = s(k) + \varepsilon(k)\]

SI suppone, inoltre, che i campioni del rumore siano incorrelati tra loro. In questa ipotesi, la media statistica tra due campioni del rumore è data da:

\[E\left\lbrack \varepsilon\left( k_{p} \right)\varepsilon^{*}\left( k_{q} \right) \right\rbrack = \overline{\varepsilon\left( k_{p} \right)\varepsilon^{*}\left( k_{q} \right)} = \sigma_{m}^{2}\delta\left( k_{p} - k_{q} \right) = \sigma_{m}^{2}\delta_{pq}\]

La media statistica tra due campioni diversi è nulla mentre se i campioni coincidono si ottiene la varianza del rumore di misura in quel campione.

Per l'ipotesi di guassianità e media nulla, il rumore è distribuito come una campana di Gauss con media nulla e varianza \(\sigma_{m}^{2}\):

\[\varepsilon\sim N\left( 0,\sigma_{m} \right)\]

La trasformata inversa di Fourier del segnale misurato nel \(k\)-spazio, \(s_{m}(k)\), restituisce la densità protonica, ovvero il segnale del voxel considerato:

\[{\widehat{\rho}}_{m}(x) = \rho(x) + \eta(x)\]

Per effetto del campionamento e troncamento, la relazione si scrive considerando i campioni \(p\Delta x\):

\[{\widehat{\rho}}_{m}(p\Delta x) = \rho(p\Delta x) + \eta(p\Delta x)\]

Il voxel signal è la trasformata inversa di Fourier del segnale campionato nel \(k\)-spazio:

\[\rho(p\Delta x) = \dfrac{1}{N}\sum_{q = - \dfrac{N}{2}}^{\dfrac{N}{2} - 1}{s(q\Delta k)\exp(j2\pi pq\Delta k\Delta x)}\]

Mentre \(\eta\) è la trasformata inversa del rumore additivo \(\varepsilon\) nel \(k\)-spazio:

\[\eta(p\Delta x) = \dfrac{1}{N}\sum_{q = - \dfrac{N}{2}}^{\dfrac{N}{2} - 1}{\varepsilon(q\Delta k)\exp(j2\pi pq\Delta k\Delta x)}\]

È possibile caratterizzare il rumore dal punto di vista statistico nel dominio dello spazio-immagine. Si inizia determinando la media statistica del rumore \(\eta\) nello spazio-immagine. Per linearità è possibile scrivere:

\[E\left\lbrack \eta(p\Delta x) \right\rbrack = \dfrac{1}{N}\sum_{q = - \dfrac{N}{2}}^{\dfrac{N}{2} - 1}{E\left\lbrack \varepsilon(q\Delta k) \right\rbrack\exp(j2\pi pq\Delta k\Delta x)}\]

Per l'ipotesi iniziale di rumore \(\varepsilon\) a media nulla, anche la media della sua antitrasformata \(\eta\) è a media nulla:

\[E\left\lbrack \varepsilon(q\Delta k) \right\rbrack = 0 \Rightarrow \ E\left\lbrack \eta(p\Delta x) \right\rbrack = 0\]

Di maggior interesse è la varianza del rumore \(\eta\) nel dominio dello spazio-immagine, per definizione data da:

\[VAR\lbrack\eta\rbrack = E\left\lbrack \eta(r\Delta x)\eta^{*}(s\Delta x) \right\rbrack\]

Applicando la relazione di inversa trasformata tra il \(k\)-spazio e lo spazio immagine si scrive:

\[\eta(r\Delta x)\eta^{*}(s\Delta x) = \left( \dfrac{1}{N}\sum_{r = - \dfrac{N}{2}}^{\dfrac{N}{2} - 1}{\varepsilon(r\Delta k)\exp(j2\pi rq\Delta k\Delta x)} \right)\left( \dfrac{1}{N}\sum_{s = - \dfrac{N}{2}}^{\dfrac{N}{2} - 1}{\varepsilon(s\Delta k)\exp(j2\pi sp\Delta k\Delta x)} \right)^{*}\]

Che può essere scritta come:

\[\eta(r\Delta x)\eta^{*}(s\Delta x) = \dfrac{1}{N^{2}}\sum_{r = - \dfrac{N}{2}}^{\dfrac{N}{2} - 1}{\sum_{s = - \dfrac{N}{2}}^{\dfrac{N}{2} - 1}{\varepsilon(r\Delta k)\varepsilon^{*}(r\Delta k)\exp\left( j2\pi p(r - s)\Delta k\Delta x \right)}}\]

Applicando l'operatore media statistica e la proprietà di linearità si ottiene:

\[E\left\lbrack \eta(r\Delta x)\eta^{*}(s\Delta x) \right\rbrack = \dfrac{1}{N^{2}}\sum_{r = - \dfrac{N}{2}}^{\dfrac{N}{2} - 1}{\sum_{s = - \dfrac{N}{2}}^{\dfrac{N}{2} - 1}{E\left\lbrack \varepsilon(p\Delta k)\varepsilon^{*}(p\Delta k) \right\rbrack\exp\left( j2\pi p(r - s)\Delta k\Delta x \right)}}\]

Per la proprietà di media nulla, \(E\left\lbrack \varepsilon(r\Delta k)\varepsilon^{*}(s\Delta k) \right\rbrack = \sigma_{m}^{2}\delta_{rs}\). Ne discende che la sommatoria è non nulla solo se \(r = s\), per cui:

\[E\left\lbrack \eta(r\Delta x)\eta^{*}(s\Delta x) \right\rbrack = \dfrac{1}{N^{2}}\sigma_{m}^{2}\sum_{r = s}^{}\delta_{rs} = \dfrac{1}{N^{2}}\sigma_{m}^{2}N = \dfrac{\sigma_{m}^{2}}{N}\]

Si pone \(\sigma_{o}^{2}\) come:

\[\sigma_{o}^{2} = E\left\lbrack \eta(r\Delta x)\eta^{*}(s\Delta x) \right\rbrack = \dfrac{\sigma_{m}^{2}}{N}\]

Si osservi che \(\sigma_{m}^{2}\) è la varianza misurata in ogni punto del \(k\)-spazio, mentre \(\sigma_{0}^{2}\) è la varianza del rumore nello spazio-immagine. Dalla relazione ricavata, la varianza misurata in ogni voxel è ridotta di un fattore \(N\) rispetto al valore che assume nello spazio \(k\) e, inoltre, il rumore presenta le stesse proprietà statistiche per ogni voxel.

In conclusione, ogni voxel è corrotto da un rumore a media nulla e varianza \(\sigma_{m}^{2}/N\). Il rumore è, quindi, equamente distribuito in tutto lo spazio-immagine, ovvero su ogni voxel dell'immagine.

È possibile estendere i ragionamenti fatto nello spazio monodimensionale a uno spazio a più dimensioni. In particolare, nel caso bidimensionale la varianza del rumore nello spazio-immagine su ogni voxel è:

\[VAR\left\lbrack \eta(p,q) \right\rbrack \propto \dfrac{\sigma_{m}^{2}}{N_{x}N_{y}}\]

Dove nella relazione non è esplicitata la dipendenza dalla temperatura o dalla resistenza \(R\) offerta dall'antenna poiché sono parametri fissi o non possono essere modificati con le sequenze di acquisizione.

Nel caso tridimensionale si ha:

\[VAR\left\lbrack \eta(p,q,r) \right\rbrack \propto \dfrac{\sigma_{m}^{2}}{N_{x}N_{y}N_{z}}\]

Con \(\sigma^{2} = 4k_{B}TRBW\).

Applicando una sequenza di acquisizione è possibile controllare diversi parametri, tra cui il numero di volte con cui si acquisisce una data slice del corpo umano. In particolare, ripetendo l'esperimento un numero \(N_{acq}\) di volte ed eseguendo una media dei segnali ricevuti, è possibile aumentare il rapporto segnale/rumore. Il segnale misurato all'\(i\)-esima acquisizione può essere scritto come:

\[s_{m.i}(k) = s(k) + \varepsilon_{i}(k)\]

Applicando l'operazione di media su \(N_{acq}\) ripetizioni dell'esperimento, si ha:

\[s_{m,av}(k) = \dfrac{1}{N_{acq}}\sum_{i = 1}^{N_{acq}}\left( s(k) + \varepsilon_{i}(k) \right)\]

Si suppone che, in ogni misura, il segnale utile \(s(k)\) non vari, ovvero eccitando il materiale allo stesso modo, il segnale prelevato presenta variazioni solamente legati al rumore.

Per la linearità dell'operatore di somma, si scrive:

\[s_{m,av}(k) = \dfrac{1}{N_{acq}}\sum_{i = 1}^{N_{acq}}{s(k)} + \dfrac{1}{N_{acq}}\sum_{i = 1}^{N_{acq}}{\varepsilon_{i}(k)}\]

Dato che il segnale utile non aria con la misura, si ha:

\[s_{m,av}(k) = \dfrac{s(k)}{N_{acq}}\sum_{i = 1}^{N_{acq}}1 + \dfrac{1}{N_{acq}}\sum_{i = 1}^{N_{acq}}{\varepsilon_{i}(k)} = \dfrac{s(k)}{N_{acq}}N_{acq} + \dfrac{1}{N_{acq}}\sum_{i = 1}^{N_{acq}}{\varepsilon_{i}(k)}\]

Dunque:

\[s_{m,av}(k) = s(k) + \dfrac{1}{N_{acq}}\sum_{i = 1}^{N_{acq}}{\varepsilon_{i}(k)}\]

In ipotesi di rumore a media nulla, se il numero delle ripetizioni è sufficientemente alto, è possibile ritenere la sommatoria al secondo membro tendente a \(0\):

\[\dfrac{1}{N_{acq}}\sum_{i = 1}^{N_{acq}}{\varepsilon_{i}(k)} \simeq 0\]

Da cui si ottiene:

\[s_{m,av}(k) \simeq s(k)\]

Operando la media su un numero di ripetizioni della sequenza \(N_{acq}\) opportunamente scelto, è possibile aumentare il rapporto segnale/rumore. Nel dettaglio, maggiore è il numero di acquisizioni, a parità di \(k\)-spazio acquisito, e migliore è la riduzione del rumore sovrapposto alla misura.

La varianza del rumore mediato è data da:

\[\sigma_{m,av}^{2} = VAR\left\lbrack s_{m,av}(k) \right\rbrack = \dfrac{1}{N_{acq}^{2}}\sum_{i = 1}^{N_{acq}}{VAR\left\lbrack s_{m.i}(k) \right\rbrack} = \dfrac{\sigma_{m}^{2}}{N_{acq}}\]

La deviazione standard, di conseguenza, è data da:

\[\sigma_{m,av} = \dfrac{\sigma_{m}}{\sqrt{N_{acq}}}\]

Con l'operazione di media, il rapporto segnale/rumore di un qualunque voxel può essere espresso, \(k\)-spazio, come il rapporto tra la media del segnale misurato e la deviazione standard:

\[SNR = \dfrac{\overline{s_{m,av}(k)}}{\sigma_{m,av}} = \sqrt{N_{acq}}\dfrac{s(k)}{\sigma_{m}}\]

Nel dominio dello spazio-immagine il segnale del pixel dipende dal volume del voxel e da altri parametri sui quali non è possibile agire come la temperatura, il campo megnatico esterno applicato, l'antenna ricevente e così via:

\[{\widehat{\rho}}_{m} \propto \Delta x\Delta y\Delta z\]

La varianza del rumore dipende dal numero di acquisizioni \(N_{acq}\), dalla banda del segnale acquisito variabile in base alla sequenza utilizzata e al numero di campioni col quale si ricostruisce l'immagine:

\[\sigma_{0}^{2} \propto \dfrac{BW}{N_{x}N_{y}N_{z}}\dfrac{1}{N_{acq}}\]

\subsection{Valutazione del rapporto segnale/rumore}\label{valutazione-del-rapporto-segnalerumore}

Il rapporto segnale/rumore in un voxel \(\Delta x\Delta y\Delta z\), nel caso più generale possibile, può essere espresso come:

\[\left. \ SNR \right|_{\Delta x\Delta y\Delta z} = \dfrac{s}{\sigma_{0}} \propto \dfrac{\Delta x\Delta y\Delta z}{\sqrt{\dfrac{BW}{N_{x}N_{y}N_{z}}}}\sqrt{N_{acq}}\]

Il rapporto segnale/rumore di un voxel dipende dalle sue dimensioni \(\Delta x\Delta y\Delta z\), dal numero di campioni acquisiti per ricostruire il voxel \(N_{x}N_{y}N_{z}\), dal numero di ripetizioni della sequenza \(N_{acq}\) e dalla banda del segnale considerato \(BW\). È noto che la banda di lettura, ovvero la banda del segnale acquisito durante il gradiente di lettura, è legata all'intervallo di campionamento temporale \(\Delta t\) dalla relazione:

\[{BW}_{R} = \dfrac{1}{\Delta t}\]

Con questa considerazione, il rapporto segnale/rumore può essere espresso come:

\[\left. \ SNR \right|_{\Delta x\Delta y\Delta z} = \dfrac{s}{\sigma_{0}} \propto \dfrac{\Delta x\Delta y\Delta z}{\sqrt{\dfrac{BW}{N_{x}N_{y}N_{z}}}}\sqrt{N_{acq}} = \Delta x\Delta y\Delta z\sqrt{N_{acq}}\sqrt{N_{x}N_{y}N_{z}\Delta t}\]

L'intervallo di acquisizione \(T_{S}\) è dato dal periodo di campionamento sull'asse di lettura, \(\Delta t\), e il numero di campioni acquisiti. Generalmente risulta che l'asse di lettura coincide con l'asse \(x\), per cui:

\[T_{S} = N_{x}\Delta t\]

Con questa definizione, il rapporto segnale/rumore di un generico voxel può essere espresso come:

\[\left. \ SNR \right|_{\Delta x\Delta y\Delta z} \propto \Delta x\Delta y\Delta z\sqrt{N_{acq}}\sqrt{N_{y}N_{z}T_{S}}\]

Da questa relazione il rapporto segnale/rumore può essere scritto in funzione dei parametri con i quali si vuole caratterizzare l'immagine o la sequenza di acquisizione.

Va considerato che i parametri del rapporto segnale/rumore non sono scollegati tra loro tramite il FOV:

\[L_{x} = N_{x}\Delta x,L_{y} = N_{y}\Delta y,\ L_{z} = N_{z}\Delta z\]

Inoltre, il FOV lungo la direzione di lettura è legato alla banda di lettura mediante il rapporto giromagnetico e il gradiente di lettura:

\[{BW}_{R} = \overline{\gamma}G_{x}L_{x}\]

Si definisce banda per voxel, nell'ipotesi che l'asse \(x\) coincida con quello di lettura, come:

\[BW/voxel = \dfrac{{BW}_{R}}{N_{x}} = \dfrac{\overline{\gamma}G_{x}L_{x}}{N_{x}}\]

Il rapporto segnale/rumore (\textbf{SNR}) è una misura fondamentale per valutare la qualità di un segnale in relazione al rumore presente. Le espressioni matematiche che descrivono l'SNR contengono una rete di \textbf{interdipendenze} tra vari parametri.

Quando si modifica uno di questi parametri, non si può considerare tale cambiamento in modo isolato: esso provoca inevitabilmente variazioni anche negli altri parametri dell'equazione. Questo implica la necessità di \textbf{valutare attentamente le conseguenze complessive} di ogni variazione.

Inoltre, le relazioni tra i parametri permettono di derivare formule alternative per l'SNR, ciascuna mirata a mettere in evidenza gli effetti prodotti dalla variazione di uno specifico gruppo di parametri. Queste formulazioni, tuttavia, sono spesso soggette a \textbf{condizioni di costanza}: determinate grandezze devono rimanere invariate, e questo vincola le possibilità di modifica degli altri fattori.

Oltre ai parametri puramente matematici, il valore dell'SNR dipende anche da \textbf{fattori pratici} come le modalità di acquisizione dei dati e le caratteristiche fisiche o biologiche dei tessuti analizzati. Questi aspetti aggiuntivi richiedono un'analisi separata, approfondita in altre sezioni del documento originale.

In sintesi, lo studio dell'SNR non può essere affrontato modificando un singolo parametro senza considerare l'effetto complessivo sul sistema. Serve un approccio sistematico, in cui i vincoli, le interrelazioni e le condizioni reali di acquisizione siano sempre presi in considerazione.

\subsection{Dipendenza del SNR dall'asse di lettura}\label{dipendenza-del-snr-dallasse-di-lettura}

Si restringe l'analisi del rapporto segnale/rumore al solo asse di lettura, supposto essere uguale all'asse \(x\). Le altre quantità lungo \(y\) e \(z\) possono essere trascurati nei ragionamenti successivi, poiché non influenzano i segnali lungo l'asse considerato. In questo caso, il rapporto segnale/rumore è dato da:

\[\left. \ SNR \right|_{\Delta x} \propto \Delta x\sqrt{T_{S}}\]

\(\Delta x\) è il Fourier pixel size, connesso alla risoluzione spaziale, mentre \(T_{S}\) è l'ampiezza della finestra di acquisizione.

Si vuole capire come si comporta il rapporto segnale/rumore al variare dei parametri nella direzione di lettura:

\begin{itemize}
\item
  Ampiezza del gradiente \(G_{x}\);
\item
  Fourier pixel size, \(\Delta x\);
\item
  FOV, \(L_{x}\);
\item
  Numero di campioni acquisiti, \(N_{x}\);
\item
  Ampiezza temporale della finestra di acquisizione, \(T_{S}\).
\end{itemize}

Si suppone che tutti i parametri siano fissati in modo che il rapporto segnale/rumore sia unitario:

\[G_{x},\Delta x,L_{x},N_{x},T_{S}:SNR = 1\]

Si raddoppia il FOV, \(L_{x}' = 2L_{x}\), acquisendo il doppio dei punti all'interno della finestra di acquisizione, \(N_{x}' = 2N_{x}\). Dalla relazione che lega il FOV con il numero di campioni acquisiti e il la risoluzione, risulta che il Fourier pixel size resta costante:

\[L_{x}' = N_{x}'\Delta x' \Longleftrightarrow 2L_{x} = 2N_{x}\Delta x' \Leftrightarrow \Delta x' = \dfrac{L_{x}}{N_{x}} = \Delta x = cost\]

Questa soluzione non modifica, quindi, la risoluzione spaziale.

Dalla relazione che lega l'ampiezza della finestra di acquisizione con il campionamento temporale, \(\Delta t\):

\[T_{S}' = N_{x}'\Delta t'\]

Mantenendo anche l'ampiezza della finestra di acquisizione, \(T_{S}' = T_{S}\), con il nuovo numero di campioni, il passo di campionamento viene dimezzato:

\[T_{S} = 2N_{x}\Delta t' \Leftrightarrow \Delta t = \dfrac{1}{2}\dfrac{T_{S}}{N_{x}} = \dfrac{\Delta t}{2}\]

Da questo risultato discende che la banda di ricezione raddoppia, infatti:

\[{BW}_{R}' = \dfrac{1}{\Delta t'} = \dfrac{2}{\Delta t} = 2BW\]

La banda di ricezione è legata al gradiente di lettura applicato \(G_{x}\) tramite il rapporto giromagnetico e il FOV. Nella nuova sequenza con \(N_{x}\) e \(L_{x}\) raddoppiati, il gradiente resta invariato:

\[{BW}_{R}' = \overline{\gamma}G_{x}'L_{x}' \Leftrightarrow 2BW = 2\overline{\gamma}G_{x}'L_{x} \Leftrightarrow G_{x}' = \dfrac{BW}{\overline{\gamma}L_{x}} = G_{x}\]

Nella condizione con \(N_{x}\) e \(L_{x}\) raddoppiati \(\Delta x\) resta invariato così come \(T_{S}\), per cui il rapporto segnale/rumore non varia. In definitiva, mantenendo costante il rapporto segnale/rumore è possibile avere un raddoppio del FOV e, quindi, dimensioni osservabili nell'immagine maggiori, mantenendo invariata anche la risoluzione spaziale, a patto di dimezzare l'intervallo di campionamento temporale.

Si vuole, ora, raddoppiare la risoluzione spaziale, \(\Delta x' = \Delta x\), ovvero rendere il voxel più grande accettando un degrado della risoluzione. Si sceglie di lasciare inalterato il FOV, \(L_{x}' = L_{x}\). Con questa scelta, il numero dei punti è dato da:

\[L_{x}' = N_{x}'\Delta x' \Leftrightarrow L_{x} = N_{x}'2\Delta x \Leftrightarrow N_{x}' = \dfrac{1}{2}\dfrac{L_{x}}{\Delta x} = \dfrac{N_{x}}{2}\]

Per mantenere costante il FOV, è necessario dimezzare il numero dei campioni acquisiti durante la finestra di acquisizione.

Lasciando inalterato l'intervallo di campionamento \(T_{S}' = T_{S}\) per non prolungare i tempi di acquisizione, l'intervallo di campionamento \(\Delta t\) deve raddoppiare:

\[T_{S}' = N_{x}'\Delta t' \Leftrightarrow T_{S} = \dfrac{N_{x}}{2}\Delta t' \Leftrightarrow \Delta t' = 2\dfrac{T_{S}}{N_{x}} = 2\Delta t\]

Nella nuova sequenza, il gradiente di lettura deve anch'esso dimezzarsi, infatti:

\[\dfrac{1}{\Delta t'} = \overline{\gamma}G_{x}'L_{x}' \Leftrightarrow \dfrac{1}{2\Delta t} = \overline{\gamma}G_{x}'L_{x} \Leftrightarrow G_{x}' = \dfrac{\overline{\gamma}L_{x}}{2\Delta t} = \dfrac{1}{2}G_{x}\]

Con queste scelte progettuali, il rapporto segnale rumore raddoppia:

\[SNR \propto \Delta x'\sqrt{T_{S}'} = 2\Delta x\sqrt{T_{S}}\]

Questo incremento del rapporto segnale/rumore determina una risoluzione spaziale ridotta, a parità di FOV. La risoluzione spaziale è degradata a causa della maggiore dimensione del voxel, tuttavia, il segnale nel voxel ha una potenza maggiore rispetto al rumore.

Si suppone, infine, di dimezzare la risoluzione spaziale, ovvero si dimezza la dimensione del voxel lungo l'asse di lettura, \(\Delta x' = \Delta x/2\). Lasciando invariati il FOV \(L_{x}' = L_{x}\), e il tempo di acquisizione \(T_{S}' = T_{S}\), risulta che il numero di campioni da acquisire per ricostruire l'immagine deve essere il doppio:

\[L_{x}' = N_{x}'\Delta x' \Leftrightarrow L_{x} = N_{x}'\dfrac{\Delta x}{2} \Leftrightarrow N_{x}' = 2\dfrac{L_{x}}{\Delta x} = 2N_{x}\]

Mentre l'intervallo di campionamento temporale deve dimezzarsi:

\[T_{S}' = N_{x}'\Delta t' \Leftrightarrow T_{S} = 2N_{x}\Delta t' \Leftrightarrow \Delta t' = \dfrac{1}{2}\dfrac{T_{S}}{N_{x}} = \dfrac{\Delta t}{2}\]

Da questo risultato si evince anche che l'ampiezza del gradiente deve raddoppiarsi:

\[\dfrac{1}{\Delta t'} = \overline{\gamma}G_{x}'L_{x}' \Leftrightarrow \dfrac{2}{\Delta t} = \overline{\gamma}G_{x}'L_{x} \Leftrightarrow G_{x}' = 2\dfrac{\overline{\gamma}L_{x}}{\Delta t} = 2G_{x}\]

Con questa scelta di parametri il rapporto segnale/rumore si dimezza:

\[{SNR}' \propto \Delta x'\sqrt{T_{S}'} = \dfrac{\Delta x}{2}\sqrt{T_{S}}\]

\begin{longtable}[]{@{}
  >{\centering\arraybackslash}p{(\linewidth - 14\tabcolsep) * \real{0.4081}}
  >{\centering\arraybackslash}p{(\linewidth - 14\tabcolsep) * \real{0.0788}}
  >{\centering\arraybackslash}p{(\linewidth - 14\tabcolsep) * \real{0.0978}}
  >{\centering\arraybackslash}p{(\linewidth - 14\tabcolsep) * \real{0.0862}}
  >{\centering\arraybackslash}p{(\linewidth - 14\tabcolsep) * \real{0.0968}}
  >{\centering\arraybackslash}p{(\linewidth - 14\tabcolsep) * \real{0.0748}}
  >{\centering\arraybackslash}p{(\linewidth - 14\tabcolsep) * \real{0.0736}}
  >{\centering\arraybackslash}p{(\linewidth - 14\tabcolsep) * \real{0.0838}}@{}}
\caption{Tabella 11.1: Schema riassuntivo dei diversi esperimenti con parametri variati}\tabularnewline
\toprule\noalign{}
\begin{minipage}[b]{\linewidth}\centering
Caso
\end{minipage} & \begin{minipage}[b]{\linewidth}\centering
\[\Delta x'\]
\end{minipage} & \begin{minipage}[b]{\linewidth}\centering
\[N_{x}'\]
\end{minipage} & \begin{minipage}[b]{\linewidth}\centering
\[L_{x}'\]
\end{minipage} & \begin{minipage}[b]{\linewidth}\centering
\[G_{x}'\]
\end{minipage} & \begin{minipage}[b]{\linewidth}\centering
\[\Delta t'\]
\end{minipage} & \begin{minipage}[b]{\linewidth}\centering
\[T_{S}'\]
\end{minipage} & \begin{minipage}[b]{\linewidth}\centering
SNR
\end{minipage} \\
\midrule\noalign{}
\endfirsthead
\toprule\noalign{}
\begin{minipage}[b]{\linewidth}\centering
Caso
\end{minipage} & \begin{minipage}[b]{\linewidth}\centering
\[\Delta x'\]
\end{minipage} & \begin{minipage}[b]{\linewidth}\centering
\[N_{x}'\]
\end{minipage} & \begin{minipage}[b]{\linewidth}\centering
\[L_{x}'\]
\end{minipage} & \begin{minipage}[b]{\linewidth}\centering
\[G_{x}'\]
\end{minipage} & \begin{minipage}[b]{\linewidth}\centering
\[\Delta t'\]
\end{minipage} & \begin{minipage}[b]{\linewidth}\centering
\[T_{S}'\]
\end{minipage} & \begin{minipage}[b]{\linewidth}\centering
SNR
\end{minipage} \\
\midrule\noalign{}
\endhead
\bottomrule\noalign{}
\endlastfoot
Aumento FOV (risol. costante) & \(\Delta x\) & \(2N_{x}\) & \(2L_{x}\) & \(G_{x}\) & \(\Delta t/2\) & \(T_{S}\) & \(1\) \\
Aumento voxel (risol. ↓) & \(2\Delta x\) & \(N_{x}/2\) & \(L_{x}\) & \(G_{x}/2\) & \(2\Delta t\) & \(T_{S}\) & \(2\) \\
Riduzione voxel (risol. ↑) & \(\Delta x/2\) & \(2N_{x}\) & \(L_{x}\) & \(2G_{x}\) & \(\Delta t/2\) & \(T_{S}\) & \(1/2\) \\
\end{longtable}

Da questi esperimenti si osserva che il rapporto segnale/rumore non può essere aumentato a piacere, poiché l'incremento di questa quantità causa l'allargamento del voxel, perdendo in risoluzione spaziale. Viceversa, la risoluzione spaziale non può essere ridotta a piacere poiché ciò determina un aumento della quota di rumore nel voxel, riducendo il rapporto segnale/rumore e un'immagine maggiormente degradata.

In base all'applicazione richiesta si sceglie il giusto compromesso tra rapporto segnale/rumore e risoluzione.

Va osservato, inoltre, che la riduzione della risoluzione spaziale determina un aumento dei gradenti di campo applicati; dunque, anche la tecnologia di imaging tramite risonanza magnetica deve essere maggiormente performante.

\subsection{Dipendenza del SNR dall'asse di phase encoding}\label{dipendenza-del-snr-dallasse-di-phase-encoding}

Il rapporto segnale/rumore, in genere, dipende dai parametri che caratterizzano l'immagine e la sequenza di acquisizione, secondo la relazione:

\[SNR/voxel \propto \Delta x\Delta y\Delta z\sqrt{N_{y}N_{z}T_{S}}\sqrt{N_{acq}}\]

Si suppone di acquisire una sola sequenza, dunque \(N_{acq} = 1\). Generalmente, la dimensione di \emph{slice selection}, \(\Delta z\), è indicato con \emph{slice thickness,} \(TH\). Trascurando i parametri legati alla direzione di lettura, si ottiene la dipendenza del segnale/rumore dai parametri relativi alle direzioni di codifica di fase e selezione della slice:

\[SNR/voxel \propto \Delta y\Delta z\sqrt{N_{y}N_{z}}\sqrt{N_{acq}}\]

I parametri nella relazione precedente sono legati tra loro mediante il FOV:

\[\left\{ \begin{matrix}
L_{y} = N_{y}\Delta y \\
L_{z} = N_{z}\Delta z
\end{matrix} \right.\ \]

L'ultima relazione può essere scritta anche usano la notazione \(\Delta z = TH\):

\[L_{z} = N_{z}TH\]

In questo contesto, ci sono solamente due modi per aumentare la risoluzione spaziale nella direzione di codifica di fase: mantenere costante il FOV o il numero di campioni.

Con la prima soluzione, il FOV viene mantenuto costante, mentre le risoluzioni spaziali \(\Delta y\) e \(\Delta z\) sono ridotte. Con questa scelta, il numero di campioni lungo la direzione di codifica di fase deve aumentare conseguentemente alla risoluzione spaziale:

\[N_{y} = \dfrac{L_{y}}{\Delta y}\]

Se, ad esempio, la risoluzione spaziale viene dimezzata, \(\Delta y' = \Delta y/2\), il numero dei punti raddoppia:

\[N_{y}' = \dfrac{L_{y}}{\Delta y'} = 2\dfrac{L_{y}}{\Delta y} = 2N_{y}\]

Il secondo metodo consiste nel diminuire le risoluzioni spaziali, \(\Delta y\) e \(\Delta z\), e mantenere costante il numero di punti su cui ricostruire l'immagine. Il FOV, di conseguenza, deve ridursi proporzionalmente alla risoluzione spaziale:

\[L_{y} = N_{y}\Delta y\]

Ad esempio, dimezzando la risoluzione spaziale, \(\Delta y' = \Delta y/2\), il FOV viene dimezzato:

\[L_{y}' = N_{y}\Delta y' = N_{y}\dfrac{\Delta y}{2} = \dfrac{L_{y}}{2}\]

Il rapporto segnale/rumore, mantenendo il FOV costante (\(L_{x}' = L_{x}\)) e dimezzando la risoluzione spaziale (\(\Delta y' = \Delta y/2\)), si riduce di un fattore \(\sqrt{2}\):

\[{SNR}'/voxel \propto \Delta y'\Delta z\sqrt{N_{y}'N_{z}} = \dfrac{\Delta y}{2}\sqrt{2N_{y}N_{z}} = \dfrac{\Delta y}{\sqrt{2}}\sqrt{N_{y}N_{z}} = \dfrac{1}{\sqrt{2}}SNR/voxel\]

Mantenendo costante il numero di punti e dimezzando la risoluzione spaziale, il rapporto segnale/rumore si dimezza anch'esso:

\[{SNR}'/voxel \propto \Delta y'\Delta z\sqrt{N_{y}N_{z}} = \dfrac{\Delta y}{2}\sqrt{N_{y}N_{z}} = \dfrac{1}{2}SNR/voxel\]

Si conclude che, per migliorare la risoluzione spaziale, è conveniente mantenere costante il FOV lungo la direzione di codifica di fase. Il rapporto segnale/rumore, in questa condizione, decresce come la radice della variazione introdotta sulla risoluzione. Ancora, se si desidera un rapporto segnale/rumore migliore è necessario aumentare la dimensione del voxel, \(\Delta y\Delta z\).

Si osservi, in fine, che un aumento del numero di punti determina un amento della complessità computazionale degli algoritmi di elaborazione dei dati. Ciò porta a un tempo di risposta maggiore da parte dell'elaboratore digitale a causa del maggior numero di dati da elaborare da parte dell'algoritmo di ricostruzione.

\subsection{Dipendenza del SNR nello spazio}\label{dipendenza-del-snr-nello-spazio}

Si considera l'espressione per il rapporto segnale/rumore nelle tre dimensioni spaziali:

\[SNR/voxel \propto \Delta x\Delta y\Delta z\sqrt{N_{y}N_{z}T_{S}}\sqrt{N_{acq}}\]

Il rapporto segnale/rumore può essere molto spunto lungo l'asse di lettura, supposto coincidente con \(x\), poiché è possibile agire più facilmente sui parametri che caratterizzano questa direzione. Viceversa, lungo l'asse di codifica di fase e selezione della slice, è necessario rispettare il criterio di Nyquist, al fine di non introdurre dei ghost nell'immagine. Ciò determina un minor margine di manovra sulla variazione della risoluzione spaziale e del FOV.

L'asse di selezione della fetta, supposto coincidente con \(z\), è legato anche alla durata complessiva dell'esame; infatti, maggiore è il numero delle slice acquisite e maggiore è il tempo necessario al fine di acquisire i dati necessari per la ricostruzione volumetrica dell'intero distretto di interessa.

Sulla base delle esigenze legate all'esame diagnostico, che richiede un certo FOV nelle direzioni spaziali \(x\), \(y\) e \(z\), si fissano le risoluzioni spaziali \(\Delta x\) e \(\Delta y\) nelle direzioni di codifica di fase e \emph{slice selection}, mentre si varia la risoluzione spaziale \(\Delta x\) e/o il tempo di acquisizione \(T_{S}\) in base alla richiesta di elevata risoluzione (primo caso) o alto rapporto segnale/rumore. In genere, si cerca sempre un compromesso tra ottima risoluzione spaziale e alto rapporto segnale/rumore.

\subsection{Rapporto contrasto/rumore}\label{rapporto-contrastorumore}

Anche il rapporto segnale/rumore più elevato non garantisce la possibilità di distinguere due oggetti diversi ma posti molto vicini tra loro.

In un'immagine diagnostica, l'SNR è un parametro importante, ma non l'unico: altrettanto fondamentale è il contrasto, definito come la differenza tra i segnali provenienti da due tessuti, indicati con \(A\) e \(B\):

\[C_{AB} = s_{A} - s_{B}\]

Dove \(s_{A}\) è il segnale del voxel del tessuto \(A\) e \(s_{B}\) del tessuto \(B\)-

\begin{figure}
\centering
\includegraphics[width=4.00903in,height=2.74013in]{media/11_SNR/image304.pdf}\caption{Tabella 11.2: Schema contrasto}
\end{figure}

Tuttavia, anche se il contrasto tra due tessuti è sufficientemente grande da distinguere i due tessuti nell'immagine, il rumore sovrapposto nel voxel può essere tale che l'occhio umano non riesca a quantificare i due tessuti come diversi.

Il parametro più accurato per quantificare quanto i tessuti siano distinguibili è il rapporto contrasto/rumore (\emph{contrast to noise ratio} o CNR), definito come il rapporto tra il contrasto tra due tessuti e il rumore che corrompe il segnale del voxel:

\[{CNR}_{AB} = \dfrac{C_{AB}}{\sigma_{m}} = \dfrac{s_{A} - s_{B}}{\sigma_{m}} = {SNR}_{A} - {SNR}_{B}\]

In definitiva, il rapporto contrasto/rumore è dato dalla differenza dei rapporti segnale/rumore del due tessuti.

Nella pratica, è possibile avere un rapporto segnale/rumore elevato nei due tessuti ma, se la differenza tra i due SNR è tale che il CNR sia molto basso, i due tessuti sono difficilmente distinguibili.

Per distinguere bene i due tessuti è necessario avere un alto valore del rapporto contrasto/rumore. Si suppone che il segnale del voxel dei due tessuti, \(s_{A}\) e \(s_{B}\), il CNR è ottenuto come distanza sull'asse \(s\) del segnale dei due tessuti, rapportati alla deviazione standard \(\sigma\), caratterizzante il rumore.

Il rumore è modellabile come una gaussiana di varianza \(\sigma\) e valor meglio \(s_{A}\) o \(s_{B}\). Dal punto di vista geometrico, quindi, dire che il CNR è abbastanza elevato da poter distinguere i due tessuti equivale ad avere due gaussiane, centrate su \(s_{A}\) e \(s_{B}\), sufficientemente separate, così da non interferire tra loro.

Se le gaussiane hanno una varianza \(\sigma\) elevata, ovvero la potenza del rumore è sufficientemente alta, le due campane interferiscono tra loro, rendendo difficile la discriminazione dei due tessuti, in quanto la differenza tra i due rapporti segnale/rumore sono molto bassi.

\begin{figure}
\centering
\includegraphics[width=6.68542in,height=3.22222in]{media/11_SNR/image305.pdf}\caption{Tabella 11.3: Rappresentazione grafica del CNR}
\end{figure}

Eseguendo un numero \(N_{acq}\) di misure e applicando l'operazione di media statistica nello spazio-immagine, la deviazione standard è scalata di un fattore \(\sqrt{N_{acq}}\):

\[\sigma_{0} = \dfrac{\sigma_{m}}{\sqrt{N_{acq}}}\]

Ne discende che il rapporto contrasto/rumore aumenta di una quantità \(\sqrt{N_{acq}}\):

\[CNR = \dfrac{s_{A} - s_{B}}{\sigma_{0}} = \ \sqrt{N_{acq}}C_{AB}\]

\subsection{Differenza di contrasto nei vari tessuti}\label{differenza-di-contrasto-nei-vari-tessuti}

La risonanza magnetica presenta una grande flessibilità nel manipolare i segnali provenienti dai tessuti, mediante varie metodiche che portano all'ottenimento di immagini a diverso contrasto.

Il metodo fondamentale per discriminare i vari tessuti in base al contrasto è quello di sfruttare le differenze nelle loro caratteristiche chimico-fisiche, in particolare la densità protonica (\(\rho\)), il tempo di rilassamento longitudinale (\(T_{1}\)) e il tempo di rilassamento trasversale (\(T_{2}\)). A seconda dei parametri della sequenza di acquisizione scelti, è possibile ottenere immagini il cui contrasto è prevalentemente determinato da uno di questi parametri. Di conseguenza, si ottengono immagini pesate in densità protonica (PD), pesate in \(T_{1}\) o pesate in \(T_{2}\) o \(T_{2}^{*}\).

\subsubsection{Pesatura dell'immagine}\label{pesatura-dellimmagine}

La densità protonica è legata al segnale misurato nel \(k\)-spazio mediante una trasformata inversa di Fourier:

\[\widehat{\rho}(x) = \int_{}^{}{s(k)\exp(j2\pi kx)dk}\]

Il segnale registrato nel \(k\)-spazio è proporzionale alla magnetizzazione trasversa \(M_{\bot}\left( t,\overset{\underline{}}{r} \right)\), ottenuto dopo un impulso di perturbazione. La componente trasversa è, poi, misurata delle antenne. In altre parole, la densità protonica, punto per punto, è proporzionale alla magnetizzazione trasversa:

\[\widehat{\rho}(x) \propto M_{\bot}\]

Si considera un volume di materia al quale applicare una sequenza del tipo gradient-echo, composta da un impulso a radiofrequenza che ribalta la magnetizzazione di \(\pi/2\), un gradiente di selezione della fetta, un gradiente di phase encoding e, in fine, un gradiente di lettura sulla quale è applicata un gradient-echo.

\begin{figure}
\centering
\includegraphics[width=6.69306in,height=6.32083in,alt={Immagine che contiene testo, diagramma, linea, Disegno tecnico Il contenuto generato dall\textquotesingle IA potrebbe non essere corretto.}]{media/11_SNR/image306.pdf}\caption{Figura .: Sequenza gradient-echo bidimensionale}
\end{figure}

Il tempo di echo, \(T_{E}\), è ottenuto quando l'area del gradiente di rifasamento uguaglia quella di defasamento. Generalmente, la finestra di acquisizione, con tempo \(T_{S}\), è centrata sul tempo di echo.

Il segnale del voxel \(\rho\) è proporzionale alla magnetizzazione trasversa che, al tempo \(t = 0\ s\), coincide con il valore di equilibrio \(M_{0}\), dato che la magnetizzazione longitudinale è ribaltata di \(\pi/2\) per l'applicazione dell'impulso a radiofrequenza. Successivamente, la magnetizzazione trasversa decade come \(T_{2}\), sebbene il segnale decada come \(T_{2}^{*}\) per le disomogeneità di campo. Al tempo d'echo, si verifica un primo fronte di rifasamento e, successivamente, un defasamento con inviluppo che decade con \(T_{2}^{*}\). A centro della finestra di acquisizione.

Il tempo di eco, \(T_{E}\), si ottiene quando l'area del gradiente di rifasamento uguaglia quella di defasamento. Generalmente, la finestra di acquisizione, di durata \(T_{S}\), è centrata sul tempo di eco.

Il segnale del voxel \(\rho\) è proporzionale alla magnetizzazione trasversa che, al tempo \(t\  = \ 0\ s\), coincide con il valore di equilibrio \(M_{0}\), poiché la magnetizzazione longitudinale è stata ribaltata di \(\pi/2\) dall'applicazione dell'impulso a radiofrequenza. Successivamente, la magnetizzazione trasversa decade con una costante di tempo \(T_{2}\), sebbene il segnale decada con \(T_{2}^{*}\) a causa delle disomogeneità di campo.

Al tempo di eco si verifica un primo fronte di rifasamento, seguito da un defasamento con un inviluppo che decade secondo \(T_{2}^{*}\), centrato nella finestra di acquisizione.

Il segnale osservato è solamente una piccola porzione interno al tempo d'echo, tale da non avvertire gli effetti dei tempi di rilassamento \(T_{1}\), \(T_{2}\) e \(T_{2}^{*}\).

Si suppone di ripetere la sequenza di eccitazione una seconda volta dopo un tempo \(T_{R}\). Si ritiene, inoltre, che il tempo di ripetizioni tra una sequenza e la successiva sia tale che gli effetti del tempo di rilassamento trasversale siano esauriti:

\[T_{R} \gg T_{2}\]

In questo modo la magnetizzazione trasversa, quando la sequenza gradient-echo è ripetuta, è nulla. Tuttavia, la magnetizzazione non recupera il valore all'equilibrio poiché il tempo di rilassamento longitudinale è maggiore di quello trasversale, \(T_{1} > T_{2}\); di conseguenza, il valore di magnetizzazione longitudinale, all'inizio della nuova sequenza è:

\[M_{z}\left( T_{R} \right) = M_{0}\left( 1 - \exp\left( - \dfrac{T_{R}}{T_{1}} \right) \right)\]

Nell'istante di ripetizione, \(t = T_{R}\), si applica l'impulso a \(\pi/2\) che ribalta la magnetizzazione longitudinale nel piano ortogonale. Ne discende che tra il primo impulso e il secondo non si registra un segnale con la stessa ampiezza ma una sua versione attenuata di un fattore dipendente da \(T_{R}\) e \(T_{1}\).

Al secondo echo, si misura, appunto, un valore di ampiezza minore della magnetizzazione trasversa di un fattore:

\[M_{\bot}\left( T_{E},T_{R} \right) = M_{0}\left( 1 - \exp\left( - \dfrac{T_{R}}{T_{1}} \right) \right)\exp\left( - \dfrac{T_{E}}{T_{2}^{*}} \right)\]

Dunque, il segnale presente nel voxel è proporzionale a questa quantità. Dalla seconda applicazione della sequenza in poi, il segnale registrato resta praticamente uguale. Ovviamente, questi ragionamenti valgono anche per la spin-echo.

In definitiva, ogni tessuto contenuto nel voxel presenta un segnale proporzionale a:

\[s\left( \rho,T_{1},T_{2} \right) \propto \rho(x,y,z) \propto M_{0}\left( 1 - \exp\left( - \dfrac{T_{R}}{T_{1}} \right) \right)\exp\left( - \dfrac{T_{E}}{T_{2}^{*}} \right)\]

Scegliendo opportunamente i tempi \(T_{R}\) e \(T_{E}\) è possibile controllare il livello del segnale del voxel, quindi la sua pesatura.

\begin{figure}
\centering
\includegraphics[width=6.69306in,height=5in,alt={Immagine che contiene testo, diagramma, linea, Parallelo Il contenuto generato dall\textquotesingle IA potrebbe non essere corretto.}]{media/11_SNR/image307.pdf}\caption{Figura .: Segnale registrato tra una sequenza di acquisizione e la successiva}
\end{figure}

\subsubsection{Pesatura in densità protonica}\label{pesatura-in-densituxe0-protonica}

È possibile realizzare una sequenza di acquisizione tale per cui il tempo di ripetizione sia sufficientemente lungo da rendere trascurabili gli effetti del rilassamento \(T_{1}\) e \(T_{2}\); inoltre, il tempo di echo deve essere abbastanza breve da poter minimizzare gli effetti del tempo di rilassamento trasversale \(T_{2}\):

\[\left\{ \begin{matrix}
T_{R} \rightarrow \infty \\
T_{E} \rightarrow 0
\end{matrix} \right.\ \]

Nella pratica, queste condizioni si traducono in:

\[\left\{ \begin{matrix}
T_{R} \gg T_{1} \\
T_{E} \ll T_{2}^{*}
\end{matrix} \right.\ \]

\begin{figure}
\centering
\includegraphics[width=6.68958in,height=5.04514in]{media/11_SNR/image308.pdf}\caption{Figura .: Sequenza per ottenere immagini pesate in densità protonica}
\end{figure}

Con questa scelta gli esponenziali presenti nella relazione del segnale nel voxel:

\[s_{0} \propto M_{0}\left( 1 - \exp\left( - \dfrac{T_{R}}{T_{1}} \right) \right)\exp\left( - \dfrac{T_{E}}{T_{2}^{*}} \right)\]

Tendono rispettivamente a:

\[\exp\left( - \dfrac{T_{R}}{T_{1}} \right) \ll 1,\ T_{R} \gg T_{1}\]

\[\exp\left( - \dfrac{T_{E}}{T_{2}^{*}} \right) \simeq 1,\ T_{E} \ll T_{2}^{*}\]

Il segnale, in queste condizioni, dipende essenzialmente dal valore della magnetizzazione all'equilibrio termodinamico \(M_{0}\), la quale a sua volta dipende dalla densità protonica:

\[s\left( T_{1},T_{2} \right) \propto M_{0} \propto \rho\]

\paragraph{Contrasto con immagini pesate in densità protonica}\label{contrasto-con-immagini-pesate-in-densituxe0-protonica}

Si considerano due tessuti \(A\) e \(B\), a cui corrispondono, rispettivamente, i segnali \(s_{A}\) e \(s_{B}\). Il contrasto tra i due tessuti è, per definizione, dato dalla differenza dei due segnali:

\[C_{AB} = s_{A} - s_{B}\]

Si sostituiscono le espressioni dei due segnali:

\[C_{AB} = M_{0,A}\left( 1 - \exp\left( - \dfrac{T_{R}}{T_{1,A}} \right) \right)\exp\left( - \dfrac{T_{E}}{T_{2,A}^{*}} \right) - M_{0,B}\left( 1 - \exp\left( - \dfrac{T_{R}}{T_{1,B}} \right) \right)\exp\left( - \dfrac{T_{E}}{T_{2,B}^{*}} \right)\]

Siccome \(T_{E} \ll T_{2}^{*}\) allora \(T_{E}/T_{2}^{*} \ll 1\), è possibile sviluppare in serie di Taylor \(\exp\left( - T_{E}/T_{2}^{*} \right)\):

\[\exp\left( - \dfrac{T_{E}}{T_{2}^{*}} \right) \simeq 1 - \dfrac{T_{E}}{T_{2}^{*}}\]

Il contrasto può essere scritto come:

\[C_{AB} \simeq M_{0,A}\left( 1 - \exp\left( - \dfrac{T_{R}}{T_{1,A}} \right) \right)\left( 1 - \dfrac{T_{E}}{T_{2,A}^{*}} \right) - M_{0,B}\left( 1 - \exp\left( - \dfrac{T_{R}}{T_{1,B}} \right) \right)\left( 1 - \dfrac{T_{E}}{T_{2,B}^{*}} \right)\]

Si svolgono i prodotti:

\[= M_{0,A}\left( 1 - \dfrac{T_{E}}{T_{2,A}^{*}} - \left( 1 - \dfrac{T_{E}}{T_{2,A}^{*}} \right)\exp\left( - \dfrac{T_{R}}{T_{1,A}} \right) \right) - M_{0,B}\left( 1 - \dfrac{T_{E}}{T_{2,B}^{*}} - \left( 1 - \dfrac{T_{E}}{T_{2,B}} \right)\exp\left( - \dfrac{T_{R}}{T_{1,B}} \right) \right) = M_{0,A}\left( 1 - \dfrac{T_{E}}{T_{2,A}^{*}} - \exp\left( - \dfrac{T_{R}}{T_{1,A}} \right) + \dfrac{T_{E}}{T_{2,A}^{*}}\exp\left( - \dfrac{T_{R}}{T_{1,A}} \right) \right) - M_{0,B}\left( 1 - \dfrac{T_{E}}{T_{2,B}^{*}} - \exp\left( - \dfrac{T_{R}}{T_{1,B}} \right) + \dfrac{T_{E}}{T_{2,B}^{*}}\exp\left( - \dfrac{T_{R}}{T_{1,B}} \right) \right) = M_{0,A} - M_{0,A}\dfrac{T_{E}}{T_{2,A}^{*}} - M_{0,A}\exp\left( - \dfrac{T_{R}}{T_{1,A}} \right) + M_{0,A}\dfrac{T_{E}}{T_{2,A}^{*}}\exp\left( - \dfrac{T_{R}}{T_{1,A}} \right) - \left( M_{0,B} - M_{0,B}\dfrac{T_{E}}{T_{2,B}^{*}} - M_{0,B}\exp\left( - \dfrac{T_{R}}{T_{1,B}} \right) + M_{0,B}\dfrac{T_{E}}{T_{2,B}^{*}}\exp\left( - \dfrac{T_{R}}{T_{1,B}} \right) \right) = M_{0,A} - M_{0,A}\dfrac{T_{E}}{T_{2,A}^{*}} - M_{0,A}\exp\left( - \dfrac{T_{R}}{T_{1,A}} \right) + M_{0,A}\dfrac{T_{E}}{T_{2,A}^{*}}\exp\left( - \dfrac{T_{R}}{T_{1,A}} \right) - M_{0,B} + M_{0,B}\dfrac{T_{E}}{T_{2,B}^{*}} + M_{0,B}\exp\left( - \dfrac{T_{R}}{T_{1,B}} \right) - M_{0,B}\dfrac{T_{E}}{T_{2,B}^{*}}\exp\left( - \dfrac{T_{R}}{T_{1,B}} \right)\]

Si riorganizza i termini del contrasto:

\[C_{AB} \simeq \left( M_{0,A} - M_{0,B} \right) - M_{0,A}\left( \dfrac{T_{E}}{T_{2,A}^{*}} + \exp\left( - \dfrac{T_{R}}{T_{1,A}} \right) \right) + M_{0,B}\left( \dfrac{T_{E}}{T_{2,B}^{*}} + \exp\left( - \dfrac{T_{R}}{T_{1,B}} \right) \right)\]

Nell'ipotesi in cui \(T_{E} \ll T_{2,A}^{*}\), \(T_{E} \ll T_{2,B}^{*}\), \(T_{R} \gg T_{1,A}\) e \(T_{R} \gg T_{1,B}\) è possibile trascurare i termini tra parantesi:

\[\dfrac{T_{E}}{T_{2,A}^{*}} \ll 1,\ \dfrac{T_{E}}{T_{2,B}^{*}} \ll 1,\ \exp\left( - \dfrac{T_{R}}{T_{1,A}} \right) \ll 1,\ \exp\left( - \dfrac{T_{R}}{T_{1,B}} \right) \ll 1\]

Di conseguenza, questi termini tendono a zero, per cui il contrasto può essere approssimato come:

\[C_{AB} \simeq M_{0,A} - M_{0,B}\]

La differenza dei segnali nei due tessuti dipende, all'incirca, dalla differenza delle due magnetizzazioni all'equilibrio nei due voxel. Siccome, la magnetizzazione dipende essenzialmente dalla densità protonica nei due voxel, il contrasto è legato alla differenza tra le densità protoniche dei due tessuti:

\[C_{AB} \propto \rho_{0,A} - \rho_{0,B}\]

La relazione \(C_{AB} \simeq M_{0,A} - M_{0,B}\) è rigorosamente valida solo nel limite \(T_{E} \rightarrow 0\) e \(T_{R} \rightarrow \infty\). In condizioni reali, il contrasto dipende anche dai tempi di rilassamento \(T_{1}\) e \(T_{2}\) dei voxel, quindi, la dipendenza dalla densità protonica non è così spiccata. La dipendenza dai tempi di rilassamento può essere controllata entro certi limiti, legati ai tempi dell'esperimento.

Le immagini ottenute con le condizioni \(T_{E} \ll T_{2}\) e \(T_{R} \gg T_{1}\), sono dette pesate in densità protonica (\emph{proton density weighted}), in quanto il contributo dominante al segnale deriva dalla densità protonica, sebbene vi sia anche la dipendenza dai tempi di rilassamento longitudinale \(T_{1}\) e trasversale \(T_{2}\).

In base all'approssimazione usata per il contrasto \(C_{AB}\) si ottengono valori diversi, anche di molti punti percentuali, quindi, nelle analisi di risonanza magnetica si utilizza la relazione non sviluppata al primo ordine di Taylor.

Le differenze tra le varie approssimazioni sono legate alle iterazioni tra i parametri tempo di ripetizione, \(T_{R}\), tempo d'echo, \(T_{E}\), tempo di rilassamento trasversale, \(T_{2}\), e longitudinale,\(T_{1}\). Per tempi di echo crescenti, il contrasto si riduce mentre al crescere del tempo di ripetizioni il contrasto aumenta. Infatti, il segnale rilevato in un voxel è descritto dall'equazione:

\[s_{0} \propto M_{0}\left( 1 - \exp\left( - \dfrac{T_{R}}{T_{1}} \right) \right)\exp\left( - \dfrac{T_{E}}{T_{2}} \right)\]

Il termine \(\exp\left( - T_{E}/T_{2} \right)\) descrive lo smorzamento della magnetizzazione trasversale. All'aumentare di \(T_{E}\), l'argomento dell'esponenziale si riduce, per cui il segnale complessivo si riduce, così come la differenza tra i segnali dei tessuti \(A\) e \(B\). Di conseguenza, \textbf{un tempo di eco più lungo riduce il contrasto}.

Il termine \(1 - \exp\left( - T_{R}/T_{1} \right)\) descrive il recupero della magnetizzazione longitudinale. Con \(T_{R}\) piccolo, il recupero non è completo, e il segnale risulta ridotto. All'aumentare di \(T_{R}\), invece, il termine \(1 - \exp\left( - T_{R}/T_{1} \right)\) tende a \(1\) e il segnale tende al massimo valore possibile \(M_{0}\). La differenza tra i segnali dei tessuti aumenta. In altre parole, \textbf{un tempo di ripetizione più lungo incrementa il contrasto}.

Questa relazione evidenzia come la scelta dei parametri sperimentali influisca sul contrasto tra tessuti e consente di ottenere immagini pesate in densità protonica, in \(T_{1}\) o in \(T_{2}\) a seconda delle esigenze diagnostiche.

Sulla base dei parametri della sequenza di ripetizione, è possibile evidenziare una caratteristica biochimica (densità protonica, tempo di rilassamento longitudinale o trasversale) dei tessuti.

\begin{longtable}[]{@{}
  >{\centering\arraybackslash}p{(\linewidth - 4\tabcolsep) * \real{0.3333}}
  >{\centering\arraybackslash}p{(\linewidth - 4\tabcolsep) * \real{0.3333}}
  >{\centering\arraybackslash}p{(\linewidth - 4\tabcolsep) * \real{0.3334}}@{}}
\caption{Tabella 11.4: Scelta dei diversi parametri della sequenza per ottenere una specifica pesatura}\tabularnewline
\toprule\noalign{}
\begin{minipage}[b]{\linewidth}\centering
Tipo di contrasto
\end{minipage} & \begin{minipage}[b]{\linewidth}\centering
\[T_{R}\]
\end{minipage} & \begin{minipage}[b]{\linewidth}\centering
\[T_{E}\]
\end{minipage} \\
\midrule\noalign{}
\endfirsthead
\toprule\noalign{}
\begin{minipage}[b]{\linewidth}\centering
Tipo di contrasto
\end{minipage} & \begin{minipage}[b]{\linewidth}\centering
\[T_{R}\]
\end{minipage} & \begin{minipage}[b]{\linewidth}\centering
\[T_{E}\]
\end{minipage} \\
\midrule\noalign{}
\endhead
\bottomrule\noalign{}
\endlastfoot
Spin density & Più lungo possibile & Più breve possibile \\
\(T_{1}\)-pesato & Dello stesso ordine di \(T_{1}\) & Più breve possibile \\
\(T_{2}\)-pasato & Più lungo possibile & Dello stesso ordine di \(T_{2}\) \\
\end{longtable}

\subsubsection{Pesatura in tempo di rilassamento longitudinale}\label{pesatura-in-tempo-di-rilassamento-longitudinale}

È possibile realizzare una sequenza di acquisizione spin-echo o gradient-echo con tempo di echo tale da non sentire gli effetti del rilassamento trasversale, \(T_{2}\), e un tempo di ripetizione \(T_{R}\) dello stesso ordine di grandezza del tempo di rilassamento longitudinale:

\[\left\{ \begin{matrix}
T_{R} \sim T_{1} \\
T_{E} \rightarrow 0
\end{matrix} \right.\ \]

Queste condizioni, all'atto pratico, si traducono in:

\[\left\{ \begin{matrix}
T_{R} \sim T_{1} \\
T_{E} \ll T_{2}^{*}
\end{matrix} \right.\ \]

In queste ipotesi, il segnale del voxel è:

\[s_{0} \propto M_{0}\left( 1 - \exp\left( - \dfrac{T_{R}}{T_{1}} \right) \right)\exp\left( - \dfrac{T_{E}}{T_{2}^{*}} \right)\]

Nell'ipotesi \(T_{E} \ll T_{2}^{*}\), il termine esponenziale tende all'unità:

\[\exp\left( - \dfrac{T_{E}}{T_{2}} \right) \sim 1\]

Per cui il segnale dipende solamente da \(M_{0}\) e \(T_{1}\):

\[s_{0} \propto M_{0}\left( 1 - \exp\left( - \dfrac{T_{R}}{T_{1}} \right) \right)\]

Se il tempo \(T_{R} \ll T_{1}\) i segnali dei voxel, il termine esponenziale \(\exp\left( - T_{R}/T_{1} \right)\) tende a zero, mentre, se \(T_{1}\) è paragonabile a \(T_{R}\), l'esponenziale assume un valore minore. Di conseguenza tessuti con minor tempo di rilassamento longitudinale presentano un \emph{voxel signal} di ampiezza maggiore.

\paragraph[Contrasto in immagini pesate in T1]{Contrasto in immagini pesate in $\mathbf{T}_{\mathbf{1}}$}

Il contrasto tra due tessuti può essere scritto come:

\[C_{AB} = s_{0,A} - s_{0,B} \simeq M_{0,A}\left( 1 - \exp\left( - \dfrac{T_{R}}{T_{1,A}} \right) \right) - M_{0,B}\left( 1 - \exp\left( - \dfrac{T_{R}}{T_{1,B}} \right) \right)\]

Dall'espressione appena individuata si evince che Il termine \(1 - \exp\left( - T_{R}/T_{1} \right)\) descrive il recupero della magnetizzazione lungo l'asse longitudinale

\begin{itemize}
\item
  Se \(T_{1}\) è lungo rispetto a \(T_{R}\), il recupero non è completo e il segnale rilevato è ridotto;
\item
  Se \(T_{1}\) è breve rispetto a \(T_{R}\), la magnetizzazione si recupera rapidamente e il segnale risulta maggiore.
\end{itemize}

In altre parole, tessuti con \(T_{1}\) \textbf{breve} producono segnale più intenso a \(T_{R}\) breve, mentre tessuti con \(T_{1}\) \textbf{lungo} producono segnale più debole se \(T_{R}\) non è sufficientemente lungo.

Esiste un unico valore del tempo di ripetizione, che, fissato i tessuti, permette di ottenere il miglior contrasto possibile per le immagini \(T_{1}\) pesate. Il valore ottimo del tempo di ripetizione è ottenuto utilizzato l'operazione di derivata:

\[\dfrac{\partial C_{AB}}{\partial T_{R}} = \dfrac{\partial}{\partial T_{R}}\left\lbrack M_{0,A}\left( 1 - \exp\left( - \dfrac{T_{R,opt}}{T_{1,A}} \right) \right) - M_{0,B}\left( 1 - \exp\left( - \dfrac{T_{R,opt}}{T_{1,B}} \right) \right) \right\rbrack = 0\]

Applicando le proprietà dell'operazione di derivata, si ottiene:

\[\dfrac{M_{0,A}}{T_{1,A}}\exp\left( - \dfrac{T_{R,opt}}{T_{1,A}} \right) - \dfrac{M_{0,B}}{T_{1,B}}\exp\left( - \dfrac{T_{R,opt}}{T_{1,B}} \right) = 0\]

Si portano i termini dipendenti dal tessuto \(B\) al secondo membro:

\[\dfrac{M_{0,A}}{T_{1,A}}\exp\left( - \dfrac{T_{R,opt}}{T_{1,A}} \right) = \dfrac{M_{0,B}}{T_{1,B}}\exp\left( - \dfrac{T_{R,opt}}{T_{1,B}} \right)\]

Si portano i termini esponenziali al primo membro e i termini lineari al secondo:

\[\exp\left( - \dfrac{T_{R,opt}}{T_{1,A}} \right)\exp\left( \dfrac{T_{R,opt}}{T_{1,B}} \right) = \dfrac{M_{0,B}}{T_{1,B}}\dfrac{T_{1,A}}{M_{0,A}}\]

Per le proprietà degli esponenziali risulta:

\[\exp\left( \dfrac{T_{R,opt}}{T_{1,B}} - \dfrac{T_{R,opt}}{T_{1,A}} \right) = \dfrac{M_{0,B}}{T_{1,B}}\dfrac{T_{1,A}}{M_{0,A}}\]

Per isolare il tempo di ripetizione \(T_{R}\) si applica il logaritmo ambo i membri:

\[\dfrac{T_{R,opt}}{T_{1,B}} - \dfrac{T_{R,opt}}{T_{1,A}} = \log\left( \dfrac{M_{0,B}}{T_{1,B}}\dfrac{T_{1,A}}{M_{0,A}} \right)\]

Per le proprietà dei logaritmi si ha:

\[\dfrac{T_{R,opt}}{T_{1,B}} - \dfrac{T_{R,opt}}{T_{1,A}} = \log\left( \dfrac{M_{0,B}}{T_{1,B}} \right) + \log\left( \dfrac{T_{1,A}}{M_{0,A}} \right) = \log\left( \dfrac{M_{0,B}}{T_{1,B}} \right) - \log\left( \dfrac{M_{0,A}}{T_{1,A}} \right)\]

Il tempo di ripetizione ottimale per ottenere il miglior contrasto tra i tessuti \(A\) e \(B\) è:

\[T_{R,opt} = \dfrac{\log\left( \dfrac{M_{0,B}}{T_{1,B}} \right) - \log\left( \dfrac{M_{0,A}}{T_{1,A}} \right)}{\dfrac{1}{T_{1,B}} - \dfrac{1}{T_{1,A}}}\]

La conoscenza a priori delle proprietà dei tessuti da visualizzare è fondamentale per poter selezionare il tempo di ripetizione tra una sequenza e la successiva. Scegliendo il valore ottimo del tempo di ripetizione si ottiene il valore massimo del contrasto tra i due tessuti.

Il massimo della funzione contrasto può essere ricavato anche analizzando l'andamento stesso di \(C_{AB}\left( T_{R} \right)\) al variare del tempo di ripetizione.

Si osservi che esiste un valore del tempo di ripetizione \(T_{R}\) in cui il contrasto è nullo.

\[C_{AB}\left( T_{R} \right) = s_{0,A} - s_{0,B} \simeq M_{0,A}\left( 1 - \exp\left( - \dfrac{T_{R}}{T_{1,A}} \right) \right) - M_{0,B}\left( 1 - \exp\left( - \dfrac{T_{R}}{T_{1,B}} \right) \right) = 0\]

Si svolgono i prodotti:

\[M_{0,A} - M_{0,A}\exp\left( - \dfrac{T_{R}}{T_{1,A}} \right) - M_{0,B} + M_{0,B}\exp\left( - \dfrac{T_{R}}{T_{1,B}} \right) = 0\]

Si portano al secondo membro i termini non dipendenti da \(T_{R}\):

\[M_{0,B}\exp\left( - \dfrac{T_{R}}{T_{1,B}} \right) - M_{0,A}\exp\left( - \dfrac{T_{R}}{T_{1,A}} \right) = M_{0,B} - M_{0,A}\]

L'equazione trascendentale e non può essere risolta per \(T_{R}\) in termini di funzioni elementari. Questo accade perché l'equazione contiene \(T_{R}\) all'interno di più esponenti con basi diverse, il che la rende intrinsecamente complessa da manipolare in modo algebrico.

Per risolvere questa equazione, sarebbe necessario utilizzare un metodo numerico (come il metodo di Newton-Raphson) per trovare un valore approssimato di \(T_{R}\).

Nel punto in cui il contrasto si annulla, anche in presenza di un elevato rapporto segnale/rumore, i due tessuti non sono distinguibili nell'immagine.

\begin{figure}
\centering
\includegraphics[width=6.67569in,height=1.82431in]{media/11_SNR/image309.pdf}\caption{Figura .: Andamento del contrasto tra materia grigia e materia bianca, materia grigia e liquido spinale, e tra materia grigia sana e lesionata}
\end{figure}

\paragraph{Esempio di segnali provenienti dai tessuti cerebrali}\label{esempio-di-segnali-provenienti-dai-tessuti-cerebrali}

Si considera, a titolo d'esempio, la materia grigia (\emph{gray matter} o GM) e la materia bianca (\emph{white matter} o WM) contenute nel cervello. La materia grigia (\emph{gray matter} o GM), essendo maggiormente acquosa, presenta una densità protonica relativa maggiore rispetto alla materia bianca. I suoi parametri tipici sono una densità protonica relativa di circa \(0.8\) e un tempo di rilassamento longitudinale (\(T_{1}\)) di circa \(0.950\ s\). La materia bianca (\emph{white matter} o WM), invece, ha un minor contenuto di liquidi, per cui la sua densità protonica è inferiore (circa \(0.65\)) e il suo tempo di rilassamento longitudinale (\(T_{1}\)) è più breve (circa \(0.600\ s\)).

In una sequenza \(T_{1}\)-pesata, caratterizzata da un tempo di ripetizione \(T_{R}\) breve, il segnale di risonanza magnetica dipende principalmente dai tempi di rilassamento \(T_{1}\) dei tessuti. Poiché la materia bianca ha un \(T_{1}\) più breve rispetto alla materia grigia, la sua magnetizzazione longitudinale recupera più rapidamente e, per valori di \(T_{R}\) brevi, il segnale emesso risulta essere maggiore. In altre parole, la materia bianca appare più "brillante" (più intensa) della materia grigia nelle immagini \(T_{1}\)-pesate.

Esiste un valore specifico del tempo di ripetizione \(T_{R}\) in cui le curve di recupero della magnetizzazione longitudinale dei due tessuti si intersecano. A questo punto di crossover, i segnali emessi dalla materia grigia e dalla materia bianca hanno la stessa intensità. Di conseguenza, in tale condizione, il contrasto tra i due tessuti è nullo, poiché la loro differenza di segnale è pari a zero.

Se il tempo di ripetizione (\(T_{R}\)) supera il punto di crossover, le curve di segnale si invertono. La materia grigia, con la sua maggiore densità protonica, recupera sufficientemente la magnetizzazione per superare il segnale della materia bianca. Di conseguenza, il contrasto cambia segno e la materia grigia apparirà più ``brillante'' rispetto alla materia bianca. Questo fenomeno dimostra come, per \(T_{R}\) lunghi, il segnale diventi sempre più dipendente dalla densità protonica del tessuto piuttosto che dal suo tempo di rilassamento \(T_{1}\).

\begin{figure}
\centering
\includegraphics[width=4.96296in,height=3.88306in]{media/11_SNR/image310.pdf}\caption{Figura .: Segnale proveniente dalla materia grigia e materia bianca}
\end{figure}

\paragraph{Tempo di ripetizione ottimo per tessuti con proprietà simili}\label{tempo-di-ripetizione-ottimo-per-tessuti-con-proprietuxe0-simili}

Si suppone che i due tessuti presentino una densità protonica simile (assunzione valida per molti tessuti cerebrali) e dei tempi di rilassamento longitudinale \(T_{1}\) che differiscono per una quantità \(\delta \ll 1\):

\[\left\{ \begin{matrix}
M_{0,A} \sim M_{0,B} \equiv M_{0} \\
T_{1,B} = T_{1,A}(1 + \delta)
\end{matrix} \right.\ \]

L'espressione del contrasto tra i due tessuti si esprime, in queste ipotesi, come:

\[C_{AB}\left( T_{R} \right) = s_{0,A} - s_{0,B} \simeq M_{0,A}\left( 1 - \exp\left( - \dfrac{T_{R}}{T_{1,A}} \right) \right) - M_{0,B}\left( 1 - \exp\left( - \dfrac{T_{R}}{T_{1,B}} \right) \right) \simeq M_{0}\left( 1 - \exp\left( - \dfrac{T_{R}}{T_{1,A}} \right) - \left( 1 - \exp\left( - \dfrac{T_{R}}{T_{1,A}(1 + \delta)} \right) \right) \right) = M_{0}\left( 1 - \exp\left( - \dfrac{T_{R}}{T_{1,A}} \right) - 1 + \exp\left( - \dfrac{T_{R}}{T_{1,A}(1 + \delta)} \right) \right)\]

Svolgendo le opportune somme, il contrasto è:

\[C_{AB}\left( T_{R} \right) \simeq M_{0}\left( \exp\left( - \dfrac{T_{R}}{T_{1,A}(1 + \delta)} \right) - \exp\left( - \dfrac{T_{R}}{T_{1,A}} \right) \right)\]

Si raccoglie \(\exp\left( - T_{R}/T_{1,A} \right)\), ottenendo:

\[C_{AB}\left( T_{R} \right) \simeq M_{0}\exp\left( - \dfrac{T_{R}}{T_{1,A}} \right)\left( \dfrac{\exp\left( - \dfrac{T_{R}}{T_{1,A}(1 + \delta)} \right)}{\exp\left( - \dfrac{T_{R}}{T_{1,A}} \right)} - 1 \right) = M_{0}\exp\left( - \dfrac{T_{R}}{T_{1,A}} \right)\left( \exp\left( - \dfrac{T_{R}}{T_{1,A}(1 + \delta)} + \dfrac{T_{R}}{T_{1,A}} \right) - 1 \right)\]

Si considera l'argomento dell'esponenziale e si esegue il minimo comune multiplo:

\[- \dfrac{T_{R}}{T_{1,A}(1 + \delta)} + \dfrac{T_{R}}{T_{1,A}} = T_{R}\left( \dfrac{- 1 + 1 + \delta}{T_{1,A}(1 + \delta)} \right) = T_{R}\left( \dfrac{\delta}{T_{1,A}(1 + \delta)} \right)\]

Per cui il contrasto è:

\[C_{AB}\left( T_{R} \right) \simeq M_{0}\exp\left( - \dfrac{T_{R}}{T_{1,A}} \right)\left( \exp\left( T_{R}\left( \dfrac{\delta}{T_{1,A}(1 + \delta)} \right) \right) - 1 \right)\]

Dato che \(\delta \ll 1\), allora \(T_{R}\delta \ll T_{1,A}(1 + \delta)\). Per tale motivo è possibile sviluppare in serie di Taylor l'esponenziale nell'espressione del contrasto, arrestando al primo ordine:

\[\exp\left( T_{R}\left( \dfrac{\delta}{T_{1,A}(1 + \delta)} \right) \right) \simeq T_{R}\left( \dfrac{\delta}{T_{1,A}(1 + \delta)} \right)\]

Poiché \(\delta \ll 1\), allora;

\[T_{R}\left( \dfrac{\delta}{T_{1,A}(1 + \delta)} \right) \simeq 1 + \left( \dfrac{T_{R}}{T_{1,A}} \right)\delta\]

Il contrasto si scrive, quindi:

\[C_{AB}\left( T_{R} \right) \simeq M_{0}\exp\left( - \dfrac{T_{R}}{T_{1,A}} \right)\left( \exp\left( T_{R}\left( \dfrac{\delta}{T_{1,A}(1 + \delta)} \right) \right) - 1 \right) \simeq M_{0}\exp\left( - \dfrac{T_{R}}{T_{1,A}} \right)\left( 1 + \left( \dfrac{T_{R}}{T_{1,A}} \right)\delta - 1 \right)\]

Semplificando i termini \(\pm 1\), si ottiene:

\[C_{AB}\left( T_{R} \right) \simeq M_{0}\delta\left( \dfrac{T_{R}}{T_{1,A}} \right)\exp\left( - \dfrac{T_{R}}{T_{1,A}} \right)\]

Il contrasto tra due tessuti è una funzione del tempo di ripetizione \(T_{R}\). Si vuole terminare \(T_{R,opt}\) tale per cui il contrasto, in caso di tessuti con proprietà simili, sia massimo. Per determinare tale valore si impone che la derivata del contrasto deve essere nulla:

\[\dfrac{\partial C_{AB}}{\partial T_{R}} = 0 \Leftrightarrow M_{0}\delta\dfrac{\partial}{\partial T_{R}}\left( \left( \dfrac{T_{R}}{T_{1,A}} \right)\exp\left( - \dfrac{T_{R}}{T_{1,A}} \right) \right) = 0\]

Svolgendo le derivate e semplificando \(M_{0}\delta\), si ha:

\[\dfrac{1}{T_{1,A}}\exp\left( - \dfrac{T_{R,opt}}{T_{1,A}} \right) + \left( \dfrac{T_{R,opt}}{T_{1,A}} \right)\left( - \dfrac{1}{T_{1,A}} \right)\exp\left( - \dfrac{T_{R,opt}}{T_{1,A}} \right) = 0\]

Semplificando \(T_{1,A}^{- 1}\) ed \(\exp\left( - T_{R,opt}/T_{1,A} \right)\) si ottiene:

\[1 - \dfrac{T_{R,opt}}{T_{1,A}} = 0\]

Per cui:

\[T_{R,opt} = T_{1,A}\]

Per i tessuti con densità protonica \(\rho_{0}\) simile e tempi di rilassamento prossimi tra loro, il valore ottimo del tempo di ripetizione tra due sequenze \(T_{R}\) per massimizzare il contrasto di un'immagine \(T_{1}\) pesata è uguale al tempo di rilassamento longitudinale minore tra i due tessuto.

\subsubsection[Pesatura in T2*]{Pesatura in $\mathbf{T}_{\mathbf{2}}^{\mathbf{*}}$}
\label{pesatura-in-T2}

Le immagini di risonanza magnetica possono mostrare un tessuto con maggiore contrasto rispetto a un altro sulla base dei tempi di rilassamento trasversali \(T_{2}\), caratteristici dei diversi tessuti.

Poiché i valori dei tempi di rilassamento trasversale \(T_{2}\) sono in genere dell'ordine delle decine di millisecondi, mentre quelli del tempo di rilassamento longitudinale \(T_{1}\) sono dell'ordine del secondo, una variazione del tempo \(T_{2}\) produce un effetto più marcato sull'intensità del segnale rispetto a una variazione equivalente di \(T_{1}\).Per questo motivo, un'alterazione del tempo \(T_{2}\) risulta più facilmente correlabile ad alcune patologie.

La pesatura in \(T_{2}\) può essere ottenuta usando una sequenza spin-echo. Tuttavia, quando i tessuti rispondono in modo inatteso al campo magnetico applicato a causa di perturbazioni, si ottengono immagini pesate in \(T_{2}^{*}\). In particolare, se le variazioni del campo avvengono in modo sufficientemente rapido tra i vari voxel, si verifica un'ulteriore perdita di segnale anche quando si usa la sequenza spin-echo. Per questo motivo, la pesatura in \(T_{2}^{*}\) è particolarmente utile per lo studio dell'attività cerebrale, dove tali effetti sono rilevanti.

Al fine di ottenere una pesatura in \(T_{2}^{*}\), la sequenza gradient-echo applicata deve essere disegnata in modo che il tempo di ripetizione sia tale da non avvertire gli effetti del rilassamento trasversale, ovvero:

\[T_{R} \gg T_{1}\]

Al limite, \(T_{R} \rightarrow \infty\). In questa condizione, il termine \(\exp\left( - T_{R}/T_{1} \right)\) nella relazione del segnale del voxel è trascurabile:

\[s_{0} \propto M_{0}\left( 1 - \exp\left( - \dfrac{T_{R}}{T_{1}} \right) \right)\exp\left( - \dfrac{T_{E}}{T_{2}} \right) \simeq M_{0}\exp\left( - \dfrac{T_{E}}{T_{2}^{*}} \right)\]

Con questa scelta il contrasto tra due tessuti può essere scritto:

\[C_{AB} = s_{0,A} - s_{0,B} \simeq M_{0,A}\exp\left( - \dfrac{T_{E}}{T_{2,A}^{*}} \right) - M_{0,B}\exp\left( - \dfrac{T_{E}}{T_{2,B}^{*}} \right)\]

Con l'ipotesi sul tempo di ripetizione \(T_{R} \gg T_{1}\), il contrasto è una funzione del tempo di echo, \(T_{E}\), intorno al quale il segnale registrato presenta il massimo valore. Il contrasto è quindi legato all'area deli gradienti della sequenza gradient-echo sull'asse di lettura al fine di formare al tempo desiderato l'echo.

La sequenza gradient-echo non produce una pesatura in \(T_{2}\) perché non usa l'impulso a \(\pi\) al fine di correggere le disomogeneità del campo. Pertanto, le immagini ottenute sono intrinsecamente pesate in \(T_{2}^{*}\). Utilizzando una sequenza spin-echo si ottengono delle relazioni simili, dipendenti dal tempo \(T_{2}\).

Per una spin-echo, noto il valore del tempo \(T_{2}\) è possibile diagrammare l'andamento del contrasto \(C_{AB}\) in funzione del tempo d'echo.

Si considera, come ad esempio, il contrasto tra materia bianca (\(\rho_{0,WM} = 0.8\), \(T_{2.WM} = 0.1\ s\)) e il liquido cerebrospinale (\(\rho_{0,CSF} = 1\), \(T_{2.WM} = 2\ s\)).

Poiché la materia bianca, essendo più solida, presenta un tempo di rilassamento trasversale molto più breve, il contrasto WM--CSF risulta massimo per tempi di echo \(T_{E}\) superiori al tempo di rilassamento trasversale della materia bianca.

\begin{figure}
\centering
\includegraphics[width=4.27564in,height=2.96504in]{media/11_SNR/image311.pdf}\caption{Figura .: Contrasto pesato \(T_{2}\) tra liquido cerebrospinale e materia bianca}
\end{figure}

Nel caso della \textbf{materia grigia} (\(\rho_{0} = 0.65\), \(T_{2} = 0.08\, s\)) rispetto alla materia bianca, le differenze nei tempi \(T_{2}\) sono minime. In questo caso, la pesatura in \(T_{2}\) non garantisce un buon contrasto, mentre il miglior contrasto WM--GM si ottiene a bassi valori del tempo di echo, in condizioni vicine alla pesatura in \textbf{densità protonica}.

\begin{figure}
\centering
\includegraphics[width=4.71154in,height=3.20426in]{media/11_SNR/image312.pdf}\caption{Figura .: Contrasto tra materia grigia e bianca}
\end{figure}

Da questo esempio si evince che il contrasto dipende dal tempo \(T_{2}\) caratteristico di un tessuto e il contrasto ottimo tra due tessuti non è detto che sia ottenibile nel limite delle immagini \(T_{2}\)-pesate.

\paragraph{Tempo di echo ottimo}\label{tempo-di-echo-ottimo}

Per le immagini \(T_{2}^{*}\)-pesate, ottenute mediante una sequenza gradient-echo, il contrasto massimo tra due tessuti lo si ottiene quando la derivata della funzione contrasto \(C_{AB}\left( T_{E} \right)\) è nulla:

\[\dfrac{\partial C_{AB}}{\partial T_{E}} = 0 \Leftrightarrow \dfrac{\partial}{\partial T_{E}}\left( M_{0,A}\exp\left( - \dfrac{T_{E}}{T_{2,A}^{*}} \right) - M_{0,B}\exp\left( - \dfrac{T_{E}}{T_{2,B}^{*}} \right) \right) = 0\]

Si esegue il calcolo della derivata:

\[- \dfrac{M_{0,A}}{T_{2,A}^{*}}\exp\left( - \dfrac{T_{E,opt}}{T_{2,A}^{*}} \right) + \dfrac{M_{0,B}}{T_{2,B}^{*}}\exp\left( - \dfrac{T_{E,opt}}{T_{2,B}^{*}} \right) = 0\]

Si portano i termini dipendenti dal tessuto \(B\) al secondo membro;

\[\dfrac{M_{0,A}}{T_{2,A}^{*}}\exp\left( - \dfrac{T_{E,opt}}{T_{2,A}^{*}} \right) = \dfrac{M_{0,B}}{T_{2,B}^{*}}\exp\left( - \dfrac{T_{E,opt}}{T_{2,B}^{*}} \right)\]

Si divide ambo i membri per \(\exp\left( - T_{E,opt}/T_{2,B}^{*} \right)\) al fine di portare la dipendenza da \(T_{E,opt}\) solo al primo membro:

\[\dfrac{M_{0,A}}{T_{2,A}^{*}}\exp\left( - \dfrac{T_{E,opt}}{T_{2,A}^{*}} \right)\exp\left( \dfrac{T_{E,opt}}{T_{2,B}^{*}} \right) = \dfrac{M_{0,B}}{T_{2,B}^{*}}\]

Si isolano i termini dipendenti da \(T_{E,opt}\):

\[\exp\left( - \dfrac{T_{E,opt}}{T_{2,A}^{*}} \right)\exp\left( \dfrac{T_{E,opt}}{T_{2,B}^{*}} \right) = \dfrac{M_{0,B}}{T_{2,B}^{*}}\dfrac{T_{2,A}^{*}}{M_{0,A}}\]

Per le proprietà degli esponenziali, risulta:

\[\exp\left( T_{E,opt}\left( \dfrac{1}{T_{2,B}^{*}} - \dfrac{1}{T_{2,A}^{*}} \right) \right) = \dfrac{M_{0,B}}{T_{2,B}^{*}}\dfrac{T_{2,A}^{*}}{M_{0,A}}\]

Si applica il logaritmo ambo i membri dell'equazione al fine di isolare \(T_{E,opt}\):

\[T_{E,opt}\left( \dfrac{1}{T_{2,B}^{*}} - \dfrac{1}{T_{2,A}^{*}} \right) = \log\left( \dfrac{M_{0,B}}{T_{2,B}^{*}}\dfrac{T_{2,A}^{*}}{M_{0,A}} \right)\]

Per le proprietà dei logaritmi è possibile riscrivere il secondo membro come:

\[T_{E,opt}\left( \dfrac{1}{T_{2,B}^{*}} - \dfrac{1}{T_{2,A}^{*}} \right) = \log\left( \dfrac{M_{0,B}}{T_{2,B}^{*}} \right) - \log\left( \dfrac{M_{0,A}}{T_{2,A}^{*}} \right)\]

Isolando \(T_{E,opt}\) si ottiene il tempo di echo tale da massimizzare il contrasto tra due tessuti:

\[T_{E,opt} = \dfrac{\log\left( \dfrac{M_{0,B}}{T_{2,B}^{*}} \right) - \log\left( \dfrac{M_{0,A}}{T_{2,A}^{*}} \right)}{\dfrac{1}{T_{2,B}^{*}} - \dfrac{1}{T_{2,A}^{*}}}\]

Questa relazione è valida sia per la sequenza gradient-echo sia per la spin-echo, a parte di sostituire \(T_{2,A}^{*}\) con \(T_{2,A}\) e \(T_{2,B}^{*}\) con \(T_{2,B}\).

Scegliendo un tempo di echo uguale a \(T_{E,opt}\) si ottiene il massimo contrasto tra due tessuti di cui sono note le proprietà biochimiche, quantificate dal tempo di rilassamento trasversale \(T_{2}\).

\paragraph{Tempo di echo ottimo per tessuti con proprietà simili}\label{tempo-di-echo-ottimo-per-tessuti-con-proprietuxe0-simili}

Si suppone che le densità protoniche di due tessuti \(A\) e \(B\) siano molto simili tra loro:

\[\rho_{0,A} \sim \rho_{0,B}\]

Siccome la densità protonica è legata alla magnetizzazione da una relazione lineare, allora le magnetizzazioni dei due tessuto sono simili:

\[M_{0,A} \sim M_{0,B} \equiv M_{0}\]

Inoltre, si suppone che i due tessuti presentano un tempo di rilassamento trasversale simile, che differiscono di una quantità \(\delta \ll 1\)

\[T_{2,B}^{*} = T_{2,A}^{*}(1 + \delta)\]

Il contrasto tra questi due tessuti, si esprime come:

\[C_{AB} = s_{0,A} - s_{0,B} \simeq M_{0,A}\exp\left( - \dfrac{T_{E}}{T_{2,A}^{*}} \right) - M_{0,B}\exp\left( - \dfrac{T_{E}}{T_{2,B}^{*}} \right) \simeq M_{0}\exp\left( - \dfrac{T_{E}}{T_{2,A}^{*}} \right) - M_{0}\exp\left( - \dfrac{T_{E}}{T_{2,A}^{*}(1 + \delta)} \right)\]

Si raccoglie il termine \(M_{0}\exp\left( - T_{E}/T_{2,A}^{*} \right)\):

\[C_{AB} \simeq M_{0}\exp\left( - \dfrac{T_{E}}{T_{2,A}^{*}} \right)\left( 1 - \dfrac{\exp\left( - \dfrac{T_{E}}{T_{2,A}^{*}(1 + \delta)} \right)}{\exp\left( - \dfrac{T_{E}}{T_{2,A}^{*}} \right)} \right) = M_{0}\exp\left( - \dfrac{T_{E}}{T_{2,A}^{*}} \right)\left( 1 - \exp\left( - \dfrac{T_{E}}{T_{2,A}^{*}(1 + \delta)} + \dfrac{T_{E}}{T_{2,A}^{*}} \right) \right)\]

Si esegue il minimo comune multiplo nell'argomento dell'esponenziale:

\[- \dfrac{T_{E}}{T_{2,A}^{*}(1 + \delta)} + \dfrac{T_{E}}{T_{2,A}^{*}} = T_{E}\dfrac{1 + \delta - 1}{T_{2,A}^{*}(1 + \delta)} = \dfrac{\delta T_{E}}{T_{2,A}^{*}(1 + \delta)}\]

Si ottiene:

\[C_{AB} \simeq M_{0}\exp\left( - \dfrac{T_{E}}{T_{2,A}^{*}} \right)\left( 1 - \exp\left( \dfrac{\delta T_{E}}{T_{2,A}^{*}(1 + \delta)} \right) \right)\]

Siccome \(\delta \ll 1\), è possibile sviluppare in serie di Taylor il termine esponenziale presente nell'espressione del contrasto:

\[\exp\left( \dfrac{\delta T_{E}}{T_{2,A}^{*}(1 + \delta)} \right) \simeq 1 + \dfrac{\delta T_{E}}{T_{2,A}^{*}(1 + \delta)}\]

Da cui si ottiene:

\[C_{AB} \simeq M_{0}\exp\left( - \dfrac{T_{E}}{T_{2,A}^{*}} \right)\left( 1 - 1 - \dfrac{\delta T_{E}}{T_{2,A}^{*}(1 + \delta)} \right) = - M_{0}\dfrac{\delta T_{E}}{T_{2,A}^{*}(1 + \delta)}\exp\left( - \dfrac{T_{E}}{T_{2,A}^{*}} \right)\]

Il segno negativo indica semplicemente che il tessuto \(B\) appare \textbf{più brillante} del tessuto \(A\) nella tecnica di MRI, poiché possiede un tempo \(T_{2}^{*}\) maggiore.

Nelle immagini MRI, di solito si visualizza il modulo del segnale, quindi, il contrasto è sempre visualizzato come un valore positivo. Tuttavia, nel calcolo matematico, il segno è importante e ha il significato fisico di maggiore intensità del tessuto \(B\).

Il contrasto massimo tra i due tessuti con caratteristiche simili tra loro è ottenuto derivando rispetto al tempo di echo \(T_{E}\) la relazione approssimata appena individuata e ponendo il risultato uguale a \(0\):

\[\dfrac{\partial C_{AB}}{\partial T_{E}} = 0 \Leftrightarrow \dfrac{\partial}{\partial T_{E}}\left( M_{0}\dfrac{\delta T_{E}}{T_{2,A}^{*}(1 + \delta)}\exp\left( - \dfrac{T_{E}}{T_{2,A}^{*}} \right) \right) = 0\]

Si svolge la derivata:

\[M_{0}\dfrac{\delta}{T_{2,A}^{*}(1 + \delta)}\exp\left( - \dfrac{T_{E,opt}}{T_{2,A}^{*}} \right) + M_{0}\dfrac{\delta T_{E,opt}}{T_{2,A}^{*}(1 + \delta)}\left( - \dfrac{1}{T_{2,A}^{*}} \right)\exp\left( - \dfrac{T_{E,opt}}{T_{2,A}^{*}} \right) = 0\]

Si mettono in evidenza le quantità in comune tra i due termini (\(M_{0}\), \(\delta\), \(1 + \delta\) e \(T_{2,A}^{*}\)) e gli esponenziali:

\[M_{0}\dfrac{\delta}{T_{2,A}^{*}(1 + \delta)}\exp\left( - \dfrac{T_{E,opt}}{T_{2,A}^{*}} \right)\left( 1 - \dfrac{T_{E,opt}}{T_{2,A}^{*}} \right) = 0\]

L'equazione è soddisfatta se risulta:

\[1 - \dfrac{T_{E,opt}}{T_{2,A}^{*}} = 0 \Leftrightarrow T_{E,opt} = T_{2,A}^{*}\]

Nel caso della sequenza spin-echo si ottiene:

\[T_{E,opt} = T_{2,A}\]

Quando due tessuti presentano caratteristiche biochimiche simili, il tempo di echo da selezionare affinché il contrasto sia massimo deve essere uguale al tempo di rilassamento trasversale minore. Da esperimenti empirici sono noti i tempi di rilassamento dei vari tessuti umani. Grazie a queste informazioni è possibile ottenere il massimo contrasto al fine di visualizzare al meglio i due tessuti.

Affinché i tessuti risultino ben visibili nell'immagine, oltre ad avere un contrasto massimo, è necessario che il rapporto contrasto/rumore sia sufficientemente elevato. Ciò equivale ad acquisire il segnale nei voxel con un soddisfacente rapporto segnale/rumore.

\subsubsection[Pesatura in T1 mediante inversion recovery]{Pesatura in $\mathbf{T}_{\mathbf{1}}$ mediante inversion recovery}
\label{pesatura-in-T1-mediante-inversion-recovery}

La sequenza di \emph{inversion recovery} permette di ottenere un contrasto dipendente dal tempo di rilassamento longitudinale \(T_{1}\)e di sopprimere il segnale di tessuti specifici.

Questa sequenza si compone di un impulso a radiofrequenza a \(\pi\) che ribalta la magnetizzazione longitudinale, che all'equilibrio termodinamico è \(M_{0}\). Dopo un tempo di inversione, indicato con \(T_{I}\), si applica una sequenza del tipo spin-echo, caratterizzata da un impulso di eccitazione a \(\pi/2\) e uno di rifasamento a \(\pi\), dopo il quale, all'istante \(T_{E}\), si forma l'echo. L'impulso a \(\pi/2\) viene applicato dopo il tempo di inversione \(T_{I}\) per convertire la magnetizzazione longitudinale residua in segnale trasversale.

\begin{figure}
\centering
\includegraphics[width=6.69167in,height=5.225in]{media/11_SNR/image313.pdf}\caption{Tabella 11.5: Sequenza inversion recovery con spin-echo al fine di ottenere immagini \(T_{1}\)-pesate}
\end{figure}

In questo contesto, il segnale registrato si esprime come:

\[s\left( T_{R},T_{I},T_{E} \right) = M_{0}\left( 1 - 2\exp\left( - \dfrac{T_{I}}{T_{1}} \right) + \exp\left( - \dfrac{T_{R}}{T_{1}} \right) \right)\exp\left( - \dfrac{T_{E}}{T_{2}} \right)\]

\(T_{R}\) è il tempo che intercorre tra l'inizio di una sequenza (impulso a\(\pi\)) e l'inizio della successiva. Il segnale a radiofrequenza a \(\pi\) è un impulso di \emph{inversione}, non di eccitazione, che è invece il ruolo dell'impulso a \(\pi/2\). Il termine dipendente da \(T_{R}/T_{1}\) è legato al ritorno all'equilibrio della magnetizzazione longitudinale con costante di tempo uguale a \(T_{1}\).

Se i tessuti hanno diverso tempo di rilassamento longitudinale, deve esistere un istante temporale in cui la differenza tra i segnali è massima. In questa condizione il contrasto assume il suo valore massimo.

\begin{figure}
\centering
\includegraphics[width=6.68333in,height=4.3in]{media/11_SNR/image314.pdf}\caption{Figura .: Massima differenza dai segnali ricevuti da materia grigia e materia bianca}
\end{figure}

Il contrasto tra due tessuti con densità protonica e tempi di rilassamento longitudinale diversi tra loro è:

\[C_{AB} = M_{0,A}\left( 1 - 2\exp\left( - \dfrac{T_{I}}{T_{1,A}} \right) + \exp\left( - \dfrac{T_{R}}{T_{1,A}} \right) \right)\exp\left( - \dfrac{T_{E}}{T_{2,A}} \right) - M_{0,B}\left( 1 - 2\exp\left( - \dfrac{T_{I}}{T_{1,B}} \right) + \exp\left( - \dfrac{T_{R}}{T_{1,B}} \right) \right)\exp\left( - \dfrac{T_{E}}{T_{2,B}} \right)\]

Al fine di avere una pesatura in \(T_{1}\) con una sequenza del tipo \emph{inversion recovery}, il tempo di ripetizione deve essere molto maggiore del tempo di rilassamento longitudinale \(T_{1}\). Il tempo di echo, inoltre, deve essere tale da non avvertire gli effetti del rilassamento trasversale:

\[\left\{ \begin{matrix}
T_{E} \ll T_{2} \\
T_{R} \gg T_{1}
\end{matrix} \right.\ \]

Con queste ipotesi è possibile trascurare il termine esponenziale \(\exp\left( - T_{R}/T_{1} \right)\), in quanto molto minore dell'unità;

\[\exp\left( - \dfrac{T_{R}}{T_{1}} \right) \ll 1\]

Inoltre, per l'ipotesi \(T_{E} \ll T_{2}\), il termine esponenziale \(\exp\left( T_{E}/T_{2} \right)\) tende all'unità:

\[\exp\left( \dfrac{T_{E}}{T_{2}} \right) \simeq 1\]

Il contrasto, con queste approssimazioni, può essere scritto come:

\[C_{AB} = M_{0,A}\left( 1 - 2\exp\left( - \dfrac{T_{I}}{T_{1,A}} \right) + \exp\left( - \dfrac{T_{R}}{T_{1,A}} \right) \right)\exp\left( - \dfrac{T_{E}}{T_{2,A}} \right) - M_{0,B}\left( 1 - 2\exp\left( - \dfrac{T_{I}}{T_{1,B}} \right) + \exp\left( - \dfrac{T_{R}}{T_{1,B}} \right) \right)\exp\left( - \dfrac{T_{E}}{T_{2,B}} \right) \simeq M_{0,A}\left( 1 - 2\exp\left( - \dfrac{T_{I}}{T_{1,A}} \right) \right) - M_{0,B}\left( 1 - 2\exp\left( - \dfrac{T_{I}}{T_{1,B}} \right) \right)\]

Al fine di identificare il tempo di inversione ottimo che massimizza il contrasto, si applica l'operazione di derivata rispetto a \(T_{I}\) e si pone il risultato uguale a \(0\):

\[\dfrac{\partial C_{AB}}{\partial T_{I}} = 0 \Leftrightarrow \dfrac{\partial}{\partial T_{I}}\left( M_{0,A}\left( 1 - 2\exp\left( - \dfrac{T_{I}}{T_{1,A}} \right) \right) - M_{0,B}\left( 1 - 2\exp\left( - \dfrac{T_{I}}{T_{1,B}} \right) \right) \right) = 0\]

Svolgendo l'operazione di derivazione, si ottiene:

\[2M_{0,A}\left( - \dfrac{1}{T_{1,A}} \right)\exp\left( - \dfrac{T_{I,opt}}{T_{1,A}} \right) + 2M_{0,B}\dfrac{1}{T_{1,B}}\exp\left( - \dfrac{T_{I,opt}}{T_{1,B}} \right) = 0\]

Si isolano i termini \(T_{I,opt}\):

\[2M_{0,A}\dfrac{1}{T_{1,A}}\exp\left( - \dfrac{T_{I,opt}}{T_{1,A}} \right) = 2M_{0,B}\dfrac{1}{T_{1,B}}\exp\left( - \dfrac{T_{I,opt}}{T_{1,B}} \right)\]

Si divide per \(\exp\left( T_{I,opt}/T_{1,B} \right)\) ambo i membri:

\[2M_{0,A}\dfrac{1}{T_{1,A}}\exp\left( - \dfrac{T_{I,opt}}{T_{1,A}} \right)\exp\left( \dfrac{T_{I,opt}}{T_{1,B}} \right) = 2M_{0,B}\dfrac{1}{T_{1,B}}\]

Si isola il tempo \(T_{I}\) dividendo per il reciproco del termine moltiplicativo il primo membro. Inoltre, per le proprietà degli esponenziali si ottiene:

\[\exp\left( - \dfrac{T_{I,opt}}{T_{1,A}} + \dfrac{T_{I,opt}}{T_{1,B}} \right) = \dfrac{M_{0,B}}{T_{1,B}}\dfrac{T_{1,A}}{M_{0,A}}\]

Si applica il logaritmo ambo i membri, si ha:

\[T_{I,opt}\left( \dfrac{1}{T_{1,B}} - \dfrac{1}{T_{1,A}} \right) = \log\left( \dfrac{M_{0,B}}{T_{1,B}}\dfrac{T_{1,A}}{M_{0,A}} \right)\]

Applicando le proprietà dei logaritmi e isolando il termine \(T_{I,opt}\), si ha:

\[T_{I,opt} = \dfrac{\log\left( \dfrac{M_{0,B}}{T_{1,B}} \right) - \log\left( \dfrac{M_{0,A}}{T_{1,A}} \right)}{\dfrac{1}{T_{1,B}} - \dfrac{1}{T_{1,A}}}\]

Il risultato ottenuto coincide con il tempo di ripetizione ottimo nella pesatura per \(T_{1}\) con sequenza gradient-echo.

In conclusione, a parità di tempo \(T_{R}\) per una sequenza spin-echo o gradient-echo, l'inversion recovery presenta un segnale circa doppio rispetto alle altre due citate. Se su pesano due tessuti in \(T_{1}\) con una sequenza di inversion recovery si ottiene un contrasto doppio a parità di altri parametri.

La sequenza di \emph{inversion recovery} permette di annullare il segnale proveniente da un tessuto \(A\), noto che sia il suo tempo di rilassamento longitudinale \(T_{1}\). È possibile scegliere di applicare l'impulso a \(\pi/2\) in corrispondenza dell'istante temporale in cui il segnale proveniente dal tessuto di interesse è nullo, poiché nulla la sua magnetizzazione longitudinale. Viceversa, il segnale proveniente dal tessuto \(B\) è diverso da zero poiché, avendo un tempo di rilassamento longitudinale diverso, allo stesso istante temporale presenta una magnetizzazione longitudinale non nulla da ribaltare mediante l'impulso a \(\pi/2\).

\begin{figure}
\centering
\includegraphics[width=6.68333in,height=4.3in]{media/11_SNR/image315.pdf}\caption{Figura .: Punto in cui il segnale del tessuto \(A\) (materia bianca) si annulla}
\end{figure}

Questa tecnica è spesso molto applicata per la rimozione del grasso. In mammografia, ad esempio, si sopprime il grasso al fine di visualizzare al meglio il tessuto mammario, mediante un'immagine \(T_{1}\)-pesata. L'imaging, ovviamente, interessa solamente i tessuti non eliminati dall'inversion recovery.

Per poter applicare la metodica di cancellazione del segnale proveniente da un tessuto è necessario avere una conoscenza a priori dei tempi di rilassamento dei tessuti di interesse. Le lesioni tumorali, ad esempio, risultano enfatizzate mediante immagini \(T_{2}\)-pesate con soppressione del grasso con mezzo di contrasto.

\subsection{Mezzo di contrasto in risonanza magnetica}\label{mezzo-di-contrasto-in-risonanza-magnetica}

Le lesioni tumorali, se sviluppate, presentano un loro microambiente, definito stroma reattivo, caratterizzato da propri tempi di rilassamento \(T_{1}\) e \(T_{2}\). Solitamente, per una buona visualizzazione del tumore si utilizza una sequenza con pesatura in \(T_{2}\).

Lo studio dei vari tessuti passa, quindi, per l'esperienza del radiologo che, in base alla struttura anatomica da visualizzare, seleziona i giusti parametri della sequenza più opportuna, sulla base delle conoscenze e studi pregressi.

Esistono casi in cui il tessuto che si vuole visualizzare non è sufficientemente in contrasto rispetto i tessuti che lo circondano. In questa evenienza si ricorre a un imaging \(T_{1}\)-pesatp con l'uso di un mezzo di contrasto che altera localmente le iterazioni magnetiche degli spin, modificando i tempi di rilassamento. In altre parole, il mezzo di contrasto introduce delle disomogeneità di campo magnetico che alterano le iterazioni tre gli spin nel tessuto, in modo da ridurre il tempo di rilassamento longitudinale di una quantità dipendente dalla sua concentrazione,

Eseguendo un imaging \(T_{1}\)-pesato è possibile risalire alla distribuzione del tracciante nel corpo, evidenziando così i tessuti con maggior concentrazione di mezzo di contrasto.

In pratica, la concentrazione \(c\) del liquido tracciante è misurata in \(mmol/L\), dove una mole è un numero di Avogadro di particelle. Quindi, iniettando un mezzo di contrasto con una certa molarità si può dimostrare che la rilassività longitudinale aumenta proporzionalmente, secondo la relazione:

\[R_{1}(c) = R_{1.0} + \alpha_{1}c\]

Con \(\alpha_{1}\) parametro dipendente dal particolare mezzo di contrasto scelto. Per definizione, \(R_{1}\) è l'inverso del tempo di rilassamento longitudinale:

\[R_{1} = \dfrac{1}{T_{1}}\]

La relazione per la rilassività può essere scritta in termini di \(T_{1}\):

\[\dfrac{1}{T_{1}(c)} = \dfrac{1}{T_{1,0}} + \alpha_{1}c\]

Dove \(T_{1}(c)\) è il tempo di rilassamento longitudinale del tessuto, dopo l'introduzione del mezzo di contrasto, e \(T_{1,0}\) il tempo prima dell'iniezione, caratteristico del tessuto,

La maggior parte dei mezzi di contrasto per risonanza magmatica sono basati sull'elemento chimico gadolinio (\(Gd\)). Questa sostanza, essendo tossica per l'organismo, non è inserita all'interno del corpo pura ma legata a delle macromolecole, come l'albumina, che eliminano l'effetto tossico. La molecola più utilizzata nella pratica clinica è detta DOTA (sale megluminico dell'acido gadoterico).

Il gadolinio presenta un comportamento paramagnetico, quindi, altera localmente il campo magnetico visto dagli spin. Ciò provoca un cambiamento del tempo di rilassamento longitudinale.

L'effetto del mezzo di contrasto si ripercuote anche sul tempo di rilassamento trasversale \(T_{2}\), secondo una relazione perfettamente analoga a quella del tempo di rilassamento longitudinale:

\[R_{2}(c) = R_{2,0} + \alpha_{2}c\]

Oppure, utilizzando i tempi di rilassamento:

\[\dfrac{1}{T_{2}(c)} = \dfrac{1}{T_{2,0}} + \alpha_{2}c\]

Dove \(T_{2,0}\) è il tempo di rilassamento trasversale caratteristico del tessuto, prima dell'introduzione del mezzo di contrasto con concentrazione \(c\).

Le due equazioni per \(R_{1}(c)\) e \(R_{2}(c)\) sono dette come relazioni di Solomon-Bloembergen (S-B). Questa teoria fornisce un quadro teorico dettagliato per spiegare e quantificare come gli ioni metallici paramagnetici (come quelli usati negli agenti di contrasto per la risonanza magnetica) influenzano i tassi di rilassamento nucleare dei nuclei circostanti, in particolare i protoni delle molecole d'acqua.

L'approssimazione lineare è valida in condizione di \emph{fast exchange limit}, il quale prevede che il tempo in cui una molecola di acqua (o un altro nucleo) interagisce con l'agente di contrasto paramagnetico è molto più breve rispetto ai tempi di rilassamento \(T_{1}\) e \(T_{2}\).

In queste condizioni:

\begin{itemize}
\item
  Le molecole d'acqua si muovono e cambiano la loro posizione tra la vicinanza dell'agente di contrasto e il resto del tessuto molto rapidamente;
\item
  L'interazione con l'agente di contrasto è di breve durata e le fluttuazioni del campo magnetico locale sono molto veloci.
\item
  I dettagli specifici dell'interazione (che sono il fulcro della teoria S-B), come il tempo di permanenza e la distanza esatta, diventano meno rilevanti.
\end{itemize}

Le costanti \(\alpha_{1}\) e \(\alpha_{2}\) sono dette, rispettivamente, rilassività o relaxivity longitudinale e trasversale del contrasto. Questi parametri dipendono direttamente dal materiale utilizzato come liquido di contrasto e sono riportati sul flaconcino del contenitore, La loro unità di misura è, nella pratica, espressa in:

\[\lbrack\alpha\rbrack = \left\lbrack \left( m\dfrac{mol}{L}s \right)^{- 1} \right\rbrack\]

Valori tipici di questi coefficienti sono di \(4 \div 5\ L/mmol\ s\). Per molti liquidi di contrasto, che tendono a ridurre il tempo di rilassamento longitudinale \(T_{1}\), i coefficienti di rilassività hanno generalmente la stessa ampiezza \(\alpha_{1} \sim \alpha_{2}\).

Inoltre, siccome \(T_{1} > T_{2} \Leftrightarrow R_{1} < R_{2}\), un aumento della concentrazione del liquido di contrasto produce un cambiamento maggiore nel tempo \(T_{1}\), quindi, il segnale ricevuto dovuto a questa riduzione ha un effetto maggiore di quella dovuta alla riduzione di \(T_{2}\). Quando il termine \(\alpha_{2}c\) diventa comparabile con la rilassività trasversale del tessuto \(R_{2,0}\), le perdite del segnale ricevuto dovute alla riduzione di \(T_{2}\) diventano importanti. In questa situazione si perdono i vantaggi del contrasto \(T_{1}\)-pesato con mezzo di contrasto.

Il punto di crossover tra le perdite di segnali dovuti alla riduzione di \(T_{2}\) e gli incrementi del segnale dovuti alla riduzione di \(T_{1}\) definisce il contrasto ottimo. Ciò permette di scegliere il valore di concentrazione del mezzo di contrasto da iniettare.

Per ottenere l'imaging con mezzo di contrasto, si acquisisce dapprima un'immagine della sezione anatomica di interesse priva del liquido di contrasto e, in seguito, si inietta il mezzo di contrasto e si esegue nuovamente l'imaging. La prima immagine permette di ottenere informazioni sui tempi di rilassamento caratteristici dei tessuti, \(T_{1,0}\) e \(T_{2,0}\).

L'introduzione del mezzo di contrasto modifica le caratteristiche di rilassamento del tessuto che si lega alla molecola biologica (come l'albumina) contenente il gadolinio. I tessuti non legati al mezzo di contrasto, invece, non modificano le proprie caratteristiche magnetiche in modo apprezzabile, per cui possiedono pressocché lo stesso segnale sia prima che dopo l'iniezione.

Eseguendo una semplice sottrazione tra l'immagine senza mezzo di contrasto e quella acquisita a valle dell'iniezione, si ottiene un'immagine contenente solamente le strutture legate al mezzo di contrasto, in cui le proprietà magnetiche sono modificate in modo apprezzabile, mentre le altre regioni sono rimosse.

In particolare, noto il tempo di rilassamento longitudinale caratteristico del tessuto, \(T_{1,0}\), e quello alterato dall'iniezione del mezzo di contrasto \(T_{1}\), è possibile ricavare la concentrazione locale del tracciante.

Questo procedimento permette di ottenere immagini funzionali ed è molto sfruttato nel campo dell'oncologia: i tumori, per accrescersi rapidamente, promuovono l'angiogenesi, ovvero la formazione di nuovi vasi sanguigni. Iniettando un mezzo di contrasto nel sangue, la zona che presenta una maggiore concentrazione del mezzo di contrasto è riccamente vascolarizzata, dunque, sede di una neoplasia.

Nella pratica, la dose del mezzo di contrasto è di \(0.1\ mmol/L\ kg\), dove l'unità di messa è riferito al quantitativo di sostanza del paziente. Ad esempio, per un paziente di \(50\ kg\)la dose di farmaco con concentrazione di:

\[0.1\dfrac{mmol}{L\ kg}50\ kg = 5mmol/L\]

Questa contrazione si distribuisce in tutto il corpo del paziente, soprattutto nelle zone che sfruttano la macromolecola contenente il gadolinio.

\subsection{Effetto di volume parziale}\label{effetto-di-volume-parziale}

Nella pratica, a causa delle dimensioni finite del volumetto elementare in cui è diviso il paziente, non è detto che un voxel contenga solamente un tessuto. Si suppone che il tessuto \(A\) sia più piccolo del voxel, il quale, di conseguenza, contiene due tipi di tessuti. Il tessuto \(A\), in altre parole, occupa sola parzialmente il volume del voxel.

\begin{figure}
\centering
\includegraphics[width=3.49167in,height=1.75664in]{media/11_SNR/image316.pdf}\caption{Figura .: Voxel contenente due tessuti diversi}
\end{figure}

Se gli effetti della sfocatura dei due tessuti a causa della PSF sono trascurabili, un semplice modello per descrivere questa situazione prevede di considerare il segnale ricevuto dal voxel come la somma dei segnali delle frazioni di tessuto contenuti nel volumetto elementare. Sia \(\alpha_{A}\) la frazione volumetrica del tessuto \(A\) nel voxel considerato:

\[\alpha_{A} = \dfrac{V_{A}}{V_{tot}}\]

Il segnale del voxel può essere scritto come:

\[s_{voxel,1} = \alpha_{A}s_{A} + \left( 1 - \alpha_{A} \right)\ s_{B}\]

In altre parole, il segnale risultante del voxel è una media pesata del segnale proveniente dai due tessuti. Questo primo modello, noto come \emph{Partial Volume Effect} (effetto del volume parziale).

Si suppone che il voxel adiacente a quello considerato contenga solamente il tessuto \(B\). Il contrasto tra i due voxel si esprime come:

\[C_{AB} = s_{voxel,2} - s_{voxel,1} = s_{B} - \left( \alpha_{A}s_{A} + \left( 1 - \alpha_{A} \right)\ s_{B} \right)\]

Si svolgono i prodotti:

\[C_{AB} = s_{B} - \alpha_{A}s_{A} - s_{B} + \alpha_{A}s_{B} = \alpha_{A}\left( s_{B} - s_{A} \right)\]

Il contrasto tra i due tessuti è una percentuale \(\alpha_{A}\) del contrasto che si avrebbe in assenza del riempimento parziale del voxel col tessuto \(A\).

\begin{figure}
\centering
\includegraphics[width=6.42556in,height=1.39179in,alt={Immagine che contiene testo, linea, schermata, Carattere Il contenuto generato dall\textquotesingle IA potrebbe non essere corretto.}]{media/11_SNR/image317.pdf}\caption{Figura .: Il voxel contenente entrambi i tessuti è posto vicino a voxel contenenti solo il tessuto \(B\)}
\end{figure}

Per migliorare il contrasto si riduce la dimensione di voxel si una quantità \(\beta < 1\).

\begin{figure}
\centering
\includegraphics[width=6.62593in,height=1.56272in,alt={Immagine che contiene testo, schermata, Carattere, linea Il contenuto generato dall\textquotesingle IA potrebbe non essere corretto.}]{media/11_SNR/image318.pdf}\caption{Figura .: Riduzione dei voxel di un fattore \(\beta\)}
\end{figure}

Con la riduzione del voxel, il tessuto \(A\) occupa una porzione volumetrica di:

\[\alpha_{A}' = \dfrac{V_{A}}{\beta V_{tot}} = \dfrac{\alpha_{A}}{\beta}\]

Affinché il modello sia valido, \(\alpha_{A}/\beta < 1\), ovvero \(\alpha_{A} < \beta\). Se \(\alpha_{A}'\) supera l'unita, il voxel ridotto è completamente riempito dal tessuto \(A\), e la formula diventa non applicabile.

La restante parte del voxel (\(1 - \alpha_{A}/\beta\)) è occupato dal tessuto \(B\). Il segnale nel voxel è, quindi, dato da:

\[s_{voxel,1}' = \dfrac{\alpha_{A}}{\beta}\left( s_{A}' \right) + \left( 1 - \dfrac{\alpha_{A}}{\beta} \right)\ \left( s_{B}' \right)\]

Dove \(s_{A}'\) e \(s_{B}'\) sono i segnali totali che si otterrebbe da un voxel interamente riempito, rispettivamente, dal tessuto \(A\) e \(B\).

Il segnale del voxel adiacente, avendo ridotto il suo volume, è scalato di un fattore \(\beta\) a causa della riduzione del voxel:

\[s_{voxel,2}' = \beta s_{B}\]

A causa della riduzione delle dimensioni del volume elementare, i segnali totali che si otterrebbero da un voxel interamente riempito da uno solo di quei tessuti, contenuti nel voxel con riempimento parziale del tessuto \(A\), sono ridotti di una quantità \(\beta\):

\[s_{A}' = \beta s_{A},s_{B}' = \beta s_{B}\]

Il segnale nel voxel considerato è:

\[s_{voxel,1}' = \dfrac{\alpha_{A}}{\beta}\left( \beta s_{A} \right) + \left( 1 - \dfrac{\alpha_{A}}{\beta} \right)\ \left( \beta s_{B} \right)\]

Il contrasto tra i due tessuti contenuti nei voxel adiacenti è, a valle della riduzione, è:

\[C_{AB} = s_{voxel,2}' - s_{voxel,1}' = \beta s_{B} - \left( \dfrac{\alpha_{A}}{\beta}\left( \beta s_{A} \right) + \left( 1 - \dfrac{\alpha_{A}}{\beta} \right)\ \left( \beta s_{B} \right) \right)\]

Svolendo i prodotti si ottiene:

\[C_{AB} = \ \beta s_{B} - \alpha_{A}s_{A} - \ \beta s_{B} + \alpha_{A}s_{B}\]

Eliminando i termini con segno opposto e raccogliendo \(\alpha_{A}\), si ottiene:

\[C_{AB} = \alpha_{A}\left( s_{B} - s_{A} \right)\]

Pur riducendo il volume del voxel di una quantità \(\beta < 1\), il contrasto tra i due tessuti non varia se la percentuale \(\alpha_{A}\) del tessuto \(A\) nel primo voxel non varia.

La visibilità dei tessuti in un'immagine non dipende solo dal contrasto, ma dal \textbf{rapporto contrasto/rumore (CNR)}, definito come il rapporto tra il contrasto e la deviazione standard del rumore (\(\sigma\)) distribuito sui voxel:

\[CNR = \dfrac{C_{AB}}{\sigma}\]

Sebbene nel modello di volume parziale il contrasto assoluto \(C_{AB}\) non cambi con la riduzione del voxel, il rapporto segnale/rumore (SNR) è direttamente influenzato. Il segnale totale raccolto da un voxel è proporzionale al suo volume. Il rumore, invece, è approssimativamente costante e non dipende dalla dimensione del voxel.

Pertanto, il rapporto segnale/rumore (SNR), che è proporzionale al segnale, è anche proporzionale al volume del voxel. Nel caso monodimensionale si ha:

\[SNR \propto \Delta x\sqrt{T_{S}}\]

La riduzione del Fourier pixel size di \(\beta\) (\(\Delta x' = \beta\Delta x\)), determina una variazione della durata della finestra di acquisizione, secondo le relazioni:

\[\left\{ \begin{matrix}
L_{x}' = N_{x}'\Delta x' \\
T_{S}' = N_{x}'\Delta t'
\end{matrix} \right.\ \]

Da cui:

\[T_{S}' = \dfrac{L_{x}'}{\Delta x'}\ \Delta t'\]

Mantenendo costante il FOV e l'intervallo di campionamento costante, la durata della finestra di acquisizione aumenta di un fattore \(\beta^{- 1}\):

\[T_{S}' = \dfrac{L_{x}'}{\Delta x'}\ \Delta t' = \dfrac{L_{x}}{\beta\Delta x}\Delta t = \dfrac{T_{S}}{\beta}\]

Con questa scelta il rapporto segnale/rumore viene ridotto di una quantità \(\sqrt{\beta}\):

\[SNR' \propto \Delta x'\sqrt{T_{S}'} = \ \beta\Delta x\sqrt{\dfrac{T_{S}}{\beta}} = \sqrt{\beta}\ \Delta x\sqrt{T_{S}}\]

Mantenendo costante la finestra di acquisizione, il numero dei campioni viene scalato di un fattore \(\beta^{- 1}\):

\[L_{x}' = N_{x}'\Delta x' \Leftrightarrow N_{x}' = \dfrac{L_{x}'}{\Delta x'} = \dfrac{L_{x}}{\beta\Delta x} = \dfrac{N_{x}}{\beta}\]

Con questa scelta il rapporto segnale/rumore viene ridotto di una quantità \(\sqrt{\beta}\):

\[SNR' \propto \Delta x'\sqrt{N_{x}'\Delta t'} = \ \beta\Delta x\sqrt{\dfrac{N_{x}}{\beta}\Delta t} = \sqrt{\beta}\ \Delta x\sqrt{N_{x}\Delta t}\]

Si suppone, ora, di mantenere costante il numero di campioni acquisiti. Il FOV deve ridursi di una quantità \(\beta\):

\[L_{x}' = N_{x}'\Delta x' = N_{x}\beta\Delta x = \beta L_{x}\]

La durata della finestra di acquisizione resta invariata, in quanto:

\[T_{S}' = \dfrac{L_{x}'}{\Delta x'}\ \Delta t' = \dfrac{\beta L_{x}}{\beta\Delta x}\Delta t = T_{S}\]

Con questa scelta il rapporto segnale/rumore viene ridotto di una quantità \(\beta\):

\[SNR' \propto \Delta x'\sqrt{N_{x}'\Delta t'} = \ \beta\Delta x\sqrt{T_{S}}\]

Il rumore con cui è acquisito il segnale del voxel si distribuisce su tutta l'immagine con uguale intensità:

\[\sigma_{0}^{2} = \dfrac{\sigma_{m}^{2}}{N_{x}}\]

Dove \(\sigma_{m}^{2}\) dipende dal sistema di acquisizione, dunque, non varia tra i vari esperimenti. Scalando il numero di campioni acquisiti per un fattore \(\beta^{- 1}\) si ottiene una riduzione della varianza del rumore che insiste sul voxel:

\[\sigma_{0}^{2} = \beta\dfrac{\sigma_{m}^{2}}{N_{x}}\]

Ne discende che il rapporto contrasto/rumore aumenta di un fattore \(\sqrt{\beta^{- 1}}\):

\[CNR = \dfrac{C_{AB}}{\sigma_{0}} = \dfrac{C_{AB}}{\sqrt{\beta^{- 1}}\sigma_{0}}\]

Con questa scelta di parametri è possibile risolvere tessuti più piccoli (tessuto \(A\)) immersi nel background (tessuto \(B\)).

La riduzione della risoluzione, ovvero la riduzione del Fourier pixel size di una quantità \(\beta\), porta a una riduzione del rapporto segnale/rumore; tuttavia, se questo parametro è sufficientemente elevato da poter subire un incremento, si migliora allo stesso tempo la visibilità dei tessuti anche in presenza di effetto di volume parziale.

\subsection{Region of intereset}\label{region-of-intereset}

Si suppone che il rumore sovrapposto all'immagine sia di tipo bianco, additivo e gaussiano, caratterizzato da una media nulla e varianza \(\sigma\). A causa delle non idealità del sistema di acquisizione, le immagini ricostruite sono generalmente complesse, per cui si preferisce visualizzare le solo immagini del modulo. Le immagini di fase sono utili per ridurre la disomogeneità di campo mediante uno \emph{shamming} attivo.

Nel processo di modulo, il rumore perde la proprietà di meda nulla poiché si perde l'alternanza casuale tra valori positivi e negativi.

\begin{figure}
\centering
\includegraphics[width=6.69306in,height=5.17222in,alt={Immagine che contiene testo, schermata, Carattere, Diagramma Il contenuto generato dall\textquotesingle IA potrebbe non essere corretto.}]{media/11_SNR/image319.pdf}\caption{Figura .: Rumore gaussiano bianco a media nulla e modulo del rumore}
\end{figure}

Inoltre, a valle dell'operazione di modulo, cadono l'ipotesi di rumore gaussiano. Si può dimostrare che la distribuzione del rumore a valle dell'operazione di modulo è:

\[f(z) = \dfrac{z}{\sigma^{2}}\exp\left( - \dfrac{z^{2}}{2\sigma^{2}} \right)\]

Dove \(Z\) è la variabile aleatoria, ottenuta come:

\[Z = \sqrt{X^{2} + Y^{2}}\]

Con \(X\) e \(Y\) variabili aleatorie di tipo gaussiane e media nulla e varianza \(\sigma^{2}\):

\[X,Y \sim N\left( 0,\sigma^{2} \right)\]

Questo risultato discende dal fatto che il modulo è la radice quadrata della somma dei quadrati della parte reale e di quella immaginaria dell'immagine. Il rumore, quindi, si distribuzione secondo il modello di Rice (o di Rayleigh se il segnale utile è nullo).

Sull'immagine, per misurare il rumore, si traccia una ROI (\emph{region of interest}) all'interno della quale, mediante software dedicati, si valuta il valor medio e la deviazione standard dei livelli di grigio relativi al distretto anatomico di interesse.

Per valutare il rumore si sceglie una ROI esterna all'immagine anatomica. In questa regione le informazioni contenute nella ROI sono legate solamente al rumore, in quanto il segnale dei tessuti è supposto essere nullo.

Note le informazioni relative alle due ROI è semplice ricavare le informazioni sul segnale e sul rumore, valutando così il rapporto segnale/rumore:

\[SNR = \dfrac{s}{\sigma}\]

Se nel backgroud si stima, ad esempio, \(\sigma = 3\), mentre il segnale nel tessuto è \(90\), il rapporto segnale/rumore valutato è:

\[SNR = \dfrac{s}{\sigma} = \dfrac{90}{3} = 30\]

\begin{center}
\vfill
    \chapter{Fat suppression}
    \label{blx:FatSupp\therefsection}
\vfill

\minitoc
\newpage
\end{center}
\justify

\subsection{Chimical shift del grasso}\label{chimical-shift-del-grasso}

Al fine di non visualizzare delle strutture anatomiche, che normalmente avrebbero un segnale paragonabile a quello del tessuto di interesse, si rende necessario l'applicazione di varie metodiche per la soppressione del grasso o dell'acqua. Ad esempio, i tessuti come i nervi ottici o il tessuto fibroghiandolare mammario sono ben visualizzati se si attua una soppressione del segnale proveniente dal grasso che circonda le strutture anatomiche di interesse.

Se non si ricorre a una sequenza di \emph{inversion recovery} seguita da una spin-echo, applicata nel momento in cui la magnetizzazione del tessuto da sopprimere è nulla, è possibile che le proprietà intrinseche delle macromolecole, che schermano il campo magnetico principale ai protoni, causano degli artefatti. Infatti, anche per un campo statico principale perfettamente uniforme, localmente a livello molecolare, sono presenti delle disomogeneità di campo dovute alla struttura della macromolecola biologica. In altre parole, un protone libero, ovvero appartenente a una molecola d'acqua, vede un campo magnetico principale diverso rispetto a un protone legato a una macromolecola. Ciò avviene perché gli elettroni orbitanti intorno ai nuclei degli atomi che compongono la molecola modificano localmente il campo magnetico.

Come esito finale, i protoni d'acqua risuonano a una frequenza di precessione di Lamour lievemente diversa da quella dei protoni legati alle macromolecole. Ad esempio, i protoni legati agli atomi di grasso presentano una frequenza di precessione inferiore di quella di un protone dell'acqua.

La differenza di precessione tra protoni legati a molecole d'acqua e legati al grasso è data da:

\[\Delta f_{fw} = f_{f} - f_{w} = - \sigma_{fw}\overline{\gamma}B_{0}\]

Dove il pedice \(f\) indica il grasso (\emph{fat}) con formula \(CH_{8}CH_{4}\), e \(w\) l'acqua (\emph{water}) con formula \(H_{2}O\).

La quantità \(\sigma_{fw}\) è un parametro positivo indicante lo shift chimico tra acqua e grasso, detto costante di \emph{shielding} o schermatura o shift chimico. In gergo si dice che i protoni sono immersi in un ambiente molecolare.

Ovviamente, i vari ambienti molecolari modificano il campo principale localmente in modo diverso, quindi, la costante di \emph{shielding} è caratteristica di ogni sostanza.

Sebbene l'equazione per lo shift di frequenza sia stata scritta per l'acqua e il grasso, essa ha validità generale a patto di cambiare il valore della costante di \emph{shielding} tra le molecole.

La maggior parte del grasso nel corpo umano presenta una costante di \emph{shielding} di:

\[\sigma_{fw} = 3.35\ ppm = 3.335 \cdot 10^{- 6}\]

Per un campo principale \(B_{0} = 1.5\ T\), ciò porta a uno shift in frequenza di:

\[\left| \Delta f_{fw} \right| = \left| - \sigma_{fw}\overline{\gamma}B_{0} \right| = 3.335 \cdot 10^{- 6} \cdot 42.6 \cdot 10^{6}\dfrac{Hz}{T}1.5\ T = 214\ Hz\]

La densità protonica dell'acqua può anche essere molto maggiore di quella del grasso in un tessuto sano, tuttavia, il segnale del voxel legato ai lipidi per una sequenza con intervalli di ripetizione di breve durata può essere molto maggiore di quello legato all'acqua, poiché quest'ultimo presenta un tempo di rilassamento longitudinale \(T_{1}\) maggiore.

Il problema dell'elevato segnale del grasso è amplificato quando l'antenna superficiale usata in risonanza è posta in prossimità del tessuto adiposo che riveste i tessuti. Inoltre, risuonando a frequenze diverse, l'adipe può introdurre una serie di artefatti nell'immagine finale. Gli artefatti da spostamento chimico, come i bordi scuri o luminosi visibili ai confini tra grasso e acqua lungo la direzione di lettura, sono un classico esempio di questo fenomeno.

L'artefatto appare come una doppia linea al confine tra tessuti a base d'acqua (come muscoli o organi) e tessuti a base di grasso:

\begin{itemize}
\item
  Banda di Bordo Scuro: A causa della loro frequenza leggermente inferiore, i segnali dei protoni del grasso vengono codificati in una posizione leggermente diversa rispetto ai protoni dell'acqua. Questo causa un'interruzione del segnale in corrispondenza del confine tra i due tessuti, creando una banda scura.
\item
  Banda di Bordo Luminoso: Allo stesso tempo, i segnali dei protoni del grasso e dell'acqua si sovrappongono in un'altra area del confine, producendo una banda luminosa.
\end{itemize}

\begin{figure}
\centering
\includegraphics[width=3.56061in,height=3.56061in]{media/12_FatSupp/image320.pdf}\caption{Figura .: Artefatto dovuto al grasso}
\end{figure}

Si applica una sequenza FID a un campione contenente grasso e acqua, caratterizzata da un impulso a \(\pi/2\) che ribalta la magnetizzazione sul piano trasverso. Si registra, poi, il segnale emesso dal ritorno all'equilibrio emesso dal voxel.

\begin{figure}
\centering
\includegraphics[width=6.69167in,height=5.16667in]{media/12_FatSupp/image321.pdf}\caption{Figura .: Sequenza FID e segnale registrato}
\end{figure}

La frequenza di risonanza dell'acqua, nel sistema rotante, è maggiore di quella del grasso di un fattore \(\Delta f_{fw}\). Questo shift in frequenza rispetto alla sequenza di Lamour genera un problema nell'immagine ricostruita se l'ampiezza della banda di accettazione del voxel è maggiore di \(f_{fw}\). Infatti, il segnale di risonanza sarà caratterizzato da due picchi, uno di ampiezza maggiore legato alla concentrazione di acqua e centrato in \(\overline{\gamma}B_{0}\) nel sistema fisso del laboratorio (o a \(0\ Hz\) nel sistema rotante) e l'altro, di intensità inferiore centrato a \(\overline{\gamma}B_{0} - \Delta f_{fw}\) nel sistema fisso da laboratorio (o a \(- \Delta f_{fw}\) nel sistema rotante.

\begin{figure}
\centering
\includegraphics[width=6.69167in,height=3.99152in]{media/12_FatSupp/image322.pdf}\caption{Figura .: Spettro del segnale registrato da acqua e grasso}
\end{figure}

Il segnale nel voxel è dato dalla combinazione di due picchi spettrali, rispettivamente a frequenza \(f_{0} = 2\pi\omega_{o} = 2\pi\gamma B_{0}\) e shiftata di \(\Delta f_{fw}\):

\[s = s(w)\exp\left( - j\omega_{0}t \right) + s(f)\exp\left( - \left( \omega_{0} + \dfrac{\Delta f_{fw}}{2\pi} \right)t \right)\]

Nel sistema rotante, la magnetizzazione dell'acqua e del grasso, a valle dell'impulso a radiofrequenza a \(\pi/2\), mostrano comportamenti diversi:

\begin{itemize}
\item
  La magnetizzazione dell'acqua resta solidale al sistema di riferimento rotante e giace sull'asse del ribaltamento su cui è applicato l'impulso;
\item
  La magnetizzazione del grasso, procedendo con una frequenza di risonanza inferiore, è ritardata rispetto all'asse di ribaltamento.
\end{itemize}

\begin{figure}
\centering
\includegraphics[width=5.19167in,height=5.78333in]{media/12_FatSupp/image323.pdf}\caption{Figura .: Vettori di magnetizzazione di acqua e tessuto adiposo nel sistema rotante}
\end{figure}

Si suppone di eseguire un imaging mediante l'applicazione di gradienti che selezionano una riga del \(k\)-spazio. Si suppone che il gradiente di lettura sia applicato lungo l'asse \(x\).

\begin{figure}
\centering
\includegraphics[width=5in,height=5.51059in]{media/12_FatSupp/image324.pdf}\caption{Figura .: FID con gradiente di lettura il quale rende la frequenza una funzione della posizione}
\end{figure}

Nel voxel di lunghezza \(\Delta x\), a causa del gradiente applicato, vi è una variazione di frequenza dipendente dalla posizione degli isocromati:

\[\omega(x) = \gamma G_{x}x + \gamma B_{0}\]

In altre parole, tra \(x\) e \(x + \Delta x\) si ha una variazione della frequenza di risonanza di \(\Delta f\) data dalla relazione:

\[\Delta f = \overline{\gamma}G_{x}\Delta x\]

Nella banda del volumetto \(\Delta x\) sono contenute le frequenze di risonanza dell'acqua e del grasso, distanziate da uno \emph{shift} frequenziale \(\Delta f_{fw}\), una quantità intrinseca e costante. Applicando il gradiente di lettura \(G_{x}\), le due frequenze misurate sono spostate in base alla loro posizione reale. Si \(x_{1}\) la posizione fissa degli isocromati legati al grasso e \(x_{2}\) quella degli isocromati legati all'acqua. La frequenza di risonanza del grasso è, quindi:

\[f_{f} = \overline{\gamma}B_{0} - \Delta f_{fw} + \overline{\gamma}G_{x}x_{1}\]

Mentre per l'acqua:

\[f_{w} = \overline{\gamma}B_{0} + \overline{\gamma}G_{x}x_{2}\]

Poiché i due isocromati si trovano in posizioni diverse, la differenza tra le loro frequenze misurate non è più semplicemente lo shift chimico, ma una quantità che dipende anche dal gradiente applicato. La differenza di frequenza tra i due segnali è data da:

\[\Delta f_{finale} = \overline{\gamma}B_{0} - \Delta f_{fw} + \overline{\gamma}G_{x}x_{1} - \overline{\gamma}B_{0} - \overline{\gamma}G_{x}x_{2} = - \Delta f_{fw} + \overline{\gamma}G_{x}\left( x_{1} - x_{2} \right)\]

Questa relazione dimostra che la differenza di frequenza tra i due segnali è la somma di due componenti: lo shift chimico intrinseco e uno shift dovuto al gradiente e alla distanza tra le posizioni reali dei due isocromati.

È possibile applicare un gradiente di lettura tale per cui le frequenze di risonanza siano portati fuori dal range di frequenze contenute nel voxel.

\begin{figure}
\centering
\includegraphics[width=6.69167in,height=3.99167in]{media/12_FatSupp/image325.pdf}\caption{Figura .: Applicazione del gradiente tale che il picco spettrale del grasso vada all'esterno della banda di ricezione del voxel}
\end{figure}

Con questa soluzione sono acquisite solamente le frequenze contenute nel voxel e, dunque, solo le componenti legati agli isocromati dell'acqua, mentre il segnale nel grasso viene soppresso.

Il segnale del grasso, in altre parole, può essere spostato e invadere i voxel vicini, un fenomeno noto come"\emph{spatial misregistration artifact} (artefatto da dislocazione spaziale). Questo artefatto si verifica perché i segnali dei protoni del grasso vengono codificati in una posizione leggermente diversa rispetto a quelli dell'acqua.

Il segnale del grasso proviene dalla posizione \(x_{1}\) con frequenza, a causa del gradiente di lettura \(f_{f}\left( x_{1} \right) = \overline{\gamma}B_{0} - \Delta f_{fw} + \overline{\gamma}G_{x}x_{1}\). Il sistema di ricostruzione mappa il segnale nella posizione \(x_{f}\) tale per cui:

\[f\left( x_{f} \right) = \overline{\gamma}B_{0} + \overline{\gamma}G_{x}x_{f}\]

Il sistema di ricostruzione interpreta il segnale ricevuto ignorando lo shift in frequenza legato allo \emph{shielding} e lo interpreta come un segnale posizionato in \(x_{f}\) a causa dell'applicazione del gradiente. Siccome i due segnali \(f_{f}\left( x_{1} \right)\) e \(f_{f}\left( x_{f} \right)\) sono uguali, è possibile determinare lo shift di posizione:

\[f_{f}\left( x_{1} \right) = f_{f}\left( x_{f} \right) \Leftrightarrow \overline{\gamma}B_{0} - \Delta f_{fw} + \overline{\gamma}G_{x}x_{1} = \overline{\gamma}B_{0} + \overline{\gamma}G_{x}x_{f}\]

Il termine \(\overline{\gamma}B_{0}\) è in comune ai due membri, dunque, può essere semplificato:

\[- \Delta f_{fw} + \overline{\gamma}G_{x}x_{1} = + \overline{\gamma}G_{x}x_{f}\]

Risolvendo l'equazione si ottiene:

\[x_{1} - x_{f} = \dfrac{\Delta f_{fw}}{\overline{\gamma}G_{x}}\]

Questa relazione dimostra che più grande è il gradiente di lettura (\(G_{x}\)), minore sarà lo shift spaziale, riducendo così l'artefatto.

La soluzione riguardante il gradiente non è ottimale in quanto il picco spettrale associato al tessuto adiposo può essere mappato in posizione diverse da quella reale nel FOV, generando un artefatto nella ricostruzione. In altre parole, il segnale del grasso determina una variazione della densità protonica del campo di vista dell'immagine, visualizzando sull'immagine il grasso in posizioni diverse da quelle realmente presenti nel corpo.

In generale, affinché non si verifichi questo artefatto, la banda di frequenza \(\Delta f\) contenuta nel voxel di ampiezza \(\Delta x\) deve essere maggiore del \emph{chimical shift}:

\[\Delta f > \sigma_{fw}\overline{\gamma}B_{0}\]

Con questa soluzione il grasso e l'acqua sono contenuti nello stesso voxel, visualizzando così il grasso nel distretto anatomico effettivamente occupato.

Se si vuole una banda nel voxel minore del \emph{chemical shift} è necessario applicare una sequenza di inversion recovery in grado di sopprimere il grasso, combinata alla sequenza di acquisizione.

La banda del voxel deve essere opportunamente scelta. È noto, infatti, che la banda del voxel è legata al rapporto segnale/rumore: minore è la banda di frequenza contenuta nel voxel e maggiore è il rapporto segnale rumore, secondo un andamento del tipo:

\[SNR \propto \dfrac{1}{\sqrt{{BW}_{R}}}\]

Questo requisito è in contrasto con l'ampliamento della banda al fine di eliminare l'artefatto legato al \emph{chemical shift}, poiché, come visto, per ridurre tale fenomeno è necessario avere una banda per voxel maggiore dello shift in frequenza delle due sostanze- Ciò produce una riduzione del rapporto segnale/rumore.

Al fine di scegliere una banda per voxel minore, così da aumentare il rapporto segnale/rumore, è necessario adottare opportune strategie di soppressione dei tessuti di non interesse come il grasso, tramite una sequenza di \emph{inversion recovery}.

\subsection{Eccitazione selettiva}\label{eccitazione-selettiva}

Oltre alla tecnica di inversion recovery per la soppressione del segnale proveniente da un tessuto, esistono molte altre metodiche come la tecnica di eccitazione selettiva.

Grasso e acqua presentano differenti frequenze di risonanza, dunque, applicando un campo principale perfettamente omogeneo, è possibile trasmettere un impulso a radiofrequenza con banda sufficientemente stretta da ribaltare il vettore magnetizzazione di una sola specie chimica.

\begin{figure}
\centering
\includegraphics[width=6.69167in,height=2.84167in,alt={Immagine che contiene testo, diagramma, Diagramma, linea Il contenuto generato dall\textquotesingle IA potrebbe non essere corretto.}]{media/12_FatSupp/image326.pdf}\caption{Figura .: Impulso a radiofrequenza per ribaltare solamente gli isocromati di un tessuto}
\end{figure}

Si vuole eleminare il segnale proveniente dal grasso; a tale scopo si applica un impulso a \(\pi/2\) in grado di eccitare solamente tale tessuto, quindi, con frequenza \(f_{0} - \Delta f_{fw}\) e a banda molto limitata così da evitare l'eccitazione dei protoni d'acqua.

\begin{figure}
\centering
\includegraphics[width=6.425in,height=5.81667in,alt={Immagine che contiene testo, diagramma, linea, schermata Il contenuto generato dall\textquotesingle IA potrebbe non essere corretto.}]{media/12_FatSupp/image327.pdf}\caption{Figura .: Eccitazione selettiva della sola magnetizzazione legata al grasso}
\end{figure}

Subito dopo l'impulso a radiofrequenza di \(\pi/2\) si applica un gradiente detto di spoiling, tipicamente lungo \(z\), generalmente asse di \emph{slice selection}. Questo gradiente ha l'effetto di variare la frequenza di risonanza all'interno della fetta eccitata, contenente solo grasso.

\begin{figure}
\centering
\includegraphics[width=5.1in,height=3.99089in,alt={Immagine che contiene testo, schermata, diagramma, linea Il contenuto generato dall\textquotesingle IA potrebbe non essere corretto.}]{media/12_FatSupp/image328.pdf}\caption{Figura .: Gradiente di spoiling}
\end{figure}

L'effetto risultate è uno sfasamento degli isocromati che porta la magnetizzazione del grasso a essere completamente nulla. Con questa tecnica la componente longitudinale è nulla per il ribaltamento mentre quella trasversale per lo sfasamento introdotto dal gradiente di spoiling.

Dopo l'applicazione dell'impulso a radiofrequenza a banda stretta e il gradiente di spoiling, il grasso non è più eccitabile. In questa condizione si dice che il grasso è saturo mentre la tecnica è nota come \emph{fat saturation} o FAT-SAT.

Successivamente, è possibile applicare una classica sequenza di acquisizione del \(k\)-spazio come la gradient-echo o spin-echo, caratterizzata da impulsi di eccitazione con banda stretta e frequenza centrale uguale a quella di risonanza dell'acqua:

\[BW = \overline{\gamma}B_{0}\]

Si suppone di applicare una sequenza gradient-echo per acquisire il segnale, dunque, esaurito il gradiente di spoiling si applica un impulso a \(\pi/2\) per eccitare la magnetizzazione dell'acqua. La sequenza gradient-echo permette di generare l'echo.

\begin{figure}
\centering
\includegraphics[width=6.69306in,height=5.50278in,alt={Immagine che contiene testo, diagramma, linea, Diagramma Il contenuto generato dall\textquotesingle IA potrebbe non essere corretto.}]{media/12_FatSupp/image329.pdf}\caption{Figura .: Sequenza per la soppressione del grasso e imaging}
\end{figure}

Con questa sequenzia si rimuove completamente gli artefatti dovuti al grasso, poiché si annulla il segnale proveniente da questo tessuto; tuttavia, l'eccitazione selettiva presenta lo svantaggio di essere molto sensibile alla disomogeneità del campo principale. In caso di disomogeneità del campo principale, gli impulsi centrati a \(f_{0} - \Delta f_{fw}\) potrebbero eccitare porzioni di acqua che risuonano a quella frequenza per l'introduzione del termine aggiuntivo, legato alle disomogeneità \(\Delta B\), nella frequenza di Lamour dell'acqua:

\[f_{w} = \overline{\gamma}B_{0} + \overline{\gamma}\Delta B\]

\(f_{w}\) potrebbe essere uguale \(f_{0} - \Delta f_{fw}\):

\[\overline{\gamma}B_{0} + \overline{\gamma}\Delta B = f_{0} - \Delta f_{fw} \Leftrightarrow \overline{\gamma}\Delta B = - \Delta f_{fw}\]

Se la condizione \(\overline{\gamma}\Delta B = - \Delta f_{fw}\) è verificata si portano porzioni di acqua alla saturazione.

Si osservi che le disomogeneità del campo \(\Delta B\) sono garantite dai costruttori di \(1\ ppm\) del campo generato all'interno una sfera con diametro di \(25\ cm\).

L'ampiezza di \(\Delta B\) è molto vicina a quella del coefficiente di shielding e ciò può determinare la mancata cancellazione del grasso o, addirittura, la soppressione del segnale proveniente da porzioni del tessuto da analizzare.

Per limitare gli effetti delle disomogeneità di campo, l'ampiezza \(\Delta B\) dovrebbe essere inferiore a \(1\ ppm\) così da non essere confrontabile col coefficiente di shielding.

\subsection{Gradient-echo per la soppressione del grasso}\label{gradient-echo-per-la-soppressione-del-grasso}

È possibile applicare dei gradienti per sopprimere il segnale proveniente da un tessuto non voluto, sfruttando l'effetto del \emph{chemical shift}.

Si applica un impulso a radiofrequenza che eccita l'intero volume contenente un campione di acqua e grasso, dunque, con un contenuto frequenziale abbastanza ampio, uguale a \(\Delta f_{fw}\).

Applicando una sequenza gradient-echo bidimensionale, si applica un gradiente di selezione della fetta, un gradiente di codifica di fase incrementato a ogni ripetizione e, in fine, il gradiente di lettura composto da un gradiente di defasamento negativo e uno di rifasamento positivo, con area solitamente doppia del primo. Quando l'area del secondo gradiente uguaglia quella del primo, all'istante \(T_{E}\), si forma l'echo.

Sia \(t = 0\ s\) l'istante di tempo al centro dell'impulso a radiofrequenza. La fase degli isocromati può variare sia per l'applicazione del gradiente di lettura sia per il \emph{chemical shift} tra acqua e grasso. La variazione della fase legata al gradiente di lettura si verifica quando quest'ultimo è applicato, secondo la relazione:

\[\phi(t) = - \int_{}^{}{\omega dt} = - \gamma G_{x}xt\]

All'applicazione del gradiente negativo nella direzione di lettura, la fase varia con legge lineare e pendenza positiva.

Nel momento in cui il gradiente è invertito, la fase decresce con pendenza uguale, essendo uguale l'ampiezza in modulo del gradiente, ma di segno opposto. Al tempo di \(t = T_{E}\), la fase si annulla per tutti gli isocromati, formando l'echo.

Lo sfasamento introdotto dal chemical shift, avendo sempre lo stesso segno, non può essere recuperato con l'inversione del gradiente. In particolare, la frequenza di risonanza del grasso è minore di quella dell'acqua, quindi, la sua fase è ritardata rispetto all'acqua nel sistema di riferimento rotante.

Nel tempo, la pendenza della fase legata al chemical shift resta costante e negativa:

\[\phi(t) = 2\pi\Delta f_{fw} = - 2\pi\left| \Delta f_{fw} \right|\]

Dal punto di vista teorico, è possibile modellare il fenomeno del chemical shift come una disomogeneità del campo magnetico principale \(B_{0}\) che detemina una frequenza di risonanza diversa per il grasso:

\[\Delta f_{fw} = \overline{\gamma}\Delta B_{fw}\]

A causa della disomogeneità di campo \(\Delta B_{fw}\), vista solamente dagli isocromati legati alle molecole di grasso, al tempo d'echo la fase del grasso non è completamente recuperata.

\begin{figure}
\centering
\includegraphics[width=6.69306in,height=6.68194in,alt={Immagine che contiene testo, diagramma, linea, Diagramma Il contenuto generato dall\textquotesingle IA potrebbe non essere corretto.}]{media/12_FatSupp/image330.pdf}\caption{Figura .: Sequenza gradient-echo con andamento della fase di acqua e grasso}
\end{figure}

Dal punto di vista analitico, la fase dell'acqua può essere scritta, durante il gradiente di rifasamento, in cui ha pendenza negativa, come:

\[\phi_{w}\left( t' \right) = - \gamma G_{x}xt'\]

Dove la definizione \(t' = t - T_{E}\) è utile a portare l'origine dei tempi al tempo d'echo. Per definizione di \(k\)-spazio (\(k_{x} = \ \overline{\gamma}G_{x}t\)), è possibile scrivere:

\[\phi_{w}\left( k_{x} \right) = 2\pi k_{x}x\]

La fase del grasso, invece, subisce l'effetto del gradiente e delle disomogeneità di campo, legati al \emph{chemical shift}. Rispetto al tempo \(t'\) durante il gradiente di rifasamento, è possibile scrivere:

\[\phi_{f}\left( t' \right) = - \gamma G_{x}xt' - \gamma\Delta B_{fw}t'\]

Si raccoglie moltiplica e divide per \(2\pi\), al fine di ottenere \(\overline{\gamma} = \gamma/2\pi\):

\[\phi_{f}\left( t' \right) = - 2\pi\overline{\gamma}G_{x}xt' - 2\pi\overline{\gamma}\Delta B_{fw}t'\]

Si raccoglie il termine \(2\pi\overline{\gamma}G_{x}t'\):

\[\phi_{f}\left( t' \right) = - 2\pi\overline{\gamma}G_{x}\left( x + \dfrac{\Delta B_{fw}}{G_{x}} \right)t'\]

Questa relazione può essere espressa in termini del \(k\)-spazio:

\[\phi_{f}\left( k_{x} \right) = - 2\pi k_{x}\left( x + \dfrac{\Delta B_{fw}}{G_{x}} \right)\]

Nel \(k\)-spazio, la fase del grasso è sfasata di una quantità \(- \ \Delta B_{fw}/G_{x}\). Siccome la fase del grasso, al tempo d'echo, non è recuperata, è necessario introdurre un termine addizionale di fase che tiene conto dello sfasamento introdotto dall'ambiente molecolare:

\[\phi_{f}\left( k_{x} \right) = - 2\pi k_{x}\left( x + \dfrac{\Delta B_{fw}}{G_{x}} \right) - 2\pi\overline{\gamma}\Delta B_{fw}T_{E}\]

Si pone:

\[\Delta\omega_{fw} = 2\pi\overline{\gamma}\Delta B_{fw}\]

La fase aggiuntiva è presente nel segnale del voxel ricostruito:

\[s\left( T_{E} \right) = s_{w} + s_{f}\exp\left( - j\Delta\omega_{fw}T_{E} \right)\]

La differenza di fase al tempo d'echo si rifletta sul segnale ricostruito come un termine di fase che rendere il segnale proveniente da grasso complesso. Dunque, il segnale del grasso dipende dalla fase \(\Delta\omega_{fw}T_{E}\), che a sua volta dipende strettamente dal tempo d'echo.

Per annullare questo segnale la tecnica prevede di acquisire due immagini, una allineata con la fase del grasso e una in opposizione di fase. Per acquisire l'immagine in fase col segnale del grasso, si sceglie il tempo di echo da che \(\Delta\omega_{fw}T_{E}\) sia un multiplo intero di \(2\pi\):

\[\Delta\omega_{fw}T_{E} = 2n\pi,n\mathbb{\in N}\]

In questa condizione, il segnale del voxel legato al grasso è puramente reale:

\[\left. \ s_{f}\exp\left( - j\Delta\omega_{fw}T_{E} \right) \right|_{\Delta\omega_{fw}T_{E} = 2n\pi} = s_{f}\]

L'immagine in fase è ottenuta sommando le immagini ricostruite di acqua e grasso:

\[s_{IN} = \left. \ s\left( T_{E} \right) \right|_{\Delta\omega_{fw}T_{E} = 2n\pi} = s_{w} + s_{f}\]

Dal punto di vista del sistema rotante, le due magnetizzazioni di acqua e grasso sono in fase, ovvero il loro sposamento differisce per un multiplo interno di \(2\pi\). Il segnale risultante è dato dalla somma dei segnali provenienti dai due tessuti.

\begin{figure}
\centering
\includegraphics[width=0.85012in,height=1.79192in,alt={Immagine che contiene schizzo, linea Il contenuto generato dall\textquotesingle IA potrebbe non essere corretto.}]{media/12_FatSupp/image331.pdf}\caption{Figura .: Magnetizzazioni in fase di acqua e grasso}
\end{figure}

La seconda immagine è ottenuta in opposizione di fase, ovvero il tempo di echo è scelto in modo tale che:

\[\Delta\omega_{fw}T_{E} = n\pi,n\mathbb{\in N}\]

In questa situazione, gli isocromati di acqua e grasso sono in opposizione di fase. Nel sistema di riferimento rotante, le due fasi differiscono per un multiplo interno di \(\pi\):

\begin{figure}
\centering
\includegraphics[width=1.03139in,height=1.69815in,alt={Immagine che contiene schizzo, unghia/chiodo, bianco e nero Il contenuto generato dall\textquotesingle IA potrebbe non essere corretto.}]{media/12_FatSupp/image332.pdf}\caption{Figura .: Magnetizzazioni in opposizione di fase di acqua e grasso}
\end{figure}

La fase dell'immagine del grasso, con questa scelta, è negativa, dunque, l'intero segnale è negativo:

\[\left. \ s_{f}\exp\left( - j\Delta\omega_{fw}T_{E} \right) \right|_{\Delta\omega_{fw}T_{E} = n\pi} = - s_{f}\]

L'immagine dovuta al grasso si sottrae a quella dell'acqua, ottenendo così il segnale del voxel:

\[s_{OP} = \left. \ s\left( T_{E} \right) \right|_{\Delta\omega_{fw}T_{E} = n\pi} = s_{w} - s_{f}\]

Acquisite le due immagini è possibile ricavare sia l'immagine dell'acqua \(s_{w}\) sia quella del grasso \(s_{f}\) mediante delle semplici operazioni di somma e sottrazione. Se, ad esempio, si sommano le due immagini in fase si risale all'immagine dell'acqua:

\[s_{OP} + s_{IN} = s_{w} - s_{f} + s_{w} + s_{f} = 2s_{w}\]

Da cui:

\[s_{w} = \dfrac{s_{OP} + s_{IN}}{2}\]

Mediante sottrazione tra le due immagini, invece, si ricava l'immagine del grasso:

\[s_{IN} - s_{OP} = s_{w} + s_{f} - s_{w} + s_{f} = 2s_{f}\]

Da cui:

\[s_{f} = \dfrac{s_{IN} - s_{OP}}{2}\]

Con questa metodica è semplice ricostruire l'immagine della sezione anatomica di interesse rimuovendo la componente del segnale del voxel indesiderata. In caso di acqua e grasso i tempi di echo da utilizzare sono, all'incirca:

\[T_{E,IN} = \dfrac{2\pi}{\Delta\omega_{fw}} \simeq 4.54\ ms\]

\[T_{E,OP} = \dfrac{\pi}{\Delta\omega_{fw}} \simeq 2.27\ ms\]

Dove \(\Delta\omega_{fw} = 2\pi\overline{\gamma}\Delta B_{fw}\) e \(\Delta B_{fw} = \gamma\Delta f_{fw} \Leftrightarrow \ \Delta f_{fw} = \Delta B_{fw}/\gamma\). Combinando si ottiene:

\[\Delta\omega_{fw} = 2\pi\ \Delta f_{fw} \simeq 1382\ rad/s\]

La metodica non richiede una grande omogeneità di campo poiché non eccita selettivamente un tessuto; tuttavia, dovendo acquisire due immagini dello stesso distretto anatomico, con tempi di echo diversi, questa strategia presenta lo svantaggio di incrementare i tempi di acquisizione.

\begin{center}
\vfill
    \chapter{Fast Imaging}
    \label{blx:refsection\therefsection}
\vfill

\minitoc
\newpage
\end{center}
\justify



\section{Fast Imaging in the Steady State}\label{fast-imaging-in-the-steady-state}

L'imaging rapido è forse una delle tecniche più interessanti della risonanza magnetica (MRI). Ci sono varie metodiche basate sull'applicazione rapida di gradienti oppure angoli di ribaltamento della magnetizzazione inferiore a \(\pi/2\).

Le tecniche di fast imaging permettono di superare di superare uno dei maggiori limiti della risonanza magnetica, ovvero l'elevato tempo di acquisizione. Se, ad esempio, il tempo di rilassamento longitudinale è di \(1\ s\) come nei tessuti più liquidi, è necessario aspettare circa \(3 \div 5\ s\) tra una ripetizione e l'altra.

Una possibile soluzione riguarda l'applicazione di un gradiente dopo l'applicazione di un impulso a radiofrequenza, che ribalta la magnetizzazione di un flip \(\vartheta < \pi/2\). Ciò implica che la magnetizzazione longitudinale, invece di essere ribaltata lungo uno degli assi del sistema rotante, è ruotata di un angolo \(\vartheta < \pi/2\). Con questa soluzione, il tempo necessario affinché la magnetizzazione torni all'equilibrio dopo l'ultimo gradiente sia inferiore.

Sia \(t = 0\ s\) il tempo di applicazione dell'impulso a radiofrequenza che ribalta la magnetizzazione di un angolo \(\vartheta\). Il vettore di magnetizzazione un instante temporale immediatamente successivo, \(t = 0^{+}\ s\), avrà due componenti longitudinale e trasversale date dalla sua proiezione lungo gli assi:

\[\left\{ \begin{matrix}
M_{z}\left( 0^{+} \right) = M_{0}\cos\vartheta \\
M_{\bot}\left( 0^{+} \right) = M_{0}\sin\vartheta
\end{matrix} \right.\ \]

\begin{figure}
\centering
\includegraphics[width=3.98717in,height=2.63469in,alt={Immagine che contiene testo, linea, Carattere, diagramma Il contenuto generato dall\textquotesingle IA potrebbe non essere corretto.}]{media/13_FastImm/image333.pdf}\caption{Figura .: Proiezione del vettore magnetizzazione lungo gli assi}
\end{figure}

Secondo le equazioni di Bloch, tra un impulso a radiofrequenza e il successivo della sequenza seguente, la magnetizzazione longitudinale tende al valore di regime \(M_{0}\) con costante di tempo \(T_{1}\). Tra i due impulsi non è detto che la magnetizzazione longitudinale raggiunga l'equilibrio termico. Quindi, se il tempo di ripetizione \(T_{R}\) è tale che la magnetizzazione longitudinale non ritorni all'equilibrio, al successivo impulso le componenti del vettore magnetizzazione saranno minori di quelle ottenute col primo impulso.

Al terzo impulso a radiofrequenza la componente longitudinale, avendo raggiunto un valore minore del primo caso, è ridotta ulteriormente. Per un numero sufficiente di cicli, il processo giunge a un equilibrio dinamico o \emph{steady state}. In altre parole, la componente longitudinale, nell'intervallo tra una sequenza e la successiva di ampiezza \(T_{R}\), recupera la stessa ampiezza che aveva prima dell'impulso a radiofrequenza. In conclusione, terminato il transitorio, la sequenza permette di eseguire l'imaging in modo rapido.

\begin{figure}
\centering
\includegraphics[width=6.69306in,height=4.15486in,alt={Immagine che contiene testo, linea, Diagramma, diagramma Il contenuto generato dall\textquotesingle IA potrebbe non essere corretto.}]{media/13_FastImm/image334.pdf}\caption{Figura .: Andamento della magnetizzazione longitudinale fino al raggiungimento dello steady state}
\end{figure}

\subsection{Valutazione del segnale registrato}\label{valutazione-del-segnale-registrato}

Per valutare i vantaggi e svantaggi e, più in generale, le caratteristiche della tecnica a steady state è necessario valutare analiticamente il segnale ricevuto. Si suppone di applicare una sequenza gradient-echo, composta da un impulso a radiofrequenza, che ribalta la magnetizzazione di un flip angle \(\vartheta\), un gradiente di selezione della fetta nell'imaging bidimensionale, un gradiente di codifica di fase, variato tra una sequenza e la successiva e, infine, un gradiente di lettura. Tra l'applicazione di una sequenza e la successiva è necessario applicare un gradiente di spoiling lungo le tre dimensioni del \(k\)-spazio, variabile da una sequenza e la successiva.

scopo principale del gradiente di spoiling è eliminare la magnetizzazione trasversale residua alla fine di ogni ciclo di acquisizione, per evitare che interferisca con il segnale del ciclo successivo. Questo è cruciale per ottenere immagini \(T_{1}\)-pesate. Avere un gradiente di spoiling con gradiente variabile in modo casuale o semi-casuale a ogni ripetizione assicura che il rifasamento degli spin sia differente per ogni ciclo, eliminando il segnale residuo. Questo è un modo per evitare l'accumulo di echi indesiderati che potrebbero compromettere il contrasto dell\textquotesingle immagine.

\begin{figure}
\centering
\includegraphics[width=6.68333in,height=6.65in]{media/13_FastImm/image335.pdf}\caption{Figura .: Sequenza gradient-echo con gradienti di spoiling di ampiezza variabile e impulso RF a \(\vartheta\)}
\end{figure}

L'applicazione dei gradienti di spoling determina il nome della sequenza short-\(T_{R}\) spoiled gradient-echo. Questi gradienti hanno il compito di sfasare la magnetizzazione trasversale prima di raggiungere il tempo. In questo modo, prima di applicare l'impulso a radiofrequenza la magnetizzazione trasversale si annulla, mentre la componente longitudinale è non nulla.

In assenza di gradienti di spoiled, all'applicazione dell'impulso a \(\vartheta\) è necessario considerare anche la componente trasversale, rendendo di conseguenza l'analisi più complessa. Nella trattazione successiva si suppone di non adoperare gradienti di spoiling al fine di studiare il caso più completo.

Sia \(t_{n}\) il tempo che intercorre tra l'impulso \(n\)-esimo e l'impulso \(n + 1\)-esimo. Tra i due impulsi deve passare un tempo \(T_{R}\). È valida la relazione:

\[nT_{R} < t_{n} < (n + 1)T_{R}\]

\begin{figure}
\centering
\includegraphics[width=5.05833in,height=0.80833in,alt={Immagine che contiene testo, schermata, linea, Carattere Il contenuto generato dall\textquotesingle IA potrebbe non essere corretto.}]{media/13_FastImm/image336.pdf}\caption{Figura .: Intervallo tra l\textquotesingle impulso \(n\)-esimo e \((n + 1)\)-esimo}
\end{figure}

Con un abuso di notazione si pone \(t_{n} = t - nT_{R}\) in modo da portare l'origine dei tempi in \(nT_{R}\). Nell'intervallo tra due ripetizioni \(t_{n} = t - nT_{R}\) rappresenta lo spostamento dall'origine dei tempi all'istante \(nT_{R}\) di applicazione dell'impulso \(n\)-esimo. Al tempo \(t_{n}\) la componente trasversale ricevuta dall'antenna è data delle equazioni di Bloch:

\[M_{\bot}\left( t_{n} \right) = M_{\bot}\left( 0^{+} \right)\exp\left( - \frac{t_{n}}{T_{2}} \right),0 < t_{n} < T_{R}\]

Dove \(t = 0^{+}\) è l'istante immediatamente successivo all'applicazione dell'impulso a radiofrequenza. La componente trasversale dall'inizio della sequenza è legata alla magnetizzazione longitudinale dalla proiezione della componente longitudinale lungo il piano trasverso:

\[M_{\bot}\left( 0^{+} \right) = M_{z}\left( 0^{-} \right)\sin\vartheta\]

Ovviamente, \({\overset{\underline{}}{M}}_{\bot}\) dipende dal valore che assume la componente longitudinale un istante prima dell'impulso a \(\vartheta\), poiché in quest'ultimo la magnetizzazione longitudinale è ribaltata quasi istantaneamente. La componente trasversa è:

\[M_{\bot}\left( t_{n} \right) = {\overset{\underline{}}{M}}_{\bot}\left( 0^{+} \right)\exp\left( - \frac{t_{n}}{T_{2}} \right) = {\overset{\underline{}}{M}}_{z}\left( 0^{-} \right)\sin\vartheta\exp\left( - \frac{t_{n}}{T_{2}} \right)\]

Per calcolare con precisione la componente trasversale è necessario esplicitare la componente longitudinale. Per l'equazione di Bloch, tra l'applicazione di un impulso e il successivo, la componente longitudinale è data da:

\[M_{z}\left( t_{n} \right) = M_{0}\left( 1 - \exp\left( - \frac{t_{n}}{T_{1}} \right) \right) + M_{z}\left( 0^{+} \right)\exp\left( - \frac{t_{n}}{T_{1}} \right)\]

La componente longitudinale dipende dal valore che aveva prima dell'impulso a radiofrequenza, che a sua volta è legata alla componente longitudinale prima dell'impulso:

\[M_{z}\left( 0^{+} \right) = M_{z}\left( 0^{-} \right)\cos\vartheta\]

L'equazione complessiva è data da:

\[M_{z}\left( t_{n} \right) = M_{0}\left( 1 - \exp\left( - \frac{t_{n}}{T_{1}} \right) \right) + M_{z}\left( 0^{-} \right)\cos\vartheta\exp\left( - \frac{t_{n}}{T_{1}} \right)\]

I transitori si estinguono con andamento esponenziale e costante di tempo \(T_{1}\). Dopo un certo numero di impulsi il transitorio si estingue, dunque, le riduzioni del segnale registrato si esauriscono. Si raggiunge così lo stato steady state nel quale tra un impulso e il successivo la magnetizzazione recupera completamente il valore che aveva prima dell'impulso a radiofrequenza.

Dal punto di vista analitico, la magnetizzazione trasversale un attimo prima dell'impulso \(n + 1\)-esimo è:

\[M_{\bot}\left( (n + 1)T_{R} \right) = M_{\bot}\left( nT_{R}^{+} \right)\exp\left( - \frac{T_{R}}{T_{2}} \right)\exp\left( - \frac{t_{n}}{T_{2}} \right)\]

Per semplicità di notazione si pone:

\[E_{2} = \exp\left( - \frac{T_{R}}{T_{2}} \right)\]

Con questa posizione la magnetizzazione trasversale si scrive come:

\[M_{\bot}\left( (n + 1)T_{R} \right) = M_{\bot}\left( nT_{R}^{+} \right)E_{2}\exp\left( - \frac{t_{n}}{T_{2}} \right)\]

La magnetizzazione trasversale all'istante \(nT_{R}^{+}\) è uguale alla proiezione della magnetizzazione longitudinale un istante prima di applicare l'impulso \(n\)-esimo:

\[M_{\bot}\left( nT_{R}^{+} \right) = M_{z}\left( nT_{R}^{-} \right)\sin\vartheta\]

Da cui:

\[M_{\bot}\left( (n + 1)T_{R} \right) = M_{z}\left( nT_{R}^{+} \right)\sin\vartheta\left( nT_{R}^{+} \right)E_{2}\]

La componente longitudinale, un attimo prima dell'applicazione dell'impulso \(n + 1\)-esimo è:

\[M_{z}\left( (n + 1)T_{R}^{-} \right) = M_{z}\left( nT_{R}^{-} \right)\cos\vartheta\exp\left( - \frac{T_{R}}{T_{1}} \right) + M_{0}\left( 1 - \exp\left( - \frac{T_{R}}{T_{1}} \right) \right)\]

Per semplicità di notazione si pone;

\[E_{1} = \exp\left( - \frac{T_{R}}{T_{1}} \right)\]

Con questa posizione, la componente longitudinale si scrive come:

\[M_{z}\left( (n + 1)T_{R}^{-} \right) = M_{z}\left( nT_{R}^{-} \right)E_{1}\cos\vartheta + M_{0}\left( 1 - E_{1} \right)\]

All'equilibrio termodinamico si instaura una componente longitudinale \(M_{ze}\) che viene recuperata nell'intervallo tra l'applicazione di un impulso a radiofrequenza e il successivo. Quindi la magnetizzazione longitudinale un istante prima dell'applicazione dell'impulso a \(\vartheta\) deve essere uguale all'istante prima l'applicazione dell'impulso successivo:

\[M_{z}\left( (n + 1)T_{R}^{-} \right) = M_{z}\left( nT_{R}^{-} \right) = M_{ze}\]

La relazione che lega la componente longitudinale con gli estremi dell'intervallo di ampiezza \(T_{R}\), \(\left\lbrack nT_{R};(n + 1)T_{R} \right\rbrack\), si scrive come:

\[M_{z}\left( (n + 1)T_{R}^{-} \right) = M_{z}\left( nT_{R}^{-} \right)E_{1}\cos\vartheta + M_{0}\left( 1 - E_{1} \right) \Leftrightarrow M_{ze} = M_{ze}E_{1}\cos\vartheta + M_{0}\left( 1 - E_{1} \right)\]

Con questa relazione è possibile ricavare il valore della magnetizzazione all'equilibrio. Si portano al primo membro i termini contenenti \(M_{ze}\):

\[M_{ze} - M_{ze}E_{1}\cos\vartheta = M_{0}\left( 1 - E_{1} \right)\]

Ricavando \(M_{ze}\) si ottiene:

\[M_{ze} = M_{0}\frac{1 - E_{1}}{1 - E_{1}\cos\vartheta}\]

Nota questa quantità è possibile analizzare il comportamento della magnetizzazione trasversale all'equilibrio. Dalla relazione \(M_{\bot}\left( (n + 1)T_{R} \right) = M_{z}\left( nT_{R}^{+} \right)\sin\vartheta\left( nT_{R}^{+} \right)E_{2}\), si ha:\(\left( t_{n} \right) = {\overset{\underline{}}{M}}_{\bot}\left( 0^{+} \right)\exp\left( - t_{n}/T_{2} \right)\), si ha:

\[M_{\bot}\left( (n + 1)T_{R} \right) = M_{ze}\sin\vartheta\left( nT_{R}^{+} \right)E_{2}\exp\left( - \frac{t_{n}}{T_{2}} \right) = M_{0}\sin\vartheta\frac{1 - E_{1}}{1 - E_{1}\cos\vartheta}\exp\left( - \frac{t_{n}}{T_{2}} \right)\]

Nota la componente trasversale della magnetizzazione è possibile ricostruire l'immagine della densità protonica del voxel, dipendente da \(\vartheta\) e il tempo di echo:

\[\widehat{\rho}\left( \vartheta,T_{E} \right) = \rho_{0}\sin\vartheta\frac{1 - E_{1}}{1 - E_{1}\cos\vartheta}\exp\left( - \frac{T_{E}}{T_{2}^{*}} \right)\]

La densità protonica ottenuta, ovvero l'immagine, non ha una pesatura semplice in \(T_{1}\), \(T_{2}\) o \(\rho_{0}\) me dipende da tutti questi parametri, secondo la relazione individuata. Tipici tempi di ripetizione sono di \(T_{E} \simeq 4\ ms\).

\begin{figure}
\centering
\includegraphics[width=6.69306in,height=2.48333in,alt={Immagine che contiene testo, linea, diagramma, Diagramma Il contenuto generato dall\textquotesingle IA potrebbe non essere corretto.}]{media/13_FastImm/image337.pdf}\caption{Figura .: Andamento del segnale del voxel proveniente dalla materia bianca e griggia al variare del tempo di ripetizione (sinistra) e del flip angle (destra)}
\end{figure}

La sequenza appena introdotta è nota come short \(T_{R}\) steady state incoherent o SSI, dove il termine incoherent si riferisce alla presenza del gradiente di spoiling che sfasa tutti gli isocromati per annullare la componente trasversa prima di ogni ripetizione. In assenza dei gradienti di spoling si ottiene una sequenza short-\(T_{R}\) steady state coherent.

\subsection[Pesatura in T1 con sequenza short-TR steady state]{Pesatura in $\mathbf{T}_{\mathbf{1}}$ con sequenza short-$\mathbf{T}_{\mathbf{R}}$ steady state}
\label{pesatura-in-T1-short-TR-steady-state}

Dalla relazione individuata per \(\widehat{\rho}\left( \vartheta,T_{E} \right)\) si evince che la pesatura dipende sia dal tempo d'echo che dal flip angle \(\vartheta\) dell'impulso a radiofrequenza. Si vuole determinare l'angono in corrispondenza del quale il segnale assume il suo valore massimo. A tale scopo si deriva rispetto \(\vartheta\) e si pone a zero la derivata

\[\frac{\partial\widehat{\rho}}{\partial\vartheta} = 0 \Leftrightarrow \frac{\partial}{\partial\vartheta}\left( \rho_{0}\sin\vartheta\frac{1 - E_{1}}{1 - E_{1}\cos\vartheta}\exp\left( - \frac{T_{E}}{T_{2}^{*}} \right) \right) = 0\]

Grazie alla linearità della derivata, si ha:

\[\left( 1 - E_{1} \right)\rho_{0}\exp\left( - \frac{T_{E}}{T_{2}^{*}} \right)\frac{\partial}{\partial\vartheta}\left( \sin\vartheta\frac{1}{1 - E_{1}\cos\vartheta} \right) = 0\]

I termini costante possono essere semplificati poiché diversi da zero, quindi:

\[\frac{\partial}{\partial\vartheta}\left( \sin\vartheta\frac{1}{1 - E_{1}\cos\vartheta} \right) = 0\]

Si esegue la derivata:

\[\cos\vartheta\frac{1}{1 - E_{1}\cos\vartheta} + \sin\vartheta\frac{1}{\left( 1 - E_{1}\cos\vartheta \right)^{2}}\left( - E_{1}\sin\vartheta \right) = 0\]

Si considera \(1 - E_{1}\cos\vartheta\) e si verifica se può essere uguale a zero:

\[1 - E_{1}\cos\vartheta = 0 \Leftrightarrow \cos\vartheta = \frac{1}{E_{1}}\]

Per definizione \(E_{1} = \exp\left( - T_{R}/T_{1} \right)\). Poiché \(T_{R}\) è dell'ordine del \(ms\) mentre \(T_{1}\) dei secondi, l'esponenziale è circa uguale all'unità:

\[\exp\left( - \frac{T_{R}}{T_{1}} \right) \simeq 1\]

La condizione \(\cos\vartheta = E_{1}^{- 1}\) è verificata quando \(\vartheta \simeq 2m\pi,m \in \mathbb{N}_{0}\). Ne discende che è possibile semplificare \(1 - E_{1}\cos\vartheta\) nell'espressione per valutare \(\vartheta\) massimo, a patto che la soluzione \(\vartheta_{opt}\) non sia prossima a \(0\). Si ottiene:

\[\cos\vartheta - E_{1}\sin^{2}\vartheta\frac{1}{1 - E_{1}\cos\vartheta} = 0\]

Si esegue il minimo comune multiplo:

\[\left( 1 - E_{1}\cos\vartheta \right)\cos\vartheta - E_{1}\sin^{2}\vartheta = 0\]

Svolgendo i prodotti si ottiene:

\[\cos\vartheta - E_{1}\cos^{2}\vartheta - E_{1}\sin^{2}\vartheta = 0\]

Si raccoglie \(- E_{1}\):

\[\cos\vartheta - E_{1}\left( \cos^{2}\vartheta + \sin^{2}\vartheta \right) = 0\]

Applicando le relazioni trigonometriche, si ottiene:

\[\cos\vartheta - E_{1} = 0\]

Da cui:

\[\cos\vartheta = E_{1}\]

Applicando la funzione inversa al coseno si ottiene l'angolo di massimo:

\[\vartheta_{opt} = \arccos\left( E_{1} \right)\]

Applicando la definizione di \(E_{1}\) si ricava:

\[\vartheta_{opt} = \arccos\left( \exp\left( - \frac{T_{R}}{T_{1}} \right) \right)\]

L'angolo \(\vartheta_{opt}\), in corrispondenza del quale il segnale del tessuto nel voxel ha il suo valore massimo, è detto angolo di Ernst ed è una quantità minore di \(\pi/2\). Questo angolo è generalmente indicato con \(\vartheta_{E}\).

Nell'intorno dell'angolo di Ernst si esaltano i tessuti che presentano un tempo di rilassamento minore, realizzando una pesatura in \(T_{1}\). Quindi, in base alla scelta dell'angolo di ribaltamento, si ottengono immagini diverse in grado di mostrare pesature diverse. Ad esempio, per piccoli flip angle la materia grigia, avendo un tempo di rilassamento longitudinale \(T_{1}\) minore, risulta più luminosa della materia bianca. Per flip angle intorno ai \(17{^\circ}\), invece, si ha un cambio di tendenza in quanto la materia grigia risulta essere meno luminosa della materia bianca.

In conclusione, note le caratteristiche dei tessuti è possibile scegliere opportunamente il valore del flip-angle.

\subsection{Angolo di Ernst per piccoli tempi di ripetizione}\label{angolo-di-ernst-per-piccoli-tempi-di-ripetizione}

Si suppone che il tempo di ripetizione tra una sequenza e l'altra sia molto minore del tempo di rilassamento longitudinale \(T_{1}\) dei tessuti:

\[T_{R} \ll T_{1}\]

In questa ipotesi, il termine \(E_{1}\) piò essere sviluppato in serie di Taylor:

\[E_{1} = \exp\left( - \frac{T_{R}}{T_{1}} \right) \simeq 1 - \frac{T_{R}}{T_{1}}\]

È valida la relazione trigonometrica:

\[\sin\vartheta = \sqrt{1 - \cos^{2}\vartheta}\]

Applicando questa relazione per l'angolo di Ernst \(\vartheta = \vartheta_{E} = \arccos\left( E_{1} \right)\) è possibile scrivere:

\[\sin\vartheta_{E} = \sqrt{1 - \cos^{2}\vartheta_{E}} = \sqrt{1 - \cos^{2}\left( \arccos\left( E_{1} \right) \right)}\]

Ma \(\cos^{2}\left( \arccos\left( E_{1} \right) \right) = E_{1}^{2}\) perché funzioni inverse, per cui:

\[\sin\vartheta_{E} = \sqrt{1 - E_{1}^{2}}\]

Applicando l'approssimazione per \(T_{R} \ll T_{1}\) si scrive:

\[\sin\vartheta_{E} = \sqrt{1 - E_{1}^{2}} \simeq \sqrt{1 - \left( 1 - \frac{T_{R}}{T_{1}} \right)^{2}}\]

Svolgendo il quadrato, si ottiene:

\[\sin\vartheta_{E} \simeq \sqrt{1 - 1 + 2\frac{T_{R}}{T_{1}} - \left( \frac{T_{R}}{T_{1}} \right)^{2}} = \sqrt{2\frac{T_{R}}{T_{1}} - \left( \frac{T_{R}}{T_{1}} \right)^{2}}\]

Siccome \(T_{R}/T_{1} \ll 1\), è possibile trascurare i termini di ordine superiore al primo, ottenendo:

\[\sin\vartheta_{E} \simeq \sqrt{2\frac{T_{R}}{T_{1}}}\]

Se \(T_{R} \ll T_{1}\), il termine \(E_{1} = \exp\left( - T_{R}/T_{1} \right) \simeq 1\). In questa condizione l'angolo di Ernst è molto minore di \(1\), poiché l'arcocoseno in prossimità di \(1\) tende a \(0\):

\[\vartheta_{E} = \arccos\left( E_{1} \right) \ll 1\]

In questa ipotesi è possibile approssimare il seno con il suo argomento, così da ricavare un'espressione semplice per l'angolo di Ernst:

\[\vartheta_{E} \simeq \sqrt{2\frac{T_{R}}{T_{1}}}\]

Con questa approssimazione, la componente trasversale del vettore di magnetizzazione, valutata nell'angolo di Ernst è:

\[M_{\bot}\left( \vartheta_{E} \right) = M_{0}\sin\vartheta_{E}\frac{1 - E_{1}}{1 - E_{1}\cos\vartheta_{E}}\exp\left( - \frac{t_{n}}{T_{2}} \right)\]

Si è visto che \(\sin\vartheta_{E} = \sqrt{1 - E_{1}^{2}}\) e che \(\vartheta_{E} = \arccos\left( E_{1} \right)\), per cui:

\[M_{\bot}\left( \vartheta_{E} \right) = M_{0}\sin\vartheta_{E}\frac{1 - E_{1}}{1 - E_{1}\cos\vartheta_{E}}\exp\left( - \frac{t_{n}}{T_{2}} \right) = M_{0}\sqrt{1 - E_{1}^{2}}\ \frac{1 - E_{1}}{1 - E_{1}\cos\left( \arccos\left( E_{1} \right) \right)}\exp\left( - \frac{t_{n}}{T_{2}} \right)\]

Da cui:

\[M_{\bot}\left( \vartheta_{E} \right) = M_{0}\sqrt{1 - E_{1}^{2}}\ \frac{1 - E_{1}}{1 - E_{1}^{2}}\exp\left( - \frac{t_{n}}{T_{2}} \right)\]

Si scompongono i termini \(1 - E_{1}^{2}\) in \(\left( 1 - E_{1} \right)\left( 1 + E_{1} \right)\):

\[M_{\bot}\left( \vartheta_{E} \right) = M_{0}\sqrt{\left( 1 - E_{1} \right)\left( 1 + E_{1} \right)}\ \frac{1 - E_{1}}{\left( 1 - E_{1} \right)\left( 1 + E_{1} \right)}\exp\left( - \frac{t_{n}}{T_{2}} \right)\]

Nella frazione è possibile semplificare \(1 - E_{1}\):

\[M_{\bot}\left( \vartheta_{E} \right) = M_{0}\ \frac{\sqrt{\left( 1 - E_{1} \right)\left( 1 + E_{1} \right)}}{\left( 1 + E_{1} \right)}\exp\left( - \frac{t_{n}}{T_{2}} \right)\]

Grazie alle proprietà delle radici, è possibile scrivere:

\[M_{\bot}\left( \vartheta_{E} \right) = M_{0}\ \sqrt{\frac{\left( 1 - E_{1} \right)\left( 1 + E_{1} \right)}{\left( 1 + E_{1} \right)^{2}}}\exp\left( - \frac{t_{n}}{T_{2}} \right)\]

Da cui:

\[M_{\bot}\left( \vartheta_{E} \right) = M_{0}\ \sqrt{\frac{1 - E_{1}}{1 + E_{1}}}\exp\left( - \frac{t_{n}}{T_{2}} \right)\]

Nella condizione \(T_{R} \ll T_{1}\) è possibile applicare lo sviluppo di Taylor per \(E_{1}\), ottenendo:

\[M_{\bot}\left( \vartheta_{E} \right) = M_{0}\ \sqrt{\frac{1 - E_{1}}{1 + E_{1}}}\exp\left( - \frac{t_{n}}{T_{2}} \right) \simeq M_{0}\ \sqrt{\frac{1 - \left( 1 - \frac{T_{R}}{T_{1}} \right)}{1 + 1 - \frac{T_{R}}{T_{1}}}}\exp\left( - \frac{t_{n}}{T_{2}} \right) =\]

Svolendo le somme, si ottiene:

\[M_{\bot}\left( \vartheta_{E} \right) \simeq M_{0}\ \sqrt{\frac{\frac{T_{R}}{T_{1}}}{2 - \frac{T_{R}}{T_{1}}}}\exp\left( - \frac{t_{n}}{T_{2}} \right)\]

Dato che \(T_{R} \ll T_{1}\) è possibile trascurare \(T_{R}/T_{1}\) al denominatore:

\[M_{\bot}\left( \vartheta_{E} \right) \simeq M_{0}\ \sqrt{\frac{\frac{T_{R}}{T_{1}}}{2 - \frac{T_{R}}{T_{1}}}}\exp\left( - \frac{t_{n}}{T_{2}} \right) \simeq M_{0}\ \sqrt{\frac{T_{R}}{2T_{1}}}\exp\left( - \frac{t_{n}}{T_{2}} \right)\]

Poiché \(T_{2} < T_{1}\) e \(0 < t_{n} < T_{R}\) il termine \(\exp\left( - t_{n}/T_{2} \right)\) tende all'unità:

\[\exp\left( - \frac{t_{n}}{T_{2}} \right) \simeq 1\]

Si ottiene:

\[M_{\bot}\left( \vartheta_{E} \right) \simeq M_{0}\ \sqrt{\frac{T_{R}}{2T_{1}}}\]

All'interno della radice quadrata, si moltiplica e divide per \(2\):

\[M_{\bot}\left( \vartheta_{E} \right) \simeq M_{0}\sqrt{\frac{T_{R}}{2T_{1}}} = M_{0}\sqrt{\frac{2T_{R}}{4T_{1}}}\]

Poiché si è visto che:

\[\vartheta_{E} \simeq \sqrt{2\frac{T_{R}}{T_{1}}}\]

È possibile concludere che:

\[M_{\bot}\left( \vartheta_{E} \right) \simeq M_{0}\frac{\vartheta_{E}}{2}\]

La magnetizzazione trasversale, nel punto di massimo \(\vartheta_{E}\) e per tempi di ripetizioni molto minori del tempo di rilassamento longitudinale, è proporzionale a metà dell'angolo di Ernst, espresso in radianti, tramite la costante \(M_{0}\).

\subsection[Pesatura in densità protonica con sequenza short-TR steady state]{Pesatura in densità protonica con sequenza short-$\mathbf{T}_{\mathbf{R}}$ steady state}
\label{pesatura-in-densita-protonica-short-TR}

La relazione generale della magnetizzazione trasversa, in funzione del flip angle, è data da:

\[M_{\bot}(\vartheta) = M_{0}\sin\vartheta\frac{1 - E_{1}}{1 - E_{1}\cos\vartheta}\exp\left( - \frac{t_{n}}{T_{2}} \right)\]

Si suppone che \(\vartheta \ll 1\). È possibile arrestare lo sviluppo del seno al primo ordine e del coseno al secondo ordine:

\[\sin\vartheta \simeq \ \vartheta\]

\[\cos(\vartheta) \simeq 1 - \frac{\vartheta^{2}}{2}\]

La componente trasversale del vettore magnetizzazione può essere approssimata come:

\[M_{\bot}(\vartheta) = M_{0}\sin\vartheta\frac{1 - E_{1}}{1 - E_{1}\cos\vartheta}\exp\left( - \frac{t_{n}}{T_{2}} \right) \simeq M_{0}\vartheta\frac{1 - E_{1}}{1 - E_{1}\left( 1 - \frac{\vartheta^{2}}{2} \right)}\exp\left( - \frac{t_{n}}{T_{2}} \right)\]

Si divide e moltiplica il secondo membro per \(1 - E_{1}\):

\[M_{\bot}(\vartheta) \simeq \frac{M_{0}\vartheta}{\frac{1 - E_{1}\left( 1 - \frac{\vartheta^{2}}{2} \right)}{1 - E_{1}}}\exp\left( - \frac{t_{n}}{T_{2}} \right)\]

Svolgendo i prodotti, si ottiene:

\[M_{\bot}(\vartheta) \simeq \frac{M_{0}\vartheta}{\frac{1 - E_{1} + E_{1}\frac{\vartheta^{2}}{2}}{1 - E_{1}}}\exp\left( - \frac{t_{n}}{T_{2}} \right)\]

Grazie alla proprietà distributiva del prodotto, si ha:

\[M_{\bot}(\vartheta) \simeq \frac{M_{0}\vartheta}{\frac{1 - E_{1}}{1 - E_{1}} + \frac{E_{1}}{1 - E_{1}}\frac{\vartheta^{2}}{2}}\exp\left( - \frac{t_{n}}{T_{2}} \right)\]

Per cui:

\[M_{\bot}(\vartheta) \simeq \frac{M_{0}\vartheta}{1 + \frac{E_{1}}{1 - E_{1}}\frac{\vartheta^{2}}{2}}\exp\left( - \frac{t_{n}}{T_{2}} \right)\]

Se il tempo di ripetizione è molto minore del tempo di rilassamento longitudinale \(T_{1}\) è possibile sviluppare \(E_{1}\) in serie di Taylor:

\[E_{1} \simeq 1 - \frac{T_{R}}{T_{1}}\]

Sostituendo, si ottiene:

\[M_{\bot}(\vartheta) \simeq \frac{M_{0}\vartheta}{1 + \frac{1 - \frac{T_{R}}{T_{1}}}{1 - \left( 1 - \frac{T_{R}}{T_{1}} \right)}\frac{\vartheta^{2}}{2}}\exp\left( - \frac{t_{n}}{T_{2}} \right) =\]

Si considera solamente il denominatore. Svolgendo le operazioni di somme:

\[1 + \frac{1 - \frac{T_{R}}{T_{1}}}{1 - \left( 1 - \frac{T_{R}}{T_{1}} \right)}\frac{\vartheta^{2}}{2} = 1 + \frac{1 - \frac{T_{R}}{T_{1}}}{1 - 1 + \frac{T_{R}}{T_{1}}}\frac{\vartheta^{2}}{2} = 1 + \frac{1 - \frac{T_{R}}{T_{1}}}{\frac{T_{R}}{T_{1}}}\frac{\vartheta^{2}}{2} =\]

Da cui:

\[= 1 + \left( \frac{T_{1}}{T_{R}} - 1 \right)\frac{\vartheta^{2}}{2}\]

Per l'ipotesi \(T_{R} \ll T_{1}\) risulta che \(T_{1} \gg T_{R}\), per cui è possibile trascurare l'unità:

\[1 + \left( \frac{T_{1}}{T_{R}} - 1 \right)\frac{\vartheta^{2}}{2} \simeq 1 + \frac{T_{1}}{T_{R}}\frac{\vartheta^{2}}{2}\]

Si è visto che:

\[\vartheta_{E} \simeq \sqrt{2\frac{T_{R}}{T_{1}}}\]

Elevando al quadrato e invertendo la relazione, si ha:

\[\vartheta_{E}^{- 2} \simeq \frac{1}{2}\frac{T_{1}}{T_{R}}\]

Sostituendo nel denominatore della magnetizzazione trasversale si ha:

\[1 + \frac{T_{1}}{T_{R}}\frac{\vartheta^{2}}{2} \simeq 1 + \frac{\vartheta^{2}}{\vartheta_{E}^{2}}\]

La componente trasversale del vettore di magnetizzazione si scrive come:

\[M_{\bot}(\vartheta) \simeq \frac{M_{0}\vartheta}{1 + \frac{\vartheta^{2}}{\vartheta_{E}^{2}}}\exp\left( - \frac{t_{n}}{T_{2}} \right) =\]

Nelle ipotesi \(T_{R} \ll T_{1}\) e \(\vartheta \ll 1\), la magnetizzazione trasversale dipende dalla magnetizzazione longitudinale all'equilibrio \(M_{0}\) e dal flip angle \(\vartheta\). Fissato quest'ultimo parametro, il segnale registrato dipende linearmente da \(M_{0}\), di conseguenza l'immagine ricostruita presenta un contrasto pesato in densità protonica.

In definitiva, con la sequenza short-T\_R steady state, se il flip angle è prossimo al valore di Ernst l'immagine risulta pesata in \(T_{1}\), mentre per angoli \(\vartheta \ll 1\) si ottiene una pesatura in densità protonica.

\subsection{Tempo necessario per raggiungere lo steady state}\label{tempo-necessario-per-raggiungere-lo-steady-state}

Si vuole valutare il tempo necessario affinché si raggiunga lo stato stazionario. Si parte dalla componente longitudinale del vettore di magnetizzazione un istante prima dell'applicazione dell'impulso \(n + 1\)-esimo

\[M_{z}\left( (n + 1)T_{R}^{-} \right) = M_{z}\left( nT_{R}^{-} \right)E_{1}\cos\vartheta + M_{0}\left( 1 - E_{1} \right)\]

Questa espressione è ricorsiva, nel senso che la componente longitudinale all'istante \((n + 1)T_{R}^{-}\) dipende dall'istante \(nT_{R}^{-}\), che a sua volta dipende dalla magnetizzazione longitudinale all'istante \((n - 1)T_{R}^{-}\) e così via. In particolare, all'istante \(t = nT_{R}^{-}\) è possibile scrivere:

\[M_{z}\left( nT_{R}^{-} \right) = M_{z}\left( (n - 1)T_{R}^{-} \right)E_{1}\cos\vartheta + M_{0}\left( 1 - E_{1} \right)\]

A sua volta \(M_{z}\left( (n - 1)T_{R}^{-} \right)\) dipende dall'istante temporale \((n - 2)T_{R}^{-}\), secondo la relazione:

\[M_{z}\left( (n - 1)T_{R}^{-} \right) = M_{z}\left( (n - 2)T_{R}^{-} \right)E_{1}\cos\vartheta + M_{0}\left( 1 - E_{1} \right)\]

Si sostituisce questo risultato nella relazione per \(M_{z}\left( nT_{R}^{-} \right)\):

\[M_{z}\left( nT_{R}^{-} \right) = M_{z}\left( (n - 1)T_{R}^{-} \right)E_{1}\cos\vartheta + M_{0}\left( 1 - E_{1} \right) = \left( M_{z}\left( (n - 2)T_{R}^{-} \right)E_{1}\cos\vartheta + M_{0}\left( 1 - E_{1} \right) \right)E_{1}\cos\vartheta + M_{0}\left( 1 - E_{1} \right) =\]

Svolgendo i prodotti si ottiene:

\[= \left( E_{1}\cos\vartheta \right)^{2}M_{z}\left( (n - 2)T_{R}^{-} \right) + M_{0}\left( 1 - E_{1} \right)E_{1}M_{z}\left( (n - 2)T_{R}^{-} \right)\cos\vartheta + M_{0}\left( 1 - E_{1} \right)\]

Nell'espressione appena individuato è presente una potenza di \(E_{1}\cos\vartheta\), pesato per il valore della magnetizzazione un attimo prima dell'impulso \((n - 2)\)-esimo.

Iterando la ricorsione si può dimostrare una relazione che tiene conto del punto iniziale in cui è applicato il primo impulso:

\[M_{z}(n) = \sum_{l = 0}^{n - 1}\left\lbrack \left( E_{1}\cos\vartheta \right)^{l}\left( 1 - E_{1} \right)M_{0} \right\rbrack + M_{0}\left( E_{1}\cos\vartheta \right)^{n}\]

Dove la sommatoria tiene conto dei contributi \(M_{0}\left( 1 - E_{1} \right)E_{1}\cos\vartheta + M_{0}\left( 1 - E_{1} \right)\), mentre la potenza \(n\)-esima generalizza il termine \(\left( E_{1}\cos\vartheta \right)^{2}\).

La relazione \(M_{z}(n)\) descrive come varia la magnetizzazione in funzione del numero di impulsi \(n\) a partire dai valori iniziali. Essendo una sommatoria geometrica, essa converge. È possibile scrivere la somma della serie:

\[M_{z}(n) = \sum_{l = 0}^{n - 1}\left\lbrack \left( E_{1}\cos\vartheta \right)^{l}\left( 1 - E_{1} \right)M_{0} \right\rbrack + M_{0}\left( E_{1}\cos\vartheta \right)^{n} = M_{0}\left( 1 - E_{1} \right)\frac{1 - \left( E_{1}\cos\vartheta \right)^{n}}{1 - E_{1}\cos\vartheta} + M_{0}E_{1}^{2n}\]

\begin{figure}
\centering
\includegraphics[width=6.525in,height=3.55556in,alt={Immagine che contiene testo, diagramma, linea, Carattere Il contenuto generato dall\textquotesingle IA potrebbe non essere corretto.}]{media/13_FastImm/image338.pdf}\caption{Figura .: Andamento della magnetizzazione longitudinale all\textquotesingle aumentare del numero di ripetizioni}
\end{figure}

La relazione può essere semplificata se valutata per un flip angle uguale a quello di Ernst, \(\vartheta = \vartheta_{E}\):

\[\left. \ M_{z}(n) \right|_{\vartheta = \vartheta_{E}} = M_{0}\left( 1 - E_{1} \right)\frac{1 - \left( E_{1}\cos\vartheta_{E} \right)^{n}}{1 - E_{1}\cos\vartheta_{E}} + M_{0}E_{1}^{2n}\]

Dove \(\vartheta_{E} = \arccos\left( E_{1} \right)\). Sostituendo questa relazione si ottiene:

\[\left. \ M_{z}(n) \right|_{\vartheta = \vartheta_{E}} = M_{0}\left( 1 - E_{1} \right)\frac{1 - \left( E_{1}\cos\left( \arccos\left( E_{1} \right) \right) \right)^{n}}{1 - E_{1}\cos\left( \arccos\left( E_{1} \right) \right)} + M_{0}E_{1}^{2n}\]

Per cui:

\[\left. \ M_{z}(n) \right|_{\vartheta = \vartheta_{E}} = M_{0}\left( 1 - E_{1} \right)\frac{1 - \left( E_{1}^{2} \right)^{n}}{1 - E_{1}^{2}} + M_{0}E_{1}^{2n}\]

Il termine \(1 - E_{1}^{2}\) può essere scomposto in \(\left( 1 - E_{1} \right)\left( 1 + E_{1} \right)\). \(\left( 1 - E_{1} \right)\) piò essere semplificato, ottenendo:

\[\left. \ M_{z}(n) \right|_{\vartheta = \vartheta_{E}} = M_{0}\frac{1 - E_{1}^{2n}}{1 + E_{1}} + M_{0}E_{1}^{2n}\]

Per valutare lo stato stazionario si passa al limite per \(n \rightarrow \infty\) di questa lezione:

\[\lim_{n \rightarrow \infty}\left. \ M_{z}(n) \right|_{\vartheta = \vartheta_{E}} = \lim_{n \rightarrow \infty}\left( M_{0}\frac{1 - E_{1}^{2n}}{1 + E_{1}} + M_{0}E_{1}^{2n} \right)\]

I termini

\[E_{1}^{2n} = \exp\left( - \frac{2nT_{R}}{T_{1}} \right) \rightarrow 0,n \rightarrow \infty\]

Per cui si ottiene:

\[\lim_{n \rightarrow \infty}\left. \ M_{z}(n) \right|_{\vartheta = \vartheta_{E}} = \lim_{n \rightarrow \infty}\left( M_{0}\frac{1 - E_{1}^{2n}}{1 + E_{1}} + M_{0}E_{1}^{2n} \right) = M_{0}\frac{1}{1 + E_{1}} = M_{ze}\]

Questo risultato coincide con il valore della magnetizzazione longitudinale allo steady state.

L'errore relativo commesso nel confondere il valore della magnetizzazione un attimo prima dell'applicazione dell'impulso \((n + 1)\)-esimo con il valore allo steady state, con un flip angle uguale a quello di Ernst, è definito come:

\[\alpha = \frac{M_{z}\left( n,\vartheta_{E} \right) - M_{ze}\left( \vartheta_{E} \right)}{M_{ze}\left( \vartheta_{E} \right)}\]

Dove:

\[M_{ze}\left( \vartheta_{E} \right) = \frac{M_{0}}{1 + E_{1}}\]

\[M_{z}\left( n,\vartheta_{E} \right) = M_{0}\frac{1 - E_{1}^{2n}}{1 + E_{1}} + M_{0}E_{1}^{2n}\]

In altre parole, \(\alpha\) è l'errore relativo commesso nel confondere il valore della magnetizzazione longitudinale dopo \(n\) impulsi col valore del limite per \(n \rightarrow \infty\), ovvero con la magnetizzazione allo steady state. Sostituendo le espressioni per \(M_{ze}\left( \vartheta_{E} \right)\) e \(M_{z}\left( n,\vartheta_{E} \right)\) si ottiene:

\[\alpha = \frac{M_{z}\left( n,\vartheta_{E} \right) - M_{ze}\left( \vartheta_{E} \right)}{M_{ze}\left( \vartheta_{E} \right)} = \frac{M_{0}\frac{1 - E_{1}^{2n}}{1 + E_{1}} + M_{0}E_{1}^{2n} - \frac{M_{0}}{1 + E_{1}}}{\frac{M_{0}}{1 + E_{1}}}\]

Si considera il numeratore e si applica la proprietà distributiva del prodotto:

\[M_{0}\frac{1 - E_{1}^{2n}}{1 + E_{1}} + M_{0}E_{1}^{2n} - \frac{M_{0}}{1 + E_{1}} = \frac{M_{0}}{1 + E_{1}} - \frac{M_{0}E_{1}^{2n}}{1 + E_{1}} + M_{0}E_{1}^{2n} - \frac{M_{0}}{1 + E_{1}} =\]

Semplificando, si ottiene:

\[= M_{0}E_{1}^{2n} - \frac{M_{0}E_{1}^{2n}}{1 + E_{1}} =\]

Raccogliendo i termini comuni, si ottiene:

\[= M_{0}E_{1}^{2n}\left( 1 - \frac{1}{1 + E_{1}} \right) =\]

Applicando il minimo comune multiplo si ottiene:

\[= M_{0}E_{1}^{2n}\left( \frac{1 + E_{1} - 1}{1 + E_{1}} \right) = M_{0}E_{1}^{2n}\left( \frac{E_{1}}{1 + E_{1}} \right)\]

Sostituendo questo risultato nell'espressione dell'errore relativo, \(\alpha\), si ottiene:

\[\alpha = \frac{M_{0}\frac{1 - E_{1}^{2n}}{1 + E_{1}} + M_{0}E_{1}^{2n} - \frac{M_{0}}{1 + E_{1}}}{\frac{M_{0}}{1 + E_{1}}} = \frac{M_{0}E_{1}^{2n}\left( \frac{E_{1}}{1 + E_{1}} \right)}{\frac{M_{0}}{1 + E_{1}}}\]

Semplificando i termini comuni tra numeratore e denominatore, ottiene l'espressione per l'errore relativo:

\[\alpha = E_{1}^{2n + 1}\]

Se si vuole un certo errore \(\alpha\) è necessario applicare un numero \(n_{\alpha}\) di impulsi, ottenuto invertendo la relazione per \(\alpha\):

\[\alpha = E_{1}^{2n_{\alpha} + 1}\]

Ricordando la definizione di \(E_{1}\), si ottiene la relazione:

\[\alpha = \left( \exp\left( - \frac{T_{R}}{T_{1}} \right) \right)^{2n_{\alpha} + 1}\]

Passando ai logaritmi, si ricava un'espressione lineare per \(n_{\alpha}\):

\[\log\alpha = - \frac{T_{R}}{T_{1}}\left( 2n_{\alpha} + 1 \right)\]

Risolvendo per \(n_{\alpha}\) si ottiene il tempo richiesto per raggiungere l'errore \(\alpha\) desiderato:

\[n_{\alpha} = - \frac{1}{2}\frac{T_{2}}{T_{R}}\log\alpha - \frac{1}{2}\]

Se, ad esempio, si vuole commettere un errore del \(10\%\) sul grasso con un impulso a radiofrequenza all'angolo di Ernst, è necessario applicare \(8\) impulsi, a \(T_{R} = 40\ ms\).

Dalla relazione \(n_{\alpha}\) si evince che il numero di impulsi è strettamente legato al tessuto da analizzare. La materia bianca (white matter), essendo più liquida del grasso, possiede un tempo di rilassamento longitudinale \(T_{1}\) maggiore del grasso. Con un tempo di ripetizione di \(40\ ms\) è necessario applicare \(18\) impulsi.

Se, infine, si vuole un'approssimazione maggiore, come a \(1\%\)m per il grasso bisogna applicare \(16\) impulsi, mentre per la materia bianca \(36\) impulsi.

In definitiva, se si vuole un'ottima approssimazione è necessario applicare un certo numero di impulsi affinché si possa ritenere raggiunto lo steady state.

Generalmente, il prodotto \(n_{\alpha}T_{R} \simeq T_{1}\), se \(\alpha\) è dell'ordine del \(10\%\). Di conseguenza, per avere un errore percentuale \(\alpha\) è necessario applicare un numero di sequenze \(n_{\alpha}\), che occupano un intervallo temporale dello stesso ordine di grandezza di \(T_{1}\) quindi dell'ordine del secondo. Successivamente, l'acquisizione risulta essere più rapida rispetto a una classica sequenza di acquisizione.

\subsection[Vantaggi e svantaggi della sequenza short-TR steady state]{Vantaggi e svantaggi della sequenza short-$\mathbf{T}_{\mathbf{R}}$ steady state}
\label{vantaggi-e-svantaggi-short-TR}

Lo svantaggio della sequenza risiede nel dover aspettare un tempo dato da un multiplo di \(T_{1}\), in base all'approssimazione desiderata, prima di eseguire l'imaging.

Questa sequenza è particolarmente utile nell'imaging tridimensionale, in cui i tempi sono molto lunghi. La sequenza, infatti, permette di ridurre il tempo di ripetizione \(T_{R}\) permettendo un vantaggio temporale.

Si osserva, infine, che, nel caso generale, la magnetizzazione longitudinale è data da:

\[M_{z}(n) = M_{0}\left( 1 - E_{1} \right)\frac{1 - \left( E_{1}\cos\vartheta \right)^{n}}{1 - E_{1}\cos\vartheta} + M_{0}E_{1}^{2n}\]

Riducendo l'angolo di ribaltamento si ottiene un'attenuazione del segnale ricevuto; tuttavia, ciò non inficia la visualizzazione dei tessuti in quanto il contrasto tra due tessuti non viene perso, anzi, la sequenza permette di enfatizzare un tessuto in base al tempo di rilassamento longitudinale o densità protonica.

Il vantaggio principale della sequenza è strettamente legato al tempo di ripetizione di circa \(40\ ms\). Ciò permette di ridurre gli artefatti da movimento del paziente all'interno del gantry.

Questa sequenza è molto utilizzata nella pratica poiché permette di ottenere un elevato contrasto tra i tessuti con un tempo ridotto.

\subsection[Applicazione della short-TR incoherent gradient-echo]{Applicazione della short-$\mathbf{T}_{\mathbf{R}}$ incoherent gradient-echo}
\label{applicazione-short-TR-incoherent-gradient-echo}

La risonanza magnetica può essere utilizzata anche per eseguire la mammografia. In questo contesto, si orienta la paziente in modo che le mammelle siano posizionate all'interno delle antenne di ricezione, dette bobine bilaterali. La paziente è posta in posizione prona all'interno del gantry. Di solito l'imaging è tridimensionale e sfrutta una sequenza di tipo short-\(T_{R}\) inchoerent gradient-echo in cui le mammelle sono sezionate secondo piani coronarici.

L'asse di frequency encoding è solitamente diretto lungo la direzione minore della mammella, campionata con circa \(96\) punti; mentre il phase encoding è campionato con un numero di punti uguale a \(128\) punti. Con \(96\) campioni è necessario ricorrere ad algoritmi iterativi sopprimere il riempimento parziale del \(k\)-spazio. L'asse longitudinale \(z\) resta invariato a patto di modificare la direzione positiva.

In base al differente flip angle si ottengono diversi contrasti: se l'angolo \(\vartheta\) è piccolo, la pesatura è in densità protonica; all'aumentare del flip angle si passa a una pesatura in \(T_{1}\).

Il tessuto fibroghiandolare è più liquido del tessuto circostante, quindi, presenta un valore di \(T_{1}\) molti longo. All'aumentare dell'angolo di ribaltamento, il segnale proveniente da questo tessuto tende a essere attenuato, apparendo più scuro nell'immagine finale; viceversa, il grasso risulta essere molto brillante. Le lesioni mammarie dovrebbero avere un tempo di rilassamento longitudinale intermedio tra il tessuto fibroghiandolare e il tessuto adiposo, dunque, appaiono con gradazione di grigio diverse dai due tessuti, per immagini pesate in \(T_{1}\)

La pesatura, a parità di rapporto segnale/rumore influenza l'imaging, quindi, modifica il rapporto contrasto/rumore, poiché cambia la visibilità nei tessuti.

\subsection[Stima del tempo T1 mediante variazioni del flip angle]{Stima del tempo $\mathbf{T}_{\mathbf{1}}$ mediante variazioni del flip angle}
\label{stima-tempo-T1-flip-angle}

L'intensità del segnale del pixel dipende dal flip angle, \(s(\vartheta)\). IN una sequenza basata su short-\(T_{R}\) incoherent gradient-echo, il segnale del voxel è data dalla relazione:

\[s\left( \vartheta,T_{E} \right) = \rho_{0}\sin\vartheta\frac{1 - E_{1}}{1 - E_{1}\cos\vartheta}\exp\left( - \frac{T_{E}}{T_{2}^{*}} \right)\]

Per stimare il tempo di rilassamento longitudinale è necessario acquisire più immagini con flip angle diversi e applicare un algoritmo di fitting come least square.

Si suppone che un voxel contenga un solo tessuto, quindi il segnale registrato dipende solo dalle sue caratteristiche biochimiche.

Eseguendo l'imaging con differenti flip angle si ottengono diversi segnali del voxel, distribuiti nel piano \(\widehat{\rho}(\vartheta) - \vartheta\). È possibile determinare i parametri del tessuto sotto analisi noti \(n\) punti sperimentali e la curva teorica che lega il segnale del voxel con il flip angle. In altre parole, si vuole determinare la curva di parametri \(\rho_{0}\), \(T_{1}\) e \(T_{2}\) che meglio descrive la distribuzione dei dati sperimentali.

\begin{figure}
\centering
\includegraphics[width=6.69306in,height=4.63333in,alt={Immagine che contiene testo, linea, schermata, Diagramma Il contenuto generato dall\textquotesingle IA potrebbe non essere corretto.}]{media/13_FastImm/image339.pdf}\caption{Figura .: Distribuzione dei punti sperimentali e curva di fit teorica}
\end{figure}

Si osservi che per una sequenza short-\(T_{R}\), il tempo di ripetizione \(T_{R}\) è molto minore del tempo di rilassamento legato alle disomogeneità di campo, \(T_{2}^{*}\), dunque, il termine esponenziale legato a questi due tempi tende all'unità:

\[E_{2} = \exp\left( - \frac{T_{R}}{T_{2}} \right) \simeq 1,T_{R} \ll T_{2}\]

Il segnale del voxel, contenente il tessuto di interesse, può essere approssimato come:

\[s(\vartheta) \simeq \rho_{0}\sin\vartheta\frac{1 - E_{1}}{1 - E_{1}\cos\vartheta}\]

La risoluzione del problema sfrutta la minimizzazione dell'errore quadratico medio tra la curva \(\widehat{\rho}(\vartheta)\), con i parametri \(\rho_{0}\), \(T_{1}\) e \(T_{2}\) da valutare, e i punti sperimentalmente misurati. Il problema della determinazione dei parametri è molto simile a una regressione lineare di tipo OLS (\emph{Ordinary Least Square}); tuttavia, il legame tra i parametri del tessuto e il flip angle è non lineare, dunque, si parla di NLS (Non-linear least squares), di più difficile risoluzione.

\subsubsection{Risoluzione numerica del metodo NLS}\label{risoluzione-numerica-del-metodo-nls}

Al fine di applicare l'algoritmo di elaborazione digitale per la valutazione dei parametri del segnale del voxel, si acquisiscono più immagini DICOM del tessuto di interesse, come il tessuto fibroghiandolare.

clear all\\
close all\\
\strut \\
percorso = \textquotesingle C:\textbackslash Users\textbackslash ausil\textbackslash OneDrive - Università di Napoli Federico II\textbackslash Università\textbackslash Ing. Biomedica\textbackslash Magistrale\textbackslash I Anno\textbackslash Strumentazione Avanzata per la Diagnosi e Terapia\textbackslash Matlab\textbackslash MRI\textbackslash MAMMELLA\_VARI\_FA\textbackslash breast\_vari\_FA\textbackslash\textquotesingle;\\
handleFigure = 0;\\
\strut \\
\% leggo una fetta centrale\\
\% tutti i flip angle\\
\strut \\
lista = dir({[}percorso \textquotesingle*.dcm\textquotesingle{]});\\
\strut \\
for k = 1:length(lista)\\
info = dicominfo({[}lista(k).folder \textquotesingle/\textquotesingle{} lista(k).name{]});\\
FA(k) = info.FlipAngle;\\
pos(k,:) = info.ImagePositionPatient;\\
end\\
\strut \\
posizioni = unique(pos(:,2));\\
indici = find(pos(:,2)==posizioni(35));\\
\strut \\
handleFigure = handleFigure + 1;\\
hf = figure(handleFigure);\\
for k = 1:3\\
IM(:,:,k) = dicomread({[}lista(indici(k)).folder \textquotesingle/\textquotesingle{} lista(indici(k)).name{]});\\
subplot(2,2,k)\\
imagesc(IM(:,:,k))\\
colormap(gray)\\
axis equal\\
axis off\\
end

\begin{figure}
\centering
\includegraphics[width=4.625in,height=3.16667in]{media/13_FastImm/image340.pdf}\caption{Figura .: Immagini mammografiche acquisite a diversi flip angle (FA) pesate in \(T_{1}\)}
\end{figure}

Per stimare i parametri, si utilizza la tecnica dei minimi quadrati non lineari (NLS), implementata in MATLAB dalla funzione lsqcurvefit. Il metodo mediante software considera nell'indentificare una ROI (Region Of Interest) sulla quale identificare i parametri di interesse. La ROI viene selezionata tramite una maschera che assume valore unitario all'interno della ragione contenente il tessuto di interesse e nullo all'esterno.

\% seleziono una regione di cui fare il grafico

bw = imfreehand(gca);

BW = bw.createMask;

handleFigure = handleFigure + 1;

hf = figure(handleFigure);

imshow(BW)

\begin{figure}
\centering
\includegraphics[width=2.175in,height=1.16667in]{media/13_FastImm/image341.pdf}\caption{Figura .: ROI della mammella}
\end{figure}

La funzione lsqcurvefit riceve in ingresso la funzione teorica che lega i parametri cercati con i dati misurati, il valore iniziale della stima dei parametri cercati, i flip angle sui quali è calcolata la curva e le misure del segnale del voxel ottenute sperimentalmente.

close all\\
clear E1est\\
TR = 0.0098;\\
FUN = @(p,teta) p(2)*sin(teta) * (1-p(1)) ./ (1-p(1)*cos(teta));\\
\strut \\
xdata = FA(indici)*pi/180;\\
x0 = {[}exp(-TR/0.9) 80{]}; \% un T1 iniziale\\
\strut \\
opzioni = optimset(\textquotesingle Display\textquotesingle,\textquotesingle off\textquotesingle);\\
\strut \\
hw = waitbar(0,\textquotesingle...\textquotesingle);\\
Nt = size(dum,1);\\
for k = 1:Nt\\
waitbar(k/Nt,hw);\\
ydata = double(dum(k,:));\\
{[}E1est(k,:),resnorm(k){]} = lsqcurvefit(FUN,x0,xdata,ydata,{[}0 0{]},{[}1 4000{]},opzioni);\\
end\\
close(hw)\\
\strut \\
T1 = -TR./log(E1est(:,1));\\
T1(T1\textgreater2)=NaN; \% valori troppo lunghi sono errori\\
\strut \\
T1v = double(BWv);\\
T1v(BWv==1)=T1;\\
\strut \\
PDv = double(BWv);\\
PDv(BWv==1) = E1est(:,2);

Noti i parametri per un certo numero di pixel, contenuti nella ROI, è possibile costruire una mappa del tempo \(T_{1}\) ed \(\rho_{0}\), in cui maggiore è il valore di questi due parametri e più luminoso è il pixel.

handleFigure = handleFigure + 1\\
hf = figure(handleFigure)\\
subplot(2,3,1)\\
imagesc(reshape(T1v,{[}size(IM,1),size(IM,2){]}))\\
colormap(gray)\\
title(\textquotesingle mappa T1\textquotesingle)\\
axis equal\\
axis off\\
\strut \\
subplot(2,3,4)\\
imagesc(reshape(PDv,{[}size(IM,1),size(IM,2){]}))\\
title(\textquotesingle mappa PD\textquotesingle)\\
axis equal\\
axis off\\
\strut \\
subplot(2,3,2)\\
imagesc(IM(:,:,1))\\
title({[}\textquotesingle flip \textquotesingle{} num2str(FA(indici(1))){]})\\
axis equal\\
axis off\\
\strut \\
subplot(2,3,3)\\
imagesc(IM(:,:,2))\\
title({[}\textquotesingle flip \textquotesingle{} num2str(FA(indici(2))){]})\\
axis equal\\
axis off\\
\strut \\
subplot(2,3,5)\\
imagesc(IM(:,:,3))\\
title({[}\textquotesingle flip \textquotesingle{} num2str(FA(indici(2))){]})\\
axis equal\\
axis off

\begin{figure}
\centering
\includegraphics[width=4.66667in,height=3.44167in,alt={Immagine che contiene schermata, design Il contenuto generato dall\textquotesingle IA potrebbe non essere corretto.}]{media/13_FastImm/image342.pdf}\caption{Figura .: Immagini \(T_{1}\)-pesate e \(\rho_{0}\) pesate}
\end{figure}

Le immagini ottenute con questa procedura non sono indipendenti tra loro, infatti, le regioni per cui il tempo di rilassamento longitudinale è elevato risultano essere più scure, essendo l'immagine pesata in \(T_{1}\). Si ottiene, in definitiva, un'immagine in cui le regioni più luminose sono associate a bassi valori di \(T_{1}\). Siccome il tempo \(T_{1}\) è caratteristico del tessuto, le immagini dipendono strettamente da esso. Ad esempio:

\begin{itemize}
\item
  Acqua pura, liquido cerebrospinale (CSF) e cisti fluide hanno tempi di rilassamento \(T_{1}\) e \(T_{2}\) molto lunghi. Questo significa che appaiono scuri nelle immagini \(T_{1}\)-pesate e molto luminosi (iperintensi) nelle immagini \(T_{2}\)-pesate;
\item
  Il tessuto fibroghiandolare (tipico, ad esempio, del parenchima mammario) ha tempi di rilassamento \(T_{1}\) e \(T_{2}\) intermedi, ovvero più brevi rispetto ai fluidi. Di conseguenza, appare di un segnale grigio intermedio sia nelle immagini \(T_{1}\) che in quelle \(T_{2}\);
\item
  Il tessuto adiposo (grasso) ha un tempo di rilassamento \(T_{1}\) molto breve, il che lo rende molto luminoso (iperintenso) nelle immagini \(T_{1}\)-pesate. Il suo \(T_{2}\) è relativamente breve, simile a quello del tessuto fibroghiandolare, apparendo quindi con un segnale grigio intermedio o leggermente più luminoso del tessuto fibroghiandolare nelle immagini \(T_{2}\)-pesate.
\item
  Lesioni neoplastiche (cancro) spesso presentano un'importante componente edematosa (gonfiore dovuto all'accumulo di fluidi) e un aumento della cellularità. L\textquotesingle edema aumenta i tempi di rilassamento \(T_{1}\) e \(T_{2}\) della lesione, rendendola generalmente più scura del tessuto fibroghiandolare circostante in immagini pesate in \(T_{1}\) e molto più luminosa (iperintensa) in \(T_{2}\). È per questo motivo che le immagini \(T_{2}\)-pesate sono spesso utilizzate per evidenziare le lesioni.
\end{itemize}

Nell'algoritmo digitale, maggiore è il numero di immagini acquisite con flip angle diversi e migliore è la valutazione dei parametri della curva di interesse; tuttavia, questa richiesta è in conflitto con la necessità di mantenere bassi i tempi di acquisizione. Nella pratica clinica non è detto che si acquisicano varie immagini con flip angle diversi. Generalmente si acquisisce una sola immagine e, se necessario, più immagini con flip angle diversi.

\subsection{Risoluzione analitica del problema NLS}\label{risoluzione-analitica-del-problema-nls}

Dal punto di vista analitico non esiste una soluzione in forma chiusa per il problema non-linear least square. Per ottenere una soluzione si esegue un algoritmo iterativo in cui individua il set di parametri che minimizza lo scarto quadratico medio.

Dato un set di misure \(y_{1},y_{2},\ldots,y_{n}\) e una serie di parametri incogniti \(x_{1},x_{2},\ldots,x_{n}\), si definiscono il vettore delle misure \(\overset{\underline{}}{y}\) e delle incognite, \(\overset{\underline{}}{x}\) come:

\[\overset{\underline{}}{y} = \left( \begin{array}{r}
y_{1} \\
y_{2} \\
 \vdots \\
y_{n}
\end{array} \right),\overset{\underline{}}{x} = \left( \begin{array}{r}
x_{1} \\
x_{2} \\
 \vdots \\
x_{n}
\end{array} \right)\]

Il problema consiste nel minimizzare lo scarto quadratico medio tra i valori misurati \(\overset{\underline{}}{y}\) e i punti teorici \(\overset{\underline{}}{f}\left( \overset{\underline{}}{x} \right)\), dipendenti dal vettore delle incognite, \(\overset{\underline{}}{x}\):

\[\min_{\overset{\underline{}}{x}}\left\| \overset{\underline{}}{y} - \overset{\underline{}}{x} \right\|^{2}\]

Il valor quadratico medio può essere espresso come sommatoria degli scarti quadratici tra le misure effettivamente eseguite e le previsioni teoriche:

\[\left\| \overset{\underline{}}{y} - \overset{\underline{}}{x} \right\|^{2} = \sum_{k}^{}\left( y_{k} - f\left( x_{k} \right) \right)^{2}\]

L'errore commesso nella minimizzazione del valor quadratico medio è detto residuo.

Per minimizzare l'errore quadratico medio si considera un vettore dei punti di misura \({\overset{\underline{}}{x}}^{0}\), assunto come punto di partenza dell'algoritmo, e si esegue un algoritmo iterativo, che aggiorna il vettore delle incognite al fine di raggiungere il minimo. Gli algoritmi dipendono dalla funzione teorica \(f\) considerata. I più famosi sono:

\begin{itemize}
\item
  \textbf{Il metodo di Newton:} sfrutta le derivate prime e seconde (matrice Hessiana) della funzione per trovare il minimo. È molto veloce se la stima iniziale è buona, ma può essere computazionalmente costoso e instabile;
\item
  \textbf{Il metodo di Gauss-Newton:} Una variante del metodo di Newton specifica per i problemi di minimi quadrati. Evita il calcolo della matrice Hessiana completa, approssimandola con lo Jacobiano. È più efficiente di Newton ma meno robusto.
\item
  \textbf{Il metodo di Levenberg-Marquardt:} Il più robusto dei tre, è un ibrido tra il metodo di Gauss-Newton e la discesa del gradiente. Adatta dinamicamente il suo comportamento per garantire la convergenza, anche con stime iniziali non precise. Per questo motivo, è il metodo più utilizzato per il fitting di curve non lineari.
\item
  \textbf{Variable Propagation (Propagazione delle Variabili):} Non è un algoritmo completo di per sé, ma una tecnica per semplificare i problemi di ottimizzazione. Separa i parametri lineari da quelli non lineari, riducendo la complessità del problema e migliorando l\textquotesingle efficienza e la stabilità del processo di fit.
\end{itemize}

Al termine del processo iterativo si ottiene una stima \({\overset{\underline{}}{x}}^{est}\) il più vicino possibile ai parametri che hanno generato le misure \(\overset{\underline{}}{y}\). Al fine di ottenere una stima fisicamente possibile, è necessario fornire dei vincoli, come limite superiore e inferiore dei parametri. Ad esempio, la densità protonica relativa \(\rho_{0}\) può assumere valori tra \(0\) e \(1\), con valore unitario in caso di acqua; oppure, il tempo di rilassamento longitudinale non può essere maggiore di \(2\ s\).

È sempre necessario valutare tutti i risultati ottenuti poiché potrebbero non avere nessun senso fisico. Ciò si verifica in presenza di dati misurati molto rumorosi.

In ambito clinico è spesso necessario avere solamente un'informazione qualitativa del tempo \(T_{1}\), ottenuta mediante pesatura in questo parametro. In altri contesti, è necessario valutare quantitativamente il tempo di rilassamento longitudinale. Solo in queste situazioni si procede con la risoluzione del problema non-linear least square e con la successiva generazione della mappa \(T_{1}\).

\subsubsection[Linearizzazione del problema NLS per stima di rho0 e T1]{Linearizzazione del problema NLS per stima di $\mathbf{\rho}_{\mathbf{0}}$ e $\mathbf{T}_{\mathbf{1}}$}
\label{linearizzazione-NLS-stima-rho0-T1}

In alcuni casi è possibile linearizzare la minimizzazione degli scarti quadratici al fine di ridurre un problema NLS in OLS.

Il segnale del voxel, nel caso di sequenza short-\(T_{R}\) dipende, secondo una legge nota, dal flip-angle (FA), densità protonica e tempo di rilassamento longitudinale \(T_{1}\) del tessuto di cui è composto il voxel:

\[s\left( \vartheta,T_{1},\rho_{0} \right) = \rho_{0}\sin\vartheta\frac{1 - E_{1}}{1 - E_{1}\cos\vartheta}\]

Si vuole linearizzare la dipendenza del segnale \(s\) dai parametri \(E_{1}\) e \(\rho_{0}\); a tale scopo si moltiplica ambo i membri per \(1 - E_{1}\cos\vartheta\):

\[\left( 1 - E_{1}\cos\vartheta \right)s\left( \vartheta,T_{1},\rho_{0} \right) = \left( 1 - E_{1} \right)\rho_{0}\sin\vartheta\]

Si divide anche per \(\sin\vartheta\):

\[\left( 1 - E_{1}\cos\vartheta \right)\frac{s\left( \vartheta,T_{1},\rho_{0} \right)}{\sin\vartheta} = \left( 1 - E_{1} \right)\rho_{0}\]

Si svolgono i prodotti al primo membro:

\[\frac{s\left( \vartheta,T_{1},\rho_{0} \right)}{\sin\vartheta} - E_{1}\cos\vartheta\frac{s\left( \vartheta,T_{1},\rho_{0} \right)}{\sin\vartheta} = \left( 1 - E_{1} \right)\rho_{0}\]

Il rapporto tra il seno e il coseno coincide con la tangente dell'angolo, per cui:

\[\frac{s\left( \vartheta,T_{1},\rho_{0} \right)}{\sin\vartheta} - E_{1}s\left( \vartheta,T_{1},\rho_{0} \right)\frac{\cos\vartheta}{\sin\vartheta} = \left( 1 - E_{1} \right)\rho_{0}\]

Dove:

\[\frac{\cos\vartheta}{\sin\vartheta} = \cot\vartheta = \frac{1}{\tan\vartheta}\]

Da cui:

\[\Longleftrightarrow \frac{s\left( \vartheta,T_{1},\rho_{0} \right)}{\sin\vartheta} - E_{1}\frac{s\left( \vartheta,T_{1},\rho_{0} \right)}{\tan\vartheta} = \left( 1 - E_{1} \right)\rho_{0}\]

Si porta al secondo membro i termini dipendenti da \(E_{1}\):

\[\frac{s\left( \vartheta,T_{1},\rho_{0} \right)}{\sin\vartheta} = \left( 1 - E_{1} \right)\rho_{0} + E_{1}\frac{s\left( \vartheta,T_{1},\rho_{0} \right)}{\tan\vartheta}\]

Rispetto al flip angle la quantità \(\left( 1 - E_{1} \right)\rho_{0}\) è costante, per cui può essere indicata semplicemente con \(c\):

\[c = \left( 1 - E_{1} \right)\rho_{0}\]

Se si pone:

\[\left\{ \begin{matrix}
y = \frac{s\left( \vartheta,T_{1},\rho_{0} \right)}{\sin\vartheta} \\
x = \frac{s\left( \vartheta,T_{1},\rho_{0} \right)}{\tan\vartheta}
\end{matrix} \right.\ \]

Si ottiene una relazione di tipo lineare dove coefficiente angolare e intercetta dipendono dai parametri \(E_{1}\) e \(\rho_{0}\):

\[y = E_{1}x + c\]

La relazione del segnale nel voxel in questo modo è stata linearizzata nei confronti di \(E_{1}\), legato a \(T_{1}\), e \(\rho_{0}\).

Per ogni \(k\)-esima sequenza di acquisizione il segnale del voxel (\(s_{k}\)) è noto, poiché misurato sperimentalmente, così come il flip angle (\(\vartheta_{k}\)) impostato dall'esterno. In altre parole, sono note le quantità \(y_{k}\) e \(x_{k}\) noti \(s_{k}\) e \(\vartheta_{k}\):

\[\left\{ \begin{matrix}
y_{k} = \frac{s_{k}}{\sin\vartheta_{k}} \\
x_{k} = \frac{s_{k}}{\tan\vartheta_{k}}
\end{matrix} \right.\ \]

È possibile definire una matrice dei coefficienti \(\overset{\underline{}}{\overset{\underline{}}{X}}\) come:

\[\overset{\underline{}}{\overset{\underline{}}{X}} = \begin{pmatrix}
x_{1} & 1 \\
x_{2} & 1 \\
 \vdots & \vdots \\
x_{n} & 1
\end{pmatrix}\]

È il vettore delle misurazioni \(\overset{\underline{}}{Y}\):

\[\overset{\underline{}}{Y} = \begin{pmatrix}
y_{1} \\
y_{2} \\
 \vdots \\
y_{n}
\end{pmatrix}\]

Il vettore dei parametri da valutare è:

\[\overset{\underline{}}{P} = \begin{pmatrix}
E_{1} \\
c
\end{pmatrix}\]

La relazione può essere espressa in forma matriciale come:

\[\overset{\underline{}}{Y} = \overset{\underline{}}{\overset{\underline{}}{X}}\begin{pmatrix}
E_{1} \\
c
\end{pmatrix}\]

Usando la teoria della minimizzazione dei minimi quadrati o OLS si ottiene una stima dei parametri \(T_{1}\) e \(c\):

\[\overset{\underline{}}{P} = \left( {\overset{\underline{}}{\overset{\underline{}}{X}}}^{T}\overset{\underline{}}{\overset{\underline{}}{X}} \right)^{- 1}{\overset{\underline{}}{\overset{\underline{}}{X}}}^{T}\overset{\underline{}}{Y}\]

La stima ottenuta con il metodo \textbf{NLS} e quella ottenuta con \textbf{OLS} linearizzato non coincidono, ma presentano differenze che evidenziano i limiti della linearizzazione.

La linearizzazione semplifica il calcolo, ma non implica necessariamente una soluzione migliore. L'OLS, in questo caso, produce un algoritmo più semplice per la stima di \(\rho_{0}\) e \(T_{1}\), ma presenta un errore maggiore perché la linearizzazione introduce una distorsione nel rumore dei dati.

Per valutare la precisione di entrambi i metodi, si possono usare provette con concentrazioni note di un mezzo di contrasto. Ogni concentrazione modifica il tempo di rilassamento \(T_{1}\) secondo le \textbf{formule di Solomon-Bloembergen}. Stimando i parametri con OLS e NLS, si può verificare quale stima si avvicina di più al valore teorico.

Si osserva che al diminuire della concentrazione del mezzo di contrasto, l\textquotesingle errore della stima aumenta, rendendo la misurazione più difficile. Questo dimostra che il metodo NLS, non richiedendo la linearizzazione del problema, è più robusto e preciso.

L'uso dei mezzi di contrasto è cruciale per rilevare patologie come le neoplasie. Il liquido di contrasto agisce come un tracciante che evidenzia la distribuzione del sangue, un parametro fondamentale per identificare l\textquotesingle{}\textbf{angiogenesi}, un processo di formazione di nuovi vasi sanguigni spesso associato alla crescita dei tumori.

\begin{figure}
\centering
\includegraphics[width=6.69306in,height=4.25208in,alt={Immagine che contiene testo, schermata, linea, Diagramma Il contenuto generato dall\textquotesingle IA potrebbe non essere corretto.}]{media/13_FastImm/image343.pdf}\caption{Figura .: Andamento dell\textquotesingle errore per NLS e OLS}
\end{figure}

\subsection{Dynamic Contrast-Enhanced (DCE) Magnetic Resonance Imaging (MRI)}\label{dynamic-contrast-enhanced-dce-magnetic-resonance-imaging-mri}

La \textbf{DCE-MRI (Dynamic Contrast-Enhanced Magnetic Resonance Imaging)} è una tecnica di risonanza magnetica che fornisce informazioni dettagliate sulla \textbf{vascolarizzazione} e sull\textquotesingle aggressività delle lesioni tumorali. Questa metodica è ampiamente utilizzata in oncologia perché i tumori, per crescere, promuovono l\textquotesingle angiogenesi, un processo che aumenta l\textquotesingle apporto di sangue e, di conseguenza, la concentrazione locale del mezzo di contrasto.

La tecnica prevede l\textquotesingle acquisizione di una serie di immagini veloci in sequenza temporale. Le immagini vengono acquisite prima, durante e dopo l\textquotesingle iniezione endovenosa di un mezzo di contrasto a base di \textbf{gadolinio}. L\textquotesingle analisi dei dati si basa sulle \textbf{curve intensità-tempo (TIC)}, che misurano come l\textquotesingle intensità del segnale (legata alla concentrazione del mezzo di contrasto) cambia nel tempo all\textquotesingle interno di una specifica \textbf{regione di interesse (ROI)} selezionata sull\textquotesingle immagine. L\textquotesingle andamento di queste curve offre indizi cruciali. In un tessuto tumorale, il grafico mostra un rapido aumento dell\textquotesingle intensità (fase di \textbf{wash-in}) seguito da una diminuzione (fase di \textbf{wash-out}). L\textquotesingle inclinazione di queste fasi dipende dall\textquotesingle aggressività del tumore: lesioni più aggressive, avendo una maggiore vascolarizzazione, mostrano un wash-in più rapido e una velocità di wash-out più elevata.

L'analisi di dati DCE-RMI, con l'ausilio di diversi approcci di elaborazione, è ampiamente utilizzata nello studio dell'angiogenesi tumorale e nello sviluppo di nuovi farmaci in grado di bloccare questo processo di crescita cellulare.

La metodica prevede che il tecnico radiologo o l'operatore selezioni la ROI di cui si vuole studiare la vascolarizzazione. I risultati, dunque, sono operatore dipendente, nel senso che due radiologi potrebbero selezione ROI lievemente diverse, ottenendo stime diverse. Inoltre, il metodo DCE-MRI non fornisce informazioni fisiopatologiche del tessuto di interesse.

Un approccio semi-quantitativo prevede di calcolare opportuni indici descrittivi della curva intensità-tempo o TIC, i quali risultano essere meno sensibili alle variazioni tra i protocolli di acquisizione e meno dipendenti da altri fattori della sequenza.

Si definiscono:

\begin{itemize}
\item
  \textbf{Tempo di arrivo nella vena (VAT):} Il tempo impiegato affinché l\textquotesingle intensità raggiunga il \(20\%\) del suo picco massimo dopo l'iniezione del mezzo di contrasto;
\item
  \textbf{Tempo al picco (TTP):} Il tempo necessario per raggiungere l\textquotesingle intensità di picco (PI), ovvero il punto più alto della curva;
\item
  \textbf{Tempo alla fase di picco (TTPP):} Il tempo per raggiungere il \(90\%\) del picco massimo.
\item
  \textbf{Rapporto di washout (WR):} La differenza tra l'intensità di picco e l'intensità al termine dello studio.
\end{itemize}

\includegraphics[width=4.70581in,height=2.46667in,alt={Schematic illustration of the time-intensity curve (TIC) and measured parameters. Hepatic vein arrival time (HVAT) was the time from contrast agent injection to 20\% of peak intensity (PI, 2 ). Time to peak (TTP) and time-to-peak phase (TTPP) were defined as the times to reach PI and 90\% PI, respectively. Washout ratio (WR) was defined as (PI À the intensity at the end of the study; }]{media/13_FastImm/image344.pdf}
\begin{enumerate}
\def\labelenumi{\arabic{enumi}.}
\setcounter{enumi}{13}
\item
  Figura .: TIC per imaging epatico ( Hepatic vein arrival time (HVAT)
\end{enumerate}

)

La metodica è molto utilizzata in oncologia poiché le neoplasie promuovono l'angiogenesi, aumentando localmente la concentrazione del mezzo di contrasto infuso nel paziente.

In presenza di tumore l'andamento del TIC, legato al tracciante nel voxel tumore, si osserva un primo assorbimento del mezzo di contrasto molto rapido; successivamente vi è un decremento con pendenza dipendente dall'aggressività tumorale. Infatti, i tumori più aggressivi, avendo una maggiore vascolarizzazione, presentano una maggiore velocità di escrezione del mezzo di contrasto. In gergo, la fase di assorbimento è detta wash-in mentre quella di espulsione wash-out. L'ipervascolarizzazione può determinare, inoltre, un picco più elevato.

\begin{figure}
\centering
\includegraphics[width=6.69306in,height=4.08958in,alt={Immagine che contiene testo, linea, Diagramma, diagramma Il contenuto generato dall\textquotesingle IA potrebbe non essere corretto.}]{media/13_FastImm/image345.pdf}\caption{Figura .: Curva TIC in presenza di neoplasia rispetto al tessuto sano}
\end{figure}

Con la metodica DCE-MRI, i tumori sono discriminato sulla base dell'assorbimento del mezzo di contrasto o wash-in e della fase di espulsione wash-out.

In letteratura si parla di quantitative imaging poiché, la metodica DCE-MRI, oltre a fornire un'immagine radiologica, permette di valutare l'aggressività tumorale sulla base dei tempi delle fasi di wash-in e wash-out.

La misura dei tempi è normalmente effettuata nelle applicazioni cliniche al fine di eseguire la diagnosi di neoplasie e seguire il loro sviluppo sotto trattamento antitumorale.

A differenza della CT con mezzo di contrasto o PET, la DCE-MRI può essere eseguita più volte sullo stesso paziente, in tempi ravvicinati, senza rischi per la salute del paziente legati a radiazioni ionizzanti.

Gli algoritmi che permettono la valutazione delle caratteristiche delle TIC devono essere estremamente precisi e semplici da usare, così da ottenere una mappa a pseudocolori indicante la velocità di efflusso e deflusso del contrasto, parametri legati all'aggressività tumorale. Le immagini a pseudocolori dipendono dal tempo di echo, dal tempo di ripetizione, dal flip angle e, ovviamente, dalle dimensioni del voxel.

La tecnica DCE-MRI così descritta può essere eseguita solamente con risonanza magnetica. È possibile, tuttavia, ottenere delle immagini funzionali analoghe con gli ultrasuoni, i quali presentano una risoluzione peggiore della risonanza magnetica. Nel caso di ultrasuoni, il mezzo di contrasto è caratterizzato da microbolle rivestite da una membrana stabilizzate di fosfolipidi, albumina o altri polimeri. Le immagini ecografiche sono anche più rumorose per cui le strutture anatomiche piccole non possono essere visualizzate.

I raggi X non permettono acquisizioni continue poiché ciò richiederebbe un aumento della dose di radiazioni assorbite dal paziente, aumentando il rischio biologico.

\subsection{Echo Planar imaging}\label{echo-planar-imaging}

Agli inizi degli anni '90 Peter Mansfield ideò una metodica nota come Echo-Planar Imaging (EPI), caratterizzata da un'elevata risoluzione temporale in quanto, con una sola eccitazione a radiofrequenza, si riesce ad acquisire un'immagine completa pesata in \(T_{2}^{*}\). Questa sequenza di imaging con risonanza magnetica permette di ottenere immagini funzionali, soprattutto delle strutture cardiache.

La sequenza EPI è basata su una classica gradient-echo, in cui si applica un impulso a radiofrequenza, un gradiente di selezione della fetta, di codifica di fase e di lettura. In una classica sequenza gradient-echo, dopo aver acquisito l'echo, nella finestra temporale centrata sul tempo d'echo \(T_{E}\), si aspetta un tempo di ripetizione \(T_{R}\), al fine di acquisire una seconda riga del \(k\)-spazio, variando i gradienti di codifica di fase e di frequenza. Per ridurre il tempo complessivo dell'esame diagnostico, la sequenza EPI prevede l'acquisizione di più righe del \(k\)-spazio con un solo impulso a radiofrequenza nel tempo di ripetizione \(T_{R}\).

Un primo modo per eseguire l'operazione di acquisizione durante l'impulso consiste nell'invertire il gradiente di lettura, generando così un rifasamento e, conseguentemente un secondo echo, al tempo \(T_{E,2}\). Data la presenza dei decadimenti con tempi \(T_{2}\) e \(T_{2}^{*}\), il secondo echo ha ampiezza minore del primo.

\includegraphics[width=5.25in,height=3.58333in,alt={MRI Physics: MRI Pulse Sequences - XRayPhysics}]{media/13_FastImm/image346.pdf}
Con questa metodica è possibile acquisire due righe di \(k\)-spazi differenti, caratterizzati da una pesatura in \(T_{2}\) diversa.

A ogni ripetizione si acquisiscono due righe di due \(k\)-spazi differenti. Di conseguenza, nello stesso tempo di imaging di una normale sequenza gradient-echo si ottengono due immagini con pesatura diverse dello stesso distretto anatomico. Questa procedura consente di ottenere due immagini, dimezzando così i tempi necessari per acquisire due immagini a contrasto diverso, in base al tempo \(T_{2}\), separatamente.

Una seconda soluzione consiste nell'aggiunta di un gradiente di codifica di fase nell'intervallo di tempo tra il primo gradiente di lettura, al tempo \(T_{E,1}\) e il secondo tempo \(T_{E,2}\). Il gradiente lungo l'asse di codifica di fase ha l'effetto di cambiare la riga del \(k\)-spazio acquisito. In particolare, dopo aver acquisito l'echo durante il tempo \(T_{E,1}\), il gradiente aggiuntivo sposta la coordinata dell'asse di codifica di fase, generalmente associata all'asse \(y\). In questo modo, col secondo gradiente di lettura si acquisisce una seconda riga del \(k\)-spazio in senso retrogrado poiché il gradiente di lettura ha polarità opposta.

Ripetendo questa sequenza un tempo \(T_{R}\) è possibile acquisire una riga pari in un verso e una dispari nel verso opposto, dimezzando i tempi di imaging per acquisire una singola immagine.

Rispetto alla soluzione precedente, in cui si acquisivano due immagini diversamente pesate, grazie al gradiente di codifica di fase interposto tra i due gradienti di lettura e le relative acquisizioni, si ottiene una singola immagine.

Nella pratica in ogni ripetizione, si varia il gradiente di codifica di fase iniziale in modo da selezionare solo una riga pari del \(k\)-spazio; le righe dispari si ottengono, all'interno della stessa sequenza, col secondo gradiente di codifica di fase, il quale sfasa gli isocrmati selezionando la successiva riga dispari del \(k\)-spazio.

In questo modo è possibile acquisire un'intera immagine in metà del tempo; tuttavia, bisogna evidenziare che l'acquisizione delle linee dispari avviene in senso discorde rispetto alle righe pari; inoltre, la pesatura in \(T_{2}\) lungo le linee pari è diversa dalla pesatura in \(T_{2}\) delle righe dispari dello stesso \(k\)-spazio, essendo acquisite con due tempi di echo diversi. L'elaborazione software deve essere tale da compensare il decadimento esponenziale tre le diverse righe.

Estendendo il ragionamento e l'elaborazione software è possibile applicare tanti gradienti di codifica di fase tra un gradiente di lettura e il successivo, in modo da acquisire completamente tutto il \(k\)-spazio relativo a quel gradiente di selezione della fetta.

A ogni tempo d'echo si acquisisce il segnale emesso dal paziente. In una sequenza reale, il gradiente di codifica di fase è applicato tra una transizione e l'altra del gradiente di lettura. Per la sua breve durata il gradiente di codifica di fase è noto come \emph{blip}.

La sequenza eredita il decadimento esponenziale con \(T_{2}\) tra le varie righe del \(k\)-spazio; inoltre, è necessario rovesciare i vettori dispari del \(k\)-spazio così da avere tutte le acquisizioni eseguite nello stesso verso.

Il vantaggio temporale di questa soluzione è palese poiché in un tempo \(T_{R}\), dell'ordine di \(1\ s\), si campiona tutto il \(k\)-spazio relativo a una singola fetta. In questo caso, si parla di single-shot EPI.

Se non si acquisisce l'intero \(k\)-spazio con una sequenza si parla di \(k\)-spazio segmentato. Questa soluzione è applicata per evitare che il decadimento esponenziale renda il segnale ricevuto dalle ultime righe di ampiezza confrontabile col rumore, non permettendo la ricostruzione.

La sequenza EPI è particolarmente utilizzata negli studi di analisi e verifica dell'attività cerebrale.

La metodica EPI permette di ottenere immagini pesate in \(T_{2}\) o \(T_{2}^{*}\). Se si esegue un'imaging pesato in \(T_{2}\), la sequenza richiede un'elevata omogeneità di campo, dunque, è necessario vere un ottimo shimming. In presenza di campi magnetici principali disomogenei, si ottiene un'immagine pesata in \(T_{2}^{*}\), non legata esclusivamente alle caratteristiche intrinseche del tessuto. In alcune applicazioni, la pesatura in \(T_{2}^{*}\) risulta vantagiosa.

\subsubsection{Gestione dei gradienti alternativa per EPI}\label{gestione-dei-gradienti-alternativa-per-epi}

Una variante della sequenza EPI non sfrutta gradienti spinti, di forma quasi onda quadra, i quali sono abbastanza complessi da generare nella pratica. Per limiti fisici, i gradienti hanno un certo periodo di salita e discesa non nulli, come si vorrebbe nel caso ideale.

Al fine di semplificare la realizzazione dei gradienti, dal punto di vista tecnologico, è possibile eseguire delle sequenze di gradienti con forma sinusoidale, di ampiezza variabile. Con questa soluzione i tempi di salita non devono essere istantanei ma smussati nel tempo, quindi, più semplici da ottenere.

Si dimostra che la sequenza di gradienti di forma sinusoidale con ampiezza crescente porta a un'acquisizione del \(k\)-spazio non rettangolare ma a spirale. Ciò complica la ricostruzione dell'immagine via software, in quanto sono necessari algoritmi di interpolazione, ma permette di ottenere hardware più semplici.

\begin{figure}
\centering
\includegraphics[width=4.92569in,height=3.52778in,alt={Gradients and k-space filling}]{media/13_FastImm/image347.pdf}\caption{Figura .: Applicazione dei gradienti e relativo riempimento del \(k\)-spazio}
\end{figure}

\begin{center}
\vfill
    \chapter{Immagini funzionali}
    \label{blx:FuncImm\therefsection}
\vfill

\minitoc
\newpage
\end{center}
\justify



\section{Tecniche di imaging strutturale e funzionale}\label{tecniche-di-imaging-strutturale-e-funzionale}

Le tecniche di \textbf{imaging funzionale} non si concentrano sulla struttura, ma su come il corpo o gli organi \textbf{lavorano}. Queste tecniche sono utilizzate per studiare i processi fisiologici come il metabolismo, il flusso sanguigno o l'attività neuronale per capire come funzionano i tessuti. Queste tecniche sono usate per studiare l\textquotesingle attività cerebrale, la diffusione di farmaci o il funzionamento di specifici organi.

\subsection{Magnetic resonance spectroscopic imaging}\label{magnetic-resonance-spectroscopic-imaging}

La Magnetic Resonance Spectroscopic Imaging (MRSI) è una metodica spettroscopica di risonanza magnetica in grado di fornire immagini funzionali, indicante il metabolismo dei tessuti.

Storicamente la risonanza magnetica nasce per scopi di spettroscopia; se, infatti, si irradia un campione con un campo elettromagnetico statico \(B_{0}\) ci si aspetterebbe che ogni protone risuoni alla frequenza di Larmor (\(\omega_{0} = \gamma B_{0}\)). In realtà, il segnale ricevuto contiene molte frequenze, legate al chemical shift. Ogni nucleo di idrogeno, infatti, percepisce il campo magnetico totale dato dal campo statico e di una disomogeneità di campo, introdotta dalla schermatura dell'ambiente molecolare in cui è inserito il protone stesso.

\begin{figure}
\centering
\includegraphics[width=4.6981in,height=3.66667in,alt={Immagine che contiene diagramma, cerchio, schermata, Diagramma Il contenuto generato dall\textquotesingle IA potrebbe non essere corretto.}]{media/14_FuncImm/image348.pdf}\caption{Figura .: Schema della sezione del corpo umano immerso in un campo magnetico}
\end{figure}

In particolare, ogni molecola produce uno specifico effetto di schermatura, quindi, lo spettro del segnale registrato possiede delle componenti spettrali, le cui frequenze dipendono dal tipo di molecole presenti. L'ampiezza del picco dipende dal numero di molecole di una specifica sostanza che risuonano alla frequenza della componente armonica registrata; in altre parole, l'ampiezza del picco spettrale dipende dalla concentrazione della molecola all'interno del corpo irradiato. La maggior parte delle molecole presenta un coefficiente di shielding \(\sigma\) di circa \(0 \div 4\ ppm\).

\begin{figure}
\centering
\includegraphics[width=6.69306in,height=3.99236in,alt={Immagine che contiene testo, diagramma, Diagramma, linea Il contenuto generato dall\textquotesingle IA potrebbe non essere corretto.}]{media/14_FuncImm/image349.pdf}\caption{Figura .: Vari picchi spettrali relativi a macromolecole diverse}
\end{figure}

Dal punto di vista analitico, il segnale decade esponenzialmente come i tempi di rilassamento \(T_{1}\) e \(T_{2}\); dunque, è necesario inserire un opportuno fattore di attenuazione o damping factor, caratteristico della molecola, al fine di descrivere al meglio il segnale registrato dal voxel.

Le sequenze utilizzate per eseguire l'imaging spettroscopico sono abbastanza semplici, poiché non prevedono la presenza di gradiente di selezione della fetta. L'imaging è, infatti, tridimensionale ed è ottenuto applicando tre gradienti di fase lungo i tre assi, al fine di acquisire un preciso voxel.

Il segnale misurato al tempo di echo è dato dalla somma di varie sinusoidi con frequenze di risonanza diverse in base alla molecola contenuta nel campione.

Dal punto di vista analitico, il segnale si esprime come una somma di vari componenti frequenziali:

\[s = \sum_{i}^{}{{\widehat{\rho}}_{i}\exp\left( j\left( \omega_{i} + \phi_{i} \right) \right)}\]

Ogni specie chimica è caratterizzata da una propria densità protonica efficace, frequenza di risonanza e sfasamento. Al fine di eccitare tutte le varie molecole, gli impulsi di eccitazione devono essere ad ampio spettro, così da selezionare l'interno range di frequenze delle specie chimiche di interesse.

Sfruttando opportunamente i gradienti di selezione lungo i tre assi è possibile effettuare la codifica di fase dei vari voxel, al fine di ottenere lo spettro di ciascuno di essi. Tramite questa informazione è possibile ricavare delle mappe di distribuzione delle varie macromolecole.

La metodica MRSI è molto utilizzata nello studio della composizione cerebrale poiché permette la generazione di mappe di distribuzione dei principali metaboliti cerebrali, come:

\begin{itemize}
\item
  N-Acetil-Aspartato (NAA), marker della funzionalità neuronale;
\item
  Creatina (Cr), indice del metabolismo energetico;
\item
  Colina (Cho), legata al metabolismo delle membrane cellulari
\end{itemize}

Per poter quantificare le varie molecole, ovvero misurare le concentrazioni, è necessario elaborare opportunamente il segnale ricevuto, mediante tecniche appropriate. Il segnale spettroscopico può essere modellato come un segnale complesso campionato, del tipo:

\[s(n) = \sum_{k = 1}^{n}{c_{k}\xi_{k}^{n}} + \varepsilon(n)\]

Dove \(c_{k} = a_{k}\exp\left( j\phi_{k} \right)\) tiene conto che la ricostruzione del \(k\)-spazio è complessa a causa degli sfasamenti, mentre \(\xi_{k}^{n} = \exp\left( - \alpha_{k} + j2\pi\nu_{k} \right)\), ovvero è il termine esponenziale che comprende il decadimento dovuto a \(T_{2}\) (se si applica una sequenza spin-echo) o \(T_{2}^{*}\) (se si applica una sequenza gradient-echo) e la frequenza di risonanza caratteristica della specie chimica. Più nel dettaglio il termine \(\alpha_{k}\) è il damping factor mentre \(a_{k}\) è legato alla concentrazione delle molecole contenute nel materiale. In ambito clinico il termine di fase ha poco interesse. Infine, il termine \(\varepsilon(n)\) rappresenta il rumore sovrapposto alla misura eseguita.

A partire dai dati acquisiti è possibile costruire una matrice dei dati con una struttura Hankel, ovvero una matrice quadrata o rettangolare in cui ogni elemento \(a_{ij}\) dipende solo dalla somma degli indici di riga e di colonna, ovvero \(a_{ij} = f(i + j - 1)\). In parole più semplici, una matrice di Hankel ha elementi costanti lungo le antidiagonali, ovvero le diagonali che salgono da sinistra verso destra.

Nel caso specifico delle misurazioni, si ha:

\[\overset{\underline{}}{\overset{\underline{}}{S}} = \begin{pmatrix}
s_{0} & s_{1} & s_{2} & \cdots & s_{M - 1} \\
s_{1} & s_{2} & s_{3} & \cdots & s_{M} \\
\cdots & \cdots & \cdots & \cdots & \cdots \\
s_{L - 1} & s_{L} & s_{L + 1} & \cdots & s_{N - 1}
\end{pmatrix}\]

In accordo con la matrice di Hankel, sulla prima riga vi sono i campioni da \(s_{0}\) a \(s_{M - 1}\), mentre sulla seconda riga vi sono i campioni da \(s_{1}\) a \(s_{M}\). In generale, ogni riga della matrice di Hankel si ottiene dalla riga precedente, traslandola verso sinistra di un campione. Tale processo è ripetuto un certo numero \(N\) di volte, fino a ottenere \(L\) righe. I parametri \(L\) e \(M\) devono essere scelti opportunamente in base alle analisi da eseguire; tuttavia, non esiste una regola precisa per la scelta di questi parametri. Nella pratica si sceglie \(M\) prossimo a \(L\).

Se il segnale fosse costituito solo da sinusoidi senza rumore la matrice dei dati \(\overset{\underline{}}{\overset{\underline{}}{S}}\) avrebbe un rango \(K\), numero delle sinusoidi che compongono il segnale. La presenza del rumore \(\varepsilon(n)\) determina che il rango della matrice sia pieno pari al \(\min(LM)\).

In assenza di rumore, dal rango della matrice \(\overset{\underline{}}{\overset{\underline{}}{S}}\) si riesce a determinare il numero di sinusoidi presenti nel segnale, note le quali è semplice determinare le frequenze di risonanze delle varie molecole contenute nel materiale.

A causa del rumore, è necessario elaborare la matrice in modo da individuare il numero delle sinusoidi effettivamente presenti nel segnale acquisito. Un modo per analizzare il rango della matrice di Henkle \(\overset{\underline{}}{\overset{\underline{}}{S}}\) consiste nella \emph{Singular Value Decomposition} (SVD), un procedimento simile alla decomposizione in autovettori e autovalori. Ordinando i valori singolari in ordine decresecente si osserva in genere una netta discontinuità tra i valori songolari corrispondneti al segnale ed i valori corrispondenti al rumore. Nel caso di decomposizione ai valori singolari, il concetto di autovettore è sostituito da quello di valore singolare.

\subsubsection{Cenni sui valori singolari}\label{cenni-sui-valori-singolari}

Una matrice \(\overset{\underline{}}{\overset{\underline{}}{X}}\) complessa con dimensioni \(N \times M\) e rango \(r\) può essere decomposta come:

\[\overset{\underline{}}{\overset{\underline{}}{X}} = \overset{\underline{}}{\overset{\underline{}}{U}}\overset{\underline{}}{\overset{\underline{}}{\Sigma}}{\overset{\underline{}}{\overset{\underline{}}{V}}}^{H}\]

Dove \(\overset{\underline{}}{\overset{\underline{}}{U}}\) è una matrice \(N \times N\), \(\overset{\underline{}}{\overset{\underline{}}{V}}\) una matrice unitaria \(M \times M\) mentre \(\overset{\underline{}}{\overset{\underline{}}{\Sigma}}\) è una matrice \(N \times M\) diagonale. Gli elementi di \(\overset{\underline{}}{\overset{\underline{}}{\Sigma}}\) diversi da zero sono detti valori singolari.

La matrice \(\overset{\underline{}}{\overset{\underline{}}{X}}{\overset{\underline{}}{\overset{\underline{}}{X}}}^{H}\) è semidefinita positiva, dunque, i suoi autovalori \(\sigma_{1}^{2},\sigma_{2}^{2},\ldots,\sigma_{M}^{2}\) sono non negativi. Siccome il rango di \(\overset{\underline{}}{\overset{\underline{}}{X}}\) è \(r\), i primi \(r\) autovalori sono non negativi, mentre i restanti \(M - r\) sono nulli.

Siano \({\overset{\underline{}}{v}}_{1},{\overset{\underline{}}{v}}_{2},\ldots,v_{M}\) gli autovettori corrispondenti agli autovalori \(\sigma_{1}^{2},\sigma_{2}^{2},\ldots,\sigma_{M}^{2}\). Si considera l'arrangiamento:

\[\overset{\underline{}}{\overset{\underline{}}{V}} = \left\lbrack {\overset{\underline{}}{\overset{\underline{}}{V}}}_{1},{\overset{\underline{}}{\overset{\underline{}}{V}}}_{2} \right\rbrack\]

Dove \({\overset{\underline{}}{\overset{\underline{}}{V}}}_{1}\) contiene le prime \(r\) colonne di \(\overset{\underline{}}{\overset{\underline{}}{V}}\), mentre \({\overset{\underline{}}{\overset{\underline{}}{V}}}_{2}\) le restanti, quindi è nulla. Risulta che:

\[{\overset{\underline{}}{\overset{\underline{}}{V}}}_{1}^{H}{\overset{\underline{}}{\overset{\underline{}}{X}}}^{H}\overset{\underline{}}{\overset{\underline{}}{X}}{\overset{\underline{}}{\overset{\underline{}}{V}}}_{1} = \left( {diag}\left( \sigma_{1},\sigma_{2},\ldots,\sigma_{r} \right) \right)^{2}\]

Ponendo:

\[{\overset{\underline{}}{\overset{\underline{}}{\Sigma}}}_{r} = {diag}\left( \sigma_{1},\sigma_{2},\ldots,\sigma_{r} \right)\]

La relazione si scrive come:

\[{\overset{\underline{}}{\overset{\underline{}}{V}}}_{1}^{H}{\overset{\underline{}}{\overset{\underline{}}{X}}}^{H}\overset{\underline{}}{\overset{\underline{}}{X}}{\overset{\underline{}}{\overset{\underline{}}{V}}}_{1} = {\overset{\underline{}}{\overset{\underline{}}{\Sigma}}}_{r}\]

Ovviamente è possibile scrivere anche la relazione per \({\overset{\underline{}}{\overset{\underline{}}{V}}}_{2}\), dove:

\[{\overset{\underline{}}{\overset{\underline{}}{V}}}_{2}^{H}{\overset{\underline{}}{\overset{\underline{}}{X}}}^{H} = \overset{\underline{}}{\overset{\underline{}}{O}}\]

Risulta che:

\[{\overset{\underline{}}{\overset{\underline{}}{\Sigma}}}_{r}^{- 1}{\overset{\underline{}}{\overset{\underline{}}{V}}}_{1}^{H}{\overset{\underline{}}{\overset{\underline{}}{X}}}^{H}\overset{\underline{}}{\overset{\underline{}}{X}}{\overset{\underline{}}{\overset{\underline{}}{V}}}_{1}{\overset{\underline{}}{\overset{\underline{}}{\Sigma}}}_{r}^{- 1} = \overset{\underline{}}{\overset{\underline{}}{I}}\]

Ponendo:

\[\overset{\underline{}}{\overset{\underline{}}{X}}{\overset{\underline{}}{\overset{\underline{}}{V}}}_{1}{\overset{\underline{}}{\overset{\underline{}}{\Sigma}}}_{r}^{- 1} = \overset{\underline{}}{\overset{\underline{}}{U}}\]

Si ottiene la relazione:

\[{\overset{\underline{}}{\overset{\underline{}}{U}}}^{H}\overset{\underline{}}{\overset{\underline{}}{U}} = \overset{\underline{}}{\overset{\underline{}}{I}}\]

Da cui si costruisce la matrice:

\[{\overset{\underline{}}{\overset{\underline{}}{U}}}^{H}\overset{\underline{}}{\overset{\underline{}}{X}}\overset{\underline{}}{\overset{\underline{}}{U}} = \left( \begin{array}{r}
{\overset{\underline{}}{\overset{\underline{}}{U}}}_{1}^{H} \\
{\overset{\underline{}}{\overset{\underline{}}{U}}}_{2}^{H}
\end{array} \right)\ \overset{\underline{}}{\overset{\underline{}}{X}}\left\lbrack {\overset{\underline{}}{\overset{\underline{}}{V}}}_{1},{\overset{\underline{}}{\overset{\underline{}}{V}}}_{2} \right\rbrack = \begin{pmatrix}
{\overset{\underline{}}{\overset{\underline{}}{\Sigma}}}_{r} & \overset{\underline{}}{\overset{\underline{}}{O}} \\
\overset{\underline{}}{\overset{\underline{}}{O}} & \overset{\underline{}}{\overset{\underline{}}{O}}
\end{pmatrix}\]

\subsubsection{Linear prediction SVD}\label{linear-prediction-svd}

Quando il segnale acquisito durante la spettroscopia non presenta rumore sovrapposto, si dimostra che il segnale soddisfa l'equazione di predizione lineare con coefficienti \(q_{k}\) del tipo:

\[{\widehat{s}}_{n} = q_{1}s_{n + 1} + q_{2}s_{n + 2} + \ldots + q_{M}s_{n + M}\]

dove i termini \(q_{k}\) sono i coefficienti del modello a predizione lineare. Se è presente del rumore, questa relazione non è esattamente valida; in tal caso, è opportuno scegliere \(M \gg K\) in modo che le componenti di rumore siano tenute in conto dai coefficienti aggiuntivi \(q_{k},k = 1,2,\ldots,M\)

Il segnale può essere espresso in termini matriciali. Si definisce il vettore dei segnali ricostruiti:

\[\widehat{\overset{\underline{}}{s}} = \left( \begin{array}{r}
\begin{array}{r}
\begin{array}{r}
{\widehat{s}}_{0} \\
{\widehat{s}}_{1}
\end{array} \\
 \vdots 
\end{array} \\
{\widehat{s}}_{N - M - 1}
\end{array} \right)\]

Il vettore dei coefficienti della regressione:

\[\overset{\underline{}}{q} = \left( \begin{array}{r}
\begin{array}{r}
\begin{array}{r}
q_{1} \\
q_{2}
\end{array} \\
 \vdots 
\end{array} \\
q_{M}
\end{array} \right)\]

E la matrice di Hankle:

\[\overset{\underline{}}{\overset{\underline{}}{S}} = \begin{pmatrix}
s_{0} & s_{1} & s_{2} & \cdots & s_{M - 1} \\
s_{1} & s_{2} & s_{3} & \cdots & s_{M} \\
\cdots & \cdots & \cdots & \cdots & \cdots \\
s_{L - 1} & s_{L} & s_{L + 1} & \cdots & s_{N - 1}
\end{pmatrix}\]

Per la decomposizione in valori singolari è possibile scrivere:

\[\overset{\underline{}}{\overset{\underline{}}{S}} = {\overset{\underline{}}{\overset{\underline{}}{U}}}^{H}\overset{\underline{}}{\overset{\underline{}}{\Sigma}}\overset{\underline{}}{\overset{\underline{}}{V}}\]

Dato il suo andamento sinusoidale del segnale acquisito, i valori singolari del rumore sono prossimi allo zero, dunque, si può ottenere una pulizia del rumore ponendo i valori singolari del rumore esattamente uguali a zero. Questo processo si chiama \textbf{troncamento della SVD}. In questo modo, si ottiene una matrice \(\widehat{\overset{\underline{}}{\overset{\underline{}}{S}}}\), a partire da quella di Henkle \(\overset{\underline{}}{\overset{\underline{}}{S}}\), con rango \(k\). La matrice ripulita non possiede più struttura di Henkel. Tale struttura può essere ripristinata ponendo su ciascuna antidiagonale il valor medio dei termini su quella diagonale. Al fine di ottenere i valori dei coefficienti del modello di predizione lineare si utilizza un algoritmo OLS:

\[\overset{\underline{}}{q} = \left( {\widehat{\overset{\underline{}}{\overset{\underline{}}{S}}}}^{T}\widehat{\overset{\underline{}}{\overset{\underline{}}{S}}} \right)^{- 1}\widehat{\overset{\underline{}}{\overset{\underline{}}{S}}}\widehat{\overset{\underline{}}{s}}\]

Dove \(\widehat{\overset{\underline{}}{s}}\) è il vettore ripulito dal rumore.

Per ottenere le componenti armoniche del segnale ripulito dal rumore, si calcolano i poli situati all'esterno al cerchio di raggio unitario, ovvero i poli il cui modulo è maggiore di \(1\). Le componenti armoniche (le sinusoidi) del segnale di risonanza magnetica sono intrinsecamente smorzate. Per modellarle correttamente il modello di predizione lineare, i poli del filtro devono trovarsi fuori dal cerchio unitario al fine di individuare le sinusoidi smrozate.

In MATLAB si utilizza la funzione henkel, per costruire una matrice di Henkel, e svd per eseguire la decomposizione ai valori singolari. Noti questi due parametri è possibile implementare l'algoritmo di pulizia.

\subsubsection{Rapporto segnale/rumore in MRSI}\label{rapporto-segnalerumore-in-mrsi}

Al fine di ottenere delle ricostruzioni mediante la tecnica del MRSI con un buon rapporto segnale/rumore è necessario aumentare la dimensione del voxel, poiché all'interno del corpo umano la concentrazione delle macromolecole è piuttosto bassa. Inoltre, l'aumento del voxel permette di compensare parzialmente l'aumento dei tempi di imaging: acquisendo un volume di \(32 \times 32 \times 16\), con un tempo di ripetizione di \(1\ s\) è necessario aspettare \(16384\ s = 273\ min = 4.55\ h\).

La tecnica di MRSI è applicata solamente in distretti anatomici con dimensione ridotta o su una singola fetta di interesse, in modo da evitare al paziente una permanenza nel gantry di quasi tre ore.

La ricostruzione delle armoniche non può essere eseguita anche adoperando una FFT poiché i picchi spettrali tendono a sovrapporsi. Con questa metodica è possibile solamente risolvere i picchi di ampiezza maggiore e ben distanziati tra loro. Gli spettri dei segnali registrati tendono ad assumere la forma di una lorentziana, dunque, con durata finita nel dominio della frequenza.

Per la ricostruzione dell'immagine è necessario avere un'ottima apparecchiatura, al fine di ridurre anche il rumore, e un software ottimizzato per la ricostruzione delle frequenze e la loro visualizzazione sulla ROI con pseudocolori. Si può ritenere che i due contributi siano pesati allo stesso modo, ovvero al \(50\%\).

\subsubsection{Utilizzato dalla MSRI in clinica}\label{utilizzato-dalla-msri-in-clinica}

La tecnica MSRA è utilizzata per l'individuazione di tumori alla prostata, al seno e altre regioni anatomiche in cui è possibile selezionare una ROI di piccola ampiezza, al fine di ridurre i tempi di acquisizione.

Anche i tumori cerebrali possono essere individuati con questa tecnica poiché queste neoplasie alterano la normale concentrazione dei metaboliti del cervello quali N-Acetil-Aspartato (NAA), Creatina (Cr) e Colina (Cho). Lo squilibrio tra questi metaboliti è indice di tumore e sono ben documentati in letteratura scientifica.

\subsubsection{Cenni sulle nuove frontiere della medicina}\label{cenni-sulle-nuove-frontiere-della-medicina}

Le tecniche di imaging funzionale con risonanza magnetica, quali DCE-MRI e MRSI, sebbene siano piuttosto recenti (rese disponibili in forma matura in clinica solamente negli ultimi \(25\) anni) forniscono una prospettiva per l'impiego del precision medicine. Questa nuova visione della medicina è ritagliata sul paziente e sfrutta metodiche che possono valutare, con elevata precisione, una patologia del paziente. La metodica permette poi di scegliere una terapia ottima e specifica per quel dato paziente.

Inoltre, con la precision medicine è possibile analizzare il progredire della malattia oppure il corretto funzionamento della terapia somministrata.

La medicina di precisione richiede un numero elevato di strumentazione, \textbf{tecnologie sofisticate e un'integrazione di dati} provenienti da diverse fonti quali imaging (sia morfologici che funzionali), genetica, dati clinici, ecc. Il focus non è sulla quantità, ma sulla complessità e l'interconnessione dei dati come:

\begin{itemize}
\item
  \textbf{Dati genetici e genomici:} Per identificare le mutazioni molecolari che guidano una malattia specifica in quel paziente.
\item
  \textbf{Dati clinici:} Le informazioni storiche del paziente, le risposte a terapie passate e i fattori di rischio.
\end{itemize}

\subsection{Applicazione della EPI in risonanza magnetica funzionale}\label{applicazione-della-epi-in-risonanza-magnetica-funzionale}

La sequenza EPI è molto sfruttata nella risonanza magnetica funzionale o fMRI (functional Magnetic Resonance Imaging). Le immagini fMRI sono indicative della funzionalità di un organo, spesso il cervello. Nella pratica, l'acronimo fMRI è legato strattamente alle applicazioni sullo studio cerebrale, poiché la metodica permette di evidenziale con mappe di pseudo-colori le zone cerebrali attivate da uno stimolo.

Con la sequenza echo-planar è possibile acquisire intere porzioni del volume cerebrale con una risoluzione temporale dell'ordine di \(2 \div 3\ s\). L'evoluzione temporale dell'attività cerebrale è dell'ordine di qualche secondo, quindi, la sequenza EPI consente di valutare l'evoluzione temporale dell'encefalo con buona risoluzione temporale.

La fMRI è una metodica di analisi complementare all'elettroencefalogramma (EEG) in cui si studiano i potenziali elettrici prelevati sullo scalpo del paziente. Essendo segnali elettrici, l'EEG può essere valutato con risoluzione spaziale anche molto spinta, in base alle caratteristiche possedute dalla circuiteria di acquisizione ed elaborazione. Le tempistiche elettriche sono dell'ordine dei \(ms\), molto più veloci, quindi, di quelle meccaniche legate al flusso di sangue sfruttato dalla fMRI. In linea teoria, l'EEG può essere acquisito anche mentre si esegue la fMRI, prestando attenzione a utilizzare elettrodi amagnetici.

La metodologia fMRI sfrutta il consumo di ossigeno da parte dei neuroni attivati durante la somministrazione di uno specifico stimolo. Quando un neurone è stimolato, infatti, assorbe una quantità di ossigeno dal sangue in concentrazione maggiore. In altre parole, il consumo di ossigeno da parte del neurone dipende dall'attività cerebrale.

L'ossigeno nel sangue è trasportato dall'emoglobina che si presenta in due forme:

\begin{itemize}
\item
  L'ossiemoglobina, se legata all'ossigeno;
\item
  Desossiemoglobina, se non legata all'ossigeno.
\end{itemize}

Studi sperimentali hanno dimostrato che la desossiemoglobina, avendo un atomo di ferro non legato all'ossigeno, presenta una coppia di elettroni spaiati (\({Fe}^{2 +}\ \)) e, di conseguenza, offre un comportamento paramagnetico. Le sostanze paramagnetiche sono debolmente attratte da un campo magnetico esterno e hanno una suscettività magnetica positiva e notevolmente più elevata rispetto alle sostanze diamagnetiche.

Quando l'emoglobina è legata all'ossigeno, non vi sono coppie di elettroni spaiati, che possono interferire col campo esterno applicato. L'ossiemoglobina presenta un comportamento diamagnetico, legato alla rotazione degli elettroni intorno al nucleo.

\begin{figure}
\centering
\includegraphics[width=6.14394in,height=3.14342in,alt={Immagine che contiene cartone animato, clipart, disegno, Elementi grafici Il contenuto generato dall\textquotesingle IA potrebbe non essere corretto.}]{media/14_FuncImm/image350.pdf}\caption{Figura .: Ossiemoglobina legata all\textquotesingle ossigeno e desossiemoglobina}
\end{figure}

La presenza di elementi paramagnetici nel sangue introduce delle disomogeneità di campo rilevate dalla tecnica di imaging con risonanza magnetica. Di conseguenza, se in un certo voxel vi è una concentrazione rilevante di ossiemoglobina non si hanno alterazioni del campo magnetico; in altre parole, in presenza di sangue ossigenato non si hanno delle alterazioni legate alla componente paramagnetiche del sangue apprezzabili. Quando, invece, il cervello estrae ossigeno dal sangue provoca un aumento della desossiemoglobina. Il voxel contenete questa sostanza vede una disomogeneità di campo magnetico legata al comportamento paramagnetico di questa molecola. Al fine di evidenziare il comportamento paramagnetico, le sequenze devono essere tali da ottenere immagini con pesatura in \(T_{2}^{*}\) al fine di evidenziare proprio le disomogeneità di campo introdotte dall'attivazione cerebrale.

Il vettore di magnetizzazione è legato al campo magnetico \(\overset{\underline{}}{H}\) tramite la suscettività magnetica \(\chi\):

\[\overset{\underline{}}{M} = \chi\overset{\underline{}}{H}\]

Anche il vettore induzione magnetica \(\overset{\underline{}}{B}\) è legato alla magnetizzazione tramite la suscettibilità \(\chi\) e la permeabilità magnetica del vuoto dalla relazione:

\[\overset{\underline{}}{B} = \dfrac{1 + \chi}{\chi}\mu_{0}\overset{\underline{}}{M}\]

Al variare della suscettività magnetica \(\chi\) cambia la magnetizzazione \(\overset{\underline{}}{M}\), dovuta al campo magnetico \(\overset{\underline{}}{B}\) applicato. Di conseguenza, gli spin vedono un campo magnetico diverso che porta il segnale di quel voxel ad avere caratteristiche differenti dai tessuti circostanti, in cui la quantità di desossiemoglobina è trascurabile. La presenza di questo tipo di emoglobina determina un cambio locale della suscettibilità \(\chi\) che porta a una variazione del vettore di magnetizzazione nel voxel.

È possibile ritenere che la suscettibilità del sangue sia data da una somma pesata delle concentrazioni di emoglobina ossigenata e deossigenata:

\[\chi_{blood} = HCT\left( \chi_{OXY}Y + (1 - Y)\chi_{DEOXY} \right)\]

Dove \(HCT\) è l'ematocrito, ovvero la componente corpuscolare del sangue, di cui il \(99\%\) è composto da globuli rossi, rispetto alla componente liquida o siero, contenente ioni e altre molecole disciolte. Il termine \(Y\) è la percentuale di emoglobina ossigenata mente \((1 - Y)\) è la percentuale di desossiemoglobina nell'ipotesi che nel sangue vi siano solo queste due forme di emoglobina. La presenza di carbossiemoglobina, prodotta quando l'emoglobina è legata al monossido di carbonio \(CO\), e la metaemoglobina, contenente un atomo di ferro del gruppo eme ossidato, è trascurata.

Al fine di ottenere la suscettività magnetica del sangue, alla componente legata alla porzione corpuscolare è necessario aggiungere anche il comportamento magnetico del plasma \((1 - HCT)\chi_{plasma}\), ottenendo:

\[\chi_{blood} = HCT\left( \chi_{OXY}Y + (1 - Y)\chi_{DEOXY} \right) + (1 - HCT)\chi_{plasma}\]

Dove \(1 - HCT\) è la percentuale di volume di sangue legata al plasma.

Quando il cervello entra in attività consuma ossigeno, facendo aumentare la percentuale di desossiemoglobina e, di conseguenza, variando la suscettività magnetica del sangue. Le variazioni della suscettibilità possono essere semplicemente calcolate come:

\[\Delta\chi_{blood} = HCT\left( \chi_{OXY}Y_{1} + \left( 1 - Y_{1} \right)\chi_{DEOXY} \right) + (1 - HCT)\chi_{plasma} - HCT\left( \chi_{OXY}Y_{2} + \left( 1 - Y_{2} \right)\chi_{DEOXY} \right) - (1 - HCT)\chi_{plasma}\]

La componente legata al plasma si semplifica, ottenendo:

\[\Delta\chi_{blood} = HCT\left( \chi_{OXY}Y_{1} + \left( 1 - Y_{1} \right)\chi_{DEOXY} \right) - HCT\left( \chi_{OXY}Y_{2} + \left( 1 - Y_{2} \right)\chi_{DEOXY} \right)\]

Si pone \(HCT\) in evidenza:

\[\Delta\chi_{blood} = HCT\left( \chi_{OXY}Y_{1} + \left( 1 - Y_{1} \right)\chi_{DEOXY} - \chi_{OXY}Y_{2} - \left( 1 - Y_{2} \right)\chi_{DEOXY} \right)\]

Svolendo i prodotti si ottiene:

\[\Delta\chi_{blood} = HCT\left( \chi_{OXY}Y_{1} + \chi_{DEOXY} - Y_{1}\chi_{DEOXY} - \chi_{OXY}Y_{2} - \chi_{DEOXY} + Y_{2}\chi_{DEOXY} \right)\]

Si semplificano i termini di segno opposto (\(\chi_{DEOXY}\)) e si raccoglie in evidenza, si ha:

\[\Delta\chi_{blood} = \ HCT\left( \chi_{OXY}Y_{1} - Y_{1}\chi_{DEOXY} - \chi_{OXY}Y_{2} + Y_{2}\chi_{DEOXY} \right) = HCT\left( \chi_{OXY}\left( Y_{1} - Y_{2} \right) + \chi_{DEOXY}\left( Y_{2} - Y_{1} \right) \right) = HCT\left( Y_{2} - Y_{1} \right)\left( - \chi_{OXY} + \chi_{DEOXY} \right)\]

Ponendo \(\Delta Y = \left( Y_{2} - Y_{1} \right)\) si ottiene:

\[\Delta\chi_{blood} = - HCT\Delta Y\left( - \chi_{OXY} + \chi_{DEOXY} \right)\]

Si mette in evidenza il segno negativo in parentesi, ottenendo:

\[\Delta\chi_{blood} = - HCT\Delta Y\left( \chi_{OXY} - \chi_{DEOXY} \right)\]

La variazione di suscettività \(\Delta\chi_{blood}\) è sufficiente per poter essere misurata dalla risonanza magnetica, nel senso che origina una variazione di segnale del voxel, per la variazione del vettore di magnetizzazione, apprezzabili per gli strumenti di risonanza magnetica. Si può dimostrare che il segnale del sangue deossigenato ha una suscettività magnetica maggiore di circa il \(20\%\) rispetto al sangue completamente ossigenato. Ne discende che il segnale prelevato ha ampiezza minore, poiché, aumentando le disomogeneità di campo, il defasamento degli isocromati è più veloce.

Gli effetti che determinano le variazioni di segnale in funzione dell'attività cerebrale sono denominati come Blood Oxygenation Level Dependent (BOLD). Vari studi hanno determinato che il contrasto BOLD a seguito dell'attività cerebrale non è dovuto al fatto che la ossiemoglobina aumenta il segnale del voxel ricevuto ma dallo spostamento della desossiemoglobina a opera della ossiemoglobina, la quale aumenta il segnale di risonanza magnetica prelevato.

Nella metodica BOLD, il meccanismo di contrasto di un certo voxel dipende, in ogni caso, dal livello di ossigeno nel sangue.

Il fenomeno del BOLD dipende essenzialmente da due eventi concomitanti:

\begin{itemize}
\item
  Uno legato alla variazione di suscettività del sangue, \(\Delta\chi_{blood}\), che varia in base al consumo di ossigeno dei neuroni;
\item
  Il flusso sanguigno che trasporta il sangue deossigenato, sostituendolo con quello ossigenato.
\end{itemize}

Si considera un vaso sanguigno in prossimità di un neurone cerebrale. A causa di una stimolazione psicofisica o cognitiva, come un potenziale evocato o ERP, dei neuroni si attivano e, per generare il potenziale d'azione, hanno bisogno di ossigeno, prelevato dal liquido interstiziale. Si attiva, così, una risposta vascolare che porta all'afflusso di sangue ossigenato, il quale, dopo aver ceduto ossigeno al liquido interstiziale, defluisce come sangue deossigenato. La variazione della suscettività del sangue \(\Delta\chi_{blood}\) varia in modo abbastanza complesso, poiché, per l'attivazione della risposta allo stimolo di alcuni neuroni, vi è una riduzione dell'ossigeno nel tessuto cerebrale, aumentando la suscettibilità del sangue, L'afflusso di ossigeno e il deflusso di sangue deossigenato determina, poi, un aumento e una riduzione della suscettibilità del sangue.

\begin{figure}
\centering
\includegraphics[width=4.55669in,height=3.03764in]{media/14_FuncImm/image351.pdf}\caption{Figura .: Neurone che preleva ossigeno per generare il potenziale d\textquotesingle azione}
\end{figure}

Quando un'area del cervello si attiva a causa di uno stimolo, l'aumento dell\textquotesingle attività neuronale crea una richiesta di ossigeno e glucosio. Dato che l'ossigeno interstiziale viene inizialmente consumato, si attiva una risposta vascolare che produce un afflusso massiccio di sangue ricco di ossigeno. Questa sequenza di eventi non è istantanea e ha un andamento ben definito nel tempo, noto come \textbf{Risposta Emodinamica} o \textbf{HRF (Hemodynamic Response Function)}. La curva è caratterizzata da:

\begin{enumerate}
\def\labelenumi{\arabic{enumi}.}
\item
  \textbf{Latenza e Calo Iniziale (Initial Dip):} Subito dopo l\textquotesingle attivazione neuronale, vi è un brevissimo calo di ossigeno locale, dovuto al consumo immediato da parte dei neuroni. Questo può portare a un leggero e rapido calo del segnale fMRI (il cosiddetto \emph{initial dip}), sebbene questo effetto sia molto piccolo e difficile da misurare con gli attuali scanner.
\item
  \textbf{Picco Positivo:} A causa della forte richiesta metabolica, il cervello attiva una \textbf{risposta vascolare} che pompa molto più sangue ossigenato del necessario. Questo eccesso di ossigeno nel sangue, che non è immediatamente consumato, porta a un aumento della concentrazione di emoglobina ossigenata (diamagnetica) rispetto a quella deossigenata (paramagnetica). Questo si traduce in una \textbf{diminuzione della suscettibilità magnetica} locale e in un forte \textbf{aumento del segnale fMRI}. Il picco del segnale si raggiunge tipicamente circa \(5 \div 6\ s\) dopo l'attivazione neuronale.
\item
  \textbf{Undershoot Post-Stimolo:} Dopo che lo stimolo è terminato e l'attività neuronale torna ai livelli di base, il flusso sanguigno si riduce, ma il volume del sangue impiega più tempo a tornare ai livelli pre-stimolo. Questa discrepanza temporale provoca un temporaneo aumento della concentrazione di deossiemoglobina, portando a una \textbf{riduzione del segnale fMRI} al di sotto del livello di base (l\textquotesingle{}\emph{undershoot}).
\item
  \textbf{Ritorno alla Linea di Base:} Infine, sia il flusso che il volume sanguigno tornano gradualmente ai loro valori basali, e il segnale fMRI torna al livello di riferimento. L'intero ciclo può durare da \(15\) a \(20\) secondi, a seconda dell'individuo e della regione cerebrale.
\end{enumerate}

La risposta vascolare è, in altre parole, la reazione del flusso sanguigno a uno stimolo psicosomatico.

\begin{figure}
\centering
\includegraphics[width=6.68958in,height=3.80278in,alt={Immagine che contiene testo, linea, diagramma, Diagramma Il contenuto generato dall\textquotesingle IA potrebbe non essere corretto.}]{media/14_FuncImm/image352.pdf}\caption{Figura .: Risposta emodinamica a una stimolazione neuronale}
\end{figure}

Per ottenere la curva della risposta vascolare è necessario utilizzare delle metodiche estremamente invasive, che richiedono l'utilizzo di elettrodi posizionati all'intero dei vasi sanguigni cerebrali. Sull'uomo queste metodiche non sono mai state applicate per problemi etici. Le curve sono state misurate empiricamente solo su animali da laboratorio come scimmie.

Dato che l'evoluzione della risposta è dell'ordine del secondo, è necessario acquisire l'intero volume cerebrale in un tempo dell'ordine del secondo. Ciò è possibile mediante l'utilizzo delle sequenze rapide, solitamente la EPI.

Per acquisire l'intero volume cerebrale si utilizzano delle fette perpendicolari all'encefalo, così da sezionarlo nel migliore dei modi. La maggior parte degli studi fMRI utilizza \textbf{fette assiali oblique}, orientate \textbf{parallelamente al piano AC-PC} (che collega il \textbf{commissura anteriore} e la \textbf{commissura posteriore}), per ottimizzare la copertura del cervello e ridurre artefatti. In questo caso, l\textquotesingle asse di slice selection è \textbf{perpendicolare a questo piano obliquo}, quindi leggermente inclinato rispetto all\textquotesingle asse verticale.

\begin{figure}
\centering
\includegraphics[width=3.77639in,height=3.14429in,alt={Immagine che contiene testo, mappa, diagramma Il contenuto generato dall\textquotesingle IA potrebbe non essere corretto.}]{media/14_FuncImm/image353.pdf}\caption{Figura .: Sezione dell\textquotesingle encefalo}
\end{figure}

Inoltre, con l'utilizzo delle sequenze veloci EPI è possibile ottenere delle immagini pesate in \(T_{2}^{*}\), necessarie per rilevare le variazioni locali del campo magnetico dovute alle differenze di suscettività \(\Delta\chi_{blood}\). Ne discende che per la tecnica fMRI l'introduzione delle disomogeneità di campo principale è fondamentale per l'imaging. Infatti, in base al defasamento degli isocromati, è possibile risalire alle zone cerebrali con maggiore concentrazione di desossiemoglobina.

A causa dell'acquisizione molto rapida acquisizione delle immagini con sequenza EPI, il segnale prelevato potrebbe risultare molto rumoroso. Dunque, le tecniche fMRI richiedono una circuiteria più sofisticata e algoritmi di ricostruzione più robusti al rumore.

\begin{figure}
\centering
\includegraphics[width=5.56061in,height=3.33648in,alt={Immagine che contiene testo, linea, Diagramma, diagramma Il contenuto generato dall\textquotesingle IA potrebbe non essere corretto.}]{media/14_FuncImm/image354.pdf}\caption{Figura .: Segnale BOLD rumoroso}
\end{figure}

Si suppone di applicare uno stimolo visivo a un paziente sottoposto all'imaging funzionale con risonanza magnetica. Dopodiché non viene somministrato nessuno stimolo. In seguito, si applica uno stimolo uguale o diverso da quello precedente.

Durante l'applicazione dello stimolo on-off, quando è applicato lo stimolo psicosomatico, si verifica una risposta emodinamica nelle zone neurali coinvolte nell'elaborazione dello stimolo, come quello occipitale per la vista. Se il paziente reagisce allo stesso modo per ogni stimolo, le forma d'onda della risposta emodinamica è la stessa.

Questa ipotesi di linearità dell'encefalo non è sempre ben verificata a causa del fenomeno di assuefazione, in cui il cervello si adatta agli stimoli ripetuti. Per evitare questo fenomeno si applicano diversi stimoli consecutivi nel tempo, come la visioni di immagini diverse oppure uno stimolo visivo e uno uditivo in ordine casuale e altro ancora.

La presenza della risposta emodinamica si traduce in una variazione della suscettività magnetica. Ciò determina una variazione del segnale proveniente dal voxel, rappresentato con una luminosità diversa. Fissato un certo voxel, quindi, la sua luminosità varia nel tempo con un andamento uguale alla risposta emodinamica cerebrale.

\begin{figure}
\centering
\includegraphics[width=6.69306in,height=2.62708in,alt={Immagine che contiene testo, Diagramma, diagramma, linea Il contenuto generato dall\textquotesingle IA potrebbe non essere corretto.}]{media/14_FuncImm/image355.pdf}\caption{Figura .: Segnale emodinamico legato a diversi stimoli ripetuti}
\end{figure}

Nell'imaging EPI il segnale è molto, quindi, la risposta all'attività cerebrale non è chiaramente visibile; tuttavia, questa misura rumorosa è l'unica possibile sull'uomo. Si rende necessaria l'estrazione del segnale utile dal rumore.

Nella visualizzazione delle immagini finali solitamente si rappresenta il distretto anatomico di interesse, generalmente l'encefalo, con gradazioni di grigio, a cui si sovrappone l'immagine funzionale, indicante il grado di attivazione cerebrale, in pseudocolori, generalmente blu e giallo.

I dati ottenuti per la ricostruzione delle immagini permettono di ricostruire l'evoluzione temporale dell'attività cerebrale, dunque, oltre alle tre dimensioni spaziali vi è una quarta dimensione ovvero il tempo. Per tale motivo i dati sono detti quadridimensionali.

L'analisi del segnale è complicata dall'esistenza di voxel di zone encefaliche diverse che non rispondono selettivamente a un singolo stimolo psicosomatico ma si attivano con stimoli diversi tra loro. Questi voxel avranno un'attività permanente visualizzata sull'immagine funzionale con uno pseudocolore. I voxel che invece rispondono a un singolo stimolo, se si è lontani dalla condizione di assuefazione, rispondono allo stesso modo allo stesso stimolo. Ciò è particolarmente verificata quando gli stimoli sono forniti in ordine casuale al paziente. Tramite le varie risposte si ricostruisce una mappa dello stimolo.

\subsubsection{Tecniche di elaborazione della risposta emodinamica}\label{tecniche-di-elaborazione-della-risposta-emodinamica}

Per elaborare i segnali prodotti dalle variazioni di suscettibilità magnetica per l'attività cerebrale si ricorre ad algoritmi di ordinary least square o OLS. Date \(n\) misurazioni si cerca la retta che minimizza lo scarto quadratico medio tra i valori misurati e la retta teorica. I dati misurati possono essere espressi in forma matriciale come:

\[\overset{\underline{}}{y} = \overset{\underline{}}{\overset{\underline{}}{X}}\overset{\underline{}}{\vartheta}\]

Dove \(\overset{\underline{}}{\overset{\underline{}}{X}}\) è la matrice degli stimoli:

\[\overset{\underline{}}{\overset{\underline{}}{X}} = \begin{pmatrix}
x_{1} & 1 \\
x_{2} & 1 \\
 \vdots & \vdots \\
x_{n} & 1
\end{pmatrix}\]

Data la matrice \(\overset{\underline{}}{\overset{\underline{}}{X}}\) e misurato il valore delle uscite \(\overset{\underline{}}{y}\), è possibile ricavare i parametri della regressione lineare mediante il metodo della pseudoinversa:

\[\overset{\underline{}}{\vartheta} = \left( {\overset{\underline{}}{\overset{\underline{}}{X}}}^{T}\overset{\underline{}}{\overset{\underline{}}{X}} \right)^{- 1}\overset{\underline{}}{\overset{\underline{}}{X}}\overset{\underline{}}{y}\]

Questo metodo può essere applicato solamente se il cervello reagisce allo stesso modo quando sollecitato dallo stesso stimolo. Implicitamente, dunque, si assume un comportamento lineare del cervello. In quest'ottica la risposta emodinamica altro non è che la risposta impulsiva del cervello, visto come un sistema lineare.

In queste ipotesi, somministrando \(n\) volte lo stesso stimolo si ottengono \(n\) risposte uguali; tuttavia, a causa del rumore sovrapposto, i segnali registrati differiscono significativamente (Figura 13.22). Per estrarre il segnale di risposta emodinamica dal rumore si esegue il fit dei dati, ovvero si calcola l'ampiezza \(B_{1}\) della risposta che meglio approssima il segnale misurato, nel senso che minimizza lo scarto quadratico medio tra la curva teorica e i dati misurati.

Ovviamente, applicando stimoli diversi si ottengono tante risposte per quanti sono gli stimoli, ognuna con propria ampiezza caratteristica. Ad esempio, si suppone di applicare due stimoli psicosomatici al paziente. Si ottengono due risposte emodinamiche di ampiezza, rispettivamente, \(B_{1}\) e \(B_{2}\).

\begin{figure}
\centering
\includegraphics[width=6.68958in,height=3.97708in]{media/14_FuncImm/image356.pdf}\caption{Figura .: Risposte emodinamiche relative a due stimoli diversi}
\end{figure}

Si ripete \(n\) l'applicazione di entrambi gli stimoli. Le due risposte, sommate \(n\) volte, permettono di ottenere \(n\) misurazioni dalle quali è possibile estrarre le ampiezze \(B_{1}\) e \(B_{2}\) separatamente.

La risposta emodinamica del paziente ai due stimoli, ripetuti \(n\) volte. può essere espressa come:

\[\overset{\underline{}}{y} = \begin{pmatrix}
{\overset{\underline{}}{B}}_{1} & {\overset{\underline{}}{B}}_{2} & \overset{\underline{}}{1}
\end{pmatrix}\left( \begin{array}{r}
b_{1} \\
b_{2} \\
q
\end{array} \right)\]

\(\overset{\underline{}}{y}\) è data dalla risposta che avrebbe dato la prima stimolazione se fosse applicata da sola \({\overset{\underline{}}{B}}_{1}\), sommata con la risposta che avrebbe generato il secondo stimolo se fosse applicato da solo \({\overset{\underline{}}{B}}_{2}\). La colonna unitaria, \(\overset{\underline{}}{1}\), tiene conto di un eventuale offset presente della risposta. In questo caso, ogni colonna \({\overset{\underline{}}{B}}_{1}\) e \({\overset{\underline{}}{B}}_{2}\) è una risposta psicosomatica.

In questo caso il vettore dei parametri incogniti della risposta emodinamica è:

\[\overset{\underline{}}{\vartheta} = \left( \begin{array}{r}
b_{1} \\
b_{2} \\
q
\end{array} \right)\]

La mappa a pseudocolori ricostruita rappresenta l'intensità dei coefficienti \(b_{1}\) e \(b_{2}\) per ogni voxel.

Nota la matrice di design \(\overset{\underline{}}{\overset{\underline{}}{X}} = \begin{pmatrix}
{\overset{\underline{}}{B}}_{1} & {\overset{\underline{}}{B}}_{2} & \overset{\underline{}}{1}
\end{pmatrix}\), contenente le informazioni su come sono organizzate le risposte emodinamiche rappresenta la risposta emodinamica attesa. Per ogni stimolo, misurando i segnali effettivamente ottenuti \(\overset{\underline{}}{y}\), è possibile applicare il metodo OLS per determinare i parametri cercati:

\[\overset{\underline{}}{\vartheta} = \left( \begin{array}{r}
b_{1} \\
b_{2} \\
q
\end{array} \right) = \left( {\overset{\underline{}}{\overset{\underline{}}{X}}}^{T}\overset{\underline{}}{\overset{\underline{}}{X}} \right)^{- 1}\overset{\underline{}}{\overset{\underline{}}{X}}\overset{\underline{}}{y}\]

È possibile eseguire, con questa metodica, la regressione lineare su tutto il voxel contemporaneamente, poiché nel vettore \(\overset{\underline{}}{y}\) vi è la risposta complessiva del segnale per ogni voxel al variare del tempo.

Successivamente, l'algoritmo prevedere un'analisi statistica che sfrutta un \(T\)-test per controllare se il valor medio della distribuzione di valori ottenuti si discosta da un riferimento fissato. Il test non prevede la conoscenza della varianza del segnale e del rumore.

\subsubsection{Applicazioni della fMRI}\label{applicazioni-della-fmri}

La risonanza magnetica funzionale è molto utilizzata in ricerca per l'analisi cerebrali, al fine di comprendere il funzionamento dell'encefalo, in base alle aree che si arrivano sulla base dello stimolo somministrato al paziente.

Con la fMRI è possibile analizzare tutte le aree cerebrali, anche quelle più profonde come il diencefalo. Questa operazione, ad esempio, non può essere eseguita con l'elettroencefalogramma o EEG, in quanto questa metodica prevede l'acquisizione dei potenziali elettrici sullo scalpo del paziente e, dunque, fornisce un'informazione integrale dell'attività cerebrale. In altre parole, l'EEG permette di ottenere delle informazioni complessive di ciò che si verifica al di sotto dello scalpo, senza fornire indicazioni su quali zone del cervello sono state attivate.

Con la metodica fMRI, in vivo, è possibile analizzare quali zone dell'encefalo sono attivate in base allo stimolo somministrato. Per tale motivo, questa sezione delle neuroscienze è in rapido sviluppo ed è ben documentata in letteratura.

Esistono anche applicazioni cliniche della metodica fRMI; infatti, essa è molto adoperata per la diagnosi e il follow-up di malattie neurodegenerative come l'Alzheimer e il Parkinson. La risonanza funzione permette di perfezionare la diagnosi, fornendo dati oggettivi sul grado e sull'avanzamento della patologia. In questo modo è possibile scegliere la terapia e seguire il suo avanzamento con elevata precisione.

In campo neuropsichiatrico o forcese, la fMRI può essere utilizzata, ad esempio, per realizzare una macchina della verità.

Ci sono state anche applicazioni poco fortunate in termini di successo per il Brain-Computer-Interface (BCI), una metodica che si basa sulla misura dell'attività cerebrale per controllare un dispositivo elettronico. Con risonanza magnetica funzionale è possibile ottenere un'informazione molto più precisa poiché fornisce informazioni sia nello spazio sia nel tempo. Tuttavia, a causa degli elevati costi della strumentazione e il tempo necessario per acquisire i dati, al giorno d'oggi questa metodica è poco accessibile.

La qualità del segnale può essere aumentata utilizzando campi magnetici più intensi. La magnetizzazione all'equilibrio dipende dal quadrato del campo principale applicato, \(B_{0}^{2}\); dunque, aumentando \(B_{0}\) anche la magnetizzazione all'equilibrio \(M_{0}\) aumenta. Di conseguenza il segnale prelevato ha ampiezza maggiore a parità di rumore sovrapposto.

I campi principali più diffusi sono a \(1.5\ T\) anche se negli ultimi \(5 \div 6\) anni sono entrati in vigore anche scanner con campi principali da \(3\ T\). Per scopi di ricerca è possibile utilizzare anche campi da \(7 \div 9\ T\), mentre per animali si può arrivare anche a \(11\ T\).

L'aumento del campo principale accresce il costo delle apparecchiature e della loro gestione; inoltre, cambiano le sequenze di acquisizione poiché i tempi di rilassamento dipendono dal campo applicato. Per questi motivi commercialmente è ancora molto diffusa la strumentazione a \(1.5\ T\) su cui viene eseguita la fMRI in ambito clinico. Con questi scanner si ottengono immagini abbastanza rumorose che potrebbero portare a false letture o risultati poco attendibili senza una buona elaborazione del segnale.

\subsection{Diffusion weighted imaging}\label{diffusion-weighted-imaging}

La Risonanza Magnetica di Diffusione (\textbf{DWI}), nota anche come \textbf{Diffusion Weighted Imaging}, è considerata una tecnica di \textbf{imaging funzionale} o, più precisamente, una tecnica che fornisce informazioni \textbf{funzionali e microstrutturali} sul tessuto.

La metodica sfrutta la diffusione mediante moto browniano dei protoni di idrogeno nel corpo umano.

Data una particella immersa in un ambiente circostante, essa non è ferma ma si muove con un andamento casuale nel tempo per il semplice moto legato all'agitazione termica. Le particelle, in instanti di tempo diversi, occupano posizione diverse all'interno dell'ambiente in cui sono immerse.

La diffusione è governata dal coefficiente di Einstein \(\mathcal{D}\) che quantifica lo spostamento della particella nel tempo. Questa grandezza è data da:

\[\mathcal{D =}\dfrac{k_{B}T}{6\pi\eta r}\]

Dove \(k_{B}\) è la costante di Boltzmann, \(T\) la temperatura assoluta della particella, \(\eta\) la viscosità del fluido nel quale la particella si muove e \(r\) il raggio della particella supposta sferica.

Il coefficiente di diffusione dell'acqua a temperatura ambiente è:

\[\mathcal{D =}\dfrac{k_{B}T}{6\pi\eta r} = \dfrac{1.38 \cdot 10^{- 23}\ \dfrac{J}{K}295\ K\ }{6\pi \cdot 1.0\dfrac{m^{2}}{s}0.1 \cdot 10^{- 9}m} = \dfrac{1.38 \cdot 10^{- 23}\ \dfrac{kg}{K}m/295\ K\ }{6\pi \cdot 1.0\dfrac{m^{2}}{s}0.1 \cdot 10^{- 9}m} = 2.2 \cdot 10^{- 12}\dfrac{m^{2}}{s} = 2.2\dfrac{pm^{2}}{s}\]

Dove \(k_{B} = 1.38 \cdot 10^{- 23}J/K\), \(T = 295\ K\), \(\eta = 1\ m^{2}/s\) e \(r = 0.1\ nm\).

In pochi istanti di tempo la molecola d'acqua percorre uno spazio di alcuni micron. La densità protonica, localmente, varia col tempo, di conseguenza, la magnetizzazione macroscopica nel tempo deve variare anch'essa per il fenomeno della diffusione protonica.

\begin{figure}
\centering
\includegraphics[width=3.53125in,height=3.53125in,alt={Immagine che contiene schizzo, disegno, clipart, design Il contenuto generato dall\textquotesingle IA potrebbe non essere corretto.}]{media/14_FuncImm/image357.pdf}\caption{Figura .: Random walk di uno spin}
\end{figure}

Nel 1956 Torrey propose l'aggiunta di due termini all'equazione di Bloch per tener conto del flusso e della diffusione dei protoni:

\[\dfrac{d\overset{\underline{}}{M}}{dt} = \gamma\overset{\underline{}}{M} \times {\overset{\underline{}}{B}}_{0} + \dfrac{M_{0} - M_{z}}{T_{1}}{\widehat{i}}_{z} - \dfrac{1}{T_{2}}{\overset{\underline{}}{M}}_{\bot} - \left( \overset{\underline{}}{\nabla} \cdot \overset{\underline{}}{v} \right)\overset{\underline{}}{M} + \overset{\underline{}}{\nabla} \cdot \left( \overset{\underline{}}{\overset{\underline{}}{\mathcal{D}}}\overset{\underline{}}{\nabla}\overset{\underline{}}{M} \right)\]

Dove \(\overset{\underline{}}{M}\) è il vettore di magnetizzazione del voxel con valore all'equilibrio \(M_{0}\), \({\overset{\underline{}}{B}}_{0}\) è il campo magnetico statico applicato, \(T_{1}\) e \(T_{2}\) sono i tempi di rilassamento del tessuto, \(\overset{\underline{}}{v}\) è il flusso e \(\overset{\underline{}}{\overset{\underline{}}{\mathcal{D}}}\) il tensore di diffusione con rango \(2\).

Il termine \(\left( \overset{\underline{}}{\nabla} \cdot \overset{\underline{}}{v} \right)\overset{\underline{}}{M}\) descrive il trasporto della magnetizzazione dovuto al flusso del fluido ed è noto come \textbf{advezione.} Questo termine è incluso solo se il fluido è comprimibile per garantire la conservazione della magnetizzazione. Per i fluidi biologici (come l'acqua nei tessuti) spesso si assume l'ipotesi di incomprimibilità è, dunque, può essere trascurato.

Infine, \(\overset{\underline{}}{\nabla} \cdot \left( \overset{\underline{}}{\overset{\underline{}}{\mathcal{D}}}\overset{\underline{}}{\nabla}\overset{\underline{}}{M} \right)\) è il termine di diffusione di Fick, che modella il flusso di magnetizzazione da aree ad alta a bassa concentrazione.

Si dimostra che, in caso di applicazione di un gradiente continuo, la componente trasversa della magnetizzazione evolve come:

\[M_{xy}(t) = M_{0}\exp\left( - \dfrac{t}{T_{2}} \right)\exp\left( - \gamma G^{2}\mathcal{D}\dfrac{t^{3}}{3} \right)\]

Dove \(G\) è l'ampiezza del gradiente applicato e \(\mathcal{D}\) la diffusività lungo la direzione trasversale. La derivazione della relazione appena citata coinvolge anche la distribuzione statistica delle frequenze dei vari isocromati.

Dalla relazione si osserva che l'attenuazione della magnetizzazione trasversale non dipende solamente dal tempo di rilassamento \(T_{2}\) ma anche dal coefficiente di diffusione del protone e dal gradiente applicato.

Nella pratica è possibile applicare delle sequenze che risaltano la diffusione degli isocromati. Prima di entrare nel dettaglio della sequenza, è comodo scrivere la magnetizzazione trasversale come:

\[M_{xy}(t) = M_{0}\exp\left( - \dfrac{t}{T_{2}} \right)\exp\left( - b\mathcal{D} \right)\]

Dove \(b\) è un coefficiente molto importante nel contesto di imaging di diffusione o Diffusion Weighting Imaging (DWI) ed è dato da:

\[b = \gamma G^{2}\dfrac{t^{3}}{3}\]

Mediante delle sequenze è possibile evidenziare come le molecole d'acqua si muovono nel tempo. A tale fenomeno corrisponde uno sfasamento degli isocromati che attenua il segnale come nella preazione per \(M_{xy}(t)\).

Il coefficiente \(b\), contenendo nella sua definizione il gradiente applicato, dipende dalla sequenza applicata per avere immagini pesate in diffusione. Indipendentemente dalla sequenza applicata la relazione:

\[M_{xy}(t) = M_{0}\exp\left( - \dfrac{t}{T_{2}} \right)\exp\left( - b\mathcal{D} \right)\]

Resta valida a patto di ridefinire il coefficiente \(b\).

La sequenza utilizzata è nota come Stejskal-Tanner o \emph{Pulsed Gradient Spin Echo} (PGSE) ed è composta da una spin-echo con gradiente pulsato detto di diffusione.

La sequenza prevede di applicare due gradienti, separati da un intervallo temporale \(\Delta\), di ampiezza \(G\) e durata \(\delta\) ai due lati di un impulso a radiofrequenza di \(\pi\). A precedere il primo gradiente vi è un impulso a \(\pi/2\).

La sequenza è sostanzialmente ottenuta dall'unione di una sequenza spin-echo con una gradient-echo. I due gradienti permettono di recuperare la disomogeneità di campo principale grazie all'impulso a \(\pi\), nel senso che al tempo d'echo, \(T_{E}\), tutti gli isocromati sono focalizzati lungo un asse del sistema rotante.

\begin{figure}
\centering
\includegraphics[width=6.69306in,height=3.83611in,alt={Immagine che contiene schermata, linea, diagramma, Diagramma Il contenuto generato dall\textquotesingle IA potrebbe non essere corretto.}]{media/14_FuncImm/image358.pdf}\caption{Figura .: Sequenza Stejskal-Tanner}
\end{figure}

Quando gli isocromati sono in movimento a seguito del primo impulso a radiofrequenza, lo sfasamento può essere modellato come la sovrapposizione dello sfasamento legato al primo gradiente:

\[\phi_{1} = \gamma\int_{0}^{\delta}{Gxdt} = \gamma G\delta x\]

Dove \(x\) è la posizione spaziale occupata dallo spin.

Il secondo contributo è lo sfasamento legato al secondo gradiente, applicato quando uno spin si è spostato dalla posizione \(x\) a \(x'\):

\[\phi_{2} = \int_{0}^{\delta}{Gx'dt} = \gamma G\delta x'\]

Tra l'applicazione del primo gradiente, quando lo spin è in posizione \(x\) e il secondo gradiente, quando lo spin è in posizione \(x'\), gli isocromati acquisiscono una differenza di fase:

\[\phi = \phi_{2} - \phi_{1} = \ \gamma G\delta x - \gamma G\delta x' = \gamma G\delta\left( x - x' \right)\]

Dopo l'applicazione dell'impulso a \(\pi\) la somma complessiva delle fasi è non nulla poiché \(x \neq x'\). Se lo spin resta nella stessa posizione, \(x = x'\), lo sfasamento è nullo e ciò porta a una fase nulla, condizione che si veridica al tempo d'echo.

Lo sfasamento introdotto dalla diffusione è legato al termine \(\exp\left( - b\mathcal{D} \right)\) dell'espressione della magnetizzazione trasversa.

Per la sequenza Stejskal-Tanner si dimostra che il termine \(b\) è dato da:

\[b = \gamma^{2}G^{2}\delta^{2}\left( \Delta - \dfrac{1}{3}\delta \right)\]

Con questa specifica sequenza, \(b\) è legato alla durata \(\delta\) dei due gradienti e all'intervallo di applicazione tra i due \(\Delta\). Lo sfasamento è quindi noto controllando l'ampiezza e la durate dei due gradienti. In conclusione, manipolando i gradienti di diffusione è possibile modificare il parametro \(b\) al dine di enfatizzare la presenza di sfasamento legato alla diffusione. Ovviamente, quando \(b\) aumenta, la magnetizzazione trasversale si riduce.

Tipicamente si acquisiscono più immagini con diversi valori di \(b\). A meno de rumore, i dati si distribuiscono su una curva esponenziale decrescente. Ripetendo la misura è possibile stimare il coefficiente \(b\) e, dunque, la diffusività \(\mathcal{D}\) dei protoni, mediante un algoritmo OLS a valle della linearizzazione della curva teorica.

\begin{figure}
\centering
\includegraphics[width=6.68958in,height=4.36389in]{media/14_FuncImm/image359.pdf}\caption{Figura .: Dati esponenziali dispersi a causa del rumore}
\end{figure}

Il coefficiente di diffusione è fondamentale poiché, in base a come l'acqua diffonde nel tessuto, è possibile discriminare un distretto anatomico sano da uno neoplastico. Ad esempio, l'acqua libera presente un ben determinato coefficiente di diffusione nel tessuto di interesse, mentre quando diffonde in un distretto anatomico altamente irregolare a causa della neoplasia, il coefficiente di diffusione risulta aumentato. L'aggressività del tumore è, quindi, valutata stimando il coefficiente di diffusione \(\mathcal{D}\), poiché maggiore è quest'ultimo e maggiore è l'irregolarità strutturare e, di conseguenza, più aggressivo è il tumore.

\subsubsection{Trattografia}\label{trattografia}

L'analisi dei coefficienti di diffusione lungo le tre dimensioni spaziali è molto importante anche nello studio del funzionamento cerebrale. È possibile eseguire una trattografiam ovvero una modellazione tridimensionale utilizzate in neuroscienze per visualizzare i tratti neurali, ovvero fasci di assoni che trasportano informazioni sensitive o motorie all\textquotesingle interno del midollo spinale e del cervello

In questa tecnica la diffusione non è studiata analizzando il comportamento complessivo nello spazio, come accade per la diffusion weigth imaging classico, ma i gradienti di diffusione sono applicati lungo i tre assi dello spazio, così da ottenere il coefficiente di diffusione \(\mathcal{D}\) lungo i tre assi mediante apposite immagini di diffusione. In questo modo, per ogni voxel è noto come l'acqua diffonde lungo gli assi.

Arrangiando i gradienti in modo particolare è possibile ottenere il tensore di diffusione contenente, oltre al coefficiente di diffusione lungo gli assi, anche i coefficienti di diffusione nelle direzioni non canoniche, ottenute combinando le proiezioni lungo gli assi. Questa metodica è nota come Diffusion Tensor Imaging (DTI)

Il tensore di diffusione è esprimibile come matrice \(3 \times 3\) simmetrica, data da:

\[\overset{\underline{}}{\overset{\underline{}}{\mathcal{D}}} = \begin{pmatrix}
\mathcal{D}_{xx} & \mathcal{D}_{xy} & \mathcal{D}_{xz} \\
\mathcal{D}_{xy} & \mathcal{D}_{yy} & \mathcal{D}_{yz} \\
\mathcal{D}_{xz} & \mathcal{D}_{yz} & \mathcal{D}_{zz}
\end{pmatrix}\]

Il tensore di diffusione permette di conoscere per ogni voxel qual è la direzione preferenziale del moto dell'acqua.

Nel DWI una volta ottenuti i coefficienti di diffusione lungo gli assi si opera una sorta di media così da calcolare come, all'incirca, l'acqua diffonde isotropicamente nel tessuto, quindi su una sfera. Al contrario, nel DTI si ottiene la direzione preferenziale lungo cui l'acqua diffonde.

La DTI è utilizzata a livello cerebrale per l'individuazione della direzione preferenziale del moto d'acqua di una determinata regione. L'assone, infatti, presenta una struttura con direzione preferenziale.

Studiando la direzione di diffusione del voxel contenente un insieme di assoni allineati tra loro è possibile evidenziare una direzione preferenziale del moto di acqua grazie al tensore di diffusione. Nella pratica si ottengono delle immagini, dette trattografie cerebrali che forniscono informazioni globali su come sono orientati gli assoni. Note le direzioni preferenziali è possibile tracciare delle linee, in pseudocolori, che collegano le varie parti dell'encefalo.

\begin{figure}
\centering
\includegraphics[width=3.17431in,height=2.81042in,alt={Fig 1}]{media/14_FuncImm/image360.pdf}\caption{Figura .: Trattografia cerebrale}
\end{figure}

Le linee, dal punto di vista analitico, presentano punto per punto come tangente il tensore di diffusione.

In vivo non esiste nessun'altra tecnica che permetta di studiare i collegamenti dei vari assoni cerebrali.

In aggiunta alla tecnica fMRI, la DTI permette anche di comprendere il motivo per cui certe zone dell'encefalo si attivano in concomitanza di certi stimoli, migliorando l'informazione fornita dalla fMRI.

Con il tensore di diffusione \(\overset{\underline{}}{\overset{\underline{}}{\mathcal{D}}}\) si evita di eseguire un numero enorme di misure, una per ogni direzione dello spazio \(\overset{\underline{}}{v}\). Per ottenere come una particella diffonda lungo una direzione \(\overset{\underline{}}{v}\) con coseni direttori \(\vartheta\) e \(\varphi\):

\[\overset{\underline{}}{v} = \left( \begin{array}{r}
\begin{array}{r}
\sin\varphi\cos\vartheta \\
\sin\varphi\sin\vartheta
\end{array} \\
\cos\varphi
\end{array} \right) = \begin{pmatrix}
v_{x} \\
v_{y} \\
v_{z}
\end{pmatrix}\]

È necessario applicare il prodotto scalare tra il tensore di tensore di diffusione e la direzione:

\[\overset{\underline{}}{\overset{\underline{}}{\mathcal{D}}} \cdot \overset{\underline{}}{v} = \begin{pmatrix}
\mathcal{D}_{xx} & \mathcal{D}_{xy} & \mathcal{D}_{xz} \\
\mathcal{D}_{xy} & \mathcal{D}_{yy} & \mathcal{D}_{yz} \\
\mathcal{D}_{xz} & \mathcal{D}_{yz} & \mathcal{D}_{zz}
\end{pmatrix} \cdot \begin{pmatrix}
v_{x} \\
v_{y} \\
v_{z}
\end{pmatrix} = \begin{pmatrix}
\mathcal{D}_{xx}v_{x} + \mathcal{D}_{xy}v_{y} + \mathcal{D}_{xz}v_{z} \\
\mathcal{D}_{xy}v_{x} + \mathcal{D}_{yy}v_{y} + \mathcal{D}_{yz}v_{z} \\
\mathcal{D}_{xz}v_{x} + \mathcal{D}_{yz}v_{y} + \mathcal{D}_{zz}v_{z}
\end{pmatrix}\]

Ovviamente, con la metodica DTI non è possibile ricostruire tutti gli assioni ma solamente la direzione preferenziale di un gruppo di essi, contenuto nel voxel con dimensione dell'ordine di \(1\ mm\).

La metodica DTI, insieme alla fMRI, è utilizzata in clinica per la diagnosi e il follow-up delle malattie neurodegenerative. È, inoltre, possibile evidenziare il corretto collegamento dei circuiti cerebrali o la presenza di patologie o interventi pregressi che hanno modificato tali collegamenti.

\begin{center}
\vfill
    \chapter{Hardware della risonanza magnetica}
    \label{blx:HWRMI\therefsection}
\vfill

\minitoc
\newpage
\end{center}
\justify


\subsection{Struttura complessiva della risonanza magnetica}\label{struttura-complessiva-della-risonanza-magnetica}

Tutte le molteplici funzionalità sono realizzare mediante l'hardware estremamente complicato della risonanza magnetica, necessario per la generazione dei campi magnetici di grande intensità o frequenze opportune.

Il magnete che genera il campo statico principale rappresenta la maggior parte del gantry, tuttavia, oltra a quest'ultimo sono presenti altri magneti per la gestione dei gradienti (5), la correzione delle disomogeneità di campo e la cancellazione del campo magnetico principale all'interno del ganty (2). Le prime antenne sono dette di shimming e le seconde di shielding.

Lo schema complessivo dell'apparecchiatura di risonanza magnetica prevede che il paziente sia posizionato sul lettino con proprietà amagnetiche e situato nel gantry (7). Ci sono poi degli avvolgimenti in spire o bobine per l'applicazione dei gradienti o di impulsi a radiofrequenza (6). Al fine di applicare questi campi, gli avvolgimenti sono opportunamente arrangiati nello spazio. Gli avvolgimenti primari del campo magnetico principale, invece, hanno generalmente una forma cilindrica (4).

Oltre questi elementi vi sono delle antenne poste sul paziente per la ricezione del segnale proveniente da zone più piccole del paziente. Queste antenne presentano la caratteristica di possedere un rapporto segnale/rumore più alto rispetto a quello ottenibile usando le stesse antenne di eccitazione per la ricezione. Ciò è legato al fatto che il segnale di rumore è tanto maggiore quanto maggiore è l'area della spira ricevente. Per tale motivo le antenne di ricezione possiedono un'area molto minore di quelle a radiofrequenza.

In una normale sequenza di eccitazione l'impulso a radiofrequenza è trasmesso dagli avvolgimenti di eccitazione presenti nel gantry e il segnale è registrato dalle antenne poste sul corpo del paziente.

Il maggiore rapporto segnale/rumore è legato anche alla stretta vicinanza con i distretti anatomici di cui si vuole eseguire l'imaging. Il segnale ricevuto da un'antenna superficiale è dovuta principalmente alle regioni del corpo sottostanti l'antenna e, in minima parte dalle zone circostanti.

Gli avvolgimenti sono leggermente sovrapposto tra loro in una modalità detta phased array, ovvero un insieme di antenne con una certa relazione di fase tra loro grazie a un'opportuna distanza e disposizione geometrica.

\begin{figure}
\centering
\includegraphics[width=5.91667in,height=2.41667in,alt={Signal-to-noise ratio comparison of phased-array vs. implantable coil for rat spinal cord MRI - ScienceDirect}]{media/15_HWRMI/image361.pdf}\caption{Figura .: Phased array}
\end{figure}

Il segnale complessivamente ricevuto deve essere ottimale nel senso che è necessario evitare le interferenze tra i vari ricevitori a interferenza distruttiva tra la magnetizzazione in una zona ricevuta da bobine affiancate. Il segnale ricevuto in quest'ultimo caso è inferiore. La necessità del phased array si comprende poiché è necessario evitare i fenomeni di interferenze.

Esistono altri oggetti nella risonanza magnetica come il contenitore che racchiude il gantry. Per le applicazioni con campi principali da \(1.5\ T\) in su questo involucro contiene una crio-camera contenente elio liquido (1) utilizzato per mantenere la temperatura del magnete principale al di sotto di una certa temperatura, prossima allo zero assoluto. Il criostato è previsto nelle apparecchiature di risonanza con un campo magnetico, generato da uno superconduttore.

\begin{figure}
\centering
\includegraphics[width=6.69306in,height=6.70278in,alt={Immagine che contiene schizzo, diagramma, disegno, Disegno tecnico Il contenuto generato dall\textquotesingle IA potrebbe non essere corretto.}]{media/15_HWRMI/image362.pdf}\caption{Figura .: Schema a blocchi della risonanza magnetica}
\end{figure}

Gli avvolgimenti di shimming permettono di ridurre la disomogeneità di campo, anche in base al paziente. Un buono shimming è fondamentale per ridurre le disomogeneità di campo, rendendo l'imaging più preciso. Invece, gli avvolgimenti di shielding o schermatura permettono di ridurre gli effetti del campo magnetico all'esterno del gantry. Questi due avvolgimenti si rendono necessari perché è fisicamente impossibile generare campi magnetici estremamente omogenei nell'area del paziente, il quale presenta dimensioni anche importanti, dell'ordine di \(90 \div 100\ cm \times 50 \div 60\ cm\). Aggiungendo campi magnetici al campo principale si riesce a garantire una disomogeneità di campo \(\Delta B\) dell'ordine di \(1\ ppm\).

Tutti gli elementi all'interno del gantry sono connessi mediante cavi che passano attraverso una camera schermata. Quest'ultimo componente, molte simile a una gabbia di Faraday, permette di ridurre le interferenze dei campi magnetici esterni. Se la struttura schermante non è completamente chiusa, ad esempio, lasciando aperta la porta, le immagini ottenute con sequenze sono molto sensibili alle disomogeneità presentano un livello di rumore molto intenso che riduce il contrasto.

L'ambiente circostante è, infatti, immerso in disturbi a radiofrequenza con frequenza di \(88 \div 108\ MHz\), vicine alle frequenze degli impulsi e dei segnali in risonanza. Ad esempio, con campi di \(1.5\ T\), la frequenza di precessione è di \(64\ MHz\), stesso ordine di grandezza dei campi esterni, che potrebbero influenzare i canali di ricezione, diventando un rumore sull'immagine.

Il personale che provvede alla preparazione del paziente e che acquisisce le immagini deve essere addestrato per chiudere correttamente la gabbia di Faraday.

La gabbia di schermatura è aperta in piccoli snodi in cui passano i cavi che trasportano il segnale registrato alla centrale di elaborazione. Il flusso di dati, ovviamente, non è unidirezionale in quanto è necessario trasmettere le configurazioni che la risonanza magnetica deve eseguire.

Le bobine di ricezione trasmettono il segnale, tramite appositi cavi, nella circuiteria di elaborazione, composta da un amplificatore a radiofrequenza, un demodulatore coerente, che porta il segnale in banda base e, infine, un convertitore analogico/digitale (ADC) che digitalizza il segnale per poi applicare gli algoritmi di ricostruzione delle immagini.

Per trasmettere la configurazione all'unità di controllo alle bobine è necessario un flusso di dati che dall'esterno della camera di Faraday giunga all'interno. In particolare, è necessario trasmettere segnali a radiofrequenza, legati agli impulsi di eccitazione verso le bobine che producono tali sollecitazioni.

L'impianto di trasmissione alle bobine si compone di un oscillatore alla frequenza di Larmor e un sintetizzatore digitale che modella la forma d'onda dell'impulso, ovvero esegue il suo shaping. Al fine di selezionare una singola fetta o il flip angle desiderato, la circuiteria deve essere tale da selezionare la durata opportuna dell'impulso. Inoltre, i segnali in uscita dal sintetizzatore devono essere sufficientemente intensi da poter eccitare il corpo del paziente. A tale scopo si utilizzano degli amplificatori opportuni.

Anche i gradienti sono modellati digitalmente per quanto riguarda la forma e le tempistiche. In seguito, sono amplificati e trasmessi alle bobine di gradiente, così da applicare la sequenza desiderata.

La ricostruzione dell'immagine avviene mediante elaboratori digitali i quali permettono anche la visualizzazione a video delle immagini tomografiche ottenute. La risonanza magnetica presenta sia un hardware che un software abbastanza complessi che, se congiunti, permettono di eseguire tutte le innumerevoli funzioni della metodica di imaging considerata.

È possibile considerare che, per ottenere una buona ricostruzione, almeno il \(50\%\) del lavoro è svolto dall'hardware mentre la restante parte dal software. L'hardware comprendere la generazione sia digitale sia analogica dei gradienti, dei campi a radiofrequenza e del campo principale.

Le risonanze di ultima generazione tendono a ridurre al minimo l'elaborazione analogica, spostando tutto il carico su quella digitale.

\subsection{Avvolgimento primario}\label{avvolgimento-primario}

Il magnete che genera il campo magnetico principale è la componente fondamentale della risonanza magnetica, poiché dalla sua intensità e omogeneità dipende l'imaging finale. Il magnete può essere realizzato con diverse tecnologie:

\begin{itemize}
\item
  Permanente;
\item
  Resistive;
\item
  Superconduttore.
\end{itemize}

La scelta della tecnologia dipende dai requisiti sul campo magnetico in termini di intensità, omogeneità spaziale, stabilità temporale, accessibilità da parte del paziente e costi di acquisto e gestione. Le apparecchiature moderne offrono campi principali che partano da qualche frazione di \(T\) fino a \(7 \div 11\ T\) nell'ambito della ricerca. In Italia sono largamente adoperati gli scanner a \(1.5\ T\) per la diagnosi clinica.

Il segnale di risonanza magnetica dipende dal quadrato del campo principale, \(B_{0}\), mentre il rapporto segnale/rumore è proporzionale a tale quantità; dunque, campi magnetici con intensità maggiore sono fondamentali al fine di avere immagini di buona qualità, caratterizzate da un basso rapporto segnale/rumore.

Tuttavia, in generale, nei tessuti biologici, per campi magnetici maggiori di \(0.5\ T\) il tempo di rilassamento cresce con l'intensità del campo; dunque, le sequenze progettate con un campo a bassa intensità non possono essere adoperate con campi più intensi, in quanto il contrasto in \(T_{1}\) sarebbe meno pronunciato. Al fine di mantenere lo stesso contrasto, al crescere del campo magnetico principale, il tempo di ripetizione \(T_{R}\) deve essere aumentato conseguentemente, aumentando i tempi di imaging. Come detto prima, l'aumento del campo principale ha il vantaggio di aumentare il rapporto segnale/rumore.

Cambiando il campo principale è necessario riprogettare anche la circuiteria logia di elaborazione, in quanto la frequenza di Larmor varia. Infatti, per un campo di \(3\ T\), la frequenza di precessione è:

\[\omega_{0} = \gamma B_{0} = 42.67\frac{MHz}{T}3\ T = 128MHz\]

In altre parole, raddoppiando il campo magnetico principale, anche la frequenza di risonanza di Larmor raddoppia. Da ciò discende che è necessario modificare gli oscillatori del mixer, gli amplificatori e la digitalizzazione del segnale stesso.

Ultimamente per la pratica clinica stanno avendo successo le macchine di risonanza magnetica a \(3\ T\), poiché permette di ottenere un SNR molto maggiore di quello che si avrebbe con un campo principale di \(1.5\ T\).

I campi con intensità di circa \(0.5\ T\) sono realizzati mediante magneti permanenti o resistivi, mentre i campi di intensità maggiore di \(1\ T\) adottano soluzioni a superconduttore, per cui sono molto pesanti e costose. Un tipico scannar da \(1.5\ T\) pesa generalmente \(6\) tonnellate, mentre quello da \(12\ T\), pesa \(12\ T\).

I magneti permanenti sono i meno costosi ma molto più pesanti, per cui è necessario adottare delle soluzioni per rafforzare il suolo al fine di sostenere il peso della strumentazione. Solitamente le apparecchiature sono poste al piano terra o interrato. Difficilmente la risonanza magnetica si trova ai piani superiori a meno che il suolo non sia stato adeguatamente rinforzato.

Si prova che la potenza depositata nei tessuti cresce con il quadrato del campo esterno applicato. Inoltre, la differenza tra le varie specie chimiche, in termini di shift della frequenza di risonanza, aumenta con l'intensità del campo. Di conseguenza, avere campi magnetici più intensi permette di ottenere ottime immagini spettroscopiche ma, allo stesso tempo, aumenta gli artefatti dovuti al chemical shift.

In genere, in campo clinico su utilizzano campo da \(0.2\ T\) fino a \(3\ T\). I magneti a basso campo, minore di \(1\ T\), sono generalmente aperti, per cui sono raccomandati laddove l'accesso del paziente è impossibilitato come in ortopedia.

Magneti fino a \(9\ T\) sono stati realizzati per ricerche neurologiche; tuttavia, per campi troppo intensi, la frequenza di Larmor, \(f_{0} = \overline{\gamma}B_{0}\), può assumere valori per cui la lunghezza d'onda del segnale a radiofrequenza sia confrontabile o minore dell'oggetto da discriminare. In tal caso, si possono generare delle onde stazionarie all'interno del corpo del paziente, visibili come aree di iperintensità o hot-spots. Queste onde indesiderate sono difficili da controllare perché legate alla geometria e alle proprietà elettriche del corpo.

\subsubsection{Magneti permanenti}\label{magneti-permanenti}

I magneti permanenti più potenti sono tipicamente costruiti con il materiale magnetico neodimio-ferro-boro (\(Nd_{2}Fe_{14}B\)). Il neodimio è una terra rara molto costosa per tale ragione non è utilizzata da solo ma in leghe con ferro e boro.

Con queste leghe è possibile ottenere campi magnetici con intensità non superiori a \(0.4\ T\). Campi più intensi renderebbero la struttura molto costosa e pesante. In ogni caso, questi magneti sono meno costosi degli scanner a \(1.5\ T\).

La risonanza magnetica a magnete permanente presenta una struttura a C, molto simili alle calamiti a ferro di cavallo. Vi sono due poli, tra i quali si genera il campo magnetico. Questi poli, appunto, sono realizzati con leghe di Neodimio.

La struttura a C è legata alla natura intrinseca del campo magnetico, le cui linee di flusso sono chiuse. Per facilitare la chiusura delle linee si usa una struttura a ferro di cavallo dove il paziente è posizionato al centro dei poi, così da essere nella regione col campo più omogeneo possibile.

Per guidare le linee di campo magnetico si utilizza un gioco di acciaio così da chiudere le linee di campo dal polo nord al sud. Questa struttura implica un peso di \(8 \div 10\) tonnellate.

\begin{figure}
\centering
\includegraphics[width=6.33422in,height=2.47951in,alt={Immagine che contiene Attrezzature mediche, macchina, design Il contenuto generato dall\textquotesingle IA potrebbe non essere corretto.}]{media/15_HWRMI/image363.pdf}\caption{Figura .: Schema della risonanza magnetica a magnete permanente ed esempio di tale scanner}
\end{figure}

Lo scanner a risonanza magnetica con magnete permanente è aperto, quindi, il paziente può comodamente occupare la posizione a lui riservata senza essere affetto da un senso di claustrofobia.

Grazie alla facilità di posizionamento, questo scanner è molto utilizzato in ortopedia, soprattutto nell'acquisizione di immagini del ginocchio o della caviglia.

Avendo un campo magnetico di intensità molto bassa, il rapporto segnale/rumore, proporzionale a \(B_{0}\), risulta essere molto limitato, dunque, le immagini acquisite sono caratterizzate da una rumorosità molto intensa. Di conseguenza, indagini che richiedono la visione di oggi con dimensioni molto ridotte, come per la mammografia, non possono essere eseguite con questo scanner. Anche le tecniche di imaging funzionale presentano una scarsa visione delle caratteristiche dei tessuti.

Le proprietà magnetiche delle leghe con cui sono realizzati i poli variano con la temperatura, la quale, di conseguenza, deve essere molto stabile. Generalmente il magnete è portato a una temperatura di \(32\ {^\circ}C\), più alta di quella ambientale. Le camere che contengono lo scanner sono mantenute alla temperatura desiderata da impianti di condizionamento, che garantiscono un'escursione massima di \(1\ {^\circ}C\).

Il calore al magnete viene fornito da una piccola resistenza di bassa potenza, tipicamente di \(200\ W\), così da bilanciare lo scambio termico tra magnete e ambiente.

La procedura di installazione prevede una fase di stabilizzazione in cui la temperatura è mantenuta costante per \(6 \div 7\) giorni, al fine di garantire un campo con intensità il più costante possibile. Per rendere più omogeneo il campo, inoltre, si inseriscono materiali magnetici o metallici che introducono dei campi tali da compensare le disomogeneità di campo.

Siccome il magnete permanente non consuma energia, il solo costo associato al suo funzionamento è offerto dall'impianto di condizionamento e di riscaldamento dei poli.

Ovviamente, se il magnete viene riscaldato eccessivamente perde le sue proprietà magnetiche. La temperatura a cui si verifica tale fenomeno è nota come temperatura di Curie e per il neodimio è di \(19\ K\), mentre per le leghe è di circa \(180\ {^\circ}C\).

\begin{longtable}[]{@{}
  >{\centering\arraybackslash}p{(\linewidth - 2\tabcolsep) * \real{0.5116}}
  >{\centering\arraybackslash}p{(\linewidth - 2\tabcolsep) * \real{0.4884}}@{}}
\caption{Tabella 16.1: Vantaggi e svantaggi dello scanner magnete permanente}\tabularnewline
\toprule\noalign{}
\begin{minipage}[b]{\linewidth}\centering
Vantaggi
\end{minipage} & \begin{minipage}[b]{\linewidth}\centering
Svantaggi
\end{minipage} \\
\midrule\noalign{}
\endfirsthead
\toprule\noalign{}
\begin{minipage}[b]{\linewidth}\centering
Vantaggi
\end{minipage} & \begin{minipage}[b]{\linewidth}\centering
Svantaggi
\end{minipage} \\
\midrule\noalign{}
\endhead
\bottomrule\noalign{}
\endlastfoot
non consuma energia & sensibile agli sbalzi termici \\
aperto & \(B_{0} < 0.4\ T\) \\
economico (per basso campo) & disomogeneo \\
\end{longtable}

\subsubsection{Magneti resistivi}\label{magneti-resistivi}

Un magnete resistivo è realizzato mediante degli avvolgimenti di un conduttore convenzionale percorso da corrente- Ovviamente, la maggior parte dell'energia fornita è convertita in energia termina, la restante piccola quota in energia magnetica. Si rende, dunque, necessario un sistema di dissipazione del calore ad acqua, che può raggiunge i \(50\ kW\), mentre per l'alimentazione del solenoide si utilizzano \(40\ kW\); per un totale di energia dissipata di \(90\ kW\) o maggiore.

Il campo magnetico può essere attivato o disattivato in base alla corrente che è fatta circolare negli avvolgimenti. Ciò rappresenta un grande vantaggio di questi scanner, poiché il consumo di energia e, quindi, i costi sono presenti solamente quando è erogata una prestazione.

La struttura dello scanner e a C, dove le estremità aperte rappresentano i poli, ovvero gli avvolgimenti resistivi. Per chiudere le linee di flusso si utilizza un giogo di acciaio che incanala il campo magnetico, riducendo la sua dispersione.

\begin{figure}
\centering
\includegraphics[width=6.18836in,height=2.65454in,alt={Immagine che contiene cartone animato, disegno Il contenuto generato dall\textquotesingle IA potrebbe non essere corretto.}]{media/15_HWRMI/image364.pdf}\caption{Figura .: Schema di uno scanner a magnete resistivo e un suo esempio}
\end{figure}

Il paziente è posizionato tra il gap di separazione tra le due elettrocalamite, quindi lo scanner è aperto. Per rendere la struttura più stabile nell'erogare il campo magnetico di solito si utilizzano due gioghi ferromagnetici.

Il campo magnetico prodotto dal magnate resistivo è minore di \(0.25\ T\) e non presenta la stessa omogeneità del campo prodotto da un magnete superconduttore. Per raggiungere un'alta omogeneità all'interno del field of view (FOV), il diametro dei poli deve essere maggiore di \(2.5\) volte il diametro del volume di imaging \(D_{FOV}\) e la separazione deve essere maggiore di \(1.5D_{FOV}\). Ad esempio, se si vuole un \(FOV = 33\ cm\) è necessario una separazione tra i poli di \(1.5 \cdot 33\ cm = 49.5\ cm\) mentre il diametro dei poli deve essere \(2.5 \cdot 33\ cm = 82.5\ cm\).

Il magnete resistivo è sensibile alla variazione di corrente, dell'ordine di \(180 \div 250\ A\); quindi, per minimizzare la disomogeneità di campo, è necessario misurare il campo magnetico concatenato ai conduttori e la differenza tra la misura e il valore desiderato è utilizzata in uno schema di feedback per la correzione della corrente che alimenta le bobine.

In genere, l'omogeneità richiesta per il campo è dell'ordine di \(1\ ppm\), quindi, la variazione di corrente non può essere maggiore di \(200 \cdot 10^{- 6}\ A\). Ciò, tuttavia, è altamente complesso da realizzare poiché la conducibilità del cavo conduttore si riduce con l'aumentare del calore, variando il campo complessivo. Dato che le correnti sono dell'ordine di \(180 \div 250\ A\), il meccanismo di feedback è molto complesso.

Per poter funzionare, i magneti resistivi dissipano molta energia sia per mantenere costante la corrente sia per mantenere la temperatura costante, dissipando il calore generato dall'enorme corrente.

Avendo campi magnetici di intensità molto limitata, le immagini sono molto rumorose rispetto a quelle ottenute con uno scannar a superconduttore. In assenza di un sistema di raffreddamento, la conducibilità varia con la temperatura, per cui anche il campo prodotto varia. Ciò può essere modellato come un campo principale con un gradiente applicato. Se ha, quindi, uno spostamento delle frequenze di risonanza degli isocromati, producendo un artefatto nelle immagini ricostruite.

\begin{longtable}[]{@{}
  >{\centering\arraybackslash}p{(\linewidth - 2\tabcolsep) * \real{0.5112}}
  >{\centering\arraybackslash}p{(\linewidth - 2\tabcolsep) * \real{0.4888}}@{}}
\caption{Figura .: Vantaggi e svantaggi dello scanner a magnete resistivo}\tabularnewline
\toprule\noalign{}
\begin{minipage}[b]{\linewidth}\centering
Vantaggi
\end{minipage} & \begin{minipage}[b]{\linewidth}\centering
Svantaggi
\end{minipage} \\
\midrule\noalign{}
\endfirsthead
\toprule\noalign{}
\begin{minipage}[b]{\linewidth}\centering
Vantaggi
\end{minipage} & \begin{minipage}[b]{\linewidth}\centering
Svantaggi
\end{minipage} \\
\midrule\noalign{}
\endhead
\bottomrule\noalign{}
\endlastfoot
consuma energia solo quando acceso & consuma molta energia \\
aperto & \(B_{0} < 0.25\ T\) \\
impianto di dissipazione calore & sensibili alle variazioni di corrente \\
\end{longtable}

\subsubsection{Magnete superconduttore}\label{magnete-superconduttore}

Gli scanner con magneti a superconduttore presentano le migliori caratteristiche in termini di stabilità e intensità del campo magnetico generato. Con questi magneti è possibile realizzare campi maggiori di \(0.3\ T\) con aperture di \(60\ cm\) del gentry.

Nel settore sanitario sono tipicamente utilizzati i campi da \(1.5\ T\), anche se di recente sono stati resi disponibili all'acquisto campi da \(3\ T\) in Italia. Per scopi di ricerca è possibile adoperare anche campi di \(7\ T\) per un massimo di \(9 \div 11\ T\) per indagini su animali. Per effetti biologici, in clinica si sceglie di non superare i \(3\ T\).

Gli scanner utilizzano materiali superconduttori nei quali scorre una corrente di un centinaio di ampere, senza che questa sia attenuata.

Il tipico scanner a magnete superconduttore presenta una geometria cilindrica con una apertura tipicamente di \(60\ cm\) di diametro. L'apertura può raggiungere anche i \(70\ cm\) o oltre per consentire l'accesso di pazienti obesi o claustrofobici.

\begin{figure}
\centering
\includegraphics[width=3.04209in,height=3.04209in]{media/15_HWRMI/image365.pdf}\caption{Figura .: Scanner con magnete a superconduttore}
\end{figure}

Il campo magnetico prodotto all'interno del gantry, indipendentemente dalle dimensioni, è molto omogeneo con delle variazioni di alcune parti per milione. L'asse maggiore del cilindro è orizzontale, a differenza delle altre due configurazioni a magnete resistivo o permanente.

All'esterno dell'avvolgimento l'intensità del campo decresce come \(r^{- 3}\). Infatti, per normative tecniche di sicurezza elettromagnetica, i campi nelle aree circostanti, a una certa distanza dal gantry, non devono essere superiori a \(0.5\ mT\). Per scanner a \(1.5\ T\), il limite va raggiunto entro \(3 \div 5\ m\), mentre per quelli a \(3\ T\) verso i \(7\ m\).Campi più intensi del limite imposto potrebbero influenzare i pacemaker, settandolo alla configurazione asincrona o a interferire altri con altri dispositivi elettronici.

\begin{figure}
\centering
\includegraphics[width=6.69306in,height=4.26944in,alt={Immagine che contiene testo, linea, Diagramma, schermata Il contenuto generato dall\textquotesingle IA potrebbe non essere corretto.}]{media/15_HWRMI/image366.pdf}\caption{Figura .: Campo magnetico interno ed esterno allo scanner}
\end{figure}

\paragraph{Materiali superconduttori}\label{materiali-superconduttori}

La proprietà di superconduttività non è semplice da spiegare poiché dovuta a fenomeni quantistici molto complessi. I materiali conduttori presentano una resistività \(\rho\) che dipende dalla temperatura secondo la fisica classica, da una legge lineare:

\[\rho = \rho_{0}\left( 1 + \alpha\left( T - T_{0} \right) \right)\]

Riducendo la temperatura, al limite per \(T \rightarrow 0\ K\), la resistività tende al valore\(\rho = \rho_{0}\left( 1 + \alpha T_{0} \right)\).

Esistono materiali per cui la resistività si abbatte molto più velocemente, fino ad annullarsi per temperature prossime allo zero assoluto, ma leggermente maggiori, dell'ordine della decina di Kelvin per i superconduttori definiti caldi.

\begin{figure}
\centering
\includegraphics[width=6.69306in,height=3.75833in,alt={Immagine che contiene testo, linea, diagramma, Diagramma Il contenuto generato dall\textquotesingle IA potrebbe non essere corretto.}]{media/15_HWRMI/image367.pdf}\caption{Figura .: Andamento della resistività al variare della temperatura per conduttore e superconduttore}
\end{figure}

Raffreddando il materiale superconduttore al di sotto della temperatura critica, \(T_{c}\), la sua resistività si annulla, così da permettere la circolazione di corrente elettrica, di intensità anche elevata, senza dissipazioni di energia.

Studi in letteratura hanno provato che la corrente circolante in un magnete superconduttore, a vent'anni dall'iniezione, restano praticamente inalterate, a meno della sensibilità dello strumento di misura utilizzato.

Il campo magnetico principale per questi scanner è generato da una corrente stabile nel tempo, dunque, si ha la sicurezza che non vi siano variazioni della sua intensità, rendendo l'imaging molto affidabile.

Ovviamente, per ottenere questi risultati il magnete deve essere mantenuto a basse temperature, inferiori al punto critico al quale la resistività si abbatte.

Il materiale superconduttore più utilizzato nella pratica di risonanza magnetica è il niobio-titanio (\(NbTi\)), la quale perde completamente le proprietà resistive a temperature prossime allo zero assoluto, ovvero \(7.2\ K\). Questa temperatura dipende dalla corrente che scorre nel materiale superconduttore; dunque, per campi intensi è necessario raffreddare il magnete a temperature inferiori.

Esistono altre leghe con proprietà simili, tuttavia, il niobio-titanio è preferibile proprio perché la temperatura di transizione assume valori maggiori rispetto a quella raggiunta dall'elio liquido e, inoltre, mantiene lo stato di superconduttore anche in presenza di un campo magnetico importante. La temperatura di fusione dell'elio è di circa \(4.2\ K\).

Esistono dei materiali ceramici che a \(77\ K\) sono superconduttori e, dunque, potrebbero essere immersi nell'azoto liquido, con un costo di circa \(0.20\, \text{\texteuro}\) al litro; piuttosto che in elio liquido, molto più costoso come \(10\, \text{\texteuro}\) al litro. Purtroppo, questi materiali perdono la proprietà di superconduttività quando sono immersi in un campo magnetico, dunque, non sono adatti per la risonanza magnetica.

Per creare un avvolgimento di niobio-titanio si inseriscono filamenti di questa lega in un supporto metallico in alluminio o rame, i quali non subiscono l'effetto del campo magnetico poiché materiali diamagnetici. Il supporto nel quale è immerso il filamento è detto coil former o formatore di bobina o supporto di avvolgimento.

Gli avvolgimenti primari sono composti tipicamente da \(6 \div 7\) filamenti di niobio-titanio, con diametro di \(20\ \mu m\) immersi in una matrice di rame o acciaio, in modo che lo spessore complessivo sia di \(2\ mm\). La matrice di rame è nota come copper matrix e permette di conferire le giuste proprietà meccaniche alla lega di niobio-titanio, che da sola risulterebbe estremamente delicata e fragile alla lettura.

\begin{figure}
\centering
\includegraphics[width=3.25045in,height=3.13585in,alt={Immagine che contiene cerchio, modello Il contenuto generato dall\textquotesingle IA potrebbe non essere corretto.}]{media/15_HWRMI/image368.pdf}\caption{Figura .: Matrice di niobio-titanio immersa nel supporto}
\end{figure}

Il rame non è un materiale superconduttore, quindi, presenta una resistenza elettrica molto maggiore del niobio-titanio, la quale è nulla:

\[R_{Cu} \gg 0\ \Omega\]

Il parallelo tra rame e lega di niobio-titanio è tale che la corrente resta confinata solamente nel materiale superconduttore, che rappresenta un corto circuito, mentre nel rame non circola corrente, in quanto schematizzabile come un circuito aperto.

Nel complesso, generalmente, l'avvolgimento è lungo \(4\ km\) circa, così che le correnti circolanti siano avvolte più volte intorno al coil formet, ottenendo il campo magnetico desiderato.

Ovviamente, ogni cavo immerso nel rame possiede una certa corrente; quindi, la corrente totale che circola nel sistema è data dalla somma delle correnti di ogni singolo filamento di niobio-titanio. Si può dimostrare che per campi dell'ordine del tesla, le correnti devono essere dell'ordine delle centinaia di ampere.

La geometria del magnete è tale che i campi prodotti dalle correnti circolanti in ogni singolo filamento di niobio-titanio si sommino, così da produrre un campo con grande omogeneità spaziale e stabilità temporale. Inoltre, siccome le correnti circolano indefinitivamente il campo magnetico è sempre acceso.

\begin{figure}
\centering
\includegraphics[width=2.18181in,height=2.36667in,alt={Immagine che contiene testo, schermata, calcolatore, illustrazione Il contenuto generato dall\textquotesingle IA potrebbe non essere corretto.}]{media/15_HWRMI/image369.pdf}\caption{Figura .: Struttura geometrica del magnete in cui il campo risultate è la somma dei campi prodotti dai filamenti di niobio-titanio}
\end{figure}

\begin{figure}
\centering
\includegraphics[width=5.48958in,height=4.89583in,alt={Immagine che contiene macchina, acciaio, Ricambio auto, metallo Il contenuto generato dall\textquotesingle IA potrebbe non essere corretto.}]{media/15_HWRMI/image370.pdf}\caption{Figura .: Struttura del magnete superconduttivo}
\end{figure}

Ancora, il consumo di energia da parte del criostato, utilizzato per mantenere la temperatura al di sotto della soglia \(7.2\ K\) e il consumo di elio liquido rendono l'apparecchiatura costosa anche in caso di non utilizzato. Dall'altro lato, l'elevata intensità, stabilità e omogeneità del campo permettono di ottenere immagini con un elevato rapporto segnale/rumore, così da rendere la ricostruzione più semplice e meno affetta da errori legati al rumore.

Per mantenere lo scanner attivo è necessario che il campo magnetico sia attivo, dunque, il sistema di raffreddamento a elio deve essere sempre in funzione. Ciò induce un dispendio energetico.

Tipicamente, gli scanner sono lunghi \(2.3\ m\) r alti \(2.5\ m\), mentre l'apertura del gantry è di \(0.9\ m\), al fine di ridurre il senso di claustrofobia dei pazienti. Anche in presenza di questi accorgimenti, il senso di chiusura potrebbe essere presente.

\begin{longtable}[]{@{}
  >{\centering\arraybackslash}p{(\linewidth - 2\tabcolsep) * \real{0.5166}}
  >{\centering\arraybackslash}p{(\linewidth - 2\tabcolsep) * \real{0.4834}}@{}}
\caption{Tabella 16.2: Vantaggi dello scanner a magnete superconduttore}\tabularnewline
\toprule\noalign{}
\begin{minipage}[b]{\linewidth}\centering
Vantaggi
\end{minipage} & \begin{minipage}[b]{\linewidth}\centering
Svantaggi
\end{minipage} \\
\midrule\noalign{}
\endfirsthead
\toprule\noalign{}
\begin{minipage}[b]{\linewidth}\centering
Vantaggi
\end{minipage} & \begin{minipage}[b]{\linewidth}\centering
Svantaggi
\end{minipage} \\
\midrule\noalign{}
\endhead
\bottomrule\noalign{}
\endlastfoot
Campo magnetico stabile e omogeneo & Consumo di elio (per la criogenia) \\
Non richiede alimentazione una volta acceso (modalità persistente) & Costosi (produzione e installazione) \\
Intensità di campo (\(\mathbf{B}_{\mathbf{0}}\)) da \(\mathbf{0.3\ T}\) fino a \(\mathbf{11\ T}\) (o più) & Sempre accesi (maggiormente difficile spegnere il campo) \\
Migliore rapporto segnale/rumore & Consumo di energia per il criostato (sistema di raffreddamento) \\
\end{longtable}

\subsubsection{Sistema di raffreddamento}\label{sistema-di-raffreddamento}

Gli avvolgimenti superconduttori sono inseriti all'interno di una struttura isolata dal punto di vista termico detta criostato. Questo elemento è realizzato tipicamente in acciaio non magnetico, contenenti schermi radiativi per impedire il trasporto di calore per diffusione, convenzione e conduzione.

Al fine di mantenere il niobi-titanio a una temperatura inferiore a quella critica, i filamenti di superconduttore sono immersi in elio liquido, contenuti in un helium vessel o vasca d'elio.

\begin{figure}
\centering
\includegraphics[width=5.74038in,height=5.60495in,alt={Immagine che contiene testo, schermata, diagramma, Parallelo Il contenuto generato dall\textquotesingle IA potrebbe non essere corretto.}]{media/15_HWRMI/image371.pdf}\caption{Figura .: Struttura dello scanner con in evidenza i coil former, il sistema di rareddamento (cry-cooler o cold-head)}
\end{figure}

A causa degli scambi termici con l'ambiente, l'elio tende a passare allo stato gassoso. Di conseguenza, la parte superiore della vasca tende a riempirsi con gas d'elio. Si rende, quindi, necessario un meccanismo che raffreddi il contenitore di elio affinché quest'ultimo sia alla temperatura di \(4.2\ K\).

Per ridurre la dissipazione di calore, il contenitore di elio è racchiuso in altre parati:

\begin{itemize}
\item
  Lo scudo termico (o thermal shield) è mantenuto a basse temperature dalla pompa criogenica (crycooler w/recondenser), ovvero una macchina frigorifera che lavora a temperature di circa \(20\ K\);
\item
  Successivamente si trovano delle camere a vuoto o vacuum vessel, molto simi a quelle presenti nei termos, che garantiscono un ulteriore isolamento verso l'ambiente in quanto il vuoto è il peggior conduttore di calore. Lo strato di vuoto tra il thermal shield e il vacuum vessel permette di ridurre la propagazione del calore verso l'interno della camera contenente l'elio liquido;
\end{itemize}

Il funzionamento della cold-head è basato sul ciclo di Gifford-McMahon che lavora tra gli \(80\ K\) e i \(20\ K\). Il ciclo sfrutta la proprietà di compressione e rarefazione del gas. Nel primo processo, la temperatura del gas aumenta, mentre nel secondo si riscalda. In particolare, la testa fredda contiene un compressore, un rigeneratore (composto da una struttura porosa) e un displacer.

\begin{figure}
\centering
\includegraphics[width=3.24291in,height=2.85833in,alt={rei\_01}]{media/15_HWRMI/image372.pdf}\caption{Figura .: Schema del ciclo di Gifford-McMahon}
\end{figure}

La valvola consente al gas di transitare tra il compressore verso la testa fredda e viceversa. Ai capi del compressore si trovano due lati: uno ad alta pressione e l'altro a bassa pressione; dunque, uno è ad alta temperatura e l'altro a bassa.

Il ciclo si compone essenzialmente di quattro fasi. Nella prima il gas ad alta pressione, tra i \(10 \div 20\ bar\) del compressore entra nella struttura attraverso il rigeneratore. Il gas, poi, finisce nella camera contenente il displacer il quale, contemporaneamente, si sopra verso sinistra. Il gas si espande passando dalla temperatura \(T_{L}\), minore di quella iniziale \(T_{H}\). Questo raffreddamento avviene poiché il gas cede calore al rigeneratore. Durante questo processo, la valvola ad alta pressione è chiusa, per cui il gas fluisce nella struttura porosa del dispacer, cendole calore.

Successivamente si chiude la valvola ad alta pressione e si apre quella a bassa pressione. Il gas si espande ulteriormente per fluire in verso opposto, riducendo ulteriormente la sua temperatura. Il dispacer è mantenuto in posizione fissa durate il processo mentre il gas assorbe calore dal lato a bassa temperatura.

Il dispacer viene, poi, spostato verso destra e il gas fluisce ancora verso il lato a bassa pressione, attraversando il rigeneratore e assorbendo calore da esso.

In fine, la cold-head viene riconnessa al lato ad alta pressione, il dispacer si sposta verso destra e il ciclo ricomincia.

\begin{figure}
\centering
\includegraphics[width=6.19452in,height=5.28356in]{media/15_HWRMI/image373.pdf}\caption{Figura .: Fasi del ciclo Gifford-McMahon}
\end{figure}

Ci sono, in definiva, due regioni dette heat-station di primo e secondo livello, che si trovano a temperatura, rispettivamente di \(80\ K\) e \(20\ K\). La loro funzione è quella di mantenere a basse temperature gli schermi cariostatici contenenti l'elio, così da minimizzare la quantità di liquido che passa allo stato gassoso.

L'efficienza del sistema è molto bassa, infatti, fornendo una potenza di \(6\ kW\) si ottiene una potenza di raffreddamento di \(100\ W\), la potenza consumata rende il sistema molto rumoroso.

Nella sala contenente lo scanner è possibile ascoltare un costante ticchettio ogni secondo, corrispondente al funzionamento del sistema di raffreddamento. Il dispacer è spostato ogni secondo per consentire il raffreddamento del magnete. Il punto a \(20\ K\) è collegato alla vasca contenente l'elio liquido e a quella termica, al fine di ridurre le dissipazioni di calore. La camera dell'elio è, quindi, mantenuta a \(20\ K\) per cui esiste una porzione di questa sostanza allo stato gassoso.

\begin{figure}
\centering
\includegraphics[width=6.69306in,height=3.19722in,alt={Immagine che contiene interno, Macchina utensile, ingegneria, Ricambio auto Il contenuto generato dall\textquotesingle IA potrebbe non essere corretto.}]{media/15_HWRMI/image374.pdf}\caption{Figura .: Posizionamento del sistema di raffreddamento nello scanner e cold-head}
\end{figure}

È importante mantenere la temperatura della camera e la quantità di elio gassoso costanti sia per conservare lo stato del niobio-titanio, sia per evitare l'esplosione della helium vacuum. Se, infatti, tutti l'elio dovesse passare allo stato gassoso, la pressione all'interno della camera sarebbe estremamente elevata da romperla.

Negli ultimi anni diversi scanner a magnete superconduttore sono stati equipaggiati con sistemi per la liquefazione dell'elio, che riutilizzano tale gas senza necessità di introdurlo nuovamente. Tali scanner sono noti come zero boil-off.

Per gli scanner che non dispongono di un sistema per il recupero dell'elio è necessario eseguire un'operazione di re-filling di gas ogni sei mesi o ogni anno circa, a seconda della qualità del criostato.

\subsubsection{Quench}\label{quench}

All'interno della camera contenente l'elio vi sono all'incirca \(2000\ L\) di elio liquido, dove, approssimativamente, \(1\ L\) di elio liquido corrisponde a \(600\ L\) di elio gassoso alla pressione atmosferica. Se tutto l'elio liquido contenuto nell'helium vessel, per effetto del calore ambientare, passasse allo stato gassoso, si avrebbe un volume di gas circa uguale a:

\[2000 \cdot 600\ L = 12 \cdot 10^{5} = 1.2\ ML\]

Questo volume di gas deve evacuare immeditatamente dalla camera dello scanner altrimenti, se il fenomeno di evaporazione non è controllato, possono verificarsi effetti esplosivim soprattutto in caso di conversione improvvisa. Ciò provoca un'elevata pressione che rompe il criostato.

Se, invece, vi è una fessura nel criostato, il gas fuoriesce nella camera dello scanner. L'elio di per sé è un gas inerte, dunque, non reagisce con i tessuti biologici; tuttavia, data il suo volume, può sostituirsi all'ossigeno, il quale fuoriesce dalla camera dello scanner. Un eventuale paziente che si trova nella camera dello scanner non avrebbe più ossigeno e potrebbe morire per asfissia. Per evitare ciò e aumentare la sicurezza del sistema si inserisce il quench.

Se l'isolamento termico non funzionasse correttamente, il filo superconduttore di niobio-titanio perderebbe localmente le sue proprietà superconduttive. Il filo mostrerebbe una resistenza elettrica sulla quale la corrente dissipa energia. Per effetto Joule il cavo di niobio-titanio si riscala e il calore generato si propaga nella struttura. Altri tratti del cavo conduttore perderebbero la proprietà di superconduttività, innescando un evento a valanga che porta la corrente di circa \(200\ A\) a circolare anche nel rame, poiché la sua resistenza diventerebbe paragonabile a quella del niobio-titanio. Ciò genererebbe ulteriori dissipazioni di energia che portano l'elio liquido a passare allo stato gassoso. In questa evenienza, la pressione all'interno del criostato aumenta e potrebbe essere tale da causarne la rottura. In gergo, questo fenomeno è detto quench e deve essere evitato al fine di evitare danni.

Al fine di aumentare la sicurezza, si inserisce il tubo di quench, ovvero un condotto di acciaio o rame che conduce il gas elio all'esterno della sala dello scanner, Ovviamente il tubo deve possedere una valvola che permette solamente il flusso di gas dall'interno verso l'esterno della camera, così da evitare la formazione di ghiaccio di ossigeno o di azoto nel tubo, che potrebbero ostruirlo.

In alcuni scenari, il fenomeno del quench potrebbe essere attivato volontariamente. Ad esempio, in caso di urgenza si può rendere necessario disattivare il campo magnetico principale in breve tempo. Per legge, nelle camere adiacenti la sala del magnete deve essere presente un interruttore detto quench-box che, se premuto, permette di spegnere il campo magnetico mediante l'evaporazione simultanea di tutto l'elio liquido.

Il quench-box arriva delle resistenze, dette di quench-heater, che riscaldano il superconduttore, innescando i meccanismi a valanga che portano alla fuoriuscita dell'elio liquido, attraverso il tubo di quench.

Ovviamente, lo spegnimento del campo prevede l'efflusso di elio verso l'ambiente esterno, comportando costi molti elevati poiché, in seguito, è necessario immettere nuovamente l'elio liquido nel helium vassel. Il costo dell'elio liquido è di circa \(10\, \text{\texteuro}\) al litro; per cui \(2000\ L\) di elio hanno un costo di circa \(20000\, \text{\texteuro}\).

Lo spegnimento del campo può essere indotto quando viene inserito un oggetto metallico nella sala del magnete. A causa dell'elevato campo magnetico, la sua forza di attrazione è così elevata da spingere l'oggetto metallico con velocità molto importanti verso lo scanner metallico. L'oggetto resta poi legato allo scanner mediante una forza estremamente intensa. Di conseguenza, per rimuovere il corpo metallico esterno è necessario spegnere il campo magnetico.

Se l'oggetto fosse sufficientemente grande potrebbe ostruire l'uscita del paziente dal grantry; per sicurezza, si rende necessario l'innesco del quench. Tale fenomeno è comunque molto raro nella pratica clinica.

\subsubsection{Fasi di ramp-up e ramp-down}\label{fasi-di-ramp-up-e-ramp-down}

La fase di ramp-up e ramp-down consistono, rispettivamente, nell'accensione e nello spegnimento controllato del campo magnetico principale.

Al dine di permettere la circolazione della corrente indefinitivamente nel tempo, il circuito superconduttore deve essere chiuso su sé stesso. L'iniezione di corrente non può avvenire mediante un interruttore poiché nessun conduttore riesce ad avere una resistenza paragonabile al superconduttore. Al fine di alimentare il cavo superconduttore si utilizza un elemento riscaldante detto switch-heater in grado di cedere calore a una porzione relativamente piccola del tratto di superconduttore posto sula sommità del magnete localmente accessibile dall'esterno. In questo tratto, il cavo di niobio-titation perde le proprietà superconduttive, assumendo una resistenza finita. Localmente, il materiale superconduttore, modellato come un induttore, risulta aperto rendendo possibile l'iniezione di corrente fino a raggiungere il valore desiderato.

\begin{figure}
\centering
\includegraphics[width=4.21934in,height=4.13021in,alt={Immagine che contiene testo, diagramma, linea, schermata Il contenuto generato dall\textquotesingle IA potrebbe non essere corretto.}]{media/15_HWRMI/image375.pdf}\caption{Figura .: Circuito di ramp-up e ramp-down}
\end{figure}

La corrente immessa nel superconduttore assume dei valori molto importanti, dai \(200\ A\) ai \(600\ A\) in base al magnete e l'intensità di campo desiderate. Nel processo bisogna tener conto anche dell'elevatissima induttanza del cavo di niobio-titanio che mantiene le sue proprietà superconduttive. L'iniezione rapida di corrente produrrebbe forti reazioni opposte da parte dell'induttore, per la legge di Faraday-Neumann-Lenz, che potrebbero generare effetti indesiderati.

Solitamente la corrente nel cavo superconduttore è variante con legge lineare con pendenza molto bassa, tale da far durare la fase di ramp-up diverse ore.

Durante la fase di ramp-up la tensione applicata di \(10\ V\) resta costante; tuttavia, per le elevate correnti raggiunte nel circuito, è necessario che il generatore sia munito di un sistema di raffreddamento ad acqua al fine di dissipare il calore prodotto.

Una volta raggiunto il valore di corrente desiderato nel superconduttore, lo switch-header è spento e il tratto riscaldato del superconduttore torna alla temperatura di lavoro. In questo modo il superconduttore si chiude nuovamente su sé stesso.

Per disattivare il campo in maniera controllata si procede con un meccanismo opposto al ramp-up, il ramp-down. Si collega il voltage supply e si riscalda localmente il superconduttore mediante gli switch-heater. La corrente scorre tra il superconduttore e il generatore esterno, dato che il tratto riscaldato si comporta come un circuito aperto.

La corrente nel circuito viene fatta decrescere lentamente, così da ridurre l'energia accumulata nell'induttore \(U_{m}\) molto lentamente:

\[U_{m} = \frac{1}{2}Li^{2}\]

La procedura di ramp-up è effettuata all'atto dell'installazione della macchina e può essere ripetuta varie volte durante il collaudo, fase nella quale si misura l'omogeneità del campo magnetico e si pone rimedio in caso di forti disomogeneità.

\subsubsection{Monitoraggio del consumo di elio}\label{monitoraggio-del-consumo-di-elio}

Il termine boil-off si riferisce alla quantità di elio che tende a passare dallo stato liquido allo stato di gas nell'unità di tempo, spesso assunta dell'ordine delle ore. Anche in presenza di isolamento termico, non è possibile eliminare l'effetto del passaggio di stato ma solo limitarlo nel tempo. Il boil-off è tipicamente di \(0.1\,\frac{L}{h}\).

Si rende necessario introdurre un meccanismo di monitoraggio del livello di elio nella forma liquida e in quella gassosa all'interno dell'helium vassel. Il monitoraggio è necessario in quanto, se una porzione eccessiva di elio passa allo stato gassoso, vi è il rischio che avvenga il fenomeno del quench. Inoltre, il gas potrebbe fuoriuscire da fessure, come il tubo di quench. In questo caso si perde la sicurezza che il niobio-titanio si comporti come un superconduttore, riducendo l'efficienza dello scanner.

Per monitorare i livelli di elio liquido si immerge un superconduttore all'interno dell'helium vassel per tutta la sua lunghezza. Il filamento è connesso a un circuito in contenente un generatore di impulsi di correnti e un dispositivo per la misura della tensione indotta dall'impulso.

\begin{figure}
\centering
\includegraphics[width=5.59453in,height=4.89652in,alt={Immagine che contiene diagramma, testo, cerchio, schizzo Il contenuto generato dall\textquotesingle IA potrebbe non essere corretto.}]{media/15_HWRMI/image376.pdf}\caption{Figura .: Principio di funzionamento di un monitor del livello di elio}
\end{figure}

La tensione misurata dipende dal livello di elio liquido presente nella vasa, poiché il filamento di niobio-titanio, essendo immerso solo parzialmente nell'elio liquido, presenta una porzione superconduttiva e la restante con comportamento resistivo, caratterizzato da una resistenza \(R\ \) non nulla. Dal punto di vista elettronico, è possibile modellare il filamento di niobio-titanio come una resistenza in serie a un corto circuito. In base alla porzione di cavo immerso nel liquido, quindi al livello di elio liquido presente nel serbatoio, la resistenza assume un determinato valore, proporzionale proprio alla porzione alla lunghezza di cavo a contatto con l'elio gassoso.

\begin{figure}
\centering
\includegraphics[width=5.18519in,height=4.31481in]{media/15_HWRMI/image377.pdf}\caption{Figura .: Schema circuitale del filamento di niobio-titanio parzialmente immerso nell\textquotesingle elio liquido}
\end{figure}

Ad esempio, se il filo è inserito al \(100\%\ \) nel liquido d'elio, la tensione è nulla; al contrario se la porzione di filo di niobio-titanio è inserito al \(70\%\) nell'elio liquido, la tensione misurata sarà uguale al \(30\%\) della tensione massima possibile \(V_{ma}\), ottenuta quando il cavo non ha proprietà superconduttive.

La corrente iniettata nel cavo è dell'ordine del centinaio di \(mV\ \), mentre la tensione letta è dell'ordine di qualche \(mV\ \), dunque, è necessario adoperare una catena, composta da un amplificatore analogico e un convertitore A/D, che fornisca direttamente l'indicazione sul livello di elio tramite un opportuno fattore di scala.

Siccome parte dell'elio passa allo stato gassoso e fuoriesce da varie aperture, è necessario un refilling, ovvero l'aggiunta periodica di nuovo elio liquido, generalmente da ogni \(6\ \)mesi a ogni anno.

Nella camera di controllo dello scanner, accanto al pulsante del quench vi è un monitor per controllare il livello di elio liquido. Se il valore di elio liquido si riduce al di sotto di una certa soglia, è suonato un allarme.

\subsubsection{Omogeneità del campo}\label{omogeneituxe0-del-campo}

La corretta formazione delle immagini in risonanza magnetica richiede un'elevata omogeneità del campo magnetico principale. L\textquotesingle imaging si basa, infatti, sulla selezione della slice e sulla codifica in frequenza, che non funzionerebbero correttamente in presenza di forti disomogeneità del campo, in quanto i vari isocromati risuonerebbero a frequenza diverse da quella attesa, ovvero quella di Larmor \(\omega_{0\,} = \gamma B_{0}\). Ancora, l'imaging di spettroscopia richiede un'alta omogeneità del campo date le piccole differenze di frequenza di precessione tra protoni del tessuto adiposo e gruppi \(CH_{3}\) presenti nei vari metaboliti.

È, dunque, necessario controllare l'omogeneità del campo principale in mod da ottenere una riduzione degli artefatti introdotti dalle variazioni non volute del campo stesso.

Quando i magneti sono installati, il campo magnetico presenta una disomogeneità di campo di circa \(100\ ppm\). Infatti, la presenza di materiali metallici e di altre eventuali apparecchiature in prossimità dello scanner può influenzare il campo magnetico principale.

Sebbene le case costruttrici garantiscano una disomogeneità di \(1\ ppm\), la presenza dell'ambiente all'interno e all'esterno della camera, nella struttura clinica adibita allo scanner, può modificare il campo magnetico principale. È, quindi, necessario adoperare delle compensazioni per ridurre l'effetto dell'ambiente.

L'omogeneità del campo è misurata all'interno di una sfera con diametro di circa \(20 \div 40\ cm\) detta \emph{Diameter Sphierical Volume} (DSV) al centro del magnete. La misura del campo all'interno della DSV è ottenuta mediante l'acquisizione di una serie di dati, distribuiti secondo una geometria prefissata, di solito situato sulla superficie sferica.

Misurando il campo su una superficie sferica, infatti, per il teorema di Gauss è possibile ricavare il campo all'interno del volume racchiuso dalla sfera stessa. Da questa misura, quindi, si ottiene una valutazione -della disomogeneità del campo all'interno della DSV.

Una disomogeneità di \(1\ ppm\) comporta un errore dell'ordine della decina di \(Hz\), con una frequenza di risonanza dei \(MHz\). Ad esempio, per un magnete a \(1.5\ T\), la disomogeneità è:

\[\Delta B = 42.6\frac{MHz}{T}1.5\ T\ 1ppm = 64\ Hz\]

Questa differenza di frequenza è dello stesso ordine di grandezza dello shift tra acqua e grasso, dove la differenza di campo è proprio di \(3.5\ ppm\). Ciò comporta una differenza di frequenza di precessione di circa \(220\ Hz\).

Per poter garantire un'omogeneità di campo dell'ordine di \(0.1\ ppm\) è necessario ricorrere alle operazioni di shimming passavo e attivo che riescono a compensare parzialmente le disomogeneità di campo.

L'omogeneità del campo viene misurata mediante sensori ad affetto Hall. Un sensore Hall è costituito da una sottile lastra di materiale \textbf{semiconduttore} (spesso silicio), attraverso la quale viene fatta scorrere una \textbf{corrente elettrica} (\(I\)) costante in senso longitudinale.

Quando il sensore è immerso in un \textbf{campo magnetico statico} (\(B_{0}\)) perpendicolare al flusso della corrente, la \textbf{Forza di Lorentz} agisce sulle cariche in movimento (elettroni):

\[F_{L} = q\left( \overset{\underline{}}{v} \times \overset{\underline{}}{B} \right)\]

Questa forza devia gli elettroni verso uno dei lati del semiconduttore.

L'accumulo di cariche su un lato e la corrispondente carenza sull\textquotesingle altro creano una \textbf{differenza di potenziale} misurabile attraverso i bordi del semiconduttore, perpendicolare sia alla corrente che al campo magnetico. Questa è chiamata \textbf{Tensione di Hall} (\(V_{H}\)). Quest'ultima è \textbf{direttamente proporzionale} all\textquotesingle intensità del campo magnetico:

\[V_{h} \propto IB_{0}\]

Mantenendo la corrente costante, il sensore misura l\textquotesingle intensità del campo magnetico statico semplicemente misurando la tensione generata. I sensori moderni integrano circuiti aggiuntivi per amplificare e linearizzare questo segnale.

\begin{figure}
\centering
\includegraphics[width=4.13542in,height=2.5625in]{media/15_HWRMI/image378.pdf}\caption{Figura .: Sensori a effetto Hall per misura del campo statico}
\end{figure}

\subsubsection{Shimming attivo}\label{shimming-attivo}

Lo shimming attivo consiste nell'iniezione di corrente in un avvolgimento, detto di shimming, per compensare le piccole disomogeneità di campo introdotte dal paziente. Quest'ultimo può introdurre delle disomogeneità di campo per la presenza sia del suo corpo, caratterizzato da un comportamento conduttivo, sia per la presenza di oggetti metallici che può contenere come delle otturazioni dentarie o protesi metalliche.

Le correnti negli avvolgimenti di shimming introducono un campo magnetico che si sovrappone al campo principale così da renderlo maggiormente omogeneo. Gli avvolgimenti di shimming attivo sono realizzati in materiale superconduttore controllato attivamente dall'operatore.

L'intensità del campo di shimming attivo è controllata mediante la valutazione dell'intensità di corrente e la posizione degli avvolgimenti nella struttura dello scanner, richiesti da parte del software dello scanner tramite lo sviluppo in armoniche sferiche.

Per particolari esami, come la spettroscopia, lo shimming attivo è effettuato sulla base di un'analisi preliminare delle distorsioni introdotte, valutate, ad esempio, con una sequenza spin-echo o FID.

Nell'imaging alle mammelle è necessario avere un'elevata omogeneità del campo poiché questi distretti anatomici sono posti all'estremità del FOV. è, dunque, importante avere uno shimming attivo per correggere le non idealità del campo per ogni paziente che si sottopone all'esame diagnostico.

\subsubsection{Campo disperso dallo scanner}\label{campo-disperso-dallo-scanner}

Il campo magnetico disperso è il campo presente all'esterno dello scanner. Allentandosi dal magnete, il campo decade come \(r^{- 3}\). In assenza di particolari accorgimenti, delle frange di campo disperso o \emph{fringe field} si estendono anche al di fuori della stanza in cui è contenuta l'apparecchiatura di risonanza magnetica.

\begin{figure}
\centering
\includegraphics[width=5.47993in,height=3.24003in,alt={Immagine che contiene diagramma, linea, cerchio, Diagramma Il contenuto generato dall\textquotesingle IA potrebbe non essere corretto.}]{media/15_HWRMI/image379.pdf}\caption{Figura .: Linee isolivello che mostrano l'intensità del fringe field a varie distanze}
\end{figure}

L'intensità di campo magnetico nelle aree attigue allo scanner, per motivi di sicurezza, non deve superare i \(0.5\ mT\). Campi di ampiezza maggiore potrebbero influenzare pacemaker e altri dispositivi elettronici. Ovviamente, esistono una serie di normative che regolano la distanza alle quali il campo deve subire una certa attenuazione.

Al fine di ridurre il campo esterno si utilizzano due tecniche:

\begin{itemize}
\item
  Il \emph{passive shielding} consiste nel montare blocchi di materiale ferromagnetico, come l'acciaio, in prossimità del magnete. In questo modo, le linee di campo sono confinate all'interno del materiale esterno, riducendo a dispersione del campo al di fuori del magnete. Purtroppo, la quantità di materiale ferromagnetico necessaria alla schermatura aumenta con l'intensità del campo magnetico principale. Ad esempio, per un campo di \(7\ T\) sono richieste \(600\) tonnellate di acciaio;
\item
  L'\emph{active shielding} consiste nell'aggiunta, nel criostato, di ulteriori avvolgimenti che creino un campo opposto al principale all'esterno dello scanner. Con questa soluzione il campo esterno al gantry decresce molto più rapidamente.
\end{itemize}

Gli avvolgimenti di shielding hanno un diametro maggiore dei primary coil del campo principale, così che il campo magnetico principale possa essere mantenuto con le caratteristiche desiderate, aumentando la corrente in entrambi gli avvolgimenti. L'avvolgimento primario e quello di shielding si respingono a vicenda, dunque, si rende necessaria una struttura di sostegno molto robusta, che bilanci le forze di repulsione.

\begin{figure}
\centering
\includegraphics[width=4.08306in,height=3.44167in,alt={Immagine che contiene testo, schermata, cerchio, arte Il contenuto generato dall\textquotesingle IA potrebbe non essere corretto.}]{media/15_HWRMI/image380.pdf}\caption{Figura .: Avvolgimenti di shielding e primario}
\end{figure}

Le forze attrattive sui materiali paramagnetici e ferromagnetici nei pressi di un magnete con shielding attivo risultano essere maggiori rispetto a un mangete non schermato poiché il gradiente di campo aumenta.

\subsubsection{Interferenze esterne}\label{interferenze-esterne}

I campi magnetici ed elettrici esterni al magnete possono alterare l'omogeneità del campo primario ed è, quindi, necessario utilizzare degli accorgimenti per ridurre le interferenze esterne.

Per risolvere il problema si pone un ulteriore avvolgimento superconduttore all'esterno del magnete principale, chiuso su sé stesso e non percorso da corrente. L'avvolgimento capta le interferenze intrappolandole al suo interno. Si induce, quindi, una corrente dell'ordine dei \(mA\) che deve essere rimossa mediante il fenomeno del quench ogni \(24\) ore.

Nei magneti a basso campo, come quelli permanente o resistivi, si utilizza un sistema noto come \emph{external fiel interference compensation} o EFI. Il sistema si compone di una sonda, che capta le interference, e da un sistema elettronico che inverte e amplifica il segnale da iniettare nelle bobine di gradiente. Si crea così un contro campo che annulla le interferenze.

\begin{figure}
\centering
\includegraphics[width=4.46667in,height=3.80017in,alt={Immagine che contiene testo, schermata, cerchio, Dispositivo di archiviazione dati Il contenuto generato dall\textquotesingle IA potrebbe non essere corretto.}]{media/15_HWRMI/image381.pdf}\caption{Figura .: Schema di feedback per l\textquotesingle attenuazione delle interferenze per magnetici permeanti o resistivi}
\end{figure}

\subsubsection{Avvolgimenti del campo magnetico principale}\label{avvolgimenti-del-campo-magnetico-principale}

Per misurare il campo magnetico si utilizzano dei sensori a effetto Hall, in cui, per effetto dei campi elettrici e magnetici, si genera una distribuzione di carica tale da generare una differenza di potenziale tra le due facce del sensore. Nota questa d.d.p. si risale al campo magnetico applicato.

Per generare il campo magnetico principale si utilizzano degli avvolgimenti coassiali dette bobine di Helmholtz. Gli avvolgimenti sono ortogonali all'asse \(z\), che passa per i loro centri.

\begin{figure}
\centering
\includegraphics[width=3.175in,height=3.54167in,alt={15-Bobine di Helmholtz. \textbar{} Download Scientific Diagram}]{media/15_HWRMI/image382.pdf}\caption{Figura .: Bobine di Helmholtz}
\end{figure}

Sia \(z_{k}\) la coordinata del centro della \(k\)-esima spira e \(a_{k}\) la posizione lungo la circonferenza. Questi due termini possono essere espressi in coordinate polari come:

\[\left\{ \begin{matrix}
a_{k} = R_{k}\sin\vartheta_{k} \\
z_{k} = R_{k}\cos\vartheta_{k}
\end{matrix} \right.\ \]

Per ogni avvolgimento è possibile scrivere un'espansione in armoniche sferiche per il campo magnetico:

\[\left\{ \begin{matrix}
B_{r}(R,\vartheta) = \frac{\mu}{2}\sum_{n = 1}^{\infty}{\frac{c_{n}}{n + 1}R^{n}P_{n}^{1}\left( \cos\vartheta \right)} \\
B_{z}(R,\vartheta) = \frac{\mu}{2}\sum_{n = 0}^{\infty}{c_{n}R^{n}P_{n}\left( \cos\vartheta \right)}
\end{matrix} \right.\ \]

Dove \(P_{n}\left( \cos\vartheta \right)\) è la funzione di Legendre e \(P_{n}^{1}\left( \cos\vartheta \right)\) la funzione di Legendre associate di primo tipo, grado \(n\) e ordine \(1\). Inoltre, le coordinate sferiche \((R,\vartheta)\) individuano la posizione del campo magnetico.

I coefficienti dello sviluppo \(c_{n}\) dipendono dal numero \(K\) degli avvolgimenti secondo la relazione:

\[c_{n} = - \sum_{k = 1}^{K}{I_{k}R_{k}^{- (n + 1)}\sin\vartheta_{k}P_{n + 1}^{1}\left( \cos\vartheta \right)}\]

\subsubsection{Sistema di generazione dei gradienti9}\label{sistema-di-generazione-dei-gradienti9}

I gradienti, al pari del campo principale, sono fondamentali per l'imaging in risonanza magnetica. In assenza di tali campi non è possibile selezionare la slice o eseguire l'operazione di codifica di frequenza.

Le bobine di gradiente sono collocate all'interno del gantry con specifici accorgimenti per ogni direzione con cui bisogna modificare il campo principale diretto lungo \({\widehat{i}}_{z}\). Di solito gli avvolgimenti di gradiente sono mantenuti su un supporto di resina ipossica che conferisce rigidità alla struttura; tuttavia, quest'ultima lo spazio per il paziente poiché il diametro del gantry passa da \(90\ cm\), in assenza delle bobine di gradiente, a \(60\ cm\).

I gradienti lungo \({\widehat{i}}_{z}\) sono generati da una coppia di avvolgimenti coassiali di raggio \(a\) e distanza \(d = \sqrt{3}a\) così da avere un gradiente lineare nello spazio degli avvolgimenti. Questo tipo di configurazione è nota come coppia di Maxwell.

\begin{figure}
\centering
\includegraphics[width=3.70885in,height=3.19836in,alt={Immagine che contiene schizzo, diagramma, disegno, linea Il contenuto generato dall\textquotesingle IA potrebbe non essere corretto.}]{media/15_HWRMI/image383.pdf}\caption{Figura .: Coppia di Maxwell per la generazione del gradiente lungo \(z\)}
\end{figure}

Le due bobine sono concentriche col magnete principale, ovvero possiedono un asse diretto lungo \({\widehat{i}}_{z}\), così che la variazione lineare prodotta dall'arrangiamento sia lungo quell'asse.

Progettando opportunamente, la coppia di Maxwell è possibile variare il campo linearmente nella regione tra le due bobine, in maniera omogenea. La linearità è assicurata nel DSV con diametro dell'ordine di \(50\ cm\).

I gradienti lungo le altre dimensioni presentano forme più complesse, generalmente formate da un rettangolo con il lato minore curvo. La corrente circola nella bobina, detta a sella o \emph{golay transvers gradient coil} producendo un campo lineare tra le due.

\begin{figure}
\centering
\includegraphics[width=6.11503in,height=4.64583in]{media/15_HWRMI/image384.pdf}\caption{Figura .: Golay transvers gradient coil}
\end{figure}

La bobina, diretta lungo un asse, genera un gradiente di campo lineare nella direzione ortogonale alle linee verticali dell'antenna. Si può, infatti, dimostrare che il campo prodotto da strisce parallele percorse da corrente uguale ma di segno opposto e dirette lungo \({\widehat{i}}_{z}\), generino un gradiente lineare lungo la direzione \({\widehat{i}}_{y}\). Ovviamente, ruotando l'arrangiamento in modo che sia disposto sulla direzione \({\widehat{i}}_{y}\), il campo prodotto varia linearmente con la coordinata \(x\).

\begin{figure}
\centering
\includegraphics[width=6.69306in,height=2.61736in]{media/15_HWRMI/image385.pdf}\caption{Figura .: Golay transvers gradient coil e campo prodotto}
\end{figure}

Il progetto delle bobine di gradiente è ottimizzato mediante una tecnica nota come target field che porta la bobina ad avere una forma a impronta digitale o fingerprint. La progettazione di queste strutture richiede l'utilizzo di software numerici che permettono di modellare le bobine immerse nella resina in base al campo voluto.

La costruzione delle bobine parte da una lastra di rame con spessore di circa \(30\ mm\), incise successivamente elettroliticamente. All'interno del rame, poi, viene fatta scorrere una corrente di circa \(500\ A\) per generare i gradienti voluti.

I moderni scanner presentano gradienti con intensità massima di \(40\ mT/m\) con un diametro delle bobine di \(60\ cm\); tuttavia, in particolari situazioni come l'imaging cerebrale, si possono utilizzare gradienti con intensità di \(80\ mT/m\) con diametro minore.

Il massimo valore del gradiente è limitato dalla tensione di alimentazione di circa \(2\ kW\) che fornisce una corrente di circa \(500\ A\). Questa corrente è dissipata in calore negli avvolgimenti di rame ed è, quindi, necessario adoperare un sistema di raffreddamento ad acqua per evitare la fusione delle bobine di gradiente.

Siccome i gradienti devono avere una durata di qualche millisecondo le correnti devono aumentare rapidamente il loro valore, passando dallo stato di riposo al regime in un tempo relativamente breve.

Si definisce slew rate, \(S_{\max}\), come il rapporto tra il massimo gradiente desiderato e il tempo di salita necessario per raggiungerlo:

\[S_{\max} = \frac{G_{\max}}{t_{rise}}\]

\begin{figure}
\centering
\includegraphics[width=5.29792in,height=2.80513in]{media/15_HWRMI/image386.pdf}\caption{Figura .: Slow rate del gradiente}
\end{figure}

È importante inoltre controllare e modellare l'omogeneità del campo nel tempo e nello spazio, il duty cycle nel tempo, il tipo di schermatura, la stabilita e la precisione con cui si producono i gradienti.

Quando la corrente nella bobina gradiente aumenta durante l'attivazione, per la legge di Lorentz, si induce una corrente nella stessa bobina che si oppone all'iniezione di corrente. Per tale motivo, il tempo di salita della corrente non può essere infintamente piccolo. Tipici valori vanno dai \(10\frac{mT}{m\ ms}\) fino a \(200\frac{mT}{m\ ms}\)

Collegando la bobina gradiente a una capacità si può ottenere un tempo di salita più breve grazie alle caratteristiche del circuito risonante che si viene a creare; tuttavia, lo svantaggio risiede nella dipendenza dei tempi di salita dalla frequenza di risonanza del circuito.

Un elevato valore di slew rate permette di ottenere tempi di salita più brevi, abilitando anche un imaging più veloce; infatti, se lo slew rate fosse troppo basso non sarebbe possibile acquisire immagini con sequenze echo-planar poiché quest'ultima richiede un sistema di gradienti molto rapido al fine di acquisire in un intervallo \(T_{R}\) potenzialmente linterno volume anatomico.

All'estremità del FOV la variazione dei gradienti è massima del campo magnetico prodotto. Si induce una fem legata alla variazione del campo magnetico nel tempo, la quale, durante la fase di salita, coincide proprio con lo slew rate:

\[fem \propto - \frac{dB}{dt} = S_{\max}\frac{Fov}{2}\]

Il campo magnetico variabile induce delle correnti nelle correnti conduttrici circostanti, quali il criostato, i supporti metallici degli avvolgimenti e nel paziente stesso, il quale presenta dei portatori di cariche ioniche nei fluidi biologici. Le correnti parassite indotte o \emph{eddy current} generano dei campi che si oppongono alla variazione del gradiente. Ciò modifica la forma del gradiente, la quale perde l'andamento desiderato:

\begin{itemize}
\item
  A causa dello slew rate non si hanno transizioni nette ma un tempo di salita finito;
\item
  Per le eddy current si ha un effetto risultante di tipo passa-basso dato dal campo indotto che si oppone al gradiente.
\end{itemize}

La forma del gradiente è data dalla combinazione dei due effetti, che determinano un andamento con tempi di salita finiti e un andamento smussato nel tempo.

\begin{figure}
\centering
\includegraphics[width=6.49583in,height=1.30833in]{media/15_HWRMI/image387.pdf}\caption{Figura .: Effetto delle eddy current sul gradiente}
\end{figure}

L'effetto delle eddy current può essere compensato mediante un active shielding che consiste nel posizionare un secondo sistema gradiente in grado di erogare un campo tale da compensare gli effetti delle variazioni all'esterno del FOV.

Inoltre, la forma della corrente erogata è struttura in modo da compensare gli effetti dello slew rete e delle eddy current. Invece di generare un gradiente con forma trapezoidale, si eroga una corrente in modo che il suo campo prodotto abbia un valore di picco più alto del valore costante. Il picco permette di compensare gli effetti passa-basso dei fenomeni parassiti. Analogamente, per avere una tradizione con una certa pendenza, si utilizza un picco negativo. L'effetto risultante è un gradiente di forma trapezoidale con un certo fronte di salita.

\begin{figure}
\centering
\includegraphics[width=6.33422in,height=1.75024in,alt={Immagine che contiene diagramma, schermata, Carattere, linea Il contenuto generato dall\textquotesingle IA potrebbe non essere corretto.}]{media/15_HWRMI/image388.pdf}\caption{Figura .: Impulso di correne per evitare gli effetti dello slew rate ed eddy current}
\end{figure}

Nota la distribuzione geometrica dei supporti conduttori e altri parametri all'interno dello scanner, con algoritmi numerici è possibile valutare la forma d'onda da erogare alle bobine di gradiente affinché il gradiente di campo magnetico abbia la forma voluta.

Non tenendo conto delle eddy current, si avrebbe un'errata ricostruzione delle immagini poiché, per la variabilità del gradiente nel tempo, non vi è più una corrispondenza biunivoca tra frequenza di precessione e posizione spaziale degli isocromati.

Le eddy current indotte nel corpo del paziente possono causare delle stimolazioni nervose o muscolari se la loro intensità supera i \(10\ \mu A\). Per tale motivo bisogna fare in modo che non si sviluppino correnti molto elevate sul paziente.

Un artefatto sull'immagine legato al sistema di generazione dei gradienti è l'overfolding artifact o artefatto da ripiegamento. Dato che il gradiente non si annulla sul bordo del FOV, esistono delle zone del corpo del paziente in cui insiste lo stesso valore del campo, ovvero due isocromati diversi, posizionati ad ascisse diverse risuonano alla stessa frequenza.

\begin{figure}
\centering
\includegraphics[width=2.8in,height=2.5982in]{media/15_HWRMI/image389.pdf}\caption{Figura .: Andamento dell\textquotesingle ampiezza del campo nel FOV}
\end{figure}

Le bobine di gradiente si trovano nel gantry e sono contenute all'interno del magnete principale. Queste bobine sono disposte nello spazio in modo da produrre gradienti lineari lungo le tre direzioni dello spazio.

\begin{figure}
\centering
\includegraphics[width=4.73333in,height=3.6538in]{media/15_HWRMI/image390.pdf}\caption{Figura .: Avvolgimenti per la generazione dei gradienti}
\end{figure}

Siccome gli avvolgimenti per la generazione dei gradienti sono all'interno del campo principale, quando la corrente scorre, queste bobine subiscono un momento torcente legati alla forza di Lorentz. Date le elevate correnti, infatti, le bobine di gradiente sono soggette a momenti torcenti, quindi forze, molto elevate che cercano di spostare il blocco di resina contenente gli avvolgimenti dalla sua posizione originaria.

Gli spostami sono molto rapidi e intensi. Ciò si traduce in urti contro i supporti metallici diamagnetici dove sono alloggiati. L'effetto risultante è molto importante in risonanza magnetica poiché produce un rumore fastidioso per il paziente, caratteristico della sequenza di acquisizione applicata e con frequenza legata al tempo d'echo, di ripetizione e di inversione. In altre parole, il rumore ascoltato dal paziente ha una certa periodicità dipendente da come sono applicati i gradienti e dal tempo che intercorre tra una sequenza e la successiva.

Per ridurre il rumore si fornisce al paziente delle cuffie isolati che, tuttavia, riescono solo ad attenuare i suoni prodotti dalle sequenze.

Se, ad esempio, in un avvolgimento di gradiente circola una corrente di \(100\ A\) e il campo esterno è di \(1.5\ T\) si instaurano forze di attrazione molto intense, invertire rapidamente con un periodo di circa \(1\ ms\). Gli avvolgimenti di gradiente sono spinti verso l'impalcatura dello scanner producendo un rumore acustico che può superare i \(100\ dB\) nei moderni scanner. Questi livelli di rumore possono provocare la perdita momentanea di udito, dunque, i dispositivi di protezione uditivi sono obbligatori durante l'esame di imaging con risonanza magnetica.

Sono state proposte in letteratura delle sequenze di gradiente in modo da produrre sinfonie classiche come quelle di Bach.

La rumorosità delle sequenze può interferire con l'imaging di risonanza funzionale o fMRI, poiché il paziente potrebbe essere distratto dal frastuono e non prestare particolare attenzione agli stimoli psicosomatici somministrati. La risposta neurale risulta così leggermente o fortemente distorta dal rumore.

\subsubsection{Sicurezza della risonanza magnetica}\label{sicurezza-della-risonanza-magnetica}

Il campo magnetico nella stanza del magnete, anche in presenza di opportuni accorgimenti come lo shielding, risulta essere abbastanza intenso. Gli oggetti metallici risentono, quindi, della forza esercitata dal campo, soprattutto in prossimità dello scanner.

Un oggetto metallico con piccola massa, in prossimità del campo magnetico generato dallo scanner può subire una forza anche di circa \(2000\ N\). Anche l'introduzione di una sedia nella camera dello scanner può portare a seri rischi per il paziente, poiché le parti conduttive dell'oggetto estraneo sono attratte dal magnete con forza molto intensa.

Se la sedia, come ogni altro oggetto, ostruisce l'apertura dello scanner, il paziente non può entrare né uscire dal gantry. Si rende necessario lo spegnimento del campo mediante il fenomeno del quench, a maggior ragione nel caso in cui il paziente resti bloccato nel gantry.

A causa dell'elevata intensità del campo magnetico, anche oggetti molto piccoli possono comportarsi come proiettili accelerati fino a raggiungere velocità di molti \(m/s\).

Più in generale è possibile classificare i materiali sulla base dei valori assunti dalla suscettività magnetica \(\chi_{m}\), legata al campo magnetico e al vettore di magnetizzazione dalla relazione:

\[\overset{\underline{}}{M} = \chi_{m}B_{0}\]

I materiali ferromagnetici possiedono una suscettività magnetica molto elevata, al limite tendente all'infinito. Questi materiali, come ferro, nichel e cobalto, in prossimità dello scanner possono diventare proiettili diretti verso il magnete principale.

I materiali diamagnetici possiedono una suscettività magnetica lievemente minore dello zero, al limite \(\chi_{m} \rightarrow 0^{-}\). A causa di ciò questi materiali sono debolmente respinti dal magnete principale.

I materiali paramagnetici possiedono una suscettività magnetica \(\chi_{m}\) lievemente maggiore dello zero e sono, dunque, lievemente attratti dal magnete principale. Rame e acciaio possiedono caratteristiche amagnetiche a patto che siano sufficientemente puri. Sotto questa ipotesi questi materiali non forniscono particolari problemi di sicurezza.

La forza con cui il campo magnetico attrae gli oggetti ferromagnetici dipende dal suo gradiente spaziale \(\overset{\underline{}}{\nabla}{\overset{\underline{}}{B}}_{0}\) e il dipolo magnetico degli atomi del materiale, \(\overset{\underline{}}{\mu}\), secondo la relazione:

\[{\overset{\underline{}}{F}}_{m} = \overset{\underline{}}{\mu} \cdot \overset{\underline{}}{\nabla}{\overset{\underline{}}{B}}_{0}\]

Al centro del magnete il gradiente è nullo, quindi non vi sono forze attrattive.

Per un materiale ferromagnetico, il rapporto tra la forza magnetica e la forza peso si esprime come:

\[\frac{F_{m}}{F_{g}} = \frac{\chi_{m}B_{0}\left| \overset{\underline{}}{\nabla}{\overset{\underline{}}{B}}_{0} \right|}{\mu_{0}\rho g}\]

Per gli scanner moderni il modulo del gradiente spaziale può essere dell'ordine della decina di \(T/m\), quindi la forza che attrae un oggetto ferromagnetico può essere anche maggiore di \(250\) volte il suo peso, il che può portare l'oggetto a muoversi con una velocità di \(200\ km/h\) in \(25\ ms\).

I moderni scanner sono muniti di sistemi, noti come \emph{active shielding}, che limitano il campo all'esterno dello scanner. Ciò implica che il gradiente spaziale di campo diventa molto intenso e, di conseguenza, la forza attrattiva cresce. Per uno scanner a \(3\ T\) non bisogna introdurre oggetti metallici entro un raggio di circa \(4\ m\), dove il gradiente si riduce a \(5\ mT/m\).

Il momento torcente che gli oggetti subiscono non dipende dal gradiente spaziale ma solo dall'intensità del campo, secondo la relazione:

\[\overset{\underline{}}{N} = \overset{\underline{}}{\mu} \times {\overset{\underline{}}{B}}_{0}\]

Il momento torcente è massimo al centro dello scanner, quindi, può dare seri problemi ai piccoli oggetti ferromagnetici impiantati nel paziente, come capsule dentarie.

Una valutazione approssimata dell'intensità del momento torcente su un oggetto di lunghezza \(L\) è dato da:

\[\frac{F_{torc}L}{F_{m}} \simeq \frac{B_{0}}{\left| \overset{\underline{}}{\nabla}{\overset{\underline{}}{B}}_{0} \right|}\]

Per un campo a \(3\ T\), con un gradiente spaziale di \(10\ T/m\) e un oggetto di \(1\ cm\), la forza torcente è circa \(300\) volte la forza magnetica:

\[F_{torc} = \frac{B_{0}}{\left| \overset{\underline{}}{\nabla}{\overset{\underline{}}{B}}_{0} \right|}\frac{1}{L}F_{m} = \frac{30\ T}{10\frac{T}{m}1 \cdot 10^{- 2}m}F_{m} = 300F_{m}\]

Le forze e i momenti torcenti sugli elementi diamagnetici e paramagnetici, come i tessuti umani, sono di piccola intensità, dunque, sono tali da non causare pericoli. Ad esempio, i globuli rossi contengono ferro nell'emoglobina. Questa componente paramagnetica porta i globuli rossi a separarsi dal plasma a causa della diversa suscettività magnetica.

Per un campo principale che presenta un prodotto \(B_{0}\left| \overset{\underline{}}{\nabla}{\overset{\underline{}}{B}}_{0} \right| = 25\ T^{2}/m\), la forza di separazione è circa dell'\(8\%\) della differenza tra le forze gravitazioni dei due elementi.

Un potenziale effetto del campo magnetico può indotto nei tessuti in cui circola una corrente ioniche, come nervi e muscoli. Sulle correnti ioniche agisce la forza di Lorentz:

\[{\overset{\underline{}}{F}}_{L} = q\overset{\underline{}}{v} \times \overset{\underline{}}{B}\]

Questi effetti sono molto limitati per poter essere di importanza biologica, tuttavia sono presenti. Ad esempio, gli ioni in movimento, disciolti nei fluidi biologici, costituiscono delle correnti su cui agisce la forza di Lorentz. Le interazioni tra le correnti ioniche e il campo magnetico sono dette magnetoidrodinamiche.

Quando il sangue, contenente particelle ioniche, scorre in una direzione perpendicolare a un campo magnetico, le particelle cariche subiscono la forza di Lorentz. Questo può generare una resistenza al flusso, ma l\textquotesingle effetto sul moto del sangue è trascurabile e comporta solo un incremento minimo della pressione.

\begin{figure}
\centering
\includegraphics[width=4.12814in,height=2.85975in,alt={Immagine che contiene testo, schermata, Carattere, design Il contenuto generato dall\textquotesingle IA potrebbe non essere corretto.}]{media/15_HWRMI/image391.pdf}\caption{Figura .: Effetto della diversa suscettività magnetica}
\end{figure}

Un effetto più significativo è legato alla separazione ionica, causata dalla forza di Lorentz, la cui direzione dipende dal segno della carica \(q\) immersa nel campo magnetico. Durante l'esame di risonanza magnetica, possono essere indotti, a livello di vasi sanguigni, dei campi elettrici che interferiscono con il segnale elettrocardiografico, complicando il monitoraggio del paziente. Questo rende più complessa l'applicazione pratica del \emph{cardiac gating}, ovvero la sincronizzazione dell'acquisizione delle immagini cardiache con l'elettrocardiogramma.

\begin{figure}
\centering
\includegraphics[width=4.3in,height=3.4in,alt={Immagine che contiene testo, schermata, diagramma Il contenuto generato dall\textquotesingle IA potrebbe non essere corretto.}]{media/15_HWRMI/image392.pdf}\caption{Figura .: Separazione degli ioni presenti nel sangue a opera del campo magnetico}
\end{figure}

Sono stati riportati in letteratura effetti importati come vertigini e nausea legati al moto delle cariche nel campo magnetico della risonanza magnetica. Questi effetti potrebbero essere dovuti all'anisotropia della suscettività magnetica o all'effetto magnetoidrodinamiche.

Accanto alla legge di Lorentz bisogna considerare anche la legge di Faraday-Neumman-Lenz:

\[\overset{\underline{}}{\nabla} \times \overset{\underline{}}{E} = - \frac{\partial\overset{\underline{}}{B}}{\partial t}\]

Possono indursi delle correnti nei tessuti biologici in movimento nella regione di spazio in cui insiste un gradiente spaziale di campo magnetico. Ad esempio, nei pressi dello scanner, dove il gradiente è più elevato, possono indursi delle correnti con intensità di \(0.1\ A/m^{2}\), legate al normale movimento dell'operatore. Ciò rappresenta un fattore di rischio in quanto i limiti fissati dall'\emph{International Commission on Non-Ionizing Radiation Protection} o ICNIRP sono di \(0.04\ A/m^{2}\).

Queste correnti possono essere molto importati sui dispositivi metallici impiantati nel paziente come peacemaker e protesi.

Nei moderni sistemi attualmente in commercio si possono raggiungere gradienti di codifica spaziale dell'ordine dei \(40\ mT/m\) fino agli \(80\ mT/m\) i quali, in un FOV di \(50\ cm\), producono una variazione del campo dai \(20\ mT\) ai \(40\ mT\), agli esterni del FOV stesso. Con uno slew rate di \(200\frac{mT}{m/s}\) l'orientamento dei campi può essere invertito ogni \(1\ ms\).

Si inducono così delle correnti nel corpo umano dovute alla legge di Faraday, che fa circolare gli ioni contenuti nei fluidi biologici. Si genera una corrente che può stimolare i tessuti cardiaci o nervosi. La curva intensità-durata mostra che con le attuali macchine è possibile stimolare i tessuti nervosi periferici ma non i tessuti cardiaci che presentano soglie di attivazione più alte. Per stimolare il miocardio tramite le eddy current è necessario adoperare gradienti spaziali maggiori di quelli in commercio.

Le correnti indotte dallo switching dei gradienti possono inoltre causare un riscaldamento di dispositivi impintati nel paziente.

Esiste una serie di normative e studi che limitano l'ampiezza e la velocità dei gradienti normalmente applicati in risonanza al fine di evitare la \emph{Peripheral Nerve Stimulation} o PNS. Per tale motivo non sono approvati per applicazione diagnostiche umane campi con intensità maggiore di \(3\ T\), i cui gradienti devono avere ampiezza maggiore e durata inferiore. Le correnti indotte sono molto più elevate e possono causare la stimolazione con maggiore probabilità.

\subsection{Sistemi di trasmissione a radiofrequenza}\label{sistemi-di-trasmissione-a-radiofrequenza}

Il sistema di trasmissione a radiofrequenza prevede un host computer nel quale sono memorizzate le sequenze di impulsi fa far eseguire alle antenne. Le sequenze sono convertire in un segnale analogico dal DAC, la cui uscita è successivamente modulata con la tecnica della \emph{Single Side Band} o SSB. In questa fase oltra alla modulazione si esegue la reiezione della portante e delle bande laterali.

Il segnale così ottenuto attraversa un blocco di amplificazione di potenza detto \emph{Radio Frequency Power Amplifier} o RFPA e, infine, trasmesso alle antenne per poter irradiare il corpo.

Il segnale generato dal blocco SSB (Single Side Band) è una modulazione a banda stretta, fondamentale per conferire selettività in frequenza all'impulso a radiofrequenza (RF). Questo consente di indirizzare l'energia RF in modo mirato, migliorando la precisione dell'eccitazione dei nuclei durante l'esame di risonanza magnetica.

\begin{figure}
\centering
\includegraphics[width=5.15in,height=0.84167in]{media/15_HWRMI/image393.pdf}\caption{Figura .: Schema di trasmissione a radiofrequenza}
\end{figure}

\subsection{Effetti del campo a radiofrequenza}\label{effetti-del-campo-a-radiofrequenza}

Anche se l'intensità del campo a radiofrequenza è dell'ordine del \(\mu T\), ovvero un ordine di grandezza inferiore rispetto al campo terreste, la sua frequenza di \(64\ MHz\) a \(1.5\ T\) o \(128\ MHz\) a \(3\ T\) può causare problemi al corpo, soprattutto legati allo sviluppo di calore.

La legge dell'induzione di Faraday impone che campi magnetici variabili inducono campi elettrici variabili, quindi una circolazione di corrente:

\[\overset{\underline{}}{\nabla} \times \overset{\underline{}}{E} = - \frac{\partial\overset{\underline{}}{B}}{\partial t}\]

La corrente indotta dalla forza elettromotrice produce delle dissipazioni di energia in calore nei materiali conduttori come il corpo umano.

Per quantificare questo effetto si introduce il tasso specifico di assorbimento o \emph{Specific Absorption Rate} (SAR), definito come:

\[SAR = \sigma\frac{E_{p}^{2}}{2\rho}\]

Dove il termine \(\sigma\) rappresenta la conducibilità del tessuto, \(E_{p}\) l'ampiezza di picco del campo elettrico variabile secondo legge sinusoidale e \(\rho\) la densità del tessuto biolofico. L'unità di misura del SAR è:

\[\lbrack SAR\rbrack = \left\lbrack \frac{W}{kg} \right\rbrack\]

Il SAR fornisce indicazioni sull'energia dissipata in calore all'interno di \(1\ kg\) di materia. Esistono delle tabelle normale indicanti il valore di SAR idoneo per la specifica applicazione.

La capacità termica \(c\) dei tessuti biologici è di circa \(4200\ J/(kg{^\circ}C)\). Dividendo questa quantità per il SAR si ottiene il tasso di riscaldamento che subisce quel tessuto:

\[TR = \frac{SAR}{c}\]

La cui unità di misura è:

\[\lbrack TR\rbrack = \left\lbrack \frac{{^\circ}C}{s} \right\rbrack\]

Tipicamente il SAR in risonanza magnetica è di \(4.2\ W/kg\), quindi, per un tessuto biologico si ha un tasso di riscaldamento di:

\[TR = \frac{SAR}{c} = \frac{4.2\frac{Js}{kg}}{4200\frac{J}{kg{^\circ}C}} = 1 \cdot 10^{- 3}\frac{{^\circ}C}{s}\]

Per incrementare la temperatura di \(1\ {^\circ}C\) è necessario fornire al tessuto energia a radiofrequenza per \(100\ s\), ovvero circa \(17\ min\), consecutivi.

Quando si progetta la sequenza di acquisizione è necessario tener conto della frequenza e dell'intensità del campo a radiofrequenza usato per eccitare gli spin di idrogeno. Questo campo deve essere tale da non fornire una quantità di energia più elevata dei valori normati.

Le frequenze non possono eccedere determinate soglie normate poiché, all'aumentare della frequenza, ci si avvicina allo spettro delle onde ionizzanti, che possono produrre effetti biologici. Per tali motivi non si utilizzano frequenze troppo elevate per la diagnosta; dunque, si limita il campo a \(3\ T\).

\subsection{Sistema di ricezione analogico}\label{sistema-di-ricezione-analogico}

Il sistema di ricezione ha subito un'evoluzione dai primi sistemi analogici degli anni '80 ai sistemi di elaborazione quasi completamente digitali.

L'architettura di base di un sistema di ricezione a radiofrequenza analogico possiede due canali per la ricezione dei segnali provenienti dalle antenne in quadratura, così da minimizzare il rumore sovrapposto al segnale utile. Il segnale ha una portante dell'ordine del \(MHz\) con una banda dell'ordine dei \(kHz\).

Ogni canale prevede un preamplificatore, in grado di operare un primo filtraggio passa banda ad alta frequenza (HightFrequency o HF). Mediante un collegamento con cavo coassiale, necessario per trasmettere i segnali date le frequenze in gioco, il segnale viene portato all'ingresso di un amplificatore passa-banda. Successivamente, il cavo è portato all'esterno della camera schermata mediante appositi accessi che consentono solamente il passaggio della guida d'onda.

Il segnale è attenuato e ulteriormente filtrato mediante un passa-banda prima di essere trasmesso a un demodulatore sincrono o coerente. Quest'ultimo si dispartisce in due canali:

\begin{itemize}
\item
  Nel primo si esegue la moltiplicazione per una sinusoide con stessa frequenza della portante e in fase col segnale stesso. Tale canale è detto reale;
\item
  Nel secondo canale si esegue la moltiplicazione con una sinusoide in quadratura, ovvero sfasata di \(\pi/2\). Questo canale è detto immaginario.
\end{itemize}

La produzione della sinusoide o di qualsiasi altra forma d'onda usata per la demodulazione è realizzata da un digital synthesizer o sintetizzatore digitale.

Dopo il moltiplicatore analogico, il segnale è filtrato passa-basso così da estrarre le sole componenti utili del segnale.

Infine, dopo un ulteriore amplificazione e filtraggio passa-basso, il segnale viene campionato e quantizzato da un ADC e trasmesso a un computer per la ricostruzione dell'immagine.

Il campionatore non richiede caratteristiche estremamente spinte poiché dopo la demodulazione il segnale presenta una banda dell'ordine del \(kHz\).

L'interno apparato di filtraggio, demodulazione, amplificazione e campionamento deve essere ripetuto per ciascuno canale, con conseguenti costi e difficoltà costruttive. La presenza di più canali, tuttavia, è necessaria per minimizzare i tempi di acquisizione delle immagini.

\begin{figure}
\centering
\includegraphics[width=6.69306in,height=2.50764in,alt={Immagine che contiene testo, Carattere, diagramma, linea Il contenuto generato dall\textquotesingle IA potrebbe non essere corretto.}]{media/15_HWRMI/image394.pdf}\caption{Figura .: Sistema di ricezione a radiofrequenza}
\end{figure}

\begin{figure}
\centering
\includegraphics[width=5.43097in,height=3.54326in,alt={Immagine che contiene testo, diagramma, Piano, Disegno tecnico Il contenuto generato dall\textquotesingle IA potrebbe non essere corretto.}]{media/15_HWRMI/image395.pdf}\caption{Figura .: Sistema completo di trasmissione e ricezione a radiofrequenza}
\end{figure}

\subsubsection{Evoluzione dell'architettura analogica per gestire più canali}\label{evoluzione-dellarchitettura-analogica-per-gestire-piuxf9-canali}

Nel tempo si è cercato di acquisire più canali anche simultaneamente da antenne posizionate intorno al paziente. Mediante dei selettori è possibile selezionare da quale distretto anatomico prelevare il segnale. In questo modo è possibile ridurre la complessità dello stesso apparato di prelievo. Infatti, invece di ripetere la circuiteria un numero di volte uguale al quantitativo di canali, più segnali sono convogliati nello stesso canale di acquisizione mediante il selettore.

\begin{figure}
\centering
\includegraphics[width=6.69306in,height=3.00347in,alt={Immagine che contiene diagramma, Piano, Carattere, linea Il contenuto generato dall\textquotesingle IA potrebbe non essere corretto.}]{media/15_HWRMI/image396.pdf}\caption{Figura .: Schema con selettore per l\textquotesingle acquisizione contemporanea da più antenne}
\end{figure}

Di solito con \(6\) antenne era necessario precedere un sistema con \(2\) canali, uno per ogni tre antenne. È possibile, ovviamente, utilizzare un'architettura con \(n\) antenne ed \(m\) canali.

Dal punto di vista della circuiteria analogica di elaborazione, il segnale è filtrato, amplificato e demodulato allo stesso modo della soluzione con \(n\) canali per \(n\) antenne di ricezione.

Le antenne sono disposte come \emph{phased array} in cui ognuna riceve un segnale da una parte del corpo leggermente diversa ma con un rapporto segnale/rumore risultante maggiore.

Il selettore permette di registrare il segnale proveniente da un'antenna piuttosto che un'altra in modo da ottenere un \(k\)-spazio abbastanza denso combinando i dati registrati opportunamente.

\subsubsection{Architettura digitale per la ricezione}\label{architettura-digitale-per-la-ricezione}

Le moderne apparecchiature di risonanza magnetica non presentano un approccio basato su uno stadio di elaborazione iniziale analogico, ma tendono a spostare tutte le operazioni sul segnale nel mondo digitale.

In particolare, nelle moderne architetture, una volta acquisito il segnale, si esegue una preamplificazione, un filtraggio passa-banda e, infine, si campiona il segnale mediante un ADC. Le operazioni di demodulazione, ulteriore filtraggio e ricostruzione delle immagini sono eseguite da un elaboratore digitale.

\begin{figure}
\centering
\includegraphics[width=6.69306in,height=1.82361in,alt={Immagine che contiene testo, linea, diagramma, Diagramma Il contenuto generato dall\textquotesingle IA potrebbe non essere corretto.}]{media/15_HWRMI/image397.pdf}\caption{Figura .: Architettura di ricezione completamente digitale}
\end{figure}

Questa soluzione richiede la presenza di un ADC con frequenza di campionamento molto alta, caratteristica possibile grazie alla moderna tecnologia digitale, che offre prestazioni e affidabilità migliori rispetto gli ADC adoperati nelle soluzioni analogiche.

La soluzione \emph{dstream} permette di semplificare l'architettura di acquisizione, garantendo un basso consumo di energia, un elevato rapporto segnale/rumore e un ampio range dinamico con cui si può superare la codifica su \(16\) bit. Inoltre, viene migliorata anche la stabilità del segnale, limitando le sue distorsioni a opera di dispositivi analogici.

La quantizzazione avviene generalmente con un numero di bit uguale a \(22 \div 26\), in base alla banda del segnale.

Tra la soluzione digitale e quella analogica cambia anche il modo in cui è concepita l'operazione della demodulazione, dal punto di vista teorico. Il teorema di Nyquist afferma che ogni segnale può essere ricostruito se campionato a una frequenza almeno doppia alla massima banda \(B\) del segnale:

\[f_{S} \geq 2B\]

Il teorema parta di lunghezza di banda e non massima frequenza, dunque, questo concetto può essere sfruttato per eseguire la demodulazione.

Per comprendere tale approccio si divide l'asse delle frequenze in intervalli di multipli interi della frequenza di campionamento \(f_{S}\), con banda \(f_{S}/2\). Ogni intervallo di ampiezza \(f_{S}/2\), del tipo \(\left\lbrack (n - 1)f_{S}/2;nf_{S}/2 \right\rbrack\) è detto Nyquist zone e rappresenta lo spettro del segnale che può essere ricostruito con la frequenza di campionamento \(f_{S}\).

\begin{figure}
\centering
\includegraphics[width=5.53202in,height=2.06696in,alt={Immagine che contiene testo, diagramma, linea, Diagramma Il contenuto generato dall\textquotesingle IA potrebbe non essere corretto.}]{media/15_HWRMI/image398.pdf}\caption{Figura .: Divisione in Nyquist zone}
\end{figure}

In generale, il campionamento di tutto lo spettro del segnale con una frequenza \(f_{S}\) comporta che tutte le armoniche dello spettro, contenute negli intervalli di ampiezza \(f_{S}\), sono riportate in banda base causando l'errore di aliasing.

Si suppone di voler ricostruire la porzione di spettro contenuta nell'intervallo \(\left\lbrack 0;f_{S}/2 \right\rbrack\). A tale scopo è necessario filtrare il segnale nella banda desiderata e campionare con una frequenza doppia della banda, ovvero proprio \(f_{S}\). Ricostruendo il segnale mediante tecniche di interpolazione si ricava il segnale associato alla banda spettrale considerata.

\begin{figure}
\centering
\includegraphics[width=5.44868in,height=2.16697in,alt={Immagine che contiene diagramma, linea, Diagramma, pendio Il contenuto generato dall\textquotesingle IA potrebbe non essere corretto.}]{media/15_HWRMI/image399.pdf}\caption{Figura .: Ricostruzione del segnale solamente nella banda \(\left\lbrack 0;f_{S}/2 \right\rbrack\)}
\end{figure}

Si vuole, ora, ricostruire il segnale nella banda \(\left\lbrack 3f_{S}/2;2f_{S} \right\rbrack.\ \)È necessario applicare un filtro passa-banda, che estragga le sole componenti di interesse del segnale acquisito. La banda del segnale così ottenuto è sempre \(f_{S}/2\), quini, campionando a frequenza \(f_{S}\) è possibile ricostruire il segnale senza perdita di informazione.

\begin{figure}
\centering
\includegraphics[width=4.79234in,height=1.96902in,alt={Immagine che contiene diagramma, linea, Diagramma, pendio Il contenuto generato dall\textquotesingle IA potrebbe non essere corretto.}]{media/15_HWRMI/image400.pdf}\caption{Figura .: Spettro del segnale acquisito a valle del filtraggio passabanda}
\end{figure}

La replicazione dello spettro sui multipli interi della frequenza di campionamento garantisce la presenza di una replica dello spettro del segnale utile anche in banda base \(\left\lbrack 0;f_{S}/2 \right\rbrack\). In altre parole, è stato effettuata un'operazione di demodulazione sul segnale con un semplice filtraggio passa-banda e un campionamento, che portano lo spettro del segnale utile in banda base.

\begin{figure}
\centering
\includegraphics[width=3.50571in,height=3.27675in,alt={Immagine che contiene linea, diagramma, Diagramma, design Il contenuto generato dall\textquotesingle IA potrebbe non essere corretto.}]{media/15_HWRMI/image401.pdf}\caption{Figura .: Ricostruzione del segnale in banda base per il fenomeno delle repliche}
\end{figure}

Questo principio può essere sfruttato in risonanza magnetica, nella soluzione \emph{dstream} per digitalizzare e demodulare il segnale proveniente dalle antenne con lo stesso processo di campionamento, congiunto a un filtraggio passa-banda.

Il segnale registrato dalle antenne è filtrato mediante un filtro a radiofrequenza di tipo passa-banda con frequenza centrale \(f_{0} = 2\pi\omega_{0}\) di \(60 \div 80\ MHz\) e un'estensione della banda di qualche centinaio di \(kHz\). In seguito, si campiona il segnale con un ADC che riesce a raggiungere una frequenza delle decine di \(MSa/s\), sottocampionando il segnale.

L'operazione di campionamento nel dominio del tempo produce una replicazione nel dominio della frequenza. Ciò determina anche in banda base vi è una replicazione dello spettro del segnale proveniente dalle antenne. Si ottiene così il segnale campionato e demodulato, pronto per essere digitalizzato ed elaborato mediante algoritmi digitali, i quali operano un ulteriore filtraggio e la ricostruzione dell'immagine.

Volendo procedere in assenza di filtraggio passa-banda, sarebbe necessario campionare il segnale con almeno una frequenza di \(128\ MSa/s\) per un campo a \(1.5\ T\). Questi campionatori ancora oggi sono complessi e costosi da realizzare.

Grazie all'interpretazione del teorema di Nyquist introdotta è possibile utilizzare dei campionatori a \(30 \div 40\ MSa/s\), molto più semplici da realizzare, con risoluzione di \(22 \div 26\) bit.

Il filtro a radiofrequenza, dal punto di vista realizzativo, è più semplice del campionatore a \(128\ MSa/s\). La soluzione adottata nel \emph{dstream} permette di semplificare il circuito di digitalizzazione.

Si osservi che, a causa della replicazione, gli spettri degli intervalli pari sono ribaltati rispetto gli spettri originali, quindi, prima di adoperare gli algoritmi di ricostruzione è necessario invertire gli spettri ottenuti con questo metodo, nota la Nyquist zone di provenienza.

\begin{figure}
\centering
\includegraphics[width=6.69167in,height=3.31667in]{media/15_HWRMI/image402.pdf}\caption{Figura .: replicazione e ribaltamento della replica pari}
\end{figure}

La soluzione digitale permette di eliminare il rumore introdotto dall'elaborazione analogica, soprattutto legato ai moltiplicatori a radiofrequenza, non facilmente realizzabili nella pratica. Questa riduzione del rumore è compensata da campionatori costati a causa dell'elevato numero di bit e le alte frequenze di campionamento.

\subsection{Convertitori analogico/digitale}\label{convertitori-analogicodigitale}

Un segnale analogico proveniente dal mondo reale, come quello registrato dalle antenne in risonanza magnetica, per poter essere elaborato da sistemi digitali, deve essere quantizzato su un numero finito di bit. Per eseguire il passaggio da analogico a digitale si utilizzano particolari circuiti detti convertitori analogico/digitalo o ADC (\emph{Analog to Digital Converter}).

\begin{figure}
\centering
\includegraphics[width=3.49167in,height=2.83333in,alt={Teorema del campionamento}]{media/15_HWRMI/image403.pdf}\caption{Figura .: Campionamento del segnale reale}
\end{figure}

Il segnale delle antenne è di natura elettrica, con variazioni continue in ampiezza e nel tempo. Per essere elaborato dalla circuiteria digitale è necessario ottenere un segnale discreto, definito su istanti di tempo predefiniti. Questa operazione è detta campionamento o sampling ed è generalmente svolta a circuiti campionatori o \emph{sample and hold}, in cui si include anche la circuiteria che si occupa di mantenere costante il valore del segnale fino all'arrivo del campione successivo.

La fase di hold è necessaria poiché la maggior parte dei circuiti che operano la conversione del campione in una parola di \(n\) bit non sono istantanei, ma necessitano di un certo tempi per effettuare le operazioni necessarie.

\subsubsection{Flash converter}\label{flash-converter}

Nelle applicazioni di risonanza magnetica si utilizzano i flash converter poiché possiedono un'elevata frequenza di campionamento, dipendente essenzialmente dalla somma del tempo di propagazione del campione e della rete di codifica. Inoltre, non è richiesto un circuito di sample and hold a valle.

L'architettura del convertitore flash prevede \(2^{n}\) resistenze, \(2^{n - 1}\) comparatori e un codificatore che trasforma l'ingresso in una codifica binaria, rappresentativa del valore in ingresso. Il codificatore, nello specifico, riceve una parola di \(2^{n}\) valori e restituisce una sua codifica su \(n\) bit.

Il segnale da campionare e digitalizzare è posto in ingresso ai morsetti non invertenti dei comparatori, mentre l'ingresso invertente è connesso a una rete che ripartisce la tensione di riferimento in \(2^{n}\) fasce, così da fissare i livelli di riferimento per ogni comparatore.

\begin{figure}
\centering
\includegraphics[width=5.30561in,height=6.28333in,alt={Flash ADC: Working and Circuit - Nerds Do Stuff}]{media/15_HWRMI/image404.pdf}\caption{Figura .: Architettura di un flash converter a \(3\) bit}
\end{figure}

Ogni comparatore commuta la propria uscita a \(1\) se la tensione del segnale è maggiore del rispettivo livello di riferimento, altrimenti è nulla. L'\(i\)-esimo livello di riferimento è dato dal partitore resistivo:

\[V_{i} = \frac{i}{2^{N}}V_{ref}\]

Affinché la \(i\)-esima uscita sia alta deve accadere che:

\[V_{in} > V_{i} = \frac{i}{2^{N}}V_{ref}\]

L'uscita di un comparatore è alta finché la tensione in ingresso non è minore della tensione di riferimento. Da quel punto in poi tutte le uscite saranno nulle poiché la tensione di riferimento tende ad aumentare con \(i\).

Se risulta che:

\[V_{in} > \frac{2^{i} - 1}{2^{N}}\]

L'uscita del compratore è sempre alta. Se, invece, risulta che:

\[V_{in} < \frac{i}{2^{N}}V_{ref}\]

Le prime \(i\) uscite sono alte, mentre le restanti \(2^{N} - i - 1\) sono basse.

In uscita ai comparatori, pilotati da un segnale di clock, non vi sono tutte le \(2^{N}\) combinazioni possibili, ma solo \(n\). È, quindi, possibile eseguire la codifica biunivoca che fa corrispondere alle \(n\) uscite dei comparatori le corrispettive codifiche su \(n\) bit.

Per avere una codifica su un numero elevato di bit è necessario aumentare esponenzialmente i componenti elettronici della rete. Ciò pone dei problemi di occupazione di area e di accuratezza legata alle tolleranze intrinseche dei resistori. Inoltre, i comparatori e le resistenze devono essere perfettamente uguali tra loro, quindi, i costi crescono poiché i componenti elettronici devono essere estremamente affidabili.

I comparatori sono pilotati da un segnale di temporizzazione e, per tale motivo, sono detti cocked; inoltre, mostrano un comportamento ibrido tra analogico e digitale. Le uscite sono aggiornate a ogni colpo di clock.

Negli ultimi anni, per il calo dei prezzi di fabbricazione, è possibile realizzare convertitori ADC di questo tipo con frequenza di campionamento intorno ai \(40 \div 60\ MHz\), con un costo limitato.

\subsubsection{Subranging ADC}\label{subranging-adc}

L'architettura di un convertitore subranging prevede la presenza di un circuito di sample and hold, a valle del quale il valore letto si dipartisce:

\begin{itemize}
\item
  Una parte entra nel convertitore A/D a \(3\) bit, dove viene digitalizzato in una stringa di bit.
\item
  La codifica del campione, per gli errori di quantizzazione non coincide con la tensione dell'ingresso. Per minimizzare l'errore di conversione il campione è confrontato con la conversione analogica della stringa di \(3\) bit precedentemente ottenuta al primo passo di conversione
\end{itemize}

La differenza dei due segnali analogici è amplificata nuovamente e convertita in un segnale digitale.

I primi \(3\) bit ottenuti con la conversione diretta del campione sono i più significativi (Most Significan Bit o MSB); mentre gli ultimi tre, ottenuto con la conversione della differenza tra l'ingresso e la sua versione quantizzata, costituiscono i bit meno significativi (Least Significan Bit o LSB).

\begin{figure}
\centering
\includegraphics[width=6.4384in,height=2.95875in,alt={Immagine che contiene testo, diagramma, Carattere, linea Il contenuto generato dall\textquotesingle IA potrebbe non essere corretto.}]{media/15_HWRMI/image405.pdf}\caption{Figura .: ADC di tipo subranging a due stati a 6 bit}
\end{figure}

Con questa architettura si ottiene una parola di \(6\) bit con una notevole riduzione dei costi rispetto al flash converter, poiché, per ottenere il numero di bit desiderato, sono stati utilizzati due convertitori analogico/digitale a \(3\) bit e un convertitore digitale/analogico a \(3\) bit. Con la soluzione flash converter sono necessari \(2^{6} = 64\) resistori e \(2^{6} - 1 = 63\) comparatori. Con \(3\) bit sono necessari \(8\) comparatori per un totale di \(16\).

Il campionamento sul segnale avviene prima in maniera grossolana e poi si raffina il risultato confrontando la conversione col segnale effettivo, al fine di minimizzare l'errore di conversione. Questo processo porta a una perdita di efficienza poiché è richiesto un tempo di conversione A/D, una conversione D/A e un'ulteriore conversione A/D.

Rispetto all'architettura flash converter, questa soluzione permette di ottenere velocità di campionamento inferiori, tuttavia, offre un range dinamico più ampio, ovvero un maggiore numero di bit, a un costo minore.

Il convertitore subranging è utilizzato in PET, poiché le frequenze in gioco non raggiungono quelle della risonanza magnetica, dunque, è possibile rilassare la frequenza di campionamento.

\begin{center}
\vfill
    \chapter{Principi fisici della PET}
    \label{blx:PrincFis\therefsection}
\vfill

\minitoc
\newpage
\end{center}
\justify

\subsection{Struttura atomica}\label{struttura-atomica}

Per comprendere il funzionamento della PET (\emph{Positrion Emission Tomography}) è necessario analizzare il funzionamento dell'atomo. Esso è formato da un nucleo, in cui sono presenti protoni e neutroni; attorno al nucleo orbitano gli elettroni. Nel complesso, il numero di protoni è pari al numero di elettroni, per cui l'atomo è detto stabile. Il numero di protoni, e, quindi, di elettroni, è detto numero atomico ed è indicato con Z.

\[\# protoni = \# elettroni = Z\  \rightarrow numero\ atomico\]

Il numero di protoni più quello dei neutroni è detto, invece, numero di massa atomica e è indicato con \(A\).

\[\# protoni + \# neutroni = A\  \rightarrow numero\ di\ massa\ atomica\]

Il numero di neutroni è spesso indicato semplicemente come \(N\):

\[\# neutroni = N\]

%\begin{figure}
%\centering
%\includegraphics[width=3.08217in,height=3.24306in,alt={P3733\#yIS1}]{media/16_PrincFis/image406.pdf}
%\caption{Figura .: Organizzazione in shell}
%\end{figure}

I livelli energetici su cui giacciano gli elettroni sono quantizzati, denotati come \emph{Shell} (gusci), e si differenziano in base alla diversa energia: il livello energetico più basso è detto K, quello immediatamente successivo L e quello successivo ancora M e così via. Ogni livello energetico possiede dei sottolivelli energetici (\emph{Subshells}) legati alla problematica del momento angolare. Questi ulteriori livelli sono indicati con le lettere s, p, d, f. Ad esempio, il livello energetico K possiede un livello s in cui è probabile trovare al massimo due elettroni con spin opposto secondo il principio di esclusione di Pauli:

\[K = s\left( 2e^{-} \right)\]

Allo stesso modo, il livello L possiede il sottolivello s, che può ospitare, come prima, al massimo 2 elettroni, e il sottolivello p, che ospita al massimo 6 elettroni poiché composto da tre orbitali, per un totale di 8 elettroni:

\[L = s\left( 2e^{-} \right) + p\left( 6e^{-} \right)\]

Nel livello energetico M, infine, vi sono i due sottolivelli già citati, s e p, e in più vi è anche il sottolivello d, che ospita al massimo 10 elettroni, per un totale di 18 elettroni:

\[M = \left( 2e^{-} \right) + p\left( 6e^{-} \right) + d\left( 10e^{-} \right)\]

Spesso l'atomo è rappresentato secondo il modello planetario in cui gli elettroni ruotano interno al nucleo. La meccanica quantistica ha sostituito il concetto deterministico di orbita con quello probabilistico di orbitale. La forma e il numero degli orbitali sono ottenuti risolvendo l'equazione di Schrödinger. L'orbitale s e unico e ha la forma di una sfera, gli orbitali p sono tre, posizionati perpendicolarmente tra loro e con una forma bilobata, gli orbitali d sono 5 mentre gli f sono 7. Questi ultimi hanno una forma molto più complessa.

\begin{figure}
\centering
\includegraphics[width=5.72917in,height=3.26695in,alt={P3742\#yIS1}]{media/16_PrincFis/image407.pdf}\caption{Figura .: Orbitali atomici}
\end{figure}

Per la PET sono di particolare interesse i nuclidi, indicati con la scrittura dove sono facilmente osservabili il numero di massa A e numero atomico Z poiché posti alla sinistra, rispettivamente in alto e in basso, del simbolo chimico della sostanza di interesse.

\[_{Z}^{A}X_{N}\]

Dove X è la specie atomica di interesse. Esistono 270 nuclidi stabili e 2700 instabili; questi ultimi sono detti radionuclidi, poiché l'instabilità determina un'emissione di radiazioni elettromagnetiche e corpuscolare. Le sostanze assumono una diversa nomenclatura in base ad alcune caratteristiche riportante in tabella:

\begin{longtable}[]{@{}
  >{\centering\arraybackslash}p{(\linewidth - 4\tabcolsep) * \real{0.3333}}
  >{\centering\arraybackslash}p{(\linewidth - 4\tabcolsep) * \real{0.3729}}
  >{\centering\arraybackslash}p{(\linewidth - 4\tabcolsep) * \real{0.2938}}@{}}
\caption{Tabella 17.1: Nomenclatura sostanze}\tabularnewline
\toprule\noalign{}
\begin{minipage}[b]{\linewidth}\centering
\textbf{NOMENCLATURA}
\end{minipage} & \begin{minipage}[b]{\linewidth}\centering
\textbf{CARATTERISTICHE}
\end{minipage} & \begin{minipage}[b]{\linewidth}\centering
\textbf{ESEMPIO}
\end{minipage} \\
\midrule\noalign{}
\endfirsthead
\toprule\noalign{}
\begin{minipage}[b]{\linewidth}\centering
\textbf{NOMENCLATURA}
\end{minipage} & \begin{minipage}[b]{\linewidth}\centering
\textbf{CARATTERISTICHE}
\end{minipage} & \begin{minipage}[b]{\linewidth}\centering
\textbf{ESEMPIO}
\end{minipage} \\
\midrule\noalign{}
\endhead
\bottomrule\noalign{}
\endlastfoot
\textbf{Isotopi} & Stesso numero atomico Z & \(_{6}^{12}{C\ e\ }_{6}^{14}C\) \\
\textbf{Isotoni} & Stesso numero di neutroni N & \(_{8}^{16}{O_{8}\ e\ _{7}^{15}N_{8}}\) \\
\textbf{Isobari} & Stesso numero di massa A & \(_{}^{131}I\ e\ _{}^{131}{Xe}\) \\
\textbf{Isomeri} & Differente energia & \(_{}^{99}{Tc}\ e\ _{}^{99m}{Tc}\) \\
\end{longtable}

Di particolare importanza sono gli isomeri dove, a causa di azioni esterne, qualche elettrone o protone si trova in uno stato energetico superiore o eccitato rispetto all'elemento base.

Le particelle elementari presentano una massa espressa in termini di energia. Dalla relazione di Einstein \(E = mc^{2}\) è noto, infatti, che la massa può essere espressa in termini di energia in eV, e viceversa.

\begin{longtable}[]{@{}
  >{\centering\arraybackslash}p{(\linewidth - 2\tabcolsep) * \real{0.5000}}
  >{\centering\arraybackslash}p{(\linewidth - 2\tabcolsep) * \real{0.5000}}@{}}
\caption{Tabella 17.2: Caratteristiche elementi atomici}\tabularnewline
\toprule\noalign{}
\begin{minipage}[b]{\linewidth}\centering
\textbf{PARTICELLA}
\end{minipage} & \begin{minipage}[b]{\linewidth}\centering
\textbf{MASSA (MeV)}
\end{minipage} \\
\midrule\noalign{}
\endfirsthead
\toprule\noalign{}
\begin{minipage}[b]{\linewidth}\centering
\textbf{PARTICELLA}
\end{minipage} & \begin{minipage}[b]{\linewidth}\centering
\textbf{MASSA (MeV)}
\end{minipage} \\
\midrule\noalign{}
\endhead
\bottomrule\noalign{}
\endlastfoot
Elettrone & 0.511 \\
Protone & 938.78 \\
Neutrone & 939.07 \\
\end{longtable}

Protone e neutrone hanno, quindi, pressocché la stessa massa mentre l'elettrone ha una massa di circa un millesimo delle due masse. Nella valutazione della massa atomica, trascurare la presenza dell'elettrone porta a un errore estremamente limitato.

\subsection{Tipi di decadimento}\label{tipi-di-decadimento}

Esistono numerosi elementi, in natura o prodotti in laboratorio mediante reazioni nucleari, che possiedono un nucleo energeticamente instabile. Questi elementi decadono, attraverso processi chimico-fisici, per trasformarsi in elementi più leggeri ed energeticamente stabili.

Il decadimento di un atomo instabile è accompagnato dall'emissione di corpuscoli carichi da parte del nucleo e/o di energia irradiata sottoforma di radiazione elettromagnetica nello spettro dei raggi \(\gamma\). Nel dettaglio:

\begin{itemize}
\item
  Il decadimento \(\alpha\) è proprio dei nuclei pesanti, come ad esempio l'uranio \(_{}^{235}U\), che, decadendo, libera, appunto, una particella \(\alpha\). Questa non è altro che un nucleo di elio (He), costituito da 2 protoni e 2 neutroni, e con una vita breve all'interno dei tessuti umani, poiché è assorbita nel giro di 0.03mm;
\item
  Il decadimento \(\beta^{-}\) è proprio dei nuclidi ricchi di neutroni: i neutroni in eccesso, non necessari per l'equilibrio del nucleo, possono decadere e trasformarsi secondo la seguente reazione:
\end{itemize}

\[n \rightarrow p + \beta^{-} + \underline{v}\]

Il neutrone si trasforma, quindi, in tre particelle: un protone, una particella \(\beta^{-}\), elettrone di carica negativa emesso dal nucleo, e un antineutrino \(\underline{v}\), una particella inerte priva di massa.

Dal punto di vista energetico, la reazione deve essere bilanciata per cui l'energia di transizione sarà uguale alla differenza di energia tra i due nuclidi. L'energia di transizione si divide tra il protone che resta nel nucleo e cambia la natura chimica dell'atomo, la particella \(\beta^{-}\) e l'antineutrino \(\underline{v}\). Quest'ultima particella è debolmente interagente con la materia ed è introdotta per bilanciare dal punto di vista energetico il decadimento;

\begin{itemize}
\item
  Il decadimento \(\beta^{+}\) è proprio dei nuclidi ricchi di protoni e avviene secondo la reazione:
\end{itemize}

\[p \rightarrow n + \beta^{+} + v\]

Il protone si trasforma, quindi, in tre particelle: un neutrone, una particella \(\beta^{+}\), ovvero un elettrone di carica positiva detto positrone, e un neutrino, antiparticella dell'antineutrino. Anche in questo caso deve esserci un bilancio energetico; poiché al secondo membro vi è un'energia, espressa in massa di:

\[m_{neutrone} = m_{protone} + m_{elettrone}\ \ \ e\ \ \ m_{\beta^{+}} = m_{elettrone}\]

è necessaria un'energia di transizione maggiore di 1.022MeV. Al primo membro, inoltre, per lo stesso motivo, vi sono \(2 \times 0.511MeV\). Questo decadimento è il più utilizzato in PET, in particolar modo quello del Fluoro (F), che, legandosi al glucosio, permette la seguente reazione:

\[_{9}^{18}{F_{9} \rightarrow_{8}^{18}{O_{10} + \beta^{+} + v}}\]

Il fluoro, legato alla macromolecola biologica metabolicamente non tossica come il glucosio, esegue il suo stesso percorso all'interno del corpo umano e, sfruttando il positrone emesso, si riesce a ricostruire le immagini della distribuzione del radionuclide;

\begin{figure}
\centering
\includegraphics[width=4.28306in,height=3.575in,alt={P3799\#yIS1}]{media/16_PrincFis/image408.pdf}\caption{Figura .: Schema riassuntivo del decadimento}
\end{figure}

\begin{itemize}
\item
  L'\emph{Electron Capture} è proprio dei nuclidi ricchi di protoni ed è un fenomeno per cui un protone può catturare un elettrone, portandolo internamente al nucleo, generando così un neutrone e un neutrino. In questo caso, l'energia di transizione deve essere minore di 1.022MeV, mentre l'energia dei nuclidi deve essere maggiore di 1.022MeV;
\item
  La Transizione isomerica avviene quando un nucleo si trova in uno stato energetico eccitato a causa di una cessione di energia da parte dell'ambiente. Se lo stato eccitato ha un tempo di vita molto lungo, allora si chiama stato metastabile. È il caso del tecnezio-99 \(_{}^{99}{Tc}\) che, tramite questa transizione, emette un fotone \(\gamma\);
\item
  La struttura atomica prevede che gli \emph{Shell} più interni, quindi, dal K all'M e così via, siano i primi a riempirsi di elettroni. Nel processo di decadimento può accadere che un elettrone del livello energetico più interno acquisisca energia tale da essere espulso dall'atomo. Questo elettrone è detto \emph{Auger Electon} e lascia una vacanza negli strani più interni della struttura atomica. Per ritornare alla condizione di riposo, un elettrone dello strato più esterno può riempire la vacanza del livello interno emettendo un fotone di energia uguale alla differenza tra i due livelli energetici. Ciò conferisce maggiore stabilità all'atomo.
\end{itemize}

Il fotone emesso dall'elettrone che riempie la vacanza può anche interagire con un altro elettrone più esterno dello stesso atomo che, essendo debolmente legato al nucleo, è espulso formando un ulteriore elettrone di Auger.

\begin{figure}
\centering
\includegraphics[width=3.41623in,height=2.56944in,alt={P3805\#yIS1}]{media/16_PrincFis/image409.pdf}\caption{Figura .: Elettrone di Auger}
\end{figure}

\subsection{Tempo di dimezzamento}\label{tempo-di-dimezzamento}

Il fenomeno del decadimento radioattivo ha una certa probabilità di occorrenza, cioè non è un fenomeno deterministico, ma è aleatorio. Sia \(P\) la probabilità di decadere, uguale per ogni atomo, e \(N(t)\) il numero di atomi radioattivi che, all'istante \(t\), possono potenzialmente decadere. Dal processo binomiale, la probabilità di avere \(k\) decadimenti su \(N(t)\) atomi è pari a:

\[P = \left( \begin{array}{r}
N(t) \\
k
\end{array} \right)p^{k}(1 - p)^{N(t) - k}\]

È possibile valutare la media statistica tra la differenza del numero di atomi radioattivi al tempo \(t + dt\) e del numero di atomi radioattivi al tempo \(t\). Per la probabilità binomiale, questo valor medio deve essere uguale a \(p\) volte il valor medio di \(N(t)\):

\[E\left\lbrack N(t + dt) - N(t) \right\rbrack = E\lbrack dN\rbrack = - E\left\lbrack N(t) \right\rbrack p\]

Se si esprime \(p\) come una probabilità nell'unità di tempo costante per tutti gli atomi di quella specie, è lecito scrivere che:

\[p = \lambda dt\]

Da cui risulta che:

\[E\left\lbrack N(t + dt) - N(t) \right\rbrack = E\lbrack dN\rbrack = - E\left\lbrack N(t) \right\rbrack\lambda dt\]

\[E\left\lbrack \dfrac{dN}{dt} \right\rbrack = - E\left\lbrack N(t) \right\rbrack\lambda\]

Si ottiene, quindi, un'equazione del tipo:

\[- \dfrac{dN}{dt} = \lambda N\]

Dove \(- \dfrac{dN}{dt}\) è il tasso di decadimento, o attività, e può essere anche indicato con \(A\). La soluzione è ovviamente del tipo esponenziale:

\[A(t) = A_{0}e^{- \lambda t}\]

Dove \(\lambda\) è la costante di decadimento o di disintegrazione ed è specifica per ogni nucleo che segue il processo del decadimento. Questo modello di decadimento statistico è coerente con le misurazioni sperimentali ed è, quindi, coerente con le applicazioni pratica.

Per praticità, poiché il fenomeno è esponenziale, si introduce il tempo di emivita, ovvero il tempo in cui l'attività del radionuclide si dimezza:

\[A\left( t_{\dfrac{1}{2}} \right) = \dfrac{1}{2}A_{0}\]

\[A_{0}e^{- \lambda t} = \dfrac{1}{2}A_{0}\]

Da quest'ultima relazione è possibile ricavare la definizione del tempo di emivita:

\[t_{\dfrac{1}{2}} = \dfrac{\log(2)}{\lambda} = \dfrac{0.693}{\lambda}\]

Si definisce vita media come il reciproco della costante di decadimento:

\[\tau = \dfrac{1}{\lambda}\]

Si può dire che un radionuclide decade del 63\% in una vita media. Infatti, risulta che:

\[A(\tau) = A_{0}e^{- \lambda\tau} = A_{0}e^{- \dfrac{\lambda}{\lambda}} = \dfrac{A_{0}}{e} = 0.63A_{0}\]

Oltre al tempo di emivita del radionuclide, bisogna considerare il tempo impiegato dall'organismo umano per eliminare il radiofarmaco somministrato. Solitamente, infatti, i radiofarmaci sono espulsi attraverso le urine in un tempo dipendente dal farmaco stesso e dalla dose. Al tempo di decadimento, quindi, si aggiunge un tempo di decadimento biologico, definito dalla costante \(\lambda_{b}\), ovvero la costante di escrezione dal sistema biologico. Il tempo di decadimento biologico è calcolato come:

\[T_{b} = \dfrac{0.693}{\lambda_{b}}\]

Si definisce una costante efficace \(\lambda_{e}\), data da:

\[\lambda_{e} = \lambda_{b} + \lambda_{p}\]

dove \(\lambda_{p}\) è la costante reale. Da questa scrittura è possibile, quindi, definire un tempo efficace, secondo la relazione:

\[\dfrac{1}{T_{e}} = \dfrac{1}{T_{b}} + \dfrac{1}{T_{p}}\]

L'unità di misura utilizzata è il Bequerel (Bq), dove:

\[1Bq = 1\ disintegrazione\ per\ secondo\ (dps)\]

Ormai obsoleto è il Curie (Ci), definito come il numero delle disintegrazioni al secondo che avvengono in un grammo di Radio-226.

\[1Ci = 3.7 \times 10^{10}dps = 37GBq\]

\subsubsection{Esempio numerico di decadimento}\label{esempio-numerico-di-decadimento}

Dato il decadimento nel tempo con legge esponenziale, l'attività di un certo radionuclide non si mantiene costante nel tempo. Dunque, note le varie costante è possibile calcolare la dose necessaria da somministrare al paziente in base all'intervallo di tempo tra produzione del tracciante e infusione per eseguire l'esame diagnostico PET.

La qualità di un'immagine PET dipende molto dall'attività del radionuclide infuso del paziente, infatti, maggiore è la dose e maggiore è il rapporto segnale/rumore. Per poter ricostruire in maniera ottima l'immagine anche dopo molte ore dalla produzione del radionuclide è necessario valutare l'attività tramite le varie costanti temporali.

Con questo procedimento è possibile ricostruire immagini funzionali di buona qualità prescindendo dall'orario in cui è eseguito l'esame diagnostico.

Una dose di \(_{}^{18}F - FDG\) ha un'attività di \(20\) \(mCi\) alle ore 10.00. Quanto era l'attività alle 7.00 e quanto sarà alle 14.00? Il tempo di emivita è 110min.

Si converte il tempo di emivita in ore:

\[t_{\dfrac{1}{2}} = 110min \Leftrightarrow \ t_{\dfrac{1}{2}} = \dfrac{110min}{60min/h} = 1.8h\]

Per esprimere l'attività da \(Ci\) a \(Bq\) è necessario utilizzare la relazione tra le due misure:

\[1Ci = 37Gbq \Leftrightarrow 20mCi = 740MBq\]

La costante di decadimento la si ottiene dalla relazione che la lega con il tempo di emivita:

\[t_{\dfrac{1}{2}} = \dfrac{0.693}{\lambda} \Leftrightarrow \lambda = \dfrac{0.693}{t_{\dfrac{1}{2}}} = \dfrac{0.693}{1.8h} = 0.385\dfrac{1}{h}\]

Dalla relazione del decadimento \(A(t) = A_{0}e^{- \lambda t}\), si ricava l'attività alle 7.00 e alle 14.00, ponendo

\[A_{0} = A(10.00) = 740MBq\]

Dunque:

\[A(7h) = A(10h)e^{- \lambda t} = 740MBq \cdot e^{- 0.385\dfrac{1}{h} \cdot 7h} \simeq 51MBq\]

\[A(14h) = A(10h)e^{- \lambda t} = 740MBq \cdot e^{- 0.385\dfrac{1}{h} \cdot 14h} \simeq 3.5MBq\]

\subsection{Interazione particelle--materia}\label{interazione-particellemateria}

Per quanto riguarda la formazione delle immagini PET, bisogna tenere conto dei vari fenomeni di interazione tra le particelle cariche, quali positroni, elettroni e particelle \(\alpha\), e la materia. Le particelle, infatti, possono interagire con la materia in vari modi:

\begin{itemize}
\item
  Espulsione di elettroni a causa della collisione di una particella carica con un elettrone atomico. In questo caso si parla di ionizzazione degli atomi;
\item
  Eccitamento di un elettrone portandolo al livello energetico superiore senza estrarlo;
\item
  Rottura di legami chimici.
\end{itemize}

Tipicamente, la particella \(\beta\), più leggera, si muove a zig-zag, urtando le varie particelle presenti intorno, mentre la particella \(\alpha\), che è molto più pesante, si muove in linea retta. Il range percorso dalle particelle, ovvero la distanza percorsa nel tessuto prima che la particella sia assorbita, dipende da vari fattori, quali energia, carica, massa e densità della materia attraversata. Quando un positrone perde la sua energia, a causa dei vari urti che subisce, e arriva a riposo, si combina con un elettrone di un atomo del tessuto assorbente. In quel momento le due particelle \(\beta^{+}\) ed \(e^{-}\) si annichiliscono e si ha l'emissione di due fotoni \(\gamma\) opposti, con energia pari a 511keV. Il processo è a carica nulla poiché le particelle hanno la stessa carica ma di segno opposto, quindi, il principio di conservazione della carica è rispettato così come quello della quantità di moto e massa-energia poiché emergono due fotoni di energia di 511keV che viaggiano in direzioni opposte.

\subsection{Principio di funzionamento della PET}\label{principio-di-funzionamento-della-pet}

Si inietta un tracciante, tipicamente a base di fluoro, nel paziente, che, essendo instabile, decade secondo un processo di decadimento \(\beta^{+}\), emettendo positroni. Questi ultimi percorrono una breve distanza all'interno dei tessuti biologici di circa 1-2mm e tipicamente 1.3mm, prima di perdere la loro energia e annichilirsi con un elettrone del tessuto. Da questo processo sono generato due fotoni \(\gamma\) che viaggiano in direzione opposta all'energia di 511keV. Questi fotoni fuoriescono dal paziente e sono catturati da un anello di detettori. Il fenomeno è poi memorizzato all'interno di un computer, che lo elabora e costruisce un'immagine. Dalla cattura del fotone \(\gamma\) in poi, il processo è totalmente digitale.

\begin{figure}
\centering
\includegraphics[width=5.80833in,height=4.24925in,alt={P3865\#yIS1}]{media/16_PrincFis/image410.pdf}\caption{Figura .: Schema di funzionamento PET}
\end{figure}

I radionuclidi più utilizzati in PET sono riportati nella seguente tabella con le rispettive caratteristiche.

\begin{longtable}[]{@{}
  >{\centering\arraybackslash}p{(\linewidth - 4\tabcolsep) * \real{0.2793}}
  >{\centering\arraybackslash}p{(\linewidth - 4\tabcolsep) * \real{0.2502}}
  >{\centering\arraybackslash}p{(\linewidth - 4\tabcolsep) * \real{0.4705}}@{}}
\caption{Tabella 17.3: Caratteristiche radionuclidi}\tabularnewline
\toprule\noalign{}
\begin{minipage}[b]{\linewidth}\centering
\textbf{RADIONUCLIDE}
\end{minipage} & \begin{minipage}[b]{\linewidth}\centering
\textbf{EMIVITA}
\end{minipage} & \begin{minipage}[b]{\linewidth}\centering
\textbf{RANGE IN ACQUA DEI POSITRONI EMESSI (mm)}
\end{minipage} \\
\midrule\noalign{}
\endfirsthead
\toprule\noalign{}
\begin{minipage}[b]{\linewidth}\centering
\textbf{RADIONUCLIDE}
\end{minipage} & \begin{minipage}[b]{\linewidth}\centering
\textbf{EMIVITA}
\end{minipage} & \begin{minipage}[b]{\linewidth}\centering
\textbf{RANGE IN ACQUA DEI POSITRONI EMESSI (mm)}
\end{minipage} \\
\midrule\noalign{}
\endhead
\bottomrule\noalign{}
\endlastfoot
\textbf{F-18} & 110min & 0.46 \\
\textbf{Rb-82} & 75s & 4.10 \\
\end{longtable}

Un parametro fondamentale è il range in mm percorso in acqua dai positroni emessi, poiché, nella PET, non è possibile rilevare il punto di emissione del positrone, ma solo il punto di annichilazione, che, evidentemente, è diverso dal primo, poiché il positrone è in grado di spostarsi del suddetto range. Pertanto, si avrà l'incertezza di circa mezzo mm sul punto di decadimento per il F-18, molto utilizzato in oncologia.

\subsection{Interazione fotoni--materia}\label{interazione-fotonimateria}

Per capire come funziona il meccanismo di ricostruzione, bisogna innanzitutto comprendere quali sono i meccanismi di interazione tra i fotoni \(\gamma\) e la materia. I possibili fenomeni sono:

\begin{itemize}
\item
  Effetto fotoelettrico;
\item
  Diffusione Rayleigh;
\item
  Effetto Compton;
\item
  Produzione di coppie.
\end{itemize}

I vari fenomeni sono generalmente proporzionali al numero atomico Z, all'energia \(E\) e alla densità \(\rho\). È possibile realizzare un grafico dove sull'asse delle ascisse sono riportate le energie, espresse in \(E = hf\), sull'asse delle ordinate il numero atomico del materiale assorbitore. Si ottiene il grafico sperimentale:

\begin{figure}
\centering
\includegraphics[width=6.05006in,height=3.34201in,alt={P3889\#yIS1}]{media/16_PrincFis/image411.pdf}\caption{Figura .: Interazioni fotoni--materia in funzione di E e Z}
\end{figure}

Si nota che, al variare dell'energia e del numero atomico cambia anche la probabilità che si verifichi una delle interazioni citate, per cui, per basse energie, si assiste alla predominanza dell'effetto fotoelettrico; viceversa, per energie molto alte, quindi, non di interesse per la PET, predomina la produzione di coppie.

L'effetto Compton, invece, si verifica molto più probabilmente in un range di energie che va dai 500keV a 7-8MeV. In particolare, all'energie della PET, intorno ai 511keV, per materiali con basso numero atomico si ha la predominanza dell'effetto Compton. Al crescere del numero atomico si assiste alla predominanza dell'effetto fotoelettrico.

Le curve, che separano la probabilità che avvenga un certo tipo di interazione invece di un altro, stanno ad indicare che, in quei punti, la probabilità che avvenga, ad esempio, l'effetto Compton o l'effetto fotoelettrico è la stessa, si ha cioè \(\sigma = \tau\).

\subsubsection{Effetto fotoelettrico}\label{effetto-fotoelettrico-1}

Nell'effetto fotoelettrico, un fotone \(\gamma\) trasferisce interamente la sua energia ad un elettrone interno (\emph{K-Shell}) e l'elettrone viene espulso.

\begin{figure}
\centering
\includegraphics[width=4.45608in,height=2.81667in,alt={P3896\#yIS1}]{media/16_PrincFis/image412.pdf}
\caption{Figura .: Effetto fotoelettrico}
\end{figure}

La probabilità che questo processo avvenga è proporzionale al cubo del numero atomico del materiale assorbente e inversamente proporzionale al cubo dell'energia del fotone incidente:

\[P \propto \dfrac{Z^{3}}{E_{\gamma}^{3}}\]

La vacanza nella \emph{K-Shell} è riempita da un elettrone proveniente dai livelli energetici superiori, a seguito dell'emissione di fotoni X oppure di un elettrone di Auger.

Se il fotone incidente ha energia inferiore all'energia di legame dell'elettrone, l'interazione fotoelettrica non può avvenire. Non appena l'energia dei fotoni incidenti diventa uguale a quella di legame, l'effetto fotoelettrico diviene un meccanismo dominante di interazione. Quando l'energia del fotone incidente aumenta ulteriormente, la probabilità di questa interazione decresce come:

\[\dfrac{1}{E_{\gamma}^{3}}\]

Dopo l'effetto fotoelettrico, l'atomo risulta ionizzato poiché ha perso un elettrone appartenente alle \emph{Shell} più interne, mentre l'elettrone espulso può interagire con altri atomi, causando ulteriori ionizzazioni. Si ha, quindi, una cascata di transizioni elettroniche al fine di riempire la vacanza creata dall'espulsione dell'elettrone interno. Questo processo si manifesta con l'emissione di fotoni con energia pari alla differenza dei due livelli energetici. Siccome ogni atomo possiede dei livelli energetici ben determinati, la radiazione emessa da un materiale è diversa da un altro. A questi fotoni emergenti si dà il nome di radiazione caratteristica. Se il numero atomico dell'atomo preso in considerazione è basso, ovvero Z è piccolo, allora si ha una bassa energia di legame K; per queste sostanze, quindi, predomina l'effetto Auger.

\subsubsection{Diffusione Rayleigh}\label{diffusione-rayleigh}

La diffusione di Rayleigh è anche detta \emph{Scattering} cioè deflessione ed è causata da un elettrone che diffonde un fotone, senza perdita di energia. Il processo è anche detto diffusione coerente.

Durante la diffusione di Rayleigh, l'elettrone aumenta temporaneamente la sua energia, ma non è rimosso dall'atomo, cosa che, invece, accade con l'effetto fotoelettrico. Le energie in gioco, infatti, non sono sufficienti a strappare o scalzare l'elettrone dall'atomo. L'elettrone ritorna al suo livello energetico iniziale emettendo un fotone che ha energia pari a quella del fotone incidente, ma in una direzione leggermente diversa. La probabilità di questa interazione è direttamente proporzionale a Z e inversamente proporzionale a E.

\subsubsection{Effetto Compton}\label{effetto-compton}

Nell'effetto Compton, un fotone \(\gamma\) incidente impatta su un atomo e trasferisce parte della sua energia ad un elettrone del livello energetico più esterno, scalzandolo. Dall'urto emergono un fotone e un elettrone, le cui energie, se sommate, devono essere circa pari all'energia del fotone incidente.

Il fotone emerso può interagire ancora con gli atomi circostanti tramite effetto fotoelettrico o effetto Compton.

\begin{figure}
\centering
\includegraphics[width=5.32548in,height=3.12897in,alt={P3910\#yIS1}]{media/16_PrincFis/image413.pdf}\caption{Figura .: Effetto Compton}
\end{figure}

L'energia dell'elettrone a riposo è data dalla relazione di Einstein:

\[E = mc^{2}\]

Inoltre, l'energia si può anche scrivere come:

\[E = hf = h\dfrac{c}{\lambda}\]

Dove \(h\) è la costante di Planck. Vale la conservazione della quantità di moto, per cui:

\[\mathbf{p}_{i} = \mathbf{p}_{f} + \mathbf{p}_{e^{-}\ diffuso}\]

Vale, inoltre, anche la conservazione dell'energia, secondo cui l'energia del fotone incidente \(E_{fi}\), sommata all'energia iniziale dell'elettrone (coincidente con la sua energia a riposo) \(E_{ei}\) è uguale alla somma dell'energia del fotone emergente \(E_{ff}\), dell'elettrone scalzato \(E_{ef}\) e della sua energia cinetica \(E_{c}\):

\[E_{e_{i}} + E_{f_{i}} = E_{f_{f}} + E_{e_{f}} + E_{c}\]

\[m_{e_{i}}c^{2} + hf_{i} = m_{e_{f}}c^{2} + hf_{f} + \dfrac{1}{2}m_{e_{f}}v^{2}\]

Dove:

\[m_{e_{f}} = \dfrac{m_{e_{i}}}{\sqrt{1 - \left( \dfrac{v^{2}}{c^{2}} \right)}}\]

Scomponendo l'equazione per la conservazione della quantità di moto si ottiene un sistema:

\[\left\{ \begin{array}{r}
\dfrac{h}{\lambda_{i}} = \dfrac{m_{e_{i}}}{\sqrt{1 - \left( \dfrac{v^{2}}{c^{2}} \right)}}\cos\varphi + \dfrac{h}{\lambda_{f}}\cos\varphi \\
\dfrac{m_{e_{i}}}{\sqrt{1 - \left( \dfrac{v^{2}}{c^{2}} \right)}}\sin\theta = \dfrac{h}{\lambda_{f}}\sin\theta
\end{array} \right.\ \]

Dal sistema, aggiungendo anche la conservazione dell'energia, si ricava che:

\[\lambda_{f} - \lambda_{i} = \mathrm{\Delta}\lambda = \dfrac{h}{m_{e_{i}}c}\left( 1 - \cos\varphi \right)\]

Siccome la lunghezza d'onda dipende dall'angolo, è possibile realizzare un grafico, da cui si evince che per basse energie incidenti (linea tratteggiata a 20keV) si ha una probabilità uniforme che i fotoni siano deflessi in tutte le direzioni dello spazio.

Se l'energia incidente aumenta (140keV), diventa più probabile che il fotone emergente abbia la stessa direzione del fotone incidente. Siccome l'angolo di incidenza può essere qualsiasi, l'energia del fotone emesso per effetto Compton varia con continuità a differenza della radiazione caratteristica per l'effetto fotoelettrico.

\begin{figure}
\centering
\includegraphics[width=6.77412in,height=5.79473in,alt={P3929\#yIS1}]{media/16_PrincFis/image414.pdf}\caption{Figura .: Grafico effetto Compton}
\end{figure}

La probabilità dell'effetto è, quindi, proporzionale alla densità del mezzo assorbente e inversamente proporzionale all'energia del fotone che attraversa la materia.

\subsection{Attenuazione della radiazione}\label{attenuazione-della-radiazione}

Nell'attraversamento della materia, un fotone può subire vari processi:

\begin{enumerate}
\def\labelenumi{\Alph{enumi})}
\item
  Il fotone non interagisce con la materia e, quindi, passa indisturbato attraverso di essa;
\item
  Il fotone può subire un'interazione fotoelettrica con l'immissione di una radiazione caratteristica dovuta alla transizione di un elettrone dal livello energetico superiore a quello inferiore e un fotoelettrone;
\item
  Una diffusione coerente di Rayleigh senza perdita di energia;
\item
  Oppure effetto Compton in cui il fotone incidente cede parte della sua energia a un elettrone atomico che viene scalzato.
\end{enumerate}

Questi effetti determinano una riduzione del numero di fotoni in uscita dalla materia.

\begin{figure}
\centering
\includegraphics[width=5.18881in,height=4.02778in,alt={P3939\#yIS1}]{media/16_PrincFis/image415.pdf}\caption{Figura .: Vare tipologie di interazioni}
\end{figure}

I vari fenomeni di assorbimento si verificano ovviamente con diversa probabilità in dipendenza del materiale assorbente, tramite il numero atomico, e dell'energia del fotone incidente.

I vari meccanismi con le relative dipendenze della probabilità con cui si verificano sono riassunti nella tabella successiva:

\begin{figure}
\centering
\includegraphics[width=6.03896in,height=2.68286in,alt={P3943\#yIS1}]{media/16_PrincFis/image416.pdf}\caption{Figura .: Effetti di assorbimento e probabilità di comparsa}
\end{figure}

Tutte le probabilità di assorbimento all'interno della materia possono essere sommate, dando una probabilità complessiva di assorbimento quantificata dal coefficiente di attenuazione lineare \(\mu\), misurato in \(cm^{- 1}\). Si definisce:

\[\mu = \tau + \sigma + \sigma_{R} + \kappa\]

Dove:

\begin{itemize}
\item
  \(\tau\) è il coefficiente dell'effetto fotoelettrico;
\item
  \(\sigma\) è il coefficiente dell'effetto Compton;
\item
  \(\sigma_{R}\) è il coefficiente della diffusione di Rayleigh;
\item
  \(\kappa\) è il coefficiente della produzione di coppie.
\end{itemize}

Il coefficiente di assorbimento lineare, dal punto di vista fisico, rappresenta la probabilità che un fotone sia assorbito in un'unità di lunghezza del materiale che attraversa.

Per l'energia della PET e della radiologia convenzionale, l'effetto Compton e fotoelettrico hanno la maggior probabilità di occorrenza. Dunque, per queste energie il coefficiente lineare di assorbimento lineare dipende essenzialmente da \(\tau\) e \(\sigma\):

\[\mu \simeq \tau + \sigma\]

Da questa relazione è possibile ricavare una legge di tipo esponenziale, per cui il numero di fotoni ad un certo spessore \emph{x} dipende dal numero di fotoni che inizialmente hanno inciso sul materiale moltiplicato per \(e^{- \mu x}\), rappresentate la probabilità che un fotone sia assorbito dal materiale nel percorrere il cammino fino a \emph{x}.

\begin{figure}
\centering
\includegraphics[width=6.21125in,height=2.37483in,alt={P3956\#yIS1}]{media/16_PrincFis/image417.pdf}\caption{Figura .: Attenuazione in scala lineare e semilogaritmica}
\end{figure}

Si definisce lo strato emivalente oppure \emph{Half Value Layer} (HVL) come lo spessore che deve avere un certo materiale con coefficiente di assorbimento lineare \(\mu\) affinché il numero dei fotoni incidente sia dimezzato. Il numero dei fotoni emergenti da un corpo di spessore \emph{x} segue la relazione:

\[N(x) = N_{0}e^{- \mu x}\]

Dove \(N_{0}\) è il numero dei fotoni incidenti mentre \(e^{- \mu x}\) la probabilità con cui essi siano attenuati. Lo spessore emivalente può essere ricavato come:

\[N(HVL) = \dfrac{N_{0}}{2} \Leftrightarrow \dfrac{N_{0}}{2} = N_{0}e^{- \mu HVL}\]

Da cui si ricava:

\[HVL = - {\dfrac{\log\left( \dfrac{1}{2} \right)}{\mu} = \dfrac{\log(2)}{\mu} = \dfrac{0.693}{\mu}}\]

Il concetto di strato emivalente è molto importante nella radioprotezione poiché è necessario che la radiazione scatterizzata dal paziente non colpisca il personale tecnico che esegue l'esame diagnostico oppure pazienti al di fuori della sala PET. Noto il coefficiente di assorbimento \(\mu\) è possibile dimensionare gli strati protettivi necessari a non diffondere la radiazione \(\gamma\) nei locali circostanti.

Spesso la schermatura è in piombo, alluminio e, dosando il giusto spessore delle pareti, si riducono i costi per la sua realizzazione fisica. I materiali utilizzati possiedono uno spessore emivalente piccolo, o equivalentemente un gande valore del coefficiente di attenuazione lineare per quella determinata energia incidente.

Un altro parametro di fondamentale importanza è il \emph{Mass Attenuation Coefficient} o il coefficiente di attenuazione di massa \(\mu_{g}\), che si ottiene dal coefficiente di attenuazione lineare, normalizzato rispetto alla densità del materiale assorbitore:

\[\mu_{g} = \dfrac{\mu}{\rho}\]

Il coefficiente di attenuazione massico è misurato in \(\dfrac{cm^{2}}{kg}\).

Questa quantità è definita perché alcuni materiali, cambiando la densità a seconda dello stato fisico, modificano anche il livello di attenuazione in funzione della densità. Si usa, quindi, il coefficiente di attenuazione massico \(\mu_{g}\) per compensare queste variazioni.

Studi sperimentali, presenti in letteratura, hanno permesso di rappresentare, in scala logaritmica, i coefficienti di attenuazione in funzione dell'energia espressa in MeV.

Per l'aria, il diagramma presenta una linea continua rappresentante il reale andamento del coefficiente di attenuazione massico dato dalla somma dei vari contributi dell'attenuazione. Le linee tratteggiate, invece, rappresentano il contributo di un singolo fenomeno di assorbimento, cioè effetto Compton, fotoelettrico ed ecc. Grazie alla scala logaritmica, la probabilità di avere l'effetto fotoelettrico o lo \emph{Scattering} di Rayleigh presentano un andamento pressoché lineare.

\begin{figure}
\centering
\includegraphics[width=3.58653in,height=3.06389in,alt={P3972\#yIS1}]{media/16_PrincFis/image418.pdf}\caption{Figura .: Coefficiente di attenuazione lineare dell'aria}
\end{figure}

Analogamente per quanto accade per l'aria, studi sperimentali hanno determinato l'andamento dei vari fenomeni di assorbimento che contribuiscono al coefficiente di assorbimento massico per acqua, piombo e tessuti molli.

Per l'acqua è possibile notare un andamento del coefficiente di attenuazione di massa molto simile a quello dell'aria. In entrambi i casi, per le energie della PET l'effetto principale che attenua la radiazione incidente è l'effetto Compton.

\begin{figure}
\centering
\includegraphics[width=3.40697in,height=2.98662in,alt={P3976\#yIS1}]{media/16_PrincFis/image419.pdf}\caption{Figura .: Coefficiente di attenuazione lineare dell'acqua}
\end{figure}

Per il piombo, è interessante osservare il fenomeno del \emph{K-Edge}, in corrispondenza del quale si ha una forte variazione del potere di assorbimento. Questo fenomeno si verifica poiché i fotoni, dell'energia in corrispondenza della quale il coefficiente di assorbimento massico presenta una discontinuità, riescono a scalzare gli elettroni più interni, situati nel \emph{K Shell}, per effetto fotoelettrico. Gli elettroni più interni richiedono, infatti, energie molto importanti per poter essere estratti. Per energie superiori il coefficiente di assorbimento aumenta poiché i fotoni sono assorbiti anche dagli elettroni nello strato K, non possibile con energie inferiori.

\begin{figure}
\centering
\includegraphics[width=3.34595in,height=2.88178in,alt={P3979\#yIS1}]{media/16_PrincFis/image420.pdf}\caption{Figura .: Coefficiente di attenuazione lineare del piombo}
\end{figure}

\begin{figure}
\centering
\includegraphics[width=4.50027in,height=4.172in,alt={P3981\#yIS1}]{media/16_PrincFis/image421.pdf}\caption{Figura .: Coefficiente di attenuazione lineare dei tessuti molli}
\end{figure}

È interessante osservare anche il confronto tra materiali differenti, come piombo, iodio, osso, tessuti molli e grasso:

\begin{figure}
\centering
\includegraphics[width=6.07301in,height=4.48349in,alt={P3984\#yIS1}]{media/16_PrincFis/image422.pdf}\caption{Figura .: Confronto tra coefficienti di attenuazione tra materiali differenti}
\end{figure}

Come si può notare, osso, tessuti molli e grasso hanno coefficienti di attenuazione più bassi rispetto a piombo e iodio; inoltre, all'aumentare dell'energia, le tre curve dei materiali biologici tendono ad unirsi. In termini pratici, se si realizza un \emph{Imaging} convenzionale ad energie molto elevate, osso, grasso e tessuti molli avranno lo stesso coefficiente di attenuazione lineare, quindi, lo stesso tipo di assorbimento. Ne consegue che non è conveniente realizzare un \emph{Imaging} convenzionale a raggi X con energie così elevate, poiché i vari tessuti non sarebbero contrastati, rendendo l'immagine poco chiara. Solitamente, si realizza un \emph{Imaging} a più bassa energia così da massimizzare il contrasto, pagando, ovviamente, il prezzo di un maggiore rumore sull'immagine.

Dal confronto con tutti i diagrammi, è possibile osservare che il coefficiente di assorbimento massico è una funzione decrescente dell'energia. Quindi, per la maggior parte dei materiali, esso si riduce all'aumentare dell'energia.

\begin{center}
\vfill
    \chapter{Principio fisico dei detettori}
    \label{blx:Detettori\therefsection}
\vfill

\minitoc
\newpage
\end{center}
\justify

\section{Detettori}\label{detettori}

I fotoni emergenti dal corpo del paziente, che non hanno interagito con i suoi tessuti, devono essere rilevati da apposite sezioni hardware dette detettori. Da uno schema generale di un detettore, è possibile notare che questo si compone essenzialmente di due parti:

\begin{enumerate}
\def\labelenumi{\arabic{enumi})}
\item
  Trasduttore, che converte il fotone in energia elettrica;
\item
  Elettronica per il \emph{Signal Processing}.
\end{enumerate}

\begin{figure}
\centering
\includegraphics[width=3.86458in,height=2.14583in,alt={P3993\#yIS1}]{media/17_Detettori/image423.pdf}\caption{Figura .: Schema generale di un detettore}
\end{figure}

Il trasduttore è la regione di detettore sensibile alla radiazione incidente, che produce eccitazione e ionizzazione. Lo stato fisico del trasduttore e la densità determinano quale dei due processi è favorito. Successivamente vi è uno stadio elettronico che converte l'interazione della radiazione con il trasduttore in un segnale elettrico misurabile, che, successivamente, è processato, analizzato e contato.

I detettori solitamente funzionano in \emph{Pulse Mode}, ovvero in modalità pulsata, poiché i vari impulsi di corrente successivi possono essere di varia intensità. Tipicamente, l'elettronica a valle, dal punto di vista del detettore, può essere schematizzata come un'impedenza di tipo RC parallelo, dove R è la resistenza di ingresso al circuito e C è la capacità equivalente del detettore e del circuito di misura. Il detettore rappresenta il forzamento del sistema RC e può essere schematizzato come un generatore di corrente.

\begin{figure}
\centering
\includegraphics[width=5.54865in,height=1.03013in,alt={P3997\#yIS1}]{media/17_Detettori/image424.pdf}\caption{Figura .: Elettronica a valle del detettore}
\end{figure}

Tale considerazione è di fondamentale importanza poiché permette di capire delle importanti caratteristiche dei detettori. Questi, infatti, possono essere di tipo \emph{Small RC} o \emph{Large RC}, a seconda che la costante di tempo \(\tau = RC\) sia piccola o grande. Se \(\tau\) è piccola, l'elettronica a valle reagisce prontamente all'impulso di corrente, quindi, la tensione misurata avrà lo stesso andamento temporale dell'impulso di corrente che arriva dal detettore.

\begin{figure}
\centering
\includegraphics[width=3.93328in,height=3.0052in,alt={P4000\#yIS1}]{media/17_Detettori/image425.pdf}\caption{Figura .: Risposta per Small R}
\end{figure}

Se, invece, \(\tau\) è grande, cioè l'elettronica ha un'evoluzione temporale molto lenta, lo stadio in uscita si comporta come un integratore, ovvero come un filtro passa-basso.

\begin{figure}
\centering
\includegraphics[width=4.61683in,height=1.90724in,alt={P4003\#yIS1}]{media/17_Detettori/image426.pdf}\caption{Figura .: Risposta per Large RC}
\end{figure}

Ovviamente, in base all'applicazione, si può scegliere uno dei due comportamenti, per cui, l'elettronica a valle è progettata per fornire la tensione di uscita desiderata.

\subsection{Caratteristiche del detettore}\label{caratteristiche-del-detettore}

Prima di analizzare in dettaglio le caratteristiche, è importante notare che i decadimenti, si verificano in maniera casuale nel tempo, di conseguenza anche l'emissione dei fotoni \(\gamma\) e l'annichilazione dei positroni sono casuali.

Gli eventi casuali e indipendenti gli uni dagli altri obbediscono al processo statistico di Poisson. Tra due eventi successivi trascorre un intervallo di tempo che ha una distribuzione esponenziale. Per un gran numero di eventi, la distribuzione di Poisson, per il teorema del limite centrale, tende a una distribuzione gaussiana.

Il detettore è caratterizzato da una serie di grandezze che ne determinano le specifiche come trasduttore da energia elettromagnetica a energia elettrica come tensione.

\begin{itemize}
\item
  L'\emph{Energy Resolution} rappresenta l'indeterminazione con cui è rilevata l'energia di un fotone. Per comprendere la distribuzione energetica del detettore, si costruisce un grafico che presenta l'ampiezza dell'impulso sull'asse delle ascisse e il numero di impulsi rilevati dal detettore sull'asse delle ordinate.
\end{itemize}

Dal punto di vista operativo, si irradia il dispositivo di rilevamento con una radiazione di energia nota e si misurano il numero di impulsi rilevati dal detettore stesso. Si ottiene un andamento del tipo:

\begin{figure}
\centering
\includegraphics[width=6.17527in,height=3.57124in,alt={P4012\#yIS1}]{media/17_Detettori/image427.pdf}\caption{Figura .: Differential Pulse Height Distribution}
\end{figure}

Si realizza, così, una sorta di istogramma delle energie. Esisterà, quindi, un certo intervallo di energie per cui il detettore riesce a rilevare un maggior numero di impulsi, cioè risulta più sensibile; mentre in altri range energetici il detettore si mostra meno sensibile.

In genere, si assiste alla presenza di una regione a bassa energia in cui i detettori hanno una buona sensibilità. Dunque, per le basse energie, i detettori sono generalmente in grado di rilevare gli impulsi, cosa che, invece, risulta più difficile per le alte energie. Si parla, perciò, di sensibilità energetica del detettore.

Idealmente, si vorrebbe che il detettore sia perfettamente sensibile alle energie di interesse come 511keV per PET, in modo da discriminare i fotoni relativi all'annichilamento rispetto a quelli generati a causa di altri effetti di \emph{Scattering} come effetto Compton o effetto fotoelettrico.

Avendo una sensibilità molto limitata, tutti i fotoni che hanno interagito con la materia non sono rilevati poiché la loro energia è esterna al piccolo intorno di 511keV.

Nella realtà il detettore si mostra sensibile per un range di energie abbastanza esteso e ciò porta a un'indeterminazione sulla misura poiché sono rilevati sia i fotoni provenienti dall'annichilamento positronico ma anche quelli deviati per effetto Compton.

\begin{figure}
\centering
\includegraphics[width=4.31484in,height=2.48637in,alt={P4019\#yIS1}]{media/17_Detettori/image428.pdf}\caption{Figura .: Risoluzione energetica caso ideale}
\end{figure}

La risoluzione energetica ha una variabilità dovuta a una serie di diverse cause:

\begin{itemize}
\item
  \emph{Drift} delle caratteristiche del detettore durante il funzionamento. Questo elemento, infatti, può cambiare le sue caratteristiche nel tempo sia a causa dell'usura sia aumenti di temperatura o altro;
\item
  Rumore casuale all'interno del detettore, dovuto alla presenza dei materiali droganti nella struttura cristallina;
\item
  Rumore della strumentazione introdotto dalla circuiteria elettronica;
\item
  Rumore statistico dovuto alla natura discreta del segnale misurato che rappresenta il contributo predominante per la degradazione dell'immagine.
\end{itemize}

La risoluzione energetica è definita considerando il picco dell'istogramma precedentemente illustrato, di cui si valuta l'ampiezza a mezza altezza (FWHM).

\begin{figure}
\centering
\includegraphics[width=4.27733in,height=3.93486in,alt={P4027\#yIS1}]{media/17_Detettori/image429.pdf}\caption{Figura .: Valutazione a mezza altezza del picco}
\end{figure}

Se l'energia è \(H_{0}\), la risoluzione si calcola come:

\[R = \frac{FWHM}{H_{0}}\]

Ovviamente, più è stretto il picco, migliore sarà la risoluzione energetica;

\begin{itemize}
\item
  L'efficienza assoluta (\(\varepsilon_{abs}\)) è definita come il numero di eventi rilevati in un certo intervallo di tempo rispetto al numero di fotoni emessi dalla sorgente nello stesso intervallo. La radiazione è emessa in maniera isotropica, cioè in tutte le direzioni. Questa quantità è ottenuta misurando il numero di quanti rilevati rispetto a quelli effettivamente emessi. I fotoni che non sono rilevati hanno subito un processo di assorbimento con la materia oppure il detettore non è stato in gradi di rilevarli per motivi di carattere statistico;
\item
  La grandezza che realmente quantifica il funzionamento del detettore è l'efficienza intrinseca (\(\varepsilon_{int}\)), definita come il numero di eventi rilevati in un dato intervallo di tempo rispetto al numero di fotoni incidenti sul detettore nello stesso intervallo. Non tutti i fotoni incidenti sul dettore sono rilevati, quindi, alcuni fotoni potrebbero essere persi, riducendo l'efficienza della trasduzione.
\end{itemize}

L'efficienza intrinseca per i fotoni \(\gamma\) coincide con la probabilità che questi hanno di impattare sul detettore, cioè con:

\[P = 1 - e^{- \mu x}\]

Dove \(\mu\) è funzione di densità, numero atomico del materiale di rilevazione ed energia della radiazione incidente ed \(e^{- \mu x}\) la probabilità che esso rilevi i fotoni. L'efficienza intrinseca è, inoltre, influenzata dallo \emph{Stopping Power} tramite \(\mu\). La relazione che intercorre tra efficienza assoluta e intrinseca è la seguente:

\[\varepsilon_{int} = \varepsilon_{abs}\frac{4\pi}{\Omega}\]

Dove \(\Omega\) è l'angolo solido del detettore visto dalla posizione della sorgente;

\begin{itemize}
\item
  L'efficienza di picco tiene conto solo di quelle interazioni che depositano l'intera energia; per ottenerne una stima si integra l'area della \emph{Pulse-Height Distribution}.
\end{itemize}

\begin{figure}
\centering
\includegraphics[width=3.74489in,height=3.56885in,alt={P4040\#yIS1}]{media/17_Detettori/image430.pdf}\caption{Figura .: Pulse-Height Distribution}
\end{figure}

\begin{itemize}
\item
  L'efficienza geometrica è il numero di fotoni incidenti sul detettore rispetto al numero di fotoni emessi dalla sorgente:
\end{itemize}

\begin{figure}
\centering
\includegraphics[width=4.1105in,height=3.20988in,alt={P4043\#yIS1}]{media/17_Detettori/image431.pdf}\caption{Figura .: Efficienza geometrica}
\end{figure}

Più grande è il detettore e più vicino alla sorgente, maggiore sarà la sua efficienza geometrica, poiché intercetta più eventi. Per la scelta del detettori è necessario selezionare dei materiali col più alto coefficiente di assorbimento e con una geometria tale da evitare la dispersione dei fotoni nello spazio. Ovviamente è da considerare anche la risoluzione geometrica legata proprio al materiale utilizzato;

\begin{itemize}
\item
  Lo \emph{Stopping Power} per i fotoni \(\gamma\) è legato al coefficiente di attenuazione lineare \(\mu\). Inoltre, è influenzato da spessore, densità, numero atomico del materiale assorbente ed energia della radiazione incidente;
\end{itemize}

\begin{figure}
\centering
\includegraphics[width=6.05302in,height=2.37086in,alt={P4047\#yIS1}]{media/17_Detettori/image432.pdf}\caption{Figura .: Caratteristiche che influenzano lo Stopping Power per diversi tipi di materiali}
\end{figure}

Lo ioduro di sodio attivato al tallio, che rappresenta l'elemento drogante, a parità di spessore con altri materiali, è sufficientemente massiccio, con un coefficiente di attenuazione lineare di 0.504cm\textsuperscript{-1}. Il germanato di bismuto, invece, presenta un coefficiente di attenuazione lineare maggiore di 0.920cm\textsuperscript{-1}. Per la realizzazione dei detettori della PET è, quindi, molto usato il germanato di bismuto per il suo grande valore di attenuazione;

\begin{itemize}
\item
  Il \emph{Dead Time} è definito come il minimo intervallo di tempo che deve intercorrere tra due eventi per poter essere rilevati separatamente. Esso dipende dal detettore e dall'elettronica posta a valle di quest'ultimo. A causa della natura casuale del decadimento radioattivo, vi saranno eventi non rilevati, poiché si verificano dopo un intervallo di tempo troppo piccolo rispetto all'evento precedente. Esistono essenzialmente due modalità di funzionamento per un detettore: paralizzabile e non paralizzabile.
\end{itemize}

Se un impulso giunge sul cristallo detettore mentre il fotone precedente è processato, questo non è rilevato e si dice che l'impulso è rigettato, per cui il detettore è detto non paralizzabile. Se il rigetto del secondo evento prolunga il \emph{Dead Time} in modo da impedire la detezione di un terzo evento, distanziato di \(\tau\) dal primo impulso, il detettore è detto paralizzabile.

Per trasdurre un fotone in un segnale elettronico, il rilevatore impiega un certo tempo \(\tau\) che scandisce la minima distanza temporale in cui due fotoni possono essere rilevati. Se il detettore è non paralizzabile e un fotone incide sul detettore mentre un fotone è processato, il tempo \(\tau\) di lavorazione del fotone non varia. Il secondo fotone è perso ma dopo un tempo \(\tau\) dal primo è possibile riconoscere nuovamente un fotone, anche se, durante il processo di conversione, altri fotoni sono giunti sul detettore.

\begin{figure}
\centering
\includegraphics[width=3.48531in,height=3.17569in,alt={P4053\#yIS1}]{media/17_Detettori/image433.pdf}\caption{Figura .: Logica del Dead Time}
\end{figure}

I materiali reali presentano delle caratteristiche intermedie tra il comportamento paralizzabile e non paralizzabile. Esistono ovviamente dei materiali che tendono ad approssimare un comportamento piuttosto che un altro.

Il \emph{Dead Time} determina la perdita di eventi, quindi, si registra un tasso di interazione inferiore rispetto a quello che realmente giunge al detettore. Inoltre, è molto importante se il tasso di conteggio è molto alto, infatti, in queste condizioni, le misure devono essere opportunamente corrette. I parametri che caratterizzano il \emph{Dead Time}, indicato con \(\tau\), sono:

\begin{itemize}
\item
  \emph{n}, ovvero il tasso reale di interazione con il detettore;
\item
  \emph{m}, ovvero il tasso registrato.
\end{itemize}

Per un impulso non paralizzabile il prodotto:

\[m\tau\]

restituisce la frazione di tempo in cui il detettore è ``morto'', nel senso che non può rilevare nessun evento di incidenza. La quantità:

\[nm\tau\]

è il tasso di perdita degli eventi veri, che può anche essere definito come:

\[n - m\]

Quindi, dall'eguaglianza delle due definizioni è possibile ottenere una relazione che permette di valutare il tasso reale incidente \emph{n} dato il tatto registrato \emph{m}:

\[nm\tau = n - m\]

Per un impulso paralizzabile, invece, si definisce la distribuzione \(P(T)\) degli intervalli tra gli eventi, data dalla distribuzione di Poisson:

\[P(T) = n\ e^{- nT}\]

Con T variabile aleatoria indicante l'intervallo tra due eventi. La probabilità di osservare un intervallo più lungo di \(\tau\) tra due eventi è:

\[ne^{- n\tau}\]

Quindi, il tasso di occorrenza degli intervalli più lunghi di \(\tau\) è pari al tasso degli eventi registrati:

\[m = n\ e^{- n\tau}\]

\begin{figure}
\centering
\includegraphics[width=5.60702in,height=3.17822in,alt={P4073\#yIS1}]{media/17_Detettori/image434.pdf}\caption{Figura .: Eventi registrati in funzione degli eventi reali}
\end{figure}

Come si nota dal grafico, è possibile schematizzare il tasso registrato \emph{m} in funzione del tasso reale \emph{n}, ottenendo una curva per il detettore paralizzabile e un'altra per il detettore non paralizzabile. La differenza sta nel fatto che, nel caso del paralizzabile, dato un certo tasso \(m_{1}\) di eventi registrati, si avrà una corrispondenza con due tassi reali differenti, mentre nel caso del non paralizzabile, la corrispondenza è biunivoca, perché dato un certo numero di eventi registrati, ci sarà un unico valore corrispondente di eventi reali. Quindi, nel caso di detettore non paralizzabile, si può ricavare, per formula inversa, il numero di eventi reali, partendo dagli eventi registrati, cosa che invece non può essere effettuata con un detettore paralizzabile a causa dell'incertezza per la non biunivocità.

Se lo stadio elettronico è in \emph{Pulse Mode}, ciascun evento è processato separatamente. L'ampiezza dell'impulso è proporzionale all'energia del fotone incidente e la durata dell'impulso limita il tasso di rilevamento, poiché ogni evento deve essere processato separatamente.

\subsection{Detettori a scintillazione}\label{detettori-a-scintillazione}

I cristalli di scintillazione sono i materiali attualmente maggiormente utilizzati per la realizzazione dei detettori della PET.

Il fenomeno della scintillazione prevede che, quando un fotone incide sul materiale scintillatore, si ha la generazione di un impulso luminoso visibile.

Gli scintillatori rilasciano luce ultravioletta o visibile, tipicamente blu, quando interagiscono per effetto fotoelettrico o effetto Compton. Essi sono realizzati con dei cristalli, tra i quali sono molto adoperati quelli basati su ioduro di sodio o germanato di bismuto, drogati con delle impurità, cioè con delle specie atomiche differenti.

Gli effetti fotoelettrico o Compton si realizzano all'interno del materiale e corrispondono ad un'eccitazione degli elettroni del materiale. Gli elettroni nei cristalli possono assumere bande energetiche discrete, dove le bande energetiche più elevate sono la banda di valenza e la banda di conduzione, separate da un \emph{Band Gap} di valori energetici proibiti per l'elettrone. La banda superiore è detta di valenza, dove risiedono gli elettroni più energetici. Se queste particelle ricevono energie tali da svincolarsi dal legame chimico in cui risiedono, passano alla banda di conduzione.

\begin{figure}
\centering
\includegraphics[width=6.19685in,height=3.3084in,alt={P4082\#yIS1}]{media/17_Detettori/image435.pdf}\caption{Figura .: Schema bande energetiche}
\end{figure}

All'interno della banda proibita possono, però, trovarsi altri livelli energetici in prossimità delle impurità droganti. In genere, quindi, il cristallo di scintillazione non è puro: se lo fosse, ci sarebbe solamente banda di valenza e banda di conduzione. Nei cristalli di scintillazione comunemente utilizzati, si introducono, invece, delle impurità droganti, che vanno ad occupare delle vacanze all'interno della struttura cristallina oppure introducono dei difetti nella struttura atomica.

\begin{figure}
\centering
\includegraphics[width=5.60868in,height=1.13067in,alt={P4085\#yIS1}]{media/17_Detettori/image436.pdf}\caption{Figura .: Impurità nei cristalli}
\end{figure}

Poiché i livelli energetici delle impurità sono differenti da quelli del resto del cristallo, in prossimità delle impurità si formano degli ulteriori livelli energetici che non sarebbero presenti normalmente nel gap energetico. Nei semiconduttori, il \emph{Band Gap} è dell'ordine di 1-1.3eV, mentre nei cristalli di scintillazione è dell'ordine di 3-4eV.

I nuovi livelli energetici introdotti nel \emph{Band Gap} hanno un ruolo fondamentale nel processo di scintillazione e sono indicati come stati attivati di eccitazione.

\subsubsection{Bande energetiche}\label{bande-energetiche}

Il cristallo puro è una successione periodica di atomi della stessa specie. Avendo una struttura periodica, anche il potenziale che si viene a creare è periodico. Ogni nucleo, infatti, genera un proprio potenziale coulombiano e, dalle sovrapposizioni con gli altri potenziali dei nuclei nella struttura cristallina, si ottiene il potenziale periodico nella struttura cristallina.

\begin{figure}
\centering
\includegraphics[width=2.82067in,height=3.89833in,alt={P4091\#yIS1}]{media/17_Detettori/image437.pdf}\caption{Figura .: Potenziale periodico dalla struttura cristallina}
\end{figure}

Un elettrone immerso nel potenziale periodico risulta possedere bande di livelli energetici permessi separati da bande proibite. Questo comportamento può essere determinato risolvendo l'equazione di Schrödinger per un elettrone posto in un potenziale periodico semplificato monodimensionale, noto come modello di Kronig-Penney.

\begin{figure}
\centering
\includegraphics[width=5.23725in,height=2.06336in,alt={P4094\#yIS1}]{media/17_Detettori/image438.pdf}\caption{Figura .: Potenziale periodico semplificato del modello Kroning-Penney}
\end{figure}

Per il teorema di Bloch, le funzioni d'onda devono avere una forma periodica del tipo:

\[\psi(x) = u(x)e^{jkx}\]

dove \(u(x)\) è una funzione periodica e \(k\) è un fattore di fase. Se il potenziale fosse ovunque nullo, ovvero se l'elettrone fosse libero, si avrebbe, includendo anche il tempo:

\[\Psi(x,t) = \psi(x) \cdot \Phi(t) = u(x)e^{jkx}e^{- j\frac{E}{\hslash}t} = u(x)e^{j\left( kx - \frac{E}{\hslash}t \right)}\]

Da cui si evince l'interpretazione di \(k\) come numero d'onda. Scrivendo le equazioni per le regioni I, in cui il potenziale è nullo, e II, dove il potenziale ha un valore \(V_{0}\), si ottengono due equazioni per \emph{u(x)} nelle due regioni:

\[\frac{d^{2}u_{I}}{dx^{2}} + 2jk\frac{du_{I}}{dx} - \left( k^{2} - \alpha^{2} \right)u_{I}(x) = 0\]

\[\frac{d^{2}u_{II}}{dx^{2}} + 2jk\frac{du_{II}}{dx} - \left( k^{2} - \beta^{2} \right)u_{II}(x) = 0\]

Dove

\[\alpha^{2} = \frac{2mE}{\hslash^{2}}\]

\[\beta^{2} = \frac{2mE}{\hslash^{2}} - \frac{2mV_{0}}{\hslash^{2}}\]

Con \(E\) energia, \(m\) massa, \(V_{0}\) potenziale nella regione II. Le soluzioni sono del tipo:

\[u_{I}(x) = Ae^{j(\alpha - k)x} + Be^{j(\alpha + k)x}\]

\[u_{II}(x) = Ce^{j(\beta - k)x} + De^{j(\beta + k)x}\]

Le condizioni al contorno da imporre per ottenere la soluzione riguardano la continuità della funzione d'onda all'interfaccia tra due regioni di potenziale. Si ottengono le condizioni al contorno:

\[\left\{ \begin{array}{r}
u_{I}(0) = u_{II}(0) \\
u_{I}(a) = u_{II}( - b) \\
\left. \ \frac{du_{I}}{dx} \right|_{x = 0} = \left. \ \frac{du_{II}}{dx} \right|_{x = 0} \\
\left. \ \frac{du_{I}}{dx} \right|_{x = a} = \left. \ \frac{du_{II}}{dx} \right|_{x = - b}
\end{array} \right.\ \]

Si ottiene così il sistema di equazioni:

\[\left\{ \begin{array}{r}
A + B = C + D\ \ \ \ \ \ \ \ \ \ \ \ \ \ \ \ \ \ \ \ \ \ \ \ \ \ \ \ \ \ \ \ \ \ \ \ \ \ \ \ \ \ \ \ \ \ \ \ \ \ \ \ \ \ \ \ \ \ \ \ \ \ \ \ \ \ \ \ \ \ \ \ \ \ \ \ \ \ \ \ \ \ \ \ \ \ \ \ \ \ \ \ \ \ \ \ \ \ \ \ \ \ \ \ \ \ \ \ \ \ \ \ \ \ \ \ \ \ \ \ \ \ \ \ \ \ \ \ \ \ \ \  \\
Ae^{j(\alpha - k)a} + Be^{j(\alpha + k)a} = \ Ce^{- j(\beta - k)b} + De^{- j(\beta + k)b}\ \ \ \ \ \ \ \ \ \ \ \ \ \ \ \ \ \ \ \ \ \ \ \ \ \ \ \ \ \ \ \ \ \ \ \ \ \ \ \ \ \ \ \ \ \ \ \ \ \ \ \ \ \ \ \ \ \ \ \ \ \ \ \  \\
Aj(\alpha - k) + Bj(\alpha + k) = Cj(\alpha - k) + Dj(\alpha + k)\ \ \ \ \ \ \ \ \ \ \ \ \ \ \ \ \ \ \ \ \ \ \ \ \ \ \ \ \ \ \ \ \ \ \ \ \ \ \ \ \ \ \ \ \ \ \ \ \ \ \ \ \ \ \ \ \ \ \ \ \ \ \ \ \ \ \  \\
Aj(\alpha - k)e^{j(\alpha - k)a} + Bj(\alpha + k)e^{j(\alpha + k)a} = Cj(\alpha - k)e^{- j(\alpha - k)b} + Dj(\alpha + k)e^{- j(\alpha + k)b}
\end{array} \right.\ \]

La risoluzione del sistema porta all'equazione

\[\cos\left( k(a + b) \right) = \frac{\alpha^{2} + \beta^{2}}{2\alpha\beta}\sin(a\alpha) + \sin(b\beta) + \cos(a\alpha)\cos(b\beta)\]

Le soluzioni dell'equazione di Schrödinger esistono se questa relazione ammette soluzioni. Se \(b \rightarrow 0\) e \(V_{0} \rightarrow \infty\), ovvero si fa tendere la regione II a un impulso di Dirac, si approssima il potenziale in corrispondenza del nucleo tendente, appunto, all'infinito. È possibile, così, semplificare l'equazione:

\[\cos(ka) = \frac{mV_{0}ab}{\hslash^{2}}\frac{\sin(a\alpha)}{a\alpha} + \cos(a\alpha)\]

Questa equazione non può essere risolta analiticamente. Visualizzando il secondo membro, posto uguale a \(f(a\alpha)\), e tenendo conto che il primo membro può assumere solo valori compresi in (-1; 1), si vede che i valori di energia \(E\) sono collegati ai valori di \(k\) ed è possibile trovare la soluzione mediante il metodo grafico:

\begin{figure}
\centering
\includegraphics[width=6.21409in,height=2.33524in,alt={P4118\#yIS1}]{media/17_Detettori/image439.pdf}\caption{Figura .: Risoluzione grafica dell'equazione}
\end{figure}

Le soluzioni sono discrete dove ogni intervallo è disgiunto dal precedente. Pertanto, le soluzioni che si ottengono corrispondono a delle bande di energia discrete e discontinue. Esistono, quindi, intervalli di energie in cui la funzione d'onda non è definita, dunque, l'elettrone non può assumere quelle energie. È possibile visualizzare le soluzioni tracciando l'andamento dell'energia in funzione del variare di \(k\):

\begin{figure}
\centering
\includegraphics[width=5.99118in,height=5.51504in,alt={P4121\#yIS1}]{media/17_Detettori/image440.pdf}\caption{Figura .: Relazione tra energia E e k}
\end{figure}

La presenza di un elemento drogante nel reticolo cristallino introduce delle bande intermedie tra la banda di conduzione e quella di valenza, note come stati eccitati di attivazione. Un elettrone può, quindi, passare dalla banda di valenza a quella intermedia se riceve energia sufficiente.

\subsubsection{Funzionamento dei cristalli di scintillazione}\label{funzionamento-dei-cristalli-di-scintillazione}

Un fotone \(\gamma\) con energia pari a 511keV può interagire con gli elettroni della banda di valenza per effetto fotoelettrico o effetto Compton, portandoli nella banda di conduzione e innescando un processo a cascata che distribuisce l'energia iniziale della radiazione incidente su tutti gli elettroni e i fotoni emessi. In seguito al passaggio di elettroni in banda di conduzione, si generano, contemporaneamente, delle lacune in banda di valenza, ovvero siti in cui non c'è carica negativa, ma positiva, che sono in grado di migrare all'interno del cristallo. Gli elettroni migrano finché la loro energia non si abbassa ulteriormente, arrivando sul bordo inferiore della banda di conduzione. A questo punto, potrebbero transitare verso la banda di valenza, cioè perdere ancora energia e andare ad occupare una lacuna, attraverso un meccanismo noto come processo di ricombinazione elettrone--lacuna.

Un altro processo possibile prevede che, se gli elettroni e le lacune si trovano in prossimità di un sito di attivazione, ovvero il sito di impurità drogante, l'elettrone, invece di transitare dalla banda di conduzione alla banda di valenza, passa verso i livelli energetici intermedi, posti nella banda proibita del sito attivatore. Dopo un tempo di permanenza variabile, dell'ordine di ns, si ha un'ulteriore perdita energetica, per cui l'elettrone transita verso un livello energetico inferiore, come ad esempio il livello più basso del sito di attivazione, noto come \emph{Ground State}, da cui passa, poi, alla banda di valenza. A ogni transizione degli elettroni da una banda di energia all'altra, si generano dei fotoni di energia pari al gap di transizione.

Nel processo di ricombinazione elettrone--lacuna la radiazione potrebbe trovarsi nell'ultravioletto, mentre nel secondo processo i siti energetici introdotti dalle impurità sono ad un livello energetico per il quale la radiazione ottenuta è nello spettro del visibile. Il fenomeno di emissione di luce visibile per breve tempo è detto fluorescenza, quindi, i processi che portano alla formazione della radiazione luminosa si verificano in tempo dell'ordine dei ns. Viceversa, quando la luce visibile viene emessa per un tempo più lungo, si parla di fosforescenza. In questo caso, i siti di attivazione possono intrappolare un elettrone anche per un tempo più lungo dei ns. La luce è, quindi, rilasciata dopo un certo ritardo dall'assorbimento dei fotoni \(\gamma\).

In genere, si preferisce la fluorescenza, in quanto, all'arrivo del fotone \(\gamma\), si vorrebbe la generazione immediata della radiazione, poiché quella generata in un secondo momento diventa un disturbo, che si sovrappone ad altre radiazioni successive.

Tuttavia, i materiali reali non sono né esattamente fluorescenti né fosforescenti ma mostrano un comportamento approssimabile in uno dei due meccanismi di funzionamento.

\subsubsection{Caratteristiche ideali di uno scintillatore}\label{caratteristiche-ideali-di-uno-scintillatore}

Nella pratica, si vorrebbe che il cristallo scintillatore presenti delle caratteristiche ideali quali:

\begin{itemize}
\item
  Breve tempo di interazione e di \emph{Stopping Time}. Per raggiungere questi obiettivi, nella PET si usano dei detettori solidi con un \emph{Stopping Time} dell'ordine dei ps;
\item
  Ad ogni fotone \(\gamma\) deve corrispondere un impulso di corrente rilevabile dalla circuiteria a valle;
\item
  Funzionamento in modalità pulsata (\emph{Pulse Mode}), al fine di rilevare ogni singolo impulso. In altre parole, la durata dell'impulso di corrente del rilevatore deve essere così breve che impulsi adiacenti siano completamente non interagenti;
\item
  Possibile conversione di fotoni \(\gamma\) in fotoni visibili. Si tende, infatti, a non utilizzare gap che generano una radiazione ultravioletta;
\item
  Conversione lineare nel senso che il numero di fotoni visibili deve essere proporzionale alla energia incidente. Con una \emph{Band Gap} di 3eV, in caso di relazione perfettamente lineare, si avrebbero circa \(1.7 \cdot 10^{5}\) fotoni luminosi.
\end{itemize}

Ovviamente i materiali non hanno un'efficienza di conversione del 100\%, quindi, il numero dei fotoni è minore;

\begin{itemize}
\item
  Il cristallo di scintillazione deve essere trasparente alla lunghezza d'onda della propria emissione, ovvero il cristallo di scintillazione non deve assorbire il fotone emesso da un elettrone che passa dalla banda di conduzione a quella di valenza. A tale scopo si inseriscono i siti di impurità così che il fotone emergente dalla transizione sito di attivazione a banda di valenza non abbia l'energia per portare altri elettroni alla banda di valenza e, quindi, non sia assorbito. In assenza di drogante non emergerebbe nessuna radiazione dal cristallo poiché appena un fotone è liberato per una transizione di stato di un elettrone, è assorbito da un altro che passa nella banda di conduzione, ricominciando il ciclo. Si vorrebbe, quindi, che lo spettro di emissione del cristallo sia disgiunto dallo spettro di assorbimento. Ciò è realizzato grazie ai droganti;
\item
  Il tempo di decadimento della luminescenza (fluorescenza) prodotta deve essere breve. Esso dipende dai materiali utilizzati e dai tipi di drogati contenuti;
\item
  Indice di rifrazione vicino a quello del vetro per l'accoppiamento con lo stadio successivo, ovvero il tubo fotomoltiplicatore.
\end{itemize}

\subsubsection{Afterglow (fosforescenza)}\label{afterglow-fosforescenza}

Nell'\emph{Afertglow}, la luminescenza si verifica con un tempo di ritardo molto lungo, il che è un fenomeno indesiderato nel caso della PET, perché potrebbe implicare che un elettrone, intrappolato in uno dei siti attivatori con tempo di decadimento molto lungo, rimanga nel sito per un tempo talmente prolungato che potrebbe decadere all'arrivo di un secondo fotone \(\gamma\). Ne discende che questo elettrone appartiene al meccanismo di scintillazione del fotone \(\gamma_{1}\), ma potrebbe decadere quando arriva il fotone \(\gamma_{2}\). Ciò rappresenta, ovviamente, un disturbo, in quanto, se gli elettroni intrappolati sono molti, si può generare una radiazione aggiuntiva non prevista. Inoltre, ciò causa un discostamento tra il comportamento ideale del cristallo scintillatore e quello effettivamente osservato nella pratica. In particolar modo, la conversione tra energia incidente e radiazione luminosa emessa è non lineare.

L'\emph{Afterglow}, quindi, non permette di riconoscere l'energia di un fotone. Ciò è un problema dal punto di vista della ricostruzione dell'immagine, perché non conoscendo l'energia del fotone, non è possibile sapere se questa è molto bassa e, quindi, se il fotone deriva da una notevole deviazione per effetto Compton oppure da un evento di annichilazione. I fotoni con energia inferiore ai 511keV dovrebbero essere, in linea teorica, scartati poiché questi derivano da iterazioni con la materia.

Purtroppo, il fenomeno non è completamente eliminabile, infatti, è sempre presente quando nel cristallo vi è del drogante. Questo fenomeno, in definitiva, porta il cristallo a emettere una radiazione luminosa secondo il meccanismo della fosforescenza.

Una sorgente di energia che determina l'\emph{Afterglow} è rappresentata dall'energia termica. Gli elettroni, se ricevono sufficiente energia termica, passano dalla banda di valenza a un sito di attivazione dove sono intrappolati e rilasciati, dopo un certo tempo variabile, durante, ad esempio, l'esame diagnostico PET. Le immagini risultanti sono, quindi, affette da un rumore di cui bisogna tener conto nel processo di ricostruzione.

\subsubsection{Quenching}\label{quenching}

\includegraphics[width=2in,height=2.23145in,alt={P4147\#y1}]{media/17_Detettori/image441.pdf}
Figura .: Variazione di energia

L'energia delle bande, infatti, varia nello spazio con andamenti anche molto complessi. Ad esempio, in un centro di attivazione, si potrebbe avere una struttura a bande con gap energetici variabili.

Questa struttura fa in modo che alcune transizioni tra i livelli energetici della banda di conduzione e della banda di valenza possano non essere radiative. Ad esempio, la transizione A--C è radiativa ad alta energia, perché il gap energetico è molto alto; allo stesso modo, la transizione B--D è radiativa a più bassa energia.

La transizione F--F', invece, potrebbe non essere radiativa, perché fuori dal range di energie che generano radiazioni visibili e, quindi, inefficaci.

Esistono, in letteratura, una serie di studi volti a determinare le caratteristiche che devono avere i materiali per limitare tale fenomeno.

In particolare, si cerca di strutturare e realizzare un materiale in modo che le transizioni energetiche siano quanto più radiative possibili e col minor \emph{Afterglow} realizzabile.

\subsubsection{Efficienza del processo di scintillazione}\label{efficienza-del-processo-di-scintillazione}

Nella maggior parte dei materiali scintillatori, per creare una coppia elettrone-lacuna occorre una energia in media pari a 3 volte il gap della banda proibita. Uno dei materiali storicamente più utilizzati è lo ioduro di sodio drogato al tallio (NaI(Tl)), che ha una grande efficienza. Per questo materiale occorrono circa 20eV per creare una coppia elettrone-lacuna.

Ad esempio, dato 1MeV di radiazione incidente, si assiste a circa \(5 \times 10^{4}\) coppie elettrone-lacuna, cioè una radiazione di uscita di \(12 \times 10^{4}eV\). Poiché ciascun fotone ha un'energia di 3eV, vi saranno \(4 \times 10^{4}\) coppie formate. L'efficienza, in questo caso, è del 12\%, che seppur complessivamente bassa, è comunque più alta rispetto a tutti gli altri materiali.

L'efficienza si definisce come:

\[Efficienza = \frac{numero\ di\ fotoni}{numero\ di\ coppie}\]

Data la sua elevata efficienza, questo materiale è preso come riferimento per calcolare l'efficienza degli altri materiali.

\subsubsection{Caratteristiche del cristallo di scintillazione}\label{caratteristiche-del-cristallo-di-scintillazione}

Per la presenza dei materiali droganti, lo spettro di emissione, legato al gap energetico, del cristallo di scintillazione non coincide con lo spettro di assorbimento. In questo modo, si ha la sicurezza che la radiazione emessa non sia assorbita dal cristallo stesso.

\begin{figure}
\centering
\includegraphics[width=6.58125in,height=4.68681in,alt={P4161\#yIS1}]{media/17_Detettori/image442.pdf}\caption{Figura .: Spettro di emissione e assorbimento del cristallo}
\end{figure}

Lo spettro di emissione del cristallo di scintillazione deve, ovviamente, essere coerente con lo spettro di assorbimento dell'elettronica a valle. Dopo il cristallo, sede della conversione tra il fotone \(\gamma\) a luce visibile, si trova il fotomoltiplicatore che converte la radiazione luminosa in energia elettrica.

Nell'accoppiamento tra il cristallo e il fotomoltiplicatore è fondamentale la conoscenza dello spettro di emissione del cristallo e di assorbimento dell'elettronica, così da accoppiarli al meglio. In altre parole, i due spettri devono essere tali da sovrapporsi il più possibile così che la maggior parte della radiazione emessa dal cristallo sia assorbita dallo stadio di fotomoltiplicazione. Il \emph{Matching} tra i due spettri garantisce, quindi, un'efficienza di conversione del fotone \(\gamma\) in impulso di corrente ottima.

Osservando lo spettro di emissione dei vari materiali è possibile ricavare informazioni legate alla radiazione emessa e il materiale con cui deve essere realizzato il \emph{Photo Multiplier Tube} o PMT. Non è detto che tutti i cristalli di scintillazione si accoppino con tutti i PMT, quindi, è necessario scegliere la coppia che permette la più alta sovrapposizione possibile tra i due spettri.

\begin{figure}
\centering
\includegraphics[width=6.69306in,height=3.00347in,alt={P4166\#yIS1}]{media/17_Detettori/image443.pdf}\caption{Figura .: Spettro di emissione di alcuni cristalli}
\end{figure}

Il NaI(Tl) emette nel blu con una lunghezza d'onda di 315-550nm a cui corrisponde un'energia dei fotoni emessi di 3eV. L'efficienza di questo materiale, sebbene sia 12\%, dipende soprattutto dall'energia incidente. Infatti, è possibile tracciare un grafico che permette di studiare come varia l'efficienza in funzione dell'energia.

\begin{figure}
\centering
\includegraphics[width=4.40219in,height=4.63619in,alt={P4169\#yIS1}]{media/17_Detettori/image444.pdf}\caption{Figura .: Grafico Efficienza-energia}
\end{figure}

Le caratteristiche di diversi materiali sono tabellate così da rendere più semplice il loro confronto:

\begin{figure}
\centering
\includegraphics[width=6.73364in,height=3.88339in,alt={P4172\#yIS1}]{media/17_Detettori/image445.pdf}\caption{Figura .: Caratteristiche dei materiali}
\end{figure}

Le efficienze riportate sono sempre valori medi, poiché si parla di processi statistici: il NaI(Tl) genera in media 38000 coppie; tuttavia, durante il suo normale funzionamento possono essere generate più o meno coppie in base a fluttuazioni quantistiche. La maggior problematica associata a questo materiale è il tempo di morto di 230ns. Un altro materiale molto importante è l'ortosilicato di lutezio (LSO) che presenta un tempo morto molto ridotto rispetto al NaI(Tl) di circa 47ns. Con questo materiale è possibile identificare anche gli impulsi molto ravvicinati tra loro. Dunque, il tasso di eventi reali può essere stimato con buona precisone a partire dal tasso di eventi misurato.

Tuttavia, rispetto allo ioduro di sodio attivato al tallio, presenta un'efficienza inferiore di 25000 di fotoni luminosi emessi per ogni MeV di energia incidente.

Affinché la radiazione luminosa non subisca deflessione importanti all'interfaccia tra cristallo scintillatore e vetro del fotomoltiplicatore, il cristallo di scintillazione deve possedere un coefficiente di riflessione quanto più simile al vetro. Generalmente, i cristalli presentano un coefficiente di riflessione variabile nel range di 1.5-1.8.

Una delle proprietà da tenere in considerazione nella scelta di un materiale piuttosto che un altro è la igroscopicità, ovvero la tendenza di un certo materiale a reagire con il vapore acquo nell'ambiente. Lo ioduro di sodio è igroscopico, quindi, il vapore acquo può degradarlo. Per evitare questo effetto, il materiale deve essere posizionato all'interno di un contenitore che lo ripara dal vapore acquo dell'aria. LSO, invece, non è igroscopico, quindi, non necessita di particolari accorgimenti per il suo mantenimento. Ovviamente questa proprietà incide sui costi della PET totale.

Non tutti i materiali possono essere lavorati per realizzare gli anelli di detettori con le dimensioni volute, mentre altri si trovano allo stato liquido o gassoso.

\begin{figure}
\centering
\includegraphics[width=6.40678in,height=3.36364in,alt={P4179\#yIS1}]{media/17_Detettori/image446.pdf}\caption{Figura .: Proprietà materiali scintillatori}
\end{figure}

Il germanato di bismuto e l'ortosilicato di lutezio attivato al cesio presentano un'alta densità, un elevato numero atomico, una robustezza meccanica e non sono idroscopi. La loro efficienza è buona con scarsa emissione secondaria. Di contro, l'ortosilicato di gadolinio (GSO) e il cadmio si sfaldano facilmente.

Il fluoruro di balio (BaF\textsubscript{2}) ha una costante di decadimento molto piccola di 0.6ns; tuttavia, ha emissioni secondarie a causa dell'\emph{Afterglow} che determina immagini molto rumorose. Inoltre, la lunghezza d'onda emessa è nello spettro dei raggi UV e, quindi, richiede dei PMT con quarzi costosi per l'accoppiamento.

Per quanto riguarda l'accoppiamento con il tubo fotomoltiplicatore, si fa riferimento al diagramma di emissione della radiazione luminosa del cristallo in funzione della lunghezza d'onda e, quindi, energia incidente.

\begin{figure}
\centering
\includegraphics[width=4.17126in,height=3.11573in,alt={P4184\#yIS1}]{media/17_Detettori/image447.pdf}\caption{Figura .: Accoppiamento PMT}
\end{figure}

Per il tempo di decadimento si osserva un andamento esponenziale in intervalli di tempo sufficientemente lunghi per il germanato di bismuto mentre per LSO è pressocché lineare:

\begin{figure}
\centering
\includegraphics[width=5.02134in,height=3.63884in,alt={P4187\#yIS1}]{media/17_Detettori/image448.pdf}\caption{Figura .: Tempo di decadimento}
\end{figure}

\subsection{Arrangiamento del blocco detettore}\label{arrangiamento-del-blocco-detettore}

\includegraphics[width=3.80208in,height=2.82292in,alt={P4190\#y1}]{media/17_Detettori/image449.pdf}
Il cristallo di scintillazione prevede delle scanalature, quindi, questo elemento deve essere lavorato in modo tale da essere tagliato per inserire le scalmanature dell'ordine 8x8 quadrati. Sono possibili anche blocchi con una divisione di 6x6. Complessivamente, il blocco detettore ha una grandezza di 4-5cm di lato per 3-4cm di spessore. Nella struttura standard, inoltre, sono previsti quattro fotomoltiplicatori dietro a un singolo cristallo di scintillazione.

Figura .: Blocco detettore

Nelle scanalature si inserisce un materiale riflettente, opaco alla radiazione, in modo tale che ciascun quadrato, che si viene a formare, assume il comportamento di una guida d'onda, poiché la radiazione può viaggiare solo in quello spessore, riflettendosi lungo le pareti. In questo modo la diffusione della luce tra un canale e l'altro è impedita.

\includegraphics[width=2.26042in,height=2.69792in,alt={P4193\#y1}]{media/17_Detettori/image450.pdf}
Figura .: Funzionamento blocco detettore monodimensionale

Questo processo è detto logica di Anger e consente di ricostruire su quale dei canali iniziali è arrivato il fotone \(\gamma\), quindi, è possibile conoscere la posizione di incidenza con un'incertezza dell'ordine di 4mm, invece di un'incertezza di 4cm, che si avrebbe se il cristallo scintillatore non fosse scanalato.

\includegraphics[width=2.52083in,height=2.40625in,alt={P4195\#y1}]{media/17_Detettori/image451.pdf}
Figura .: Blocco detettore visto dall'alto

\[X_{\gamma} = \frac{(B + D) - (A + C)}{A + B + C + D}\]

\[Y_{\gamma} = \frac{(A + B) - (C + D)}{A + B + C + D}\]

Le coordinate \({(X}_{\gamma},Y_{\gamma})\) sono variabili tra -1 e 1: sono 1 o -1 quando non si hanno fotoni luminosi su tutti i rispettivi detettori, ovvero verso gli estremi oppure una certa percentuale all'avvicinarsi del centro, in cui assumono valori intermedi. Le correnti, in definitiva, dipenderanno dal PMT che ha catturato la radiazione, quindi, dal fatto che quest'ultima è arrivata solo su A, solo su B o al 50\% tra A e B. La logica di Anger permette di ridurre l'incertezza di 4cm a 4mm, ottenendo immagini molto più precise.

A rigor di logica, maggiore è il numero di fotomoltiplicatori e migliore è la risoluzione spaziale dell'immagine; tuttavia, esistono dei limiti costruttivi legati alle minime dimensioni fisicamente realizzabili dei fotomoltiplicatori stessi. Nelle apparecchiature molto recenti e ancora poco diffuse non si utilizzano i PMT ma tecnologie a semiconduttore.

I blocchi detettori (cristallo di scintillazione e PMT) con dimensioni di 4-5cm, assemblati in 8 unità, formano un modulo; l'insieme di più moduli costituiscono l'anello di detezione. Più anelli di detezione sono uniti per realizzare il \emph{Gantry} della macchina PET.

\begin{figure}
\centering
\includegraphics[width=6.58133in,height=4.84167in,alt={P4201\#yIS1}]{media/17_Detettori/image452.pdf}\caption{Figura .: Anello di detezione}
\end{figure}

I blocchi di detettori hanno una dimensione di circa \(50x50x30{mm}^{3}\), il numero di scanalature tipicamente è di 8x8. Il PMT possiede, a sua volta, una certa geometria. In particolare, può avere sezione trasversale quadrata o rotonda. Per ridurre il senso di claustrofobia, tipicamente gli scanner hanno una dimensione minima di 56.2cm fino a massimo di 70cm.

Il numero di blocchi detettori che costituiscono gli anelli detettori va da 144 a 288. Questa quantità è legata alla precisione con cui si riesce a rilevare la posizione di incidenza di un fotone \(\gamma\), che a sua volta si traduce in qualità dell'immagine ricostruita. Maggiore è il numero di blocchi detettori e migliore è la risoluzione dell'immagine.

Le dimensioni del cristallo scintillatore sono riportante in termini di direzione transassiale, assiale e radiale. Infatti, dato che il cristallo è immesso in un anello detettore ha una dimensione radiale, ovvero nella dimensione che lo congiunge con il centro dell'anello di detezione, assiale, parallela all'asse longitudinale del paziente, e transassiale, ovvero tangente alla circonferenza dell'anello. Le tre dimensioni sono ovviamente diverse: la dimensione radiale si aggira intorno ai 20-30mm, quella assiale dai 4.05mm a 8mm mentre la transiassiale tra 4mm e 6.39mm.

Il materiale con cui è realizzato il cristallo ha uno certo \emph{Stopping Power} legato al coefficiente di attenuazione lineare. Chiaramente maggiore è lo spessore del cristallo, cioè maggiore è la dimensione radiale, maggiore è la probabilità che il fotone \(\gamma\) si intercettato. Uno spessore radiale limitato può aumentare la percentuale dei fotoni \(\gamma\) non rilevati perdendo così di efficienza. L'energia elettromagnetica incidente, infatti, può non essere assorbita sulla superficie esterna del cristallo scintillatore ma in profondità della sua struttura.

Ovviamente, maggiore è lo spessore del materiale e maggiore è il percorso che i fotoni luminosi devo compiere all'interno del cristallo, quindi, maggiore è la probabilità che essi siano assorbiti. È necessario scegliere un giusto spessore così da trovare un compromesso tra aumento dell'efficienza e radiazione emessa dal cristallo scintillatore. La perdita di alcuni fotoni luminosi può essere tollerata a patto che vi sia un aumento del numero di fotoni \(\gamma\) intercettati dal cristallo.

Il numero degli anelli di detezione è legato al numero di eventi rilevati. Infatti, i due fotoni emergenti dal corpo non sono diretti secondo una direzione prestabilita, ma ogni direzione dello spazio presenta la stessa probabilità di essere intrapresa. Una parte dei fotoni emergono anche parallelamente all'asse longitudinale del corpo e, a causa della struttura aperta dell'anello di detettori, non sono intercettati. I fotoni che viaggiano con un certo angolo rispetto alla normale sono rilevati se il numero degli anelli di rilevazione è adatto. Ciò si traduce in una migliore ricostruzione dell'immagine e aumento dell'efficienza totale del sistema. Tuttavia, il costo del macchinario PET aumenta poiché è maggiore il numero dei cristalli di scintillazione e PMT. Generalmente, il numero degli anelli di detezione è compreso tra i 18 e 32.

Lo spessore della fetta in PET è maggiore di quella in risonanza magnetica ed è in media uguale a 4mm contro l'1mm della risonanza magnetica. La grandezza dei voxel deriva proprio alle dimensioni dei cristalli scintillatori, delle scanalature e dei PMT.

La finestra di coincidenza temporale è un fattore molto importante per poter considerare due fotoni \(\gamma\) rilevati da detettori opposti come risultati di evento di annichilazione. Questa finestra temporale è generalmente di 8-12.5ns. Al di fuori di tale intervallo i due fotoni non sono considerati provenienti da uno stesso evento di annichilazione e sono, di conseguenza, scartati. Se il tempo morto è troppo elevato, anche la finestra temporale di detezione risulta essere abbastanza estesa, dunque, fotoni non appartenenti allo stesso fenomeno di annichilazione potrebbero essere interpretati come tali, portando a un errore nella ricostruzione delle immagini.

La sola contemporaneità non basta per identificare se due fotoni provengono dallo stesso evento di annichilazione. I moderni materiali presentano un una risoluzione energetica di 350-650keV. Se i fotoni rientrano in questo intervallo energetico sono accettabili per la ricostruzione delle immagini.

Altre caratteristiche tipiche dei blocchi di detezione sono riportati nelle seguenti tabelle:

\begin{figure}
\centering
\includegraphics[width=5.01326in,height=2.62679in,alt={P4213\#yIS1}]{media/17_Detettori/image453.pdf}\caption{Figura .: Caratteristiche blocchi detettore}
\end{figure}

\begin{figure}
\centering
\includegraphics[width=6.12814in,height=4.42157in,alt={P4215\#yIS1}]{media/17_Detettori/image454.pdf}\caption{Figura .: Caratteristiche scanner PET commerciali}
\end{figure}

Esistono detettori a scintillazione a fotodiodo, cioè dei semiconduttori sensibili ai raggi \(\gamma\), che sono usati particolarmente nelle PET/MRI, le quali, però, sono molto poco diffuse a causa dei costi attualmente molto elevati e la mancanza di un campo di applicazione principe come l'oncologia per la PET.

Queste macchine oltre a presentare gli anelli detettori tipici della PET e tutte le problematiche associate alla risonanza magnetica dovute alla generazione dei gradienti, il raffreddamento del superconduttore e così via, ci sono anche questioni legate all'interazione magnetica dei fotomoltiplicatori che ha portato all'introduzione di tecnologie a semiconduttore.

\begin{center}
\vfill
    \chapter{Principio fisico del tubo fotomoltiplicatore}
    \label{blx:PTM\therefsection}
\vfill

\minitoc
\newpage
\end{center}
\justify

\section{Tubo fotomoltiplicatore (PMT)}\label{tubo-fotomoltiplicatore-pmt}

Il fotomoltiplicatore o PMT (\emph{Photo-Multiplier Tube}) è situato tra il cristallo di scintillazione e l'elettronica di elaborazione. Questo componente intercetta la radiazione luminosa emergente dal cristallo e, tramite un vetro di accoppiamento, la trasforma in un impulso di corrente.

\begin{figure}
\centering
\includegraphics[width=6.37276in,height=2.70756in,alt={P4221\#yIS1}]{media/18_PTM/image455.pdf}\caption{Figura .: Posizione del PMT nel detettore}
\end{figure}

\begin{figure}
\centering
\includegraphics[width=6.22527in,height=2.79303in,alt={P4223\#yIS1}]{media/18_PTM/image456.pdf}\caption{Figura .: Schema interno di un PMT}
\end{figure}

La struttura, fondamentalmente, si compone di un fotocatodo su cui arriva la luce incidente proveniente dal cristallo. La luce è convertita in elettroni grazie all'effetto fotoelettrico. Gli elettroni sono accelerati all'interno del tubo, in cui è praticato il vuoto (vi è, infatti, una pressione di \(10^{- 4}Pa\)), attraverso degli elettrodi, detti dinodi, mantenuti ad un potenziale diverso rispetto al fotocatodo. L'elettrone emergente dal fotocatodo possiede energia molto limitata, e, per effetto del campo elettrico arriva, sul primo dinodo con una certa energia cinetica. L'elettrone, collidendo con il dinodo, cede l'energia cinetica ai suoi elettroni, scalzandoli. Ad un elettrone incidente coincidono vari elettroni scalzati, che vengono ugualmente accelerati verso il secondo dinodo e così via, innescando, dunque, un fenomeno a valanga. I dinodi, in totale, sono una decina, per cui, ad un elettrone incidente proveniente dal fotocatodo, coincide una corrente di alcuni nA, che può essere rilevata dall'elettronica a valle.

Più nel dettaglio della PET, il blocco detettore dei fotoni \(\gamma\) si compone di un cristallo di scintillazione, in cui i fotoni \(\gamma\) a 511keV sono convertiti in fasci luminosi. Dato che la luce visibile ha un'energia più bassa, dell'ordine di 2-3eV, in uscita al cristallo scintillatore vi sono qualche centinaio di migliaio di fotoni luminosi che, poi, sono convertiti dal fotocatodo in un fascio elettronico da amplificare mediante dinodi. Non esiste una corrispondenza esattamente biunivoca tra fotoni \(\gamma\) incidenti ed elettroni emessi, ma dalla corrente in uscita dal fotomoltiplicatore, dovuta all'effetto a valanga dei dinodi, è possibile risalire all'energia incidente sul cristallo scintillatore. La corrente in uscita dal fotomoltiplicatore deve essere opportunamente rilevata e amplificata per essere leggibile dalla circuiteria a valle.

\subsection{Fotocatodo}\label{fotocatodo}

Il fotomoltiplicatore sfrutta la struttura a bande caratteristiche dei solidi cristallini, usati per la realizzazione del cristallo scintillatore. Per un metallo le bande di conduzione e valenza coincidono, per semiconduttore le due bande sono separate da un gap energetico di qualche eV, per gli isolanti, invece, le due bande sono distanziate da un gap energetico molto elevato.

La struttura a bande si ritrova anche nel fotocatodo del fotomoltiplicatore, generalmente realizzato in un materiale alcalino. Questo elemento possiede una banda di valenza e conduzione separate da un gap energetico e un'interfaccia tra il materiale e il vuoto. Quando la luce di energia \(hf\) incide su degli elettroni della banda di valenza, questi acquisiscono energia per passare alla banda di conduzione. Gli elettroni con sufficiente energia riescono a percorrere lo spazio che li separa tra la posizione originale e lo spazio esterno del materiale. In altre parole, gli elettroni percorrono tutto lo spessore del materiale fino a fuoriuscire nel vuoto, dove sono accelerati dai dinodi. Durante il percorso compiuto all'interno del materiale, gli elettroni possono perdere energia emettendo altri fotoni. In questo caso, le particelle cariche potrebbero non avere energia sufficiente a superare la barriera di potenziale tra la banda di conduzione e il vuoto. L'energia somministrata necessaria che un elettrone deve possedere affinché passi dalla banda di conduzione al vuoto sia detto lavoro di estrazione o energia di estrazione. Questa quantità è legata all'effetto fotoelettrico spiegato da Einstein.

Ovviamente non tutti gli elettroni riescono a superare la barriera di potenziale poiché l'energia acquisita potrebbe non essere sufficiente o dissipata all'interno del materiale stesso. Quindi, il costruttore del fotocatodo deve tener conto di questi fenomeni così da poter realizzare una barriera di potenziale più piccola possibile per aumentare il numero di elettroni fuoriusciti dal materiale. Anche lo spessore del trasduttore è importante poiché i materiali devono essere molto sottili o film da depositare sulla finestra di ingresso. Con questa soluzione gli elettroni devono percorrere un cammino ridotto, rendendo, quindi, meno probabile le dissipazioni energetiche nel cristallo stesso.

\begin{figure}
\centering
\includegraphics[width=6.53904in,height=4.5625in,alt={P4231\#yIS1}]{media/18_PTM/image457.pdf}\caption{Figura .: Bande di un cristallo scintillatore}
\end{figure}

Un parametro importante di questi materiali è l'efficienza quantica o \emph{Quantum Efficiency} (QE) definito come il numero di fotoelettroni emessi rispetto al numero di fotoni incidenti. Idealmente, si vorrebbe che a ogni fotone corrisponda un elettrone estratto, ovvero un 100\% di QE. Tuttavia, nella realtà l'efficienza è dell'ordine del 10-20\%, quindi, per estrarre un elettrone è necessario irradiare il fotocatodo con un certo numero di fotoni. Questo fenomeno si verifica poiché non è detto che l'energia di un elettrone sia tale da superare la barriera di potenziale o che sia conservata nel tragitto tra la posizione originaria e il vuoto.

La fotoemissione si compone di tre passi:

\begin{itemize}
\item
  L'assorbimento di un fotone in cui si assiste al trasferimento di energia da un fotone incidente e un elettrone;
\item
  Il moto dell'elettrone verso l'interfaccia metallo-vuoto;
\item
  La fuga dell'elettrone dalla barriera di potenziale all'interfaccia possibile solo se l'elettrone ha energia sufficiente a superare la barriera di potenziale.
\end{itemize}

Le perdite di energia possono verificarsi per molti fenomeni anche non strettamente legati al cammino dell'elettrone nel fotocatodo. Possono verificarsi, ad esempio, dei fenomeni riflessivi tra il cristallo scintillatore e il fotocatodo che determinano un parziale assorbimento della radiazione incidente. La porzione di energia incidente estrae elettroni che possono collidere, perdendo così energia, sia con altri elettroni o con il reticolo durante nel cammino verso l'interfaccia fotocatodo-vuoto. Quando l'elettrone collide col reticolo si parla di \emph{Phonon-Scattering} mentre se incide con un altro elettrone si parla di \emph{Electron-Scattering}. Questi due fenomeni aumentano il numero di particelle cariche che non hanno energia sufficiente a superare la barriera di potenziale.

\subsubsection{Metalli per fotocatodo}\label{metalli-per-fotocatodo}

La realizzazione dei fotocatodi pone una serie di problematiche costruttive che riducono il rendimento della conversione fotoni-elettroni. I metalli presentano una superficie molto riflettente, quindi, la quantità di radiazione assorbita è molto limitata. Possono verificarsi anche un gran numero di collisioni tra elettroni liberi e reticolo o altri elettroni. Per tale motivo la profondità di fuga, ovvero il cammino compiuto per giungere all'interfaccia metallo-vuoto, deve essere di 1nm.

L'energia necessaria affinché un elettrone di un metallo superi la barriera di potenziale deve essere maggiore di 3eV; tuttavia, la luce visibile possiede un'energia minore di tale soglia. Quindi, per produrre fotoelettroni, bisogna spostare lo spettro di emissione verso i raggi UV. Per superare tale limite si utilizzano i metalli alcalini per i quali il lavoro di estrazione è minore di 3eV.

Si può vedere che, in genere, l'efficienza dei metalli nella regione visibile è dell'ordine di 1 fotoelettrone per 1000 fotoni incidenti:

\[QE = \dfrac{1}{1000} = 0.1\%\]

Questa quantità risulta comunque essere molto variabile in base al metallo considerato e le condizioni ambientali.

\subsubsection{Semiconduttori}\label{semiconduttori}

I semiconduttori, a 0K, possiedono elettroni più energetici nella banda di valenza, mentre la banda di conduzione è vuota. Le due bande sono separate da un'energia detta gap di circa 2-3eV. A temperature maggiori dello zero assoluto, alcuni elettroni sufficientemente energetici si spostano nella banda di conduzione.

Gli elettroni nella banda di valenza di questi materiali assorbono fotoni solamente se possiedono un'energia maggiore del gap energetico. In particolare, per la fotoemissione, l'elettrone deve assorbire un'energia maggiore dell'affinità elettronica che tiene conto dell'energia di legame con la quale è legata al reticolo. Ne discende che il fotone non solo deve avere un'energia almeno uguale al gap energetico ma deve tener conto anche dell'affinità elettronica. Il fotone deve avere un'energia almeno uguale alla somma delle due energie:

\[E_{f_{tot}} = E_{G} + E_{A}\]

A favore di questi materiali vi sono alcune proprietà tipiche come il basso numero di collisioni degli elettroni liberi nella banda di conduzione. Ciò comporta che gli elettroni possono percorrere anche una profondità di fuga di 10nm.

Con gli attuali processi tecnologici è possibile realizzare semiconduttori che possiedono un'energia di estrazione, somma del gap e dell'affinità elettronica, anche minore di 2eV.

\subsubsection{Affinità elettronica negativa}\label{affinituxe0-elettronica-negativa}

Normalmente nei materiali semiconduttori le bande di conduzione e valenza sono separate da un gap energetico. Tipicamente il vuoto si trova a un potenziale maggiore della banda di conduzione. Si parla, dunque, di affinità elettronica positiva. Tuttavia, con le attuali metodiche, è possibile produrre dei materiali in cui la banda di conduzione possiede un'energia superiore al vuoto. In questo caso si parla di affinità elettronica negativa e ogni elettrone, per poter fuggire, deve avere un'energia lievemente maggiore a quella di conduzione.

In altre parole, il fotone deve possedere almeno un'energia uguale al gap tra banda di valenza e conduzione per poter estrarre il fotoelettrone.

I materiali con affinità elettronica negativa sono fondamentali per realizzare fotocatodi con elevata efficienza quantica. Un esempio di materiale con affinità negativa è offerto dal fosfato di gallio attivato al cesio o GaP(Cs) che a sua volta possiede delle impurità di atomi zinco (Zn) che realizzano l'accettore, mentre il cesio è posto sulla superficie.

Lo zinco attrae gli elettroni dal cesio che, quindi, è ionizzato positivamente. Gli elettroni in fondo alla banda di conduzione, con energie inferiori, possono fuggire con una profondità di fuga di 100nm.

\subsubsection{Efficienza quantica dei vari materiali}\label{efficienza-quantica-dei-vari-materiali}

Per quantizzare l'efficienza quantica dei vari materiali si osservi che essa dipende all'energia dei fotoni incidenti. Studi in letteratura hanno permesso di ottenere una curva efficienza quantica-energia incidente per vari materiali.

Da questi digrammi è possibile osservare che per metalli di spessore uguale a 400nm, l'efficienza quantica varia nel range di 25-30\% per ogni keV di energia incidente. Per misurare l'efficienza quantica si irradia un fotocatodo di ioduro di sodio drogato al tallio o NaI(Tl), considerato come riferimento da confrontare con altri materiali.

Se l'efficienza è intorno al 30\% gli elettroni emessi per ogni keV sono circa 8-10. Per tale motivo si rende necessaria la fase di moltiplicazione elettronica operata dai dinodi.

Si noti che se la profondità di penetrazione e la probabilità di fuga sono piccole, è maggiore la probabilità di estrarre un elettrone. In questo caso, si potrebbe pensare di aumentare l'efficienza ottimizzando la probabilità di fuga mediante un fotocatodo molto sottile. Tuttavia, ciò provocherebbe un aumento della trasmissione della radiazione.

L'utilizzo dei semiconduttori drogati in modo da avere un'affinità elettrica negativa come il GaP(Cs) permette un aumento dell'efficienza, grazie all'elevata profondità di fuga.

\begin{figure}
\centering
\includegraphics[width=4.85057in,height=4.73958in,alt={P4262\#yIS1}]{media/18_PTM/image458.pdf}\caption{Figura .: Efficienza quantica in funzione dell'energia dei fotoni per vari materiali}
\end{figure}

\subsubsection{Caratteristiche di un fotocatodo}\label{caratteristiche-di-un-fotocatodo}

Per scegliere adeguatamente un fotocatodo di un fotomoltiplicatore, uno dei parametri da considerare è l'efficienza quantica che può andare dal 12\% fino a un massimo di 30\% nei materiali con prestazioni migliori. Non è possibile convertire tutti i fotoni ricevuti in fotoelettroni ma, in media, si assiste a un numero di fotoelettroni emessi legato al QE. Nel processo di trasduzione una quota di energia è sempre persa per vari fenomeni come le dissipazioni in calore.

È importante anche un elevato grado di accoppiamento con il cristallo scintillare, così che i due materiali abbiano un indice di rifrazione quando più simile possibile. Ciò consente di ridurre al minimo la radiazione riflessa. Ovviamente, è necessario che il cristallo scintillatore emetta fotoni luminosi con lunghezza d'onda tale da eccitare il fotocatodo.

È, inoltre, importante considerare il tempo morto di un materiale così da poter ricostruire l'informazione ricevuta sull'energia nel modo più affidabile possibile.

I costruttori dei materiali inseriti in un fotomoltiplicatore forniscono delle tabelle contenenti tutte le caratteristiche e proprietà del materiale utilizzato. Ciò consente di selezionare un materiale piuttosto che un altro in base alle specifiche del progetto e le applicazioni future.

\subsection{Dark Current}\label{dark-current}

La \emph{Dark Current} è un parametro che riguarda il tubo fotomoltiplicatore nel suo complesso ed è definita come la corrente che fluisce nell'anodo dello strumento quando non vi è nessuna radiazione incidente. In linea teorica, ci si aspetterebbe che, quando non vi è nessuna radiazione emessa dal cristallo scintillatore, non sia prodotto nessun fotoelettrone e, di conseguenza, la corrente in uscita al fotomoltiplicatore sia nulla. Tuttavia, esistono dei fenomeni per cui si può generare una corrente anche in assenza di energia luminosa in entrata.

La corrente oscura limita la soglia dei fotoni poiché si sovrappone alla corrente prodotta dalla trasduzione e amplificazione elettronica, introducendo così una certa quota di rumore additivo. Ciò rende complicata la valutazione dell'ampiezza e durata dell'impulso di corrente.

La \emph{Dark Current} può essere prodotta da:

\begin{itemize}
\item
  Dispersione ohmica dovuta all'imperfetto isolamento all'interno del tubo e sul contenitore del tubo stesso. Si producono così delle correnti parassite dal contenitore verso l'anodo che corrompono la misura;
\item
  L'emissione termoionica, dovuto al riscaldamento del catodo. Quando il catodo o il tubo si riscaldano un numero maggiore di elettroni possiede l'energia necessaria a superare il gap, passando dalla banda di valenza a quella di conduzione. Ciò incrementa il numero di elettroni, che diventa maggiore di quello atteso, e anche l'emissione in assenza di radiazione luminosa incidente. Anche i dinodi possono emettere degli elettroni per emissione termoionica. In questo caso si corrompe l'emissione secondarie;
\item
  Gli effetti rigenerativi sono di minor importanza nei fotomoltiplicatori.
\end{itemize}

Nei metalli l'emissione termoionica è data da tutti gli elettroni che oltrepassano la barriera di potenziale tra la banda di conduzione e il vuoto. La densità di corrente prodotta segue la l'equazione di Richardsson:

\[J = 4\pi emk_{B}^{2}\dfrac{T}{h^{3}}e^{- \dfrac{\varphi}{k_{B}T}}\ \]

Dove \(e\) è la carica elementare dell'elettrone, \(m\) la sua massa, \(k_{B}\) la costante di Boltzmann, \(h\) la cosante di Planck, \(T\) la temperatura assoluta e \(\varphi\) la \emph{Work-Function}.

A seconda della temperatura, esiste una densità di corrente termoionica formata da elettroni che sfuggono dal materiale. La corrente è generata indipendentemente dal fascio luminoso in ingresso e forma una porzione della \emph{Dark Current}.

La corrente di disturbo in funzione della temperatura è fornita dal costruttore del fotomoltiplicatore come diagrammi della densità di corrente in funzione della temperatura, parametrizzati la \emph{Dark Emission}, valutata come numero di elettroni emessi al secondo su unità di superficie. La \emph{Dark Current} può assumere valori anche dell'ordine dei pA che, amplificate insieme al segnale utile, costituiscono un rumore di sottofondo su correnti già con intensità molto limitate.

\begin{figure}
\centering
\includegraphics[width=2.79331in,height=4.60417in,alt={P4281\#yIS1}]{media/18_PTM/image459.pdf}\caption{Figura .: Dark Current in funzione della temperatura per vari materiali}
\end{figure}

I semiconduttori sono anche loro soggetti a correnti termoioniche, date dagli elettroni che passano dalla banda di valenza a quella di conduzione. Per questi materiali, l'equazione di Richardsson va modificata sostituendo al lavoro di estrazione la somma del gap energetico e dell'affinità elettronica:

\[J = 4\pi emk_{B}^{2}\dfrac{T}{h^{3}}e^{- \dfrac{E_{A} + E_{G}}{k_{B}T}}\]

La quantità \(E_{A} + E_{G}\) è proprio l'energia necessaria che bisogna fornire all'elettrone nella banda di valenza per essere estratto. È importante sottolineare questo punto poiché per i metalli le due bande coincidono, quindi, un elettrone di valenza è anche di conduzione. Invece, per un semiconduttore le due bande sono separate da un gap energetico che l'elettrone deve superare prima di poter essere estratto.

\subsection{Emissione secondaria}\label{emissione-secondaria}

Il fenomeno dell'emissione secondaria è dovuto all'impatto di elettroni sufficientemente energetici, estratti per effetto foto-ionico o per correnti dispersione, sulla superficie dinodi, che a loro volta producono altri elettroni in cascata. Questo processo può essere quantizzato da un'efficienza, indicata con \(\delta\) e detta \emph{Secondary Emission Ratio}, dato dal rapporto di numero di elettroni secondari emessi e il numero di elettroni primari incidenti:

\[\delta = \dfrac{N_{s}}{N_{e}}\]

Ovviamente si vorrebbe che questa quantità sia più costante possibile e di valore sufficientemente elevato, così da sapere che la quota di rumore è molto limitata e non varia nel tempo. Nella pratica, questa quantità non è costante ma dipende da numerosi fattori statistici. La corrente raccolta dopo il processo di amplificazione elettronica è a sua volta dipendente dall'efficienza: se per ogni fotoelettrone incidente si generano \(\delta\) elettroni secondari, a fine processo si generano \(\delta^{n}\) elettroni secondari, dove \(n\) è il numero dei dinodi.

Se \(\delta\) è costante nel tempo e sufficientemente elevato, il numero degli elettroni secondari cresce rapidamente con il numero dei dinodi così da avere una corrente in uscita di ampiezza sufficientemente intensa da poter essere rilevata.

Ogni volta che un elettrone primario incide sul dinodo, interagisce con un elettrone nel materiale, eccitandolo. L'elettrone eccitato passa nello stato energetico superiore e, se possiede un'energia sufficientemente elevata, supera la barriera di potenziale all'interfaccia tra materiale e vuoto. Tutti gli elettroni che superano la barriera di potenziale formano gli elettroni secondari che sono poi emessi all'esterno del dinodo ed accelerati dal campo elettrico verso il dinodo successivo.

A seconda dei materiali, il \emph{Secondary Emission Ratio} varia da 10 a 20, mentre la probabilità di fuga diminuisce all'aumentare dello spessore del materiale stesso.

\begin{figure}
\centering
\includegraphics[width=5.81858in,height=3.88542in,alt={P4293\#yIS1}]{media/18_PTM/image460.pdf}\caption{Figura .: Probabilità di fuga in funzione della profondità di fuga}
\end{figure}

Il \emph{Secondary Emission Ratio}, conseguentemente anche la probabilità di fuga, dipende anche dall'energia dell'elettrone primario incidente. In quest'ottica, è importante selezionare la giusta differenza di potenziale tra due dinodi così che gli elettroni acquisiscano la giusta energia in eV.

\begin{figure}
\centering
\includegraphics[width=3.64583in,height=2.63037in,alt={P4296\#yIS1}]{media/18_PTM/image461.pdf}\caption{Figura .: Andamento dell'emissione secondaria}
\end{figure}

La forma dei dinodi, della struttura acceleratrice e dell'intero fotomoltiplicatore sono parametri importanti da considerare in fase di progetto. Esistono, infatti, differenti modi in cui i dinido possono essere arrangiati nello spazio.

Tuttavia, la configurazione geometrica deve essere tale che tra un dinodo e l'altro la dispersione elettronica sia minima, ovvero che la maggior parte degli elettroni emessi da un dinodo siano accettati dal dinodo successivo.

Le diverse traiettorie compiute dagli elettroni, le diverse forme e posizioni spaziali dei dinodi determinano le classi dei fotomoltiplicatori.

La struttura interna del tubo fotomoltiplicatore può essere arrangiata secondo diverse geometrie come la circolare, in cui i dinodi sono delle semicirconferenze e la differenza di potenziale è tale accelerare gli elettroni e condurli sul dinodo successivo con la minima dispersione possibile. Nella maggior parte dei casi pratici, si preferisce utilizzare delle strutture a tubo cilindrico con dinodi sempre di forma semicircolare.

\begin{figure}
\centering
\includegraphics[width=4.71875in,height=3.24896in,alt={P4302\#yIS1}]{media/18_PTM/image462.pdf}\caption{Figura .: Struttura circolare di un PMT}
\end{figure}

\subsection{Ottica elettronica}\label{ottica-elettronica}

La geometria del fotocatodo ha un impatto fondamentale sulla tempistica poiché i fotoni estraggono elettroni lungo tutto la sua superficie. Gli elettroni estratti seguiranno percorsi diversi per arrivare al primo dinodo in base alla posizione in cui sono stati estratti. In altre parole, se il fotoelettrone è emesso in corrispondenza dei bordi, per raggiungere il centro deve percorrere un tragitto più lungo rispetto a un altro fotoelettrone emesso direttamente al centro. La posizione di estrazione dell'elettrone primario influenza, in definitiva, il tempo con cui arriva sul primo dinodo.

\begin{figure}
\centering
\includegraphics[width=5.5625in,height=2.4083in,alt={P4306\#yIS1}]{media/18_PTM/image463.pdf}\caption{Figura .: Diverso cammino compiuto dagli elettroni}
\end{figure}

Una possibile soluzione a questo inconveniente consiste nell'utilizzare uno schermo di ingresso ricurvo verso l'esterno così che i fotoelettroni percorrono sempre lo stesso cammino, uguale al raggio della circonferenza del fotocatodo.

\begin{figure}
\centering
\includegraphics[width=4.76042in,height=3.94392in,alt={P4309\#yIS1}]{media/18_PTM/image464.pdf}\caption{Figura .: Cristallo scintillatore curvo}
\end{figure}

La tempistica del cammino è dell'ordine di 0.1ns e se varia in base alla pozione, varia anche il tempo di percorrenza dal fotocatodo all'anodo. Il tempo complessivo medio tra l'arrivo di un fotone al fotocatodo e l'acquisizione dell'impulso all'anodo è detto \emph{Electron Transit Time}. Questo tempo è dell'ordine di 20-80ns e dipende molto dal tempo di percorrenza del fotoelettrone emesso dal fotocatodo sul primo dinodo, poiché risulta essere quello più variabile. Il tempo di percorrenza influenza la durata temporale dell'impulso e la sua ampiezza poiché introduce un ritardo fisso e, dunque, non un rumore casuale dannoso. In realtà, si assiste a una distribuzione dei tempi di arrivo all'anodo che causa uno \emph{Spread} intorno al valor medio, poiché esistono comunque delle fluttuazioni del tempo di transito.

Si suppone che gli elettroni arrivino in determinati istanti di tempo. All'inizio vi è una certa quantità di elettroni che arriva sull'anodo quasi contemporaneamente e altri con un ritardo maggiore poiché possiedono tempi di transito più lunghi. Si vorrebbe, in generale, che tutti gli elettroni arrivassero nello stesso istante poiché si genererebbe un impulso di corrente con durata e ampiezza dipendente dalla quantità di carica depositata sull'anodo. Siccome la corrente è la derivata della carica, quando arrivano i primi impulsi molto ravvicinati tra loro, la corrente aumenta rapidamente per poi decrescere. Il tempo di salita della corrente è un parametro fondamentale per la scelta del fotomoltiplicatore ed è riportato nei dati tecnici. Sia i tempi di salita che di discesa della corrente sono legati all'\emph{Electron Transit Time}.

A causa degli elettroni con tempo di transito più lunghi, la forma d'onda, oltre al picco principale, presenta una serie di picchi di intensità minore in corrispondenza dell'arrivo dell'elettrone sull'anodo.

L'impulso di corrente così misurato non comprenderebbe tutte le cariche giunte sull'anodo. Per risolvere la problematica dei picchi spuri si preferisce prelevare una tensione ottenuta integrando la corrente, cosa che avviene mediante un condensatore.

\begin{figure}
\centering
\includegraphics[width=5.4375in,height=3.4834in,alt={P4315\#yIS1}]{media/18_PTM/image465.pdf}\caption{Figura .: Corrente discreta, corrente e tensione}
\end{figure}

La misura così ottenuta è meno soggetta alla presenza dell'\emph{Electron Transit Time}. Si osserva, infatti, che la tensione, dopo un certo periodo di tempo, raggiunge un valore di regime proporzionale alla quantità di carica complessivamente accumulata sull'anodo. Quest'ultima quantità è proporzionale alla radiazione luminosa in entrata al fotomoltiplicatore che a sua volta dipende dall'energia del fotone \(\gamma\) incidente. In definitiva, la tensione misurata è legata all'energia del fotone incidente. Da questa conoscenza è possibile discriminare se il fotone rilevato sia proveniente da un evento di annichilazione o da un effetto di \emph{Scattering}.

Per limitare il più possibile il fenomeno del diverso percorso di transito, si progetta il PMT in modo che la distanza tra fotocatodo e il primo anodo sia grande rispetto le distanze inter-dinodo successive. Ciò garantisce maggiore uniformità nei cammini seguiti dai vari elettroni. Inoltre, la curvatura del fotocatodo minimizza la dispersione dei tempi di percorrenza dei vari elettroni.

Si vede che le velocità iniziali degli elettroni si distribuisce secondo una statistica ben determinata. Questo effetto può essere compensato aumentando la tensione di accelerazione tra i vari dinodi.

La distribuzione dei tempi dipende dal numero iniziale di fotoelettroni per impulso ed è legata inversamente con il quadrato del numero dei fotoelettroni.

Per valutare la tempistica del processo è fondamentale avere informazioni sull'\emph{High Light Output} dello scintillatore.

\subsection{Alimentazione del PMT}\label{alimentazione-del-pmt}

L'alimentazione del tubo fotomoltiplicatore è fondamentale poiché i dinodi devono essere posti al giusto potenziale. Subito dopo il fotocatodo vi è anche una griglia di accelerazione che convoglia tutti gli elettroni prodotti sul primo dinodo.

I potenziali di accelerazione sono creati ponendo i dinodi come nodi di un partitore resistivo con resistenze tutte uguali. La tensione che complessivamente cade su ogni resistenza è, quindi, la stessa. Tra due dinodi si assiste a una differenza di potenziale sempre uguale e, di solito, posta a 100V. In presenza di 10 dinodi l'alimentazione complessiva deve essere di 1kV in modo da ripartire 100V su ogni resistenza del partitore collegate ai dinodi. Con questa scelta, l'energia che acquisiscono gli elettroni nel inter-distanza tra due dinodi è di 100eV.

\begin{figure}
\centering
\includegraphics[width=4.89583in,height=1.56575in,alt={P4326\#yIS1}]{media/18_PTM/image466.pdf}\caption{Figura .: Schema di alimentazione dei Dinodi}
\end{figure}

Oltre alle problematiche di sicurezza legate alla gestione di tensioni così elevate, esistono dei limiti tecnici legati alla diversa corrente che scorre nei vari stadi. Nei primi stadi, la corrente ha intensità molto limitata poiché è basso il numero degli elettroni coinvolti. All'aumentare degli stadi, si ha un'amplificazione elettronica di tipo esponenziale. L'elevato numero di elettroni produce una corrente sufficientemente intensa che si manifesta tra un dinodo e il successivo.

Se esiste una corrente nel dinodo, allora la corrente circolante nel ramo che lo congiunge con il partitore è non nulla. Per la legge di Kirchhoff ai nodi deve necessariamente accadere che la corrente che scorre nel partitore sia ridotta. Si perde così l'ipotesi di partitore resistivo, in favore di una configurazione più complessa che prevede la verifica delle ipotesi di corrente trascurabile solo tra i dinodi del primo tratto del fotomoltiplicatore.

Nell'ultimo tratto la corrente tra due dinodi è tale da ridurre la differenza di potenziale tra i due elementi. Gli elettroni non acquisiscono più un'energia di 100eV ma inferiore, riducendo, di conseguenza, il numero di elettroni secondari estratti.

Per bilanciare le correnti inverse tra gli ultimi dinodi, sono inseriti dei condensatori in parallelo alle resistenze con lo scopo di mantenere costante la tensione anche in presenza di rapide variazioni dovute all'arrivo degli elettroni.

Le capacità di stabilizzazione forniscono ai dinodi la carica persa durante l'impulso ed è poi ricaricato dal partitore nel periodo tra due impulsi, in modo che il successivo fascio elettronico veda una differenza di potenziale di 100V. La carica della capacità deve essere circa 100 volte la carica emessa da dinodo, così da avere una variazione dell'1\% della tensione sul dinodo.

La corrente circolante nel tubo fotomoltiplicatore dovrebbe essere mantenuta piccola così da limitare le dissipazioni di calore e i costi dell'intera struttura. Si suppone, ad esempio di avere 1000 fotoelettroni prodotti dal fotocatodo, un guadagno sia di 10\textsuperscript{6} e un numero di impulsi al secondo di 10\textsuperscript{6}. Il valor medio della corrente all'anodo è:

\[10^{3} \cdot 10^{6} \cdot 10^{5} \cdot 1.6 \cdot 10^{- 19} = 16nA\]

Si usa stabilizzare la tensione di alimentazione dell'intero sistema con dei diodi Zener o transistor. È, infatti, importante mantenere basso il valore del \emph{Ripple} per evitare variazioni del guadagno col tempo. Inoltre, conviene mantenere il fotocatodo al potenziale di riferimento dato che è in contatto col cristallo mentre l'anodo è posto al potenziale positivo.

L'anodo è, quindi, sottoposto a una tensione elevata e, per passare l'impulso di corrente agli stadi successivi, bisogna posizionare un condensatore di accoppiamento per limitare la tensione, ponendo la resistenza di carico a terra. La struttura si comporta come un filtro passa-alto che blocca la tensione di alimentazione in DC e lascia passare solamente l'impulso di corrente, della durata di qualche ns, al carico.

Un'altra soluzione per l'alimentazione consiste nel porre l'anodo a massa e il fotocatodo al potenziale negativo.

\begin{figure}
\centering
\includegraphics[width=6.25in,height=1.98108in,alt={P4338\#yIS1}]{media/18_PTM/image467.pdf}\caption{Figura .: Schema di alimentazione alternativo}
\end{figure}

Questa soluzione porta l'anodo a una tensione nulla senza avere la necessità della capacità di accoppiamento, ma il fotocatodo è in contatto diretto col cristallo scintillatore, che, dunque, potrebbe portarsi in tensione. In questo caso, è necessario isolare opportunamente i due componenti per evitare problemi di sicurezza ed elevate \emph{Dark Current}.

\subsection{Forma dell'impulso}\label{forma-dellimpulso}

L'amplificatore a valle dell'anodo possiede un'impedenza in ingresso che può essere schematizzata come un cappio RC.

Il circuito di ingresso presenta a sua volta una costante di tempo \(\tau\) che deve essere opportunamente dimensionata per essere accoppiata con le tempistiche di ritardo all'interno del fotomoltiplicatore.

La forma dell'impulso di corrente in uscita dal PMT dipende dalla costante di tempo del circuito anodico secondo una forma d'onda che ricalca quella di emissione del cristallo.

Nello specifico, la corrente è caratterizzata da una costante primaria di decadimento del cristallo \(\lambda\), secondo la relazione:

\[i(t) = i_{0}e^{- \lambda t}\]

Questa relazione è sempre più verificata quanto più lo \emph{Spread} del tempo di emissione è piccolo rispetto alla costante di decadimento.

Sia \(Q\) la carica raccolta nell'intero impulso, risulta che:

\[Q = \int_{0}^{\infty}{i(t)dt} = i_{0}\int_{0}^{\infty}{e^{- \lambda t}dt} = \dfrac{i_{0}}{\lambda} \Leftrightarrow i_{0} = Q\lambda\]

La corrente nel dispositivo di misura può essere scritta come:

\[i(t) = Q\lambda e^{- \lambda t}\]

Per risolvere la rete e determinare la tensione in uscita, si passa nel dominio dei fasori. La corrente si ripartisce nelle due impedenze che rappresentano, quindi, un partitore di corrente:

\includegraphics[width=3.97639in,height=3.77639in,alt={P4353\#y1}]{media/18_PTM/image468.pdf}
\[{\dot{I}}_{C} = \dfrac{{\dot{Y}}_{C}}{G + {\dot{Y}}_{C}}\dot{I} = \dfrac{j\omega C}{\dfrac{1}{R} + j\omega C}\dot{I}\]

\[\dfrac{j\omega C}{\dfrac{1}{R} + j\omega C} = \dfrac{j\omega CR}{1 + j\omega CR}\]

La tensione indotta sulla capacità è data da:

\[{\dot{V}}_{C} = {\dot{Z}}_{C}{\dot{I}}_{C}\]

Da cui:

\[{\dot{V}}_{C} = \dfrac{1}{j\omega C}\dfrac{j\omega CR}{1 + j\omega CR}\dot{I} = \dfrac{R}{1 + j\omega CR}\dot{I}\]

\[R\dot{I} = {\dot{V}}_{C} + j\omega CR{\dot{V}}_{C}\]

Passando nel dominio del tempo si ottiene la relazione ingresso-uscita della rete elettrica:

Figura .: Imprendenza di ingresso degli stadi di prelievo

\[Ri(t) = v_{C} + RC\dfrac{dv_{C}}{dt}\]

Particolarizzando al caso in esame, è noto il forzamento di tipo esponenziale:

\[RQ\lambda e^{- \lambda t} = v_{C} + RC\dfrac{dv_{C}}{dt}\]

Si pone \(\theta = \dfrac{1}{RC}\) reciproco della costante di tempo del circuito RC di prelievo del segnale, l'equazione della rete può essere scritta come:

\[Q\lambda e^{- \lambda t} = \dfrac{1}{R}v_{C} + \dfrac{1}{\theta R}\dfrac{dv_{C}}{dt}\]

Si passa all'omogenea associata:

\[\dfrac{1}{R}v_{C} + \dfrac{1}{\theta R}\dfrac{dv_{C}}{dt} = 0\]

La cui soluzione è semplice:

\[{v_{C}}_{0} = ke^{- \theta t}\]

Una soluzione particolare dell'equazione differenziale è un esponenziale con stesso tempo di decadimento \(\lambda\):

\[u(t) = Fe^{- \lambda t}\]

Sostituendo nell'equazione differenziale si ottiene:

\[Q\lambda e^{- \lambda t} = \dfrac{1}{R}Fe^{- \lambda t} - \dfrac{\lambda}{\theta R}Fe^{- \lambda t}\]

\[Q\lambda = \dfrac{F}{R}\left( 1 - \dfrac{\lambda}{\theta} \right) \Leftrightarrow F = \dfrac{RQ\lambda}{\left( 1 - \dfrac{\lambda}{\theta} \right)} = \dfrac{RQ\lambda}{(\theta - \lambda)}\theta = \dfrac{RQ\lambda}{(\theta - \lambda)}\dfrac{1}{RC} = \dfrac{1}{(\theta - \lambda)}\dfrac{Q\lambda}{C}\]

La soluzione particolare è, dunque:

\[u(t) = \dfrac{1}{(\theta - \lambda)}\dfrac{Q\lambda}{C}\ e^{- \lambda t} = - \dfrac{1}{(\lambda - \theta)}\dfrac{Q\lambda}{C}e^{- \lambda t}\ \]

L'integrale generale dell'equazione differenziale è, quindi, del tipo:

\[v_{C}(t) = ke^{- \theta t} - \dfrac{1}{(\lambda - \theta)}\dfrac{Q\lambda}{C}e^{- \lambda t}\]

Se si pone la condizione iniziale che \(v_{C}(0) = 0\) dovuta all'assenza di corrente nella resistenza all'avvio del circuito, si ottiene:

\[k = \dfrac{1}{(\lambda - \theta)}\dfrac{Q\lambda}{C}\]

La soluzione della rete elettrica è stata determinata

\[v_{C}(t) = \dfrac{1}{(\lambda - \theta)}\dfrac{Q\lambda}{C}\left( e^{- \theta t} - e^{- \lambda t} \right)\]

A seconda della costante di tempo RC l'equazione ottenuta per la tensione sul cappio RC può essere approssimata in diversi modi.

Se il circuito di ingresso dell'elettronica è un \emph{Large RC}, rispetta la condizione:

\[\dfrac{1}{\lambda} \ll \dfrac{1}{\theta}\]

La tensione sulla capacità può essere espressa come:

\[v_{C}(t) \simeq \dfrac{Q}{C}\left( e^{- \theta t} - e^{- \lambda t} \right)\]

Nella condizione iniziale, ovvero per:

\[t \ll \dfrac{1}{\theta}\]

È possibile approssimare \(e^{- \theta t}\) con l'unità:

\[v_{C}(t) \simeq \dfrac{Q}{C}\left( 1 - e^{- \lambda t} \right)\]

Nella condizione opposta, invece, in cui:

\[t \gg \dfrac{1}{\lambda}\]

Risulta che \(e^{- \theta t} \gg e^{- \lambda t}\), la tensione può essere scritta come:

\[v_{C}(t) \simeq \dfrac{Q}{C}e^{- \theta t}\]

Da queste approssimazioni è possibile ricavare la forma della tensione sulla capacità: crescente con costante di tempo \(\lambda\) nel tratto iniziale e decresce con costante di tempo \(\theta\) per tempi lunghi. L'ampiezza massima dell'impulso è nota ed è data dalla quantità di carica che genera l'impulso sulla capacità. Questa capacità deve essere scelta piccola così da massimizzare il valore del picco.

Questa soluzione offre il vantaggio di essere poco influenzata dal rumore elettronico di fondo.

\begin{figure}
\centering
\includegraphics[width=3.73454in,height=1.41042in,alt={P4399\#yIS1}]{media/18_PTM/image469.pdf}\caption{Figura .: Tensione per Large RC}
\end{figure}

È possibile progettare la rete RC in modo che la sua evoluzione temporale sia piccola, ottenendo così uno \emph{Small RC}, nel senso che:

\[\dfrac{1}{\lambda} \gg \dfrac{1}{\theta}\]

La tensione può essere scritta come:

\[v_{C}(t) \simeq \dfrac{Q}{C}\dfrac{\lambda}{\theta}\left( e^{- \theta t} - e^{- \lambda t} \right)\]

Nei primi istanti dell'evoluzione, ovvero per:

\[t \ll \dfrac{1}{\lambda}\]

Si ottiene l'andamento

\[v_{C}(t) \simeq \dfrac{Q}{C}\dfrac{\lambda}{\theta}\left( 1 - e^{- \lambda t} \right)\]

Nel caso opposto, invece, per

\[t \gg \dfrac{1}{\theta}\]

Si ha:

\[v_{C}(t) \simeq \dfrac{Q}{C}\dfrac{\lambda}{\theta}e^{- \theta t}\]

La curva risulta essere abbastanza limitata nel tempo e, quindi, riesce a seguire l'impulso di corrente così come si verifica. Per costanti di tempo RC grandi invece, l'impulso è slargato nel tempo, quindi, non si riesce a seguire correttamente l'evoluzione temporale degli impulsi di corrente.

Per costanti RC piccole, l'ampiezza del picco dipende da \(\dfrac{Q}{C}\dfrac{\lambda}{\theta}\), ovvero è proporzionale alla carica che a sua volta dipende dell'energia del fotone \(\gamma\) incidente. Tuttavia, essendo \(\dfrac{1}{\lambda} \gg \dfrac{1}{\theta}\), allora \(1 \gg \dfrac{\lambda}{\theta}\). L'impulso di tensione possiede un'ampiezza molto minore rispetto ai circuiti \emph{Large RC}.

\begin{figure}
\centering
\includegraphics[width=6.00544in,height=2.12174in,alt={P4415\#yIS1}]{media/18_PTM/image470.pdf}\caption{Figura .: Tensione per Small RC}
\end{figure}

La corrente anodica discreta non è visibile nei circuiti con grandi RC per l'effetto dell'integrazione che essi compiono sul segnale in ingresso, mentre per piccoli RC si ha un'influenza dell'impulso di corrente legate a fluttuazioni statistiche del tempo di percorrenza degli elettroni all'interno del fotomoltiplicatore. Con i circuiti di ingresso ad ampia costante RC si produce un segnale con un rapporto segnale/rumore molto più alto ma, tuttavia, si perde l'informazione sulla durata temporale degli impulsi. Gli \emph{Small RC}, al contrario, seguono fedelmente le tempistiche degli impulsi di corrente ma sono soggetti alle fluttuazioni della corrente anodica discreta. I piccoli \emph{Spike} di corrente costituiscono un disturbo per gli stati elettronici di prelievo e misura a valle.

\begin{center}
\vfill
    \chapter{Circuiti di elaborazione in PET}
    \label{blx:refsection\therefsection}
\vfill

\minitoc
\newpage
\end{center}
\justify

\section{Circuiti elettronici per la PET}\label{circuiti-elettronici-per-la-pet}

Nel \emph{Gantry} si verifica un'annichilazione che produce due fotoni \(\gamma\) che viaggiano in direzione opposta. Questi fotoni sono intercettati da una coppia di detettori, ognuno formato da uno scintillatore e dei fotomoltiplicatori. In uscita da quest'ultimo si ha un impulso di corrente, convertito a sua volta in un impulso di tensione.

Come primo passo si amplifica il segnale in uscita dal fotomoltiplicatore con un blocco di preamplificazione che fornisce un primo guadagno e un successivo stadio di amplificazione che porta il segnale al livello voluto.

\begin{figure}
\centering
\includegraphics[width=5.7037in,height=3.13909in,alt={P4423\#yIS1}]{media/19_Circuiti/image471.pdf}\caption{Figura .: Logica di elaborazione PET}
\end{figure}

In seguito, sono eseguiti due processi in parallelo: la discriminazione dell'energia contenuta nell'impulso di tensione, importante per determinare se il fotone proviene da un evento di \emph{Scattering} o no, e la discriminazione della posizione, con la logica di Anger, in base alla coordinate del reticolato scintillatore, e la tempistica di arrivo, che, insieme alla tempistica proveniente dall'altro fotone \(\gamma\) dello stesso evento di annichilazione, permette di rilevare se i due fotoni sono contemporanei. Il circuito di coincidenza consente di verificare la differenza temporale tra l'arrivo dei due \(\gamma\) su due diversi detettori. Se questo tempo è minore di 10ns i due fotoni sono considerati come contemporanei e, quindi, provenienti dallo stesso evento di annichilazione. In caso di esito positivo la posizione dei fotoni è memorizzata così da poter ricostruire l'immagine.

Per ogni coppia di detettori la circuiteria deve essere esattamente uguale alle altre coppie di rilevatori. Ciò comporta una serie di problemi costruttivi poiché il \emph{Gantry} è composto da centinaia di coppie ed è complesso realizzare così tanti componenti elettronici esattamente uguali. Delle variazioni di amplificazioni, valutazioni energetiche e temporali possono portare, ad esempio, all'errata valutazione di un fotone che potrebbe essere scartato sebbene non abbia interagito con la materia.

Il processo di rilevazione passa per la PHA (\emph{Pulse-Height Analyzer}) che presenta un andamento a campana nell'intorno di 511keV e un altro picco in prossimità di energie più basse. Se l'energia del fotone non rientra nella finestra energetica considerata, il fotone è scartato.

L'intera logica deve possedere una tempistica estremamente rapida così da rilevare il maggior numero possibile di annichilazione.

Lo strumento di misura deve avere un'impedenza di ingresso molto elevata così da poter trascurare la corrente che scorre nel circuito di prelievo. In questo modo si minimizzano gli effetti di carico tra la sorgente e lo strumento di misura.

L'impedenza in uscita deve essere molto basse così da minimizzare le perdite di segnale quando si collega un carico al circuito di misura.

Per la trasmissione del segnale si utilizzano dei cavi coassiali, formati da un conduttore centrale e una maglia metallica separati da un dielettrico, prevalentemente polietilene. La maglia metallica è ricoperta a sua volta da un isolante per evitare la dispersione dei campi e problematiche di sicurezza.

Il cavo coassiale più diffuso è indicato con RG-58C/U caratterizzato da:

\begin{itemize}
\item
  Dielettrico in polietilene;
\item
  Impedenza caratteristica di 50Ω;
\item
  Capacità lungo la linea di 100pF/m;
\item
  Velocità di propagazione di 0.66c, ovvero il 66\% della velocità della luce.
\end{itemize}

L'utilizzo dei cavi coassiali è fondamentale date le alte frequenze raggiunte dal segnale in PET.

Per evitare riflessioni bisogna adattare l'impedenza della linea con quella del carico del circuito di prelievo e amplificazione del segnale. A tale scopo bisogna annullare il coefficiente di riflessione all'interfaccia sorgente/linea e linea/carico.

Dal punto di visto analitico, il coefficiente di riflessione è dato da:

\[\Gamma(d) = \frac{\dot{Z}(d) - {\dot{Z}}_{C}}{\dot{Z}(d) + {\dot{Z}}_{C}}\]

Per annullare questa quantità, il carico deve essere uguale all'impedenza caratteristica del cavo coassiale.

Nei casi più semplici la linea può essere chiusa su un carico con impedenza uguale a quella caratteristica, su un cortocircuito o circuito aperto. Nel primo caso la linea trasmette completamente il segnale senza subire riflessioni; invece, negli altri due casi si ha una riflessione dell'onda trasmessa. Se la linea è chiusa su un cortocircuito o \emph{Short}, il coefficiente di riflessione è:

\[\Gamma = - 1\]

L'onda, giunta al carico, è riflessa completamente con polarità opposta.

Chiudendo la linea di trasmissione su un circuito aperto, il coefficiente di riflessione è:

\[\Gamma = 1\]

Il segnale trasmesso, alla sezione del carico, è riflesso mantenendo la polarità.

\subsection{Tipi di impulsi}\label{tipi-di-impulsi}

Con riferimento al cavo coassiale è possibile distinguere gli impulsi lenti e veloci in base alla durata dell'impulso rispetto alla velocità di trasmissione lungo il cavo coassiale:

\begin{itemize}
\item
  Gli impulsi veloci possiedono un \emph{Rise-Time} comparabile o minore rispetto il tempo di transito lungo la linea;
\item
  Gli impulsi lenti, invece, possiedono un \emph{Rise-Time} molto maggiore del tempo di propagazione lungo la linea.
\end{itemize}

Conoscere le costanti di tempo con cui il segnale evolve è fondamentale per classificare l'impulso.

Si suppone che l'impulso di corrente sia in uscita a un circuito \emph{Large RC}, le equazioni che descrivono la forma d'onda sono note:

\[\left\{ \begin{array}{r}
v_{C}(t) \simeq \frac{Q}{C}\left( 1 - e^{- \lambda t} \right),\ \ t \ll \frac{1}{\theta} \\
v_{C}(t) \simeq \frac{Q}{C}e^{- \theta t},\ \ t \gg \frac{1}{\lambda}
\end{array} \right.\ \]

Si suppone che \(\theta = 0.1MHz\) e \(\lambda = 15MHz\).

Il tempo di salita può essere stimato come tre-quattro volte la costante di tempo con cui l'impulso raggiunge il regime:

\[\tau = 3\frac{1}{\lambda} = 12ns\]

Se la linea di trasmissione è lunga 1m, il tempo con cui il segnale è trasmesso è dato da:

\[t = \frac{L}{v} = \frac{1m}{0.66 \cdot 3.00 \cdot \frac{10^{8}m}{s}} = 5.05ns\]

Il tempo di salita del segnale è maggiore del tempo di propagazione, quindi, si conclude che con queste caratteristiche l'impulso è lento.

\subsection{Preamplificatore}\label{preamplificatore}

Il blocco di preamplificazione ha lo scopo di amplificare il debole segnale proveniente dal fotomoltiplicatore, aggiungendo una minima quantità di rumore. Questo elemento deve essere il più vicino possibile al fotomoltiplicatore così da ridurre la lunghezza dei cavi e, di conseguenza, gli effetti parassiti e le interferenze elettromagnetiche.

Il blocco di preamplificazione ha lo scopo di disaccoppiare lo stato di fotoamplificazione con gli elementi a valle della circuiteria elettronica. Questa funzione è svolta grazie all'elevata impedenza di ingresso e la bassa impedenza di uscita. Gli amplificatori di tensione ricevono in ingresso una tensione e in uscita una tensione amplificata di una quantità \(A\), detta guadagno. Questa soluzione è nota come \emph{Voltage-Sensitive} e può essere realizzata anche con un amplificatore invertente o non invertente.

Si considera un amplificatore invertente dato dal circuito:

\begin{figure}
\centering
\includegraphics[width=4.15741in,height=2.09113in,alt={P4465\#yIS1}]{media/19_Circuiti/image472.pdf}\caption{Figura .: Amplificatore invertente}
\end{figure}

Idealmente, l'operazionale possiede un guadagno infinito, tuttavia, nella pratica tale quantità è elevata ma non infinita. Se risulta che:

\[A \gg \frac{R_{2}}{R_{1}}\ \]

Allora la funzione di trasferimento è data da:

\[V_{out} = - \frac{R_{2}}{R_{1}}V_{in}\]

Dove \(V_{in}\) è l'ampiezza dell'impulso di tensione in uscita dal fotomoltiplicatore legato sia alla carica Q depositata sull'anodo sia dalla capacità C del circuito anodica d'uscita.

Se il segnale è ottenuto come la sovrapposizione di un corrente alternata e una continua, è possibile filtrare la componente continua e a bassa frequenza introducendo una capacità in ingesso all'amplificatore nella configurazione invertente.

\begin{figure}
\centering
\includegraphics[width=3.78176in,height=2.68519in,alt={P4473\#yIS1}]{media/19_Circuiti/image473.pdf}\caption{Figura .: Configurazione invertente con capacità di accoppiamento}
\end{figure}

In questo caso, la funzione di trasferimento può essere scritta nel dominio dei fasori:

\[G(\omega) = - \frac{R_{3}}{R_{2} + Z_{C}} = - \frac{j\omega CR_{3}}{1 + j\omega CR_{2}}\]

La risposta in frequenza ha un carattere passa-alto con frequenza di taglio e guadagno di cortocircuito noti:

\[\left\{ \begin{array}{r}
G = - \frac{R_{3}}{R_{2}}\  \\
f_{c} = \frac{1}{2\pi R_{2}C}
\end{array} \right.\ \]

Se si vuole realizzare un filtro passa-alto con frequenza di taglio di 100Hz e un guadagno di 100 allora bisogna dimensionare i componenti tali da soddisfare le relazioni:

\[\left\{ \begin{array}{r}
\frac{R_{3}}{R_{2}} = \ 100 \\
\frac{1}{2\pi R_{2}C} = 100Hz
\end{array} \right.\ \]

Se si pone \(R_{2} = 1k\mathrm{\Omega}\), allora la risoluzione del sistema è semplice:

\[\left\{ \begin{array}{r}
R_{3} = \ 100k\mathrm{\Omega} \\
C = 1.6\mu F
\end{array} \right.\ \]

Il blocco della componente in continua può essere eseguito anche tramite un amplificatore non invertente con una capacità in ingresso di accoppiamento.

\begin{figure}
\centering
\includegraphics[width=6.9081in,height=2.69403in,alt={P4484\#yIS1}]{media/19_Circuiti/image474.pdf}\caption{Figura .: Amplificatori non invertenti}
\end{figure}

La resistenza \(R_{1}\) è utilizzata per limitare la corrente di polarizzazione, dell'ordine di 10nA o minore, dell'operazionale. Questa resistenza è ottenuta col parallelo di \(R_{2}\) e \(R_{3}\):

\[R_{1} = R_{2}||R_{3} = \frac{R_{2}R_{3}}{R_{2} + R_{3}}\]

Con questa soluzione il guadagno e la frequenza di taglio sono dati dalle relazioni:

\[\left\{ \begin{array}{r}
G = 1 + \frac{R_{3}}{R_{2}}\  \\
f_{c} = \frac{1}{2\pi R_{2}C}
\end{array} \right.\ \]

Se si vuole ottenere un guadagno di 100 e una frequenza di taglio a 100Hz, in questo caso, ponendo \(R_{2} = 1k\mathrm{\Omega}\) si ottiene:

\[\left\{ \begin{array}{r}
R_{3} = \ 99k\mathrm{\Omega} \\
C = 1.6 \mu F
\end{array} \right.\ \]

Con i due stadi preamplificatori si realizza, contemporaneamente, il blocco della componete continua e il disaccoppiamento del tubo fotomoltiplicatori dalla circuiteria a valle.

\subsection{Main Amplifier}\label{main-amplifier}

Successivamente al preamplificatore si trova il circuito amplificatore vero e proprio con lo scopo di aumentare il livello del segnale. Inoltre, questo stadio consente di fornire, attraverso il \emph{Pulse Shaping}, una certa forma al segnale per permettere il \emph{Processing} del segnale stesso agli stati successivi.

Le elaborazioni devono essere eseguite mantenendo un giusto compromesso tra conservazione delle informazioni caratterizzanti dell'ingresso, quali ampiezza e il \emph{Timing}, cioè la tempistica dell'impulso con cui si rileva la coincidenza.

Lo \emph{Shaping} è essenziale per evitare il fenomeno del \emph{Pulsa Pile-Up} in cui la forma degli impulsi potrebbero facilmente generare distorsioni di ampiezza, soprattutto nel caso di eventi molto vicini tra loro.

Se gli impulso sono molto vicini tra loro, potrebbero sovrapporsi falsando così la risoluzione energetica. Questo fenomeno è noto come \emph{Pulse Pile-Up} ovvero l'impilamento degli impulsi.

Gli impulsi sono legati agli eventi statistici di rilevazione dei fotoni \(\gamma\) provenienti da processi di annichilazione, che si verificano con velocità di decadimento dell'ordine di 200-300MBq quindi, milioni di eventi al secondo. La distanza temporale tra due eventi è dell'ordine di un centesimo di \(\mu s\).

Se la durata dell'impulso di tensione, rilevato dalla strumentazione a valle, è più lunga di qualche decina di ns, può accedere che due impulsi successivi interagiscano tra di loro, sommandosi, e dando luogo a un impulso con ampiezza maggiore e, quindi, con un contenuto informativo sulla carica accumulata sull'anodo alterato. In definitiva, l'interferenza tra i due impulsi provoca un artefatto nella ricostruzione dell'energia associata al fotone incidente.

Molto probabilmente l'effetto di \emph{Pulse Pile-Up} determina l'eliminazione dell'impulso poiché la sua energia non rientra nella finestra energetica selezionata. Per tale ragione è necessario modificare la forma dell'impulso, riducendo la loro durata temporale. In questo modo impulsi vicini nel tempo non interagiscono tra di loro e, di conseguenza, non si verifica l'effetto del \emph{Pulsa Pile-Up}. Gli impulsi sono così correttamente associati alla propria energia poiché il valore del picco non è influenzato da segnali ravvicinati.

\begin{figure}
\centering
\includegraphics[width=4.85754in,height=6.33333in,alt={P4501\#yIS1}]{media/19_Circuiti/image475.pdf}\caption{Figura .: Pulse Pile-Up in alto}
\end{figure}

\subsubsection{Pulse Shaping}\label{pulse-shaping}

Il processo analogico di modellazione dell'impulso o \emph{Pulse Shaping} può essere effettuato mediante numerose metodiche, tra cui la più comune riguarda l'utilizzo di una cascata di circuito CR e RC, separati da uno stadio di disaccoppiamento. I buffer sono utilizzati anche per adattare il segnale di ingresso e il segnale elaborato con l'uscita.

Il primo circuito CR si comporta come un derivatore, ovvero un filtro passa-alto, che lascia passare le sole componenti ad alta frequenza. Il secondo stadio, invece, si comporta come un integratore, che nel dominio della frequenza rappresenta un filtro passa-basso. La cascata dei due circuiti si comporta come un filtro passa-banda.

\begin{figure}
\centering
\includegraphics[width=5.52882in,height=3.47619in,alt={P4506\#yIS1}]{media/19_Circuiti/image476.pdf}\caption{Figura .: Circuito CR-RC}
\end{figure}

Se in ingresso si pone un segnale molo lento, schematizzabile come un gradino, il primo stadio CR lascia passare solo le componenti in alta frequenza associate ai fronti di salita e discesa. Il segnale continuo decade con una costante di tempo \(\tau_{d}\) dipendente dai parametri C ed R della rete differenziatore. Il segnale così ottenuto è, poi, filtrato passa-basso, quindi, i fronti di salita sono smussati e variano con costante di tempo \(\tau_{i}\). La forma d'onda risultate è un impulso con durata finita dipendente dalle costanti di tempo scelte per realizzare il sistema.

Siccome i due stadi sono separati da un buffer è possibile analizzare la risposta al gradino separatamente. Se le due costanti di tempo sono diverse, il segnale in uscita dal primo stadio deve rispettare l'equazione ottenuta dalla legge di Kirchhoff per le tensioni:

\[\left\{ \begin{array}{r}
\frac{dv_{C}}{dt} + \frac{1}{R_{1}C_{1}}v_{C} = \frac{E_{in}}{R_{1}C_{1}} \\
v_{C}(0) = 0
\end{array} \right.\ \]

L'integrale generale dell'equazione differenziale è dato da:

\[v_{C}(t) = ke^{- \frac{t}{\tau_{d}}} + E_{in}\]

Imponendo la condizione iniziale:

\[v_{C}(t) = E_{in}\left( 1 - e^{- \frac{t}{\tau_{d}}} \right)\]

Questa tensione rappresenta il forzamento del secondo stadio, governato, quindi, dalle equazioni:

\[\left\{ \begin{array}{r}
\frac{dv_{C}}{dt} + \frac{1}{R_{2}C_{2}}v_{C} = \frac{E_{in}}{R_{2}C_{2}}\left( 1 - e^{- \frac{t}{\tau_{d}}} \right) \\
v_{C}(\infty) = 0
\end{array} \right.\ \]

La soluzione del problema è nota:

\[v_{C}(t) = \frac{\tau_{i}}{\left( \tau_{d} - \tau_{i} \right)}E_{in}\left( e^{- \frac{t}{\tau_{d}}} - e^{- \frac{t}{\tau_{i}}} \right)\]

Nel caso in cui le due costanti di tempo siano uguali, per ottenere l'espressione dell'uscita è necessario eseguire un'operazione di limite:

\[\lim_{\tau_{i} \rightarrow \tau_{d}}{v_{C}(t)} = \lim_{\tau_{i} \rightarrow \tau_{d}}{\frac{\tau_{i}}{\left( \tau_{d} - \tau_{i} \right)}E_{in}\left( e^{- \frac{t}{\tau_{d}}} - e^{- \frac{t}{\tau_{i}}} \right)} = E_{in}\lim_{\tau_{i} \rightarrow \tau_{d}}{\frac{e^{- \frac{t}{\tau_{d}}}\left( 1 - e^{- \frac{t}{\tau_{i}}}e^{\frac{t}{\tau_{d}}} \right)\tau_{i}}{\tau_{d}\left( 1 - \frac{\tau_{i}}{\tau_{d}} \right)}\frac{{(E}_{in}e^{- \frac{t}{\tau_{d}}})}{\tau_{d}}} = = \lim_{\tau_{i} \rightarrow \tau_{d}}{\tau_{i}\frac{\left( 1 - e^{- \frac{t}{\tau_{i}}}e^{\frac{t}{\tau_{d}}} \right)}{\left( 1 - \frac{\tau_{i}}{\tau_{d}} \right)}} = \frac{t}{\tau_{d}}E_{in}e^{- \frac{t}{\tau_{d}}}\]

La soluzione CR-RC, grazie al filtraggio passa-banda del segnale, permette di incrementare il rapporto segnale/rumore.

\subsubsection[Gaussian CR-(RC)n]{\emph{Gaussian} CR-(RC)\textsuperscript{n}}
\label{gaussian-cr-rcn}

Ponendo in ingresso un circuito CR e tanti circuiti RC in cascata, opportunamente separati da stadi di buffer, si riesce a ottenere una forma d'onda a campana di Gauss. È comodo utilizzare questa metodica poiché la campana di Gauss è caratterizzata solo dall'ampiezza e della durata, legate alla deviazione standard. Queste informazioni permettono di discriminare l'energia del fotone incidente e l'istante di tempo in cui si è verificato la ricezione. La durata di ogni campana, ovviamente, deve essere progettata così da evitare il fenomeno del \emph{Pulse Pile-Up}. A fine condizionamento analogico, le varie gaussiane prodotte, oltre ad avere la stessa ampiezza, presentano un picco direttamente collegato alla carica accumulata sull'anodo.

È possibile dimostrare che con quattro stadi di tipo RC posti in cascata la forma d'onda ottenuta ben approssima una gaussiana. Per ottenere lo stesso effetto è possibile utilizzare filtri attivi passa-basso di tipo \emph{Sallen-Key}.

\begin{figure}
\centering
\includegraphics[width=6.44459in,height=1.76191in,alt={P4525\#yIS1}]{media/19_Circuiti/image477.pdf}\caption{Figura .: Cella Sallen-Key}
\end{figure}

Dato che il filtro \emph{Sallen-Key} è composto da due blocchi integratori RC, avrà una funzione di trasferimento a due poli del tipo

\[H(s) = \frac{A_{0}}{1 + \omega_{c}\left( C_{1}\left( R_{1} + R_{2} \right) + \left( 1 - A_{0} \right)R_{1}C_{2} \right)s + \omega_{c}^{2}R_{1}R_{2}C_{1}C_{2}s^{2}}\]

\subsubsection{Undershoot}\label{undershoot}

Considerare un ingresso a gradino equivale a considerare il caso limite in cui il segnale del fotomoltiplicatore risulta essere molto più lungo delle costanti di tempo del circuito di filtraggio. In questa condizione è possibile considerare l'elaborazione come integrazione infinita sul segnale in ingresso.

Nella realtà, i segnali provenienti dai preamplificatori presentano una durata finita legata al tempo di decadimento. Di conseguenza si assiste a una visibile modifica nella risposta dello \emph{Shaper} come l'\emph{Undershoot}. Questo problema deriva dall'operazione di derivazione del primo blocco CR che opera un filtraggio passa-alto sul segnale in ingresso.

Quest'ultimo possiede una propria costante di tempo diversa dalla rete elettrica, e ciò può portare a delle componenti negative con durata molto ampia. Sommandosi con gli altri impulsi, queste componenti negative possono causare dei problemi di \emph{Pile-Up} portando a una sottostima dell'energia del fotone incidente poiché l'ampiezza dell'impulso successivo si somma con una componente negativa e, dunque, risulta essere complessivamente ridotta di intensità.

\begin{figure}
\centering
\includegraphics[width=6.45274in,height=1.90909in,alt={P4533\#yIS1}]{media/19_Circuiti/image478.pdf}\caption{Figura .: Impulso originale}
\end{figure}

\begin{figure}
\centering
\includegraphics[width=6.29786in,height=1.97403in,alt={P4535\#yIS1}]{media/19_Circuiti/image478.pdf}\caption{Figura .: Undershoot}
\end{figure}

La problematica può essere risolta cancellando lo zero-polo nella funzione di trasferimento mediante un circuito del tipo:

\begin{figure}
\centering
\includegraphics[width=5.45329in,height=2.09524in,alt={P4538\#yIS1}]{media/19_Circuiti/image479.pdf}\caption{Figura .: Filtro passa-banda}
\end{figure}

Dove sono presenti due circuiti RC in cascata, separati da uno stadio di buffer. La funzione di trasferimento è ottenuta moltiplicando le funzioni di trasferimento di ogni singolo stadio:

\[H(s) = \frac{R_{1}\left( R_{pz}C_{1}s + 1 \right)}{\left( 1 + s\tau_{2} \right)\left( R_{pz} + R_{1}\left( R_{pz}C_{1}s + 1 \right) \right)}\]

Il numeratore introduce uno zero e, quindi, rappresenta il termine di differenziazione del segnale.

Se si pone \(R_{pz} = \frac{\tau_{2}}{C_{1}}\) la funzione di trasferimento si riduce a una semplice cascata di filtri RC passivi:

\[H(s) = \frac{R_{1}}{\left( \frac{\tau_{2}}{C_{1}} + R_{1} + \frac{\tau_{1}\tau_{2}}{C_{1}}s \right)}\]

Con questa soluzione si rimuove il fenomeno dell'\emph{Undershoot} così da avere un impulso di durata finita senza nessun attraversamento per lo zero.

\begin{figure}
\centering
\includegraphics[width=5.45647in,height=1.80952in,alt={P4546\#yIS1}]{media/19_Circuiti/image480.pdf}\caption{Figura .: Uscita dallo Shaping}
\end{figure}

\subsubsection{Baseline Shift}\label{baseline-shift}

La capacità della rete CR, presente all'interno della soluzione di \emph{Shaping CR-RC}, blocca la componente continua poiché mostra un'elevatissima impedenza a basse frequenze. La tensione ai capi della resistenza presenta solamente una componete alternata con valor medio nullo: la forma d'onda presenta componenti positive e negative che sottendono la stessa area così da avere un valor medio nullo. Questo disturbo è noto come \emph{Baseline Shift} e può causare distorsioni d'ampiezza di impulsi ravvicinati nel tempo.

\begin{figure}
\centering
\includegraphics[width=5.3221in,height=3.9881in,alt={P4550\#yIS1}]{media/19_Circuiti/image481.pdf}\caption{Figura .: Baseline Shift}
\end{figure}

Il \emph{Baseline Shift} determina la comparsa di altri \emph{Undershoot} che portano a una sottostima dell'ampiezza del picco e, di conseguenza, dell'energia del fotone incidente.

Per risolvere il problema introdotto dalle capacità di accoppiamento si adotta una metodica nota come \emph{Baseline Restoration} che prevede l'utilizzo del circuito RC la cui uscita è pilotata da un interruttore:

\begin{figure}
\centering
\includegraphics[width=3.67354in,height=2.31226in,alt={P4554\#yIS1}]{media/19_Circuiti/image482.pdf}\caption{Figura .: Circuito di risoluzione del Baseline Shift}
\end{figure}

L'interruttore è aperto finché l'impulso è positivo. Non appena il segnale attraverso lo zero, l'interruttore è chiuso e il segnale è posizionato a massa con una costante di tempo data da:

\[\tau = \left( R + R_{0} \right)C\]

Il circuito, con l'interruttore chiuso, si comporta come un filtro passa-alto che cortocircuita verso masse le componenti negative del segnale.

Nei circuiti reali la funzione dall'interruttore è eseguita da diodi o altri componenti non lineari che realizzano un raddrizzatore a singola semionda.

\begin{figure}
\centering
\includegraphics[width=3.97014in,height=3.04289in,alt={P4560\#yIS1}]{media/19_Circuiti/image483.pdf}\caption{Figura .: Circuito reale per risolvere il Baseline Shift}
\end{figure}

Nei due diodi identici D1 e D2, polarizzati direttamente, si lascia fluire una corrente di piccola entità \(2i\) che si ripartisce in modo uguale nei due elementi non lineari. Quando il segnale in ingresso è nullo il potenziale del nodo di ingresso coincide con il potenziale di rifermento poiché, avendo un nodo in comune, i due diodi si portano allo stesso potenziale. Virtualmente, quindi, il nodo di ingresso è collegato al riferimento.

Se l'impulso di tensione in ingresso è positivo, il diodo D1 è contro-polarizzato, quindi, la corrente \(2i\) scorre completamente verso massa tramite D2. Il nodo di ingresso, appena a valle della capacità si porta alla tensione dell'impulso di tensione del fotomoltiplicatore.

Quando la tensione in ingresso è negativa, D1 entra in conduzione (potenziale del catodo maggiore dell'anodo negativo) così come il diodo D2. Ciò porta il nodo di entrata al potenziale nullo finché il segnale in ingresso non supera la soglia di conduzione del diodo.

Lo \emph{Shift} della linea di base si presenta con i circuiti di \emph{Shaping} monopolari. Esso può essere eliminato ricorrendo a uno \emph{Shaping} bipolare dei segnali, che, tuttavia, riduce il rapporto segnale/rumore rispetto a un circuito monopolare.

\subsection{Pulse Shaping con Single Delay Line}\label{pulse-shaping-con-single-delay-line}

La modellazione dell'impulso con un circuito \emph{Single Delay Line} rappresenta un'alternativa al circuito CR-RC e produce ancora un impulso monopolare. La soluzione prevede la sostituzione della capacità con una linea di trasmissione chiusa su un corto circuito che riflette il segnale in ingresso. Con questa soluzione non si ricorre a filtri passa-alto o passa-basso, quindi, non si modificano il fronte di salita e discesa né si produce il decadimento esponenziale del segnale. Ciò peggiora il rapporto segnale/rumore poiché le componenti di disturbo non sono cancellate.

\begin{figure}
\centering
\includegraphics[width=5.85903in,height=2.29565in,alt={P4568\#yIS1}]{media/19_Circuiti/image484.pdf}\caption{Figura .: Single Delay Line}
\end{figure}

La linea di trasmissione è progettata in modo che la sua lunghezza assicuri un tempo di propagazione uguale a \(2T\), con \(T\) durata per giungere al carico. Si suppone di trasmettere sulla linea un segnale a gradino. Quando il segnale è trasmesso, giunge sul carico cortocircuito dove è riflesso e ribaltato. Il segnale viaggia, quindi, in senso retrogrado e con polarità opposta. L'onda progressiva e regressiva, sulla linea di trasmissione, si sommano per dare origine a un segnale di durata\(\ 2T\).

\begin{figure}
\centering
\includegraphics[width=4.20432in,height=4.17391in,alt={P4571\#yIS1}]{media/19_Circuiti/image485.pdf}\caption{Figura .: Shaping Single Delay Line con impulso reale}
\end{figure}

Se l'impulso trasmesso non è rettangolare, può verificarsi il fenomeno dell'\emph{Undershoot} dovuto alla sovrapposizione della porzione di onda progressiva con l'onda regressiva di ampiezza, in modulo, maggiore. Ciò provoca un impulso positivo di durata \(2T\) e la restante porzione dell'impulso è negativa poiché la somma tra le due onde porta all'attraversamento per lo zero.

\begin{figure}
\centering
\includegraphics[width=4.35714in,height=3.03072in,alt={P4574\#yIS1}]{media/19_Circuiti/image486.pdf}\caption{Figura .: Single Delay Line con impulso reale}
\end{figure}

Questo problema è facilmente risolto utilizzando un carico con impedenza minore di quella caratteristica della linea di trasmissione. Infatti, il carico, oltre a ribaltare e riflettere il segnale, ne causa un'attenuazione tanto maggiore quanto minore è il coefficiente di riflessione alla sezione del carico. Progettando la linea di trasmissione e il carico è possibile fare in modo che l'impulso risultante abbia durata di \(2T\) e che le componenti che non rientrano in questa durata temporale siano esattamente uguali e opposte all'onda riflessa che viaggia in senso opposto. Il risultato è un impulso monofasico con durata finita e ampiezza legata all'energia del fotone incidente.

\begin{figure}
\centering
\includegraphics[width=4.79762in,height=3.05508in,alt={P4577\#yIS1}]{media/19_Circuiti/image487.pdf}\caption{Figura .: Single Delay Line chiusa su impedenza}
\end{figure}

\subsection{Double Delay Line}\label{double-delay-line}

Il \emph{Double Dalay Line} permette di ottenere gli impulsi bipolari ponendo in cascata due linee di trasmissione con stesso ritardo e separate da uno stadio di Buffer. Questo circuito non introduce nessun filtraggio, quindi il SNR è minore rispetto al circuito CR-RC. Tuttavia, le linee di trasmissione possono avere velocità di propagazione molto spinte e, di conseguenza, un conteggio di eventi più elevato.

\begin{figure}
\centering
\includegraphics[width=4.88095in,height=1.80922in,alt={P4581\#yIS1}]{media/19_Circuiti/image488.pdf}\caption{Figura .: Double Delay Line}
\end{figure}

Se si trasmette un segnale a gradino, alla prima interfaccia di separazione tra la linea di trasmissione e il cortocircuito, si genera un'onda riflessa di polarità opposta che, sovrapponendosi con l'onda progressiva, determina la formazione di un impulso rettangolare.

L'impulso si propaga lungo la linea ed è posto in ingresso al secondo stadio, nel quale si propaga con la stessa velocità del primo. Quindi, dopo un tempo uguale alla sua durata è riflesso con polarità opposta. In uscita, quindi, vi è un impulso bipolare ottenuto dal segnale a gradino positivo e la sua versione ritardata di \(2T\) e con polarità opposta.

\begin{figure}
\centering
\includegraphics[width=4.24999in,height=3.75556in,alt={P4585\#yIS1}]{media/19_Circuiti/image489.pdf}\caption{Figura .: impulso con Double Delay Line}
\end{figure}

Il passaggio per lo zero è importante per la rilevazione della tempistica poiché, in genere, i circuiti che permettono di discriminare questo evento presentano le prestazioni migliori rispetto al passaggio per il massimo.

Nella realtà, l'impulso di corrente con durata di qualche ns è convertito in tensione dallo stadio di prelievo e amplificato da un preamplificatore. Il segnale così ottenuto presenta una durata molto maggiore di quella originale e un fronte di discesa con un'evoluzione esponenziale. Per limitare l'impulso nel tempo si esegue l'operazione di \emph{Shaping} che permette di avere una tempistica standardizzata, mantenendo le informazioni sul picco massimo di tensione legato alla quantità di carica accumulata sull'anodo e, quindi, dall'energia del fotone \(\gamma\) incidente.

\begin{figure}
\centering
\includegraphics[width=5.92857in,height=4.36633in,alt={P4589\#yIS1}]{media/19_Circuiti/image490.pdf}\caption{Figura .: Varie elaborazioni degli impulsi del PMT}
\end{figure}

\subsection{Discriminatore differenziale}\label{discriminatore-differenziale}

Prelevato il segnale in uscita dal fotomoltiplicatore e modificata la forma, è necessario elaborare le informazioni riguardanti gli istanti di occorrenza e le ampiezze degli impulsi. La prima conoscenza è fondamentale per rilevare le coincidenze di due fotoni provenienti dello stesso evento di annichilazione, mentre la seconda permette di ricavare l'energia dei fotoni incidenti. Una prima soluzione potrebbe essere la conversione del segnale analogico elaborato in digitale. Tuttavia, a causa delle brevi durate degli impulsi, i campionatori necessari dovrebbero avere un'estrema frequenza di campionamento. Ancora oggi, quindi, molte elaborazioni, come la discriminazione energetica, sono svolte mediante circuiti analogici.

\begin{figure}
\centering
\includegraphics[width=6.69555in,height=1.50217in,alt={P4593\#yIS1}]{media/19_Circuiti/image491.pdf}\caption{Figura .: Elaborazione digitale del segnale PET}
\end{figure}

L'energia dei vari impulsi può subire delle fluttuazioni energetiche dovute a fenomeni statistici. Ciò impone l'uso di una finestra di 350-650keV in cui i fotoni sono considerati provenienti da un evento di annichilazione e non \emph{Scattering}. La soglia da adottare per la rilevazione energetica non è unica. Se la soglia fosse unica esisterebbe una scarsa tolleranza al rumore poiché vi sarebbero delle oscillazioni intorno alla soglia che genererebbero delle commutazioni spurie nel comparatore.

\begin{figure}
\centering
\includegraphics[width=3.96009in,height=4.90476in,alt={P4596\#yIS1}]{media/19_Circuiti/image492.pdf}\caption{Figura .: Commutazioni spurie}
\end{figure}

Per evitare questi effetti si utilizzano dei comparatori a isteresi, realizzati con un \emph{Trigger} di Schmitt. Nella tecnica a isteresi la soglia non è fissa ma varia in un determinato intervallo, la cui ampiezza dipende dal livello di rumore del segnale in ingresso così da ridurre le commutazioni spurie. La realizzazione circuitale del comparatore a isteresi è la seguente:

\begin{figure}
\centering
\includegraphics[width=3.48399in,height=2.79762in,alt={P4599\#yIS1}]{media/19_Circuiti/image493.pdf}\caption{Figura .: Commutatore di Schmitt}
\end{figure}

Ovviamente se il segnale in ingresso è minore della soglia \(V_{S}(t) - V_{in} > 0\), l'OpAmp satura verso l'alimentazione positiva; mentre, se la soglia è minore dell'ingresso \(V_{S}(t) - V_{in} < 0\), si ha la saturazione verso l'alimentazione negativa.

La soglia di tensione del comparatore è variabile, nel senso che assume valori diversi a seconda dell'uscita. È possibile scrivere:

\[V_{S}(t) = V_{R} + R_{2}i(t) = V_{R} + \frac{R_{2}}{R_{1} + R_{2}}\left( V_{out}(t) - V_{R} \right)\]

Se la tensione di ingresso è minore della tensione di soglia superiore o \emph{High Threshold,} l'uscita sarà alta.

\[V_{S}^{H}(t) = V_{R} + \frac{R_{2}}{R_{1} + R_{2}}\left( V_{H} - V_{R} \right)\]

Una volta che il segnale in ingresso supera la soglia l'uscita si porta al valore basso della saturazione negativa. Il valore della soglia si riduce portandosi al valore \emph{Low Threshold}.

\[V_{S}^{L}(t) = V_{R} + \frac{R_{2}}{R_{1} + R_{2}}\left( V_{L} - V_{R} \right)\]

\begin{figure}
\centering
\includegraphics[width=6.51538in,height=2.28571in,alt={P4608\#yIS1}]{media/19_Circuiti/image494.pdf}\caption{Figura .: Andamento del segnale e della soglia}
\end{figure}

Il valore dell'isteresi è dato dall'ampiezza della striscia ottenuta come differenza delle due soglie:

\[S = \frac{R_{2}}{R_{1} + R_{2}}\left( V_{H} - V_{L} \right)\]

I segnali, per poter essere accettati, devono essere contenuti nella striscia di tensioni. I valori dei limiti superiore e inferiore sono imposti in modo che le energie rientrino nell'intervallo di 350-650keV. Le energie maggiore di 511keV possono essere dovute a \emph{Dark Current} oppure a fluttuazioni statistiche che portano il cristallo scintillatore a emettere un numero di fotoni maggiore del valore medio. Ancora, gli atomi droganti possono introdurre delle radiazioni \emph{Afterglow} indipendenti dai fotoni \(\gamma\) incidenti. Per energie non comprese nell'intervallo energetico di 350-650keV, il dato è considerato corrotto dal rumore e, quindi, scartato.

Le soglie del commutatore a isteresi sono in volt, quindi, è necessario utilizzare apposite conversioni per fare in modo che la banda energetica sia quella voluta. I valori delle soglie dipendono dalla quantità di carica accumulata sull'anodo, il guadagno dell'amplificazione e dall'energia dei fotoni incidenti.

\subsection{Conversione digitale}\label{conversione-digitale}

Nella conversione A/D il range della tensione analogica da convertire è diviso in un certo numero di intervalli, idealmente con ampiezza uguale, e a ciascun intervallo è associato un livello digitale, codificato su un certo numero di bit. La funzione ingresso-uscita del convertitore A/D è detta a scalino.

\begin{figure}
\centering
\includegraphics[width=4.61597in,height=3.88642in,alt={P4616\#yIS1}]{media/19_Circuiti/image495.pdf}\caption{Figura .: Caratteristica ingresso-uscita di un ADC}
\end{figure}

In ascissa è riportata l'ampiezza relativa del range di tensioni da codificare con \(V_{in}\) e con \(V_{refLo}\) gli estremi dell'intervallo da codificare. L'intervallo è rapportato al valore di riferimento, spesso di 3.3V.

Sulle ordinate si pongono i \(2^{m}\ \)livelli dove \(m\) è il numero di bit utilizzati. Si definisce risoluzione \(Q\) come:

\[Q = \frac{E_{FSR}}{2^{m}}\]

Q rappresenta l'ampiezza ideale degli intervalli di tensione da codificare. Possono, comunque, presentarsi degli errori dovuto allo scostamento della caratteristica reale da quella ideale.

L'ampiezza degli intervalli, quindi, può variare e, per quantificare tale effetto, si definisce l'errore come non-linearità differenziale:

\[DNL(k) = \frac{W(k) - Q}{Q}\]

Dove \(W(k)\) è l'ampiezza del \(k\)-esimo intervallo.

\subsubsection{Elaborazione digitale}\label{elaborazione-digitale}

Le elaborazioni dei segnali nel mondo digitale presentano una maggiore flessibilità e, inoltre, la manipolazione delle forme d'onda, lo \emph{Shaping} e tutti gli altri processi sono realizzati con le caratteristiche volute.

I segnali digitali sono meno affetti da artefatti, a eccezione dell'errore di quantizzazione, poiché, data la caratteristica ingresso-uscita dei componenti digitali, un eventuale rumore sovrapposto è reiettato dal primo stadio di elaborazione.

Nel caso specifico della PET, la tempistica degli impulsi è dell'ordine dei ns e ciò richiede un convertitore analogico/digitale con elevatissime frequenze di campionamento.

Il risultato è una limitazione dell'accuratezza temporale poiché eventi più veloci non possono essere rilevati. I contatori sono elementi essenziali nell'hardware della PET poiché permettono di memorizzare il numero di eventi rilevato. Esso è realizzato mediante la connessione in cascata di vari flip-flop, temporizzati da un unico segnale di clock.

Per la PET si utilizzano contatori asincroni, in cui il segnale di clock è posto in ingresso solamente al primo stadio. Questo commuta e porta la sua uscita al flip-flop successivo che, quindi, cambia stato a sua volta. Con \(n\) bistabili si possono memorizzare \(2^{n}\) parole binarie.

I flip-flop o bistabili sono circuiti elettronici sequenziali molto semplici utilizzati nell'elettronica digitale come elementi di memoria elementare poiché possono conservare lo stato alto o basso. Il flip-flop J-K è il più semplice e presenta due ingressi J e K per il cambio stato e uno di sincronizzazione.

\begin{figure}
\centering
\includegraphics[width=5.76389in,height=3.83222in,alt={P4632\#yIS1}]{media/19_Circuiti/image496.pdf}\caption{Figura .: registri contatori}
\end{figure}

\subsection{Time Pick-Off}\label{time-pick-off}

Per permettere il conteggio è necessario discriminare gli eventi utili e considerare solo quelli che possiedono determinate caratteristiche come segnali di \emph{Trigger}. Esistono alcuni criteri per generare il segnale di \emph{Trigger} attraverso l'individuazione di un instante di occorrenza di un evento, detto \emph{Time Pick-Off}.

La rilevazione del fronte di salita può essere effettuata mediante il conteggio del tempo necessario affinché l'impulso superi una certa soglia. Questa metodica pone una serie di problematiche di rilevazione. Infatti, può accadere che le ampiezze degli impulsi siano diverse tra loro, quindi, un segnale con ampiezza minore, a parità di forma d'onda, impiega un tempo maggiore a superare la soglia rispetto a un segnale con ampiezza maggiore. In altre parole, due impulsi contemporanei ma con energia differenti saranno rilevati in istanti di tempo successivi poiché il segnale a energia maggiore raggiunge prima il valore della soglia. Analogamente, due impulsi non contemporanei ma con ampiezza diversa possono essere rilevati nello stesso istante poiché raggiungono la soglia in intervalli di tempo differenti. La problematica associata all'attraversamento della soglia per segnali con diverse energie è detta \emph{Amplitude Walk}.

\begin{figure}
\centering
\includegraphics[width=6.68646in,height=1.1039in,alt={P4637\#yIS1}]{media/19_Circuiti/image497.pdf}\caption{Figura .: Errore da Amplitude Walk}
\end{figure}

La presenza del rumore sovrapposto agli impulsi elettrici può essere tale che la rilevazione avvenga prima o dopo l'effettivo superamento della soglia da parte della forma d'onda non corrotta dal rumore. Questo fenomeno è note come \emph{Time Jitter} e porta a un errore di rilevazione nel superamento della soglia.

Il \emph{Time Jitter} e l'\emph{Amplitude Walk} determinano degli errori nella rilevazione della tempistica degli impulsi con effetti sulla ricostruzione dell'immagine.

Per risolvere questi inconvenienti si ricorre a impulsi bipolari, caratterizzati da una componente positiva e una negativa. Invece di rilevare il passaggio per una soglia, si determina l'istante di tempo in cui l'impulso oltrepassa lo zero, detto istante di \emph{Zero-Crossing}. Per realizzare queste forme d'onda è comodo utilizzare i \emph{Dual Daley Line} che, tuttavia, non filtrano il segnale.

Lo \emph{Shaping} della forma d'onda impulsiva deve essere tale che la durata della fase positiva sia uguale a quella negativa. In questo modo, indipendentemente dall'ampiezza del segnale, l'istante di \emph{Zero-Crossing} è costante per ogni impulso: due impulsi contemporanei, attraversano lo zero nello stesso istante di tempo. Si riduce, quindi, il fenomeno dell'\emph{Amplitude Walk}.

\begin{figure}
\centering
\includegraphics[width=4.69792in,height=2.17142in,alt={P4643\#yIS1}]{media/19_Circuiti/image498.pdf}\caption{Figura .: Attraversamento per lo zero di un impulso bifasico}
\end{figure}

Il circuito per la rilevazione dello \emph{Zero-Crossing} è dato da:

\begin{figure}
\centering
\includegraphics[width=4.43698in,height=2.72917in,alt={P4646\#yIS1}]{media/19_Circuiti/image499.pdf}\caption{Figura .: Circuito di rilevazione dello zero-crossing}
\end{figure}

La configurazione presentata è sostanzialmente un comparatore con isteresi, dove l'ingresso è comparato con una soglia. L'uscita è alta se l'ingresso supera la soglia, altrimenti è bassa. I diodi D1 e D2 fungono da circuiti di protezione poiché evitano che tensioni estremamente elevate siano poste in ingresso all'operazionale. La capacità nella rete di retroazione è detta di \emph{Speed-Up} e permette la commutazione rapida del fronte d'onda raccolto al secondario.

Il fenomeno dell'isteresi determina la rilevazione del passaggio per lo zero quando la tensione si trova in un suo intorno. Infatti, è possibile dimostrare che l'evento di passaggio per lo zero appartiene a una fascia di {[}-2.5mV 2.5mV{]} all'interno del quale avviene la comparazione con isteresi. Data la soglia mobile, questa comparazione risulta essere più robusta rispetto al rumore sovrapposto e, inoltre, l'errore introdotto dall'ampiezza dell'isteresi è comunque contenuto. Ad esempio, se si amplifica il segnale fino a raggiungere un valore dell'ordine della decina di volt, l'errore commesso è 0.5\%.

\subsubsection{Costant Fraction Timing}\label{costant-fraction-timing}

La \emph{Costant Fraction Timing} permette di utilizzare un impulso monopolare per lo \emph{Zero-Crossing} mediante un'opportuna operazione di \emph{Shaping}. Dell'impulso in ingresso, come a esempio un gradino reale di polarità negativa, si produce una versione attenuata di un fattore \(F < 1\) costante. Successivamente, il segnale di partenza è ritardato, invertito e sommato con la sua versione attenuata. Il risultato è una forma d'onda con uno \emph{Zero-Crossing} con una parte negativa di ampiezza massima \(Fv_{in}\). La forma dell'impulso finale è controllata e presenta un passaggio per lo zero così da rendere più semplice la determinazione delle occorrenze.

\begin{figure}
\centering
\includegraphics[width=2.63621in,height=3.90625in,alt={P4652\#yIS1}]{media/19_Circuiti/image500.pdf}\caption{Figura 20.31: Shaping con Constant Fraction Timing}
\end{figure}

Dal punto di vista hardware, questa soluzione utilizza un amplificatore con guadagno minore dell'unità, un ritardatore, realizzato ad esempio con una linea di ritardo, e un sommatore.

Il \emph{Leading Edge} funziona meglio quando la soglia è al 10-20\% del livello massimo, ottenuto in assenza di \emph{Amplitude Walk}, e, inoltre, il \emph{Time Jitter} risulta essere ridotto. Ciò porta a considerare il passaggio dello zero nella forma d'onda manipolata con buona approssimazione uguale all'istante di occorrenza dell'evento.

\subsection{Coincidenza}\label{coincidenza}

Una volta rilevata l'occorrenza temporale dei vari impulsi è necessario valutare se due eventi sono coincidenti. Come prima istanza si cerca di determinare la distribuzione della tempistica di un impulso attraverso un diagramma detto \emph{Time-Spectrum}.

L'impulso prodotto da una sorgente di annichilazione è inviato a due canali nei quali si misura l'istante di occorrenza di ciascun impulso tramite dei circuiti di \emph{Time Pick-Off}. Il primo canale permette di rilevare l'occorrenza dell'evento che avvia lo \emph{Start} di un timer. Il secondo canale presenta un circuito di \emph{Time Pick-Off}, identico al primo, seguito da una linea di ritardo costante T. L'evento di attraversamento dello zero del secondo canale, posticipato rispetto al primo, determina il segnale di \emph{Stop} del conteggio da parte del contatore. Un analizzatore a valle quantifica il tempo del conteggio tra l'evento di \emph{Start} (passaggio per lo zero del primo canale) e di \emph{Stop} (passaggio per lo zero del secondo canale).

\begin{figure}
\centering
\includegraphics[width=6.69583in,height=0.77083in,alt={P4659\#yIS1}]{media/19_Circuiti/image501.pdf}\caption{Figura .: Logica di rilevazione delle coincidenze}
\end{figure}

In assenza di rumore e fluttuazioni statistiche, tutti gli impulsi presenterebbero un'occorrenza uguale al tempo di ritardo introdotto nel secondo canale T.

\begin{figure}
\centering
\includegraphics[width=3.55052in,height=2.54167in,alt={P4662\#yIS1}]{media/19_Circuiti/image502.pdf}\caption{Figura .: Time-Spectrum ideale}
\end{figure}

A causa dei disturbi dovuti a parametri costruttivi, il \emph{Time Jitter} e l'\emph{Amplitude Walk} è introdotta una certa aleatorietà nel ritardo tra gli istanti di occorrenza. I fenomeni di \emph{Time Jitter} e \emph{Amplitude Walk} ovviamente sono indipendenti tra i due canali poiché i due circuiti non sono esattamente uguali, di conseguenza non rilevano lo \emph{Zero-Crossing} esattamente nello stesso punto del segnale. Può capitare che un circuito di \emph{Time Pick-Off} rilevi lo zero prima dell'altro e viceversa.

La distribuzione non è, quindi, impulsiva ma avrà un certo \emph{Spread}, tendente a una gaussiana, con valor medio T. Rappresentando la distribuzione temporale in un piano dove sull'asse delle ascisse vi è il ritardo e sull'asse delle ordinate il numero di conteggi per canale si ottiene il \emph{Time-Spectrum}.

\begin{figure}
\centering
\includegraphics[width=5.64097in,height=2.97917in,alt={P4666\#yIS1}]{media/19_Circuiti/image503.pdf}\caption{Figura .: Time-Spectrum reale}
\end{figure}

Si considerano ora il caso in cui la sorgente di impulsi sia racchiusa tra due detettori che introducono a loro volta altri fenomeni di ritardo, modificando lo spettro del tempo.

\begin{figure}
\centering
\includegraphics[width=6.18031in,height=2.4434in,alt={P4669\#yIS1}]{media/19_Circuiti/image504.pdf}\caption{Figura .: Logica realmente utilizzata in PET}
\end{figure}

Tipicamente gli eventi rilevati da due detettori sono divisi in tre categorie: normali, \emph{Scatter} e \emph{Random}. Tutti gli eventi, indipendentemente dalla loro origine sono detti \emph{Prompt Coincidences}.

\begin{itemize}
\item
  Gli eventi normali, in cui entrambi gli impulsi provenienti da un processo di annichilazione sono rilevati, sono detti \emph{True Coincidence}. In questo caso, dato che i due circuiti di \emph{Time Pick-Off} non sono esattamente uguali tra loro, tra lo \emph{Start} e lo \emph{Stop} non passa un tempo esattamente uguale a T ma che può essere leggermente maggiore o minore, distribuito secondo una certa legge probabilistica;
\item
  Gli eventi \emph{Scatter} sono dovuti all'interazioni dei fotoni \(\gamma\) all'interno della materia. I fotoni sono deviati e perdono energia a causa dell'effetto Compton o fotoelettrico. Ciò potrebbe portare alla rilevazione solo di un fotone da parte di un detettore. Può anche accadere che, pur arrivando entrambi sui detettori, uno dei fotoni non è rilevato. In questo caso, il fotone rilevato avvia i processi che portano all'inizio del conteggio da parte del contatore; tuttavia, non essendo stato rilevato il secondo fotone non si produce un segnale di \emph{Stop}. Il contatore verrà fermato casualmente da un'annichilazione successiva, che non corrisponde all'evento che ha avviato lo \emph{Start}. Questo tipo di occorrenze introduce una componente continua nel \emph{Time-Spectrum} poiché, in un certo istante, la probabilità di rilevare un evento di \emph{Scatter} è mediamente costante. Questa componente continua è detta \emph{Chance Intervals} poiché il contatore può contare tempi molti lunghi o piccoli in modo causale;
\item
  Gli eventi di tipo \emph{Random} si verificano in presenza di più sorgenti di radioisotopo. Nel corpo umano, una volta immesso il tracciante, nascono degli eventi contemporanei di annichilazione. Ciò dà luogo a più rilevazioni simultanee che introducono una quota di rumore sull'immagine.
\end{itemize}

Il \emph{Time-Spectrum} presenta, quindi, un picco in corrispondenza del ritardo T tra i due canali, dove si verificano il maggior numero di eventi rilevati, che formano il \emph{True} delle coincidenze. A causa degli eventi \emph{Scatter} sono possibili eventi non legati alla stessa annichilazione che determinano ritardi più lunghi o più corti di T con la medesima probabilità. Lo spettro è, quindi, dato da:

\begin{figure}
\centering
\includegraphics[width=6.68472in,height=2.89097in,alt={P4676\#yIS1}]{media/19_Circuiti/image505.pdf}\caption{Figura .: Time-Spectrum con coincidenze casuali}
\end{figure}

Bisogna scartare le false coincidenze dovute agli effetti di \emph{Scattering} che rientrano in un ritardo ammissibile ovvero appartenenti all'area riservata alle \emph{True Coincidence}. È possibile determinare la probabilità con cui si verificano i \emph{Chance Intervals} osservando che la probabilità che sia trascorso un tempo T senza aver rilevato un evento di \emph{Stop} è:

\[P_{T} = e^{- r_{2}T}\ \]

Dove \(r_{2}\) è il \emph{Rate} di intervalli rilevati dal detettore D2.

La probabilità che al tempo T vi sia un segnale di \emph{Stop} è ottenuta moltiplicando la probabilità che sia trascorso un tempo T senza aver rilevato un evento di \emph{Stop} per il \emph{Rate} di intervalli rilevati dal detettore D2:

\[P_{stop} = {r_{2}e}^{- r_{2}T}\]

La probabilità che sia rilevato anche un evento di \emph{Start} è ottenuta moltiplicando la probabilità \(P_{stop}\) per il tasso di eventi rilevati dal primo detettore:

\[P = {r_{1}r_{2}e}^{- r_{2}T}\]

Questa quantità coincide con la probabilità che si verifichi un evento di \emph{Scatter} con ritardo T tra l'evento di \emph{Start} e \emph{Stop} non correlati a un solo evento di annichilazione. Se T è piccolo, essa si riduce al prodotto tra le frequenze medie di rilevazione.

Si dice tasso complessivo di coincidenze casuali come:

\[2\tau r_{1}r_{2}\]

Dato dal prodotto dei due tassi di eventi rilevati di entrambi i detettori per la finestra di tempo \(\tau\) in cui i due eventi sono ritenuti simultanei. Due eventi, infatti, non sono mai veramente simultanei ma tra di loro vi è un certo ritardo dovuto al diverso spazio percorso dai fotoni dal paziente verso i detettori.

Nell'apparecchiatura PET si utilizzano due canali dove in un ramo si trovano un circuito di \emph{Time Pick-Off} per il rilevamento per lo zero e un ritardatore costante. Il secondo ramo, invece, oltre al \emph{Time Pick-Off} presenta un circuito di ritardo variabile.

\begin{figure}
\centering
\includegraphics[width=5.27083in,height=1.68759in,alt={P4690\#yIS1}]{media/19_Circuiti/image506.pdf}\caption{Figura .: Logica con ritardo costante e variabile lungo le linee}
\end{figure}

I due ritardi, posti diversi tra loro, sono necessari per poter stimare il numero delle coincidenze causali che rientrano in quelle ammissibili. A tale scopo si utilizza un'unità di coincidenza che verifica se due eventi rientrano nella finestra temporale di durata \(\tau\).

Tramite l'unità di coincidenza è semplice stimare il tasso di coincidenza casuale e, noto anche il tasso di eventi rilevato, è possibile valutare il tasso di coincidenze vere mediante la sottrazione tra il tasso totale e casuale di coincidenze.

\begin{figure}
\centering
\includegraphics[width=6.01968in,height=7in,alt={P4694\#yIS1}]{media/19_Circuiti/image507.pdf}\caption{Figura .: Time-Spectrum corretto}
\end{figure}

In assenza di correzione, nella ricostruzione delle immagini si considerano anche eventi non legati a un fenomeno di annichilazione.

La finestra di risoluzione \(\tau\) non può essere troppo piccola perché altrimenti si perdono delle coincidenze vere, ma nemmeno troppo grande perché altrimenti si avrebbe una dispersione elevata che comporta il riconoscimento di eventi di \emph{Scatter} come \emph{True}. Tipico valore della finestra è di 12ns.

Dal punto di vista hardware, i circuiti di coincidenza contengono delle porte logiche o sommatori: i due ingressi possono essere sia posti in ingresso a un'unità logica la cui uscita è alta nel momento in cui entrambi gli ingressi sono alti, sia una somma e selezionare una soglia per verificare l'istante in cui si ha la coincidenza.

\begin{figure}
\centering
\includegraphics[width=5.21507in,height=3.40298in,alt={P4699\#yIS1}]{media/19_Circuiti/image508.pdf}\caption{Figura .: Rilevazione della coincidenza digitale}
\end{figure}

Se si utilizza una porta NAND allora quando i due impulsi sono entrambi alti l'uscita è bassa. Dunque, l'evento di coincidenza si verifica quando l'uscita della porta logica è bassa.

Un circuito complessivo per l'acquisizione ed elaborazione del segnale PET è il seguente:

\begin{figure}
\centering
\includegraphics[width=6.61522in,height=3.63542in,alt={P4703\#yIS1}]{media/19_Circuiti/image509.pdf}\caption{Figura .: Circuito per coincidenze e discriminazione energetica}
\end{figure}

L'uscita dei fotomoltiplicatori è posta in ingresso a degli amplificatori. Il sommatore riceve in ingresso tutti i segnali amplificati e li invia a un blocco per la discriminazione dell'istante temporale di occorrenza. In parallelo a questo ramo, vi sono gli integratori con \emph{Reset} che ricevono solamente un canale in uscita dall'amplificatore. Le uscite di questi blocchi sono sommate e posti in ingresso a un discriminatore dell'ampiezza dell'impulso o PHA. Se gli eventi sono simultanei e le energie rientrano nell'intervallo di 350-650keV l'uscita della porta AND è alta e abilita il conteggio del numero di eventi rilevato.

Il circuito prevede anche una sezione in cui si determina la posizione spaziale del fotone \(\gamma\) sulla \(\gamma\)-camera, secondo la logica di Anger.

I segnali risultanti dalle elaborazioni sono poi digitalizzati e inviati a un elaboratore digitale che si occupa della ricostruzione dell'immagine a partire dal numero di fotoni ricevuto e dalla loro posizione spaziale.

\begin{center}
\vfill
    \chapter{Elaborazione dei dati acquisiti in PET}
    \label{blx:refsection\therefsection}
\vfill

\minitoc
\newpage
\end{center}
\justify

\section{Acquisizione dei dati}\label{acquisizione-dei-dati}

Una volta discriminato l'energia dell'impulso e la coincidenza temporale tra due eventi, per ricostruire l'immagine finale è necessario eseguire delle ulteriori elaborazioni sulla modalità di acquisizione dei dati. Per eliminare dai dati utilizzati per la ricostruzione delle immagini i fotoni provenienti da iterazioni con la materia, che aggiungerebbero solo rumore sull'immagine, si ricorre alla collimazione elettronica.

\subsection{Collimazione elettronica: sinogramma}\label{collimazione-elettronica-sinogramma}

Per realizzare un'immagine con buona qualità è necessario acquisire il maggior numero possibile di eventi di annichilazione. Una volta che il positrone si annichila con un elettrone atomico genera una coppia di fotoni \(\gamma\) diretti in qualsiasi direzione dello spazio. Se si utilizza un singolo anello di detettori si rischia di perdere un gran numero di eventi di annichilazione in cui i fotoni \(\gamma\) non si propagano in modo perpendicolare al piano dell'anello. Per tale motivo si realizzano degli scanner PET con più linee di detettori in modo da prelevare anche i fotoni emessi con un certo angolo fino a un valore limite. Questa problematica non si presenta in CT poiché si ha l'emissione di un numero di fotoni X molto più elevato e, quindi, il rumore e le perdite risultano essere contenute.

Il paziente è posto al centro del \emph{Gantry}, nel FOV, dove inizia a emettere fotoni \(\gamma\), alcuni dei quali possono interagire con la materia. Per la presenza dei due fotoni associati a un singolo evento di annichilazione è possibile scartare i fotoni scatterati mediante la collimazione elettronica.

La collimazione elettronica si basa sul concetto di \emph{Line Of Response} o LOR, una linea che unisce due detettori appartenente al FOV. La LOR identifica la linea lungo cui i fotoni provenienti da un evento di annichilazione di un positrone con un elettrone viaggiano verso i detettori. Di conseguenza, se due fotoni colpiscono due detettori posti su una LOR, vi è una certa probabilità che i due eventi appartengano a un processo di annichilazione. Ovviamente esistono delle complicanze, dovute a fenomeni statistici, che possono portare a false letture.

\begin{figure}
\centering
\includegraphics[width=2.49426in,height=2.53913in,alt={P4715\#yIS1}]{media/20_ElaDati/image510.pdf}\caption{Figura .: Esempi di LOR}
\end{figure}

Tramite la definizione di LOR è possibile realizzare un sinogramma nel contesto della PET. Il sinogramma è un grafico cartesiano dove sull'asse delle ordinate vi è l'angolo di inclinazione di una certa LOR rispetto, solitamente, a una linea orizzontale. Ad esempio, nell'immagine precedente, la LOR A identifica l'angolo nullo mentre B e C presentano rispetto alla LOR A un certo angolo via via crescente. La LOR D è perpendicolare alla LOR A e, dunque, nel sinogramma corrisponde al limite superiore dell'asse delle ordinate.

Sull'asse delle ascisse si trova il \emph{Displacement}, ovvero lo scostamento della LOR rispetto al centro del \emph{Gantry}. Ad esempio, la LOR D passa per il centro e, quindi, ha \emph{Displacement} nullo mentre la LOR A presenta la distanza massima rispetto al centro.

\begin{figure}
\centering
\includegraphics[width=6.39097in,height=3.28264in,alt={P4719\#yIS1}]{media/20_ElaDati/image511.pdf}\caption{Figura .: LOR e rispettivo sinogramma}
\end{figure}

Il sinogramma, quindi, instaura una trasformazione geometrica biunivoca tra le varie LOR del \emph{Gantry} e un punto nel piano \emph{Angle-Displacement} che caratterizza l'inclinazione della LOR rispetto l'orizzontale e la distanza dal centro. Noti i due parametri è univocamente determinata la LOR.

Se si considera un punto nel \emph{Gantry} e si tracciano tutte le possibili LOR, si ottiene una curva con un andamento sinusoidale. Da questo risultato discende il nome sinogramma.

La corrispondenza geometrica è fondamentale poiché all'atto dell'acquisizione dei dati, si valuta il numero di eventi che si verificano per ogni LOR. Ogni coppia di detettori, infatti, misura un certo numero di coincidenze diverso da un'altra coppia che individua una LOR diversa.

Siccome i detettori sono in numero finito, non è possibile avere tutti gli angoli e \emph{Dispacement} del piano ma solamente dei punti discretizzati. Il numero di coincidenze è inserito in una struttura dati simile a una matrice, corrispondente al piano discretizzato del sinogramma. Ogni punto del piano \emph{Angle-Displacement} contiene, quindi, informazioni sul numero di conteggi effettuato su una LOR.

Il sinogramma, altro non è un istogramma bidimensionale dove i due punti del piano individuano la LOR mentre sul terzo asse vi è il numero di conteggi, spesso rappresentato sul piano del sinogramma come gradi di grigio.

In presenza di diverse strutture nella sezione del corpo umano, si ottiene un sinogramma più complesso in cui, ad ogni punto dello spazio-immagine, è associata una certa sinusoide poiché attraversato da LOR con angoli e distanze dal centro diverse. Ovviamente LOR parallele generano sinusoidi parallele.

Nell'immagine sottostante si vede una \emph{Slice} del paziente e il relativo sinogramma. Gli agenti con maggior quantità di tracciante sono rappresentati da gradazioni di grigio più scure rispetto a regioni con minor quantità di radiofarmaco.

\begin{figure}
\centering
\includegraphics[width=6.71629in,height=3.55844in,alt={P4728\#yIS1}]{media/20_ElaDati/image512.pdf}\caption{Figura .: Esempio di sinogramma}
\end{figure}

Per costruire un sinogramma si fissa un particolare angolo e si analizza il numero di eventi di coincidenza rilevato da LOR parallele tra loro.

La linea nel sinogramma rappresenta, quindi, la proiezione del paziente lungo l'angolo considerato poiché mostra come il tracciante si distribuisce lungo una linea.

In particolare, nella regione in cui la densità di tracciante è più alta, emergono più coppie di eventi rilevati dai detettori inclinati dell'angolo scelto e, dunque, sono più scure rispetto a regioni con piccole concentrazioni di radio nucleotide.

Il sinogramma è difficilmente interpretabile, quindi, è successivamente elaborato per estrarre l'immagine della \emph{Slice} del corpo.

Una modalità di rappresentazione del sinogramma consiste nel mostrare il diagramma per ogni fetta: si suppone di acquisire le \emph{Slice} del corpo secondo piani di detezione ortogonali al paziente. Rappresentando i dati ottenuti da tutte le \emph{Slice} in termini di \emph{Angle} e \emph{Dispacement} si ottiene un sinogramma del tipo:

\begin{figure}
\centering
\includegraphics[width=3.17532in,height=4.7907in,alt={P4735\#yIS1}]{media/20_ElaDati/image513.pdf}\caption{Figura .: Sinogramma di tutte le fette}
\end{figure}

Considerando invece, una rappresentazione in cui il sinogramma è ristretto a un angolo si ottiene una proiezione di una singola \emph{Slice} del corpo:

\begin{figure}
\centering
\includegraphics[width=4.4672in,height=3.77907in,alt={P4738\#yIS1}]{media/20_ElaDati/image514.pdf}\caption{Figura .: Sinogramma di una fetta fissato l'angolo}
\end{figure}

Visualizzando le varie fette relative a un particolare angolo si riesce a intravedere una possibile struttura anatomica relativa alla \emph{Slice}.

Entrambe le visualizzazioni non sono utilizzate poiché, mediante algoritmi digitali, si ricostruisce il volume tridimensione \emph{Slice} per \emph{Slice} così come avviene in CT.

\subsection{Piani di detezione}\label{piani-di-detezione}

Si suppone di mettere in coincidenza ciascun detettore di un anello con i detettori dello stesso anello. Questa soluzione limita l'acquisizione ai soli eventi che si verificano nei piani trasversi al paziente. Ponendo \(N\) anelli di rilevatori in serie si ottengono \(N\) piani paralleli che sezionano il paziente nel \emph{Ganty} in modo ortogonale all'asse principale.

Le coincidenze avvenute lungo LOR che congiungono anelli differenti non sono rilevate. Di conseguenza, una coppia di fotoni che viaggia trasversalmente al \emph{Gantry} non è rilevata.

Per illustrare il modo in cui i vari detettori di un anello sono correlati con altri detettori di altro si utilizza il michelogramma. Il grafico è costruito riportando sull'asse delle ascisse e delle ordinate i numeri dei detettori di due anelli. Nelle coppie di anelli posti in detezione si inserisce un asterisco.

Nella logica prima citata, il michelogramma mostra una retta diagonale che unisce ogni detettore con un altro dello stesso anello. Questa modalità è la più semplice per rilevare i fotoni \(\gamma\) provenienti da eventi di annichilazione, ma non è l'unica possibile.

\begin{figure}
\centering
\includegraphics[width=5.21875in,height=2.6138in,alt={P4747\#yIS1}]{media/20_ElaDati/image515.pdf}\caption{Figura .: Coincidenze tra i detettori di un anello e michelogramma corrispondente}
\end{figure}

È possibile realizzare la circuiteria di controllo in modo che ogni anello sia in coincidenza anche con gli anelli adiacenti, ad esempio il primo anello è messo in relazione col secondo e con se stesso e così via. Ciò permette di individuare degli ulteriori piani obliqui, nella direzione di un anello il successivo, rispetto alla direzione longitudinali su cui poter ricavare le proiezioni dei radionuclidi. Se si pongono \(N\) anelli con questa logica di coincidenza si individuano \(N(N - 1)\) piani di detezione ovvero si ha una maggiore copertura del paziente.

\begin{figure}
\centering
\includegraphics[width=6.08333in,height=3.09653in,alt={P4750\#yIS1}]{media/20_ElaDati/image516.pdf}\caption{Figura .: Coincidenze tra i detettori di due anelli adiacenti e michelogramma corrispondente}
\end{figure}

La maggior copertura degli anelli di detezione comporta dei circuiti di coincidenza tra i due anelli. Se per un due detettori si hanno un centinaio di componenti elettronici da utilizzare, per mettere in relazione tutti i detettori di un anello in coincidenza bisogna utilizzare circa:

\[\frac{100 \cdot 99}{2}\sim 5000\]

Nei circuiti di coincidenza. La relazione tra gli \(N\) detettori di un anello con quello successivo richiede circa:

\[5000 \cdot 2N\]

Il costo complessivo dell'apparecchiatura aumenta notevolmente poiché aumentano i costi associati alla circuiteria di controllo e coincidenza. I costruttori possono prevedere anche quattro anelli in congiunzione, raddoppiando il numero degli elementi circuitali necessari alla realizzazione dello scanner.

Iterando il ragionamento delle coincidenze è possibile mettere in relazione ogni detettore con tutti gli altri così da poter ottenere la copertura massima del paziente. A questa soluzione corrisponde il maggior numero di eventi rilevati poiché i fotoni emergenti in modo isotropo possono essere intercettati da quasi tutte le direzioni oblique rispetto all'asse del \emph{Gantry}.

Tuttavia, questa soluzione porta a una complessità circuitale e computazionale degli algoritmi di ricostruzione che aumentano notevolmente il costo totale dell'apparecchiatura. In ogni caso la ricostruzione dell'immagine risulta essere molto più accurata per l'elevato numero di eventi valutati.

Questa configurazione è tipica delle apparecchiature più recenti poiché i costi dell'elettronica e della fabbricazione sono stati ridotti negli ultimi anni. Ciò ha reso possibile congiungere ogni anello con tutti gli altri del \emph{Gantry}.

\begin{figure}
\centering
\includegraphics[width=5.32836in,height=2.73142in,alt={P4760\#yIS1}]{media/20_ElaDati/image517.pdf}\caption{Figura .: Coincidenze tra i detettori di tutti gli anelli e michelogramma corrispondente}
\end{figure}

Nelle applicazioni pratiche, non sempre è richiesta la massima copertura possibile del paziente, ad esempio, ci sono casi in cui l'esame radiologico richiede un \emph{Imaging} bidimensionale basato su alcune fette particolari del paziente. A tale scopo i costruttori prevedono dei setti interplanari, realizzati con materiale pesante assorbente le radiazioni \(\gamma\), evitando il rilevamento di fotoni che prevengono all'esterno di una \emph{Slice}. I setti, generalmente in piombo o tungsteno, sono, quindi, posti in modo da intercettare e assorbire tutti i fotoni che non appartengono alla \emph{Slice} di interesse. Questa soluzione, se da un lato permette di ridurre il rumore sovrapposto all'immagine poiché gli eventi indesiderati sono eliminati, dall'altro presenta lo svantaggio di ridurre proprio il numero totale di eventi rilevati e ciò comporta una minor qualità dell'immagine.

I setti interplanari sono configurati dal radiologo mediante il tavolo di comando: questi elementi sono retraibili a seconda del tipo di \emph{Imaging} da eseguire sul paziente per effettuare la diagnosi del suo stato di salute.

\begin{figure}
\centering
\includegraphics[width=3.47761in,height=3.37176in,alt={P4764\#yIS1}]{media/20_ElaDati/image518.pdf}\caption{Figura .: Presenza dei setti interplanari}
\end{figure}

\subsection{Sensibilità}\label{sensibilituxe0}

La sensibilità è un parametro molto importante legato al numero di eventi contati. Intuitivamente è semplice rendersi conto che i fotoni dalla zona centrale del paziente possono essere intercettati con maggior probabilità. Infatti, un fotone emergente dalla regione centrale del \emph{Gatry} può essere rilevato da uno qualsiasi degli anelli, se questi sono messi tutti in coincidenza.

Se l'evento di annichilazione, invece, si verificare nelle zone periferiche del paziente, verso la pelle, i fotoni prodotti hanno maggior probabilità di essere rilevati dagli anelli posti in periferia o quello immediatamente in prossimità.

È possibile concludere che l'intero processo di acquisizione dei dati al centro del paziente è più sensibile rispetto alla periferia del corpo. Riportando su un diagramma la sensibilità in funzione della posizione assiale si ottiene un grafico diverso in base al tipo di acquisizione:

\includegraphics[width=4.67639in,height=5.86736in,alt={P4770\#y1}]{media/20_ElaDati/image519.pdf}
Per un'acquisizione 3D, invece, bisogna fornire il \emph{Ring Difference} (\(rd\)), cioè il numero di anelli messi in congiunzione. Se, ad esempio, 11 su 16 anelli sono tra loro in coincidenza, la regione centrale presenta una sensibilità molto elevata mentre per le regioni periferiche essa decresce dato che le coppie di fotoni, provenienti dalle zone periferiche dal centro del \emph{Gantry}, sono rilevabili solo da pochi anelli. La sensibilità in funzione della posizione assiale presenta una forma del tipo trapezoidale.

Figura .: Andamento della sensibilità per varie coincidenze

Nel caso in cui tutti gli anelli sono messi in congiunzione tra loro, il punto al centro del \emph{Gantry} presenta la massima sensibilità di rilevazioni. Progressivamente, tutte le altre posizioni all'interno del \emph{Gantry}, all'allontanarsi dal centro, presenteranno un valore di sensibilità sempre più ridotto. La sensibilità in funzione della posizione assiale ha un andamento di tipo triangolare.

In definitiva, quindi, all'aumentare del numero di anelli in congiunzione, si riduce l'ampiezza della regione in cui la sensibilità può essere considerata costante. Quindi, per avere il maggior numero di conteggi e la sensibilità più costante possibile, è necessario spostare il lettino porta-paziente così da cambiare il distretto anatomico situato al centro del \emph{Gantry}.

Il lettino è spostato in modo tale che la sensibilità delle aree laterali, non situate mai al centro del \emph{Gantry} si sommino. Il risultato è una sensibilità sufficientemente costante sulla maggior parte del corpo del paziente. Questo processo è essenziale per poter eseguire una PET \emph{Total Body} dove, per ricostruire allo stesso modo tutte le regioni del corpo, è essenziale il movimento del lettino porta-paziente all'interno del \emph{Gantry}.

Se il lettino non fosse mosso durante l'esame strumentale, le regioni centrali del corpo risulterebbero meglio ricostruite di altre poiché il numero dei fotoni rilevati è maggiore proprio in virtù della maggiore sensibilità.

Prima di ricostruzione l'immagine, è necessario notare che l'acquisizione dei dati è influenzata da una serie di fattori quali:

\begin{itemize}
\item
  La normalizzazione, dovuta alle disuguaglianze tra coppie di detettori;
\item
  L'attenuazione dei fotoni, a causa delle possibili interazioni con la materia;
\item
  Coincidenze \emph{Random};
\item
  Coincidenze di \emph{Scatter};
\item
  \emph{Deat Time};
\item
  \emph{Radial Elongation}.
\end{itemize}

\subsubsection{Normalizzazione}\label{normalizzazione}

I moderni scanner PET presentano dai 10000 a 20000 detettori e centinaia di fotomoltiplicatori, quindi, l'architettura hardware è estremamente complessa.

A questi numeri bisogna aggiungere la replicazione della circuiteria di amplificazione e rilevazione delle coincidenze per ogni coppia di detettori secondo il michelogramma.

Ovviamente l'elettronica di controllo, prelievo e del cristallo scintillatore presentano delle differenze che determinano la variazione della sensibilità di una coppia di detettori. Inoltre, i parametri con cui la circuiteria analogica è realizzata dipendono dall'invecchiamento dei componenti o dall'aumento di temperatura locale, che potrebbe modificare il guadagno solo per un numero finito di detettori.

Sono presenti altri fenomeni quali la \emph{Dark Current}, anch'essa dipendente dalla temperatura, e le variazioni del fattore di emissione secondario.

Il blocco di scintillazione presenta delle dimensioni abbasta grandi, dell'ordine di 4cm x 4cm x 4cm, quindi, è molto complesso realizzare un drogaggio perfettamente omogeneo. Esisterà, quindi, una certa distribuzione del drogante all'interno della struttura cristallina. Ne discende che i livelli energetici di scintillazione presentano una certa distribuzione all'interno del cristallo, sicuramente diversa da un altro cristallo dello stesso macchinario.

Di conseguenza, la sensibilità tra varie coppie di detettori dipende, oltre che dalla posizione rispetto il \emph{Gantry}, anche dal guadagno dei PMT e dalla variabilità fisica e costruttiva dei detettori stessi e dei cristalli scintillatori. Il risultato finale è una non uniformità dei dati acquisiti dovuti a coppie più sensibile di altre e viceversa. Sistematicamente, per ragioni costruttive, alcuni rilevatori captano più fotoni rispetto ad altri.

Per compensare la disomogeneità dei dati si esegue una procedura di normalizzazione, applicata, di solito, con una cadenza temporale piuttosto spinta, come una volta a settimana di notte. Ciò è essenziale per evitare che le variazioni parametriche, che avvengono per l'invecchiamento dell'elettronica, possano pesare sui dati acquisiti e, in definitiva, sull'immagine ricostruita.

La procedura di normalizzazione è normata da norme europee recepite dall'ente CEI in Italia ed è eseguita esponendo uniformemente tutti i detettori a una sorgente del radionuclide Germanio-68 (\(_{}^{68}{Ge}\ \)) che emette fotoni di energia uguale a 511keV. La sorgente è posizionata al centro del \emph{Gantry}, in assenza di paziente, così da eccitare tutti i detettori allo stesso modo.

La procedura di irradiazione può durare dalle 6 alle 8 ore, in base alla sorgente, così da poter acquisire un numero statisticamente significativo di fotoni. Per tale motivo, solitamente, la normalizzazione è eseguita di notte.

Una volta determinato il sinogramma dei fotoni rilevati durante il periodo di esposizione, si raccolgono i dati sia 2D che 3D e, a partire dai quali, si determinano i coefficienti di normalizzazione. In particolare, per ogni coppia di detettori in coincidenza, si calcola il fattore di normalizzazione come:

\[F_{i} = \frac{A_{mean}}{A_{i}}\]

Dove \(A_{mean}\) è il conteggio medio delle coincidenze di tutte le LOR, noto dal sinogramma, mentre \(A_{i}\) è il conteggio delle coincidenze sulla LOR i-esima che unisce la coppia di detettori considerati.

Il fattore di normalizzazione permette di confrontare le prestazioni di una coppia di detettori con le prestazioni media. Se, infatti, la coppia di detettori è più sensibile rispetta alla media risulta che \({A_{mean} < A_{i} \Leftrightarrow \ F}_{i} < 1\), viceversa, se è meno sensibile \({A_{mean} > A_{i} \Leftrightarrow \ F}_{i} > 1\).

Quando si esegue l'acquisizione del sinogramma sul paziente nella \emph{Routine} clinica, la i-esima coppia di detettori rileva un conteggio di coincidenze \(C_{i}\) da correggere mediante il fattore di normalizzazione \(F_{i}\) nel seguente modo:

\[C_{norm,i} = {C_{i}F}_{i}\]

Moltiplicando il conteggio della i-esima LOR per il fattore di normalizzazione, si compensa la maggiore o minore sensibilità della coppia di detettori rispetto alla media.

Grazie ai conteggi normalizzati, si riesce a ricostruire delle immagini con qualità molto migliore rispetto all'assenza di normalizzazione poiché quest'ultime, presenterebbero degli artefatti legati alla maggiore o minore intensità del sinogramma non per il diverso numero di conteggi ma per la diversa sensibilità delle coppie rilevatori.

\subsubsection{\texorpdfstring{Attenuazione dei fotoni \(\gamma\)}{Attenuazione dei fotoni γ}}\label{attenuazione-dei-fotoni-ux3b3}

I fotoni \(\gamma\), prodotti da un processo di annichilazione, sono attenuati nei vari tessuti incontrati poiché possono interagire con la materia per effetto Compton o fotoelettrico.

Con riferimento a un materiale omogeneo, la probabilità che un fotone emerga e che sia rilevato da un detettore è:

\[P = e^{- \mu d}\]

Dove \(\mu\) è il coefficiente di attenuazione lineare, mentre \(d\) la distanza percorsa da un fotone nella materia.

Un evento di annichilazione di positrone con un elettrone produce due fotoni di energia uguale a 511keV che viaggiano in direzione opposte. Il rilevamento di un fotone \(\gamma\) è indipendente dalla cattura del secondo fotone, poiché se un fotone è assorbito dalla materia non è detto che anche l'altro interagisca con essa. La probabilità di intercettare entrambi i fotoni è, quindi, ottenuta come prodotto delle probabilità di intercettare i fotoni:

\[P = e^{- \mu d_{1}}e^{- \mu d_{2}}\]

Con \(d_{1}\) percorso compiuto da un fotone e \(d_{2}\) il cammino nella materia dell'altro. Sia \(D\) lo spessore del corpo, la probabilità di rilevare entrambe i fotoni può essere scritta come:

\[P = e^{- \mu D}\]

Esistono delle zone del corpo da cui i fotoni emergono con maggiore probabilità e sono, quindi, maggiormente intercettati dai detettori. In altre zone del corpo umano, invece, vi è una maggiore probabilità di assorbimento e, di conseguenza, una minore probabilità di essere intercettati dai rilevatori.

Questa differenza di emissione potrebbe creare degli artefatti nell'immagine ricostruita poiché le zone con maggior probabilità di emissione dei fotoni risulteranno avere una maggiore concentrazione di tracciante rispetto alle zone con minor emissione. L'artefatto determina la rappresentazione della distribuzione dell'attività del radionuclide con maggiore intensità luminosa nelle regioni con maggior probabilità di emissione dei fotoni e, quindi, non direttamente collegata alla maggiore o minore attività.

La descrizione matematica della probabilità di emissione di entrambi i fotoni nei mezzi eterogeni come il corpo del paziente richiede l'utilizzo della nozione di integrale:

\[P = P_{1}P_{2} = e^{- \int_{a}^{x}{\mu ds}}e^{- \int_{x}^{b}{\mu ds}}\]

Dove \(x\) è il punto in cui avviene l'annichilazione mentre \(a\) e \(b\) sono gli estremi del segmento che congiunge il punto di annichilazione con i detettori. La probabilità può essere scritta come esponenziale dell'integrale di linea del coefficiente di assorbimento del corpo del paziente:

\[P = e^{- \int_{a}^{b}{\mu\left( \mathbf{r} \right)ds}}\]

L'integrale è ovviamente esteso alla LOR che congiunge i due detettori.

Dal numero di conteggi rilevati dalle varie LOR e nota la probabilità che quella LOR possa intercettare i fotoni in base ai tessuti incontrati, è possibile correggere l'immagine tenendo conto dell'attenuazione, che contribuisce al rumore sovrapposto all'immagine.

Si osservi che la probabilità che un fotone sia assorbito dipende dal materiale incontrato e dal percorso che esso compie all'interno del corpo umano. Così i fotoni generati nelle regioni più interne del corpo umano hanno una maggiore probabilità di essere attenuati rispetto a quelli emessi in prossimità della superficie corporea o nei polmoni poiché il cammino compiuto ha durata minore e i tessuti incontrati sono meno densi.

L'immagine risultate, quindi, mostra una maggiore attività apparente in prossimità della pelle e nei polmoni. Questi artefatti devono essere compensati per ottenere delle immagini che mostrino solamente la distribuzione dell'attività effettiva nel corpo del paziente.

Per tale motivo è nata la PET/CT ovvero un'apparecchiatura che, all'interno dello stesso \emph{Gantry}, contiene sia il tubo radiogeno che la strumentazione necessaria per la rilevazione degli eventi di annichilazione. La PET/CT non nasce per ottenere delle immagini CT con informazioni di carattere funzionale, ma per correggere gli artefatti dell'immagine PET.

Dal punto di vista tecnologico il \emph{Gantry} della PET/CT è composto da due apparecchiature in linea l'una con l'altra. Si presentano ovviamente sia le problematiche della CT, legate al tubo rotante e all'alimentazione di circa 100kV, sia della PET, dovuta ai detettori posti ad anello e all'alimentazione a 1kV circa dei fotomoltiplicatori.

Il protocollo clinico prevede prima l'acquisizione delle immagini CT, in assenza di liquido di contrasto, poiché con le CT spirali si riesce a eseguire l'\emph{Imaging Total Body} in pochi secondi, e successivamente si procede con l'esame PET dalla durata di 30-45min per il \emph{Total Body}.

Gli elevati tempi di scansione sono dovuti ai tempi di decadimento del tracciante, dell'ordine di un paio d'ore.

Se il centro radiologico acquista il tracciante dall'esterno, il tempo per eseguire l'esame strumentale PET aumenta perché l'attività del radio-tracciante è ridotta, quindi, è necessario un maggior tempo per rilevare un numero di conteggi sufficienti alla ricostruzione dell'immagine.

\begin{figure}
\centering
\includegraphics[width=4.80088in,height=2.06944in,alt={P4825\#yIS1}]{media/20_ElaDati/image520.pdf}\caption{Figura .: Schema di PET/CT}
\end{figure}

Storicamente la PET nasce come strumentazione a sé, molto utilizzata nel campo dell'analisi cerebrale. Con l'avvento dell'fMRI la macchina PET si è spostata sulle applicazioni oncologiche.

Nelle immagini PET la morfologia dell'organo non risulta essere ben evidenziata, mentre la componente funzionale è ben visibile, sebbene sia difficile comprendere il distretto anatomico. La logica di rappresentazione delle immagini è inversa rispetto alla CT: le porzioni che possiedono un'elevata attività sono rappresentare col nero, mentre le zone con minor attività col bianco. Le sole immagini PET sono difficilmente analizzabili e presentano gli errori dovuti alla probabilità di attenuazione dei fotoni \(\gamma\).

Per correggere gli artefatti e, allo stesso tempo, ottenere immagini di carattere morfo-funzionale si ricorre al PET/CT. In assenza di correzione, si osserva una maggiore attività sulla superficie del corpo poiché il fotone \(\gamma\), dovendo attraversare una minor quantità di tessuto se emerge tangenzialmente, presenta una maggior probabilità di essere rilevato. Le correzioni introdotte dalle immagini CT riguardano, dunque, prevalentemente la superficie del corpo e i polmoni.

\begin{figure}
\centering
\includegraphics[width=5.35387in,height=3.46296in,alt={P4830\#yIS1}]{media/20_ElaDati/image521.pdf}\caption{Figura .: In alto immagini non corrette, in basso immagini corrette con la CT}
\end{figure}

La CT permette di ottenere delle immagini che mostrano come varia il coefficiente di attenuazione lineare del corpo del paziente. Dalle immagini morfologiche della CT è, inoltre, possibile ricavare le distanze \(D\) che i vari fotoni percorrono all'interno del corpo umano.

Con le due informazioni delle immagini CT, cioè distanza e coefficiente di assorbimento, è possibile stimare la probabilità con cui i fotoni sono attenuati all'intero dell'organismo e, tramite questa conoscenza, si operano le correzioni.

Nello specifico, le immagini CT permettono di valutare l'integrale di linea del coefficiente di assorbimento lineare lungo una LOR.

\[\int_{a}^{b}{\mu\left( \mathbf{r} \right)ds}\]

Il coefficiente di assorbimento lineare dipende anche dell'energia della radiazione assorbita. Dunque, il coefficiente \(\mu\) ricostruito con la CT non coincide, numericamente, con quello presente nella probabilità che i due fotoni \(\gamma\) siano rilevati.

In radiologia convenzionale, i raggi X presentano energia dell'ordine di 80-120keV mentre per la PET i fotoni \(\gamma\) hanno energia di 511keV e, quindi, i coefficienti di attenuazione possono essere anche molto differenti tra loro.

L'immagine CT non può essere utilizzata per la correzione dell'immagine PET senza applicare una procedura di elaborazione che permetta di stimare i coefficienti di attenuazione lineare alle energie della PET a partire dalle misurazioni ottenute con le energie della radiologia convenzionale.

La correzione richiede delle segmentazioni automatiche, ovvero degli algoritmi che automaticamente dividono l'immagine CT in tessuti omogenei (in gergo si dice che le immagini sono segmentate).

Questa operazione risulta essere abbastanza complessa da realizzare nella pratica e può indurre una serie di artefatti nell'immagine ricostruita a causa di una segmentazione non perfetta.

Sulla base della conoscenza dell'attenuazione di quel tessuto omogeneo alle energie della CT si riesce a determinare il coefficiente di attenuazione alle energie della PET mediante operazioni di prolungamento.

Questo processo è possibile poiché le curve del coefficiente di attenuazione massico sono state determinate sperimentalmente e pubblicate sulla letteratura scientifica.

Per la valutazione della probabilità che entrambi i fotoni siano rilevati, le case produttrici degli scanner PET/CT introducono una serie di algoritmi proprietario le cui linee generali ricalcano la segmentazione automatica dell'immagine CT e il ricalco del coefficiente di attenuazione.

Confrontando le immagini ottenute con la CT, la PET e la PET/CT è possibile analizzare le differenze e migliorie introdotte dall'ultima metodica di \emph{Imaging}:

\begin{figure}
\centering
\includegraphics[width=6.25552in,height=4.07407in,alt={P4845\#yIS1}]{media/20_ElaDati/image522.pdf}\caption{Figura .: Rispettivamente immagini CT, PET e PET/CT}
\end{figure}

Le immagini PET/CT sono mostrate in pseudo-colori indicanti l'attività nel corpo del paziente. Solitamente si sceglie la mappa termica dove il blu indica assenza di attività mentre il rosso la massima attività.

Con la CT gli organi molli sono molto pochi contrastati ma con riferimenti anatomici molto più evidenti rispetto a un'immagine PET. La fusione delle due immagini porta a risultati con informazioni morfologiche e funzionali corrette.

\subsubsection{False coincidenze}\label{false-coincidenze}

Le coincidenze possono essere \emph{True}, \emph{Scatter} o multiple:

\begin{itemize}
\item
  \emph{True Coincidences} in cui l'evento di annichilazione è rilevato mediante la cattura di entrambi i fotoni emergenti sulla LOR percorsa;
\item
  Le coincidenze di tipo \emph{Scatter} possono verificarsi quando, in un evento di annichilazione, un fotone è deviato per le interazioni con la materia. Se esso non riduce la sua energia al di sotto della soglia di rilevazione, i due fotoni sono rilevati su una LOR diverse da quella originale. Si produce così una LOR apparente che introduce una quota di rumore sull'immagine;
\item
  Nelle coincidenze \emph{Random} due eventi di annichilazione avvengono quasi simultaneamente e, per caso, due fotoni delle due diverse coppie di fotoni sono assorbiti dalla materia. I restanti due fotoni viaggiano lungo le rispettive LOR fino ad essere rilavati dai detettori come eventi contemporanei poiché giungo sull'anello con un ritardo ammissibile. La logica di controllo non riconosce l'evento di annichilazione su una delle due LOR effettiva ma su una terza, intermedia tra le due reali. In altre parole, invece di riconoscere i due eventi, si riconosce un solo evento in una posizione diversa da quella effettiva, mostrando un'attività in un distretto anatomico anche molto diverso da quelli originale. Mentre l'evento \emph{Scatter} può essere scartato sulla base della discriminazione energetica se il fotone \(\gamma\) riduce la sua energia al di sotto della soglia di rilevazione per effetto Compton, l'evento \emph{Random} non può essere rilevato poiché i fotoni giungono sull'anello di rilevatori con la giusta energia.
\end{itemize}

La correzione degli eventi \emph{Random} può essere eseguita essendoci la possibilità di stimare il numero di queste coincidenze che possono verificarsi all'interno dello scanner noti i tassi di conteggio su due detettori:

\[2\tau r_{1}r_{2}\]

Non è, quindi, possibile determinare se un evento rilevato sia da associare a una coincidenza \emph{True} oppure \emph{Random}, ma è possibile conoscere il numero medio degli eventi casuali;

\begin{itemize}
\item
  Le coincidenze multiple sono più rare ma sono molto più complesse da determinare. Esse si verificano quando due eventi di annichilazione occorrono all'incirca nello stesso momento e un fotone di una coppia è attenuato dalla materia. Sul detettore giungono, quindi, tre fotoni in coincidenza temporale che individuerebbero tre LOR. Una di queste può essere scartata poiché congiunge due detettori secondo una linea non appratente per il FOV. Viceversa, la LOR reale e quella appratente sono entrambe plausibili, quindi, il sistema elettronico potrebbe scegliere di scartare una delle due oppure conteggiarle entrambe. In ogni caso si generano degli errori nella ricostruzione delle immagini.
\end{itemize}

\begin{figure}
\centering
\includegraphics[width=4.83666in,height=4.47222in,alt={P4858\#yIS1}]{media/20_ElaDati/image523.pdf}\caption{Figura .: Possibili coincidenze}
\end{figure}

\subsubsection{Distorsioni geometriche}\label{distorsioni-geometriche}

Prima di poter ricostruire l'immagine PET è necessario tener conto anche delle distorsioni geometriche introdotte da come sono disposti i detettori nel \emph{Gantry}. Generalmente nel \emph{Gantry} i detettori sono distribuiti uniformemente lungo l'anello. Di conseguenza, le LOR al centro dell'anello presentano una certa spaziatura tra di loro, che si riduce all'allontanarsi dal centro. Vi è, dunque, un certo grado di disomogeneità lungo tutto il raggio dell'anello.

La distorsione generata è nota come effetto ad arco e deve essere compensata per ricostruire al meglio le immagini. Questi errori sono più visibili per pazienti molto massicci poiché rientrano in un maggior numero di zone con diversa spaziatura tra detettori a cui corrisponde una diversa risoluzione spaziale.

Oltre all'architettura ad anello sono possibili altre geometrie per l'arrangiamento dei detettori come, ad esempio, a semianello rotante, \emph{array} di detettori lineari disposti a esagono o una coppia \emph{array} paralleli tra loro. Per le due geometrie aperte, il \emph{Gantry} deve essere messo in rotazione per rilevare quanti più eventi possibili.

\begin{figure}
\centering
\includegraphics[width=6.29975in,height=5.79167in,alt={P4864\#yIS1}]{media/20_ElaDati/image524.pdf}\caption{Figura .: Possibili Gantry con spaziature tra i detettori}
\end{figure}

Le soluzioni aperte sono state le prime ad essere realizzate poiché permettono di abbattere i costi della circuiteria elettronica di controllo. Al giorno d'oggi, la maggior parte degli scanner PET prevede un'architettura ad anello di detettori fisso intorno al paziente, situato nel FOV.

\subsubsection{Radial Elongation}\label{radial-elongation}

Un altro errore introdotto nell'acquisizione dei dati è dovuto alla regione in cui il fotone è assorbito all'interno del cristallo scintillatore con uno spessore di 3-4cm. Nei due casi limiti:

\begin{itemize}
\item
  Il fotone \(\gamma\) incide sul detettore ed è subito convertito in radiazione luminosa;
\item
  Oppure lo stesso fotone può essere assorbito alla fine del cristallo scintillatore.
\end{itemize}

Sulla base della posizione lungo lo spessore del cristallo scintillatore in corrispondenza della quale il fotone \(\gamma\) è assorbito, possono essere identificate delle LOR diversa, spaziate di una certa quantità \(d\). Ovviamente, non è possibile conoscere la posizione dell'evento di annichilazione a causa della diversa posizione di assorbimento del fotone \(\gamma\) incidente nel cristallo scintillatore. Quindi, l'algoritmo di elaborazione ricostruisce l'immagine considerando una LOR media individuata dalla stessa coppia di detettori. Ciò introduce una certa indeterminazione sulla posizione di annichilazione e, di conseguenza, delle maggiori dimensioni del voxel.

\begin{figure}
\centering
\includegraphics[width=4.69828in,height=4.14583in,alt={P4872\#yIS1}]{media/20_ElaDati/image525.pdf}\caption{Figura .: Indeterminazione sulla LOR per diversa posizione di assorbimento nel cristallo}
\end{figure}

Tutti gli errori che affliggono i dati misurati si cumulano e fanno sì che il voxel non possa avere dimensioni inferiori di un certo limite, generalmente di 6mm. Le immagini PET sono molto meno definite e contrastate rispetto alle immagini di Risonanza Magnetica che presentano un voxel di circa 1mm e alle immagini della CT, con un voxel di 1mm o inferiore. La risoluzione spaziale della PET è, quindi, intrinsecamente molto limitata.

Per ottenere un'immagine più accurata sono applicate, simultaneamente, la correzione dell'attenuazione, la normalizzazione, il filtraggio e infine si procede con la ricostruzione dell'immagine.

\begin{figure}
\centering
\includegraphics[width=6.50733in,height=7.77083in,alt={P4876\#yIS1}]{media/20_ElaDati/image526.pdf}\caption{Figura .: Processo di correzione delle immagini}
\end{figure}

\subsection{Considerazioni su FOV e michelogramma}\label{considerazioni-su-fov-e-michelogramma}

Dal punto di vista geometrico è utile introdurre il FOV o \emph{Field Of View} come l'angolo limite per accettare gli eventi di coincidenza per ogni singolo detettore.

Non ha, quindi, senso mettere in coincidenza ogni detettore con tutti gli altri presenti nell'anello poiché è altamente improbabile che delle coincidenze \emph{True} si verifichino all'esterno del paziente situato nel FOV. In questo modo è possibile ottenere un risparmio della circuiteria di controllo poiché non ha senso mettere in coincidenza due detettori connessi da una LOR che non passa per il paziente.

Ogni detettore è in coincidenza con il semianello corrispondente così che le LOR prodotte coprano solamente tutto o parte del FOV. Quindi, sulla base della scelta progettuale del FOV si eseguono le opportune coincidenze tra i detettori dello stesso anello.

\begin{figure}
\centering
\includegraphics[width=6.69583in,height=3.01667in,alt={P4882\#yIS1}]{media/20_ElaDati/image527.pdf}\caption{Figura .: Coincidenze tra i detettori di un anello}
\end{figure}

La scelta di mettere in comunicazione solamente i detettori di uno stesso anello che coprono il FOV permette di abbattere i costi dell'apparecchiatura poiché si riducono i componenti elettronici necessari per realizzare lo scanner.

\subsection{Malfunzionamento degli anelli detettori}\label{malfunzionamento-degli-anelli-detettori}

Un'altra correzione può riguardare il malfunzionamento di uno dei due recettori posti in coincidenza. Tale malfunzionamento può essere evidenziato e corretto osservando che tutte le LOR che fanno capo a un singolo detettore sono allocate su una diagonale del sinogramma.

Si acquisisce, quindi, il sinogramma posizionando un \emph{Phantom} con un coefficiente di assorbimento lineare costante sorgente di raggi \(\gamma\). Si analizza poi il sinogramma e se si evidenziano delle linee nere allora si è in presenza di un malfunzionamento e dalla sua posizione si risale alla coppia di detettori guasti.

\begin{center}
\vfill
    \chapter{Ricostruzioni delle immagini PET}
    \label{blx:RicImmPET\therefsection}
\vfill

\minitoc
\newpage
\end{center}
\justify

\section{Ricostruzione delle immagini PET}\label{ricostruzione-delle-immagini-pet}

I metodi di ricostruzione delle immagini possono essere classificati essenzialmente in due categorie:

\begin{itemize}
\item
  I metodi basati sulla trasformata di Radon sono ottenuti mediante retroproiezione filtrata. Questo meccanismo di ricostruzione, pur essendo quello storicamente più utilizzato, non presenta le migliori caratteristiche di ricostruzione. Ad esempio, la quantità di fotoni rilevati, quindi, l'intensità del sinogramma, sono molto limitate rispetto alle intensità di fotoni in radiologia convenzionale. Quindi, gli algoritmi della CT non offrono le stesse prestazioni in PET;
\item
  In PET sono stati predisposti altri algoritmi alternativi basati sulla ricostruzione iterativa che presentano una ricostruzione molto migliore di quelli basati sulla trasformata di Radon.
\end{itemize}

La qualità di un'immagine può essere valutata mediante apposite procedure statistiche in cui sono mostrate le stesse immagini a diversi radiologi in vari momenti della giornata.

Lo stesso radiologo, infatti, può eseguire due refertazioni diverse della stessa immagine in diversi momenti della giornata poiché la lettura delle immagini radiografiche richiede una certa dose di attenzione che si riduce con la stanchezza.

Questa caratteristica è detta variabilità intrasoggettiva poiché si verifica all'interno dello stesso soggetto a seconda della sua stanchezza. Mostrare la stessa immagine in momenti diversi del giorno consente di superare questa variabilità. Le analisi statistiche su più radiologi diversi permettono poi di superare anche la variabilità intersoggettiva tra i diversi radiologi.

Dalle valutazioni statistiche si ricava un indice numerico che quantifica la qualità di un algoritmo e la bontà con cui ricostruisce le immagini.

\subsection{Assorbimento dei fotoni}\label{assorbimento-dei-fotoni}

Sia \(N(a)\) il numero dei fotoni emessi da un punto di ascissa \(a\) su una LOR congiungente due detettori e appartenente al FOV. I fotoni emessi dall'evento di annichilazione incontrano un coefficiente di assorbimento lineare \(\mu(s)\) dove \(s\) è il punto della linea attraversata dal fotone, ovvero un punto sulla LOR.

Il numero dei fotoni emergenti \(dN\) dopo aver attraversato uno strato di materia \(ds\) centrato nel punto \((a,s)\) sufficientemente piccolo tale da poter considerare il materiale omogeneo, è proporzionale, secondo la legge di Lamber-Beer, al numero di fotoni che attraversa lo spessore di materiale per il suo coefficiente di attenzione, ovvero:

\[dN = - \mu(s)N(a)ds\]

Integrando l'equazione lungo la LOR individuata si ottiene:

\[N(x) = N(a)e^{- \int_{a}^{x}{\mu(s)ds}}\]

Il numero di fotoni che giungono sul detettore, distante \(d_{2}\) dal punto di annichilazione, è dato dal numero di fotoni che non sono attenuati dalla materia:

\[N\left( d_{2} \right) = N(a)e^{- \int_{a}^{d_{2}}{\mu(s)ds}}\]

L'esponenziale, dipendente dal coefficiente di assorbimento lineare, rappresenta la probabilità che il fotone sia assorbito lungo il percorso tra il punto di annichilazione e il detettore lungo la LOR:

\begin{figure}
\centering
\includegraphics[width=5.4533in,height=1.40278in,alt={P4905\#yIS1}]{media/21_RicImmPET/image528.pdf}\caption{Figura .: Distanze percorse dai due fotoni emergenti dal sito di annichilazione}
\end{figure}

La logica di controllo della PET impone che per ricostruire il punto di annichilazione, e, quindi, l'immagine, è necessario rilevare entrambi i fotoni prodotti dall'annichilazione del positrone con un elettrone della materia. Va considerata anche la probabilità che il secondo fotone, percorrendo una distanza \(d_{1}\), raggiunga il secondo detettore. Scelto il verso di percorrenza positivo che va dal punto \(a\) verso il primo detettore allora si ha:

\[N\left( d_{1} \right) = N(a)e^{- \int_{d_{1}}^{a}{\mu(s)ds}}\]

Il numero dei fotoni che sono rilevati contemporaneamente è dato dal prodotto dei numeri di fotoni rilevati su ciascun detettore poiché il rilevamento di uno è indipendente dall'arrivo dell'altro sul secondo detettore:

\[N\left( d_{1},d_{2} \right) = N(a)e^{- \int_{d_{1}}^{a}{\mu(s)ds}}e^{- \int_{a}^{d_{2}}{\mu(s)ds}} = N(a)e^{- \int_{d_{1}}^{d_{2}}{\mu(s)ds}}\]

Il numero di coincidenze dovuto agli eventi di annichilazione dipende dall'esponenziale dell'integrale di linea del coefficiente di attenuazione lungo la LOR che congiunge i due detettori:

\[\int_{d_{1}}^{d_{2}}{\mu(s)ds}\]

Per ogni punto \(a\) della LOR, esiste una sorgente puntiforme che emette \(N(a)\) fotoni a loro volta attenuati nel percorso tra la sorgente e i detettori. Il numero di fotoni emessi dalla sorgente dipende dalla quantità di tracciante lungo la LOR \(\lambda(s)\) per lo spessore infinitesimo della sorgente di radiazione \(ds\) secondo la relazione:

\[dN(a) = \lambda(s)ds\]

Per ottenere il numero di coppie di fotoni generati lungo tutta la LOR basta semplicemente integrare tutti i contributi elementari sul percorso che unisce i due detettori:

\[N\left( d_{1},d_{2} \right) = N(a)e^{- \int_{d_{1}}^{d_{2}}{\mu(s)ds}}\]

Ma è noto che:

\[N(a) = \int_{d_{1}}^{d_{2}}{\lambda(s)ds}\ \]

Da cui è possibile scrive che:

\[N\left( d_{1},d_{2} \right) = \int_{d_{1}}^{d_{2}}{\lambda(s)ds}e^{- \int_{d_{1}}^{d_{2}}{\mu(s)ds}}\]

In questo modo si considerano tutti i fotoni emergenti all'interno del paziente fissata una LOR.

In PET, il coefficiente di assorbimento non è noto e non può essere stimato se non mediante la PET/CT. La conoscenza di questo parametro permette di correggere le immagini PET con la stima della probabilità di attenuazione dei fotoni. In CT questa problematica non si presenta poiché la sorgente di radiazione è esterna, quindi, i fotoni attraversano i tessuti umani e la radiazione residua incide sui rilevatori. La tecnica trasmissiva permette di ricavare l'integrale di linea del coefficiente di attenuazione lineare note che siano l'intensità di radiazione iniziale e l'intensità misurata sulla lastra radiografica o detettori allo stato solido.

L'intensità di radiazione può essere espressa come il numero di fotoni che attraversa una data superficie, quindi, noto il numero dei fotoni emessi e rilevati è possibile valutare l'integrale di linea del coefficiente di attenuazione lineare come:

\[\dfrac{N\left( d_{2} \right)}{N_{0}} = e^{- \int_{a}^{d_{2}}{\mu(s)ds}}\]

In PET, invece, la radiazione non è generata all'esterno ma all'interno del corpo. A priori, quindi, non è possibile conoscere la radiazione emessa da un punto del corpo poiché non è nota la distribuzione del tracciante all'interno dell'organismo.

Prima dell'introduzione della PET/CT, si posizionava una sorgente radioattiva, solitamente germanio-68, all'esterno del paziente che emanava un'intensità di fotoni \(\gamma\) nota. Questi fotoni attraversando il paziente, permettevano di ricostruire una mappa del coefficiente di attenuazione lineare dei tessuti all'energia dei fotoni \(\gamma\) di 511keV allo stesso modo della CT.

Al giorno d'oggi, questa metodica non è più utilizzata poiché il germanio-68 decade con un tempo di emivita delle ore, quindi, per acquisire un numero di fotoni abbastanza elevato è necessario aspettare un tempo molto lungo. In aggiunta, va considerato anche tempo dell'\emph{Imaging} PET stesso, ottenuto mediante il radiotracciante. In totale ci vorrebbe un tempo di un'ora circa per la radiazione trasmissiva e circa 45min per l'esame PET vero e proprio. I tempi estremamente lunghi portano alla ricostruzione di immagini con numerosi e notevoli artefatti da movimento.

D'altro canto, la CT permette di ottenere delle immagini \emph{Total Body} in pochi secondi, dunque, riduce notevolmente gli artefatti da movimento a discapito di una compensazione del coefficiente di attenuazione lineare che non risulta essere perfetta.

Mediante la scansione trasmissiva che sia essa con CT o radioisotopo, è possibile stimare l'integrale di linea del coefficiente di assorbimento lineare come:

\[\dfrac{N\left( d_{2} \right)}{N_{0}} = e^{- \int_{a}^{d_{2}}{\mu(s)ds}}\]

Ovviamente nel caso della PET/CT il coefficiente \(\mu(s)\) deve essere ricalcolare per le energie tipiche della PET.

Il numero di fotoni congiuntamente rilevati su entrambi i detettori può essere scritto come:

\[N\left( d_{1},d_{2} \right) = \int_{d_{1}}^{d_{2}}{\lambda(s)ds}\ e^{- \int_{d_{1}}^{d_{2}}{\mu(s)ds}} = \dfrac{N\left( d_{2} \right)}{N_{0}}\int_{d_{1}}^{d_{2}}{\lambda(s)ds}\]

Da quest'ultima relazione è possibile risalire all'integrale di linea della distribuzione del tracciante lungo la LOR individuata dai due detettori come:

\[N\left( d_{1},d_{2} \right)\dfrac{N_{0}}{N\left( d_{2} \right)}\  = \int_{d_{1}}^{d_{2}}{\lambda(s)ds}\]

Con degli algoritmi di ricostruzione, noti il numero di fotoni congiuntamente rilevati da due detettori posti su una LOR e il numero di fotoni trasmesso e ricevuto, posti entrambi su una stessa retta coincidente con la LOR, è possibile determinare il modo in cui il radionuclide è distribuito nel corpo del paziente.

\subsection{Filtered Back Projection o FBP}\label{filtered-back-projection-o-fbp}

Siccome la formula per la ricostruzione è molto simile a quella ottenuta in CT, si potrebbe pensare di utilizzare degli algoritmi basati sulla retroproiezione filtrata o FBP (\emph{Filtered Back Projection}) per invertire la trasformata di Radon.

Si definisce la proiezione o trasformata di Radon di una funzione \(f(x,y)\) l'integrale di linea della funzione \(f(x,y)\) lungo una linea parallela all'asse del detettore ad una distanza \(\xi\)' dall'origine. L'operazione di proiezione corrisponde ad una operazione di campionamento spaziale e si può scrivere, per una data funzione \(f(x,y,k)\):

\[p_{\gamma}(\xi) = R\left\lbrack f(x,y) \right\rbrack = \iint_{\mathbb{R}^{2}}^{}{f(x,y)\delta\left( x\cos(\gamma) + \sin(\gamma) - \xi \right)dxdy} = \iint_{\mathbb{R}^{2}}^{}{f\left( \mathbf{r} \right)*\delta\left( \mathbf{r} - \mathbf{L} \right)d^{(2)}\mathbf{r}}\]

\(\mathbf{L}\) è un vettore che rappresenta la linea lungo cui avviene la proiezione mentre \(\mathbf{r}\) è il vettore posizione di un generico punto nel piano.

Fissato il punto nel piano \(\mathbf{r}\), al variare del parametro \(\gamma\) si ottiene la proiezione della funzione \(f\left( \mathbf{r} \right)\) lungo tutte le rette passante per il punto in esame, ovvero si costruisce il sinogramma.

A partire dal sinogramma è possibile ricostruire l'immagine sfruttando un risultato noto come teorema della fetta centrale che lega la trasformata di Fourier monodimensionale della trasformata di Radon con la trasformata bidimensionale della funzione immagine, ristretta a una retta passante per l'origine dello spazio-frequenza e formante con l'asse delle ascisse un angolo \(\gamma\):

\[P_{\gamma}(q) = \left. \ F(q,p) \right|_{(q,\gamma)}\]

Si può dimostrare che questo metodo introduce una \emph{Point Spread Function} o PSF del tipo \(\dfrac{1}{r}\) che provoca uno sfocamento dell'immagine ricostruita:

\[g(x,y) = f\left( \mathbf{r} \right)*\dfrac{1}{\left| \mathbf{r} \right|}\]

L'effetto introdotto dalla PSF è detto \emph{Blurring} e per ridurlo si sfruttano algoritmi basati sulla retroproiezione filtrata. Questa metodica sfrutta dei filtri passa-basso che, inoltre, permettono di ridurre il rumore introdotto e il fenomeno di Gibbs. La frequenza di taglio del filtro deve essere opportunamente scelta per ridurre il rumore ma allo stesso tempo è necessario ridurre lo sfocamento introdotto dalla reiezione delle alte frequenze.

\[g(x,y) = f\left( \mathbf{r} \right)*\dfrac{1}{\left| \mathbf{r} \right|}*h\left( \mathbf{r} \right)\]

I filtri implementati molto spesso sono anche dedicati solo a particolari distretti corporei, in cui il rumore può presentarsi con diverse caratteristiche. Le proprietà che deve possedere un filtro molto spesso sono determinate su base empirica. Uno dei filtri più utilizzati è la finestra di Hamming.

\begin{figure}
\centering
\includegraphics[width=3.94958in,height=3.47015in,alt={P4950\#yIS1}]{media/21_RicImmPET/image529.pdf}\caption{Figura .: Risposta in frequenza dei vari filtri}
\end{figure}

La retroproiezione filtrata è stata storicamente utilizzata nelle applicazioni PET per ricostruire l'immagine, tuttavia, a causa del ridotto numero di fotoni conteggiati rispetto ai fotoni X in CT, questa metodica restituisce immagini con scarsa risoluzione e qualità.

Anche ricostruendo l'immagine con un \emph{Kernel}, ovvero una matrice di convoluzione o maschera, opportuno le ricostruzioni sono molto rumorose e presentano l'artefatto a stella tipico. Ciò è dovuto, appunto, allo scarso numero di fotoni rilevato che rende il segnale molto più rumoroso di un segnale CT.

\begin{figure}
\centering
\includegraphics[width=6.37547in,height=4.86458in,alt={P4954\#yIS1}]{media/21_RicImmPET/image530.pdf}\caption{Figura .: Effetto del Kernel sull'immagine}
\end{figure}

Per ricostruire l'immagine PET si sfruttano degli algoritmi iterativi dove, per passi successivi, si corregge il sinogramma fino a che il sinogramma attuale e quello precedente non differiscano per una determinata soglia. Gli algoritmi basati su FBP presentano un numero finito di passi predefinito, quindi, risultano essere più veloci rispetto ad algoritmi iterativi.

Per analizzare quale metodo di ricostruzione offra le caratteristiche migliori si considerano dei fantocci o \emph{Phantom} contenenti delle sorgenti di radionuclidi. I fantocci presentano una forma cilindrica al cui interno sono contenuti altri cilindri ripieni del mezzo di contrasto.

Si osserva che, a parità di esposizione, i metodi basati sulla retroproiezione filtrata permettano di ricostruire immagini molto meno chiare e nitide rispetto a un algoritmo iterativo tipo OSEM. Per piccoli numeri di fotoni, l'immagine con FBP è decisamente poco leggibile poiché non sono evidenziate e contrastate le strutture interne del fantoccio. Il metodo iterativo, invece, permette di ottenere una buona stima della \emph{Slice} già dopo 10s dell'esposizione.

Ovviamente, per entrambe le possibili ricostruzioni, maggiore è il tempo di esposizione e migliore è la risoluzione dell'immagine.

\begin{figure}
\centering
\includegraphics[width=6.68681in,height=2.19306in,alt={P4960\#yIS1}]{media/21_RicImmPET/image531.pdf}\caption{Figura .: Immagini con FBP in alto e con OSEM in basso}
\end{figure}

La retroproiezione filtrata risultata inefficiente, non solo per la scarsità delle coppie di fotoni assorbiti ma anche per la degradazione del segnale a opera del rumore Compton e di Poisson. Gli eventi di annichilazione, infatti, risultano rispettare la statistica di Poisson poiché l'interazione dei positroni con gli elettroni sono eventi statistici e indipendenti tra i vari atomi di radionuclide così come i fotoni che colpiscono il cristallo scintillatore. Ne discende che il numero delle coppie di fotoni incidenti sui detettori è assimilabile al numero di eventi verificatosi in un determinato intervallo spaziale. La distribuzione che meglio approssima il comportamento del rumore quantico sovrapposto al segnale segue la statistica di Poisson:

\[P_{N}(\lambda) = \dfrac{\lambda\left( \mathbf{r} \right)^{N\left( \mathbf{r} \right)}e^{- \lambda\left( \mathbf{r} \right)}}{N\left( \mathbf{r} \right)!}\]

Dove \(\mathbf{r}\) è la coordinata spaziale, \(N\) il numero di coppie ricevute, \(\lambda\) il numero medio di coppie.

La distribuzione di Poisson può assumere con probabilità diversa da zero solamente valori interi non negativi.

\begin{figure}
\centering
\includegraphics[width=4.14405in,height=3.21875in,alt={P4966\#yIS1}]{media/21_RicImmPET/image532.pdf}\caption{Figura .: Andamento della PDF di Poisson}
\end{figure}

Il modello di Poisson descrive tutti gli esperimenti di conteggio di piccole manifestazioni come l'arrivo delle coppie, a patto di scegliere il parametro \(\lambda\) adeguatamente. Si dimostra anche che il valor medio e la varianza della PDF di Poisson coincidono con \(\lambda\).

Il rapporto segnale/rumore può essere espresso come:

\[SNR = \dfrac{E\left\lbrack P_{N} \right\rbrack}{VAR\left\lbrack P_{N} \right\rbrack} = \sqrt{\lambda}\]

Le fluttuazioni introdotte da questo rumore diminuiscono all'aumentare del numero medio di coppie rilevato, ma ciò comporta una maggior dose al paziente che a sua volta aumenta i tempi dell'esame diagnostico. È, quindi, necessario un giusto compromesso tra tempistiche, dose assorbita dal paziente e SNR.

All'aumentare del numero di eventi, per il teorema del limite centrale, la distribuzione di Poisson tende a una gaussiana \(N\) di media e varianza \(\lambda\):

\[P_{N}(\lambda)\sim N(\lambda,\lambda)\]

In questo caso, il rumore quantico può essere modellato come un rumore gaussiano le cui proprietà statistiche dipendo dal numero medio di coppie di fotoni che incidono sui detettori.

\subsection{Metodi iterativi}\label{metodi-iterativi}

I maccanismi di elaborazione dei dati PET secondo protocolli interativi prevedono la ricostruzione delle immagini mediante approssimazioni successive che partano da un'immagine iniziale per raggiungere una stima sempre migliore dell'anatomia effettiva del paziente. Il processo di ricostruzione procede per un numero di passi non prevedibile a propri poiché è eseguito finché il sinogramma corrente non differisce dal sinogramma precedente di una certa soglia. Nella pratica il numero di passi che un algoritmo iterativo compie per ottenere una soddisfacente qualità di costruzione non è molto elevato. I tempi di ricostruzione con i moderni calcolatori non sono così troppo elevati ma la qualità è molto migliore di un'immagine ottenuto con retroproiezione filtrata.

Si è visto che le immagini CT ricostruite con algoritmi iterativi presentano delle caratteristiche molto migliori rispetto a quelle ottenute con il classico procedimento basato sulla trasformata inversa di Radon. Ciò consente di ridurre il dosaggio di radiazione X fornita al paziente a parità di qualità dell'immagine. Gli algoritmi usati in CT appartengono alla tipologia ASIR o \emph{Adaptive Statistical Iterative Reconstruction}.

Dal punto di vista operativo, i metodi iterativi prevedono uno schema ciclico basato su una stima iniziale dell'immagine (\emph{Estimated Image}) a partire da congetture iniziali (\emph{Initial Guess}). Si calcolano delle proiezioni che permettono di ricostruire il sinogramma corrispondente all'immagine stimata e lo si compara con le proiezioni effettivamente misurate dalla PET. La differenza tra le proiezioni misurate, provenienti dal corpo del paziente, e quelle stimate, basate sulle congetture iniziale e i successivi aggiusti, è poi utilizzata per correggere l'immagine corrente. Dopodiché l'algoritmo procede finché la discrepanza tra le proiezioni effettivamente misurate e quelle stimate non sono inferiori di una certa soglia o finché le due immagini stimate da due cicli diversi non differiscono per una piccola quantità.

\begin{figure}
\centering
\includegraphics[width=5.22619in,height=2.11408in,alt={P4979\#yIS1}]{media/21_RicImmPET/image533.pdf}\caption{Figura .: Schema di un algoritmo iterativo}
\end{figure}

\subsubsection{Forward Projection}\label{forward-projection}

Un possibile metodo per l'esecuzione di un algoritmo iterativo consiste nel valutare una distribuzione pesata dall'errore così da ottenere una proiezione diretta. Si indica con \(q_{i}\) il numero di fotoni che incide sull'i-esimo pixel dell'immagine, dato dalla relazione:

\[q_{i} = \sum_{j}^{}{a_{ij}C_{j}}\]

Dove \(C_{j}\) è l'attività nel j-esimo pixel mentre \(a_{ij}\) è la probabilità che l'emissione del pixel \(j\) sia registrata nella i-esima LOR.

Si osservi che nella notazione per indicare i pixel si è utilizzato un solo indice \(j\). I pixel sono, quindi, numerati in maniera progressiva a partire dal primo elemento della riga. Così la prima riga sarà numerata da \(1\) fino a \(n\), la seconda da \(n + 1\) a \(2n\) e così via.

\begin{figure}
\centering
\includegraphics[width=3.23243in,height=2.30208in,alt={P4987\#yIS1}]{media/21_RicImmPET/image534.pdf}\caption{Figura .: Schema dei parametri in esame}
\end{figure}

Nota l'attività di un pixel j-esimo e la probabilità che questo stesso pixel sia rilevato dalla LOR i-esima e sommate per tutti i pixel è possibile ricavare il numero di fotoni contati dalla i-esima LOR. In altre parole, \(q_{i}\) rappresenta la stima della proiezione dell'immagine se le attività fossero esattamente uguali a \(C_{j}\).

Se si indica con \(p_{i}\) le proiezioni effettivamente misurate sulla LOR i-esima, l'errore può essere definito come la differenza tra la proiezione effettiva e quella stimata:

\[\varepsilon = p_{i} - q_{i}\]

L'errore è distribuito poi in maniera pesata sui pixel di una LOR così da poter correggere l'immagine stimata. Ai parametri \(C_{j}\) devono essere aggiunti delle quantità \(\mathrm{\Delta}C_{j}\) valutata secondo la relazione:

\[\mathrm{\Delta}C_{j} = \dfrac{a_{ij}\left( p_{i} - q_{i} \right)}{\sum_{j}^{}a_{ij}} = \dfrac{a_{ij}\varepsilon}{\sum_{j}^{}a_{ij}}\]

In quest'ottica, i coefficienti \(a_{ij}\) rappresentano i fattori di peso con cui distribuire l'errore e di conseguenza correggere l'immagine.

Ricalcolati i coefficienti \(C_{j}\), si ripete la valutazione dell'immagine e la stima dell'orrore che poi è distribuito sull'intera stima, così da poter correggere l'immagine corrente. Il processo si ripete finché l'errore tra le due immagini non si riduce al di sotto di una soglia.

\subsubsection{Classificazione dei metodi in base alle correzioni}\label{classificazione-dei-metodi-in-base-alle-correzioni}

I metodi iterativi per la ricostruzione delle immagini possono essere classificati in base a come si corregge l'immagine stimata:

\begin{itemize}
\item
  Nella correzione \emph{Point-By-Point} si calcolano gli errori di tutte le LOR passanti per un punto, si corregge quel punto per poi passare a un altro e così via;
\item
  Nella correzione \emph{LOR-By-LOR} si calcolano gli errori per ogni LOR e si applica la correzione sui voxel di una singola LOR per poi passare alla successiva;
\item
  Per la ricostruzione simultanea, invece, l'immagine è ricostruita o \emph{updatata} simultaneamente.
\end{itemize}

Questi metodi possono essere implementati in varie tipologie di algoritmi, tra cui quelle attualmente più utilizzate sono:

\begin{itemize}
\item
  \emph{Maximum Likelihood Expectation Maximization} o MLEM basato, come suggerisce il nome, sulla massima verosimiglianza in grado di massimizzare il valore atteso. Esso richiede molte interazioni per poter ricostruire un'immagine ottimale;
\item
  \emph{Ordered Subset Expectation Maximization} o OSEM è una variante più efficiente della tipologia MLEM. In questo algoritmo le proiezioni sono ordinate in sottogruppi, dove N sottogruppi di LOR corrispondo a N interazioni eseguite con la tipologia MLEM.
\end{itemize}

Ovviamente aumentando il numero di segmenti N aumenta anche il tempo necessario per ottenere il risultato.

Gli algoritmi basati sulla retroproiezione filtrata devono lavorare su un'immagine in cui sono già state effettuate le correzioni quali normalizzazioni, riduzione del rumore, delle coincidenze \emph{Random}, di \emph{Scatter} e delle attenuazioni.

Negli algoritmi MLEM e OSEM, la maggior parte delle correzioni preliminari possono essere inglobate all'interno della stessa procedura di elaborazione. Inoltre, con queste tipologie non si osservano gli artefatti introdotti dalla retroproiezione, come gli artefatti a stella, e, per le correzioni inglobate, si migliore il rapporto segnale/rumore.

Una futura tipologia di algoritmi utilizzati sono i \emph{Row Action Maximum Likelihood} in cui si sfruttano delle sequenze di proiezioni ortogonali che determinano una convergenza più veloce degli OSEM.

Gli algoritmi di retroproiezione non permettono di visualizzare lesioni di piccole dimensioni e, per quelle di ampiezza maggiore, non presentano contorni netti e definiti. Nei polmoni, invece, qualsiasi tipo di lesione risulta essere poco visibile per gli artefatti a stella. Questi problemi si risolvono con gli algoritmi iterativi.

\begin{figure}
\centering
\includegraphics[width=6.22619in,height=2.20865in,alt={P5009\#yIS1}]{media/21_RicImmPET/image535.pdf}\caption{Figura .: Confronto tra immagini con FBP e OSEM. A polmoni, B fegato sano, C fegato con tumore, D mammella}
\end{figure}

\subsubsection{Ricostruzione 3D}\label{ricostruzione-3d}

I metodi di ricostruzione iterativi si prestano molto meglio alla ricostruzione di volumi tridimensionali \emph{Slice} per \emph{Slice} mentre gli algoritmi di retroproiezione filtrata permettono di ricostruire solamente una \emph{Slice} alla volta. Il volume paziente deve essere poi ricostruito mediante altre elaborazioni di interpolazione.

L'algoritmo OSEM è intrinsecamente tridimensionale e ciò consente di avere una maggiore sensibilità assiale poiché aumentano i piani di detezione e di conseguenza aumenta il numero di fotoni ricevuto. Il numero di LOR risulta essere molto maggiore del numero di pixel e ciò porta a un aumento della memoria occupata dall'algoritmo, la sua complessità computazionale e il tempo necessario per ottenere il risultato. Tuttavia, con gli elaboratori moderni questa soluzione permette la ricostruzione dell'intero volume paziente in tempi ragionevoli.

La direzione di una LOR nello spazio è individuata dai coseni direttori che essa forma con gli assi. Il versore di una qualsiasi LOR può essere espresso come:

\[\widehat{n} = \left( \begin{array}{r}
n_{x} \\
n_{y} \\
n_{z}
\end{array} \right) = \left( \begin{array}{r}
\cos\vartheta\cos\varphi \\
\sin\vartheta\cos\varphi \\
\sin\varphi
\end{array} \right)\]

Ponendo \(\varphi = 0\) si ottiene la direzione della LOR situata nel piano \(x - y\), ovvero si ricade nel caso bidimensionale.

I metodi iterativi possono essere applicati direttamente in 3D anche se i tempi di calcolo aumentano notevolmente perché la parametrizzazione del sinogramma si complica enormemente.

Per ridurre il carico computazione è possibile esegue un'operazione di \emph{Rebinning} di dati 3D in dati 2D. In particolare, la procedura di \emph{Single Slice Rebinning} o SSRB consiste nell'assegnare le LOR non planari, ovvero con \(\varphi \neq 0\), a piani bidimensionali che passano per il loro punto medio.

Questo metodo è, quindi, equivalente a un'acquisizione \emph{Multi-Ring} e presenta ottime prestazioni per le LOR centrali.

Un altro metodo utilizzabile è il \emph{Fourier Rebinning} o FORE che consiste nell'applicare la trasformata di Fourier bidimensionale ai sinogramma obliqui.

\begin{figure}
\centering
\includegraphics[width=6.64354in,height=5.30952in,alt={P5021\#yIS1}]{media/21_RicImmPET/image536.pdf}\caption{Figura .: Tecnica di Rebinning}
\end{figure}

\subsubsection{Approccio Bayesiano}\label{approccio-bayesiano}

Il principio di funzionamento del MLEM si basa su un approccio statistico detto Bayesiano. Siano:

\begin{itemize}
\item
  \(Q\) le misure effettuate. Le misure sulla i-esima LOR sono indicate con \(q_{i}\);
\item
  \(\Lambda\) l'immagine ricostruita tramite le attività stimata \(C_{j}\) di ogni voxel;
\item
  \(p(\Lambda)\) la probabilità a priori che può essere stimata;
\item
  \(p\left( Q \middle| \Lambda \right)\) la probabilità di ottenere i dati misurati \(Q\) dalle immagini ricostruite \(\Lambda\);
\item
  \(p\left( \Lambda \middle| Q \right)\) la probabilità a posteriori, ovvero la probabilità di tutte le attività \(\Lambda\) avendo osservato i conteggi Q.
\end{itemize}

L'ultima probabilità è legata alle altre tramite la relazione di Bayes:

\[p\left( \Lambda \middle| Q \right) = \dfrac{p\left( Q \middle| \Lambda \right)p(\Lambda)}{p(Q)}\ \]

Dove \(p(Q)\)è la probabilità di avere le proiezioni Q.

Nell'ottica dell'approccio Bayesiano, \(p\left( Q \middle| \Lambda \right)\) è la verosimiglianza, ovvero la probabilità che le misure effettive \(Q\) siano rispettate dall'immagine ricostruita \(\Lambda\).

\subsubsection{Maximum A Posteriori Probability}\label{maximum-a-posteriori-probability}

L'approccio a massima probabilità a posteriori o MAP (\emph{Maximum A Posteriori Probability}) consiste nel massimizzare la probabilità a posteriori \(p\left( \Lambda \middle| Q \right)\) per ottenere la ricostruzione dell'immagine più probabilmente compatibile con le misurazioni \(Q\) ottenute. In altre parole, si vuole far in modo che la probabilità di ottenere l'immagine \(\Lambda\) abbia la massima probabilità date le misure eseguite.

Solitamente si ritiene che la probabilità a priori \(p(\Lambda)\) sia costante, ovvero che tutte le immagini abbiano la stessa probabilità di essere corrette. Questa considerazione consente di semplificare l'analisi poiché non è nota l'anatomia del paziente prima di eseguire l'\emph{Imaging}, ma non è sempre verificata, ad esempio, la probabilità di ottenere un'immagine con due fegati è nulla.

Fissata \(p(\Lambda)\), per massimizzare \(p\left( \Lambda \middle| Q \right)\) è necessario variare il parametro \(\Lambda\). Dato che \(p(Q)\) è costante rispetto a \(\Lambda\) e noto avendo eseguito le misure, massimizzare la probabilità a posteriori \(p\left( \Lambda \middle| Q \right)\) equivale a massimizzare la verosimiglianza.

\[\max{\left\{ p\left( \Lambda \middle| Q \right) \right\} = \max\left\{ p\left( Q \middle| \Lambda \right) \right\}\ }\]

Si indica con \(\lambda_{j}\) l'attività locale nel pixel ovvero il numero di fotoni emessi nell'unità di tempo nel voxel j-esimo. Sebbene l'immagine sia una matrice tridimensionale, i pixel sono numerati con un solo indice riga per riga, una \emph{Slice} alla volta. Dunque, una volta definito l'ordine i pixel, sono tutti identificati con un unico indice. L'immagine totale \(\Lambda\) è una matrice formata da tutte le attività locali nel pixel:

\[\Lambda = \left\{ \lambda_{j} \right\}_{j = 1,\ldots,J}\ \]

Siano poi \(c_{ij}\) i coefficienti di attenuazione, ovvero la sensibilità del detettore i-esimo nei confronti del pixel j-esimo. \(c_{ij}\) è la probabilità che l'i-esimo detettori rilevi l'attività del j-esimo pixel.

Fissato \(i\), sommando su tutti i \(J\) pixel il prodotto di \(c_{ij}\) e \(\lambda_{j}\) si ottiene il numero dei fotoni che dovrebbe essere rilevato dall'i-esimo detettore:

\[r_{i} = \sum_{j = 1}^{J}{c_{ij}\lambda_{j}}\]

Il costruttore dello scanner PET fornisce una mappa tridimensionale dei coefficienti \(c_{ij}\) che, quindi, sono noti e dipendenti dal particolare scanner PET.

Il metodo MAP, rispetto al metodo della retroproiezione filtrata, ingloba al suo interno la distribuzione di Poisson degli eventi probabilistici di annichilazione. Infatti, la probabilità di misurare un numero di fotoni \(q_{i}\) sul detettore \(i\), quando il valore atteso è il numero di fotoni che dovrebbe essere rilevato dall'i-esimo detettore \(r_{i}\), segue la relazione:

\[p\left( q_{i} \middle| r_{i} \right) = \dfrac{r_{i}^{q_{i}}e^{- r_{i}}}{q_{i}!}\]

Dato che ogni fotone che giunge sul detettore i-esimo è indipendente dagli altri, la probabilità complessiva di osservare le misurazioni, quando il valore atteso è l'immagine ricostruita \(\Lambda\), è data dal prodotto delle singole probabilità \(p\left( q_{i} \middle| r_{i} \right)\).

\[p\left( Q \middle| \Lambda \right) = \prod_{i}^{}\dfrac{r_{i}^{q_{i}}e^{- r_{i}}}{q_{i}!}\]

Dove \(p\left( Q \middle| \Lambda \right)\) è la verosimiglianza delle osservazioni data l'immagine stimata corrente legate alle attività dei singoli pixel \(\lambda_{j}\). La quantità \(r_{i}\) dipende proprio dall'attività nel j-esimo pixel \(\lambda_{j}\). Dunque, la verosimiglianza dipende anche dall'immagine corrente tramite gli elementi \(\lambda_{j}\) che la compongono.

Le misurazioni sono state eseguite prima dell'elaborazione delle immagini, quindi, \(q_{i}\) è costante tra un'iterazione e l'altra. Per poter massimizzare la verosimiglianza è necessario variare i parametri \(\lambda_{j}\), ovvero l'immagine corrente. Variando l'immagine complessiva, varia il numero di fotoni atteso sull'i-esimo detettore \(r_{i}\) e, in definitiva, la verosimiglianza. Trascurando \(q_{i}!\) al denominatore poiché costante, per calcolare il valore massimo della verosimiglianza si applica il logaritmo a entrami i membri dell'espressione:

\[p\left( Q \middle| \Lambda \right) = \prod_{i}^{}\dfrac{r_{i}^{q_{i}}e^{- r_{i}}}{q_{i}!} \Leftrightarrow \log{p\left( Q \middle| \Lambda \right)} = \sum_{i}^{}\left( q_{i}\log r_{i} - r_{i} \right)\]

Si definisce \emph{Likelihood} come:

\[{L\left( Q \middle| \Lambda \right) = log}{p\left( Q \middle| \Lambda \right)}\]

Dato che il logaritmo è una funzione monotona, invece di massimizzare la verosimiglianza, si massimizza la \emph{Likelihood}. Sostituendo a \(r_{i}\) la sua espressione si ottiene:

\[L\left( Q \middle| \Lambda \right) = \sum_{i}^{}{\left( q_{i}\log{\sum_{j = 1}^{J}{c_{ij}\lambda_{j}}} - \sum_{j = 1}^{J}{c_{ij}\lambda_{j}} \right)\ }\]

Se la matrice \(C\) dei coefficienti di attenuazione del pixel j-esimo rispetto all'i-esimo detettore ha rango massimo, l'hessiano, ovvero la matrice delle derivate seconde rispetto a \(\lambda_{j}\), della funzione di \emph{Likelihood} è definito negativo (\(\mathbf{x}H\mathbf{x}^{T} < 0,\ \forall\mathbf{x}\epsilon\mathbb{R}^{n},\ \mathbf{x} \neq \mathbf{0}\). Ne discende che i suoi autovalori sono tutti negativi).

Per ottenere il massimo della \emph{Likelihood} bisogna porre il suo gradiente uguale a zero, ovvero:

\[\dfrac{\partial L}{\partial\lambda_{j}} = \dfrac{\partial}{\partial\lambda_{j}}\sum_{i}^{}\left( q_{i}\log{\sum_{j = 1}^{J}{c_{ij}\lambda_{j}}}\  - \sum_{j = 1}^{J}{c_{ij}\lambda_{j}} \right) = 0\]

\[\dfrac{\partial}{\partial\lambda_{j}}\sum_{i}^{}{(q_{i}}\log{\sum_{j = 1}^{J}{c_{ij}\lambda_{j}}}\  - \sum_{j = 1}^{J}{c_{ij}\lambda_{j}}) = \sum_{i}^{}\left( \dfrac{q_{i}}{\sum_{j = 1}^{J}{c_{ij}\lambda_{j}}}\sum_{j = 1}^{J}c_{ij} - \sum_{j = 1}^{J}c_{ij} \right) = = \sum_{i}^{}\left( \left( \dfrac{q_{i}}{\sum_{j = 1}^{J}{c_{ij}\lambda_{j}}} - 1 \right)\sum_{j = 1}^{J}c_{ij} \right)\]

In definitiva, per massimizzare la funzione di \emph{Likelihood}, bisogna porre

\[\sum_{i}^{}\left( \left( \dfrac{q_{i}}{\sum_{j = 1}^{J}{c_{ij}\lambda_{j}}} - 1 \right)\sum_{j = 1}^{J}{c_{ij}\lambda_{j}} \right) = 0\]

Da questa espressione è possibile calcolare le attività che devono avere i pixel affinché la \emph{Likelihood} sia massima.

Per trovare la soluzione di questo sistema di equazioni si dovrebbe invertire la matrice di coefficienti \(C\) di ordine \(I \times J\), dove \(I\) è il numero di detettori dell'ordine di 64x144x16=147456, dove 144 sono i detettori su un anello, 16 gli anelli e 64 gli elementi della griglia del cristallo scintillatore, e J il numero di pixel dell'ordine di 128x128=16384.

La matrice \(C\) dei coefficienti di peso \(c_{ij}\) possiede un numero elevatissimo di elementi dato da 16384x147456 il che rende la sua inversione un'operazione complessa dal punto di vista computazionale e soggetta a enormi errori poiché per eseguire l'inversione devono essere eseguite una serie di operazioni all'interno delle quali gli errori si propagano.

Quindi, anche un piccolo errore iniziale nel calcolo della matrice si propaga fino a ottenere un'inversa corrotta da rumori molto ampi.

La soluzione dei \(\lambda_{j}\) non è ottenibile in forma chiusa per ragioni di tipo numeriche. Per risolvere il problema si ricorre ad algoritmi di tipo iterativo.

\subsubsection{Gradient Ascent}\label{gradient-ascent}

Un metodo per la ricerca del massimo è detto \emph{Gradient Ascent} che si basa sulla ricerca del massimo tramite la valutazione della direzione di massima salita. Un algoritmo speculare è detto \emph{Gradient Vessel} in cui si ricerca la direzione di massima discesa e, dunque, riesce a trovare il minimo. I passi da eseguire per entrambe le tipologie sono identici a patto di invertirne il segno.

L'algoritmo parte da un'immagine arbitraria \(\Lambda\), ad esempio, un'immagine costante monocromatica. Per ogni iterazione si calcola il gradiente per ciascun elemento della matrice \(\Lambda\), \(\lambda_{j}\), così da ottenere la direzione di massima salita. Il gradiente è, infatti, il vettore che permette di determinare la direzione nello spazio delle \(\lambda_{j}\) in cui la \emph{Likelihood} presenta la massima variazione.

Per ottenere la stima all'iterazione successiva si aggiunge alla stima corrente \(\lambda_{j}\) un'opportuna quantità ricavata dal gradiente. La risoluzione analitica del problema sfrutta la teoria del \emph{Non-Linear Least Square}.

Ciò rende l'algoritmo robusto ma lento ed è per questo motivo che si adotta un'altra tipologia di soluzioni note come \emph{Expectation Maximization}.

\subsubsection{Expectation Maximization}\label{expectation-maximization}

L'algoritmo \emph{Expectation Maximization} è molto utilizzato nella pratica clinica per ricostruire le immagini PET poiché risulta essere molto più rapido rispetto al \emph{Gradient Ascent}. Questa soluzione cerca il massimo della \emph{Likelihood} in un modo alternativo che consente di ripetere le iterazioni un numero limitato di volte.

Per ottenere il massimo della \emph{Likelihood} si introduce una variabile aleatoria incognita poiché non osservabile \(X = \left\{ x_{ij} \right\}\), dove \(x_{ij}\) è il numero di fotoni provenienti dalla posizione j-esima e rilevati dall'i-esimo detettore.

La variabile \(X\) obbedisce alla statistica di Poisson e i suoi valori sono detti variabili complete. Si pone che il valor medio \(x_{ij}\) dato \(\Lambda\) uguale alla relazione:

\[E\left\lbrack x_{ij} \middle| \Lambda \right\rbrack = c_{ij}\lambda_{j}\]

Ovvero, il valor medio dei fotoni provenienti dalla posizione j-esima e rilevati dal detettore i-esimo, data l'immagine \(\Lambda\), è uguale all'attività del j-esimo pixel \(\lambda_{j}\) per la sensibilità del detettore i-esimo rispetto al pixel j-esimo.

Con l'introduzione delle variabili complete \(x_{ij}\), la \emph{Likelihood}, logaritmo naturale della verosimiglianza, può essere espresso come doppia sommatoria sui pixel (\(j\)) e detettori (\(j\)):

\[L_{x}(X,\Lambda) = \sum_{i}^{}{\sum_{j = 1}^{J}{{(x}_{ij}\log\left( c_{ij}\lambda_{j} \right) - c_{ij}\lambda_{j})}}\]

Dove si suppone noto, per ogni voxel, quanti fotoni siano stati emessi verso un certo detettore. Le variabili complete sostituiscono, quindi, le misure realmente determinate \(q_{i}\).

Ovviamente, non essendo effettivamente noti i valori di \(x_{ij}\) non è possibile calcolare la \emph{Likelihood} \(L_{x}(X,\Lambda)\). Tuttavia, è possibile calcolare il valor medio di tale quantità attraverso un procedimento noto come \emph{Expectation-step} o semplicemente E-step:

\[{E\lbrack L_{x}}(X,\Lambda)\left| \ Q,\Lambda^{old}\  \right\rbrack = E\left\lbrack \sum_{i}^{}{\sum_{j = 1}^{J}{{(x}_{ij}\log\left( c_{ij}\lambda_{j} \right) - c_{ij}\lambda_{j})}} \right\rbrack = \ \sum_{i}^{}{\sum_{j = 1}^{J}{{(E\lbrack x}_{ij}\rbrack\log\left( c_{ij}\lambda_{j} \right) - c_{ij}\lambda_{j})}}\]

Calcolando il valor medio è possibile, quindi, dimostrare che esso dipende solamente dall'attività del j-esimo pixel \(\lambda_{j}\). In questo modo è possibile procedere con la massimizzazione della funzione di verosimiglianza media calcolata nell'E-step con la stima precedente dell'immagine \(\Lambda^{old}\). Il processo di massimizzazione è noto come M-step e permette di determinare una nuova immagine \(\Lambda\) tale da massimizzare la verosimiglianza.

Nell'E-step, si pone \(n_{ij}\) il valore atteso di \(x_{ij}\). La sommatoria su \(j\) di tutti i valor medi delle \(x_{ij}\) dovrebbe essere uguale alle osservazioni effettivamente misurate dal detettore i-esimo:

\[\sum_{j}^{}{E\left\lbrack x_{ij} \right\rbrack = q_{i} \Leftrightarrow}\sum_{j}^{}{n_{ij} = q_{i}}\]

Infatti, deve risultare anche che \(n_{ij} = c_{ij}\lambda_{j}\). La sommatoria si riduce proprio alle misure effettuate:

\[\sum_{j}^{}{c_{ij}\lambda_{j} = q_{i}}\]

Si può dimostrare che la quantità \(n_{ij}\) è dato da:

\[n_{ij} = c_{ij}\lambda_{j}^{old}\dfrac{q_{i}}{\sum_{i}^{}{c_{ij}\lambda_{j}^{old}}}\]

Il valor medio della \emph{Likelihood} dipende dall'immagine stimata nello step precedente, i cui coefficienti sono noti, e da un'immagine \(\Lambda\) da valutare in modo da ottimizzare la verosimiglianza.

Nel M-Step, si procede con il calcolo del gradiente del valor medio e la sua posizione a zero così da poter determinare le attività correnti del j-esimo pixel che massimizzano la \emph{Likehood}:

\[\dfrac{\partial}{\partial\lambda_{j}}{E\lbrack L_{x}}(X,\Lambda)\left| \ Q,\Lambda^{old}\  \right\rbrack = \dfrac{\partial}{\partial\lambda_{j}}\sum_{i}^{}{\sum_{j = 1}^{J}{n_{ij}\ \log\left( c_{ij}\lambda_{j} \right) - c_{ij}\lambda_{j})}} = 0\]

Dove, derivando rispetto a \(\lambda_{j}\) la sommatoria su \(j\) si elimina poiché, tra tutti gli elementi, solamente il j-esimo è non nullo:

\[\dfrac{\partial}{\partial\lambda_{j}}\sum_{i}^{}{\sum_{j = 1}^{J}{n_{ij}\log\left( c_{ij}\lambda_{j} \right) - c_{ij}\lambda_{j})}} = \sum_{i}^{}\left( \dfrac{n_{ij}}{\lambda_{j}} - c_{ij} \right)\]

Ponendo uguale a zero si ottiene l'attività del j-esimo pixel che massimizza la \emph{Likelihood}:

\[\sum_{i}^{}{\left( \dfrac{n_{ij}}{\lambda_{j}} - c_{ij} \right) = 0}\]

La cui soluzione è:

\[\lambda_{j} = \dfrac{\sum_{i}^{}n_{ij}}{\sum_{i}^{}c_{ij}}\]

Sostituendo ai valori attesi delle variabili complesse l'espressione determinata:

\[n_{ij} = c_{ij}\lambda_{j}^{old}\dfrac{q_{i}}{\sum_{i}^{}{c_{ij}\lambda_{j}^{old}}}\]

Si ottiene una relazione che lega l'attività del j-esimo pixel che massimizza la \emph{Likelihood} in funzione delle misure effettuate e le attività dell'immagine precedente:

\[\lambda_{j} = \lambda_{j}^{old}\dfrac{1}{\sum_{m}^{}c_{mj}}\ \sum_{i}^{}{c_{ij}\dfrac{q_{i}}{\sum_{k}^{}{c_{ik}\lambda_{k}^{old}}}}\]

Dove \(\dfrac{1}{\sum_{m}^{}c_{mj}}\) è la \emph{Sensitivity} dello scanner per il j-esimo pixel mentre \(\sum_{k}^{}{c_{ik}\lambda_{k}^{old}}\) è la proiezione che sarebbe misurata se l'immagine precedente fosse la vera immagine. Entrambi i fattori rappresentano dei pesi con cui stimare la nuova immagine.

Nonostante la sua apparente complessità analitica, l'algoritmo \emph{Expectation Maximization} presenta un'ottima ricostruzione e un tempo di calcolo ridotto.

Infatti, vari studi dimostrano che il numero di iterazioni necessario per un'ottima ricostruzione delle immagini è di circa 10. Per un'iterazione, in generale, è necessario aspettare un tempo dell'ordine di 1s, quindi, l'immagine è ricostruita dopo circa 10s.

Ogni iterazione, infatti, comprende fondamentalmente delle moltiplicazioni in cui alcuni termini restano costanti poiché indipendenti dall'immagine precedente.

Uno dei problemi principali di questo algoritmo riguarda la convergenza del risultato all'immagine reale.

Infatti, per un'immagine non affetta da rumore è possibile notare che, all'aumentare del numero di iterazioni, la verosimiglianza cresce mentre l'errore quadratico medio si riduce.

Viceversa, quando i dati sono rumorosi, esiste un minimo locale dell'errore quadratico medio, intorno alla decina di iterazioni, dopodiché aumenta.

È fondamentale controllare il numero di iterazioni da compiere affinché il contenuto informativo dell'immagine non risulti essere completamente corrotto dal rumore.

\begin{figure}
\centering
\includegraphics[width=6.05218in,height=4.41667in,alt={P5111\#yIS1}]{media/21_RicImmPET/image537.pdf}\caption{Figura .: Immagine priva di rumore (DX) e con rumore (SX). Con 64 iterazioni l'immagine rumorosa è illeggibile}
\end{figure}

La \emph{Maximum Likelihood Expectation Maximization} o MLEM non è realmente implementata negli scanner PET poiché ogni costruttore realizza una propria variante dell'algoritmo.

Esiste, inoltre, una variante ancora più efficiente dell'algoritmo detta OSEM (\emph{Ordered Subset Expectation Maximization}) che possono essere sia bidimensionali che tridimensionali.

\subsection{Effetto di volume parziale}\label{effetto-di-volume-parziale-1}

La PET presenta la problematica che i voxel, in cui è scomponibile il corpo del paziente sono sufficientemente grandi, dell'ordine di 6mm lato.

Ciò comporta che le immagini sono affette dalla problematica del volume parziale che si verifica quando un voxel contiene uno o più tessuti e si manifesta con una riduzione complessiva del contrasto, definito come la capacità di distingue due strutture molto vicine tra loro.

L'oggetto emettente di piccole dimensioni o \emph{Hot-Spot} nel voxel sembra avere un'attività inferiore di quella reale perché essa è distribuita sull'intero voxel, con dimensioni molto più grandi.Tale effetto è compensato mediante l'introduzione di un coefficiente di correzione, definito come il rapporto tra l'attività ricostruita e l'attività vera, su una regione di interesse o ROI più piccola del doppio della risoluzione spaziale.

\begin{figure}
\centering
\includegraphics[width=3.10614in,height=3.19792in,alt={P5119\#yIS1}]{media/21_RicImmPET/image538.pdf}\caption{Figura .: Fantoccio e immagine PET}
\end{figure}

Esistono degli studi in letteratura che analizzando il contrasto tra masse tumorali e il \emph{Background}. In particolare, essi evidenziano che le masse tumorali sono ben visibili quando vi è un elevato numero di conteggi e il rapporto tra la luminosità delle cellule tumorali rispetto al \emph{Background} è dell'ordine della decina. Da ciò si evince la necessità di rilevare la gran parte degli eventi di annichilazione, poiché, in caso contrario, l'immagine risulta essere poco visibile anche con un contrasto elevato.

Se il contrasto di per sé è basso, anche con un numero di conteggi molto elevato l'immagine non permette di distinguere le masse tumorali rispetto al \emph{Background}.

\begin{figure}
\centering
\includegraphics[width=5.15717in,height=3.97727in,alt={P5123\#yIS1}]{media/21_RicImmPET/image539.pdf}\caption{Figura .: Ricostruzione dell'immagine tumore con diversi conteggi e contrasto rispetto lo sfondo}
\end{figure}

Generalizzando, il contrasto dipende anche dall'intensità del radio tracciate rispetto al \emph{Background} nella struttura anatomica. Più è elevato è il rapporto tra la luminosità della struttura anatomica irradiante rispetto il \emph{Background} e maggiore è il numero di conteggi, allora più nitida è l'immagine.

\subsection{Protocollo di ricostruzione PET/CT}\label{protocollo-di-ricostruzione-petct}

Per ottenere immagini PET molto più affidabili e priva di artefatti si ricorre al protocollo PET/CT che prevede una scansione preliminare con la CT spirale per identificare le zone di cui eseguire l'\emph{Imaging}. Si ottengono così delle immagini, dette \emph{µ-Image}, che mostrano come è distribuito il coefficiente di attenuazione lineare all'interno del corpo del paziente.

Si inietta poi il liquido di contrasto e si esegue la scansione PET volta a ottenere le immagini di emissione che forniscono informazioni sulla distribuzione del radionuclide all'interno del corpo del paziente.

Dalla fusione delle immagini trasmissive della CT ed emissive della PET si ottiene un'immagine globale che mostra sia l'anatomia del paziente che l'attività metaboliche dei suoi tessuti mediante delle scale di pseudocolori.

\begin{figure}
\centering
\includegraphics[width=6.69306in,height=2.68361in,alt={P5130\#yIS1}]{media/21_RicImmPET/image540.pdf}\caption{Figura .: Protocollo PET/CT}
\end{figure}

La fusione delle immagini è un problema non del tutto chiuso. Per sovrapporre l'immagine PET alla CT è necessario eseguire delle operazioni di interpolazione poiché il voxel della prima metodica è più grande della seconda. La CT, infatti, presenta una dimensione del voxel dell'ordine di 1mm o minore, mentre la PET di 6mm.

Dal \emph{Dicom Info} è possibile rilevare la \emph{Slice Location} ovvero l'ascissa lungo l'asse \(z\) della \emph{Slice} di interesse sia per le immagini CT che PET.

A causa dei voxel di ampiezza diversa, non è possibile avere la stessa ascissa \(z\) per le due \emph{Slice} da combinare. Dunque, per combinare le due immagini, si scelgono due \emph{Slice} che differiscono per una piccola quantità, inferiore del millimetro.

Nei DICOM sono contenute anche le informazioni sull'orientamento del paziente grazie alla voce \emph{ImageOrientationPatient}. Questa informazione è fornita come un vettore contenente i coseni direttori degli assi rispetto al sistema di riferimento identificato dallo scanner. In alcuni esami l'asse lungo cui si posiziona il paziente può non coincidere con quello dello scanner, ad esempio, in cardiologia l'asse di \emph{Imaging}, di solito, coincide con l'asse cardiaco.

Con la PET/CT si tende ad allineare le fette poiché la ricostruzione PET procede, solitamente, lungo piani ortogonali all'asse \(z\).

Con \emph{PixelSpacing,} invece, si indica lo spessore di un voxel nelle direzioni \(x\) e \(y\), mentre con \emph{ImagePositionPatient} si ottengono informazioni sul posizionamento del primo pixel, in alto a sinistra, dell'immagine nel sistema di riferimento paziente.

Con le informazioni contenute nei \emph{Dicom Info} è possibile leggere le immagini CT e PET e ricavarne i dettagli. Solitamente, oltre a non essere allineate, le immagini CT presenta una dimensione di 512x512 mentre l'immagine PET di 128x128, quindi, risulta avere un minor numero di pixel poiché questi possiedono dimensioni maggiori. Da ciò è ovvio che un singolo voxel dell'immagine PET contiene più voxel della CT.

Prima di sovrapporre le immagini è necessario riallinearle. A tale scopo si ricorre alle informazioni contenute nel DICOM, secondo il quale, le coordinate tridimensionali di un punto nell'immagine nel sistema di riferimento paziente, identificato da un quaternione ovvero un vettore di quattro componenti, possono essere ricavate dalle coordinate-immagine \((i,\ j,\ 0)\) moltiplicate per la matrice le cui tre righe sono i coseni direttori dei punti dell'immagine seguita da una colonna di zeri, rappresentante la coordinata \(z\), e poi un'altra contenente la posizione del punto alla sinistra dell'immagine:

\[\left( \begin{array}{r}
p_{x} \\
p_{y} \\
p_{z} \\
1
\end{array} \right) = \begin{pmatrix}
\begin{array}{r}
X_{x} \\
X_{y} \\
X_{z} \\
0
\end{array} & \begin{matrix}
\begin{array}{r}
Y_{x} \\
Y_{y} \\
Y_{z} \\
0
\end{array} & \ 
\end{matrix}\begin{matrix}
\begin{array}{r}
0 \\
0 \\
0 \\
0
\end{array} & \begin{array}{r}
S_{x} \\
S_{y} \\
S_{z} \\
0
\end{array}
\end{matrix}
\end{pmatrix}\left( \begin{array}{r}
i \\
j \\
0 \\
1
\end{array} \right)\]

Per ogni pixel della matrice dell'immagine è possibile costruire il suo quaternione, dunque, ogni pixel dell'immagine sia CT che PET possiede la sua quaterna di coordinate. La coordinata \(z\) per una \emph{Slice} è fissata e, quindi, non ha senso considerarla. Per tale motivo essa è posta a zero.

La griglia risultante per la CT avrà 512x512x4 componenti, mentre per la PET ha una griglia di quaternioni di 128x128x4.

Con questa procedura si ottengono le coordinate dei punti nel sistema di riferimento solidale col \emph{Gantry} dello scanner, ovvero i punti di entrambe le immagini sono riferite a un unico sistema di riferimento. In questo modo è possibile eseguire l'interpolazione così da mostrare le informazioni di un voxel della PET su più voxel della CT.

Mediante algoritmi di interpolazione bidimensionali (\emph{interp2} in Maltab), a partire delle due griglie costruite sulle immagini PET e CT e dall'attività misurate nelle immagini PET, sono valutati i valori dell'immagine PET in un punto in cui non è stata misurata l'attività. In particolare, si vuole ricavare il valore dell'immagine PET negli stessi punti dell'immagine CT così da costruire un'immagine dell'emissione con stessa dimensione dell'immagine di trasmissione. L'interpolazione, in definitiva, è necessaria per poter sovrapporre le due immagini.

L'interpolazione sfrutta la conoscenza dei valori assunti da una funzione in determinati punti discreti per valutare il valore del pixel in punti non misurati mediante una stima pesata con diverse relazioni in base al tipo di interpolazione che si vuole ottenere. Se i pesi sono segmenti di rette, l'interpolazione è detta bilineare.

\begin{figure}
\centering
\includegraphics[width=3.33681in,height=3.18472in,alt={P5146\#yIS1}]{media/21_RicImmPET/image541.pdf}\caption{Figura .: Schema di interpolazione bilineare}
\end{figure}

Con l'interpolazione bilineare, l'immagine risultate PET/CT presenta dei contorni più sfumati poiché l'algoritmo di interpolazione si comporta come un filtraggio passa-basso.

\begin{longtable}[]{@{}
  >{\raggedright\arraybackslash}p{(\linewidth - 2\tabcolsep) * \real{0.5000}}
  >{\raggedright\arraybackslash}p{(\linewidth - 2\tabcolsep) * \real{0.5000}}@{}}
\caption{Figura .: Immagine CT e immagine PET/CT con interpolazione lineare}\tabularnewline
\toprule\noalign{}
\begin{minipage}[b]{\linewidth}\centering
\includegraphics[width=2.99057in,height=2.99057in,alt={P5149C1T9\#yIS1}]{media/21_RicImmPET/image542.pdf}\end{minipage} & \begin{minipage}[b]{\linewidth}\centering
\includegraphics[width=2.99664in,height=3.01887in,alt={P5150C2T9\#yIS1}]{media/21_RicImmPET/image543.pdf}\end{minipage} \\
\midrule\noalign{}
\endfirsthead
\toprule\noalign{}
\begin{minipage}[b]{\linewidth}\centering
\includegraphics[width=2.99057in,height=2.99057in,alt={P5149C1T9\#yIS1}]{media/21_RicImmPET/image542.pdf}\end{minipage} & \begin{minipage}[b]{\linewidth}\centering
\includegraphics[width=2.99664in,height=3.01887in,alt={P5150C2T9\#yIS1}]{media/21_RicImmPET/image543.pdf}\end{minipage} \\
\midrule\noalign{}
\endhead
\bottomrule\noalign{}
\endlastfoot
\end{longtable}

La bilineare non è sempre preferibile rispetto ad altre metodiche. Nella pratica, spesso si frutta un'interpolazione della \emph{Nearest} in cui si assegna ai punti intorno a un pixel misurato, il valore dell'attività nota.

Il punto incognito assume, quindi, il valore del pixel con luminosità nota più vicino. L'immagine ottenuta mostra un certo grado di granulosità dovuta proprio alla conservazione della maggior dimensione dei pixel della PET. Ingrandendo, infatti, si osserva un quadrato funzionale omogeneo sovrapposto all'immagine anatomica.

\begin{longtable}[]{@{}
  >{\raggedright\arraybackslash}p{(\linewidth - 2\tabcolsep) * \real{0.4982}}
  >{\raggedright\arraybackslash}p{(\linewidth - 2\tabcolsep) * \real{0.5018}}@{}}
\caption{Figura .: Differenza tra bilineare (SX) e Nearest (DX)}\tabularnewline
\toprule\noalign{}
\begin{minipage}[b]{\linewidth}\centering
\includegraphics[width=3.16403in,height=3.1875in,alt={P5155C1T10\#yIS1}]{media/21_RicImmPET/image543.pdf}\end{minipage} & \begin{minipage}[b]{\linewidth}\centering
\includegraphics[width=3.20798in,height=3.18229in,alt={P5156C2T10\#yIS1}]{media/21_RicImmPET/image544.pdf}\end{minipage} \\
\midrule\noalign{}
\endfirsthead
\toprule\noalign{}
\begin{minipage}[b]{\linewidth}\centering
\includegraphics[width=3.16403in,height=3.1875in,alt={P5155C1T10\#yIS1}]{media/21_RicImmPET/image543.pdf}\end{minipage} & \begin{minipage}[b]{\linewidth}\centering
\includegraphics[width=3.20798in,height=3.18229in,alt={P5156C2T10\#yIS1}]{media/21_RicImmPET/image544.pdf}\end{minipage} \\
\midrule\noalign{}
\endhead
\bottomrule\noalign{}
\endlastfoot
\end{longtable}

L'ultimo aspetto riguarda la visione dell'immagine stesse. La sovrapposizione non è banale poiché è necessario unire un'immagine a colori con una in scala di grigio. Scegliendo opportunamente i pesi della sovrapposizione è possibile enfatizzare un aspetto funzionale in pseudocolori oppure l'espetto morfologico in scala di grigi.

Questa configurazione è gestita dal radiologo mediante manopole o pulsanti sulla \emph{Console} di visualizzazione.

\begin{longtable}[]{@{}
  >{\raggedright\arraybackslash}p{(\linewidth - 2\tabcolsep) * \real{0.5000}}
  >{\raggedright\arraybackslash}p{(\linewidth - 2\tabcolsep) * \real{0.5000}}@{}}
\caption{Figura .: A sinistra immagine più anatomica che funzionale a destra più funzionale che anatomica}\tabularnewline
\toprule\noalign{}
\begin{minipage}[b]{\linewidth}\centering
\includegraphics[width=3.14603in,height=3.125in,alt={P5161C1T11\#yIS1}]{media/21_RicImmPET/image545.pdf}\end{minipage} & \begin{minipage}[b]{\linewidth}\centering
\includegraphics[width=3.125in,height=3.125in,alt={P5162C2T11\#yIS1}]{media/21_RicImmPET/image546.pdf}\end{minipage} \\
\midrule\noalign{}
\endfirsthead
\toprule\noalign{}
\begin{minipage}[b]{\linewidth}\centering
\includegraphics[width=3.14603in,height=3.125in,alt={P5161C1T11\#yIS1}]{media/21_RicImmPET/image545.pdf}\end{minipage} & \begin{minipage}[b]{\linewidth}\centering
\includegraphics[width=3.125in,height=3.125in,alt={P5162C2T11\#yIS1}]{media/21_RicImmPET/image546.pdf}\end{minipage} \\
\midrule\noalign{}
\endhead
\bottomrule\noalign{}
\endlastfoot
\end{longtable}

Esistono anche metodi di visualizzazione multiplanari in cui lo stesso distretto anatomico è mostrato secondo i piani transassiali, sagittali e coronale, simultaneamente per CT, PET e la loro fusione. Quest'ultima permette di riconoscere, in modo chiaro, le regioni che presentano un'attività maggiore e, quindi, una maggiore concentrazione di radionuclide.

Un altro protocollo di visualizzazione delle immagini è il MIP o \emph{Maximum Intensity Projection} in cui ogni pixel dell'immagine assume solo il massimo valore di attività che lo interessano. Si ignorano così le integrazioni delle varie attività.

\subsection{Artefatti nelle immagini PET}\label{artefatti-nelle-immagini-pet}

Le immagini PET possono essere affette da una serie di errori che portano a una scorretta valutazione diagnostica.

La presenza di oggetti metallici come protesi o otturazione dentarie in CT genera artefatti a stella molto forti a causa dell'elevato assorbimento di questi materiali alle energie della radiologia convenzionale. L'immagine PET permette di risolvere parzialmente la presenza degli artefatti metallici correggendo l'attenuazione.

\begin{figure}
\centering
\includegraphics[width=6.68681in,height=1.21667in,alt={P5170\#yIS1}]{media/21_RicImmPET/image547.pdf}\caption{Figura .: A immagine CT, B immagine PET, C immagine corretta}
\end{figure}

Siccome l'esame PET ha una durata di 30-45min diventano importanti gli artefatti da movimento. A questa categoria appartengono gli artefatti legati al respiro del paziente durante l'esame e visibile prevalentemente nella regione diaframmatica e polmonare.

Se il radio tracciante scorre in prossimità dei polmoni, a causa del movimento ritmico legato alla respirazione, i punti di annichilazione si trovano ad altezze diverse per ogni istante di tempo. I fotoni emergenti impattano su diverse LOR falsando così la misura che, a sua volta, provoca un artefatto nell'immagine ricostruita. Ovviamente, siccome la CT non presenta questa problematica, non è ottima la correzione delle immagini in prossimità dei polmoni.

La risoluzione dell'artefatto da movimento legato alla respirazione è ancora in corso di studio poiché a oggi non esiste un metodo efficacie per la sua risoluzione. Alcuni studi prevedono l'\emph{Imaging} con semplici telecamere che registrano i movimenti del paziente, per poi ricostruire l'immagine in maniera sincrona con i movimenti respiratori. Analogamente, è possibile utilizzare delle semplici fasce elastiche che memorizzano i movimenti toracici mediante variazioni della resistenza di elementi sensibili al loro interno.

L'artefatto legato alla respirazione può portare a diagnosi errate poiché alcuni errori legati alla respirazione potrebbero essere scambiate per lesioni del parenchima del fegato posto subito sotto i polmoni. In presenza di lesione, invece, queste potrebbero essere spostate al di sotto o al di sopra della posizione effettiva.

\begin{figure}
\centering
\includegraphics[width=5.89167in,height=5.36404in,alt={P5176\#yIS1}]{media/21_RicImmPET/image548.pdf}\caption{Figura .: Esempio di artefatto da respirazione.}
\end{figure}

Gli artefatti da iniezione sono dovuti alla distribuzione non uniforme del tracciante. Infatti, nel sito di infusione del radionuclide si osserva un'attività maggiore che deve essere considerata al momento della refertazione. Non conoscendo il punto di iniezione si potrebbe evidenziare questa regione come lesionata portando a una diagnosi errata.

\begin{figure}
\centering
\includegraphics[width=6.67767in,height=2.78261in,alt={P5179\#yIS1}]{media/21_RicImmPET/image549.pdf}\caption{Figura .: Artefatto da iniezione}
\end{figure}

Se, per motivi diagnosti, la CT è eseguita mediante contrasto radiopaco, le immagini PET corrette con CT risultati avranno un'intensa attività nelle zone perfuse dal tracciare della CT.

\begin{figure}
\centering
\includegraphics[width=6.68472in,height=1.57639in,alt={P5182\#yIS1}]{media/21_RicImmPET/image550.pdf}\caption{Figura .: Artefatti da mezzo di contrasto radiopaco}
\end{figure}

Infine, se il FOV della CT è differente da quello dalla PET, che generalmente ha estensione minore, potrebbero esserci delle zone dell'immagine PET non corrette, soprattutto in pazienti corpulenti. Ciò potrebbe portare a delle omissioni di informazioni cliniche rilevanti come lesioni tumorali maligne in prossimità della superficie del corpo come la pelle.

\begin{figure}
\centering
\includegraphics[width=6.31137in,height=4.87736in,alt={P5185\#yIS1}]{media/21_RicImmPET/image551.pdf}\caption{Figura .: Artefatto da FOV (melanoma non corretto)}
\end{figure}

\begin{center}
\vfill
    \chapter{Prestazioni di uno scanner PET}
    \label{blx:PrestazioniPET\therefsection}
\vfill

\minitoc
\newpage
\end{center}
\justify

\section{Prestazioni di uno scanner PET}\label{prestazioni-di-uno-scanner-pet}

La valutazione delle prestazioni di uno scanner PET può essere molto utile in fase di acquisizione, di collaudo e di manutenzione.

\subsection{Risoluzione spaziale}\label{risoluzione-spaziale}

Tra le caratteristiche principali di uno scanner PET vi è la risoluzione spaziale, influenzata da vari fattori quali:

\begin{itemize}
\item
  Dimensione del detettore che, generalmente, presentano una dimensione caratteristica dell'ordine di 20cm circa. La dimensione genera un'indeterminazione nell'individuazione della LOR per la costruzione del sinogramma. Infatti, la posizione della LOR non può essere individuata con una precisione dell'ordine del mm, ma dell'ordine del cm;
\item
  Il \emph{Positron Range} è il cammino percorso dal positrone prima di annichilarsi con l'elettrone e, quindi, prima dell'emissione dei fotoni;
\item
  Non collinearità dei fotoni;
\item
  Metodo di ricostruzione come \emph{Filtered Back Projection} o iterativo;
\item
  Localizzazione dei detettori.
\end{itemize}

\subsubsection{Detector Size}\label{detector-size}

Detta \(d\) la dimensione del detettore, generalmente la risoluzione è variabile tre \(\frac{d}{2}\) al centro del \emph{Gantry} oppure \(d\) all'avvicinarsi del detettore. La risoluzione non è la stessa in tutto il \emph{Gantry}, ma dipende anche da fattori geometrici che riguardano la relazione geometrica tra detettori posti in coincidenza per determinate LOR.

Per risoluzione si intende la differenza minima tra due oggetti rilevabili, ovvero due sorgenti puntiformi di radioattività. Tali sorgenti possono essere rilevate con maggiore risoluzione in prossimità del centro piuttosto che alla periferia, verso i detettori per motivi geometrici.

\subsubsection{Positron Range}\label{positron-range}

Dall'istante in cui il positrone è emesso dalla sorgente di radioattività fino a quello in cui quest'ultimo si annichila con l'elettrone atomico e si assiste all'emersione dei due fotoni, intercorre un certo tempo in cui il positrone percorre un certo spazio all'interno del paziente. Il positrone compie una serie di urti fino a perdere l'energia cinetica, così da rendere possibile l'annichilazione dopo aver percorso un certo cammino detto \emph{Positrion Range}. Tale range spaziale varia all'interno dei materiali e dipende dalle sostanze radioattive. Tipicamente, nell'acqua, il F-18 emette un positrone con un \emph{Positron Range} che in media si aggira al di sotto del mm. Altre sostanze, sempre in acqua, possono avere \emph{Positron Range} dell'ordine di diversi mm. Ciò influenza la risoluzione nel senso che non è possibile misurare con esattezza il punto in cui è avvenuto il decadimento, ma piuttosto i punti in cui sono avvenute le annichilazioni, i quali sono spostati rispetto al punto in cui si ha il decadimento. Usando materiali con range positronico basso (come ad es. il F-18) tale effetto impatto meno sulla degradazione della risoluzione.

\begin{figure}
\centering
\includegraphics[width=4.3381in,height=1.89583in,alt={P5203\#yIS1}]{media/22_PrestazioniPET/image552.pdf}
\caption{Figura .: Positrion Range}
\end{figure}

\subsubsection{Non Colinearity}\label{non-colinearity}

I fotoni che sono emessi dall'evento di annichilazione non sono esattamente collineari. Questi non emergono con uno sfasamento di 180° ma vi è una piccola deviazione dovuta al principio di conservazione della quantità di moto. Il positrone, quando arriva in corrispondenza dell'elettrone, non possiede una quantità di moto esattamente nulla, dunque, l'energia complessivamente presente nell'evento di annichilamento non è esattamente pari a quella dei due fotoni, e, dovendosi conservare la quantità di moto, i due fotoni emergono con direzioni non collineari poiché, appunto la somma delle quantità di moto non è esattamente nulla. Da ciò deriva che l'angolo formato dalle due traiettorie non è esattamente pari a 180°. Si stima, con l'ausilio di studi statistici, che la massima deviazione è di circa 0.25°. Tale deviazione può divenire un problema nel momento in cui la distanza tra i due detettori è elevata e, dunque, è importante in scanner PET con grandi \emph{Gantry}. L'errore è stimato grossolanamente come \(0.0022D\) dove \(D\) è diametro del \emph{Gantry}.

\begin{figure}
\centering
\includegraphics[width=3.02324in,height=2.64167in,alt={P5207\#yIS1}]{media/22_PrestazioniPET/image553.pdf}\caption{Figura .: Non collinearità}
\end{figure}

\subsubsection{Metodo di ricostruzione}\label{metodo-di-ricostruzione}

La risoluzione dell'immagine legata al metodo di ricostruzione non è facilmente quantificabile. Per tale motivo sono proposti dei fattori moltiplicativi dell'ordine di 1.2-1.5. Per tener conto del metodo di ricostruzione si assegna un fattore che entra nella formula complessiva per il calcolo della risoluzione per ogni algoritmo di ricostruzione. In particolare, nel caso della \emph{Filtered Back Projection}, la ricostruzione è influenzata dal tipo di filtri e dalle frequenze di \emph{Cut-Off} utilizzate prima dell'antitrasformata di Radon.

\subsubsection{Localizzazione dei detettori}\label{localizzazione-dei-detettori}

Un ulteriore problema si riscontra nella localizzazione dei detettori, soprattutto nel caso in cui sono utilizzati detettori a blocchi, ovvero anelli di detettori. Si possono, infatti, generare errori di circa 2.2mm per BGO, mentre per altri materiali con una \emph{Light Output} o efficienza di scintillazione elevata tale errore può essere minore.

\subsubsection{Risoluzione spaziale totale}\label{risoluzione-spaziale-totale}

La risoluzione spaziale complessiva, indicata con \(R_{t}\), può essere valutata con la formula:

\[R_{t} = K\sqrt{R_{i}^{2} + \ R_{p}^{2} + R_{a}^{2} + \ R_{L}^{2}}\]

Dove \(R_{i}\) è la risoluzione associata al \emph{Detector Size}, \(R_{p}\) al \emph{Positron Range}, \(R_{a}\) alla \emph{Non Colinearity}, \(R_{L}\) alla localizzazione o \emph{Localisation} dei detettori, e \(K\) è una costante dipendente dal metodo di ricostruzione utilizzato.

\begin{figure}
\centering
\includegraphics[width=4.3002in,height=3.48333in,alt={P5217\#yIS1}]{media/22_PrestazioniPET/image554.pdf}
\caption{Figura .: Vari errori nella ricostruzione dell'immagine}
\end{figure}

Gli errori si sommano tra di loro e danno origine a una risoluzione totale di circa 5-6mm che peggiora all'allontanarsi dal centro del \emph{Gantry}.

\begin{figure}
\centering
\includegraphics[width=6.6875in,height=3.53125in,alt={P5220\#yIS1}]{media/22_PrestazioniPET/image555.pdf}\caption{Figura .: Parametri di alcuni costruttori}
\end{figure}

La FWHM (\emph{Full Width At Half Maximum}) è riportata lungo le direzioni transassiale, ovvero nel piano della \emph{Slice}, ortogonale al paziente, e assiale, lungo il piano del paziente.

\subsection{Sensitivity}\label{sensitivity}

La \emph{Sensitivity} è un parametro molto importante per identificare le \emph{performance} della PET, definita come il numero di eventi rilevati come coincidenza nell'unità di tempo per unità di attività nella sorgente. Essa è misurata in \(\frac{Cps}{MBq}\), ovvero conteggi per secondo rispetto ai Megabecquerel emessi dalla sorgente. Questo parametro dipende da:

\begin{itemize}
\item
  Efficienza geometrica tra sorgente e detettore;
\item
  Efficienza del detettore, quindi, dalle sue caratteristiche peculiari;
\item
  \emph{PHA Window}, ovvero la \emph{Pulse High Analizer} per discriminare gli eventi di annichilamento dagli eventi di \emph{Scatter} su base energetica;
\item
  \emph{Dead time}.
\end{itemize}

\subsection{Noise}\label{noise}

Gli eventi tipici della PET seguono la statistica di Poisson e, quindi, la deviazione standard è legata al numero di aventi rilevati \(N\) della relazione:

\[\sigma = \frac{1}{\sqrt{N}}\]

Si potrebbe pensare che il rumore possa essere ridotto aumentando il numero di eventi \(N\) rilevati. Ciò non è possibile poiché bisognerebbe aumentare la dose somministrata al paziente, aumento, così, l'attività e, di conseguenza, il numero di eventi. Ciò, tuttavia, porta a:

\begin{itemize}
\item
  Un aumento delle coincidenze casuali, dette eventi \emph{Random};
\item
  Il \emph{Dead Time} influisce sulla ricezione degli eventi. Se aumenta la dose il tempo morto influisce sulla rilevazione degli eventi portando a una riduzione dei conteggi;
\item
  Un problema clinico relativo all'aumento della radioattività sul paziente che vede così aumentare la probabilità di sviluppare neoplasie nel futuro.
\end{itemize}

Anche l'aumento dei tempi di esposizione, a parità di dose, risulta svantaggioso soprattutto per una problematica relativa agli artefatti da movimento.

Si introduce l'NECR (\emph{Noise Equivalent Count Rate}) come indice per quantificare il rumore e valutare le prestazioni dello scanner in termini di rumorosità dell'immagine.

\[NECR\  = \ \frac{T^{2}}{T + R + S}\]

Dove \(T\) sono le coincidenze vere, \(R\) quelle di \emph{Random} e \(S\) quelle di \emph{Scatter}.

Tale parametro può essere quantificato e riportato su diagrammi, dove sulle ascisse è posto l'attività della sorgente in kBq e sulle ordinate i conteggi per secondo.

\begin{figure}
\centering
\includegraphics[width=6.64252in,height=3.05208in,alt={P5241\#yIS1}]{media/22_PrestazioniPET/image556.pdf}\caption{Figura .: Andamento dell'NECR}
\end{figure}

Gli eventi veri sono rappresentati dalla curva indicata con \emph{Trues}. All'aumento dell'attività corrisponde un aumento degli eventi veri, fino ad un valore massimo. A causa dell'aumento del numero di eventi \emph{Random}, l'NECR presenta un aumento fino ad una certa dose, per poi esibire una diminuzione. Quindi, avere dosi eccessivamente alte non porta un beneficio in termini di rumore.

La misura dell'NECR può essere effettuata con fantocci come cilindri con materiale radioattivo di dimensione e radioattività determinata e normate.

Generalmente, si usano fantocci di 20cm di diametro e dal sinogramma sono conteggiati gli eventi \emph{Prompt}, ovvero tutti gli eventi, senza distinzione tra le tipologie di evento.

Per misurare l'NECR devono essere misurati gli eventi \emph{Scatter} e gli eventi \emph{Random} in modo che dalla sottrazione dal totale si possano stimare i soli eventi \emph{True}.

La \emph{Scatter Fraction}, frazione di eventi \emph{Scatter} rispetto al totale, può essere misurata come numero di eventi \emph{Scatter} su numero di eventi \emph{Promp:}

\[SF = \ \frac{C_{s}}{C_{p}}\]

È necessario, quindi, stimare il numero di eventi \emph{Scatter} con varie metodiche.

\subsection{Contrasto}\label{contrasto}

Il contrasto in PET è definito come il numero di eventi registrato in un tessuto A rispetto a quello registrato in un tessuto B, normalizzati rispetto ad uno dei due tessuti.

\[C = \ \frac{A - B}{B}\]

Tale rapporto dipenderà dall'attività emessa, dagli eventi \emph{Scatter}, dalla dimensione della lesione da osservare e dagli artefatti da movimento.

\subsection{Controlli di qualità}\label{controlli-di-qualituxe0}

Generalmente su uno scanner PET si eseguono una serie di controllo per attestarne la qualità della misura e, di conseguenza, dell'immagine ricostruita. I controlli di qualità possono essere eseguiti con una cadenza temporale diversa. In particolare, si distinguono in controlli quotidiani e settimanali. I primi, sono eseguiti una volta al giorno e consentono di valutare l'uniformità del sinogramma.

Dal punto di vista operativo, si analizza il sinogramma di due giorni consecutivi ricavato con la stessa sorgente, per lo stesso numero di ore. Il sinogramma deve essere uniforme nei due giorni consecutivi e per valutarne il grado di uniformità, si misura lo scarto quadratico medio tra i due sinogrammi punto per punto. Ovviamente, tra i due sinogrammi può esserci una certa percentuale di tolleranza, riportata nelle norme.

Per eseguire la procedura si espongono i detettori a una sorgente di radioattività isotropica come Ge-68 o Cs-137 contenute in un fantoccio cilindrico con un diametro di circa 20cm. Il sinogramma acquisito è uniforme, poiché, ragionevolmente, ogni detettore conta lo stesso numero di fotoni, essendo la sorgente isotropica, ed è poi confrontato con quello registrato nelle sedute precedenti. La differenza tra i due sinogrammi la si esprime come variazione media degli scarti quadratici delle efficienze dei detettori.

Se il valore ottenuto è maggiore di 2.5, è necessario ricalibrare lo scanner. Per valori troppo elevati è necessario un intervento del costruttore dello scanner. Questa metodica permette di verificare il corretto funzionamento dei detettori, poiché, in caso di malfunzionamento di uno di essi, nel sinogramma compare una linea scura in corrispondenza del detettore malfunzionante.

I controlli settimanali prevedono una serie valutazioni volte a identificare lo stato dell'apparecchiatura e i suoi parametri di calibrazione.

\begin{itemize}
\item
  Il \emph{System Calibration} consiste nel posizionare un \emph{Phantom} al centro del FOV e irradiare i detettori in modo uniforme. Le immagini ottenute sono poi controllate col fine di cercare delle disuniformità dovute a una scorretta calibrazione dei fotomoltiplicatori e dei circuiti di rilevazione delle energie. Le immagini dovrebbero essere, infatti, delle circonferenze con uguale attività per ogni \emph{Slice} del fantoccio. Se ciò non accade si procede con la calibrazione;
\item
  Il \emph{Plane Efficiency} permette di ottenere una stima delle variazioni di efficienza tra i vari piani dello scanner.
\end{itemize}

Tra un piano e l'altro, infatti, potrebbero esserci delle variazioni di efficienza a causa, ad esempio, delle diversità dell'elettronica di controllo. Esaminando le varie immagini ottenute per le fette del \emph{Phantom} cilindrico, è possibile osservare se tra di esse esiste qualche variazione nella direzione transassiale. In caso affermativo, si procede con il calcolo del fattore di correzione per ottenere immagini uniformi per ogni piano in cui è sezionabile il paziente;

\begin{itemize}
\item
  La normalizzazione permette di ricavare le disomogeneità tra i vari detettori dovuti ai fotomoltiplicatori, localizzazioni dei detettori e variazioni fisiche degli stessi o dell'elettronica di controllo legate, ad esempio, alla loro degradazione nel tempo. Questo processo è realizzato mediante una provetta di Ge-68 posizionato parallelamente all'asse longitudinale oppure un \emph{Phantom Standard}.
\end{itemize}

La sorgente utilizzata presenta una bassa attività per evitare delle perdite di conteggio dovute al \emph{Dead Time} del cristallo scintillatore, essendo gli eventi di annichilazioni più lontani nel tempo. Per avere un'ottima esposizione e, di conseguenza, un conteggio di eventi statisticamente significativo, l'acquisizione dei fotoni dura delle ore e per tale motivo è sempre eseguito di notte. Il controllo della normalizzazione può essere effettuato anche con cadenza mensile.

\begin{figure}
\centering
\includegraphics[width=6.44167in,height=3.40489in,alt={P5265\#yIS1}]{media/22_PrestazioniPET/image557.pdf}\caption{Figura .: Fantocci per la calibrazione}
\end{figure}

Nell'immagine precendente è possibile osservare i risultati delle procedure di normalizzazione: a sinistra vi è l'immagine ottenuta con detettori non normalizzati, dove si nota la presenza di \emph{Pattern} dovuta alla differenza in termini di sensibilità tra i vari detettori; mentre l'immagine a destra, ottenuta con detettori normalizzati, non presenta i \emph{Pattern} dovuti ai detettori, ma solo il rumore.

\subsection{Test ACR (American College of Radiology)}\label{test-acr-american-college-of-radiology}

\includegraphics[width=4.21875in,height=1.60069in,alt={P5269\#y1}]{media/22_PrestazioniPET/image558.pdf}
Le normative utilizzano una serie di fantocci, composti da un insieme di cilindri riempiti con materiale radioattivo in modo da simulare i diversi tipi di tessuto. Si effettuano così le varie prove per valutare contrasto, uniformità e risoluzione spaziale dello scanner PET in base alle possibili attività presenti all'interno del paziente.

Figura .: Test ACR

\subsubsection{Test di accettazione}\label{test-di-accettazione}

Con i fantocci della ACR è effettuata una prima operazione di verifica dello scanner PET in occasione del collaudo del dispositivo, per verificare:

\begin{itemize}
\item
  Il rispetto dei parametri dichiarati dal costruttore;
\item
  Risoluzione assiale e trasversale;
\item
  \emph{Sensitivity};
\item
  \emph{Scatter Fraction};
\item
  NECR.
\end{itemize}

Le normative internazionali alle quale ci si attiene per la PET sono:

\begin{itemize}
\item
  1991, SNM (\emph{Society of Nuclear Medicine});
\item
  1994, NEMA (\emph{National Electrical Manufacturers Association});
\item
  2001, NEMA NU 2-2001 che stabilisce delle linee guida recepite anche dalla UE.
\end{itemize}

\begin{quote}
\includegraphics[width=4.7in,height=1.88596in,alt={P5282\#yIS1}]{media/22_PrestazioniPET/image561.pdf}\end{quote}

\begin{figure}
\centering
\includegraphics[width=4.8in,height=0.72208in,alt={P5283\#yIS1}]{media/22_PrestazioniPET/image561.pdf}\caption{Figura .: Fantocci delle normative}
\end{figure}

A sinistra dell'immagine sono mostrati i fantocci previsti dallo standard NEMA 1994; a destra quelli del 2001.

In quest'ultimo standard sono proposti due tipi di \emph{Phantom}: uno cilindrico all'interno del quale deve essere posto il materiale radioattivo ed uno molto più piccolo realizzato in sei strati di alluminio concentrici che servono a misurare la \emph{Sensitivity} che deve essere, appunto, misurata per diversi diametri di alluminio.

Attraverso le procedure normate è possibile valutare la \emph{Full Width At Half Maximum} della PSF, mediante l'utilizzo di una sorgente puntiforme. La sei sorgenti di F-18 sono poste in capillari di vetro di 1cc e collocati in diverse posizioni all'interno del \emph{Gantry}.

I fantocci sono posizionati ad 1cm e 10cm sull'asse verticale e a 10cm sull'asse orizzontale.

In questo modo è possibile valutare la risposta impulsiva in vari punti, nelle diverse direzioni della \emph{Slice}, poiché le sorgenti possono essere approssimate come impulsi di Dirac diretti lungo i due assi del piano.

Ad una distanza di ¼ del FOV assiale si pone la stessa terna di sorgenti in modo da valutare, allo stesso modo, la PSF.

Figura .: Schema di posizionamento dei Phantom per misurare la FWHM

Ogni sorgente puntiforme fornisce una FWHM e, da tutte queste misure, si effettua la media in modo da ottenere un FWHM complessiva.

In genere la risoluzione è migliore al centro e peggiora verso la periferia del FOV.

Figura .: PSF a diverse distanze

\subsubsection{Scatter Fraction}\label{scatter-fraction}

Per la valutazione della \emph{Scatter Fraction}, così come suggerito dalla NEMA, si procede con step ben determinati:

\begin{itemize}
\item
  Si acquisiscono i fotoni, proveniente da un fantoccio di 20cm di diametro, fino a che i \emph{Random Events} e \emph{Dead Time Loss} sono trascurabili. Ciò si traduce in una proceduta con durata di diverse ore;
\item
  Successivamente, si usano dei sinogrammi che corrispondono alle LOR situate nel FOV. Dato che ogni scanner presenta un FOV diverso in base al costruttore la NEMA stabilisce che le LOR devono trovarsi in un diametro di 24cm, così da confrontare le misure tra i diversi costruttori;
\item
  Il fantoccio, posizionato in modo da essere coassiale con l'asse del \emph{Gantry}, ha un diametro di 20cm, mentre le LOR considerate competono a un raggio 24cm. Esistono 4cm in cui non vi è la sorgente di radiazione e, quindi, è lecito non aspettarsi eventi. Tutti le occorrenze esterne al fantoccio, appartenenti ai 4cm compresi tra \emph{Phantom} e FOV, sono considerate eventi di \emph{Scatter}. Gli eventi \emph{Random}, per l'elevato tempo trascorso, sono ritenuti trascurabili;
\item
  Si indica con \(C_{t}\) il numero eventi totale e \(C_{s}\) il numero eventi \emph{Scatter};
\item
  La \emph{Scatter Fraction} è valutata come:
\end{itemize}

\[SF\  = \frac{C_{s}}{C_{t}}\]

\begin{itemize}
\item
  Mentre gli eventi di coincidenza veri, avendo trascurato gli eventi \emph{Random}, come:
\end{itemize}

\[R_{true} = \frac{C_{t} - C_{s}}{tempo\ di\ esposizione}\]

\subsubsection{Sensitivity}\label{sensitivity-1}

La \emph{Sensitivity}, dal punto di vista analitico è definita come il conteggio al secondo rispetto all'attività della sorgente:

\[S\  = \frac{Cps}{MBq}\]

Essa è valutata mediante un \emph{Phantom} capillare NEMA di 70cm rivestito con lamine di metallo con differente spessore ed emettente un livello di attività bassa per ridurre \emph{Random Events} e \emph{Dead-Time Loss}. La sorgente di radioattività è poi racchiusa in lamelle di metallo, generalmente in cinque lamelle con differente spessore. Teoricamente il conteggio dovrebbe essere realizzato senza rivestimento metallico, tuttavia, sperimentalmente, è stato osservato che, con l'uso del rivestimento metallico e una procedura più laboriosa, si riesce ad ottenere una stima migliore della \emph{Sensitivity}.

Si eseguono 5/6 scansioni con diversi spessori della lamina di metallo e da queste si riesce a ottenere un \emph{Count-Rate} in funzione dello spessore della lamina attraverso i sinogrammi effettivamente misurati. Successivamente si esegue una regressione lineare per ottenere il \emph{Count-Rate} senza metallo, indicata con \(R_{0}\). Questo processo può essere, ad esempio, eseguito mediante un algoritmo OLS, nell'ipotesi che i logaritmi dei \emph{Count-Rate} giacciano su una retta.

Con questi procedimenti, la \emph{Sensitivity} può essere valutata come rapporto tra il \emph{Count-Rate} in senza metallo di rivestimento e l'attività della sorgente:

\[S = \frac{R_{0}}{A}\]

\subsubsection{Count Rate Losses}\label{count-rate-losses}

La perdita del tasso di conteggio è valutata mediante un fantoccio di F-18 che emette un'elevata attività, rilevata da detettori con una finestra PHA di 410-665keV. Il processo di acquisizione dei fotoni procede finché l'attività della sorgente non è così bassa, da poter trascurare il \emph{Dead Time Loss} e i \emph{Random Events}.

Il \emph{Count-Rate} è stimato, poi, come il numero totale di eventi sul tempo di esposizione dei detettori alla sorgente:

\[R_{t} = \frac{numero\ totale\ dieventi}{tempo\ di\ esposizione}\]

Dopo aver stimato lo \emph{Scatter Fraction} con uno dei metodi precedenti, raccomandanti dalle normative, si valuta il numero di conteggi veri come differenza tra il \emph{Count-Rate} e lo \emph{Scatter Fraction}:

\[R_{true} = \ R_{t} - R_{scatter}\]

Infine, il \emph{Noise Equivalent Count Rate} è valutato come rapporto tra il quadrato degli eventi \emph{True}, rapportato al \emph{Count-Rate}:

\[NECR\  = \frac{\left( R_{true} \right)^{2}}{R_{t}}\]

\subsection{PET/MRI}\label{petmri}

Realizzata una PET/CT è stato pensato di affiancare gli scanner PET agli MRI per ottenere gli scanner PET/MRI. La presenza delle due metodice in un unico \emph{Gantry} determina la nascita di problemi riguardanti la coesistenza delle due differenti tecnologie. Si può pensare ad una struttura in cui i due scanner sono coassiali e posti in maniera lineare l'uno dopo all'altro. Un'altra soluzione potrebbe essere quella di inserire uno scanner PET all'interno di uno MRI. In ogni caso vi è però un problema dovuto alla sensibilità del fotomoltiplicatore all'intenso campo magnetico prodotto dalla risonanza magnetica: un campo magnetico esercita una forza sulle cariche. Si può dimostrare, infatti, che una carica immessa all'interno di un campo magnetico percorre un'elica attorno all'asse magnetico. Per tale motivo gli scanner PET/MRI non sono costruiti con fotomoltiplicatori, ma con altri tipi di rilevatori basati su semiconduttori, i quali non risentono di tale problematica.

\begin{figure}
\centering
\includegraphics[width=6.43385in,height=3.31482in,alt={P5324\#yIS1}]{media/22_PrestazioniPET/image564.pdf}\caption{Figura .: Esempio di scanner PET/MRi}
\end{figure}

I costruttori hanno risolto le problematiche tecnologiche legate all'affiancamento delle due metodiche. Tuttavia, non esiste ancora un'applicazione clinica/scientifica principe della PET/MRI. I

nizialmente si era pensato ad uno studio che riguardasse l'analisi da un punto di vista cerebrale, affiancando all'analisi PET un'analisi fMRI, usando marcatori per analizzare il metabolismo cerebrale e l'effetto BOLD (\emph{Blood Oxygenation Level Dependent}) per studiare l'attivazione cerebrale.

Gli studi effettuati in questo senso sono stati condotti da un punto di vista scientifico/divulgativo accademico. Potrebbe non essere giustificata però come tecnologia da un punto di vista clinico/diagnostico.

Si stanno studiando traccianti molecolari particolari utili per lo studio tumorale. Si pensa a molecole in grado di marcare il tumore in MRI e allo stesso tempo radioattive in modo tale da essere rilevabili in PET. Ciò aumenterebbe l'utilità cliniche dello scanner PET/MRI

\subsection{PET Time Of Flight}\label{pet-time-of-flight}

\includegraphics[width=4.13194in,height=2.15625in,alt={P5331\#y1}]{media/22_PrestazioniPET/image565.pdf}
\begin{figure}
\centering
\includegraphics[width=6.51103in,height=3.93518in,alt={P5332\#yIS1}]{media/22_PrestazioniPET/image566.pdf}\caption{Figura .: PET Time Of Light}
\end{figure}

Figura .: Immagini corrette con informazioni sul tempo di volo

Le PET future saranno probabilmente dotate di questa tecnologia così da migliorare la qualità dell'immagine e, allo stesso tempo, migliorare la diagnosi.

Questa metodica non è presente su tutti gli scanner poiché necessita di materiali con caratteristiche molto spinte, che aumentano il costo complessivo dell'intera strumentazione.

\subsection{Assorbimento di fluorodesossiglucosio}\label{assorbimento-di-fluorodesossiglucosio}

Ogni pixel dell'immagine presenta un valore numerico associato all'attività del tracciante in quel determinato punto. Ciò ovviamente non è sufficiente per poter confrontare misure tra diversi pazienti. È necessario avere uno strumento che consenta di ottenere informazione sull'aggressività delle diverse lesioni e che consenta, inoltre, nel tempo, di comprendere la risposta di una lesione ad un determinato trattamento, ad esempio, tramite una riduzione dei livelli di aggressività.

A tale scopo è stato sviluppato lo \emph{Standard Uptake Value} o SUV come il rapporto tra l'attività in un certo voxel rispetto all'attività totale somministrata al paziente, normalizzato il suo peso. Analiticamente, questa quantità si esprime come:

\[SUV = \frac{Act_{voi}}{\frac{Act_{administered}}{BW}}\lbrack = \rbrack\frac{\frac{kBq}{mL}}{\frac{MBq}{kg}}\ \]

Questo fattore consente di normalizzare le diverse misure tra i vari pazienti. Il fattore peso del paziente è essenziale nel normalizzare e standardizzare la misura, in vista di esami effettuati a lunga distanza di tempo, considerando eventuali perdite di peso a causa della terapia somministra.

In oncologia il SUV è lo standard nell'effettuazione delle misurazioni. Un valore maggiore o uguale a 2 indica una lesione aggressiva poiché assorbe in modo significativo il fluorodesossiglucosio.

\begin{center}
\vfill
    \chapter{Tecnologia dual-energy X-ray}
    \label{blx:refsection\therefsection}
\vfill

\minitoc
\newpage
\end{center}
\justify

\section{Dual Energy X-Ray}\label{dual-energy-x-ray}

Molte apparecchiature, come CT e MOC, che misurano la presenza di minerali nel tessuto osseo, si basano sul concetto di \emph{Dual Energy}. In tutte le apparecchiature convenzionali si sfrutta l'ipotesi di fascio di raggi X monocromatico, ovvero tutti i raggi X abbiano la stessa frequenza e, per l'equazione di Planck, anche la stessa energia:

\[E = hf\]

Ciò è utile perché, nel caratterizzare i tessuti con il coefficiente di attenuazione lineare, si riscontra che tale coefficiente dipende dall'energia e, quindi, il confronto dei tessuti è più agevole a parità di energia irradiata.

Il principio dei raggi X \emph{Dual Energy} sfrutta le differenze nel coefficiente di attenuazione di massa per differenti materiali, al variare dell'energia, mediante l'erogazione di due fasci di raggi X monocromatici. Per un materiale fissato:

\[\mu = \mu(E)\]

L'andamento del coefficiente di attenuazione lineare in funzione dell'energia è \href{https://www.nist.gov/pml/x-ray-mass-attenuation-coefficients}{scaricabile} per un consulto. Ad alte energie, ovvero nel range energetico della PET, si nota che tutti i materiali presentano, approssimativamente, lo stesso coefficiente di attenuazione massico. Nel range energetico della CT, è possibile discriminare i tessuti in termini di \(\mu\) poiché i vari materiali biologici attenuano i raggi X in maniera differente.

\begin{figure}
\centering
\includegraphics[width=5.85313in,height=4.48148in,alt={P5498\#yIS1}]{media/23_DualEn/image567.pdf}\caption{Figura .: µ di diversi materiali al variare dell'energia.}
\end{figure}

Nella pratica sono molto utilizzati due approcci nell'applicazioni \emph{Dual Energy}:

\begin{itemize}
\item
  L'\emph{Energy Subtraction}, in cui si combinano, in maniera pesata, due immagini dello stesso oggetto ottenute con differenti energie. Un esempio di questa metodica è la \emph{Digital Subtraction Angiography} (DSA), utilizzata per rilevare la presenza di stenosi nei vasi sanguigni.
\end{itemize}

Irradiando un distretto anatomico con due differenti energie, nella sottrazione tra le due immagini, i tessuti che hanno lo stesso coefficiente di attenuazione, come l'aria e i tessuti molli, sono cancellati mentre il tessuto di interesse è ben visibile;

\begin{itemize}
\item
  Nel \emph{Basis Material Decomposition}, invece, le immagini ottenute con differenti energie possono essere decomposte nella somma pesata di due materiali detti materiali di base. Questa metodica è utilizzata nelle CT di ultima generazione e nelle MOC e permette di ricavare i materiali all'interno della struttura irradiata con i due fasci monocromatici a diversa energia.
\end{itemize}

Con quest'ultimo approccio è possibile individuare una base nello spazio dei materiali che consente poi di identificare i vari materiali.

\subsection{Rappresentazione dei coefficienti di attenuazione}\label{rappresentazione-dei-coefficienti-di-attenuazione}

Negli anni `70, Klein e Nishina studiarono gli andamenti del coefficiente di attenuazione massico in funzione dell'energia della radiazione e si resero conto che certi motivi teorici permettevano di decomporre il \(\mu(E)\) nella somma di più contributi.

I processi dominanti l'assorbimento della radiazione, nel range di energie usate per la diagnostica, sono l'effetto fotoelettrico e l'effetto Compton. I coefficienti di attenuazione dipendono dall'energia. Al di sopra del \emph{K-Edge} la funzione \(\frac{\mu(E)}{\rho}\), coefficiente di attenuazione lineare di massa, può essere decomposta in funzioni opportune del tipo:

\[\frac{\mu(E)}{\rho} = a_{p}f_{p}(E) + \ a_{c}f_{KN}(E)\]

Dove \(a_{p}\) e \(a_{c}\) sono i coefficienti dello sviluppo corrispondenti, rispettivamente, all'effetto fotoelettrico e Compton. Si può dimostrare che le funzioni della scomposizione sono:

\[\left\{ \begin{array}{r}
f_{p}(E) = \ \frac{1}{E^{3}}\ \ \ \ \ \ \ \ \ \ \ \ \ \ \ \ \ \ \ \ \ \ \ \ \ \ \ \ \ \ \ \ \ \ \ \ \ \ \ \ \ \ \ \ \ \ \ \ \ \ \ \ \ \ \ \ \ \ \ \ \ \ \ \ \ \ \ \ \ \ \ \ \ \ \ \ \ \ \ \ \ \ \ \ \ \ \ \ \ \ \ \ \ \ \ \ \ \ \ \ \ \ \ \ \ \ \ \ \ \ \ \ \  \\
f_{KN}(E) = \ \frac{1 + E}{E^{2}}\left\lbrack \frac{2(1 + E)}{1 + 2E} - \frac{1}{E}\ln(1 + 2E) \right\rbrack + \frac{1}{2E}\ln(1 + 2E) - \frac{(1 + 3E)}{(1 + 2E)^{2}}
\end{array} \right.\ \]

Equivalentemente si ha:

\[\mu(E) = a_{p}f_{p}(E) + \ a_{c}f_{KN}(E)\]

Ridefinendo opportunamente i coefficienti \(a_{p}\) e \(a_{c}\) che rappresentano i pesi con cui combinare i contributi dell'effetto fotoelettrico e Compton.

La funzione \(f_{p}(E)\) approssima la dipendenza energetica della interazione fotoelettrica, mentre la funzione \(f_{KN}(E)\), detta funzione di Klein-Nishina, fornisce la dipendenza energetica della sezione d'urto totale per effetto Compton.

Rappresentando le due funzioni in scala logaritmica si nota un comportamento prettamente lineare con qualche discostamento difficilmente percettibile. Sommando le due funzioni si ottiene proprio la curva del coefficiente di attenuazione massico in cui è possibile osservare i due tratti lineari a pendenza differente.

\begin{figure}
\centering
\includegraphics[width=5.38019in,height=4.2037in,alt={P5516\#yIS1}]{media/23_DualEn/image568.pdf}\caption{Figura .: Andamento delle due funzioni dello sviluppo}
\end{figure}

I coefficienti dello sviluppo sono espressi in funzione dei parametri fisici del materiale, mediante delle relazioni deducibili da motivi quantomeccanici e sono:

\[\left\{ \begin{array}{r}
a_{p} = K_{1}\frac{\rho}{A}Z^{n},\ \ n \simeq 4 \\
a_{c} = K_{2}\frac{\rho}{A}Z\ \ \ \ \ \ \ \ \ \ \ \ \ \ \ \ \ \ \ \ \ \ \ 
\end{array} \right.\ \]

Con \(K_{1}\) e \(K_{2}\) costanti, \(\rho\) densità di massa, \(A\) peso atomico, \(Z\) numero atomico. Tale espressione è alla base del \emph{Dual Energy}. I due coefficienti sono, infatti, legati al peso atomico e al numero atomico e, quindi, in qualche modo legati univocamente al materiale, cioè la specie atomica in esame. Inoltre, nel range 30-200keV, includente la radiologia convenzionale, l'errore di rappresentazione, valutato come errore tra la funzione osservata realmente e quella ottenuta dallo sviluppo, è dell'ordine dell'1\%.

Applicando un metodo OLS (in MatLab \textbackslash{} per evitare problematiche associate all'inversione della matrice) per ottenere il fitting delle funzioni \(f_{p}(E)\) e \(f_{KN}(E)\) alle energie di interesse, è possibile dimostrare che i residui, identificati come la differenza tra il valore reale e il modello, assoluti sono più alti alle basse energie e si riducono alle alte energie. I residui percentuali, ottenuti rapportando i residui assoluti rispetto alle misure effettuate, sono al di sotto dell'1\%. Dunque, l'approssimazione utilizzata introduce degli errori completamente trascurabili.

\begin{figure}
\centering
\includegraphics[width=4.94101in,height=4.00926in,alt={P5522\#yIS1}]{media/23_DualEn/image569.pdf}\caption{Figura .: Andamento dei residui}
\end{figure}

Iodio e osso si differenziano di molto, mentre tutti gli altri materiali sono abbastanza simili. Ogni coppia di coordinate del digramma in basso identifica un materiale diverso.

\includegraphics[width=3.03133in,height=2.39815in,alt={P5525\#yIS1}]{media/23_DualEn/image570.pdf}
Figura .: Andamento di \(a_{c}\) in funzione \(a_{p}\)

\subsection{Dual Energy Contrast-Enhanced Digital Mammography (DE-CEDM)}\label{dual-energy-contrast-enhanced-digital-mammography-de-cedm}

Come esempio di \emph{Energy Subtraction} si considera la \emph{Dual Energy Contrast-Enhanced Digital Mammography} o DE-CEDM. In questo tipo di mammografia digitale si acquisiscono due immagini ad alta e bassa energia dopo la somministrazione del mezzo di contrasto a base di iodio. Successivamente, la sottrazione pesata consente di cancellare il tessuto della mammella.

Lo iodio presenta un picco per il \emph{K-Edge} a circa 33keV e tale fenomeno può essere sfruttato per enfatizzare i tessuti e le regioni in cui è stato iniettato il mezzo di contrasto. Il mezzo di contrasto è iniettato nella paziente circa due minuti prima di effettuare l'esame in modo da poter fluire all'interno dei vasi.

La paziente è posta nell'apparecchiatura di radiologia, la quale genera due fasci monocromatici a energie differenti, effettuando due acquisizioni in maniera non simultanea: una prima acquisizione è effettuata ad alta energia mentre la seconda a bassa energia.

In particolare, sia \(\Phi_{0}^{H}(i,\ j)\) l'intensità del fascio di raggi X ad alta energia incidente sul punto \((i,j)\)dell'immagine quando non c'è la paziente; mentre \(\Phi^{H}(i,\ j)\) è l'intensità dovuta all'attenuazione all'interno della paziente con un fascio ad alta energia. Queste due quantità sono legate dalla seguente espressione:

\[\Phi^{H}(i,\ j) = \Phi_{0}^{H}(i,\ j)e^{- \mu_{b}^{H}(T - t) - \mu_{I}^{H}t}\]

Dove \(\mu_{b}^{H}\) è il coefficiente di attenuazione ad alta energia del tessuto mammario (\emph{Breast}); mentre \(\mu_{I}^{H}\) è il coefficiente di attenuazione dello iodio alla stessa energia; \(t\) è lo spessore del tessuto attraversato dal mezzo contrasto; \(T\) è lo spessore complessivo della mammella; \(T\  - \ t\) è, dunque, lo spessore del solo tessuto fibroghiandolare mammario. Tale relazione indica che \(\Phi^{H}\) ad alta energia nel punto \((i,j)\) è uguale alla quantità di raggi X che arriverebbero se non fosse presente la paziente, per l'esponenziale del coefficiente di attenuazione ad alta energia nello spessore del solo tessuto mammario e il coefficiente di attenuazione dello iodio, moltiplicati per i propri spessori. Un raggio X, attraversando il tessuto mammario, incontra uno spessore \(t\) di iodo e la restante porzione di tessuto fibroghiandolare. Entrambi i materiali attenuano la radiazione in maniera proporzionale ai rispettivi coefficienti di attenuazione.

A bassa energia si ritrova una situazione analoga, descritta dall'equazione:

\[\Phi^{L}(i,\ j) = \Phi_{0}^{L}(i,\ j)e^{- \mu_{b}^{L}(T - t) - \mu_{I}^{L}t}\]

Applicando i logaritmi ed effettuando una soppressione pesata, mediante un coefficiente di peso \(w_{I}\), per ogni pixel delle due immagini, si ottiene:

\[\log{\Phi^{H}(i,\ j)} - w_{I}\log{\Phi^{L}(i,\ j)} = \log{\Phi_{0}^{H}(i,\ j)e^{- \mu_{b}^{H}(T - t) - \mu_{I}^{H}t}} - w_{I}\log{\Phi_{0}^{L}(i,\ j)e^{- \mu_{b}^{L}(T - t) - \mu_{I}^{L}t} =}\]

\[= \log{\Phi_{0}^{H}(i,\ j)} - \mu_{b}^{H}(T - t) - \mu_{I}^{H}t - w_{I}\left( \log{\Phi_{0}^{L}(i,\ j)} - \mu_{b}^{L}(T - t) - \mu_{I}^{L}t \right)\]

Posto:

\[k = \log{\Phi_{0}^{H}(i,\ j)} - w_{I}\log{\Phi_{0}^{L}(i,\ j)}\]

Si ottiene l'equazione:

\[\log{\Phi^{H}(i,\ j)} - w_{I}\log{\Phi^{L}(i,\ j) =}k - \mu_{b}^{H}(T - t) - \mu_{I}^{H}t - w_{I}\left( - \mu_{b}^{L}(T - t) - \mu_{I}^{L}t \right)\]

La costante \(k\) dipende dall'intensità del fascio iniziale alle due energie e una serie di fattori dovuti ai differenti coefficienti di attenuazione del tessuto e del mezzo di contrasto.

Scegliendo opportunamente il peso come:

\[w_{I} = \ \frac{\mu_{b}^{H}}{\mu_{b}^{L}}\]

Si ottiene la cancellazione del tessuto mammario e nell'immagine risultante è mostrata soltanto la differenza dipendente dal mezzo di contrasto:

\[\log{\Phi^{H}(i,\ j)} - w_{I}\log{\Phi^{L}(i,\ j) =}k - \mu_{b}^{H}(T - t) - \mu_{I}^{H}t - w_{I}\left( - \mu_{b}^{L}(T - t) - \mu_{I}^{L}t \right) = = k - \mu_{b}^{H}(T - t) - \mu_{I}^{H}t - \ \frac{\mu_{b}^{H}}{\mu_{b}^{L}}\left( - \mu_{b}^{L}(T - t) - \mu_{I}^{L}t \right) = = k - \mu_{b}^{H}(T - t) - \mu_{I}^{H}t + \mu_{b}^{H}(T - t) + \frac{\mu_{b}^{H}}{\mu_{b}^{L}}\mu_{I}^{L}t = k - \mu_{I}^{H}t + w_{I}\mu_{I}^{L}t\]

In definitiva, riarrangiando l'equazione si ottiene:

\[\log{\Phi^{H}(i,\ j)} - w_{I}\log{\Phi^{L}(i,\ j) =}\ k - \left( \mu_{I}^{H} - w_{I}\mu_{I}^{L} \right)t\ \]

I coefficienti di attenuazione spesso non sono noti alle energie delle radiazioni utilizzate per ottenere l'\emph{Imaging} mammario. In questo caso, è necessario ricorrere a un'operazione di interpolazione (con la funzione \emph{interp1} in MatLab che riceve i punti effettivamente misurati e quali energie si vuole la funzione).

Alle energie di 33keV e 49keV (\emph{infoH.KVP} in DICOM), i tessuti mammari non sono ben visibili.

\begin{figure}
\centering
\includegraphics[width=4.29355in,height=3.22222in,alt={P5552\#yIS1}]{media/23_DualEn/image572.pdf}\caption{Figura .: Immagini delle mammelle a 33 e 49keV}
\end{figure}

Dovendo confrontare diverse immagini è importante che lo spessore della mammella non vari. In DICOM questa informazione si trova alla voce \emph{BodyPartThickness} e generalmente è di circa 90mm.

Le immagini ottenute con la metodica del \emph{Dual Energy Contrast-Enhanced Digital Mammography} dipendono dal coefficiente di peso scelto. Infatti, in base a come si sceglie la quantità \(w_{I}\) si ottengono immagini differenti. Se si pone tale quantità uguale a 0.7, effettivo rapporto tra i due coefficienti di assorbimento della mammella alle due diverse energie, si ottiene un'immagine che permette di evidenziare la presenza o meno di masse tumorali.

Ponendo \(w_{I} = 0\), l'immagine mostrata presenta una gradazione di grigio speculare rispetto alla classica visualizzazione delle immagini radiologiche.

Per avere un ottimo contrasto le energie con cui è irradiata la mammella sono scelte in prossimità del \emph{K-Edge} dello iodo, quindi, energie leggermente inferiori e superiori alla discontinuità del suo coefficiente di attenuazione.

\begin{figure}
\centering
\includegraphics[width=5.82672in,height=4.95513in,alt={P5558\#yIS1}]{media/23_DualEn/image573.pdf}\caption{Figura .: Da sinistra coefficiente di peso 0.0-7.1}
\end{figure}

\subsection{Attenuazione lineare}\label{attenuazione-lineare}

Per poter comprendere come funzioano le applicazioni in \emph{Dual Energy} è necessario studiare come i fotoni X siano attenuati dalla materia.

La relazione che lega l'intensità \(I\) del fascio di raggi X lungo il cammino \(S\) misurata con detettori:

\[I = \int_{}^{}{S(E)e^{- \int_{S}^{}{\mu(x,y,E)ds}}dE\ }\]

Dove \(\mu\) dipende dall'energia oltre che dalla posizione nel piano. L'integrale di \(\mu\) lungo una determinata direzione nel piano restituisce l'attenuazione complessiva a quel determinato valore di energia.

L'esponenziale dell'integrale interno restituisce proprio la quantità di raggi che incidono sul detettore. È necessario, poi, moltiplicare questa quantità per lo spettro energetico \(S(E)\) della sorgente al determinato valore di energia \(E\).

La scrittura:

\[S(E)e^{- \int_{S}^{}{\mu(x,y,E)ds}}\]

Fornisce l'attenuazione subita dai fotoni X con energia \(S(E)\) nell'attraversamento della materia con coefficiente lineare \(\mu(x,y,E)\).

Integrando su tutte le possibili energie si ottiene l'intensità del fascio emergente sul detettore, emesso dalla sorgente X, quando lo spettro è policromatico.

Considerando uno spettro \(S(E)\) costante, composto da fasci mono-energetici ad energia \(E\), l'intensità di radiazione emergente dal corpo del paziente può essere scritta come:

\[I = I_{0}e^{- \int_{S}^{}{\mu(x,y,E)dS}}\]

Dove risulta che:

\[I_{0} = \int_{}^{}{S(E)dE\ }\]

L'immagine all'energia \(E\) può essere ottenuta come:

\[M(E) = - \log\frac{I}{I_{0}} = \int_{S}^{}{\mu(x,y,E)dS}\]

Dove \(I_{0}\) è l'intensità di radiazione che investe il paziente, mentre \(I\) l'intensità emergente. Applicando la decomposizione in effetto Compton ed effetto fotoelettrico si scompone l'immagine come una somma di termini:

\[M(E) = \int_{S}^{}{\left( a_{p}(x,y)f_{p}(E) + \ a_{c}(x,y)f_{KN}(E) \right)dS}\]

Si ottiene, in defiitiva, una somma in cui la dipendenza dalle coordinate \((x,y)\) la si ritrova nei coefficienti \(a_{p}\) e \(a_{c}\), mentre la dipendenza dall'energia la si ha nelle due funzioni. È possibile, quindi, porre:

\[A_{p} = \int_{S}^{}{a_{p}(x,y)dS}\]

\[A_{c} = \int_{S}^{}{a_{c}(x,y)dS}\]

L'immagine può essere ottenuta come sovrapposizione dei due effetti:

\[M(E) = A_{p}f_{p}(E) + A_{c}f_{KN}(E)\]

In definitiva, l'immagine ad una certa energia \(E\) può essere decomposta in due immagini di base che, combinate mediante i pesi \(A_{p}\) e \(A_{c}\), permettono, appunto, di ottenere l'immagine originale.

Se il materiale fosse omogeneo allora risulterebbe che:

\[A_{p} = \int_{S}^{}{a_{p}(x,y)dS} = a_{c}L\]

\[A_{c} = \int_{S}^{}{a_{c}(x,y)dS} = a_{c}L\]

Rendendo la trattazione molto più semplice.

\subsection{\texorpdfstring{Esempi di applicazione di attenuazione in \emph{Dual Energy}}{Esempi di applicazione di attenuazione in Dual Energy}}\label{esempi-di-applicazione-di-attenuazione-in-dual-energy}

\subsubsection{Identificazione di un materiale}\label{identificazione-di-un-materiale}

Misurando il logaritmo dell'attenuazione di un materiale incognito \(x\) omogeneo di spessore noto \(L\) a due diverse energie, si avranno due immagini:

\[\left\{ \begin{array}{r}
M\left( E_{h} \right) = L\left( a_{p}^{x}f_{p}\left( E_{h} \right) + a_{c}^{x}f_{KN}\left( E_{h} \right) \right) \\
M\left( E_{l} \right) = L\left( a_{p}^{x}f_{p}\left( E_{l} \right) + a_{c}^{x}f_{KN}\left( E_{l} \right) \right)
\end{array} \right.\ \]

Le due costanti incognite \(a_{p}^{x}\) ed \(a_{c}^{x}\), caratteristiche del materiale sconosciuto, possono essere risolte facilmente mediante il sistema di due equazioni in due incognite.

Con questa metodica è possibile identificare il materiale note che siano le sue proprietà di attenuazione per due fasci monocromatici.

Tali metodiche sono utilizzate, ad esempio, nei detettori negli aeroporti che devono analizzare la composizione dei materiali all'interno di bagagli, supponendo che gli spessori non siano noti.

\subsubsection{Misura dello spessore del materiale}\label{misura-dello-spessore-del-materiale}

Dati due materiali noti, \(\alpha\) e \(\beta\), di due spessori incogniti differenti \(L_{\alpha}\) ed \(L_{\beta}\)\textsubscript{,} le due equazioni risultanti a due energie differenti per il logaritmo dell'attenuazione saranno:

\[\left\{ \begin{array}{r}
M\left( E_{h} \right) = L_{\alpha}\left( a_{p}^{\alpha}f_{p}\left( E_{h} \right) + a_{c}^{\alpha}f_{KN}\left( E_{h} \right) \right) + L_{\beta}\left( a_{p}^{\beta}f_{e}\left( E_{h} \right) + a_{c}^{\beta}f_{KN}\left( E_{h} \right) \right) \\
M\left( E_{l} \right) = L_{\alpha}\left( a_{p}^{\alpha}f_{p}\left( E_{l} \right) + a_{c}^{\alpha}f_{KN}\left( E_{l} \right) \right) + L_{\beta}\left( a_{p}^{\beta}f_{e}\left( E_{l} \right) + a_{c}^{\beta}f_{KN}\left( E_{l} \right) \right)
\end{array} \right.\ \]

Noti i materiali è possibile misurare lo spessore del mezzo di contrasto e, quindi, del tessuto ghiandolare tramite la loro composizione biochimica.

Questa metodica è sfruttata, ad esempio, in mammografia in cui, noti i coefficienti di attenuazione della mammella e del mezzo di contrasto, è possibile misurare lo spessore del tessuto ghiandolare e della zona perfusa dal tracciante.

\subsection{Basis Material Decomposition}\label{basis-material-decomposition}

La dipendenza energetica delle due funzioni \(f_{p}(E)\) e \(f_{KN}(E)\) è tale che, scegliendo due materiali predefiniti opportuni, uno con \(Z\) basso, rappresentante l'effetto Compton, e l'altro con \(Z\) alto, che esprime l'effetto fotoelettrico, si può decomporre il \(\frac{\mu}{\rho}\) di un qualsiasi materiale nella base costituita dai due materiali prescelti. Infatti, si considerano due materiali \(\alpha\) e \(\beta\), caratterizzati da \(\mu_{\alpha}(E)\) e \(\mu_{\beta}(E)\) scelti come materiale di base. Per ciascun dei due è possibile scrivere l'equazione di decomposizione come:

\[\left\{ \begin{array}{r}
\left( \frac{\mu}{\rho} \right)_{\alpha}(E) = a_{p}^{\alpha}f_{p}(E) + \ a_{c}^{\alpha}f_{KN}(E) \\
\left( \frac{\mu}{\rho} \right)_{\beta}(E) = a_{p}^{\beta}f_{p}(E) + \ a_{c}^{\beta}f_{KN}(E)
\end{array} \right.\ \]

Pertanto, il coefficiente di attenuazione massico per un materiale incognito \(x\) può essere decomposto secondo l'equazione:

\[\left( \frac{\mu}{\rho} \right)_{x}(E) = {c_{\alpha}^{x}\left( \frac{\mu}{\rho} \right)}_{\alpha}(E) + c_{\beta}^{x}\left( \frac{\mu}{\rho} \right)_{\beta}(E)\]

Ovvero un qualsiasi materiale incognito può essere espresso come combinazione dei due materiali noti, mediante i due coefficienti di attenuazione dei due materiali predeterminati, opportunamente pesanti. Una scrittura equivalentemente è la seguente:

\[\mu_{x}(E) = {c_{\alpha}^{x}\ \mu}_{\alpha}(E) + c_{\beta}^{x}\mu_{\beta}(E)\]

Per ottenere il valore dei pesi, si sostituisce il valore dei coefficienti di attenuazione massici dei materiali della base, espressi secondo le equazioni Klein e Nishina, nella scomposizione del materiale incognito. Ovvero:

\[\left( \frac{\mu}{\rho} \right)_{x}(E) = c_{\alpha}^{x}\left( a_{p}^{\alpha}f_{p}(E) + \ a_{c}^{\alpha}f_{KN}(E) \right) + c_{\beta}^{x}\left( a_{p}^{\beta}f_{p}(E) + \ a_{c}^{\beta}f_{KN}(E) \right) = = \left( c_{\alpha}^{x}a_{p}^{\alpha} + c_{\beta}^{x}a_{p}^{\beta} \right)f_{p}(E) + \left( c_{\alpha}^{x}a_{c}^{\alpha} + c_{\beta}^{x}a_{c}^{\beta} \right)f_{KN}(E)\]

Dunque, per il principio di identità dei polinomi, i pesi sono ottenuti come:

\[\left\{ \begin{array}{r}
a_{c}^{x} = c_{\alpha}^{x}a_{c}^{\alpha} + \ c_{\beta}^{x}a_{c}^{\beta} \\
a_{p}^{x} = c_{\alpha}^{x}a_{p}^{\alpha} + \ c_{\beta}^{x}a_{p}^{\beta}
\end{array} \right.\ \]

Il coefficiente di Compton del materiale \(x\) è dato, quindi, dalla somma pesata dei coefficienti dei due materiali di base e, analogamente, anche per il coefficiente relativo all'effetto fotoelettrico.

\subsubsection{Decomposizione in immagini di base}\label{decomposizione-in-immagini-di-base}

Si considera un materiale \(\xi\) di spessore \(L_{\xi}\). La sua proiezione radiografica può essere identificata con un vettore bidimensionale nella base dei due materiali prescelti:

\[\left\{ \begin{array}{r}
A_{\alpha}^{\xi} = \int_{S}^{}{c_{\alpha}^{\xi}(x,y)dS} \\
A_{\beta}^{\xi} = \int_{S}^{}{c_{\beta}^{\xi}(x,y)ds}
\end{array} \right.\ \]

Se i materiali scelti sono omogenei, allora è valida la semplificazione:

\[\left\{ \begin{array}{r}
A_{\alpha}^{\xi} = \int_{S}^{}{c_{\alpha}^{\xi}(x,y)dS} = \ c_{\alpha}^{\xi}L_{\xi} \\
A_{\beta}^{\xi} = \int_{S}^{}{c_{\beta}^{\xi}(x,y)ds} = \ c_{\beta}^{\xi}L_{\xi}
\end{array} \right.\ \]

L'immagine radiografica, alle due energie, può essere espressa come:

\[M^{\xi}(E) = A_{\alpha}^{\xi}\mu_{\alpha}^{\xi}(E) + A_{\beta}^{\xi}\mu_{\beta}^{\xi}(E)\]

Per ogni pixel dell'immagine, i due coefficienti \(A_{\alpha}^{\xi}\) e \(A_{\beta}^{\xi}\) diventano una coppia di coordinate o, equivalentemente, un numero complesso, dotato di modulo ed angolo.

La fase \(\theta^{\xi}\) del vettore \(\left( A_{\alpha}^{\xi},A_{\beta}^{\xi} \right)\ \)dipende solo dal materiale \(\xi\), secondo la relazione:

\[\theta^{\xi} = \tan^{- 1}\left( \frac{A_{\beta}^{\xi}}{A_{\alpha}^{\xi}} \right)\]

Se il mezzo è omogeneo allora:

\[\theta^{\xi} = \tan^{- 1}{\left( \frac{A_{\beta}^{\xi}}{A_{\alpha}^{\xi}} \right) = \tan^{- 1}\left( \frac{c_{\beta}^{\xi}L_{\xi}}{c_{\alpha}^{\xi}L_{\xi}} \right) = \tan^{- 1}\left( \frac{c_{\beta}^{\xi}}{c_{\alpha}^{\xi}} \right)}\]

Il modulo, invece, si esprime come:

\[m^{\xi} = \sqrt{\left( A_{\beta}^{\xi} \right)^{2} + \left( A_{\alpha}^{\xi} \right)^{2}}\]

Dunque, il modulo del vettore dipende sia dal materiale \(\xi\) che dal suo spessore. Se, quest'ultimo è omogeneo, allora:

\[m^{\xi} = \sqrt{\left( A_{\beta}^{\xi} \right)^{2} + \left( A_{\alpha}^{\xi} \right)^{2}} = \sqrt{\left( c_{\alpha}^{\xi}L_{\xi} \right)^{2} + \left( c_{\beta}^{\xi}L_{\xi} \right)^{2}\ } = L_{\xi}\sqrt{\left( c_{\alpha}^{\xi} \right)^{2} + \left( c_{\beta}^{\xi} \right)^{2}}\]

Il vantaggio della rappresentazione di un materiale incognito in termini di materiale di base risiede nella possibilità di identificare il primo materiale mediante dall'angolo che viene a formarsi nel piano.

È possibile una rappresentazione in forma complessa:

\[M^{\xi}(E) \approx \left( A_{\alpha}^{\xi},A_{\beta}^{\xi} \right) \approx m^{\xi}e^{j\theta^{\xi}}\]

La lunghezza del vettore nello stesso piano identifica lo spessore del materiale incognito considerato.

\subsubsection{Identificazione del materiale}\label{identificazione-del-materiale}

Si conseidera un materiale incognito \(x\) di spessore omogeneo. Si eseguono due proiezioni con energie differenti \(E_{h}\) e \(E_{l}\). Siano \(M\left( E_{h} \right)\) e \(M\left( E_{l} \right)\) le due immagini ottenute con le due differenti energie. Per ogni pixel dell'immagine è possibile scrivere la decomposizione in base di materiali:

\[\left\{ \begin{array}{r}
M\left( E_{h} \right) = A_{\alpha}^{x}\mu_{\alpha}\left( E_{h} \right) + A_{\beta}^{x}\mu_{\beta}\left( E_{h} \right) \\
M\left( E_{l} \right) = A_{\alpha}^{x}\mu_{\alpha}\left( E_{l} \right) + A_{\beta}^{x}\mu_{\beta}\left( E_{l} \right)
\end{array} \right.\ \]

Conoscendo i coefficienti di attenuazione lineare per i materiali di base alle due energie in questione, ovvero note le quantità \(\mu_{\alpha}\left( E_{h} \right)\), \(\mu_{\alpha}\left( E_{l} \right)\), \(\mu_{\beta}\left( E_{h} \right)\), \(\mu_{\beta}\left( E_{l} \right)\), è possibile calcolare, ad esempio con algoritmi OLS, le rappresentazioni vettoriali di ciascun pixel nella base dei materiali \(\left( A_{\alpha}^{x},A_{\beta}^{x} \right)\) poiché le altre quantità sono note, avendo eseguito le proiezioni e scelto i materiali della base. Le immagini risultanti sono dette immagini di base.

Ora, combinando le due immagini di base con due coefficienti dati da \(\sin \Phi\) e \(\cos \Phi\), l'immagine risultante sarà:

\[C = A_{\alpha}^{x}\cos \Phi + A_{\beta}^{x}\sin \Phi\]

La relazione scritta altro non è che il prodotto scalare dei due vettori \(\left( A_{\alpha}^{x},A_{\beta}^{x} \right)\), identificato nel piano dei materiali da un certo angolo \(\theta\), e il versore \(\left( \sin \Phi,\cos \Phi \right)\). Empiricamente, con l'uso di un calcolatore, si può variare \(\Phi\) fino a ottenere un prodotto scalare nullo, cioè un'immagine combinata \(C\) in cui il materiale in oggetto è rimosso. All'angolo \(\Phi\) per cui si ottiene ciò, i vettori \(\left( A_{\alpha}^{x},A_{\beta}^{x} \right)\) e \(\left( \sin \Phi,\cos \Phi \right)\) sono ortogonali e il materiale sconosciuto è cancellato. Pertanto, noto l'angolo \(\Phi\), è possibile risalire all'angolo \(\theta\) dalla relazione:

\[\theta + \Phi = \frac{\pi}{2}\]

\begin{figure}
\centering
\includegraphics[width=3.06072in,height=2.45in,alt={P5641\#yIS1}]{media/23_DualEn/image574.pdf}\caption{Figura .: Cancellazione del materiale \(\xi\)}
\end{figure}

E, da quest'ultimo è possibile ricavare il materiale, secondo la relazione \(\theta^{\xi} = \tan^{- 1}\left( \frac{c_{\beta}^{x}}{c_{\alpha}^{x}} \right)\). Ciò vale, ovviamente, se il materiale è unico.

\subsubsection{Identificazione con più materiali}\label{identificazione-con-piuxf9-materiali}

Nella proiezione di un oggetto composto da più materiali omogenei i vettori nella rappresentazione nei materiali di base si sommano. Infatti, presi due materiali \(\xi\) ed \(\eta\), siano \(L_{\xi}\) ed \(L_{\eta}\) i loro spessori. La proiezione è ottenuta integrando i coefficienti di assorbimento lungo lo spessore complessivo, ovvero:

\[M(E) = \int_{}^{}{\mu(x,y,E)ds} = \int_{\xi}^{}{\mu_{\xi}(E)d\xi} + \int_{\eta}^{}{\mu_{\eta}(E)d\eta} = \mu_{\xi}(E)L_{\xi} + \mu_{\eta}(E)L_{\eta}\]

Per cui, decomponendo nella base dei materiali per ogni componente dell'immagine, si ottiene:

\[M(E) = \left( A_{\alpha}^{\xi}\mu_{\alpha}(E) + A_{\beta}^{\xi}\mu_{\beta}(E) \right) + \left( A_{\alpha}^{\eta}\mu_{\alpha}(E) + A_{\beta}^{\eta}\mu_{\beta}(E) \right)\]

Raccogliendo, è possibile scrivere l'equazione come:

\[M(E) = \left( A_{\alpha}^{\xi} + A_{\alpha}^{\eta}\  \right)\mu_{\alpha}(E) + \left( A_{\beta}^{\xi} + A_{\beta}^{\eta} \right)\mu_{\beta}(E)\]

Da cui si evince che le componenti dell'immagine dell'oggetto composto è associata alla somma delle componenti degli oggetti corrispondenti.

Se si applica la combinazione delle immagini di base mediante la pesatura con \(\sin \Phi\) e \(\cos \Phi\) si ottiene:

\[C = \left( A_{\alpha}^{\xi} + A_{\alpha}^{\eta}\  \right)\cos \Phi + \left( A_{\beta}^{\xi} + A_{\beta}^{\eta} \right)\sin \Phi\]

Riarragiando si ha:

\[C = \left( A_{\alpha}^{\xi}\cos \Phi + A_{\beta}^{\xi}\sin \Phi \right) + \left( A_{\alpha}^{\eta}\cos \Phi + A_{\beta}^{\eta}\sin \Phi \right)\]

Pertanto, si può variare \(\Phi\) fino a ottenere la cancellazione di uno dei materiali, uno per volta, in maniera selettiva. Questo processo, in generale, può essere realizzato grazie alla riconoscibilità geometrica degli oggetti. A quel punto il valore \(\Phi\) corrispondente permette di valutare il materiale cancellato.

Mediante algoritmi ed elaborazioni digitali è possibile risalire ai materiali contenuti all'interno di un dato oggetto.

\section{Tipologie di tubi radiogeni}\label{tipologie-di-tubi-radiogeni}

In base al numero e la posizione dei tubi radiogeni si determinano le caratteristiche della strumentazione CT che esegue un esame diagnostico con la metodica del \emph{Dual Energy}. Vi sono essenzialmente tre tipologie di macchinari:

\begin{enumerate}
\def\labelenumi{\Alph{enumi})}
\item
  Il tubo radiogeno è unico ed è alimentato in maniera alternata a bassa ed alta energia, producendo rispettivamente raggi X monocromatici a frequenza maggiore e minore. Questa soluzione è la più semplice dal punto di vista costruttivo, ma risulta essere complicato il cambio del livello energetico in maniera rapida;
\item
  Vi possono essere due tubi ad energia fissa, disposti in maniera ortogonale tra loro. Questa architettura presenta lo svantaggio che le due immagini ricostruite sono sfasate. Dunque, vi sono problemi di sincronia;
\item
  Può esserci un unico tubo ad energia fissata, ad ampio spettro energetico, quindi, non monocromatico. I detettori, in questo caso, sono sensibili solo a un determinato valore di energia. Con questa soluzione si ottengono le due immagini differenti; tuttavia, il fascio erogato presenta un'intensità molto variabile e ciò comporta degli artefatti nell'immagine ricostruita.
\end{enumerate}

\begin{figure}
\centering
\includegraphics[width=6.65466in,height=3.75in,alt={P5665\#yIS1}]{media/23_DualEn/image575.pdf}\caption{Figura .: Schema dei possibili tubi radiogeni}
\end{figure}

Con queste strumentazioni è possibile eseguire la cancellazione delle strutture ossee in CT oppure delle immagini in pseudocolori indicati la distribuzione del tracciante all'interno del corpo umano.

In ogni caso, il contrasto tra i vari tessuti risulta essere molto migliore rispetto alle apparecchiature CT che utilizzano un solo fascio di raggi X monocromatico, grazie proprio all'utilizzo della metodica dei \emph{Dual Energy}.

Le tipologie di immagini che possono essere ricostruite sono le seguenti:

\begin{figure}
\centering
\includegraphics[width=6.54545in,height=6.3215in,alt={P5670\#yIS1}]{media/23_DualEn/image576.pdf}\caption{Figura .: Varie immagini ricostruite}
\end{figure}

\begin{center}
\vfill
    \chapter{Tecniche di visualizzazione delle immagini}
    \label{blx:VisualImm\therefsection}
\vfill

\minitoc
\newpage
\end{center}
\justify

\subsection{Rappresentazione multiplanare}\label{rappresentazione-multiplanare}

Molti programmi di visualizzazione delle immagini radiografiche, che siano esse CT, PET o MRI, permettono di analizzare un distretto anatomico lungo i tre assi in cui è possibile sezionare il paziente, ovvero assiale, frontale e sagitale. Questa rappresentazione è nota come \emph{Multiplanar Reconstruction} o MPR.

In MatLab è possibile implementare tale funzione mediante delle GUI in cui sono presenti tre finestre, nelle quali si visualizza l'immagine radiografica, sezionata rispetto a un piano, e uno \emph{Slider} che permette di scorrere una delle immagini lungo la direzione di taglio. Per rendere più chiara la visione, sono inseriti degli assi di riferimento che indicano la posizione della \emph{Slice} rispetto a quel piano di sezione.

\begin{figure}
\centering
\includegraphics[width=4.85924in,height=4.25833in,alt={P5681\#yIS1}]{media/24_VisualImm/image577.pdf}\caption{Figura .: Sezione anatomica con GUI}
\end{figure}

Per programmare una GUI è buona norma associare a ogni evento una o più azioni. Il paradigma di programmazione è basato sulla combinazione di azione dell'utente con \emph{Routine} eseguita corrispondente all'iterazione con l'esterno. A ogni azione, lo stato della GUI è aggiornato, con un meccanismo proprio della programmazione orientata agli oggetti.

La \emph{callback} contiene il codice da eseguire all'atto dell'interazione con l'utente. In questo caso, nella finestra di apertura vi è un pulsate che permette di scegliere il file immagine (\emph{uigetfile}) che poi verrà letto (\emph{load}). Con \emph{guidata} si aggiorna lo stato della GUI, ovviamente bisogna fornire l'oggetto (\emph{hObject}) e i dati salvati nella struttura dati nota come \emph{handles}.

Allo scorrere dello \emph{Slider}, la \emph{callback} aggiorna il valore corrente del piano, che sezione il paziente, relativo all'azione eseguita, e aggiorna le immagini nelle finestre coerentemente col piano considerato. Infine, si cambia la posizione delle linee rappresentanti la posizione della \emph{Slice} nel corpo del paziente, in base sempre alla posizione dei tre piani.

\subsection{Maximum Intensity Projection}\label{maximum-intensity-projection}

Una delle tecniche di visualizzazione molto utilizzate nelle applicazioni radiologiche è la \emph{Maximum Intensity Projection} o MIP, ovvero una proiezione a intensità massima. L'immagine mostrata è bidimensionale, con una determinata angolazione, ed è ottenuta proiettando i voxel con intensità massima lungo tutti i raggi di proiezione.

Dall'occhio umano partono dei raggi verso il volume osservato, e, nell'attraversare i tessuti, incontrano una serie di voxel di intensità differenti. Di questi voxel, poi, è visualizzato solamente quello a massima intensità. Ad esempio, in CT, osservando la zona cerebrale, il cranio risulta essere l'elemento più assorbente ed è, quindi, mostrato sull'immagine. Questa visualizzazione può essere ruotata in modo da vedere le proiezioni dei voxel con maggiore intensità secondo diverse angolazioni.

La metodica è molto utilizzata per poter ricostruire adeguatamente le strutture anatomiche, dunque, avere a disposizione un \emph{Volume Rendering} di tipo MIP.

\begin{longtable}[]{@{}
  >{\raggedright\arraybackslash}p{(\linewidth - 2\tabcolsep) * \real{0.5000}}
  >{\raggedright\arraybackslash}p{(\linewidth - 2\tabcolsep) * \real{0.5000}}@{}}
\caption{Figura .: Immagini di visione MIP per diversi angoli}\tabularnewline
\toprule\noalign{}
\begin{minipage}[b]{\linewidth}\centering
\includegraphics[width=3.13644in,height=3.275in,alt={P5691C1T13\#yIS1}]{media/24_VisualImm/image578.pdf}\end{minipage} & \begin{minipage}[b]{\linewidth}\centering
\includegraphics[width=3.1625in,height=3.3in,alt={P5692C2T13\#yIS1}]{media/24_VisualImm/image579.pdf}\end{minipage} \\
\midrule\noalign{}
\endfirsthead
\toprule\noalign{}
\begin{minipage}[b]{\linewidth}\centering
\includegraphics[width=3.13644in,height=3.275in,alt={P5691C1T13\#yIS1}]{media/24_VisualImm/image578.pdf}\end{minipage} & \begin{minipage}[b]{\linewidth}\centering
\includegraphics[width=3.1625in,height=3.3in,alt={P5692C2T13\#yIS1}]{media/24_VisualImm/image579.pdf}\end{minipage} \\
\midrule\noalign{}
\endhead
\bottomrule\noalign{}
\endlastfoot
\end{longtable}

A differenza della lastra radiografica tradizionale, il raggio ottico che attraversa il volume tridimensionale non esegue nessun integrale, quindi, non si proietta l'intero volume su un piano ma solamente i punti con intensità maggiore.

A volte è comodo semplicemente proiettare tutto il volume tridimensionale su un piano. Così facendo si ottiene un'immagine molto più simile a una lastra radiografica, che, tuttavia, può essere visualizzata secondo vari piani nello spazio. La metodica è detta \emph{X-Ray} e mostra immagini molto complesse del tipo:

\begin{figure}
\centering
\includegraphics[width=3.45498in,height=3.54762in,alt={P5697\#yIS1}]{media/24_VisualImm/image580.pdf}\caption{Figura .: Visione X-Ray}
\end{figure}

Dal punto di vista dell'implementazione, è necessario prelevare le coordinate dello \emph{Slider} dalle quali si ricava la matrice di rotazione. Con questa metodica è possibile ruotare il volume-paziente secondo vari angoli dipendenti da come l'operatore interagisce con la GUI.

Si calcolano poi le coordinate del volume ruotate e, infine, si esegue un'interpolazione (\emph{interp3(x,y,z,vol,cx,cy,cz)}) poiché è necessario valutare il coefficiente di assorbimento lineare nelle nuove coordinate.

Per la rappresentazione MIP bisogna trovare il massimo lungo una data direzione del raggio ottico, mentre per la visione \emph{X-Ray} è necessario eseguire una somma di tutte le intensità dei voxel lungo un piano.

\subsection{Tecniche di visualizzazioni quadridimensionali}\label{tecniche-di-visualizzazioni-quadridimensionali}

Una delle tecniche di visualizzazione delle immagini radiologiche, ideata negli ultimi 15-20 anni, riguarda la visualizzazione quadridimensionale. In particolare, invece di mostrare a video dei volumi tridimensionali, si acquisisce un volume più volte nel tempo. Al radiologo è, poi, mostrato come il volume varia nel tempo grazie a più acquisizioni.

La metodica dei quadrivettori è molto utilizzata con la tecnica del \emph{Dynamic Contrast Enhanced MRI}, in cui si segue l'evoluzione del contrasto iniettato per via endovenosa. Mostrando a video come il liquido di contrasto si distribuisce nel corpo umano è possibile evidenziare le zone riccamente vascolarizzate, indici di una probabile lesione neoplastica.

Una volta caricata la fetta, sulla base delle informazioni temporali è possibile costruire una mappa temporale che fornisce informazioni su come si distribuisce il contrasto. La mappa restituisce l'integrale della curva temporale, punto per punto. La GUI, poi, gestisce il passaggio su un voxel: ponendo il cursore su un voxel è visualizzata la curva temporale associata a quell'elemento costituente dell'immagine.

Il livello di concentrazione del mezzo di contrasto, dove la curva temporale è pressocché piatta, è costante nel tempo. Nelle regioni in cui si ha un aumento rapido, il mezzo di contrasto è stato assorbito velocemente nel tempo. La mappa è molto simile alla fetta della MRI all'istante iniziale di infusione del farmaco, tuttavia, presenta delle regioni anatomiche sovrapposte che non erano visibili inizialmente. Col passare del tempo, infatti, queste regioni assorbono una quantità di mezzo di contrasto sempre maggiore, diventando così sempre più brillanti in un \emph{Imaging} T1-pesato.

La visualizzazione dei dati temporali permette di riconoscere delle curve che presentano un andamento compatibile con il tipo di lesioni. Studi in letteratura hanno evidenziato che neoplasie molto aggressive presentano un andamento differente dalle lesioni meno pericolose.

Le lesioni benigne sono caratterizzate da una salita molto rapida, seguita poi da un \emph{Plateau} in cui il livello di contrasto resta all'incirca costante. Per le lesioni magline, invece, si assiste a una rapida salita del livello di contrasto seguita, poi, da una rapida discesa. Il flusso sanguigno nella lesione è così elevato che il liquido di contrasto è prelevato rapidamente, nella fase di \emph{Wash-In}, ed escreto altrettanto velocemente nella fase di \emph{Wash-Out}.

\begin{figure}
\centering
\includegraphics[width=6.39513in,height=3.72619in,alt={P5709\#yIS1}]{media/24_VisualImm/image581.pdf}\caption{Figura .: In alto Slice, in basso la mappa temporale a destra la curva temporale}
\end{figure}

In questo caso, la funzione principale carica e aggiorna la GUI in base ai movimenti del cursore sulla mappa temporale della \emph{Slice}. Per gestire il mouse si usa la funzione \emph{WindowsButtonMotionFcn}. Per la gestione del mouse, si preleva la posizione del cursore tramite \emph{CurrentPoint}, si arrotondano le coordinate e si determina se queste rientrano nella matrice contenente i dati. Se il cursore si muove sulla mappa temporale, le coordinate della sua posizione sono utilizzate per indicizzare la matrice delle curve temporali. Prelevati i dati della curva temporale associati al voxel selezionato, si procede col \emph{plot}.

La mappa temporale, invece, è situata nella \emph{callback} del pulsante ``Mappa'' ed è realizzata mediante una somma sulla dimensione temporale (3), di tutte le \emph{Slice} ottenuta a intervalli di tempo differenti.


\medskip

\newpage
\printbibheading[
heading=bibintoc,
title={Bibliografia}
]
\bibbysection[heading=bibbysubsect]

\end{document}
