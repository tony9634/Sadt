%!TEX bibfile = bibliografia.bib

%Questo è il preambolo, dove si inseriscono i pacchetti e le impostazioni che servono per compilare il documento. Quanto scritto dopo il simbolo '%' è solo un commento e serve a fini dimostrativi.
\documentclass[a4paper, 13, oneside]{report}
\linespread{1.5} %interlinea
\pagestyle{plain}
\usepackage{geometry} %margini
\geometry{a4paper, top=2.5cm, bottom=2cm, left=2.5cm, right=2cm, bindingoffset=5mm}
\usepackage{graphicx}

\usepackage{listings}
\usepackage{xcolor}

\lstset{
  language=Matlab,
  basicstyle=\ttfamily\small,
  keywordstyle=\color{blue},
  commentstyle=\color{green!60!black},
  stringstyle=\color{red},
  numbers=left,
  numberstyle=\tiny,
  stepnumber=1,
  numbersep=5pt,
  backgroundcolor=\color{gray!10},
  frame=single,
  breaklines=true,
  captionpos=b,
  showspaces=false,
  showstringspaces=false,
  showtabs=false,
  tabsize=2
}
\usepackage{multicol} %più colonne
\usepackage{ragged2e} %allineamento testo
\usepackage{caption}
\usepackage{pdfpages}
\usepackage{csquotes}
\usepackage{subfigure}
\usepackage{amsmath}
\usepackage{mathtools}
\usepackage{amssymb}
\usepackage{esint}
\usepackage{pgfplotstable}
\usepackage{pgfplots}
\pgfplotsset{compat=1.7}
\captionsetup{textfont=sl}
\maxdeadcycles=300

\usepackage{morefloats}

\usepackage[italian]{babel} %lingua principale

\usepackage{minitoc} %mini sommario a inizio capitolo
\nomtcrule 
\addto{\captionsitalian}{% Making babel aware of special titles
  \renewcommand{\mtctitle}{Sommario}
}
\setcounter{secnumdepth}{3}
\usepackage[sorting=none,backend=biber,refsection=chapter]{biblatex} %bibliografia
\defbibheading{bibbysubsect}{}
\addbibresource{bibliografia.bib}


\usepackage{comment}
\usepackage{xpatch}
\usepackage{blindtext}


\makeatletter

\xpatchcmd{\@makeschapterhead}{%
  \Huge \bfseries  #1\par\nobreak%
}{%
  \Huge \bfseries\centering #1\par\nobreak%
}{\typeout{Patched makeschapterhead}}{\typeout{patching of @makeschapterhead failed}}

\xpatchcmd{\@makechapterhead}{%
  \huge\bfseries \@chapapp\space \thechapter
}{%
  \huge\bfseries\centering \@chapapp\space \thechapter
}{\typeout{Patched @makechapterhead}}{\typeout{Patching of @makechapterhead failed}}

\makeatother
\usepackage{csvsimple}
\usepackage{fancyhdr}
\usepackage[export]{adjustbox}

\usepackage{hyperref} %hyperlink
\hypersetup{
    colorlinks=true,
    citecolor=cyan,
    linkcolor=black,
    urlcolor=black,
    pdftitle=Appunti del Corso Di Strumentazione Avanzata per La Diagnosi e Terapia,
    pdfauthor=Antonio Ausilio,
    }

\usepackage{booktabs} %per le tabelle
\usepackage{multirow}
%\usepackage[table,xcdraw]{xcolor}
\usepackage{graphicx}

\usepackage{lscape}
\usepackage{float}
\usepackage{wrapfig}
\usepackage{listings}

\usepackage{longtable}
\usepackage{array}
\usepackage{enumitem}

\newcolumntype{C}[1]{>{\centering\arraybackslash}p{#1}}
%\renewcommand{\arraystretch}{1.4} % <-- Aggiunge spazio tra le righe

%Da qui in poi inizia il documento.
\begin{document}
\begin{center}
    \Huge\textbf{Università degli Studi di Napoli “Federico II”}
    
    \vspace{1cm}

    \includegraphics[width=0.40\textwidth]{media/Logo.pdf}\\[1cm]
    
    \Huge\textbf{Appunti del Corso Di Strumentazione Avanzata per La Diagnosi e Terapia}
    
    \vspace{1cm}
    
    \large A.A. 2019/2020
    
    \vspace{1cm}
    
    \textbf{Docente: Mario Sansone}\\
    (Beppe Vessicchio)
    
    \vfill
    
    \begin{flushright}
    \textbf{A cura di}\\
    \textit{
    Abagnale Francesca M54000961\\
    Ausilio Antonio M54000985\\
    Senese Cristian M54000955}
    \end{flushright}
\end{center}

\newpage


\dominitoc
\tableofcontents

\begin{center}
\vfill
    \chapter{Meccanica classica}
    \label{blx:refsection\therefsection}
\vfill

\minitoc
\newpage
\end{center}
\justify

\section{Cenni di meccanica classica}\label{cenni-di-meccanica-classica}

La \textbf{meccanica} è la branca della fisica che studia il moto dei corpi materiali \cite{landau1994meccanica}. In base alle caratteristiche fisiche della materia considerata, sono state sviluppate diverse teorie meccaniche, suddivise principalmente in:

\begin{itemize}
\item
 \textbf{Meccanica classica}: descrive sistemi di dimensioni macroscopiche che si muovono a velocità molto inferiori rispetto a quella della luce;
\item
 \textbf{Meccanica statistica}: applicabile a sistemi costituiti da un numero elevato di particelle, delle quali si analizzano le proprietà medie;
\item
 \textbf{Meccanica relativistica}: tratta sistemi non quantistici che si muovono a velocità prossime a quella della luce;
\item
 \textbf{Meccanica quantistica}: si occupa di sistemi su scala atomica e subatomica, dove gli effetti quantistici risultano predominanti.
\end{itemize}

\section{Meccanica newtoniana}\label{meccanica-newtoniana}
La \textbf{meccanica classica} si fonda sulla descrizione dei fenomeni fisici secondo l'approccio introdotto da Newton nel XVII secolo. Essa interpreta il moto della natura in termini di forze \(\vec{F}\) e accelerazioni \(\vec{a}\). Il punto di vista \textbf{newtoniano è di tipo locale}: conoscendo le forze agenti su una particella e il loro andamento temporale, è possibile determinare il moto di quest'ultima.

L'equazione fondamentale della meccanica newtoniana è:

\[\vec{F} = m\vec{a}\]

dove \(m\) è la massa della particella (assunta costante) e \(\vec{a}\) è la sua accelerazione. L'accelerazione è definita come la derivata della velocità rispetto al tempo:

\[\vec{a} = \dfrac{d\vec{v}}{dt}\]

Si definisce la \textbf{quantità di moto} (o \textbf{momento lineare}) \(\vec{p}\) come il prodotto tra la massa e la velocità della particella:

\[\vec{p} = m\vec{v}\]

Il vettore \(\vec{p}\) ha la stessa direzione e verso del vettore velocità \(\vec{v}\). L'equazione fondamentale della meccanica può essere riscritta in termini di quantità di moto:

\[\vec{F} = m\vec{a} = m\dfrac{d\vec{v}}{dt} = \dfrac{d}{dt}(m\vec{v}) = \dfrac{d\vec{p}}{dt}\]

Questa forma è valida anche nel caso in cui la massa \(m\) non sia costante nel tempo, ad esempio, nel caso di corpi che perdono massa come una navetta spaziale.

L'equazione può essere estesa a un sistema di \(n\) particelle, con masse \(\{ m_{1},m_{2},\ldots,m_{n}\}\) e forze \(\{{\vec{f}}_{1},{\vec{f}}_{2},\ldots,{\vec{f}}_{n}\}\), ottenendo:

\[
{\vec{F}}_{\text{tot}} = \sum_{k = 1}^{n}{\vec{f}}_{k} = \sum_{k = 1}^{n}\dfrac{d{\vec{p}}_{k}}{dt}
\]

Questa espressione rappresenta il teorema della \textbf{dinamica dei sistemi di particelle}, il quale afferma che la somma delle forze agenti su un sistema di particelle è uguale alla derivata temporale della somma delle quantità di moto delle singole particelle.

\begin{figure}[ht]
\centering
\includegraphics[width=2.65748in,height=2.12598in,alt={Immagine che contiene linea Il contenuto generato dall'IA potrebbe non essere corretto.}]{media/1_Meccanica/image2.pdf}\caption{Sistema di particelle soggetto a forze esterne}
\end{figure}

Per una particella soggetta a una forza \(\vec{F}\), si definisce \textbf{momento torcente} (o \textbf{momento della forza}) \(\vec{N}\) (talvolta indicato anche con \(\vec{\tau}\)):

\[\vec{N} = \vec{r} \times \vec{F}\]

dove \(\vec{r}\) è il vettore posizione della particella rispetto a un polo (origine del sistema di riferimento).

\begin{figure}[ht]
\centering
\includegraphics[width=2.62626in,height=1.47606in,alt={Immagine che contiene linea, schizzo, diagramma Il contenuto generato dall'IA potrebbe non essere corretto.}]{media/1_Meccanica/image3.pdf}\caption{Definizione del momento torcente rispetto a un punto fisso}
\end{figure}

Analogamente, si definisce \textbf{momento angolare} (o \textbf{quantità di moto angolare}) \(\vec{L}\):

\[\vec{L} = \vec{r} \times \vec{p}\]

\begin{figure}[ht]
\centering
\includegraphics[width=2.71741in,height=1.60327in,alt={Immagine che contiene testo, diagramma, design Il contenuto generato dall'IA potrebbe non essere corretto.}]{media/1_Meccanica/image4.pdf}\caption{Momento angolare di una particella rispetto a un'origine fissa}
\end{figure}

Sostituendo la definizione di quantità di moto:

\[\vec{L} = \vec{r} \times (m\vec{v}) = m(\vec{r} \times \vec{v})\]

Applicando la derivata rispetto al tempo:

\[\begin{aligned}
\dfrac{d\vec{L}}{dt} & = \dfrac{d}{dt}(\vec{r} \times \vec{p}) = \dfrac{d\vec{r}}{dt} \times \vec{p} + \vec{r} \times \dfrac{d\vec{p}}{dt}
\end{aligned}\]

Nel caso in cui il polo sia fisso:

\[\dfrac{d\vec{r}}{dt} \times \vec{p} = \vec{v} \times \vec{p} = \vec{0}\]

perché \(\vec{v}\) e \(\vec{p}\) sono paralleli. Resta quindi:

\[\dfrac{d\vec{L}}{dt} = \vec{r} \times \dfrac{d\vec{p}}{dt} = \vec{r} \times \vec{F} = \vec{N}\]

Questa equazione prende il nome di \textbf{teorema del momento angolare}.

Per un sistema di particelle, il momento torcente totale è:

\[{\vec{N}}_{\text{tot}} = \sum_{k = 1}^{n}{{\vec{r}}_{k} \times {\vec{f}}_{k}} = \sum_{k = 1}^{n}{{\vec{r}}_{k} \times \dfrac{d{\vec{p}}_{k}}{dt}} = \dfrac{d\vec{L}}{dt}\]

Conoscendo le forze agenti, è possibile determinare il moto della particella integrando l'equazione fondamentale:

\[\vec{F} = \dfrac{d\vec{p}}{dt}\]

Integrando nel tempo tra un istante iniziale \(t_{0}\) e un istante generico \(t\), si ottiene:

\[\int_{t_{0}}^{t}{\vec{F}\, dt} = \vec{p}(t) - \vec{p}(t_{0}) = m\lbrack\vec{v}(t) - \vec{v}(t_{0})\rbrack\]

Assumendo \(\vec{v}(t_{0}) = \vec{0}\), si ha:

\[\int_{t_{0}}^{t}{\vec{F}\, dt} = m\vec{v}(t)\]

Integrando una seconda volta, si ottiene lo spostamento:

\[\int_{t_{0}}^{t}{\left( \int_{t_{0}}^{t'}{\vec{F}\, dt''} \right)dt'} = m\lbrack\vec{s}(t) - \vec{s}(t_{0})\rbrack\]

dove \(\vec{s}(t)\) è il vettore spostamento.

Il modello newtoniano descrive accuratamente molti fenomeni osservati nella sua epoca, come il moto dei pianeti, il comportamento dei corpi sotto l'azione di forze, e le interazioni meccaniche quotidiane.

\section{Meccanica lagrangiana}\label{meccanica-lagrangiana}
La \textbf{meccanica lagrangiana} è una formulazione matematica della meccanica introdotta nel XVIII secolo da Joseph-Louis Lagrange, come riformulazione della meccanica newtoniana \cite{arnold1992matematici, landau1994meccanica}.

Questa descrizione parte da un punto di vista globale e si propone di determinare il moto di un sistema minimizzando una funzione chiamata \emph{azione}, che dipende dall'intero percorso del sistema. Il modello lagrangiano è particolarmente utile, poiché consente di descrivere non solo fenomeni della meccanica classica, ma anche situazioni della \textbf{meccanica quantistica}.

Consideriamo un sistema composto da \(N\) particelle. La descrizione del loro moto secondo la meccanica newtoniana richiede l'uso di uno spazio tridimensionale cartesiano: a ciascuna particella \(m_{i}\) è associata una terna di coordinate \((x_{i},y_{i},z_{i})\), che variano nel tempo. Questo approccio porta alla necessità di risolvere \(3N\) equazioni differenziali per determinare il moto di tutte le particelle.

Tuttavia, spesso le particelle sono soggette a vincoli che limitano il loro moto a determinate traiettorie o superfici. In questi casi, è possibile descrivere il sistema utilizzando \emph{coordinate generalizzate} \(q_{i}\), con \(i = 1,2,\ldots,s\), dove \(s\) rappresenta il numero dei \emph{gradi di libertà} del sistema.

Tra tutte le curve che collegano un punto \(\vec{A}\) al tempo \(t_{0}\) con un altro punto \(\vec{B}\) al tempo \(t_{1}\), esiste una traiettoria unica che rende stazionaria l'azione, ovvero l'integrale della funzione lagrangiana nel tempo.

\begin{figure}[ht]
\centering
\includegraphics[width=3.9759in,height=1.98795in,alt={Immagine che contiene calligrafia, linea Il contenuto generato dall'IA potrebbe non essere corretto.}]{media/1_Meccanica/image5.pdf}\caption{Esempio di moto su traiettoria a azione stazionaria tra due punti nel tempo.}
\end{figure}
\subsection{Lemma 1: Principio di Azione Stazionaria di Hamilton}\label{lemma-1-principio-di-minimizzazione}

Il moto della particella, ovvero la traiettoria \(q_{i}(t)\) e la velocità con cui essa viene percorsa \({\dot{q}}_{i}(t)\), deve rendere stazionario l'integrale \cite{arnold1992matematici}:

\[S = \int_{t_{0}}^{t_{1}}{L\left( q_{i},{\dot{q}}_{i},t \right)\, dt}\]

L'integrale \(S\) è detto \emph{azione}, \(t\) è la variabile temporale, mentre \(L\) è la funzione \emph{lagrangiana}, o semplicemente \emph{lagrangiana}, del sistema di particelle. Dal punto di vista dimensionale, la lagrangiana è omogenea all'energia, e quindi ha le stesse dimensioni del joule:

\[\lbrack L\rbrack = \lbrack J\rbrack\]

dove \(\lbrack L\rbrack\) indica le dimensioni fisiche della lagrangiana e \(\lbrack J\rbrack\) quelle dell'energia, ovvero:

\[\lbrack J\rbrack = kg \cdot m^{2} \cdot s^{- 2}\]

\subsection{Lemma 2: equazione di Eulero-Lagrange}\label{lemma-2-equazione-di-eulero-lagrange}

Per descrivere il moto di una particella o di un sistema di particelle, la lagrangiana deve soddisfare l'equazione di Eulero-Lagrange \cite{landau1994meccanica}:

\begin{align*}
\dfrac{d}{dt}\left(\dfrac{\partial L}{\partial\dot{q}_i}\right) - \dfrac{\partial L}{\partial q_i} &= 0, \ i=1,2,\dots,s
\end{align*}

Poiché l'azione \(S\) deve essere stazionaria, una sua variazione \(\delta S\), dovuta a una perturbazione dello spostamento \(\delta\vec{q}\) e della velocità \(\delta\dot{\vec{q}}\), è tale che:

\[S + \delta S = \int_{t_{0}}^{t_{1}}{L\left( \vec{q} + \delta\vec{q},\dot{\vec{q}} + \delta\dot{\vec{q}} \right)dt}\]

Dove \(\vec{q}\) è una traiettoria che rende stazionaria l'azione \(S\). Il Principio di Hamilton (o di Minima Azione) stabilisce che la traiettoria rende l'azione stazionaria o un estremo (ovvero $\delta S=0$). Non è garantito che sia un minimo poiché può essere un massimo o un punto di sella.

Siccome le variazioni di spostamento e velocità sono molto minori delle rispettive qualità:

\[
\delta q \ll q,\ \delta\dot{q} \ll \dot{q}
\]

è possibile sviluppare la lagrangiana in serie di Taylor, arrestando lo sviluppo al primo ordine, nell'intorno del punto \(\left( \vec{q},\dot{\vec{q}} \right)\):

\[L\left( \vec{q} + \delta\vec{q},\dot{\vec{q}} + \delta\dot{\vec{q}} \right) \simeq L\left( \vec{q},\dot{\vec{q}} \right) + \vec{\nabla}L \cdot \left( \delta\vec{q},\delta\dot{\vec{q}} \right) =\]

Per definizione di gradiente si ha:

\[= L\left( \vec{q},\dot{\vec{q}} \right) + \left( \dfrac{\partial L}{\partial\vec{q}},\dfrac{\partial L}{\partial\dot{\vec{q}}} \right) \cdot \left( \delta\vec{q},\ \delta\dot{\vec{q}} \right) =\]

Svolgendo l'operazione di prodotto scalare si ha:

\[= L\left( \vec{q},\dot{\vec{q}} \right) + \dfrac{\partial L}{\partial\vec{q}}\delta\vec{q} + \dfrac{\partial L}{\partial\dot{\vec{q}}}\ \delta\dot{\vec{q}}\]

Dove il gradiente è un vettore colonna mentre le coordinate generalizzate dei vettori riga. Sostituendo nell'espressione della variazione dell'azione, si ottiene:

\[S + \delta S = \int_{t_{0}}^{t_{1}}{\left\lbrack L\left( \vec{q},\dot{\vec{q}} \right) + \dfrac{\partial L}{\partial\vec{q}}\delta\vec{q} + \dfrac{\partial L}{\partial\dot{\vec{q}}}\ \delta\dot{\vec{q}} \right\rbrack dt}\]

Per la linearità dell'integrale si scrive:

\[S + \delta S = \int_{t_{0}}^{t_{1}}{L\left( \vec{q},\dot{\vec{q}} \right)dt} + \int_{t_{0}}^{t_{1}}{\left( \dfrac{\partial L}{\partial\vec{q}}\delta\vec{q} + \dfrac{\partial L}{\partial\dot{\vec{q}}}\ \delta\dot{\vec{q}} \right)dt}\]

Dove:

\[S = \int_{t_{0}}^{t_{1}}{L\left( \vec{q},\dot{\vec{q}} \right)dt}\]

Da cui si ottiene:

\[S + \delta S = S + \int_{t_{0}}^{t_{1}}{\left( \dfrac{\partial L}{\partial\vec{q}}\delta\vec{q} + \dfrac{\partial L}{\partial\dot{\vec{q}}}\ \delta\dot{\vec{q}} \right)dt}\]

Semplificando \(S\), si ottiene un'espressione per la variazione dell'azione \(\delta S\):

\[\delta S = \int_{t_{0}}^{t_{1}}{\left( \dfrac{\partial L}{\partial\vec{q}}\delta\vec{q} + \dfrac{\partial L}{\partial\dot{\vec{q}}}\ \delta\dot{\vec{q}} \right)dt}\]

La traiettoria perturbata \(\delta\vec{q}\) ha in comune con la traiettoria \(\vec{q}\) il punto iniziale \(\vec{A}\) all'istante \(t_{0}\) e il punto di fine \(\vec{B}\) all'istante \(t_{1}\), ne discende che:

\[\begin{cases}
\vec{q}\left( t_{0} \right) = \vec{q}\left( t_{0} \right) + \delta\vec{q}\left( t_{0} \right) = \vec{A} \\
\vec{q}\left( t_{1} \right) = \vec{q}\left( t_{1} \right) + \delta\vec{q}\left( t_{1} \right) = \vec{B}
\end{cases} \]

Non variando il punto iniziale, risulta che nei punti iniziali non vi sono perturbazioni:

\[\delta\vec{q}\left( t_{0} \right) = \vec{0},\ \ \delta\vec{q}\left( t_{1} \right) = \vec{0}\]

\begin{figure}[ht]
\centering
\includegraphics[width=5.17616in,height=2.58808in,alt={Immagine che contiene Carattere, linea, diagramma Il contenuto generato dall'IA potrebbe non essere corretto.}]{media/1_Meccanica/image6.pdf}\caption{Perturbazione della traiettoria}
\end{figure}

Se risulta che:

\[\dot{\vec{q}} = \dfrac{d\vec{q}}{dt}\]

Allora, deve accadere che:

\[\delta\dot{\vec{q}} = \dfrac{d}{dt}\left( \delta\vec{q} \right)\]

Dunque, la variazione di azione può essere espressa come:

\[\delta S = \int_{t_{0}}^{t_{1}}{\left( \dfrac{\partial L}{\partial\vec{q}}\delta\vec{q} + \dfrac{\partial L}{\partial\dot{\vec{q}}}\ \delta\dot{\vec{q}} \right)dt} = \int_{t_{0}}^{t_{1}}{\dfrac{\partial L}{\partial\vec{q}}\delta\vec{q}dt} + \int_{t_{0}}^{t_{1}}{\dfrac{\partial L}{\partial\dot{\vec{q}}}\ \dfrac{d}{dt}\left( \delta\vec{q} \right)dt}\]

Si considera la quantità:

\[\dfrac{d}{dt}\left( \dfrac{\partial L}{\partial\dot{\vec{q}}}\delta\vec{q} \right)\]

Questa può essere riscritta ricorrendo alle proprietà del prodotto:

\[\dfrac{d}{dt}\left( \dfrac{\partial L}{\partial\dot{\vec{q}}}\delta\vec{q} \right) = \left( \dfrac{d}{dt}\dfrac{\partial L}{\partial\dot{\vec{q}}} \right)\delta\vec{q} + \dfrac{\partial L}{\partial\dot{\vec{q}}}\dfrac{d}{dt}\left( \delta\vec{q} \right)\]

Da cui è possibile ricavare:

\[\dfrac{\partial L}{\partial\dot{\vec{q}}}\dfrac{d}{dt}\left( \delta\vec{q} \right) = \dfrac{d}{dt}\left( \dfrac{\partial L}{\partial\dot{\vec{q}}}\delta\vec{q} \right) - \left( \dfrac{d}{dt}\dfrac{\partial L}{\partial\dot{\vec{q}}} \right)\delta\vec{q}\]

Sostituendo nel secondo integrale della variazione dell'azione si ottiene:

\[\delta S = \int_{t_{0}}^{t_{1}}{\left\lbrack \dfrac{\partial L}{\partial\vec{q}}\delta\vec{q} + \dfrac{d}{dt}\left( \dfrac{\partial L}{\partial\dot{\vec{q}}}\delta\vec{q} \right) - \left( \dfrac{d}{dt}\dfrac{\partial L}{\partial\dot{\vec{q}}} \right)\delta\vec{q} \right\rbrack dt}\]

Raccogliendo \(\delta\vec{q}\) tra il primo e l'ultimo termine, si ha:

\[\delta S = \int_{t_{0}}^{t_{1}}{\left\lbrack \left( \dfrac{\partial L}{\partial\vec{q}} - \dfrac{d}{dt}\dfrac{\partial L}{\partial\dot{\vec{q}}} \right)\delta\vec{q} + \dfrac{d}{dt}\left( \dfrac{\partial L}{\partial\dot{\vec{q}}}\delta\vec{q} \right) \right\rbrack dt} = \int_{t_{0}}^{t_{1}}{\left( \dfrac{\partial L}{\partial\vec{q}} - \dfrac{d}{dt}\dfrac{\partial L}{\partial\dot{\vec{q}}} \right)\delta\vec{q}dt} + \int_{t_{0}}^{t_{1}}{\dfrac{d}{dt}\left( \dfrac{\partial L}{\partial\dot{\vec{q}}}\delta\vec{q} \right)dt}\]

Si considera l'ultimo integrale. Risulta che:

\[\int_{t_{0}}^{t_{1}}{\dfrac{d}{dt}\left( \dfrac{\partial L}{\partial\dot{\vec{q}}}\delta\vec{q} \right)dt} = \left. \ \ \dfrac{\partial L}{\partial\dot{\vec{q}}}\delta\vec{q} \right|_{t_{0}}^{t_{1}}\ \]

Ma, poiché \(\delta\vec{q}\left( t_{0} \right) = \delta\vec{q}\left( t_{1} \right) = \vec{0}\), l'integrale è nullo. In definitiva, si ottiene:

\[\delta S = \int_{t_{0}}^{t_{1}}{\left( \dfrac{\partial L}{\partial\vec{q}} - \dfrac{d}{dt}\dfrac{\partial L}{\partial\dot{\vec{q}}} \right)\delta\vec{q}dt}\]

Dato che l'azione deve essere stazionaria, la sua variazione deve essere nulla \(\delta S = 0\). Per cui si ha:

\[\int_{t_{0}}^{t_{1}}{\left( \dfrac{\partial L}{\partial\vec{q}} - \dfrac{d}{dt}\dfrac{\partial L}{\partial\dot{\vec{q}}} \right)\delta\vec{q}dt} = 0\]

Se l'integrale è nullo, allora la funzione integranda deve essere nulla:

\[\left( \dfrac{\partial L}{\partial\vec{q}} - \dfrac{d}{dt}\dfrac{\partial L}{\partial\dot{\vec{q}}} \right)\delta\vec{q} = 0\]

Se la variazione della traiettoria, \(\delta\vec{q}\), è non nulla, si ritrova l'equazione di Eulero-Lagrange.

\subsubsection{Particella in coordinate cartesiane}\label{particella-in-coordinate-cartesiane}

Per una \textbf{particella libera} (cioè in assenza di forze conservative, dove l'energia potenziale è nulla) in moto in uno spazio cartesiano, la lagrangiana \(L\) è data da:

\[
L = \dfrac{1}{2}mv^{2}
\]

Dove

\[
v^{2} = {\dot{x}}^{2} + {\dot{y}}^{2} + {\dot{z}}^{2}
\]

In altre parole, in questo caso la lagrangiana coincide con l'energia cinetica della particella. Inoltre, la descrizione lagrangiana si riduce alla modellazione newtoniana, dato che il sistema di riferimento corrisponde con quello cartesiano.

Tuttavia, nella pratica non è sempre conveniente descrivere il moto in coordinate cartesiane. In presenza di vincoli o geometrie particolari, è utile ricorrere alle coordinate generalizzate \(\vec{q}\) e \(\dot{\vec{q}}\). È quindi necessario esprimere l'energia del sistema in funzione di queste coordinate.

\subsection{Lemma 3: energia cinetica in coordinate generalizzate}\label{lemma-3-energia-cinetica-in-generalizzate}

Per un sistema di \(N\) particelle nelle coordinate generalizzate \(\vec{q}\) e \(\dot{\vec{q}}\), l'energia cinetica è:

\[
T = T\left( q_{i},{\dot{q}}_{i} \right),\ \ i = 1,\ \ldots,N
\]

Ovvero l'energia cinetica è una funzione anche delle coordinate generalizzate \(q_{i}\) oltre che della velocità \({\dot{q}}_{i}\).

Si considerano \(N\) funzioni \(f_{i}\) che legano le coordinate cartesiane \(\left( x_{i},y_{i},z_{i} \right)\) con le coordinate generalizzate \(\left( q_{1},q_{2},\ldots,q_{s} \right)\). Ad esempio, si considera:

\[
x_{i} = f_{i}\left( q_{1},\ q_{2},\ \ldots,\ q_{s} \right),\ \ i = 1,2,\ldots,N
\]

Dove:

\[q_{k} = q_{k}(t),\ k = 1,2,\ldots,s\]

Si deriva \(x_{i}\) rispetto al tempo al fine da ottenere la velocità:

\[{\dot{x}}_{i} = \dfrac{d}{dt}f_{i}\left( q_{1},q_{2},\ldots,\ q_{s} \right),\ \ i = 1,2,\ldots,N\]

Per la derivata della funzione composta si ha:

\[{\dot{x}}_{i} = \dfrac{d}{dt}f_{i}\left( q_{1},\ q_{2},\ \ldots,\ q_{s} \right) = \sum_{k = 1}^{s}{\dfrac{\partial f_{i}}{\partial q_{k}}\dfrac{dq_{k}}{dt}} = \sum_{k = 1}^{s}{\dfrac{\partial f_{i}}{\partial q_{k}}{\dot{q}}_{k}}\]

Si eleva al quadrato \({\dot{x}}_{i}\), ottenendo:

\[{\dot{x}}_{i}^{2} = \left( \sum_{k = 1}^{s}{\dfrac{\partial f_{i}}{\partial q_{k}}{\dot{q}}_{k}} \right)^{2} = \sum_{k = 1}^{s}{\dfrac{\partial f_{i}}{\partial q_{k}}{\dot{q}}_{k}}\sum_{k = 1}^{s}{\dfrac{\partial f_{i}}{\partial q_{k}}{\dot{q}}_{k}}\]

È possibile esprimere il secondo membro come doppia sommatoria sugli indici \(k\) e \(j\):

\[{\dot{x}}_{i}^{2} = \sum_{k = 1}^{s}{\sum_{j = 1}^{s}{\dfrac{\partial f_{i}}{\partial q_{j}}{\dot{q}}_{j}\dfrac{\partial f_{i}}{\partial q_{k}}{\dot{q}}_{k}}} = \sum_{k = 1}^{s}{\sum_{j = 1}^{s}{\dfrac{\partial f_{i}}{\partial q_{j}}\dfrac{\partial f_{i}}{\partial q_{k}}{\dot{q}}_{j}}{\dot{q}}_{k}}\]

L'energia cinetica totale del sistema è:

\[T = \dfrac{1}{2}\sum_{i = 1}^{N}{m_{i}{\dot{x}}_{i}}^{2} = \dfrac{1}{2}\sum_{i = 1}^{N}m_{i}\sum_{k = 1}^{s}{\sum_{j = 1}^{s}{\dfrac{\partial f_{i}}{\partial q_{j}}\dfrac{\partial f_{i}}{\partial q_{k}}{\dot{q}}_{j}}{\dot{q}}_{k}}\]

Si definiscono coefficienti metrici:

\[a_{kj}(q) = \sum_{i = 1}^{N}{m_{i}\dfrac{\partial f_{i}}{\partial q_{j}}\dfrac{\partial f_{i}}{\partial q_{k}}}\]

Allora l'energia cinetica può essere scritta come forma quadratica nelle velocità generalizzate:

\[T = \dfrac{1}{2}\sum_{k = 1}^{s}{\sum_{j = 1}^{s}a_{kj}(q){\dot{q}}_{k}{\dot{q}}_{j}}\]

Dato che \(a_{kj}\) è un parametro dipendente dalla posizione generalizzata \(q\), l'energia cinetica totale dipende dalla velocità e dalla posizione generalizzate.

\subsection{Lemma 4: energia potenziale in coordinate generalizzate}\label{lemma-4-energia-potenziale-in-coordinate-generalizzate}

È conveniente esprimere anche l'energia potenziale in funzione delle coordinate generalizzate. Si considera un sistema di \(N\) particelle. L'energia di interazione tra le particelle è indicata con \(U\left( q_{i} \right),\ i = 1,2,\ldots,N\) ed è denotata come energia potenziale. In meccanica classica questa quantità dipende solamente dalla posizione.

In particolare, se le coordinate generalizzate \(q_{i}\) descrivono completamente la configurazione del sistema, allora l'energia potenziale può essere scritta come:

\[U = U(q_{1},q_{2},\ldots,q_{s})\]

dove \(s\) è il numero di coordinate generalizzate.

La meccanica classica assume che le interazioni tra le particelle avvengano istantaneamente, ovvero non considera gli effetti di propagazione dei cambiamenti nei campi di forza. Tuttavia, secondo la teoria della relatività di Einstein, le interazioni si propagano attraverso campi con velocità finita, non superiore alla velocità della luce \(c\). Questo implica che la descrizione classica è un'approssimazione valida solo quando le velocità coinvolte sono molto inferiori a \(c\) e gli effetti di ritardo possono essere trascurati.

\subsection{Lemma 5: forma della lagrangiana}\label{lemma-5-forma-della-lagrangiana}

La lagrangiana può essere espressa come funzione dell'energia cinetica \(T\) e dell'energia potenziale \(U\), secondo la relazione \cite{arnold1992matematici, landau1994meccanica}:

\[
L\left( q_{i},{\dot{q}}_{i} \right) = T\left( q_{i},{\dot{q}}_{i} \right) - U\left( q_{i} \right),\quad i = 1,\ldots,N
\]

La lagrangiana \(L\left( q_{i},{\dot{q}}_{i} \right)\) è, dunque, una funzione delle coordinate generalizzate posizione e velocità, che a loro volta dipendono dal tempo:

\[L : TQ \times \mathbb{R} \rightarrow \mathbb{R}\]

Dove \(TQ\) indica il \textbf{fibrato tangente} dello spazio delle configurazioni \(Q\), dove:

\begin{itemize}
\item Un elemento di \(Q\) è semplicemente una configurazione \(\vec{q}\);
\item Un elemento di \(TQ\) è una coppia \(\left( \vec{q},\dot{\vec{q}} \right)\), ovvero traiettoria e velocità della particella.
\end{itemize}

Si considera l'azione \(S\), data per definizione da:

\[S = \int_{t_{1}}^{t_{2}}{L\left( q_{i},{\dot{q}}_{i} \right)dt}\]

L'azione dipende dalle coordinate generalizzate, funzioni del tempo, dunque, è un funzionale poiché associa a ogni funzione \(\vec{q}(t)\) un valore numerico. Dunque, l'azione è definita nello spazio vettoriale delle traiettorie generalizzate \(\mathbb{V =}\left\{ \vec{q}(t) \right\}\) a valori in \(\mathbb{R}\):

\[S:\mathbb{V \rightarrow R}\]
\subsection{Lemma 6: legame tra lagrangiana ed equazioni di Newton}\label{lemma-6-legame-tra-lagrangiana-ed-equazioni-di-newton}

Per individuare una correlazione tra la meccanica newtoniana e quella lagrangiana si scrive l'equazione di Eulero-Lagrange in coordinate cartesiane. In generale, l'equazione di Eulero-Lagrange può essere espressa come:

\[\dfrac{d}{dt}\dfrac{\partial L}{\partial{\dot{q}}_{i}} - \dfrac{\partial L}{\partial q_{i}} = 0\]

È noto che la lagrangiana è data da:

\[L\left( q_{i},{\dot{q}}_{i} \right) = T\left( q_{i},{\dot{q}}_{i} \right) - U\left( q_{i} \right)\]

Per cui, è possibile scrivere:

\[\dfrac{d}{dt}\dfrac{\partial}{\partial{\dot{q}}_{i}}\left\lbrack T\left( q_{i},{\dot{q}}_{i} \right) - U\left( q_{i} \right) \right\rbrack - \dfrac{\partial}{\partial q_{i}}\left\lbrack T\left( q_{i},{\dot{q}}_{i} \right) - U\left( q_{i} \right) \right\rbrack = 0\]

Passando alle coordinate cartesiane si ha:

\[\dfrac{\partial}{\partial q_{i}} = \dfrac{\partial x_{i}}{\partial q_{i}}\dfrac{\partial}{\partial x_{i}} = \dfrac{\partial}{\partial x_{i}}\]

Questa semplificazione è vera solo se si assume che la $i$-esima coordinata generalizzata $q_i$ coincida con la $i$-esima coordinata cartesiana $x_i$. Analogo procedimento può essere eseguito per passare da \({\dot{q}}_{i}\) a \({\dot{x}}_{i}\).

In coordinate cartesiane, l'energia cinetica dipende solamente dalla velocità \({\dot{x}}_{i}\) mentre l'energia potenziale solamente dalla posizione \(x_{i}\). L'equazione di Eulero-Lagrange è:

\[\dfrac{d}{dt}\dfrac{\partial}{\partial{\dot{x}}_{i}}\left\lbrack T\left( {\dot{x}}_{i} \right) - U\left( x_{i} \right) \right\rbrack - \dfrac{\partial}{\partial x_{i}}\left\lbrack T\left( {\dot{x}}_{i} \right) - U\left( x_{i} \right) \right\rbrack = 0 \Leftrightarrow \dfrac{d}{dt}\dfrac{\partial}{\partial{\dot{x}}_{i}}T\left( {\dot{x}}_{i} \right) + \dfrac{\partial}{\partial x_{i}}U\left( x_{i} \right) = 0\]

Da cui si ottiene:

\[\dfrac{d}{dt}\dfrac{\partial}{\partial{\dot{x}}_{i}}T\left( {\dot{x}}_{i} \right) = - \dfrac{\partial}{\partial x_{i}}U\left( x_{i} \right)\]

Siccome l'energia cinetica dipende solo dalla velocità, risulta:

\[\dfrac{\partial}{\partial x_{i}}T\left( {\dot{x}}_{i} \right) = 0\]

È possibile scrivere:

\[\dfrac{\partial}{\partial x_{i}}U\left( x_{i} \right) = \dfrac{\partial}{\partial x_{i}}\left\lbrack T\left( {\dot{x}}_{i} \right) - U\left( x_{i} \right) \right\rbrack = - \dfrac{\partial L}{\partial x_{i}}\]

L'energia cinetica per un sistema di \(N\) particelle, in coordinate cartesiane, è:

\[T = \dfrac{1}{2}\sum_{i = 1}^{N}{m_{i}{\dot{x}}_{i}^{2}}\]

Applicando la derivata rispetto a \({\dot{x}}_{i}\) si ottiene:

\[\dfrac{\partial T}{\partial{\dot{x}}_{i}} = \dfrac{\partial T}{\partial{\dot{x}}_{i}}\left( \dfrac{1}{2}\sum_{i = 1}^{N}{m_{i}{\dot{x}}_{i}^{2}} \right) = \dfrac{1}{2}\dfrac{\partial T}{\partial{\dot{x}}_{i}}\left( m_{1}{\dot{x}}_{1}^{2} + m_{2}{\dot{x}}_{2}^{2} + \ldots + m_{i}{\dot{x}}_{i}^{2} + \ldots + m_{N}{\dot{x}}_{N}^{2} \right) = m_{i}{\dot{x}}_{i}\]

Derivando tale quantità rispetto al tempo si ottiene il primo membro dell'equazione di Eulero-Lagrange in coordinate cartesiane:

\[\dfrac{d}{dt}\dfrac{\partial T}{\partial{\dot{x}}_{i}} = \dfrac{d}{dt}\left( m_{i}{\dot{x}}_{i} \right) = m_{i}{\ddot{x}}_{i}\]

Dunque, poiché:

\[\dfrac{d}{dt}\dfrac{\partial}{\partial{\dot{x}}_{i}}T\left( {\dot{x}}_{i} \right) + \dfrac{\partial}{\partial x_{i}}U\left( x_{i} \right) = 0\]

risulta:

\[m_{i}{\ddot{x}}_{i} = - \dfrac{\partial}{\partial x_{i}}U\left( x_{i} \right)\]

Dalla relazione:

\[L\left( q_{i},{\dot{q}}_{i} \right) = T\left( q_{i},{\dot{q}}_{i} \right) - U\left( q_{i} \right)\]

È possibile scrivere che:

\[\dfrac{\partial}{\partial x_{i}}U\left( x_{i} \right) = \dfrac{\partial}{\partial x_{i}}\left\lbrack T\left( {\dot{x}}_{i} \right) - L\left( x_{i},{\dot{x}}_{i} \right) \right\rbrack = - \dfrac{\partial L}{\partial x_{i}}\]

Per cui si ottiene:

\[
m_{i}{\ddot{x}}_{i} = - \dfrac{\partial}{\partial x_{i}}U\left( x_{i} \right) = - \dfrac{\partial L}{\partial x_{i}}
\]

Dal secondo principio della dinamica è noto che:

\[
m_{i}{\ddot{x}}_{i} = f_{i} = \dfrac{dp_{i}}{dt}
\]

Dove \(f_{i}\) è la forza agente mentre \(p_{i}\) la quantità di moto.

Ne discende che la quantità di moto \(p_{i}\) e la forza \(f_{i}\) sono legate all'energia potenziale \(U\) e alla lagrangiana \(L\) dalle relazioni \cite{arnold1992matematici, landau1994meccanica}:

\[
\begin{cases}
\displaystyle p_{i} = \dfrac{\partial L}{\partial{\dot{q}}_{i}} \\
\displaystyle f_{i} = - \dfrac{\partial L}{\partial q_{i}}
\end{cases}
\]

Le due relazioni sono note come definizioni fondamentali della meccanica lagrangiana. La prima è detta momento coniugato o generalizzato, mentre la seconda forza generalizzata.

Si considera l'equazione di Eulero-Lagrange e si sostituisce la relazione per il momento coniugato generalizzato:

\[\dfrac{d}{dt}\dfrac{\partial L}{\partial{\dot{q}}_{i}} - \dfrac{\partial L}{\partial q_{i}} = 0 \Leftrightarrow \dfrac{d}{dt}p_{i} = \dfrac{\partial L}{\partial q_{i}}\]

Da cui risulta:

\[{\dot{p}}_{i} = \dfrac{\partial L}{\partial q_{i}}\]

\subsection{Principio di conservazione}\label{principio-di-conservazione}

Le leggi di conservazione dell'energia, del momento lineare e del momento angolare sono una conseguenza delle simmetrie fondamentali dello spazio e del tempo.

In particolare, queste leggi di conservazione sono connesse all'invarianza delle leggi fisiche del sistema rispetto a determinate trasformazioni. Questo legame profondo fu dimostrato da Emmy Noether nel 1915 \cite{arnold1992matematici}. Secondo il suo teorema, a ogni simmetria continua e differenziabile delle leggi fisiche corrisponde una quantità conservata:

\begin{itemize}
\item
  Dall'omogeneità temporale discende la conservazione dell'energia. L'omogeneità temporale implica che le leggi fisiche non cambiano se un fenomeno viene traslato nel tempo;
\item
  Dall'omogeneità spaziale deriva la conservazione del momento lineare (o quantità di moto). L'omogeneità spaziale implica che le leggi fisiche non variano se il sistema viene traslato nello spazio;
\item
  Dall'isotropia dello spazio discende la conservazione del momento angolare. L'isotropia spaziale implica che le leggi fisiche non cambiano se il sistema viene ruotato nello spazio.
\end{itemize}

\subsection{Lemma 7: conservazione dell'energia}\label{lemma-7-conservazione-dellenergia}

L'energia totale \(E = T + U\) si conserva grazie alla proprietà di omogeneità temporale. Infatti, se il tempo è omogeneo e il sistema isolato, per definizione la lagrangiana non dipende esplicitamente dal tempo; invece, tale dipendenza è presente nelle coordinate generalizzate \cite{landau1994meccanica}:

\[L = L\left( \vec{q},\dot{\vec{q}} \right),\ \ \vec{q} = \vec{q}(t),\dot{\vec{q}} = \dot{\vec{q}}(t)\]

La derivata rispetto al tempo della lagrangiana, per la derivata delle funzioni composte, può essere espressa come:

\[\dfrac{d}{dt}L\left( \vec{q},\dot{\vec{q}} \right) = \dfrac{\partial L}{\partial\vec{q}}\dfrac{d\vec{q}}{dt} + \dfrac{\partial L}{\partial\dot{\vec{q}}}\dfrac{d\dot{\vec{q}}}{dt}\]

Dove i prodotti tra vettori sono da intenderi come prodotti scalari. Dunque, ricorrendo alla simbologia sulle derivate temporali, si può scrivere:

\[\dfrac{d}{dt}L\left( \vec{q},\dot{\vec{q}} \right) = \dfrac{\partial L}{\partial\vec{q}}\dot{\vec{q}} + \dfrac{\partial L}{\partial\dot{\vec{q}}}\vec{\ddot{q}}\]

Si considera la quantità:

\[\dfrac{d}{dt}\left( \dfrac{\partial L}{\partial\dot{\vec{q}}}\dot{\vec{q}} \right)\]

Svolgendo la derivata si ottiene:

\[\dfrac{d}{dt}\left( \dfrac{\partial L}{\partial\dot{\vec{q}}}\dot{\vec{q}} \right) = \dfrac{d}{dt}\left( \dfrac{\partial L}{\partial\dot{\vec{q}}} \right)\dot{\vec{q}} + \dfrac{\partial L}{\partial\dot{\vec{q}}}\dfrac{d\dot{\vec{q}}}{dt} = \dfrac{d}{dt}\left( \dfrac{\partial L}{\partial\dot{\vec{q}}} \right)\dot{\vec{q}} + \dfrac{\partial L}{\partial\dot{\vec{q}}}\vec{\ddot{q}}\]

Si isola il termine contenente \(\vec{\ddot{q}}\):

\[\dfrac{\partial L}{\partial\dot{\vec{q}}}\vec{\ddot{q}} = \dfrac{d}{dt}\left( \dfrac{\partial L}{\partial\dot{\vec{q}}}\dot{\vec{q}} \right) - \dfrac{d}{dt}\left( \dfrac{\partial L}{\partial\dot{\vec{q}}} \right)\dot{\vec{q}}\]

Si sostituisce questo risultato nella derivata temporale della lagrangiana:

\[\dfrac{dL}{dt} = \dfrac{\partial L}{\partial\vec{q}}\dot{\vec{q}} + \dfrac{\partial L}{\partial\dot{\vec{q}}}\vec{\ddot{q}} = \dfrac{\partial L}{\partial\vec{q}}\dot{\vec{q}} + \dfrac{d}{dt}\left( \dfrac{\partial L}{\partial\dot{\vec{q}}}\dot{\vec{q}} \right) - \dfrac{d}{dt}\left( \dfrac{\partial L}{\partial\dot{\vec{q}}} \right)\dot{\vec{q}}\]

Si porta il termine \(\dfrac{d}{dt}\left( \dfrac{\partial L}{\partial\dot{\vec{q}}}\dot{\vec{q}} \right)\) al primo membro:

\[\dfrac{dL}{dt} - \dfrac{d}{dt}\left( \dfrac{\partial L}{\partial\dot{\vec{q}}}\dot{\vec{q}} \right) = \dfrac{\partial L}{\partial\vec{q}}\dot{\vec{q}} - \dfrac{d}{dt}\left( \dfrac{\partial L}{\partial\dot{\vec{q}}} \right)\dot{\vec{q}} \Leftrightarrow \dfrac{d}{dt}\left( L - \dfrac{\partial L}{\partial\dot{\vec{q}}}\dot{\vec{q}} \right) = \left\lbrack \dfrac{\partial L}{\partial\vec{q}} - \dfrac{d}{dt}\left( \dfrac{\partial L}{\partial\dot{\vec{q}}} \right) \right\rbrack\dot{\vec{q}}\]

Per l'equazione di Eulero-Lagrange, il secondo membro è nullo:

\[\dfrac{\partial L}{\partial\vec{q}} - \dfrac{d}{dt}\left( \dfrac{\partial L}{\partial\dot{\vec{q}}} \right) = \vec{0}\]

Dunque:

\[\dfrac{d}{dt}\left( L - \dfrac{\partial L}{\partial\dot{\vec{q}}}\dot{\vec{q}} \right) = 0\]

Integrando rispetto al tempo si ottiene:

\[L - \dfrac{\partial L}{\partial\dot{\vec{q}}}\dot{\vec{q}} = cost\]

Per definizione di lagrangiana, risulta che:

\[L = T - U\]

Inoltre, l'energia potenziale non dipende dalla velocità, dunque:

\[\dfrac{\partial L}{\partial\dot{\vec{q}}} = \dfrac{\partial}{\partial\dot{\vec{q}}}(T - U) = \dfrac{\partial L}{\partial\dot{\vec{q}}}\]

Dalle relazioni tra meccanica lagrangiana e newtoniana è possibile scrivere:

\[\dfrac{\partial L}{\partial\dot{\vec{q}}} = {\vec{p}}^{T} = m{\dot{\vec{q}}}^{T}\]

Moltiplicando ambo i membri per \(\dot{\vec{q}}\), si ha:

\[\dfrac{\partial L}{\partial\dot{\vec{q}}}\dot{\vec{q}} = m{\dot{\vec{q}}}^{T}\dot{\vec{q}} = m{\dot{q}}^{2}\]

Per sistemi dove l'energia cinetica è una funzione omogenea di secondo grado rispetto alle velocità, è possibile affermare che:

\[m{\dot{q}}^{2} = 2T\]

Per cui:

\[L - \dfrac{\partial L}{\partial\dot{\vec{q}}}\dot{\vec{q}} = cost \Leftrightarrow T - U - 2T = cost\]

A meno di un segno, risulta:

\[
T + U = cost
\]

In definitiva, si è dimostrato che la conservazione dell'energia totale (T+U) è una diretta conseguenza della simmetria temporale delle leggi fisiche.

\subsection{Lemma 8: conservazione della quantità di moto}\label{lemma-8-conservazione-della-quantituxe0-di-moto}

Il momento lineare:

\[\vec{p} = \sum_{i = 1}^{N}{\vec{p}}_{i} = \sum_{i = 1}^{N}{m\vec{v}}_{i}\]

Si conserva lungo la direzione per cui l'energia potenziale resta invariata, quindi lungo traiettorie equipotenziali.

Si suppone che lo spazio sia omogeneo, dunque, una qualsiasi traslazione \(\delta\vec{r}\) del sistema non deve cambiare la lagrangiana; in altre parole, deve risultare che \(\delta L = 0\).

La variazione della lagrangiana può essere espressa come:

\[\delta L = \dfrac{\partial L}{\partial\vec{r}}\delta\vec{r}\]

Si considera l'equazione di Eulero-Lagrange \cite{landau1994meccanica}:

\[\dfrac{\partial L}{\partial\vec{q}} - \dfrac{d}{dt}\left( \dfrac{\partial L}{\partial\dot{\vec{q}}} \right) = \vec{0}\]

Passando a coordinate cartesiane, \(\vec{q}\) coincide con la posizione \(\vec{r}\), mentre \(\dot{\vec{q}}\) con la velocità \(\vec{v}\). L'equazione di Eulero-Lagrange può essere scritta come:

\[\dfrac{\partial L}{\partial\vec{r}} - \dfrac{d}{dt}\left( \dfrac{\partial L}{\partial\vec{v}} \right) = \vec{0} \Leftrightarrow \dfrac{\partial L}{\partial\vec{r}} = \dfrac{d}{dt}\left( \dfrac{\partial L}{\partial\vec{v}} \right)\]

Per la proprietà di omogeneità spaziale, la derivata rispetto alla posizione della lagrangiana è nulla:

\[\dfrac{\partial L}{\partial\vec{r}} = \vec{0}\]

Per cui risulta che:

\[\dfrac{d}{dt}\left( \dfrac{\partial L}{\partial\vec{v}} \right) = \vec{0}\]

Integrando rispetto al tempo si ottiene:

\[
\dfrac{\partial L}{\partial\vec{v}} = \vec{const}
\]

In altre parole, il gradiente della lagrangiana rispetto alla velocità è costante.

In coordinate cartesiane, è possibile scrivere che:

\[L = \dfrac{1}{2}\sum_{i = 1}^{N}{m_{i}v_{i}^{2}} - U\left( x_{1},x_{2}\ldots,x_{N} \right) = \dfrac{1}{2}\sum_{i = 1}^{N}{m_{i}{\vec{v}}_{i} \cdot {\vec{v}}_{i}} - U\left( x_{1},x_{2}\ldots,x_{N} \right)\]

Derivando rispetto a \(\vec{v}\), si ottiene:

\[
\dfrac{\partial L}{\partial\vec{v}} = \dfrac{\partial}{\partial\vec{v}}\left( \dfrac{1}{2}\sum_{i = 1}^{N}{m_{i}v_{i}^{2}} - U\left( x_{1},x_{2}\ldots,x_{N} \right) \right) = \sum_{i = 1}^{N}{m_{i}{\vec{v}}_{i}} = \sum_{i = 1}^{N}{\vec{p}}_{i} = \vec{const}
\]

La quantità di moto o momento lineare, in definitiva, si conserva in ipotesi di omogeneità spaziale.

\subsection{Lemma 9: conservazione del momento angolare}\label{lemma-9-conservazione-del-momento-angolare}

La proiezione del momento angolare \cite{landau1994meccanica}:

\[\vec{L} = \sum_{i = 1}^{N}{{\vec{r}}_{i} \times {\vec{p}}_{i}}\]

si conserva lungo direzioni in cui il potenziale \(U\) presenta delle simmetrie. Ad esempio, se il potenziale ha simmetria cilindrica, la proiezione del momento angolare lungo quest'asse si conserva.

Si suppone che lo spazio sia isotropo, dunque, una rotazione \(\delta\vec{r} = d\vec{\varphi} \times \vec{r}\) non deve modificare la lagrangiana. Il termine $\varphi$ rappresenta l'asse di una simmetria parziale del potenziale.

Anche la velocità \(\vec{v}\) subisce una rotazione \(\delta\vec{v} = d\vec{\varphi} \times \vec{v}\) dovuta a \(d\vec{\varphi}\). La variazione \(\delta L\) della lagrangiana, dovuta alla rotazione \(d\vec{\varphi}\), può essere espressa come differenziale:

\[\delta L = \dfrac{\partial L}{\partial\vec{v}} \cdot \delta\vec{v} + \dfrac{\partial L}{\partial\vec{r}} \cdot \delta\vec{r}\]

Per l'ipotesi di isotropia, la variazione \(\delta L = 0\), per cui si ha:

\[\dfrac{\partial L}{\partial\vec{v}} \cdot \delta\vec{v} + \dfrac{\partial L}{\partial\vec{r}} \cdot \delta\vec{r} = 0\]

Si sostituiscono le relazioni per le variazioni di spostamento e velocità in termini di \(d\vec{\varphi} \times \vec{r}\):

\[\delta\vec{r} = d\vec{\varphi} \times \vec{r},\ \ \delta\vec{v} = d\vec{\varphi} \times \vec{v}\]

Si ottiene:

\[\dfrac{\partial L}{\partial\vec{v}} \cdot \left( d\vec{\varphi} \times \vec{v} \right) + \dfrac{\partial L}{\partial\vec{r}} \cdot \left( d\vec{\varphi} \times \vec{r} \right) = 0\]

Esplicitando le derivate, si ha:

\[\sum_{i = 1}^{N}\left\lbrack \dfrac{\partial L}{\partial{\vec{v}}_{i}} \cdot \left( d\vec{\varphi} \times {\vec{v}}_{i} \right) + \dfrac{\partial L}{\partial{\vec{r}}_{i}} \cdot \left( d\vec{\varphi} \times {\vec{r}}_{i} \right) \right\rbrack = 0\]

Dati tre vettori generici \(\vec{a}\), \(\vec{b}\) e \(\vec{c}\), si dimostra:

\[
\left( \vec{a} \times \vec{b} \right) \cdot \vec{c} = \left( \vec{b} \times \vec{c} \right) \cdot \vec{a} = \left( \vec{c} \times \vec{a} \right) \cdot \vec{b}
\]

Applicando tale relazione, è possibile scrivere \(d\vec{\varphi}\) in prodotto scalare con l'operazione di prodotto vettoriale tra gli altri due vettori:

\[\sum_{i = 1}^{N}\left\lbrack \dfrac{\partial L}{\partial{\vec{v}}_{i}} \cdot \left( d\vec{\varphi} \times {\vec{v}}_{i} \right) + \dfrac{\partial L}{\partial{\vec{r}}_{i}} \cdot \left( d\vec{\varphi} \times {\vec{r}}_{i} \right) \right\rbrack = \sum_{i = 1}^{N}\left\lbrack d\vec{\varphi} \cdot \left( {\vec{v}}_{i} \times \dfrac{\partial L}{\partial{\vec{v}}_{i}} \right) + d\vec{\varphi} \cdot \left( {\vec{r}}_{i} \times \dfrac{\partial L}{\partial{\vec{r}}_{i}} \right) \right\rbrack = 0\]

Dall'equivalenza con la meccanica classica, è noto che:
\[
\begin{cases}
\displaystyle \dfrac{\partial L}{\partial{\vec{v}}_{i}} = {\vec{p}}_{i} \\
\displaystyle \dfrac{\partial L}{\partial{\vec{r}}_{i}} = {\vec{f}}_{i} = \displaystyle \dfrac{d{\vec{p}}_{i}}{dt}
\end{cases}
\]

Dunque, si ha:

\[\sum_{i = 1}^{N}\left( {\vec{v}}_{i} \times {\vec{p}}_{i} + {\vec{r}}_{i} \times \dfrac{d{\vec{p}}_{i}}{dt} \right) \cdot d\vec{\varphi} = 0\]

Poiché \({\vec{p}}_{i} = m_{i}{\vec{v}}_{i}\), la quantità di moto è parallela alla velocità, dunque, il loro prodotto vettorale è nullo:

\[{\vec{v}}_{i} \times {\vec{p}}_{i} = \vec{0}\]

Resta, dunque:

\[\sum_{i = 1}^{N}\left( {\vec{r}}_{i} \times \dfrac{d{\vec{p}}_{i}}{dt} \right) \cdot d\vec{\varphi} = 0\]

Si considera la quantità:

\[\dfrac{d}{dt}\left( {\vec{r}}_{i} \times {\vec{p}}_{i} \right)\]

Applicando le proprietà delle derivate, si ha:

\[\dfrac{d}{dt}\left( {\vec{r}}_{i} \times {\vec{p}}_{i} \right) = \dfrac{d{\vec{r}}_{i}}{dt} \times {\vec{p}}_{i} + {\vec{r}}_{i} \times \dfrac{d{\vec{p}}_{i}}{dt}\]

La derivata temporale della posizione coincide con la velocità istantanea:

\[\dfrac{d{\vec{r}}_{i}}{dt} = {\vec{v}}_{i}\]

Siccome \({\vec{v}}_{i} \times {\vec{p}}_{i} = \vec{0}\), risulta:

\[\dfrac{d}{dt}\left( {\vec{r}}_{i} \times {\vec{p}}_{i} \right) = {\vec{r}}_{i} \times \dfrac{d{\vec{p}}_{i}}{dt}\]

Sostituendo questo risultato nella relazione:

\[\sum_{i = 1}^{N}\left( {\vec{r}}_{i} \times \dfrac{d{\vec{p}}_{i}}{dt} \right) \cdot d\vec{\varphi} = 0\]

Si ha:

\[
\sum_{i = 1}^{N}\left( {\vec{r}}_{i} \times \dfrac{d{\vec{p}}_{i}}{dt} \right) \cdot d\vec{\varphi} = \dfrac{d}{dt}\sum_{i = 1}^{N}\left( {\vec{r}}_{i} \times {\vec{p}}_{i} \right) \cdot d\vec{\varphi} = 0
\]

Da questa relazione discende la conservazione del momento angolare lungo la direzione \(d\vec{\varphi}\) di simmetria del potenziale \(U\).
\subsection{Lagrangiana per pendolo}\label{lagrangiana-per-pendolo}

Si considera un corpo di massa \(m\) sospeso a un filo di lunghezza \(l\) nel campo gravitazionale con accelerazione \(g\).

\begin{figure}[ht]
\centering
\includegraphics[width=1.62881in,height=2.32892in,alt={Immagine che contiene schizzo, diagramma, linea, disegno Il contenuto generato dall'IA potrebbe non essere corretto.}]{media/1_Meccanica/image7.pdf}\caption{Pendolo semplice}
\end{figure}

Il pendolo possiede un solo grado di libertà, ovvero la rotazione intorno al proprio polo. Sia \(\vartheta\) l'angolo di cui la massa \(m\) è inclinata rispetto la verticale. La funzione lagrangiana data da:

\[L\left( \vartheta,\dot{\vartheta} \right) = T - U\]

Nel moto del pendolo, la velocità \(v\) è legata alla velocità angolare \(\dot{\vartheta}\), dovuta allo spostamento angolare, dalla relazione:

\[v = l\dot{\vartheta}\]

L'energia cinetica del sistema si scrive, quindi, come:

\[T = \dfrac{1}{2}m\left( l\dot{\vartheta} \right)^{2}\]

Per l'energia potenziale, \(U\), la componente verticale è data dalla proiezione della posizione della massa rispetto alla verticale, ovvero \(l\cos\vartheta\). La differenza di altezza rispetto al punto di equilibrio è data da:

\[\Delta h = l - l\cos\vartheta = l(1 - cos\vartheta)\]

Dunque, l'energia potenziale è data da:

\[U = mgl(1 - cos\vartheta)\]

Esplicitando l'energia potenziale e cinetica, la lagrangiana è data da:

\[L\left( \vartheta,\dot{\vartheta} \right) = T - U = \dfrac{1}{2}m\left( l\dot{\vartheta} \right)^{2} - mgl(1 - cos\vartheta)\]

La funzione lagrangiana deve soddisfare l'equazione di Eulero-Lagrange:

\[\dfrac{d}{dt}\dfrac{\partial L}{\partial\dot{\vartheta}} - \dfrac{\partial L}{\partial\vartheta} = 0\]

Sostituendo l'equazione ottenuta per \(L\), si ha:

\[\dfrac{d}{dt}\left( ml^{2}\dot{\vartheta} \right) + mgl\sin\vartheta = 0\]

\[ml^{2}\ddot{\vartheta} + \ mgl\sin\vartheta = 0\]

Semplificando \(m\) ed \(l\) si ha:

\[\ddot{\vartheta} + \dfrac{g}{l}\sin\vartheta = 0\]

Si definisce pulsazione naturale del sistema:

\[
\omega = \sqrt{\dfrac{g}{l}}
\]

Con questa definizione, l'equazione può essere scritta come:

\[
\ddot{\vartheta} + \omega^{2}\sin\vartheta = 0
\]

Tale equazione non ammette soluzione in forma chiusa a meno di considerare l'approssimazione per piccole oscillazioni:

\[
\vartheta \ll 1
\]

In questo caso, l'equazione si scrive come:

\[
\ddot{\vartheta} + \omega^{2}\vartheta = 0
\]

In definitiva, si ottiene l'equazione dell'oscillatore armonico.

\subsection{Lagrangiana per il doppio pendolo}\label{lagrangiana-per-il-doppio-pendolo}

Si vuole scrivere la lagrangiana per un doppio pendolo, costituito da due masse, \(m_{1}\) e \(m_{2}\) connesse tra loro. La prima massa è collegata al fulcro mediante un cavo di lunghezza \(l_{1}\); la seconda è connessa a \(m_{1}\) mediante un cavo di lunghezza \(l_{2}\)

\begin{figure}[ht]
\centering
\includegraphics[width=1.49765in,height=2.0292in,alt={Immagine che contiene diagramma, linea, design Il contenuto generato dall'IA potrebbe non essere corretto.}]{media/1_Meccanica/image8.pdf}\caption{Doppio pendolo}
\end{figure}

Si proiettano le componenti delle lunghezze \(l_{1}\) e \(l_{2}\) sugli assi cartesiani:

\[l_{1}: \begin{cases}
x_{1} = l_{1}\sin\vartheta_{1} \\
y_{1} = - l_{1}\cos\vartheta_{1}
\end{cases}\ \]

\[l_{2}: \begin{cases}
x_{2} = x_{1} + l_{2}\sin\vartheta_{2} \\
y_{2} = y_{1} - l_{2}\cos\vartheta_{2}
\end{cases}\]

Le componenti della velocità possono essere valutate, derivando rispetto al tempo le equazioni ottenute:

\[
\begin{cases}
\displaystyle{\dot{x}}_{1} = l_{1}\dfrac{d}{dt}\sin\vartheta_{1} = l_{1}\cos\vartheta_{1}\dfrac{d\vartheta_{1}}{dt} \\
\displaystyle{\dot{y}}_{1} = - l_{1}\dfrac{d}{dt}\cos\vartheta_{1} = l_{1}\sin\vartheta_{1}\dfrac{d\vartheta_{1}}{dt}
\end{cases}
\]

\[
\begin{cases}
\displaystyle{\dot{x}}_{2} = l_{1}\cos\vartheta_{1}\dfrac{d\vartheta_{1}}{dt} + l_{2}\cos\vartheta_{2}\dfrac{d\vartheta_{2}}{dt} \\
\displaystyle{\dot{y}}_{2} = l_{1}\sin\vartheta_{1}\dfrac{d\vartheta_{1}}{dt} + l_{2}\sin\vartheta_{2}\dfrac{d\vartheta_{2}}{dt}
\end{cases} \]

La configurazione del sistema può essere determinata noti i parametri \(\vartheta_{1}\) e\(\vartheta_{2}\), dunque, il sistema presenta \(2\) gradi di libertà. Infatti, rispetto al fulcro, le due masse possono ruotare relativamente, dunque, l'energia cinetica comprende sia la velocità di transizione che rotazione:

\[T = \dfrac{1}{2}m_{1}v_{1}^{2} + \dfrac{1}{2}m_{2}v_{2}^{2}\]

La velocità al quadrato della massa \(m_{1}\) è data da:

\[
v_{1}^{2} = {\dot{x}}_{1}^{2} + {\dot{y}}_{1}^{2} = \left( l_{1}\cos\vartheta_{1}\dfrac{d\vartheta_{1}}{dt} \right)^{2} + \left( l_{1}\sin\vartheta_{1}\dfrac{d\vartheta_{1}}{dt} \right) = l_{1}^{2}{\dot{\vartheta}}_{1}^{2}\left( \cos^{2}\vartheta_{1} + \sin^{2}\vartheta_{1} \right) \Leftrightarrow v_{1}^{2} = l_{1}^{2}{\dot{\vartheta}}_{1}^{2}
\]

La velocità al quadrato della massa \(m_{2}\) è data da:

\[\begin{aligned}
v_{2}^{2} & = {\dot{x}}_{2}^{2} + {\dot{y}}_{2}^{2} = \left( l_{1}\cos\vartheta_{1}\,{\dot{\vartheta}}_{1} + l_{2}\cos\vartheta_{2}\,{\dot{\vartheta}}_{2} \right)^{2} + \left( l_{1}\sin\vartheta_{1}\,{\dot{\vartheta}}_{1} + l_{2}\sin\vartheta_{2}\,{\dot{\vartheta}}_{2} \right)^{2} \\
 & = l_{1}^{2}{\dot{\vartheta}}_{1}^{2}\left( \cos^{2}\vartheta_{1} + \sin^{2}\vartheta_{1} \right) + l_{2}^{2}{\dot{\vartheta}}_{2}^{2}\left( \cos^{2}\vartheta_{2} + \sin^{2}\vartheta_{2} \right) + 2l_{1}l_{2}{\dot{\vartheta}}_{1}{\dot{\vartheta}}_{2}\left( \cos\vartheta_{1}\cos\vartheta_{2} + sin\vartheta_{1}\sin\vartheta_{2} \right) \\
 & = l_{1}^{2}{\dot{\vartheta}}_{1}^{2} + l_{2}^{2}{\dot{\vartheta}}_{2}^{2} + 2l_{1}l_{2}{\dot{\vartheta}}_{1}{\dot{\vartheta}}_{2}cos(\vartheta_{1} - \vartheta_{2})
\end{aligned}\]

Ricorrendo alle identità trigonometriche, è possibile scrivere:

\[v_{2}^{2} = l_{1}^{2}{\dot{\vartheta}}_{1}^{2} + l_{2}^{2}{\dot{\vartheta}}_{2}^{2} + 2l_{1}l_{2}{\dot{\vartheta}}_{1}{\dot{\vartheta}}_{2}\cos\left( \vartheta_{1} - \vartheta_{2} \right)\]

L'energia cinetica si scrive come:

\[T = \dfrac{1}{2}m_{1}l_{1}^{2}{\dot{\vartheta}}_{1}^{2} + \dfrac{1}{2}m_{2}\left\lbrack l_{1}^{2}{\dot{\vartheta}}_{1}^{2} + l_{2}^{2}{\dot{\vartheta}}_{2}^{2} + 2l_{1}l_{2}{\dot{\vartheta}}_{1}{\dot{\vartheta}}_{2}\cos\left( \vartheta_{1} - \vartheta_{2} \right) \right\rbrack\]

Bisogna valutare anche l'energia potenziale. Quest'ultima è data dalla somma delle energie potenziali delle due masse:

\[U = m_{1}g\Delta h_{1} + m_{2}g\Delta h_{2}\]

Dove:

\[\Delta h_{1} = l_{1} - l_{1}\cos\vartheta_{1} = l_{1}\left( 1 - cos\vartheta_{1} \right),\ \ \Delta h_{2} = l_{1} + l_{2} - l_{1}\cos\vartheta_{1} - l_{2}\cos\vartheta_{2} = l_{1}\left( 1 - cos\vartheta_{1} \right) + l_{2}\left( 1 - cos\vartheta_{2} \right)\]

Sostituendo, si ottiene:

\[U = m_{1}gl_{1}\left( 1 - cos\vartheta_{1} \right) + m_{2}gl_{1}\left( 1 - cos\vartheta_{1} \right) + m_{2}gl_{2}\left( 1 - cos\vartheta_{2} \right)\]

La lagrangiana per questo sistema è data da:

\begin{align*}
L\left( \vartheta_{1},\vartheta_{2},{\dot{\vartheta}}_{1},{\dot{\vartheta}}_{2} \right) 
&= \dfrac{1}{2}m_{1}l_{1}^{2}{\dot{\vartheta}}_{1}^{2} 
+ \dfrac{1}{2}m_{2}\left( l_{1}^{2}{\dot{\vartheta}}_{1}^{2} + l_{2}^{2}{\dot{\vartheta}}_{2}^{2} + 2l_{1}l_{2}{\dot{\vartheta}}_{1}{\dot{\vartheta}}_{2}\cos\left( \vartheta_{1} - \vartheta_{2} \right) \right) +\\
&\quad - m_{1}gl_{1}\left( 1 - \cos\vartheta_{1} \right) 
- m_{2}gl_{1}\left( 1 - \cos\vartheta_{1} \right) 
- m_{2}gl_{2}\left( 1 - \cos\vartheta_{2} \right)
\end{align*}

La funzione lagrangiana deve soddisfare l'equazione di Eulero-Lagrange:

\[\dfrac{d}{dt}\dfrac{\partial L}{\partial\dot{\vec{\vartheta}}} - \dfrac{\partial L}{\partial\vec{\vartheta}} = \vec{0}\]

L'equazione si traduce in due equazioni, relative a \(\vartheta_{1}\) e \(\vartheta_{2}\):

\[ \begin{cases}
\displaystyle \dfrac{d}{dt}\left( \dfrac{\partial L}{\partial{\dot{\vartheta}}_{1}} \right) - \dfrac{\partial L}{\partial\vartheta_{1}} = 0 \\
\displaystyle \dfrac{d}{dt}\left( \dfrac{\partial L}{\partial{\dot{\vartheta}}_{2}} \right) - \dfrac{\partial L}{\partial\vartheta_{2}} = 0
\end{cases} \]

Dove le derivate sono:

\[\begin{cases}
\displaystyle\dfrac{\partial L}{\partial{\dot{\vartheta}}_{1}} = \left( m_{1} + m_{2} \right)l_{1}^{2}{\dot{\vartheta}}_{1} + m_{2}l_{1}l_{2}{\dot{\vartheta}}_{2}\cos\left( \vartheta_{1} - \vartheta_{2} \right) \\
\displaystyle\dfrac{\partial L}{\partial{\dot{\vartheta}}_{2}} = m_{2}l_{2}^{2}{\dot{\vartheta}}_{2} + m_{2}l_{1}l_{2}{\dot{\vartheta}}_{1}\cos\left( \vartheta_{1} - \vartheta_{2} \right) \\
\displaystyle\dfrac{\partial L}{\partial\vartheta_{1}} = - m_{2}l_{1}l_{2}{\dot{\vartheta}}_{1}{\dot{\vartheta}}_{2}\sin\left( \vartheta_{1} - \vartheta_{2} \right) + \left( m_{1} + m_{2} \right)gl_{1}\sin\vartheta_{1} \\
\displaystyle\dfrac{\partial L}{\partial\vartheta_{2}} = m_{2}l_{1}l_{2}{\dot{\vartheta}}_{1}{\dot{\vartheta}}_{2}\sin\left( \vartheta_{1} - \vartheta_{2} \right) + m_{2}gl_{2}\sin\vartheta_{2}
\end{cases}\]

Eseguendo le derivate temporali e riarrangiando i termini, si ottengono le due equazioni:

\[
\begin{cases}
\left( m_{1} + m_{2} \right)l_{1}^{2}{\ddot{\vartheta}}_{1} + m_{2}l_{1}l_{2}{\ddot{\vartheta}}_{2}\cos\left( \vartheta_{1} - \vartheta_{2} \right) + m_{2}l_{1}l_{2}{\dot{\vartheta}}_{2}^{2}\sin\left( \vartheta_{1} - \vartheta_{2} \right) + \left( m_{1} + m_{2} \right)gl_{1}\sin\vartheta_{1} = 0 \\
m_{2}l_{2}{\ddot{\vartheta}}_{2} + m_{2}l_{1}l_{2}{\ddot{\vartheta}}_{1}\cos\left( \vartheta_{1} - \vartheta_{2} \right) - m_{2}l_{1}l_{2}{\dot{\vartheta}}_{1}^{2}\sin\left( \vartheta_{1} - \vartheta_{2} \right) + m_{2}gl_{2}\sin\vartheta_{2} = 0
\end{cases}
\]

Risolte le due equazioni, si ottiene la traiettoria, descritta da \(\vartheta_{1}\) e \(\vartheta_{2}\), del doppio pendolo.

\includegraphics[width=6.25in,height=2.38333in,alt={Sequenza temporale del moto del doppio pendolo in 10 istanti successivi.}]{media/1_Meccanica/image9.pdf}

La Figura 1.8 mostra una sequenza di fotogrammi che rappresentano l'evoluzione temporale del doppio pendolo. Ogni riquadro corrisponde a un istante successivo, e le posizioni delle due masse sono tracciate in base agli angoli \(\vartheta_{1}(t)\) e \(\vartheta_{2}(t)\).

Il comportamento del sistema è altamente non lineare e sensibile alle condizioni iniziali: anche piccole variazioni iniziali possono produrre traiettorie molto diverse. Questo fenomeno è noto come \textbf{caos deterministico}.

La traiettoria delle masse non segue un percorso regolare, ma mostra oscillazioni complesse e interazioni dinamiche tra i due bracci del pendolo. Questo rende il doppio pendolo un sistema ideale per lo studio della dinamica non lineare.

\section{Descrizione hamiltoniana}\label{descrizione-hamiltoniana}

La descrizione hamiltoniana privilegia le variabili momento lineare o quantità di moto \(\vec{p}\) e la posizione generalizzata della particella \(\vec{q}\) \cite{landau1994meccanica}. Questa teoria sfrutta una funzione \(H\) detta hamiltonina, data da:

\[
H\left( \vec{p},\vec{q} \right) = E = T + U
\]

Questa relazione è valida se:

\begin{itemize}
\item
 Nel sistema vi sono solo forze conservative;
\item
 La Lagrangiana non dipende esplicitamente dal tempo, dunque, il sistema è isolato;
\item
 L'energia cinetica è una funzione quadratica omogenea delle velocità generalizzate.
\end{itemize}

Privilegiando la quantità di moto e la posizione generalizzata della particella, l'approccio hamiltoniano permette di descrivere l'evoluzione del sistema nel tempo come una traiettoria nello spazio posizione-quantità di moto, detto spazio delle fasi.

\subsection{Lemma 10: equazioni di Hamilton}\label{lemma-9-equazione-di-hamilton}

Si considera il differenziale della funzione hamiltoniana:

\[dH\left( \vec{p},\vec{q} \right) = \sum_{i = 1}^{N}\left( \dfrac{\partial H}{\partial p_{i}}dp_{i} + \dfrac{\partial H}{\partial q_{i}}dq_{i} \right)\]

Si differenzia anche la lagrangiana:

\[dL\left( \vec{q},\dot{\vec{q}} \right) = \sum_{i = 1}^{N}\left( \dfrac{\partial L}{\partial q_{i}}dq_{i} + \dfrac{\partial L}{\partial{\dot{q}}_{i}}d{\dot{q}}_{i} \right)\]

Combinando l'equazione del momento coniugato generalizzato con quella di Eulero-Lagrange, si ottiene:

\[
\begin{cases}
\displaystyle p_{i} = \dfrac{\partial L}{\partial{\dot{q}}_{i}} \\
\displaystyle {\dot{p}}_{i} = \dfrac{\partial L}{\partial q_{i}}
\end{cases}
\]

Sostituendo le relazioni precedenti, il differenziale della lagrangiana può essere scritto come:

\[dL\left( \vec{q},\dot{\vec{q}} \right) = \sum_{i = 1}^{N}\left( {\dot{p}}_{i}dq_{i} + p_{i}d{\dot{q}}_{i} \right)\]

Si considera la quantità:

\[d\left( p_{i}{\dot{q}}_{i} \right) = p_{i}d{\dot{q}}_{i} + dp_{i}{\dot{q}}_{i}\]

Da cui si ottiene:

\[p_{i}d{\dot{q}}_{i} = d\left( p_{i}{\dot{q}}_{i} \right) - dp_{i}{\dot{q}}_{i}\]

Si sostituisce questo risultato nel differenziale della lagrangiana:

\[dL\left( \vec{q},\dot{\vec{q}} \right) = \sum_{i = 1}^{N}\left( {\dot{p}}_{i}dq_{i} + p_{i}d{\dot{q}}_{i} \right) = \sum_{i = 1}^{N}\left\lbrack {\dot{p}}_{i}dq_{i} + d\left( p_{i}{\dot{q}}_{i} \right) - {\dot{q}}_{i}dp_{i} \right\rbrack = \sum_{i = 1}^{N}\left( {\dot{p}}_{i}dq_{i} - {\dot{q}}_{i}dp_{i} \right) + \sum_{i = 1}^{N}{d\left( p_{i}{\dot{q}}_{i} \right)}\]

Riordinando i termini, si scrive:

\[dL - \sum_{i = 1}^{N}{d\left( p_{i}{\dot{q}}_{i} \right)} = \sum_{i = 1}^{N}\left( {\dot{p}}_{i}dq_{i} - {\dot{q}}_{i}dp_{i} \right)\]

Grazie alla proprietà di linearità del differenziale, si può scrivere:

\[d\left( L - \sum_{i = 1}^{N}{p_{i}{\dot{q}}_{i}} \right) = \sum_{i = 1}^{N}\left( {\dot{p}}_{i}dq_{i} - {\dot{q}}_{i}dp_{i} \right)\]

Moltiplicando per \(- 1\), si ottiene:

\[d\left( \sum_{i = 1}^{N}{p_{i}{\dot{q}}_{i}} - L \right) = \sum_{i = 1}^{N}\left( {\dot{q}}_{i}dp_{i} - {\dot{p}}_{i}dq_{i} \right)\]

Risulta che:

\[\sum_{i = 1}^{N}{p_{i}{\dot{q}}_{i}} = \sum_{i = 1}^{N}{m_{i}{\dot{q}}_{i}{\dot{q}}_{i}} = \sum_{i = 1}^{N}{m_{i}{\dot{q}}_{i}^{2}} = 2T\]

Di conseguenza:

\[\sum_{i = 1}^{N}{p_{i}{\dot{q}}_{i}} - L = 2T - T + U = E\]

Con questo risultato, è possibile scrivere:

\[dE = \sum_{i = 1}^{N}\left( {\dot{q}}_{i}dp_{i} - {\dot{p}}_{i}dq_{i} \right)\]

Per definizione di hamiltoniana, è possibile scrivere:

\[dE = dH = \sum_{i = 1}^{N}\left( {\dot{q}}_{i}dp_{i} - {\dot{p}}_{i}dq_{i} \right)\]

Confrontando questo risultato con il differenziale dell'hamiltoniana:

\[dH\left( \vec{p},\vec{q} \right) = \sum_{i = 1}^{N}\left( \dfrac{\partial H}{\partial p_{i}}dp_{i} + \dfrac{\partial H}{\partial q_{i}}dq_{i} \right)\]

Si ottiene:

\[
\sum_{i = 1}^{N}\left( {\dot{q}}_{i}dp_{i} - {\dot{p}}_{i}dq_{i} \right) = \sum_{i = 1}^{N}\left( \dfrac{\partial H}{\partial p_{i}}dp_{i} + \dfrac{\partial H}{\partial q_{i}}dq_{i} \right)
\]

Confrontati i coefficienti dei due polinomi, si ottengono le equazioni, note come \textbf{equazioni canoniche di Hamilton} \cite{arnold1992matematici}:

\[
\begin{cases}
\displaystyle {\dot{q}}_{i} = \dfrac{\partial H}{\partial p_{i}} \\
\displaystyle {\dot{p}}_{i} = - \dfrac{\partial H}{\partial q_{i}}
\end{cases}
\]

A meno di un segno, queste equazioni sono simmetriche rispetto a quelle relative alla lagrangiana.


\subsection{Parentesi di Poisson}\label{parentesi-di-poisson}

Si considera una qualunque grandezza \(f\), funzione delle coordinate generalizzate \(\vec{q}\) e del momento lineare \(\vec{p}\):

\[
f = f\left( \vec{q},\vec{p} \right)
\]

La sua derivata temporale è valutata mediante la proprietà delle derivate dalle funzioni composte:

\[\dfrac{df}{dt} = \sum_{i = 1}^{N}\left( \dfrac{\partial f}{\partial q_{i}}\dfrac{dq_{i}}{dt} + \dfrac{\partial f}{\partial p_{i}}\dfrac{dp_{i}}{dt} \right) = \sum_{i = 1}^{N}\left( \dfrac{\partial f}{\partial q_{i}}{\dot{q}}_{i} + \dfrac{\partial f}{\partial p_{i}}{\dot{p}}_{i} \right)\]

Per le proprietà della funzione di Hamilton:

\[\begin{cases}
\displaystyle {\dot{q}}_{i} = \dfrac{\partial H}{\partial p_{i}} \\
\displaystyle {\dot{p}}_{i} = - \dfrac{\partial H}{\partial q_{i}}
\end{cases}
\]

La derivata temporale della funzione \(f\), può essere scritta come:

\[\dfrac{df}{dt} = \sum_{i = 1}^{N}\left( \dfrac{\partial f}{\partial q_{i}}\dfrac{dq_{i}}{dt} + \dfrac{\partial f}{\partial p_{i}}\dfrac{dp_{i}}{dt} \right) = \sum_{i = 1}^{N}\left( \dfrac{\partial f}{\partial q_{i}}{\dot{q}}_{i} + \dfrac{\partial f}{\partial p_{i}}{\dot{p}}_{i} \right) = \sum_{i = 1}^{N}\left( \dfrac{\partial f}{\partial q_{i}}\dfrac{\partial H}{\partial p_{i}} - \dfrac{\partial f}{\partial p_{i}}\dfrac{\partial H}{\partial q_{i}} \right)\]

Per semplificare la notazione si introduce la **parentesi di Poisson** \cite{arnold1992matematici}:

\[
\left\{ f,H \right\} = \sum_{i = 1}^{N}\left( \dfrac{\partial f}{\partial q_{i}}\dfrac{\partial H}{\partial p_{i}} - \dfrac{\partial f}{\partial p_{i}}\dfrac{\partial H}{\partial q_{i}} \right)
\]

\subsection{Hamiltoniana per sistema con un grado di libertà}\label{hamiltoniana-per-sistema-con-un-grado-di-libertuxe0}

Si vuole valutare la funzione di Hamilton per un sistema con un grado di libertà immerso in un potenziale quadratico, come la forza di richiamo elastica. Questa condizione si applica anche nei punti di minimo del potenziale \(U\), in cui vale un'approssimazione del secondo ordine. Per definizione, l'hamiltoniana è data da:

\[H = E = T + U\]

Dove:

\[T = \dfrac{1}{2}mv^{2}\]

Si scrive l'energia cinetica \(T\) in funzione della quantità di moto. Risulta:

\[p = mv\]

Elevando al quadrato, si ottiene:

\[p^{2} = m^{2}v^{2}\]

Isolando la velocità:

\[v^{2} = \dfrac{p^{2}}{m^{2}}\]

Si sostituisce questo risultato nell'energia cinetica:

\[
T = \dfrac{1}{2}mv^{2} = \dfrac{1}{2}m\dfrac{p^{2}}{m^{2}} = \dfrac{p^{2}}{2m}
\]

Il potenziale, invece, dipende solamente dalla posizione, per cui è dato da:

\[
U = \dfrac{1}{2}kq^{2}
\]

L'hamiltoniana può essere scritta come:

\[
H = \dfrac{p^{2}}{2m} + \dfrac{1}{2}kq^{2}
\]

Si applicano le proprietà dell'hamiltoniana, dunque, si eseguono le derivate parziali:

\[
\begin{cases}
\displaystyle {\dot{q}}_{i} = \dfrac{\partial H}{\partial p_{i}} \\
\displaystyle {\dot{p}}_{i} = - \dfrac{\partial H}{\partial q_{i}}
\end{cases} \Leftrightarrow \begin{cases}
\displaystyle \dot{q} = \dfrac{p}{m} \\
\dot{p} = - kq
\end{cases}
\]

Risolvendo il sistema, si ottiene l'andamento della traiettoria generalizzata \(q\). A tale scopo si deriva la prima equazione rispetto al tempo:

\[\ddot{q} = \dfrac{\dot{p}}{m}\]

Sostituendo la seconda equazione, si ottiene l'equazione dell'oscillatore armonico, la cui soluzione è nota:

\[
\ddot{q} = - \dfrac{k}{m}q
\]

\section{Metodo di Eulero}\label{metodo-di-eulero}

Si considera il sistema di equazioni differenziali del secondo ordine, nelle funzioni incognite \(y_{1}\) e \(y_{2}\):

\[
\begin{cases}
{\ddot{y}}_{1} = f_{1}\left( {\dot{y}}_{i},{\dot{y}}_{2},y_{1},y_{2} \right) \\
{\ddot{y}}_{2} = f_{2}\left( {\dot{y}}_{i},{\dot{y}}_{2},y_{1},y_{2} \right)
\end{cases}
\]

Dove \(f_{1}\) e \(f_{2}\) sono due funzioni qualsiasi che legano la derivata seconda di una funzione incognite con le derivate prime e funzioni incognite stesse. Per rendere il sistema del primo ordine si considerano due variabili ausiliarie, \(A_{1}\) e \(A_{2}\), definite come:

\[
\begin{cases}
A_{1} = {\dot{y}}_{1} \\
A_{2} = {\dot{y}}_{2}
\end{cases}
\]

Si ottiene così un sistema del primo ordine con quattro funzioni incognite, \(A_{1}\), \(A_{2}\), \(y_{1}\) e \(y_{2}\):

\[
\begin{cases}
A_{1} = {\dot{y}}_{1} \\
A_{2} = {\dot{y}}_{2} \\
{\dot{A}}_{1} = f_{1}\left( A_{1},A_{2},y_{1},y_{2} \right) \\
{\dot{A}}_{2} = f_{2}\left( A_{1},A_{2},y_{1},y_{2} \right)
\end{cases}
\]

Il sistema può essere risolto mediante il metodo di Eulero degli elementi finiti. Si considera la prima equazione:

\[A_{1} = \dfrac{dy_{1}}{dt} \Leftrightarrow dy_{1} = A_{1}dt\]

Passando agli incrementi finiti, è possibile approssimare l'equazione:

\[
\Delta y_{1} \simeq A_{1}\Delta t
\]

Per tutte le altre equazioni è possibile procedere allo stesso modo:

\[
\begin{cases}
\Delta y_{1} \simeq A_{1}\Delta t \\
\Delta y_{2} \simeq A_{2}\Delta t \\
\Delta A_{1} \simeq f_{1}\left( A_{1},A_{2},y_{1},y_{2} \right)\Delta t \\
\Delta A_{2} \simeq f_{2}\left( A_{1},A_{2},y_{1},y_{2} \right)\Delta t
\end{cases}
\]

Dividendo l'intervallo temporale in intervalli sufficientemente piccoli è possibile determinare la soluzione approssimata del sistema. Tale metodo è in grado di fornire soluzioni valide solamente se le funzioni incognite non variano troppo rapidamente rispetto gli intervalli di tempo scelti per l'analisi \(\Delta t\). Con funzioni rapidamente variabili sono possibili anche errori importanti.

\subsection{Risoluzione sistema con MatLab}\label{risoluzione-sistema-con-matlab}

Si considera il seguente sistema di equazioni differenziali del secondo ordine, nelle funzioni incognite \(y_{1}\) e \(y_{2}\) \cite{landau1994meccanica}:

\[
\begin{cases}
{\ddot{y}}_{1} = - 5.5y_{1} + 1.1y_{2} \\
{\ddot{y}}_{2} = 1.1y_{1} - 1.2y_{2}
\end{cases}
\]

Ponendo \(z_{1} = {\dot{y}}_{1},z_{2} = {\dot{y}}_{2}\), si ottiene il sistema:

\[
\begin{cases}
{\dot{y}}_{1} = z_{1} \\
{\dot{y}}_{2} = z_{2} \\
{\dot{z}}_{1} = - 5.5y_{1} + 1.1y_{2} \\
{\dot{z}}_{2} = 1.1y_{1} - 1.2y_{2}
\end{cases}
\]

Si pone il sistema in forma matriciale:

\[
\begin{pmatrix}
{\dot{y}}_{1} \\
{\dot{y}}_{2} \\
{\dot{z}}_{1} \\
{\dot{z}}_{2}
\end{pmatrix} = \begin{pmatrix}
0 & 0 & 1 & 0 \\
0 & 0 & 0 & 1 \\
- 5.5 & 1.1 & 0 & 0 \\
1.1 & - 1.2 & 0 & 0
\end{pmatrix}\begin{pmatrix}
y_{1} \\
y_{2} \\
z_{1} \\
z_{2}
\end{pmatrix}
\]

Utilizzando le variabili ausiliarie \(z_{1}\) e \(z_{2}\), è possibile avere una matrice quadrata. Si pone:

\[\textbf{A} = \begin{pmatrix}
- 5.5 & 1.1 \\
1.1 & - 1.2
\end{pmatrix}\]

La matrice dei termini noti \(\textbf{C}\) può essere scritta come:

\[\textbf{C} = \begin{pmatrix}
0 & 0 & 1 & 0 \\
0 & 0 & 0 & 1 \\
- 5.5 & 1.1 & 0 & 0 \\
1.1 & - 1.2 & 0 & 0
\end{pmatrix} = \begin{pmatrix}
{\textbf{0}}_{2 \times 2} & {\textbf{I}}_{2 \times 2} \\
\textbf{A} & \textbf{0}_{2 \times 2}
\end{pmatrix}\]

Dove \({\textbf{I}}_{2 \times 2}\) è la matrice identità \(2 \times 2\), mentre \(\textbf{0}_{2 \times 2}\) è la matrice nulla \(2 \times 2\).

Definendo \(\vec{y}\) il vettore delle funzioni incognite:

\[\vec{y} = \begin{pmatrix}
y_{1} \\
y_{2} \\
z_{1} \\
z_{2}
\end{pmatrix}\]

Il sistema può essere scritto come:

\[
\dot{\vec{y}} = \textbf{C}\vec{y}
\]

La soluzione di questa equazione è del tipo:

\[
\vec{y} = \vec{k}\exp\left( \lambda\textbf{I}t \right)
\]

Sostituendo nel sistema, si ottiene:

\[\lambda\vec{k}\textbf{I}\exp\left( \lambda\textbf{I}t \right) = \textbf{C}\vec{k}\exp\left( \lambda\textbf{I}t \right)\]

Poiché la funzione esponenziale è sempre non nulla, esiste l'inversa a \(\vec{k}\exp\left( \lambda\textbf{I}t \right)\), si ha:

\[\lambda\textbf{I} = \textbf{C} \Leftrightarrow \textbf{C} - \lambda\textbf{I} = \textbf{0}\]

Affinché il sistema ammetta soluzioni non banali bisogna porre:

\[
\det\left( \textbf{C} - \lambda\textbf{I} \right) = 0
\]

Di conseguenza, \(\lambda\) sono gli autovalori della matrice dei coefficienti. Si calcolano, dunque, gli autovalori:

\[\det\left( \textbf{C} - \lambda\textbf{I} \right) = \begin{vmatrix}
- \lambda & 0 & 1 & 0 \\
0 & - \lambda & 0 & 1 \\
- 5.5 & 1.1 & - \lambda & 0 \\
1.1 & - 1.2 & 0 & - \lambda
\end{vmatrix} = 0\]

La cui soluzioni sono:

\[
\begin{cases}
\lambda_{1} = j2.4011 \\
\lambda_{2} = - j2.4011 \\
\lambda_{3} = j0.9669 \\
\lambda_{4} = - j0.9669
\end{cases}
\]

Di conseguenza, le soluzioni sono del tipo:

\[
y_{i} = k_{i,1}\cos\left( \omega_{i,1}t + \vartheta_{i,1} \right) + k_{i,2}\cos\left( \omega_{i,2}t + \vartheta_{i,2} \right),\ i = 1,2
\]

Dove \(k_{i,j}\) e \(\vartheta_{i,j}\), con \(i,j = 1,2\) sono costanti ottenute imponendo le condizioni iniziali; mentre \(\omega_{i,1}\) sono le pulsazioni naturali del sistema.

Per ottenere la soluzione si ricorre a MATLAB. Per prima cosa, si pulisce l'ambiente.

\begin{lstlisting}
clear all
close all
\end{lstlisting}

Si definiscono i parametri del sistema. La matrice dei coefficienti è definita come globale perché deve essere letta anche da una funzione, richiamata dalla \emph{main function}.

\begin{lstlisting}
global A
A=[-5.5 1.1;1.1 -1.2]; %matrice dei coefficienti del sistema
Ts=0.001;
t_span=0:Ts:200;
y0 = {.5,.5,0,0}'; %si usa il trasposto perche' e' necessario avere un vettore colonna
\end{lstlisting}

Si risolve il sistema mediante ode45, il risolutore di equazioni differenziali. Bisogna utilizzare una funzione che implementa il sistema di equazioni differenziali.

\begin{lstlisting}
eq=@sistema\_f;
[t,s]=ode45(eq,t\_span,y0);
a\_val=eig(A);
w=abs(a\_val);
w1=sqrt(w(1))
w2=sqrt(w(2))
\end{lstlisting}

risulta che:

\[\omega_{1} = 2.4011\]

\[\omega_{2} = 0.9669\]

Si plottano le due funzioni e i picchi spettrali.

\begin{lstlisting}
subplot(1,2,1)
plot(t,[s(:,1) s(:,2)])
subplot(1,2,2)
L = length(t);
fax = (1/Ts)*(0:L-1)/L; %Si normalizza l'asse delle frequenze
plot(fax,abs(fft(s(:,1:2))))
set(gca,'xlim',[0 1])
\end{lstlisting}

\begin{figure}[ht]
\centering
\includegraphics[width=4.93333in,height=3.95833in,alt={P754\#yIS1}]{media/1_Meccanica/image10.pdf}\caption{Andamento delle soluzioni e relativa risposta spettrale}
\end{figure}

Come si vede dalla trasformata di Fourier, le soluzioni del sistema accoppiato contengono due frequenze di oscillazione ben distinte (\(\omega_{1}\) e \(\omega_{2}\)). A causa dell'interazione tra le due funzioni, entrambe le soluzioni \(y_{1}\) e \(y_{2}\) sono una combinazione di queste due frequenze, che sono differenti dal caso non interagente.

\begin{center}
\vfill
    \chapter{Meccanica relativistica}
    \label{blx:refsection\therefsection}
\vfill

\minitoc
\newpage
\end{center}
\justify

\section{Discordanza tra Meccanica ed Elettromagnetismo}\label{discordanza-meccanica-elettromagnetismo}

Verso la fine del XIX secolo, emerse una profonda discordanza tra le leggi della \textbf{meccanica classica}, basate sulle \textbf{trasformazioni di Galileo}, e quelle dell'\textbf{elettromagnetismo}, descritte dalle equazioni di Maxwell. Le equazioni di Maxwell prevedevano che la luce, e più in generale ogni radiazione elettromagnetica, si propagasse nel vuoto con una velocità costante $c$, pari a circa $3 \times 10^8$ m/s, in \textbf{tutti i sistemi di riferimento inerziali}. Tuttavia, se si applicavano le \textbf{trasformazioni galileiane per la composizione delle velocità}, un osservatore in moto relativo $v$ avrebbe dovuto misurare una velocità $c' = c \pm v$. Questo risultato era in palese contraddizione con la previsione di Maxwell che $c$ fosse una \textbf{costante universale}, rendendo le leggi dell'elettromagnetismo non invarianti sotto trasformazioni galileiane, a differenza di quelle della meccanica.

L'\textbf{esperimento di Michelson-Morley}, condotto nel 1887, mirava originariamente a misurare la velocità della Terra rispetto al presunto \textbf{etere luminifero}. Il celebre \textbf{risultato nullo} dell'esperimento dimostrò l'impossibilità di rilevare tale moto, fornendo una cruciale evidenza sperimentale che supportò l'idea che la \textbf{velocità della luce $c$ è la stessa} in tutti i sistemi di riferimento inerziali, indipendentemente dal moto della sorgente o dell'osservatore. Questa evidenza sperimentale non era conciliabile con la fisica basata sulle trasformazioni di Galileo, dimostrando la necessità di una profonda \textbf{revisione delle fondamenta della fisica classica} e portando alla formulazione della Teoria della Relatività Ristretta.

\subsection{Trasformazioni galileiane}\label{trasformazioni-galileiane}

Per comprendere le previsioni della meccanica classica si considerino due sistemi di riferimento inerziali, $K$ e $K^'$, in moto relativo uniforme l'uno rispetto all'altro con velocità costante $\vec{v}$. Si assume che il moto del sistema $K'$ rispetto a $K$ avvenga unicamente lungo l'asse $x$, in modo che gli assi restino paralleli e che $K$ e $K'$ coincidano all'istante $t=t'=0$.

\begin{figure}[h!]
\centering
\begin{tikzpicture}[>=stealth,scale=1.2]

% Assi principali (sistema non primato)
\draw[->] (0,-0.2,0) -- (2,-0.2,0) node[below] {$x$};
\draw[->] (0,-0.2,0) -- (0,2,0) node[left] {$y$};
\draw[->] (0,-0.2,0) -- (-1,-1,0) node[left] {$z$};

% Assi primati
\draw[->] (1,0,0) -- (3,0,0) node[below] {$x'$};
\draw[->] (1,0,0) -- (1,2,0) node[left] {$y'$};
\draw[->] (1,0,0) -- (0,-1,0) node[left] {$z'$};

% Velocità V
\draw[->] (1,1.7,0) -- (2,1.7,0) node[right] {$V$};

% Punti
\node[circle,fill,inner sep=2pt] (B) at (1.3,0) {};
\node[circle,fill,inner sep=2pt] (A) at (1.8,0) {};
\node[circle,fill,inner sep=2pt] (C) at (2.3,0) {};

\node[above] at (B) {B};
\node[above] at (A) {A};
\node[above] at (C) {C};

% Frecce attorno ad A
\draw[<->] (1.3,0.45) -- (2.3,0.45);

\end{tikzpicture}
\caption{Sistemi di riferimento in moto relativo}
\label{fig:2_SistRef}
\end{figure}

L'ipotesi fondamentale della meccanica classica è il \textbf{tempo assoluto}, ovvero che il tempo scorra nello stesso modo per tutti gli osservatori inerziali ($t = t'$). In queste condizioni, le \textbf{Trasformazioni Galileiane di Posizione} si scrivono come:

\[
\begin{cases}
x = x' + vt \\
y = y' \\
z = z' \\
t = t'
\end{cases}
\]

Derivando queste relazioni rispetto al tempo, si ottengono le \textbf{Trasformazioni Galileiane di Velocità} o regola di composizione classica delle velocità:

\[
\begin{cases}
v_{x} = v'_{x} + v \\
v_{y} = v'_{y} \\
v_{z} = v'_{z}
\end{cases}
\]

Da queste relazioni, si evince che se una radiazione elettromagnetica viaggia con velocità $c$ lungo l'asse $x'$ nel sistema $K'$ (quindi $v'_{x} = c$), un osservatore nel sistema $K$ misurerebbe una velocità $v_x$ pari a $c + v$ (se $K'$ si allontana da $K$). Questa previsione, basata sul \textbf{Principio di Relatività Galileiana} e sull'ipotesi di \textbf{tempo assoluto}, è in \textbf{netto contrasto} con il risultato nullo dell'esperimento di Michelson-Morley e con la costanza della velocità della luce \cite{kittel1965meccanica}.

Questo profondo conflitto teorico-sperimentale evidenziò che l'ipotesi classica di $t=t'$ non è valida quando si considerano fenomeni ad alta velocità \cite{kittel1965meccanica}.

\section{Relatività ristretta}\label{relativita-ristretta}

Assumendo che il principio di relatività sia valido, all'inizio del '900 Einstein dimostrò che il concetto di tempo non è assoluto ma \textbf{relativo}. Egli si basò su due postulati fondamentali:

\begin{enumerate}
    \item Le leggi della fisica sono le stesse in tutti i sistemi di riferimento inerziali (Principio di Relatività).
    \item La velocità della luce nel vuoto ($c$) è la stessa in tutti i sistemi di riferimento inerziali, indipendentemente dal moto della sorgente o dell'osservatore (Principio di Invarianza di $c$).
\end{enumerate}

Si considerano due sistemi inerziali $K$ e $K'$, di cui il primo fermo, mentre il secondo in moto lungo la direzione delle $x$ positive con velocità costante $v$. Il secondo postulato di Einstein implica la non assolutezza del tempo. Per dimostrarlo, si considera l'esperimento mentale dei tre punti. Siano $A$, $B$ e $C$ tre punti sull'asse delle $x'$, con $A$ e $C$ equidistanti da $B$, sorgente luminosa. I tre punti sono solidali con il sistema di riferimento $K'$.

\begin{itemize}
    \item Per un osservatore solidale con $K'$: Egli vede i tre punti fermi. Poiché la luce procede con velocità $c$ in entrambe le direzioni e le distanze $BA$ e $BC$ sono uguali, la luce giunge \textbf{contemporaneamente} sui punti $A$ e $C$.
    \item Per un osservatore solidale con $K$: I tre punti si muovono con velocità $v$ nel verso positivo delle $x$. In particolare, $A$ si muove verso la luce emessa da $B$, mentre $C$ si allontana da essa. Poiché la velocità della luce deve essere $c$ anche in $K$, essa, dovendo percorrere cammini diversi, raggiunge prima il punto $A$ e, successivamente, il punto $C$.
\end{itemize}

Da questo esempio discende che, cambiando sistema di riferimento, gli eventi che in un sistema di riferimento $K'$ sono contemporanei, non lo sono nel sistema di riferimento $K$. Si conclude che la \textbf{simultaneità} è un concetto relativistico legato al sistema di riferimento considerato.

Per rispettare il principio di invarianza della velocità della luce, le trasformazioni tra i sistemi $K$ e $K'$ devono essere diverse da quelle galileiane. A tale scopo si supponga che un'onda luminosa sferica si propaghi partendo dall'origine dei due sistemi all'istante $t=t'=0\ \text{s}$ e si propaghi con velocità $c$. L'equazione che descrive il fronte d'onda deve essere la stessa nei due sistemi:

\[
\begin{cases}
x^{2} + y^{2} + z^{2} = r^{2} \\
{x'}^{2} + {y'}^{2} + {z'}^{2} = {r'}^{2}
\end{cases} 
\]

dove, i raggi sono dati da:

\[
r = ct, \quad \text{e} \quad r' = ct'
\]

Dato che gli assi $y$ e $z$ sono perpendicolari al moto, si ha $y=y'$ e $z=z'$. Per cui, è possibile scrivere:


\[
\begin{cases}
x^{2} + y^{2} + z^{2} = c^{2}t^{2} \\
{x'}^{2} + {y'}^{2} + {z'}^{2} = c^{2}{t'}^{2}
\end{cases} 
\]

Siccome gli eventi non sono simultanei nei due sistemi di riferimento, \(t' \neq t\).

Einstein postulò un legame di tipo lineare tra le coordinate, un'ipotesi necessaria per la ricerca della trasformazione, nella forma:

\[
\begin{cases}
x' = ax + bt \\
t' = gx + et
\end{cases}
\]

Si sostituisce queste relazioni nell'equazione che descrive l'espansione dell'onda in \(K'\):

\[
\left(ax + bt\right)^{2} + y^{2} + z^{2} = c^{2}\left(gx + et\right)^{2}
\]

Sviluppando i quadrati:

\[
a^{2}x^{2} + 2abxt + b^{2}t^{2} + y^{2} + z^{2} = c^{2}g^{2}x^{2} + 2c^{2}gext + c^{2}e^{2}t^{2}
\]

Raccogliendo, si ha:

\[
\left( a^{2} - c^{2}g^{2} \right)x^{2} + \left( 2ab - 2c^{2}ge \right)xt + y^{2} + z^{2} = \left( c^{2}e^{2} - b^{2} \right)t^{2}
\]

Questa equazione descrive l'onda nel sistema di riferimento \(K\), quindi, deve essere paragonata all'equazione:

\[
x^{2} + y^{2} + z^{2} = c^{2}t^{2}
\]

Questa equazione deve essere proporzionale a $x^{2} + y^{2} + z^{2} - c^{2}t^{2} = 0$. Uguagliando i coefficienti, si ottiene il sistema di equazioni:

\[
\begin{cases}
a^{2} - c^{2}g^{2} = 1 \quad \text{(coeff. di } x^2) \\
2ab - 2c^{2}ge = 0 \quad \text{(coeff. di } xt) \\
c^{2}e^{2} - b^{2} = c^{2} \quad \text{(coeff. di } t^2)
\end{cases}
\]

Il sistema ottenuto presenta tre equazioni, nelle incognite $a$, $b$, $e$ e $g$. È, dunque, necessario aggiungere una quarta equazione che leghi i coefficienti alla velocità relativa $v$. A tale scopo, si considera un punto fermo nel sistema $K'$, per il quale $dx' = 0$. Dalla trasformazione di $x'$, differenziando, si ha:

\[
dx' = a\ dx + b\ dt = 0 \Leftrightarrow \dfrac{dx}{dt} = - \dfrac{b}{a}
\]

La quantità $dx/dt$ è la velocità di un punto fermo in $K'$ misurata in $K$. Poiché il sistema $K'$ si muove rispetto a $K$ con velocità $v$, si deve avere:

\[
\dfrac{dx}{dt} = v
\]

Pertanto, è valida la relazione:

\[
v = - \dfrac{b}{a} \Leftrightarrow b = -av
\]

Si ottiene il sistema risolvibile:

\[
\begin{cases}
a^{2} - c^{2}g^{2} = 1 \quad \text{(Eq. 1)} \\
ab - c^{2}ge = 0 \quad \text{(Eq. 2)} \\
c^{2}e^{2} - b^{2} = c^{2} \quad \text{(Eq. 3)} \\
b = - av \quad \text{(Eq. 4)}
\end{cases}
\]

Si considera la terza equazione:

\[c^{2}e^{2} - b^{2} = c^{2}\]

Si sostituisce l'ultima:

\[c^{2}e^{2} - v^{2}e^{2} = c^{2}\]

Da cui è possibile ricavare \(e\):

\[e^{2} = \dfrac{c^{2}}{c^{2} - v^{2}} = \dfrac{1}{1 - \left( \dfrac{v}{c} \right)^{2}}\]

\[e = \dfrac{1}{\sqrt{1 - \left( \dfrac{v}{c} \right)^{2}}}\]

Noto \(e\) è possibile risalire a \(b\) dalla quarta equazione:

\[b = - ve = - \dfrac{v}{\sqrt{1 - \left( \dfrac{v}{c} \right)^{2}}}\]

Dalla seconda equazione è possibile ricavare \(g\) in funzione di \(a\):

\[c^{2}ge = ab \Leftrightarrow g = \dfrac{1}{c^{2}}a\dfrac{b}{e}\]

Per la quarta equazione, si ha:

\[g = - \dfrac{v}{c^{2}}a\]

Si sostituisce questo risultato nella prima equazione:

\[a^{2} - c^{2}g^{2} = 1 \Leftrightarrow a^{2} - c^{2}\dfrac{v^{2}}{c^{4}}a^{2} = 1 \Leftrightarrow \left( 1 - \dfrac{v^{2}}{c^{2}} \right)a^{2} = 1 \Leftrightarrow a = \dfrac{1}{\sqrt{1 - \left( \dfrac{v}{c} \right)^{2}}}\]

Noto \(a\), è possibile risalire a \(g\):

\[g = - \dfrac{v}{c^{2}}a = - \dfrac{v}{c^{2}}\dfrac{1}{\sqrt{1 - \left( \dfrac{v}{c} \right)^{2}}}\]

I coefficienti determinati sono, dunque:

\[
\begin{cases}
a = \dfrac{1}{\sqrt{1 - \left( \dfrac{v}{c} \right)^{2}}} \\
b = - \dfrac{v}{\sqrt{1 - \left( \dfrac{v}{c} \right)^{2}}} \\
g = - \dfrac{v}{c^{2}}\dfrac{1}{\sqrt{1 - \left( \dfrac{v}{c} \right)^{2}}} \\
e = \dfrac{1}{\sqrt{1 - \left( \dfrac{v}{c} \right)^{2}}}
\end{cases}
\]

Dato che il rapporto della velocità di movimento del sistema di riferimento sulla velocità della luce compare frequentemente, si pone:

\[
\dfrac{v}{c} = \beta
\]

La quantità:

\[
\gamma = \dfrac{1}{\sqrt{1 - \left( \dfrac{v}{c} \right)^{2}}} = \dfrac{1}{\sqrt{1 - \beta^{2}}}
\]

È detto \textbf{fattore di Lorentz}. Con queste definizioni, i coefficienti delle equazioni di composizione possono essere scritti come:

\[
\begin{cases}
a = \gamma \\
b = - v\gamma \\
 g = - \dfrac{v}{c^{2}}\gamma \\
e = \gamma
\end{cases} \Leftrightarrow \begin{cases}
a = \gamma \\
b = - \beta\gamma c \\
 g = - \dfrac{\beta}{c}\gamma \\
e = \gamma
\end{cases} 
\]

Le leggi di trasformazioni tra il sistema \(K\) e \(K'\) sono dette \textbf{trasformazioni di Lorentz} e sono \cite{landau1975campi,feynman1964vol1}:

\[
\begin{cases}
x' = ax + bt \\
y' = y \\
z' = z \\
t' = gx + et
\end{cases} \Leftrightarrow \begin{cases}
x' = \gamma (x - vt) \\
y' = y \\
z' = z \\
t' = \gamma \left( t - \dfrac{vx}{c^{2}} \right)
\end{cases}
\]

\subsection{Composizione delle velocità}\label{composizione-delle-velocituxe0}

Si determinano, ora, le leggi di trasformazione per la velocità; a tale scopo si differenziano le trasformazioni di Lorentz:

\[
\begin{cases}
dx' = \gamma (dx - v\,dt) \\
dy' = dy \\
dz' = dz \\
dt' = \gamma \left( dt - \dfrac{v}{c^{2}}\,dx \right)
\end{cases}
\]

Dividendo una delle variazioni spaziali per $dt'$ si ottengono le velocità lungo gli assi nel sistema $K'$.

La velocità $v_{x'}$ è data dal rapporto $dx'/dt'$. Sostituendo le espressioni differenziali:
\[
v_{x'} = \dfrac{dx'}{dt'} = \dfrac{\gamma(dx - v\,dt)}{\gamma\left( dt - \dfrac{v}{c^{2}}\,dx \right)} = \dfrac{dx - v\,dt}{dt - \dfrac{v}{c^{2}}\,dx}
\]

Al secondo membro si divide numeratore e denominatore per $dt$ per far comparire le velocità $v_x = dx/dt$:

\[
v_{x'} = \dfrac{\dfrac{dx}{dt} - v}{1 - \dfrac{v}{c^{2}}\dfrac{dx}{dt}}
\]

Posto $v_{x} = dx/dt$ come la componente di velocità lungo l'asse \(x\) con cui si muove un fenomeno nel sistema \(K\), si ottiene la formula di composizione relativistica:

\[
v_{x'} = \dfrac{v_{x} - v}{1 - \dfrac{v_{x}v}{c^{2}}}
\]

Lungo \(y'\), la velocità è data da:

\[
v_{y'} = \dfrac{dy'}{dt'} = \dfrac{dy}{\gamma\left( dt - \dfrac{v}{c^{2}}\,dx \right)}
\]

Moltiplicando e dividendo per \(dt\) al secondo membro, si ha:

\[
v_{y'} = \dfrac{\dfrac{dy}{dt}}{\gamma\left( 1 - \dfrac{v}{c^{2}}\dfrac{dx}{dt} \right)}
\]

Dove compare la componente di velocità lungo l'asse \(x\) con cui si muove un fenomeno nel sistema \(K\):

\[
\dfrac{dy}{dt} = v_{y} \Rightarrow v_{y'} = \dfrac{v_{y}}{\gamma\left( 1 - \dfrac{v_{x}v}{c^{2}} \right)}
\]

Analogamante, per la componente lungo \(z'\) si ha:

\[
v_{z'} = \dfrac{dz'}{dt'} = \dfrac{dz}{\gamma\left( dt - \dfrac{v}{c^{2}}\,dx \right)} = \dfrac{v_{z}}{\gamma\left( 1 - \dfrac{v_{x}v}{c^{2}} \right)}
\]

Si considera il caso in cui la velocità $v_{x} = c$ (un raggio di luce nel sistema $K$). Nel sistema $K'$, la velocità $v_{x'}$ diventa:

\[
v_{x'} = \left. \ \dfrac{v_{x} - v}{1 - \dfrac{v_{x}v}{c^{2}}} \right|_{v_{x} = c} = \dfrac{c - v}{1 - \dfrac{cv}{c^{2}}} = \dfrac{c - v}{1 - \dfrac{v}{c}} = \dfrac{c - v}{\dfrac{c - v}{c}} = c
\]

Le trasformazioni di Lorentz per la composizione delle velocità verificano, quindi, il \textbf{Principio di Invarianza della Velocità della Luce}, risolvendo la contraddizione presente nelle leggi di composizione galileiane.

\subsection{Fattore di Lorentz}\label{fattore-di-lorentz}

Il \textbf{fattore di Lorentz} ($\gamma$) è definito come:

\[
\gamma = \dfrac{1}{\sqrt{1 - \left( \dfrac{v}{c} \right)^{2}}}
\]

dove $v$ è la velocità relativa tra i sistemi di riferimento e $c$ è la velocità della luce.

Le velocità considerate nella maggior parte delle applicazioni pratiche sono molto minori della velocità della luce:

\[
\dfrac{v}{c} \ll 1
\]

In questo caso, il fattore $\left(v/c\right)^{2}$ è trascurabile, e il fattore di Lorentz tende all'unità:

\[
\gamma \simeq 1 \quad \text{per } v \ll c
\]

In questo limite, la relatività ristretta ricade nella \textbf{meccanica classica} (o meccanica newtoniana). La meccanica classica, in ultima analisi, è un caso particolare della meccanica relativistica di Einstein, valido quando gli effetti relativistici sono minimi.

Al crescere della velocità $v$, il fattore di Lorentz, sempre positivo, tende a crescere. Nel caso limite in cui la velocità si avvicina alla velocità della luce ($v \rightarrow c$), il denominatore tende a zero:

\[
\lim_{v \to c} \gamma = \lim_{v \to c} \dfrac{1}{\sqrt{1 - \left( \dfrac{v}{c} \right)^{2}}} \rightarrow \infty
\]

\begin{figure}[h!]
\centering
\begin{figure}[h!]
\centering
\begin{tikzpicture}

\begin{axis}[
    axis lines=middle,
    xlabel={$V$},
    ylabel={$\gamma$},
    xmin=0, xmax=1.2,
    ymin=1, ymax=6,
    xtick={0.2,0.4,0.6,0.8, 1},
    xticklabels={$0.2c$, $0.4c$, $0.6c$, $0.8c$, $c$},
    ytick={1},
    domain=0:0.99,
    samples=200,
    smooth,
    width=11cm,
    height=7cm,
    axis line style={->},
    tick style={black},
]

% Curva gamma
\addplot[black, thick] {1/sqrt(1-x^2)};

% Linea tratteggiata vicino a c
\draw[dashed] (axis cs:0.99,0) -- (axis cs:0.99,10);

\end{axis}

\end{tikzpicture}
\caption{Andamentodel fattorediLorentz al variare della velocità}
\label{fig:2_GammaFactor}
\end{figure}
\caption{Andamento del fattore di Lorentz al variare della velocità}
\label{fig:2_GammaFactor}
\end{figure}


La velocità della luce ($c$) rappresenta un \textbf{limite irraggiungibile} per qualsiasi corpo massivo secondo la meccanica relativistica.

\subsection{Dilatazione dei tempi}\label{dilatazione-dei-tempi}

Si considerano due sistemi di riferimento inerziali \(K\) e \(K'\), in moto relativo tra loro con velocità \(v\) lungo la direzione delle \(x\) positive. Si suppone che \(K\) sia fermo.

Per ricavare $dt$ in funzione di $dt'$ e $dx'$, si parte dalle Trasformazioni di Lorentz:

\[
\begin{cases}
dx' = \gamma(dx - \beta c)dt \\
dy' = dy \\
dz' = dz \\
dt' = \gamma\left( - \dfrac{\beta}{c}dx + dt \right)
\end{cases}
\]

Si considerano solamente la prima e la quarta equazione. Tramite queste si ricavano le coordinate del sistema di riferimento fisso \(K\) in funzione di quelle relative a \(K'\)

\[
\begin{cases}
dx' = \gamma(dx - \beta c)dt \\
dt' = \gamma\left( - \dfrac{\beta}{c}dx + dt \right)
\end{cases}  \Leftrightarrow \begin{cases}
\dfrac{1}{\gamma}dx' = dx - \beta c\ dt \\
\dfrac{1}{\gamma}dt' = - \dfrac{\beta}{c}\ dx + dt
\end{cases} \Leftrightarrow \begin{cases}
dx = \dfrac{1}{\gamma}dx' + \beta c\ dt \\
dt = \dfrac{1}{\gamma}dt' + \dfrac{\beta}{c}\ dx
\end{cases}
\]

Si sostituisce la prima equazione nella seconda:

\[
dt = \dfrac{1}{\gamma}dt' + \dfrac{\beta}{c}\ dx \Leftrightarrow dt = \dfrac{1}{\gamma}dt' + \dfrac{\beta}{c}\ \left( \dfrac{1}{\gamma}dx' + \beta c\ dt \right)
\]

Risolvendo, si ha:

\[
dt = \dfrac{1}{\gamma}dt' + \dfrac{\beta}{\gamma c}\ dx' + \beta^{2}dt
\]

Si portano i termini contenenti \(dt\) al primo membro:

\[
\left( 1 - \beta^{2} \right)dt = \dfrac{1}{\gamma}dt' + \dfrac{\beta}{\gamma c}\ dx'
\]

Isolando \(dt\), si ha:

\[
dt = \dfrac{1}{\gamma\left( 1 - \beta^{2} \right)}\left( dt' + \dfrac{\beta}{c}\ dx' \right)
\]

Per definizione, il fattore di Lorentz può essere scritto come:

\[
\gamma = \dfrac{1}{\sqrt{1 - \beta^{2}}}
\]

Sostituendo questo risultato nell'espressione per \(dt\) si ottiene:

\[
dt = \dfrac{\sqrt{1 - \beta^{2}}}{\left( 1 - \beta^{2} \right)}\left( dt' + \dfrac{\beta}{c}\ dx' \right) = \dfrac{1}{\sqrt{1 - \beta^{2}}}\left( dt' + \dfrac{\beta}{c}\ dx' \right) = \gamma\left( dt' + \dfrac{\beta}{c}\ dx' \right)
\]

Noto \(dt\), è possibile ricavare \(dx\) dalla prima equazione:

\[
dx = \dfrac{1}{\gamma}dx' + \beta c\ dt = \dfrac{1}{\gamma}dx' + \beta c\gamma\left( dt' + \dfrac{\beta}{c}\ dx' \right) = \dfrac{1}{\gamma}dx' + \beta c\gamma\ dt' + \beta^{2}\gamma\ dx'
\]

Si raccoglie \(\gamma\) al secondo membro:

\[
dx = \gamma\left\lbrack \left( \dfrac{1}{\gamma^{2}} + \beta^{2} \right)dx' + \beta c\ dt' \right\rbrack
\]

Dove, per definizione del fattore di Lorentz:

\[
\dfrac{1}{\gamma^{2}} + \beta^{2} = 1 - \beta^{2} + \beta^{2} = 1
\]

Con questo risultato, si ottiene l'espressione per \(dx\):

\[
dx = \gamma\left( dx' + \beta c\ dt' \right)
\]

Le due equazioni, per \(dx\) e \(dt\), sono, in definitiva:

\[
\begin{cases}
dx = \gamma\left( dx' + \beta c\ dt' \right) \\
dt = \gamma\left( dt' + \dfrac{\beta}{c}\ dx' \right)
\end{cases}
\]

Si considera un orologio o un fenomeno solidale col sistema di riferimento \(K'\), ovvero fermo rispetto a esso. Ne discende che \(dx' = 0\), in quanto la variazione lungo l'asse \(x\) è nulla, essendo, appunto, il punto fermo. Nel sistema di riferimento \(K\), la variazione temporale è data da:

\[
dt = \left. \gamma\left( dt' + \dfrac{\beta}{c}\ dx' \right) \right|_{dx' = 0} = \gamma\ dt'
\]

Si definisce tempo proprio \(\tau\) il tempo che una particela vede scorrere nel sistema di riferimento in cui è in quiete. Con questa definizione, la relazione può essere scritta come:

\[
dt = \gamma\ d\tau
\]

L'intervallo di tempo $dt$ misurato in $K$ (dove l'orologio è in moto) è maggiore dell'intervallo di tempo proprio $d\tau$ (dove l'orologio è fermo). Ciò significa che, per l'osservatore in $K$, \textbf{gli orologi in movimento scorrono più lentamente} del suo orologio. Questo è l'effetto noto come dilatazione dei tempi.

Per apprezzare tale effetto, noto come dilatazione dei tempi, il fattore di Lorentz \(\gamma\) deve essere significativamente maggiore di \(1\). Tale condizione si verifica quando la velocità con cui si muove \(K'\) rispetto a \(K\) deve essere prossima alla velocità della luce.

Per apprezzare tale effetto, il fattore di Lorentz \(\gamma\) deve essere significativamente maggiore di \(1\).

\begin{itemize}
 \item \textbf{Limite Classico} ($v \ll c$): In questo limite, il fattore di Lorentz è circa unitario ($\gamma \simeq 1$), dunque la dilatazione dei tempi non è apprezzabile:
    
 \[
 dt \simeq d\tau
 \]
 \item \textbf{Limite Relativistico} ($v \to c$): Nel caso limite in cui \(K'\) si muove alla velocità della luce, il fattore di Lorentz tende all'infinito ($\gamma \to \infty$):
 \[
 dt = \left. \ \gamma\ d\tau \right|_{v \to c} \rightarrow \infty
 \]
    Ciò implica che, per un osservatore esterno, il tempo di un oggetto che viaggia alla velocità della luce (se fosse possibile) si fermerebbe.
\end{itemize}

A differenza degli intervalli di tempo misurati nei vari sistemi di riferimento, il tempo proprio \(\tau\) è univocamente determinato noto \(\gamma\), ovvero la velocità relativa di \(K'\) rispetto al sistema di riferimento \(K\), in cui si effettua la misura.

\subsection{Contrazione delle lunghezze}\label{contrazione-delle-lunghezze}

Si considerino due sistemi di riferimento inerziali $K$ e $K'$, in moto relativo con velocità $v$ (con $\beta = v/c$) lungo la direzione positiva dell'asse $x$. Si assuma che il sistema $K$ sia fermo e si osservano due punti nel sistema $K'$ nello stesso istante; pertanto la variazione temporale è nulla: $dt' = 0$.

Nel sistema in moto $K'$ si misura una distanza $dx'$, mentre un osservatore nel sistema fisso $K$ misura una distanza $dx$, legata a $dx'$ dalle trasformazioni di Lorentz:

\[
\begin{cases}
dx = \gamma\left( dx' + \beta c\ dt' \right) \\
 dt = \gamma\left( dt' + \dfrac{\beta}{c}\ dx' \right)
\end{cases} 
\]

dove $\gamma = 1/\sqrt{1-\beta^2}$.

Per misurare la lunghezza di un oggetto in movimento, l'osservatore nel sistema $K$ deve determinare le posizioni delle sue estremità simultaneamente nel proprio sistema, cioè $dt=0$. Impostando quindi la seconda equazione a zero si ottiene:

\[
dt = \gamma\left( dt' + \dfrac{\beta}{c}\ dx' \right) \Leftrightarrow\ \gamma\left( dt' + \dfrac{\beta}{c}\ dx' \right) = 0
\]

Da cui si ricava:

\[
dt' = - \dfrac{\beta}{c}\ dx'
\]

Sostituendo questo risultato nella prima equazione:

\[
dx = \gamma\left( dx' + \beta c\, dt' \right)
     = \gamma\left( dx' + \beta c \left( - \dfrac{\beta}{c}\, dx' \right) \right)
     = \gamma\left( dx' - \beta^2 dx' \right).
\]

Raccogliendo $dx'$:

\[
dx = \gamma\, dx' (1 - \beta^2).
\]

Poiché $1 - \beta^2 = \gamma^{-2}$, si ottiene:

\[
dx = \gamma dx' \left(\dfrac{1}{\gamma^2}\right) 
\]

Semplificando $\gamma$ si ottiene la relazione che lega la lunghezza nel sistema $K$, ovvero $dx$, con quella nel sistema $K'$, $dx'$:

\[
dx = \dfrac{1}{\gamma} dx'
\]

Poiché \(\gamma > 1,v > 0\), la lunghezza misurata nel sistema $K$, in cui l'oggetto è in moto, risulta dunque minore rispetto alla misura $dx'$ ottenuta nel sistema $K'$, in cui l'oggetto è in quiete. La distanza tra due punti è massima nel sistema in cui essi sono fermi; inoltre, a differenza delle misure ottenute negli altri sistemi, essa è univocamente determinata noto $\gamma$, cioè la velocità relativa tra i sistemi.

Infine, noto $\gamma$ e $dx'$, è sempre possibile determinare la corrispondente misura $dx$ in qualsiasi sistema di riferimento inerziale.

\subsection{Energia e quantità di moto relativistiche}\label{energia-e-quantituxe0-di-moto-relativistiche}

Si ricavano le relazioni di energia e quantità di moto nella teoria della Relatività Ristretta partendo dal **Principio di Azione Stazionaria** (o di Hamilton). L'azione $S$ di una particella libera è definita come l'integrale della Lagrangiana $L$ rispetto al tempo coordinato $t$:

\[
S = \int_{t_1}^{t_2}{L dt}
\]

L'azione $S$ deve essere uno \textbf{scalare di Lorentz}, ovvero deve avere lo stesso valore in tutti i sistemi di riferimento inerziali. Affinché $S$ sia invariante, la Lagrangiana $L$ non può essere una semplice estensione della Lagrangiana classica.

Si osserva che l'elemento di linea $ds$ e il tempo proprio $d\tau$ sono invarianti di Lorentz:

\[
ds^2 = c^2 dt^2 - d\vec{x} \cdot d\vec{x} = c^2 d\tau^2
\]

Il tempo proprio $d\tau$ è legato al tempo misurato in un sistema di riferimento inerziale ($dt$) tramite il fattore di Lorentz $\gamma$:

\[
d\tau = \frac{1}{\gamma} dt = \sqrt{1 - \frac{v^{2}}{c^{2}}}dt
\]

Per garantire l'invarianza dell'azione, l'integrando deve essere una quantità proporzionale al tempo proprio, l'unico intervallo di tempo universale per la particella. Si definisce quindi l'azione come:

\[
S = \int_{\tau_1}^{\tau_2}{\alpha d\tau}
\]

dove $\alpha$ è una costante scalare da determinare.

Sostituendo l'espressione per $d\tau$, l'azione può essere riscritta in funzione del tempo coordinato $t$:

\[
S = \int_{t_1}^{t_2}{\alpha \sqrt{1 - \frac{v^{2}}{c^{2}}}dt}
\]

Da cui si ricava la Lagrangiana Relativistica $L^r$:

\[
L^{r} = \alpha \sqrt{1 - \frac{v^{2}}{c^{2}}}
\]

\[S = \int{\alpha d\tau} = \int{\alpha\sqrt{1 - \dfrac{v^{2}}{c^{2}}}dt}\]

La meccanica relativistica deve inglobare la meccanica classica, nel caso limite in cui \(v \ll c\). In questa condizione l'azione deve essere uguale a quella classica, ovvero, la quantità \(\alpha\) deve coincidere con la lagrangiana classica. Per \(v \ll c\), risulta che:

\[\sqrt{1 - \dfrac{v^{2}}{c^{2}}} \simeq \left( 1 - \dfrac{1}{2}\dfrac{v^{2}}{c^{2}} \right),\ \ v \ll c\]

Con questo risultato l'azione, nel limite classico, può essere scritta come:

\[s = \int{\alpha\sqrt{1 - \dfrac{v^{2}}{c^{2}}}dt} \simeq \int{\alpha\left( 1 - \dfrac{1}{2}\dfrac{v^{2}}{c^{2}} \right)dt}\]

Nell'approssimazione classica, l'azione è data da:

\[S = \int{L_{class}dt}\]

L'integrando coincide con la lagrangiana relativistica $L_{appr}^r$, nel limite delle meccanica classica, in cui le velocità in gioco sono molto minori della velocità della luce. Questa quantità deve coincidere con la lagrangiana classica ($L_{class}$), a meno di una costante additiva, in quanto le equazioni di Eulero-Lagrange dipendono solamente dalla derivata della lagrangiana.

Per una particella libera in assenza di un campo di potenziale, la lagrangiana coincide con l'energia cinetica, per cui l'azione è data da:

\[S = \int{L_{class}dt} = \int{\dfrac{1}{2}m_{0}v^{2}dt}\]

Dove \(m_{0}\) è la massa inerziale. Confrontando l'azione ottenuta nel limite classico, per \(v \ll c\), con quella scritta nella teoria classica, si uguagliano i termini in \(v^{2}\): poiché la parte costante non influenza le equazioni di Eulero-Lagrange. Si ottiene:

\[- \dfrac{1}{2}\alpha\dfrac{v^{2}}{c^{2}} = \dfrac{1}{2}m_{0}v^{2}\]

Da cui si ricava \(\alpha\) come:

\[\alpha = - m_{0}c^{2}\]

Con questo risultato si ha la certezza che, nel limite \(v \ll c\), la lagrangiana relativistica e quella classica coincidano, a meno di una costante \(- m_{0}c^{2}\). Infatti, risulta che:

\[
L_{appr}^r=\alpha\left( 1 - \dfrac{1}{2}\dfrac{v^{2}}{c^{2}} \right) = - m_{0}c^{2}\left( 1 - \dfrac{1}{2}\dfrac{v^{2}}{c^{2}} \right) = - m_{0}c^{2} + \dfrac{1}{2}m_{0}v^{2} = - m_{0}c^{2} + L_{class}
\]

La costante non influenza il risultato dell'equazione di Eulero-Lagrange, dunque, può essere trascurata senza problemi.

La lagrangiana deve rispettare il principio di Fermat del minor tempo, secondo il quale, tra tutti i possibili percorsi che uniscono due punti,
un raggio di luce segue il cammino che richiede il minor tempo e che, di conseguenza, rende l'azione stazionaria.

L'azione relativistica può essere espressa come:

\[s = \int{\alpha\sqrt{1 - \dfrac{v^{2}}{c^{2}}}dt} = \int{- m_{0}c^{2}\sqrt{1 - \dfrac{v^{2}}{c^{2}}}dt}\]

La lagrangiana relativistica è, dunque:

\[L^{r} = - m_{0}c^{2}\sqrt{1 - \dfrac{v^{2}}{c^{2}}} = - m_{0}c^{2}\sqrt{1 - \dfrac{\vec{v} \cdot \vec{v}}{c^{2}}}\]

Dall'equivalenza tra meccanica newtoniana e meccanica lagrangiana è possibile ricavare la definizione di momento lineare generalizzato alla meccanica relativistica:

\[\vec{p} = \dfrac{\partial L}{\partial\vec{v}} = \dfrac{\partial}{\partial\vec{v}}\left( - m_{0}c^{2}\sqrt{1 - \dfrac{\vec{v} \cdot \vec{v}}{c^{2}}} \right) = \dfrac{m_{0}\vec{v}}{\sqrt{1 - \dfrac{v^{2}}{c^{2}}}}\]

Da questa relazione è possibile osservare che la massa relativistica è legata alla massa inerziale dalla relazione:

\[m = \dfrac{m_{0}}{\sqrt{1 - \dfrac{v^{2}}{c^{2}}}} = \gamma m_{0}\]

\(m_{0}\) è detta massa a riposo e rappresenta la quantità di materia che possiede un corpo da fermo. La massa \(m\) di una particella dipende, invece, dalla velocità con cui si muove. Nel limite classico, \(v \ll c\), il fattore di Lorentz è prossimo all'unità, per cui, la massa relativistica \(m\) coincide con la massa a riposo:

\[m \simeq m_{0},\ \ v \ll c\]

Se la velocità \(v\) tende a raggiungere la velocità della luce, il fattore di Lorentz tende a diventare infinito; di conseguenza, anche la massa relativistica tende a divergere:

\[m \rightarrow \infty,\ \ v \rightarrow \infty\]

\begin{figure}[ht]
\centering
\begin{tikzpicture}

\begin{axis}[
    axis lines=middle,
    xlabel={$v$},
    ylabel={$m$},
    xmin=0, xmax=1,
    ymin=0, ymax=8.5,
    xtick={1},
    xticklabels={},
    ytick={1,2,3,4,5,6,7,8},
    yticklabels={$m_0$, $2m_0$, $3m_0$, $4m_0$, $5m_0$, $6m_0$, $7m_0$, $8m_0$},
    width=10cm,
    height=8cm,
    domain=0:0.99,
    samples=200,
    smooth,
    axis line style={->},
    tick style={black},
]

% Curva massa relativistica m = gamma m0
\addplot[red, thick] {1/sqrt(1 - x^2)};

% Linea orizzontale della massa a riposo
\addplot[red, thick] coordinates {(0,1) (1,1)};

% Linea verticale in v = c
\draw[dashed] (axis cs:1,0) -- (axis cs:1,8.5);

% Etichetta "massa relativistica"
\node at (axis cs:0.55,3.3) {\small massa relativistica};

% Etichetta "c" posizionata più in alto
\node at (axis cs:0.98,1.5) {$c$};

\end{axis}

\end{tikzpicture}


\caption{Andamento della massa relativistica in funzione della velocità}
\label{fig:2_MassRelav}
\end{figure}

La divergenza della massa relativistica al crescere della velocità spiega perché non sia possibile superare la velocità della luce \(c\). Infatti, l'energia fornita a una particella in parte ne aumenta la velocità e in parte ne accresce la massa relativistica. Di conseguenza, per superare la velocità della luce \(c\) è necessario fornire un'energia infinita, violando il principio di conservazione dell'energia.

La quantità di moto, a differenza del limite classico, è legata alla massa relativistica, che a sua volta dipende dalla velocità. Di conseguenza, la quantità di moto non dipende più in modo lineare dalla velocità:

\[\vec{p} = m\vec{v}\]

Essa presenta, invece, una relazione più complessa, poiché anche la massa dipende dalla velocità:

\[\vec{p} = \dfrac{m_{0}\vec{v}}{\sqrt{1 - \dfrac{v^{2}}{c^{2}}}}\]

La massa \(m\), nel piano \(p - v\), rappresenta la pendenza della curva quantità di moto in funzione della velocità. Nella teoria relativistica, non si ha una retta. L'andamento della quantità di moto relativistica è lineare nel limite classico, dunque per \(v \ll c\).

\begin{figure}[h!]
\centering
\begin{tikzpicture}
\begin{axis}[
    width=15cm,
    height=12cm,
    xlabel={velocità [m/s]},
    ylabel={quantità di moto [kg$\cdot$m/s]},
    xmin=0, xmax=3e8,
    ymin=0, ymax=2e-21,
    xtick={0,0.5e8,1e8,1.5e8,2e8,2.5e8,3e8},
    xticklabel style={/pgf/number format/fixed},
    yticklabel style={/pgf/number format/fixed},
    scaled y ticks=false,
    scaled x ticks=false,
    y label style={at={(axis description cs:-0.1,.5)},rotate=180},
    x label style={at={(axis description cs:0.5,-0.1)}},
    legend style={draw=none},
]

% Parametri
\def\m{9.11e-31} % massa elettrone
\def\c{3e8}      % velocità della luce

% Curva relativistica
\addplot[blue, thick, domain=0:2.99e8, samples=200] 
    {(\m*x)/sqrt(1-(x/\c)^2)};
\addlegendentry{Relativistica}

% Curva classica
\addplot[red, dashed, domain=0:3e8, samples=100] 
    {\m*x};
\addlegendentry{Approssimazione classica}

% Linea verticale c
\addplot[green, dashed] coordinates {(3e8,0) (3e8,2e-21)};

% Etichetta
\node at (axis cs:2.5e8,1.5e-22) {approssimazione classica};

\end{axis}
\end{tikzpicture}

\caption{Andamento della massa relativistica in funzione della velocità}
\label{fig:2_MassaRelativistica}
\end{figure}

Per valutare l'energia relativistica si ricorre alla definizione di lagrangiana:

\[L^{r} = T - U\]

L'energia totale del sistema è data da:

\[E = T + U\]

Nota la lagrangiana è possibile ricavare l'energia totale del sistema. Sommando, infatti, le due equazioni, si ha:

\[L^{r} + E = T - U + T + U = 2T \Leftrightarrow E = 2T - L^{r}\]

È possibile scrivere che:

\[2T = \vec{p} \cdot \vec{v} = \dfrac{\partial L}{\partial\vec{v}} \cdot \vec{v}\]

Con questo risultato l'energia totale può essere espressa come:

\[E = \dfrac{\partial L}{\partial\vec{v}} \cdot \vec{v} - L\]

Sostituendo le espressioni ricavate per la quantità di moto e lagrangiana nell'ambito della teoria relativistica, si ha:

\[E = \dfrac{\partial L}{\partial\vec{v}} \cdot \vec{v} - L = \dfrac{m_{0}\vec{v}}{\sqrt{1 - \dfrac{v^{2}}{c^{2}}}} \cdot \vec{v} + m_{0}c^{2}\sqrt{1 - \dfrac{v^{2}}{c^{2}}} = \dfrac{m_{0}v^{2}}{\sqrt{1 - \dfrac{v^{2}}{c^{2}}}} + m_{0}c^{2}\sqrt{1 - \dfrac{v^{2}}{c^{2}}}\]

Si esegue il minimo comune multiplo al secondo membro e, successivamente, si svolgono i prodotti:

\[E = \dfrac{m_{0}v^{2} + m_{0}c^{2}\left( 1 - \dfrac{v^{2}}{c^{2}} \right)}{\sqrt{1 - \dfrac{v^{2}}{c^{2}}}} = \dfrac{m_{0}v^{2} + m_{0}c^{2} - m_{0}v^{2}}{\sqrt{1 - \dfrac{v^{2}}{c^{2}}}}\]

L'energia totale è, quindi, data da:

\[E = \dfrac{m_{0}}{\sqrt{1 - \dfrac{v^{2}}{c^{2}}}}c^{2}\]

Questa relazione rappresenta una delle equazioni più note di Einstein e della relatività ristretta \cite{landau1994meccanica,feynman1964vol1}. Il termine \(m_{0}c^{2}\) rappresenta un termine energetico, legato allo stato di quiete della particella. Infatti, nel limite classico, è possibile approssimare in serie di Taylor il denominatore:

\[
\dfrac{1}{\sqrt{1 - \dfrac{v^{2}}{c^{2}}}}\simeq 1 + \dfrac{1}{2}\dfrac{v^{2}}{c^{2}}
\]

Sostituendo tale risultato nell'equazione per l'energia $E$, si ottiene:

\[
E = \dfrac{m_{0}}{\sqrt{1 - \dfrac{v^{2}}{c^{2}}}}c^{2}\simeq m_{0}c^{2}\left(1 + \dfrac{1}{2}\dfrac{v^{2}}{c^{2}}\right)
\]

Svolgendo i prodotti, si ricava l'espressione per l'energia totale nel limite classico:

\[
E\simeq m_{0}c^{2} +\dfrac{1}{2}m_{0}v^{2} ,\ \ v \ll c
\]

L'energia totale della particella è data dalla somma dell'energia a riposo ($m_{0}c^{2}$) e del termine di energia cinetica caratteristico della meccanica classica per una particella libera.

Se, invece, la velocità \(v\) approssima quella della luce, l'energia tende a divergere:

\[E \rightarrow \infty,\ \ v \rightarrow \infty\]

Questo risultato ribadisce il concetto che, per portare una particella con massa a riposo \(m_{0}\) da ferma alla velocità della luce \(c\), bisogna fornire un'energia infinita.

Anche in meccanica relativistica valgono i teoremi di conservazione, tuttavia, i principi di conservazione della massa e dell'energia sono sostituiti dal principio di conservazione della massa-energia.

\section{Quadrivettori}\label{quadrivettori}

Lo spazio-tempo o \textbf{cronotopo} è lo spazio quadridimensionale introdotto da Einstein nella relatività ristretta, composto da tre coordinate spaziali e una temporale. Ogni fenomeno fisico è descritto da eventi nello spazio-tempo del tipo \((ct,x,y,z)\), detti quadrivettori. È necessario introdurre come prima componente \(ct\) in modo da avere delle quantità dimensionalmente omogenee nel quadrivettore; inoltre, il quadrivettore è denotato con l'apice \(\alpha\):

\[s^{\alpha} = \begin{pmatrix}
ct, x, y, z
\end{pmatrix}
\]

Si definisce \textbf{intervallo spazio-temporale} $s^2$ la quantità:

\[
s^{2} = c^{2}t^{2} - \left( x^{2} + y^{2} + z^{2} \right)
\]

Il termine $x^{2} + y^{2} + z^{2}$ coincide con la distanza euclidea. L'intervallo $s^{2}$ è una quantità \text{invariante} rispetto alle trasformazioni di Lorentz \cite{landau1994meccanica}. L'invarianza dell'intervallo rispetto al quadrato della distanza euclidea è dovuta al fatto che lo spazio-tempo non è piatto, ma quadridimensionale e definito dalla \textbf{metrica di Minkowski}, non euclidea. La definizione di \(s\) è scelta in modo da ottenere equazioni simili alla meccanica classica.

Si definisce quadrivettore velocità $u^\alpha$ la derivata del quadrivettore spostamento $s^{\alpha}$ rispetto al \textbf{tempo proprio} $\tau$ della particella:

\[
{u}^{\alpha} = \dfrac{d{s}^{\alpha}}{d\tau} =
\begin{pmatrix}
 c\dfrac{dt}{d\tau},  \dfrac{dx}{d\tau},  \dfrac{dy}{d\tau},  \dfrac{dz}{d\tau}
\end{pmatrix}
\]

La variazione del \textbf{tempo proprio} $d\tau$ è legata alla variazione del tempo $dt$, osservata in un qualsiasi sistema di riferimento inerziale, dalla relazione:

\[
d\tau = \dfrac{1}{\gamma}dt \Leftrightarrow d\tau = \sqrt{1 - \dfrac{v^{2}}{c^{2}}}dt
\]

Il quadrivettore velocità, espresso in termini della velocità classica $\vec{v} = (v_x, v_y, v_z)$, è:

\[
{u}^{\alpha} = \dfrac{d{s}^{\alpha}}{d\tau} = \dfrac{1}{\sqrt{1 - \dfrac{v^{2}}{c^{2}}}}\dfrac{d{s}^{\alpha}}{dt} = \gamma\dfrac{d{s}^{\alpha}}{dt}
\]

Svolgendo l'operazione di derivata, il quadrivettore velocità si esprime come:

\[
{u}^{\alpha} = \gamma \begin{pmatrix}
 c\dfrac{dt}{dt},  \dfrac{dx}{dt},   \dfrac{dy}{dt},  \dfrac{dz}{dt}
\end{pmatrix} = \gamma \begin{pmatrix}
c, v_{x}, v_{y}, v_{z}
\end{pmatrix}
\]

In meccanica, il vettore velocità lungo i tre assi si esprime come:

\[
\vec{v} = \begin{pmatrix}
v_{x}, v_{y}, v_{z}
\end{pmatrix}
\]

Il quadrivettore velocità può essere espresso come:

\[{u}^{\alpha} = \gamma
\begin{pmatrix}
c,\vec{v}
\end{pmatrix}
\]

È possibile esprimere il quadrivettore in termini di quantità di moto ed energia. Infatti, per definizione di quantità di moto, risulta:

\[
\vec{p} = m_{0} \vec{v} \Leftrightarrow \vec{v} = \dfrac{1}{m_{0}}\vec{p}
\]

L'energia, invece, può essere espressa come:

\[
E = m_{0} c^{2} \Leftrightarrow c = \dfrac{E}{m_{0}c}
\]

Il quadrivettore velocità, espresso in termini energetici, è dato da:

\[{u}^{\alpha} = \gamma \begin{pmatrix}
 \dfrac{E}{m_{0}c}, \dfrac{1}{m_{0}}\vec{p}
\end{pmatrix}
\]

È possibile definire anche l'operazione di moltiplicazione tra quadrivettori. A tale scopo si adopera la notazione di Einstein secondo la quale ogni indice che compare all'interno di un fattore più di una volta viene sommato al variare di tutti i possibili valori che l'indice può assumere. Ad esempio, il prodotto vettoriale tra vettori  \(n\)-dimensionali, mediante la notazione di Einstein, è:

\[
\vec{x}\times\vec{y} = \sum_{i=1}^{n}{\left(\sum_{j=1}^{n}{\left(\sum_{k=0}^{n}{\left(\varepsilon_{ijk} x_{j} y_{k} \vec{e}_{i}\right)}\right)}\right)} = \varepsilon_{ijk} x^{j} y^{k} \vec{e}^{i}
\]

Dove \(\varepsilon_{ijk}\) è il simbolo di Levi-Civita e \(\vec{e}_{i}\) è la \(i\)-esimo vettore della base canonica di \(\mathbb{R}^{3}\).

Si applica tale notazione al fine di valutare il modulo del quadrivettore velocità:

\[
u_{\alpha}u^{\alpha} = \gamma^{2}\left( c^{2} - v^{2} \right) = c^{2}\gamma^{2}\left( 1 - \dfrac{v^{2}}{c^{2}} \right)
\]

Dopo aver raccolto \(c^{2}\), per definizione di \(\gamma\), è possibile scrivere:

\[
u_{\alpha}u^{\alpha} = c^{2}\gamma^{2}\gamma^{- 2} = c^{2}
\]

Il modulo quadro del quadrivettore velocità è la velocità della luce al quadrato, $c^{2}$, ed è \textbf{invariante rispetto alle trasformazioni di Lorentz}.

Si valuta il modulo quadro del quadrivettore quantità di moto:

\[
{p}^{\alpha} = m_{0}{u}^{\alpha} = m_{0}\gamma\begin{pmatrix}
c,\vec{v}
\end{pmatrix} = \begin{pmatrix}
m_{0}\gamma c,m_{0}\gamma\vec{v}
\end{pmatrix}
\]

Il termine \(m_{0}\gamma\) rappresenta la massa relativistica \(m\), dunque, la relazione \({p}^{\alpha}\) può essere scritta come:

\[
 {p}^{\alpha} = \begin{pmatrix}
m c,m\vec{v}
\end{pmatrix}
\]

Si è visto che \(E=mc^{2}\Leftrightarrow mc=E/c\) e \(\vec{p}=m\vec{v}\). Da queste relazione si evince che il quadrivettore quantità di moto è:

\[
{p}^{\alpha} = \begin{pmatrix}
 \dfrac{E}{c}, \vec{p}
\end{pmatrix}
\]

Si valuta il modulo del quadrivettore quantità di moto:

\[
p^{\alpha}p^{\alpha} = \dfrac{E^{2}}{c^{2}} - p^{2}
\]

dove \(E = \gamma m_{0}c^{2}\) e \(p = \gamma m_{0}v\), per cui:

\[
p_{\alpha}p^{\alpha} = \dfrac{E^{2}}{c^{2}} - p^{2} = \dfrac{\gamma^{2}m^{2}_{0}c^{4}}{c^{2}} - \gamma^{2}m^{2}_{0}v^{2} = \gamma^{2}m^{2}_{0}c^{2} - \gamma^{2}m^{2}_{0}v^{2}
\]

Raccogliendo \(\gamma^{2}m^{2}_{0}c^{2}\) si ha:

\[
p^{\alpha}p^{\alpha} = \gamma^{2}m^{2}_{0}c^{2}\left( 1 - \dfrac{v^{2}}{c^{2}} \right)
\]

Per definizione del fattore di Lorentz si ha:

\[
p_{\alpha}p^{\alpha} = m^{2}_{0}c^{2}
\]

Siccome \(E = m_{0}c^{2}\), la quantità di moto può essere espressa anche in termini energetici:

\[
p_{\alpha}p^{\alpha} = m_{0}E
\]

Infine, il vettore quantità di moto \(\vec{p}\) può essere espresso in termini energetici, infatti:

\[
E = m_{0}c^{2} \Leftrightarrow m_{0} = \dfrac{E}{c^{2}}
\]

Per cui:

\[
\vec{p} = m_{0}\vec{v} \Leftrightarrow \vec{p} = \dfrac{E}{c^{2}}\vec{v}
\]

Tale equazione è molto utile nel campo della medicina radiologica in cui le particelle raggiungono quasi la velocità della luce. In questo caso, la quantità di moto è data da:

\[
p = \dfrac{E}{c^{2}}c = \dfrac{E}{c}
\]

Questa relazione descrive la quantità di moto dei fotoni. 

\subsection{Legge di trasformazione dei quadrivettori}\label{legge-di-trasformazione-dei-quadrivettori}

Per comodità è possibile scrivere le trasformazioni di Lorentz in forma matriciale; così è più semplice determinare il modo in cui un quadrivettore in un sistema di riferimento \(K\) si trasforma in un quadrivettore nel sistema di riferimento \(K'\).

Si parte dalle trasformazioni di Lorentz:

\[ \begin{cases}
 t' = \gamma\left( - \dfrac{\beta}{c}x + t \right) \\
x' = \gamma(x - \beta ct) \\
y' = y \\
z' = z
\end{cases} 
\]

Si moltiplicano entrambi i membri della prima equazione, in modo da ottenere le quantità \(ct\) e \(ct'\) presenti nei quadrivettori dello spostamento:

\[
\begin{cases}
ct' = \gamma ct - \beta\gamma x \\
x' = - \gamma\beta ct + \gamma x \\
y' = y \\
z' = z
\end{cases}
\]

Si pone \({s'}^{\alpha}\) il quadrivettore spostamento nel sistema di riferimento \(K'\) e \({s}^{\alpha}\) il quadrivettore spostamento nel sistema di riferimento \(K\):

\[{s'}^{\alpha} =\begin{pmatrix}
ct' \\
x' \\
y' \\
z'
\end{pmatrix},\ \ {s}^{\alpha} = \begin{pmatrix}
ct \\
x \\
y \\
z
\end{pmatrix}
\]

La matrice di trasformazione di Lorentz è data da:

\[
\boldsymbol{\Lambda} = \begin{pmatrix}
\gamma & - \beta\gamma & 0 & 0 \\
- \beta\gamma & \gamma & 0 & 0 \\
0 & 0 & 1 & 0 \\
0 & 0 & 0 & 1
\end{pmatrix}
\]

Le trasformazioni di Lorentz in forma matriciale si scrivono come:

\[
\begin{pmatrix}
ct' \\
x' \\
y' \\
z'
\end{pmatrix} = \begin{pmatrix}
\gamma & - \beta\gamma & 0 & 0 \\
 - \beta\gamma & \gamma & 0 & 0 \\
0 & 0 & 1 & 0 \\
0 & 0 & 0 & 1
\end{pmatrix} \begin{pmatrix}
ct \\
x \\
y \\
z
\end{pmatrix}
\]

In forma compatta, si ha:

\[
{s'}^{\alpha} =\boldsymbol{\Lambda} {s}^{\alpha}
\]

Derivando rispetto al tempo si ottengono le trasformazioni della velocità per passare dal sistema \(K\) a \(K'\):

\[
\dfrac{d{{s}'}^{\alpha}}{dt} = \boldsymbol{\Lambda}\dfrac{d{s}^{\alpha}}{dt}
\]

La matrice di trasformazione di Lorentz è costante rispetto al tempo poiché la velocità relativa tra i due sistemi di riferimento è fissata, dunque, i termini \(\beta\) e \(\gamma\) sono costanti.

Si è visto che:

\[
\gamma\dfrac{d{s}^{\alpha}}{dt} = {u}^{\alpha} \Leftrightarrow \dfrac{d{s}^{\alpha}}{dt} = \dfrac{1}{\gamma}{u}^{\alpha} =\gamma\left( \dfrac{E}{m_{0}c},\dfrac{1}{m_{0}}\vec{p} \right)
\]

Per cui è possibile scrivere le equazioni di composizione della velocità in forma matriciale:

\[
\begin{pmatrix}
 \dfrac{E'}{m_{0}c} \\
 \dfrac{p_{x'}}{m_{0}} \\
 \dfrac{p_{y'}}{m_{0}} \\
 \dfrac{p_{z'}}{m_{0}}
\end{pmatrix} = \begin{pmatrix}
\gamma & - \beta\gamma & 0 & 0 \\
 - \beta\gamma & \gamma & 0 & 0 \\
0 & 0 & 1 & 0 \\
0 & 0 & 0 & 1
\end{pmatrix} \begin{pmatrix}
 \dfrac{E}{m_{0}c} \\
 \dfrac{p_{x}}{m_{0}} \\
 \dfrac{p_{y}}{m_{0}} \\
 \dfrac{p_{z}}{m_{0}}
\end{pmatrix} \Leftrightarrow  \begin{pmatrix}
 \dfrac{E'}{m_{0}c} \\
v_{x'} \\
v_{y'} \\
v_{z'}
\end{pmatrix} = \begin{pmatrix}
\gamma & - \beta\gamma & 0 & 0 \\
 - \beta\gamma & \gamma & 0 & 0 \\
0 & 0 & 1 & 0 \\
0 & 0 & 0 & 1
\end{pmatrix} \begin{pmatrix}
 \dfrac{E}{m_{0}c} \\
v_{x} \\
v_{y} \\
v_{z}
\end{pmatrix}
\]

Moltiplicando entrambi i membri per \(m_{0}\) si ottiene la trasformazione della quantità di moto:

\[
\begin{pmatrix}
 \dfrac{E'}{c} \\
p_{x'} \\
p_{y'} \\
p_{z'}
\end{pmatrix}  = \begin{pmatrix}
\gamma & - \beta\gamma & 0 & 0 \\
 - \beta\gamma & \gamma & 0 & 0 \\
0 & 0 & 1 & 0 \\
0 & 0 & 0 & 1
\end{pmatrix} \begin{pmatrix}
 \dfrac{E}{c} \\
p_{x} \\
p_{y} \\
p_{z}
\end{pmatrix}
\]

Le componenti dei quadrivettori spostamento, velocità e quantità di moto variano in base al sistema di riferimento, tuttavia, il loro quadrato è costante, dunque, è invariante rispetto alle trasformazioni di Lorentz.

\section{Urto anelastico}\label{urti-anelastico}

Un urto anelastico è un particolare tipo di urto in cui si conserva solamente la quantità di moto del sistema, mentre l'energia cinetica è non si conserva ma si trasforma in massa a riposo. Fondamentalmente le due particelle dopo l'urto si fondono in un'unica particella, come, ad esempio, due nuclei di idrogeno collidono, producendo un nucleo di elio.

Si pone l'origine del sistema di riferimento \(S\) nel centro di massa delle due particelle. Se le due particelle di masse identiche \(m\) hanno velocità uguali ed oppose, risulta che la somma delle quantità di moto deve essere nulla:

\[{\vec{p}}_{1} + {\vec{p}}_{2} = \gamma m_{0}{\vec{v}}_{1} + \gamma m_{0}{\vec{v}}_{2} = \vec{0}\]

Poiché:

\[m_{1} = m_{2} = m_{0},\ \ {\vec{v}}_{1} = - {\vec{v}}_{2}\]

\begin{figure}[ht]
\centering
\begin{tikzpicture}[>=Stealth, font=\small]

% Prima riga: particelle A e B
% Particella A
\fill[blue] (-3,1.5) circle (0.3);
\node[below=4pt] at (-3,1.4) {\textbf{A}};
\node[above=4pt] at (-3,1.6) {$m$};

% Freccia A
\draw[->, thick] (-2.6,1.5) -- (-1,1.5);
\node[above=4pt] at (-1.8,1.5) {$u'_A = u'_0$};

% Particella B
\fill[red] (3,1.5) circle (0.3);
\node[below=4pt] at (3,1.4) {\textbf{B}};
\node[above=4pt] at (3,1.6) {$m$};

% Freccia B
\draw[->, thick] (2.6,1.5) -- (1,1.5);
\node[above=4pt] at (1.8,1.5) {$u'_B = -u'_0$};

% Asse x' superiore
\draw[->, thick] (-4,0.7) -- (4,0.7) node[right] {$x'$};

% Linea separatrice
\draw[thick] (-4,0) -- (4,0);

% Seconda riga: particella C
\fill[blue] (-0.3,-1.5) circle (0.3);
\fill[red] (0.3,-1.5) circle (0.3);
\node[below=4pt] at (0,-1.6) {\textbf{C}};
\node[above=6pt] at (0,-1.4) {$M_0$};

% Velocità C
\node[right=8pt] at (0.6,-1.5) {$u'_C = 0$};

% Asse x' inferiore
\draw[->, thick] (-4,-2.3) -- (4,-2.3) node[right] {$x'$};

\end{tikzpicture}
\label{fig:2_UrtoAnelastico}
\caption{Urto anelastico tra due particelle}
\end{figure}

L'energia di una particella secondo la meccanica relativistica è data da:

\[E = \gamma m_{0}c^{2}\]

Siccome le velocità, in modulo, e le masse delle due particelle sono uguali, anche le energie coincidono, ovvero:

\[E_{1} = E_{2} = E\]

Dopo l'urto la particella risultate è in quiete nel centro di massa, per cui il fattore di Lorentz è unitario. L'energia della particella risultante, di massa a riposo \(M_{0}\), è, dunque:

\[E = M_{0}c^{2},\ \ \gamma = 1\]

Per la conservazione dell'energia, la somma delle energie delle particelle prima dell'urto deve essere uguale all'energia della particella dopo l'urto, \(E_{R}\):

\[E_{R} = E_{1} + E_{2} = 2E\]

Esplicitando le equazioni, si ha:

\[2\gamma m_{0}c^{2} = M_{0}c^{2}\]

Semplificando \(c^{2}\), si ha:

\[M_{0} = 2\gamma m_{0} = 2\dfrac{m_{0}}{\sqrt{1 - \dfrac{v^{2}}{c^{2}}}}\]

Poiché le due particelle, prima dell'urto, erano in moto, il fattore di Lorentz è maggiore dell'unità, di conseguenza, la massa \(M_{0}\) della particella risultante a valle dell'urto è maggiore della somma delle due masse a riposo iniziali. Questo fenomeno può essere spiegato ammettendo che l'energia cinetica posseduta dalle particelle prima dell'urto sia convertita in massa a riposo.

\section{Conservazione dell'energia relativistica}\label{conservazione-dellenergia-relativistica}

Il principio di conservazione dell'energia può essere verificato a partire a partire dal principio di conservazione del momento lineare e applicando le leggi di trasformazioni di Lorentz al quadrivettore energia-momento. Si considera un sistema di riferimento \(S'\) con origine non coincidente con il centro di massa del sistema. Supponendo che il moto avvenga solamente lungo una direzione, è possibile scrivere:

\[p_{x_{1}'} + p_{x_{2}'} = p_{x_{3}'}\]

Dove \(1\), \(2\) e \(3\) distinguono le tre particelle prima e dopo l'urto. Si applica la trasformazione di Lorentz per la quantità di moto:

\[
\begin{pmatrix}
 \dfrac{E'}{c} \\
p_{x'} \\
p_{y'} \\
p_{z'}
\end{pmatrix} = \begin{pmatrix}
\gamma & - \beta\gamma & 0 & 0 \\
 - \beta\gamma & \gamma & 0 & 0 \\
0 & 0 & 1 & 0 \\
0 & 0 & 0 & 1
\end{pmatrix} \begin{pmatrix}
 \dfrac{E}{c} \\
p_{x} \\
p_{y} \\
p_{z}
\end{pmatrix}
\]

Da cui si ricavano le equazioni:

\[
\begin{cases}
 \dfrac{E'}{c} = \gamma\dfrac{E}{c} - \beta\gamma p_{x} \\
 p_{x'} = - \beta\gamma\dfrac{E}{c} + p_{x}
\end{cases}
\]

Utilizzando la seconda equazione è possibile riscrivere il bilancio della quantità di modo:

\[p_{x_{1}'} + p_{x_{2}'} = p_{x_{3}'} \Leftrightarrow - \beta\gamma\dfrac{E_{1}}{c} + \gamma p_{x_{1}} - \beta\gamma\dfrac{E_{2}}{c} + \gamma p_{x_{2}} = - \beta\gamma\dfrac{E_{3}}{c} + \gamma p_{x_{3}}\]

Se la particella risultante è in quiete nel sistema \(S\), la sua quantità di moto \(p_{x_{3}}\) è nulla. Raccogliendo la primo membro si ha:

\[- \beta\gamma\dfrac{E_{1}}{c} - \beta\gamma\dfrac{E_{2}}{c} + \gamma\left( p_{x_{1}} + p_{x_{2}} \right) = - \beta\gamma\dfrac{E_{3}}{c}\]

Per il principio di identità di polinomi deve risultare che:

\[\begin{cases}
p_{x_{1}} + p_{x_{2}} = 0 \\
 - \beta\gamma\dfrac{E_{1}}{c} - \beta\gamma\dfrac{E_{2}}{c} = - \beta\gamma\dfrac{E_{3}}{c}
\end{cases} 
\]

Semplificando i termini comuni nella seconda equazione, si ha:

\[E_{1} + E_{2} = E_{3}\]

Da cui discende la conservazione dell'energia.

\section{Esempio carica in moto in campo elettrico}\label{esempio-carica-in-moto-in-campo-elettrico}

Si considera una particella carica in un campo elettrico \({\vec{E}}_{ext}\) uniforme in una regione dello spazio e costante nel tempo. La carica è soggetta alla forza di Lorentz:

\[\vec{F} = q{\vec{E}}_{ext}\]

Per la meccanica newtoniana, risulta:

\[\vec{F} = \dfrac{d\vec{p}}{dt} = \dot{\vec{p}}\]

Per cui la legge di Lorentz si scrive anche:

\[\dot{\vec{p}} = q{\vec{E}}_{ext}\]

\begin{figure}[ht]
\centering
\begin{tikzpicture}[>=Stealth, font=\large]

% Piastre
\draw[thick] (-3,-4) rectangle (-2,4);
\draw[thick] (3,-4) rectangle (2,4);

% Cariche sulle piastre
\foreach \y in {-3.5,-2.5,-1.5,-0.5,0.5,1.5,2.5,3.5} {
    \node at (-2.5,\y) {\textbf{+}};
    \node at (2.5,\y) {\textbf{-}};
}

% Linee di campo elettrico
\foreach \y in {-3.5,-2.5,-1.5,-0.5,0.5,1.5,2.5,3.5} {
    \draw[->, red, thick] (-2,\y) -- (2,\y);
}

% Etichetta E
\node[red, above] at (0,4.2) {\textbf{E}};
\draw[->, red, thick] (0,4.1) -- (0,3.7);

% Delta V
\node[above] at (0,5) {$\Delta V$};
\draw[decorate, decoration={brace, amplitude=10pt}] (-3,4.6) -- (3,4.6);

% Particella positiva
\draw[thick] (-1.5,0) circle (0.4);
\node at (-1.5,0) {\textbf{+}};
\draw[->, thick] (-1.1,0) -- (-0.3,0);
\node[above] at (-0.85,-0.5) {$\vec{v}$};

% Distanza d
\draw[<->, thick] (-1.8,-4.5) -- (1.8,-4.5);
\node[below] at (0,-4.5) {$d$};

\end{tikzpicture}

\label{fig:2_CaricaCampo}
\caption{Particella immersa in un campo elettrico costante e uniforme}
\end{figure}

Si suppone che il moto avvenga solamente lungo l'asse \(x\):

\[{\dot{p}}_{x} = qE_{ext,x}\]

Si integra tale equazione rispetto al tempo. In ipotesi di velocità iniziale nulla, si ha:

\[p_{x} = \int_{0}^{t}{qE_{ext,x}dt'} = qE_{ext,x}t\]

Nella teoria classica, la velocità della particella cresce linearmente nel tempo. Infatti, dividendo l'equazione per la quantità di moto appena ottenuta per la massa, risulta:

\[v_{x} = \dfrac{q}{m}E_{ext,x}t\]

Secondo la meccanica relativistica, la velocità non può aumentare indefinitamente, in quando la velocità della luce è un limite invalicabile. A tale scopo, si studia lo stesso fenomeno mediante un approccio relativistico. Il modulo quadro del quadrivettore quantità di moto è dato da:

\[p_{\alpha}p^{\alpha} = \dfrac{E^{2}}{c^{2}} - p_{x}^{2}\]

Dove \(p_{\alpha}p^{\alpha} = m^{2}_{0}c^{2}\), per cui si ha:

\[m^{2}_{0}c^{2} = \dfrac{E^{2}}{c^{2}} - p_{x}^{2}\]

Da questa equazione, si ricava l'energia della particella \(E\):

\[E = c\sqrt{m^{2}_{0}c^{2} + p^{2}}\]

La quantità di moto è data da \(p_{x} = qE_{ext,x}t\), per cui si ottiene:

\[E = c\sqrt{m^{2}_{0}c^{2} + q^{2}E_{ext,x}^{2}t^{2}}\]

La velocità della particella è legata all'energia dalla relazione:

\[\vec{p} = \dfrac{E}{c^{2}}\vec{v}\]

Proiettando questa equazione sull'asse del moto, si ha:

\[p_{x} = \dfrac{E}{c^{2}}v_{x} \Leftrightarrow v_{x} = \dfrac{c^{2}}{E}p_{x}\]

Sostituendo le relazioni per \(p_{x}\) e l'energia, si ricava

\[v_{x} = \dfrac{c^{2}}{E}p_{x} = \dfrac{c^{2}qE_{ext,x}t}{c\sqrt{m^{2}_{0}c^{2} + q^{2}E_{ext,x}^{2}t^{2}}} = \dfrac{cqE_{ext,x}t}{\sqrt{m^{2}_{0}c^{2} + q^{2}E_{ext,x}^{2}t^{2}}}\]

Per \(t \rightarrow \infty\) la velocità tende a quella della luce:

\[\lim_{t \rightarrow \infty}v_{x} = \lim_{t \rightarrow \infty}\dfrac{cqE_{ext,x}t}{\sqrt{m^{2}_{0}c^{2} + q^{2}E_{ext,x}^{2}t^{2}}} \rightarrow \dfrac{cqE_{ext,x}t}{qE_{ext,x}t} = c\]

Tale equazione è coerente con il principio di limite teorico invalicabile per la velocità della luce. Infatti, affinché una particella di massa a riposo \(m_{0}\) raggiunga la velocità della luce deve essere accelerata per un tempo indefinito.

Per piccoli intervalli temporali, è valida l'approssimazione classica in cui la velocità è proporzionale al tempo \(t\) mediante una costante di proporzionalità:

\[v_{x} \simeq \dfrac{q}{m}E_{ext,x}t,\ \ t \ll \dfrac{m_{0}c}{qE_{ext,x}}\]

\begin{figure}[ht]
\centering
\begin{tikzpicture}
\begin{axis}[
    width=14cm,
    height=8cm,
    xlabel={tempo [s]},
    ylabel={velocità},
    xmin=0, xmax=0.01,
    ymin=0, ymax=4e8,
    xtick={0,0.001,0.002,0.003,0.004,0.005,0.006,0.007,0.008,0.009,0.01},
    yticklabel style={/pgf/number format/fixed},
    scaled y ticks=false,
    scaled x ticks=false,
    y label style={at={(axis description cs:-0.2,.5)},rotate=180},
    x label style={at={(axis description cs:0.5,-0.1)}},
]

% Parametri
\def\c{3e8} % velocità della luce
\def\a{1.5e11} % accelerazione costante (esempio)

% Curva relativistica (approssimazione)
\addplot[blue, thick, domain=0:0.01, samples=200] 
    {\c*(1 - exp(-\a*x/\c))}; % modello esponenziale per saturazione
\addlegendentry{Relativistica}

% Curva classica
\addplot[green, dashed, domain=0:0.0025, samples=100] 
    {\a*x};
\node[above, rotate=66] at (axis cs:0.002,2.8e8) {approssimazione classica};

% Linea limite c
\addplot[red, dash dot, domain=0:0.01] {\c};
\node[above] at (axis cs:0.008,3e8) {limite c};

\end{axis}
\end{tikzpicture}

\label{fig:2_velocitàTempo}
\caption{Andamento della velocità in funzione del tempo per particella accelerata}
\end{figure}

Nel limite classico, il concetto di massa inerziale coincide con quello di massa a riposo relativistico.
\begin{center}
\vfill
    \chapter{Elettromagnetismo}
    \label{blx:refsection\therefsection}
\vfill

\minitoc
\newpage
\end{center}
\justify


\section{Cenni di elettromagnetismo}\label{cenni-di-elettromagnetismo}

Le \textbf{equazioni di Maxwell}, formalizzate verso la metà dell'Ottocento dal fisico James Clerk Maxwell, non solo sintetizzarono le leggi dell'elettricità e del magnetismo note all'epoca (come le leggi di Faraday e Ampère), ma predissero l'esistenza delle onde elettromagnetiche e dimostrarono che la luce stessa è un'onda elettromagnetica \cite{landau1975campi, purcell1985elettromagnetismo, feynman1964vol2, halliday2004fisica2}. Queste equazioni sono ancora valide in molte applicazioni pratiche moderne, con poche modifiche rispetto alla formulazione originale.

Il concetto di \textbf{spazio-tempo} (nella teoria \textbf{Relatività Ristretta}) fu introdotto da Albert Einstein nel 1905 proprio per risolvere il conflitto tra la meccanica newtoniana (che era invariante solo rispetto alle trasformazioni Galileiane) e le equazioni di Maxwell (che sono invece invarianti rispetto alle trasformazioni di Lorentz).

Il campo elettromagnetico è descritto in modo formale dai due campi vettoriali: il \textbf{campo elettrico} ($\vec{E}$) e il \textbf{campo di induzione magnetica} ($\vec{B}$). Questi due campi non sono entità separate e indipendenti, ma sono le manifestazioni di un unico tensore elettromagnetico.

Il campo magnetico ($\vec{B}$) è essenzialmente una manifestazione relativistica del campo elettrico ($\vec{E}$). Ossia, il campo magnetico osservato in un sistema di riferimento $K$ è in parte il risultato della trasformazione del campo elettrico e delle cariche in moto osservate in un altro sistema $K'$

Quando una carica $q$ si muove in una regione di spazio, essa è soggetta alla Forza di Lorentz ($\vec{F} = q(\vec{E} + \vec{v} \times \vec{B})$) \cite{landau1975campi, purcell1985elettromagnetismo, feynman1964vol2, halliday2004fisica2}. Se si considera un sistema di riferimento $K'$ solidale con la carica ($\vec{v}' = 0$), su di essa agisce solo la forza elettrica ($\vec{F}' = q\vec{E}'$). Pertanto, in tale sistema, qualsiasi forza percepita dalla carica deve essere attribuita al solo campo elettrico trasformato $\vec{E}'$, mentre il termine magnetico è nullo.

\section{Equazioni di Maxwell nel vuoto in forma locale}\label{equazioni-di-maxwell-nel-vuoto-in-forma-locale}

Sebbene sia possibile descrivere l'interazione elettromagnetica mediante l'uso del solo campo elettrico, per semplicità si ammette l'esistenza anche del campo magnetico. In quest'ottica, il campo elettrico e il campo di induzione magnetica sono due manifestazioni dello stesso campo: il campo elettromagnetico, descritto, in assenza di mezzo materiale (nel vuoto), dalle equazioni differenziali in forma locale:

\[\begin{cases}
 \vec{\nabla} \cdot \vec{E} = \dfrac{\rho}{\varepsilon_{0}} & \text{(Legge di Gauss per l'Elettricità)} \\
\vec{\nabla} \cdot \vec{B} = 0 & \text{(Legge di Gauss per il Magnetismo)} \\
 \vec{\nabla} \times \vec{E} = - \dfrac{\partial\vec{B}}{\partial t} & \text{(Legge di Faraday-Neumann-Lenz)} \\
 \vec{\nabla} \times \vec{B} = \mu_{0}\left( \vec{J} + \varepsilon_{0}\dfrac{\partial\vec{E}}{\partial t} \right) & \text{(Teorema di Ampère-Maxwell)}
\end{cases}
\]

La prima equazione è la legge di Gauss per il campo elettrico e afferma che la divergenza del campo elettrico è proporzionale alla densità di carica $\rho$ presente nel volume di riferimento, rapportata alla costante dielettrica nel vuoto \(\varepsilon_{0}\).

La seconda equazione è la legge di Gauss per il magnetismo, secondo cui la divergenza del campo di induzione magnetica $\vec{B}$ è nulla. Questa legge implica che le linee di campo magnetico non hanno sorgenti o pozzi (non esistono monopoli magnetici) e che il campo $\vec{B}$ è solenoidale. Dal punto di vista fisico, questa legge implica che le linee di campo magnetico entrano ed escono dal volumetto \(d\Omega\) su cui si calcola la divergenza, che, di conseguenza, è nulla \figurename~\ref{fig:CampSolenoidale}.

La terza equazione è la legge di Faraday-Neumann-Lenz e stabilisce che un campo di induzione magnetica variabile nel tempo ($\partial\vec{B}/\partial t$) induce un campo elettrico rotazionale ($\vec{\nabla} \times \vec{E}$). In altre parole, la variazione del flusso del campo magnetico attraverso un circuito elettrico genera una forza elettromotrice indotta e, di conseguenza, una corrente elettrica indotta

\begin{figure}[ht]
\centering
\includegraphics[width=1.90691in,height=1.76389in,alt={P1154\#yIS1}]{media/3_Elettromagnetismo/image18.pdf}\caption{Flusso campo magnetico attraverso superficie chiusa}\label{fig:CampSolenoidale}
\end{figure}

La quarta equazione è il teorema di Ampère-Maxwell, secondo cui il rotore del campo $\vec{B}$ è determinato dalla densità di corrente di conduzione ($\vec{J}$) più la corrente di spostamento ($\varepsilon_{0}\partial\vec{E}/\partial t$). La corrente di spostamento fu introdotta da Maxwell per garantire la conservazione della carica. Secondo questo teorema, dal punto di vista globale, un campo magnetico indotto intorno a un circuito chiuso qualsiasi è proporzionale alla corrente elettrica concatenata \(\mu_{0}\vec{J}\) al circuito più la corrente di spostamento (\(\mu_{0}\varepsilon_{0}\partial\vec{E}/\partial t\)) attraverso la superficie chiusa.

\subsection{Equazione di continuità della carica}
Dalle equazioni di Maxwell è possibile ricavare la legge di continuità della carica. A tale scopo si applica la divergenza all'equazione di Ampere-Maxwell:

\[
\vec{\nabla} \cdot \vec{\nabla} \times \vec{B} = \mu_{0}\vec{\nabla} \cdot \vec{J} + \mu_{0}\varepsilon_{0}\vec{\nabla} \cdot \dfrac{\partial\vec{E}}{\partial t}
\]

Il primo membro è la divergenza di un rotore che, per qualsiasi campo vettoriale derivabile due volte, è sempre nulla, ovvero \(\vec{\nabla} \cdot \vec{\nabla} \times \vec{B} = 0\). Si ottiene:

\[\mu_{0}\vec{\nabla} \cdot \vec{J} + \mu_{0}\varepsilon_{0}\vec{\nabla} \cdot \dfrac{\partial\vec{E}}{\partial t} = 0\]

È possibile scambiare il simbolo di derivata temporale con quello di divergenza:

\[\mu_{0}\vec{\nabla} \cdot \vec{J} + \mu_{0}\varepsilon_{0}\dfrac{\partial}{\partial t}\left( \vec{\nabla} \cdot \vec{E} \right) = 0\]

La divergenza del campo elettrico è data dalla prima equazione di Maxwell. Semplificando anche \(\mu_{0}\), si ottiene:

\[\vec{\nabla} \cdot \vec{J} + \varepsilon_{0}\dfrac{\partial}{\partial t}\left( \dfrac{\rho}{\varepsilon_{0}} \right) = 0\]

Dunque, si è ottenuta la legge di conservazione della carica:

\[\vec{\nabla} \cdot \vec{J} + \dfrac{\partial\rho}{\partial t} = 0\]

Tale equazione afferma che, se la carica in un volume varia nel tempo, deve esistere una corrente in ingresso o in uscita dal volume che sostenga tale variazione di carica.

\subsection{Risoluzione delle equazioni di Maxwell omogenee}\label{risoluzione-delle-equazioni-maxwell-omogenee}

Si considera il caso di assenza di sorgenti, ovvero, si vuole risolvere le equazioni di Maxwell lontano dalle sorgenti di campo \(\vec{J}\) e \(\rho\)

\[\begin{cases}
 \vec{\nabla} \cdot \vec{E} = \dfrac{\rho}{\varepsilon_{0}} \\
\vec{\nabla} \cdot \vec{B} = 0 \\
\vec{\nabla} \times \vec{E} = - \dfrac{\partial\vec{B}}{\partial t}  \\
\vec{\nabla} \times \vec{B} = \mu_{0}\left( \vec{J} + \varepsilon_{0}\dfrac{\partial\vec{E}}{\partial t} \right)
\end{cases}
\]

Si vuole risolvere il sistema di equazioni in modo da determinare il campo induzione magnetica \(\vec{B}\). A tale scopo si applica il rotore all'equazione di Ampere-Maxwell:

\[\vec{\nabla} \times \vec{\nabla} \times \vec{B} = \mu_{0}\varepsilon_{0}\vec{\nabla} \times \dfrac{\partial\vec{E}}{\partial t}\]

Il rotore del rotore può essere scritto come:

\[\vec{\nabla} \times \vec{\nabla} \times = \vec{\nabla}\left( \vec{\nabla} \cdot \  \right) - \nabla^{2}\ \]

Dove \(\nabla^{2}\) è l'operatore laplaciano che, scritto in coordinate cartesiane, è dato da:

\[\nabla^{2} = \Delta = \dfrac{\partial^{2}}{\partial x^{2}} + \dfrac{\partial^{2}}{\partial y^{2}} + \dfrac{\partial^{2}}{\partial z^{2}}\]

L'equazione di Ampere-Maxwell si può scrivere, invertendo l'operatore derivata temporale col rotore, come:

\[\vec{\nabla} \times \vec{\nabla} \times \vec{B} = \mu_{0}\varepsilon_{0}\vec{\nabla} \times \dfrac{\partial\vec{E}}{\partial t} \Leftrightarrow \vec{\nabla}\left( \vec{\nabla} \cdot \ \vec{B} \right) - \nabla^{2}\vec{B} = \mu_{0}\varepsilon_{0}\dfrac{\partial}{\partial t}\left( \vec{\nabla} \times \vec{E} \right)\]

La divergenza del campo induzione magnetica è nulla, inoltre, sostituendo la terza equazione di Maxwell si ha:

\[- \nabla^{2}\vec{B} = \mu_{0}\varepsilon_{0}\dfrac{\partial}{\partial t}\left( - \dfrac{\partial\vec{B}}{\partial t} \right) \Leftrightarrow - \nabla^{2}\vec{B} = - \ \mu_{0}\varepsilon_{0}\dfrac{\partial^{2}\vec{B}}{\partial t^{2}}\]

L'equazione:

\[\nabla^{2}\vec{B} - \ \mu_{0}\varepsilon_{0}\dfrac{\partial^{2}\vec{B}}{\partial t^{2}} = 0\]

È detta equazione delle onde o di d'Alembert. Si definisce operatore di d'Alembert come:

\[\square = \nabla^{2} - \dfrac{1}{c^{2}}\dfrac{\partial^{2}}{\partial t^{2}}\]

Dove \(c\) è la velocità di propagazione dell'onda. In questo caso, \(c\) coincide con la velocità della luce nel vuoto. Dalle equazioni di Maxwell discende che:

\[c = \dfrac{1}{\sqrt{\mu_{0}\varepsilon_{0}}} \simeq 3.00 \cdot 10^{8}\ m/s\]

\subsection{Risoluzione delle equazioni di Maxwell in presenza di sorgenti}\label{risoluzione-delle-equazioni-di-maxwell}

Si vuole risolvere le equazioni di Maxwell in prossimità delle sorgenti \(\vec{J}\) e \(\rho\). In questo modo è possibile ricostruire l'intero campo elettromagnetico generato dalle sorgenti.

\[\begin{cases}
 \vec{\nabla} \cdot \vec{E} = \dfrac{\rho}{\varepsilon_{0}} \\
\vec{\nabla} \cdot \vec{B} = 0 \\
\vec{\nabla} \times \vec{E} = - \dfrac{\partial\vec{B}}{\partial t}  \\
\vec{\nabla} \times \vec{B} = \mu_{0}\left( \vec{J} + \varepsilon_{0}\dfrac{\partial\vec{E}}{\partial t} \right)
\end{cases}
\]

Dall'equazione di Gauss per il magnetismo, risulta che il campo induzione magnetica è solenoidale, quindi, è possibile definire un potenziale vettore \(\vec{A}\) tale che:

\[\vec{\nabla} \times \vec{A} = \vec{B}\]

Si sostituisce tale definizione nella terza equazione di Maxwell:

\[\vec{\nabla} \times \vec{E} = - \dfrac{\partial\vec{B}}{\partial t} = - \dfrac{\partial}{\partial t}\left( \vec{\nabla} \times \vec{A} \right)\]

L'operatore \(\vec{\nabla}\) è indipendente dalla derivata temporale, dunque, è lecita la loro inversione:

\[\vec{\nabla} \times \vec{E} = - \vec{\nabla} \times \dfrac{\partial\vec{A}}{\partial t}\]

Portando tutto al primo membro e applicando la linearità dell'operatore rotore, è possibile scrivere:

\[\vec{\nabla} \times \left( \vec{E} + \dfrac{\partial\vec{A}}{\partial t} \right) = \vec{0}\]

Il campo \(\vec{E} + \partial\vec{A}/\partial t\) è irrotazionale, per cui, è possibile definire un potenziale scalare \(\phi\) tale che:

\[\vec{E} + \dfrac{\partial\vec{A}}{\partial t} = - \vec{\nabla}\phi\]

Il potenziale scalare $\phi$ è generalmente definito con un segno negativo. Nell'elettrostatica, il potenziale scalare concide con la tensione $\vec{E} = -\vec{\nabla}V$.

In presenza di un campo magnetostatico, la derivata temporale di \(\vec{B}\) è nulla, per cui il potenziale scalare coincide con il potenziale elettrico:

\[\vec{E} = - \vec{\nabla}\phi\]

La forza elettrica ($\vec{F}$) deve essere opposta al gradiente del potenziale scalare ($\phi$) perché il potenziale $\phi$ rappresenta l'energia potenziale elettrica per unità di carica (ossia, la tensione o voltaggio).In fisica, una forza conservativa (come la forza elettrica) tende sempre a muovere un oggetto nella direzione in cui l'energia potenziale diminuisce più rapidamente. Il gradiente ($\vec{\nabla}\phi$) punta, per definizione, nella direzione in cui il potenziale aumenta più rapidamente. Per questo motivo, la relazione fondamentale tra campo elettrico ($\vec{E}$) e potenziale elettrico ($\phi$) deve includere un segno negativo

 potenziali vettore $\vec{A}$ e scalare $\phi$ non sono univocamente determinati. Per tale motivo, si adotta una condizione di Gauge per disaccoppiare le equazioni differenziali. La Gauge di Lorentz, data da:

\[\vec{\nabla} \cdot \vec{A} + \mu_0\varepsilon_{0}\dfrac{\partial\phi}{\partial t} = 0\]

è invariante quando si passa da un sistema di riferimento inerziale a un altro tramite le trasformazioni di Lorentz,il che è un requisito fondamentale della Relatività Ristretta. Per tale motivo si parla di condizione covariante in relatività.

Sotto la condizione di Gauge di Lorentz, le Equazioni di Maxwell si riducono a due equazioni d'onda disaccoppiate per $\vec{A}$ e $\phi$ con termini sorgente. La soluzione di queste equazioni porta ai potenziali ritardati.

Si dimostra che, per una sorgente puntiforme, il potenziale vettore, nel dominio della frequenza, è dato da:

\[\vec{A} = \dfrac{\mu_{0}}{4\pi}\dfrac{\vec{J}}{\left| \vec{r} - {\vec{r}}_{0} \right|}e^{- jk\left| \vec{r} - {\vec{r}}_{0} \right|}\]

Dove \({\vec{r}}_{0}\) è la posizione in cui sono posizionate le sorgenti, mentre \(\vec{r}\) è il punto nello spazio in cui si valuta il campo. La quantità:

\[g\left( \vec{r} \right) = \dfrac{1}{4\pi}\dfrac{1}{\left| \vec{r} \right|}e^{- jk\left| \vec{r} \right|}\]

È la funzione di Green o risposta impulsiva del mezzo, lineare e isotropo, visto come un sistema ingresso-uscita. Si definisce il numero d'onda (o costante di propagazione nel vuoto) come:

\[k = \omega\sqrt{\mu_{0}\varepsilon_{0}}=\dfrac{\omega}{c}\]

\subsection{Equazioni di Maxwell nel vuoto in forma globale}\label{equazioni-di-maxwell-nel-vuoto-in-forma-globale}

Le equazioni di Maxwell in forma locale non possono essere utilizzate in presenza di brusche variazioni della superficie dei volumi sui quali calcolare i campi. In questo caso si ricorre alla forma globale o integrale:

\[\begin{cases}
 \oiint_{S}{\vec{E} \cdot d\vec{S}\ } = \dfrac{Q}{\varepsilon_{0}} \\
 \oiint_{S}{\vec{B} \cdot d\vec{S}\ } = 0  \\
 \oint_{\partial S}{\vec{E} \cdot d\vec{s}} = - \dfrac{\partial}{\partial t}\int_{\Sigma}{\vec{B} \cdot d\vec{\Sigma}}  \\
 \oint_{\partial S}{\vec{B} \cdot d\vec{s}} = \mu_{0}I + \mu_{0}\varepsilon_{0}\dfrac{\partial}{\partial t}\int_{\Sigma}{\vec{E} \cdot d\vec{\Sigma}}
\end{cases}
\]

Dove \(S\) è una superficie all'interno della quale si vuole valutare il campo, mentre \(\Sigma\) è una superficie aperta avente per contorno \(\partial S\). \(Q\) è la quantità di carica contenuta in \(S\) e \(I\) la corrente che attraversa \(\Sigma\).

\section{Equazioni di Maxwell nel mezzo}\label{equazioni-di-maxwell-nel-mezzo}

In presenza di un mezzo materiale si introducono i vettori di induzione elettrica \(\vec{D}\) e di campo magnetico \(\vec{H}\). Il vettore \(\vec{B}\) è detto vettore di induzione magnetica, mentre \(\vec{E}\) rappresenta il campo elettrico.

I campi di induzione descrivono il comportamento del materiale a seguito dell'applicazione dei rispettivi campi. Con l'introduzione dei campi \(\vec{E}\), \(\vec{D}\), \(\vec{H}\) e \(\vec{B}\), le equazioni di Maxwell in forma locale si scrivono come:

\[
\begin{cases}
\vec{\nabla} \cdot \vec{D} = \rho_{lib} \\
\vec{\nabla} \cdot \vec{B} = 0 \\
\vec{\nabla} \times \vec{E} = - \dfrac{\partial \vec{B}}{\partial t} \\
\vec{\nabla} \times \vec{H} = \vec{J}_{lib} + \dfrac{\partial \vec{D}}{\partial t}
\end{cases}
\]

Dove \(\rho_{lib}\) e \(\vec{J}_{lib}\) sono le sorgenti libere ovvero le cariche e correnti che possono muoversi liberamente all’interno o all’esterno di un materiale, non vincolate alla struttura atomica o molecolare del mezzo. Queste sorgenti possono essere controllate direttamente, ad esempio, applicando una tensione o facendo passare una corrente in un conduttore. In contrapposizione, ci sono le sorgenti vincolate (o di legame), che derivano dal comportamento microscopico delle molecole e atomi del materiale. I vettori di polarizzazione e di magnetizzazione dipendono da tali sorgenti.

Il vettore di induzione elettrica è definito come:

\[
\vec{D} = \varepsilon_{0}\vec{E} + \vec{P}
\]

dove \(\vec{P}\) è il vettore di polarizzazione; esso descrive come i dipoli elettrici contenuti nel materiale si orientano sotto l'effetto del campo elettrico.

L'induzione magnetica è invece definita come:

\[
\vec{B} = \mu_{0}(\vec{H} + \vec{M})
\]

dove \(\vec{M}\) è il vettore di magnetizzazione e descrive come i dipoli magnetici nel materialesi orientano sotto l’azione del campo magnetico applicato H \(\vec{H}\).

Per i mezzi lineari e isotropi esistono due costanti, dette suscettibilità elettrica \(\chi_{e}\) e suscettibilità magnetica \(\chi_{m}\), che rappresentano i coefficienti di proporzionalità, rispettivamente, tra i campi di polarizzazione e magnetizzazione e i relativi campi applicati:

\[
\vec{P} = \varepsilon_{0}\chi_{e}\vec{E}, \qquad \vec{M} = \chi_{m}\vec{H}
\]

Sostituendo il vettore di polarizzazione nella definizione del vettore di induzione elettrica si ottiene:

\[
\vec{D} = \varepsilon_{0}\vec{E} + \vec{P} = \varepsilon_{0}\vec{E} + \varepsilon_{0}\chi_{e}\vec{E} = \varepsilon_{0}\left( 1 + \chi_{e} \right)\vec{E}
\]

Si definisce la costante dielettrica del mezzo, per un materiale lineare e isotropo, come:

\[
\varepsilon = \varepsilon_{0}\left( 1 + \chi_{e} \right)
\]

Con questa definizione si ha:

\[
\vec{D} = \varepsilon\vec{E}
\]

Analogamente, per il campo di induzione magnetica:

\[
\vec{B} = \mu_{0}\vec{H} + \mu_{0}\vec{M} = \mu_{0}\vec{H} + \mu_{0}\chi_{m}\vec{H} = \mu_{0}\left( 1 + \chi_{m} \right)\vec{H}
\]

Si definisce la permeabilità magnetica del mezzo, per un materiale lineare e isotropo, come:

\[
\mu = \mu_{0}\left( 1 + \chi_{m} \right)
\]

Il campo di induzione magnetica si scrive quindi come:

\[
\vec{B} = \mu\vec{H}
\]

\begin{figure}[ht]
\centering
\includegraphics[width=3.19044in,height=2.4711in,alt={P1240\#yIS1}]{media/3_Elettromagnetismo/image19.pdf}
\caption{Dipolo elettrico e magnetico.}
\end{figure}

Infine, la densità di corrente totale si suddivide in una componente libera, associata alle cariche mobili, e in una componente vincolata, dovuta alla polarizzazione e alla magnetizzazione del materiale. Si scrive dunque:

\[
\vec{J} = \vec{J}_{lib} + \vec{J}_{vinc}
\]

La densità di corrente vincolata (o di legame) è data da:

\[
\vec{J}_{vinc} = \dfrac{\partial \vec{P}}{\partial t} + \vec{\nabla} \times \vec{M}
\]

Per un mezzo lineare e isotropo, sostituendo le relazioni costitutive si ottiene:

\[
\vec{J}_{vinc} = \varepsilon_0 \chi_e \dfrac{\partial \vec{E}}{\partial t} + \left(\vec{\nabla} \times \left(\chi_{m} \vec{H}\right)\right)
\]

Se il mezzo è omogeneo, $\chi_m$ è una costante spaziale. In questa ipotesi, la densità di corrente vincolata si scrive come:

\[
\vec{J}_{vinc} = \varepsilon_0 \chi_e \dfrac{\partial \vec{E}}{\partial t} + \chi_{m}\left(\vec{\nabla} \times \vec{H}\right)
\]

La densità di corrente totale \(\vec{J}\) dipende dunque sia dal campo elettrico sia dal campo magnetico. Nella quarta equazione di Maxwell, tuttavia, la densità di corrente si riferisce solo alla parte libera:

\[
\vec{\nabla} \times \vec{H} = \vec{J}_{lib} + \dfrac{\partial \vec{D}}{\partial t}
\]

In definitiva, i campi fondamentali che descrivono le forze elettromagnetiche sono \(\vec{E}\) e \(\vec{B}\), mentre \(\vec{D}\) e \(\vec{H}\) sono campi ausiliari introdotti per semplificare le equazioni di Maxwell in presenza di un mezzo materiale (come un dielettrico o un magnete), inglobando negli stessi gli effetti microscopici di polarizzazione (\(\vec{P}\)) e magnetizzazione (\(\vec{M}\)).

\section{Forza di Lorentz}\label{forza-di-lorentz}

I campi elettrico \(\vec{E}\) e di induzione magnetica \(\vec{B}\) esercitano una forza su una particella di carica \(q\) che si muove con velocità \(\vec{v}\) in una regione di spazio contenente tali campi. L’espressione della forza totale è:

\[
\vec{F} = q\vec{E} + q\vec{v} \times \vec{B}
\]

Questa interazione è nota come \textbf{forza di Lorentz}. Essa si compone di due contributi distinti:

\begin{itemize}
    \item \textbf{Forza elettrica}: \(\vec{F}_{E} = q\vec{E}\).
    Questa forza è parallela o antiparallela al campo elettrico, a seconda del segno della carica \(q\), ed è indipendente dalla velocità \(\vec{v}\) della particella.
    
    \item \textbf{Forza magnetica}: \(\vec{F}_{B} = q(\vec{v} \times \vec{B})\).
    Questa forza è sempre perpendicolare sia alla velocità \(\vec{v}\) sia al campo di induzione magnetica \(\vec{B}\). Poiché la direzione della forza è perpendicolare allo spostamento della particella, la forza magnetica \textit{non compie lavoro} sulla carica.
\end{itemize}

\section{Parallelismo tra campo elettrico e magnetico}\label{parallelismo-tra-campo-magnetico-ed-elettrico}

Le equazioni di Maxwell possono essere rese formalmente simmetriche introducendo le \textit{cariche magnetiche} e le \textit{correnti magnetiche}, concetti ipotetici non osservati sperimentalmente ma utili per evidenziare l’analogia tra fenomeni elettrici e magnetici.

Si indichi con \(U\) l’energia potenziale di un dipolo elettrico \(\vec{d}\) o magnetico \(\vec{\mu}\), immerso rispettivamente in un campo elettrico o magnetico.  
Conoscendo \(U\), è possibile determinare la forza \(\vec{F}\) e il momento torcente \(\vec{N}\) agente sul dipolo.

Per il campo elettrico:

\[
\begin{cases}
 U = - \vec{d} \cdot \vec{E} \\
 \vec{F} = - \vec{\nabla}U = \vec{\nabla}\left( \vec{d} \cdot \vec{E} \right) \\
 \vec{N} = \vec{d} \times \vec{E} \\
 \phi(\vec{r}) = \dfrac{1}{4\pi\varepsilon_0}\int \dfrac{\rho(\vec{r}')}{R}\,dV' \\
 \vec{E} = - \vec{\nabla}\phi \\
 \nabla^{2}\phi = -\dfrac{\rho}{\varepsilon_{0}}
\end{cases}
\]

Dove:
\begin{itemize}
    \item \(\vec{d}\) è il \textbf{momento di dipolo elettrico}, vettore diretto dalla carica negativa a quella positiva, proporzionale al modulo della carica e alla distanza di separazione;
    \item \(\vec{E}\) è il \textbf{campo elettrico}, che rappresenta la forza per unità di carica esercitata nello spazio;
    \item \(\phi\) è il \textbf{potenziale elettrico scalare}, da cui il campo deriva come gradiente negativo;
    \item \(\rho\) è la \textbf{densità di carica elettrica}, sorgente del campo elettrico;
    \item \(\varepsilon_0\) è la \textbf{costante dielettrica del vuoto}, che stabilisce la relazione tra campo elettrico e densità di carica.
\end{itemize}

Per il campo magnetico risulta:

\[
\begin{cases}
 U = - \vec{\mu} \cdot \vec{B} \\
 \vec{F} = - \vec{\nabla}U = \vec{\nabla}\left( \vec{\mu} \cdot \vec{B} \right) \\
 \vec{N} = \vec{\mu} \times \vec{B} \\
 \vec{A}(\vec{r}) = \dfrac{\mu_0}{4\pi}\int \dfrac{\vec{J}(\vec{r}')}{R}\,dV' \\
 \vec{B} = \vec{\nabla} \times \vec{A} \\
 \nabla^{2}\vec{A} = -\mu_{0}\vec{J}
\end{cases}
\]

Dove:
\begin{itemize}
    \item \(\vec{\mu}\) è il \textbf{momento di dipolo magnetico}, legato al moto circolare di cariche (correnti) e orientato secondo la regola della mano destra rispetto al verso della corrente;
    \item \(\vec{B}\) è il \textbf{campo magnetico}, che descrive l’azione sui dipoli magnetici e sulle particelle cariche in movimento;
    \item \(\vec{A}\) è il \textbf{potenziale vettore magnetico}, da cui si ricava il campo magnetico mediante il rotore;
    \item \(\vec{J}\) è la \textbf{densità di corrente elettrica}, sorgente del campo magnetico;
    \item \(\mu_0\) è la \textbf{permeabilità magnetica del vuoto}, che regola l’intensità del campo magnetico generato dalle correnti.
\end{itemize}

Queste relazioni mostrano il parallelismo formale tra le grandezze elettriche \((\vec{d}, \vec{E}, \phi, \rho, \varepsilon_0)\) e le corrispondenti magnetiche \((\vec{\mu}, \vec{B}, \vec{A}, \vec{J}, \mu_0)\).

\begin{table}[h!]
\centering
\begin{tabular}{@{} c c l @{}}
\toprule
\textbf{Elettrico} & \textbf{Magnetico} & \textbf{Descrizione} \\ 
\midrule
$\vec{d}$ & $\vec{\mu}$ & Momento di dipolo (separazione di cariche o corrente circolare) \\
$\vec{E}$ & $\vec{B}$ & Campo vettoriale (forza per unità di carica o effetto su dipoli) \\
$\phi$ & $\vec{A}$ & Potenziale (scalare o vettoriale) da cui deriva il campo \\ 
$\rho$ & $\vec{J}$ & Sorgente del campo (cariche o correnti) \\
$\varepsilon_0$ & $\mu_0$ & Costanti fondamentali del vuoto \\
\bottomrule
\end{tabular}
\caption{Confronto tra le grandezze elettriche e magnetiche e loro significato fisico.}
\label{tab:parallelo-elettromagnetico}
\end{table}

Questa analogia sottolinea la profonda simmetria formale tra elettricità e magnetismo, unificate nella teoria elettromagnetica di Maxwell.

\section{Forza e coppia su una piccola spira}\label{forza-e-coppia-su-una-piccola-spira}

Si vuole determinare la forza totale agente su una piccola spira di superficie \(S\), percorsa da una corrente \(I\), immersa in un campo di induzione magnetica uniforme.

\begin{figure}[ht]
\centering
\includegraphics[width=0.35\textwidth]{media/3_Elettromagnetismo/image20.pdf}
\caption{Spira percorsa da corrente in un campo magnetico.}
\end{figure}

Il momento magnetico dipolare è dato da:
\[
\vec{\mu} = I S\,\hat{\imath}_n
\]
dove \(\hat{\imath}_n\) è la normale alla superficie \(S\), orientata in modo da vedere la corrente ruotare in senso antiorario.

Si consideri un tratto elementare \(d\vec{s}\) della spira; su di esso agisce la forza di Lorentz, poiché le cariche al suo interno sono messe in moto dalla corrente elettrica:
\[
d\vec{F} = I\,d\vec{s} \times \vec{B}
\]
Integrando su tutta la lunghezza della spira si ottiene:
\[
\vec{F} = \oint_{\partial S} I\,d\vec{s} \times \vec{B}
\]
Poiché la corrente e il campo \(\vec{B}\) sono costanti, possono essere portati fuori dal segno di integrale:
\[
\vec{F} = I\left( \oint_{\partial S} d\vec{s} \right) \times \vec{B}
\]
Siccome la spira è in equilibrio, la somma di tutti i contributi elementari è nulla; di conseguenza la forza risultante è nulla:
\[
\vec{F} = \vec{0}
\]
ovvero:
\[
I\left( \oint_{\partial S} d\vec{s} \right) \times \vec{B} = \vec{0}
 \Longleftrightarrow
\left( \oint_{\partial S} d\vec{s} \right) \times \vec{B} = \vec{0}
\]

Anche se la forza netta agente sulla spira è nulla, possono comunque esistere momenti torcenti. Poiché la forza agente sulla spira elementare è nulla, il momento delle forze risulta indipendente dal polo scelto. La coppia agente sull’elemento infinitesimo è data da:
\[
d\vec{N} = \vec{r} \times d\vec{F}
\]
La coppia, o momento torcente, può essere denotata anche con \(\vec{\tau}\). Integrando lungo la spira si ottiene:
\[
\vec{N} = \oint_{\partial S} \vec{r} \times d\vec{F}
\]
Sostituendo la forza di Lorentz \(d\vec{F} = I\,d\vec{s} \times \vec{B}\), si ha:
\[
\vec{N} = I \oint_{\partial S} \vec{r} \times (d\vec{s} \times \vec{B})
\]
Dati tre vettori, è valida l’identità:
\[
\vec{a} \times (\vec{b} \times \vec{c}) = (\vec{a}\cdot\vec{c})\,\vec{b} - (\vec{a}\cdot\vec{b})\,\vec{c}
\]
Applicando tale relazione, l'espressione del momento si scrive come:
\[
\vec{N} = I \oint_{\partial S} \big[(\vec{r}\cdot\vec{B})\,d\vec{s} - \vec{B}\,(\vec{r}\cdot d\vec{s})\big]
\]
Poiché il vettore \(\vec{r}\) appartiene alla spira, risulta \(d\vec{s} = d\vec{r}\), e dunque:
\[
\vec{N} = I \oint_{\partial S} \big[(\vec{r}\cdot\vec{B})\,d\vec{r} - \vec{B}\,(\vec{r}\cdot d\vec{r})\big]
\]
Il termine \(\vec{r}\cdot d\vec{r}\) può essere riscritto come:
\[
\vec{r}\cdot d\vec{r} = \dfrac{1}{2}\,d(\vec{r}\cdot\vec{r})
\]
per cui:

\[\vec{N} = I\oint_{\partial S}{\left( \vec{r} \cdot \vec{B} \right)d\vec{r}} - I\oint_{\partial S}{\vec{B}\left( \vec{r} \cdot d\vec{r} \right)} = I\oint_{\partial S}{\left( \vec{r} \cdot \vec{B} \right)d\vec{r}} - \dfrac{1}{2}I\vec{B}\oint_{\partial S}{d\left( \vec{r} \cdot \vec{r} \right)}\]

L'ultimo termine è l'integrale esteso a una linea chiusa di una forma differenziale esatta, dunque, è nullo:

\[\oint_{\partial S}{d\left( \vec{r} \cdot \vec{r} \right)} = 0\]

Pertanto, la coppia risulta:

\[\vec{N} = I\oint_{\partial S}{\left( \vec{r} \cdot \vec{B} \right)d\vec{r}}\]

Per risolvere tale integrale si considera la quantità \(d\left( \left( \vec{r} \cdot \vec{B} \right)\vec{r} \right)\); si applicano le proprietà del differenziale:

\[d\left( \left( \vec{r} \cdot \vec{B} \right)\vec{r} \right) = \left( d\vec{r} \cdot \vec{B} \right)\vec{r} + \left( \vec{r} \cdot \vec{B} \right)d\vec{r} + \left( \vec{r} \cdot d\vec{B} \right)\vec{r}\]

Dato che il campo è uniforme, \(d\vec{B} = 0\). Per cui:

\[d\left( \left( \vec{r} \cdot \vec{B} \right)\vec{r} \right) = \left( d\vec{r} \cdot \vec{B} \right)\vec{r} + \left( \vec{r} \cdot \vec{B} \right)d\vec{r}\]

Si ricava la quantità \(\left( \vec{r} \cdot \vec{B} \right)d\vec{r}\), presente nell'espressione della coppia:

\[\left( \vec{r} \cdot \vec{B} \right)d\vec{r} = d\left( \left( \vec{r} \cdot \vec{B} \right)\vec{r} \right) - \left( d\vec{r} \cdot \vec{B} \right)\vec{r}\]

Per applicare nuovamente la proprietà sul prodotto vettore di tre vettori, si aggiunge e sottrae \(\left( \vec{r} \cdot \vec{B} \right)d\vec{r}\) al secondo membro:

\[\left( \vec{r} \cdot \vec{B} \right)d\vec{r} = d\left( \left( \vec{r} \cdot \vec{B} \right)\vec{r} \right) - \left( d\vec{r} \cdot \vec{B} \right)\vec{r} - \left( \vec{r} \cdot \vec{B} \right)d\vec{r} + \left( \vec{r} \cdot \vec{B} \right)d\vec{r}\]

Dove:

\[\left( \vec{r} \cdot \vec{B} \right)d\vec{r} - \left( d\vec{r} \cdot \vec{B} \right)\vec{r} = \vec{r} \times d\vec{r} \times \vec{B}\]

Per cui si ha:

\[\left( \vec{r} \cdot \vec{B} \right)d\vec{r} = d\left( \left( \vec{r} \cdot \vec{B} \right)\vec{r} \right) + \vec{r} \times d\vec{r} \times \vec{B} - \left( \vec{r} \cdot \vec{B} \right)d\vec{r} \Leftrightarrow 2\left( \vec{r} \cdot \vec{B} \right)d\vec{r} = d\left( \left( \vec{r} \cdot \vec{B} \right)\vec{r} \right) + \vec{r} \times d\vec{r} \times \vec{B}\]

In definitiva, si ha:

\[\left( \vec{r} \cdot \vec{B} \right)d\vec{r} = \dfrac{1}{2}d\left( \left( \vec{r} \cdot \vec{B} \right)\vec{r} \right) + \dfrac{1}{2}\vec{r} \times d\vec{r} \times \vec{B}\]

Sostituendo tale risultato nell’espressione della coppia si ha::

\[\vec{N} = I\oint_{\partial S}{\left( \vec{r} \cdot \vec{B} \right)d\vec{r}} = \dfrac{1}{2}\ I\oint_{\partial S}{d\left( \left( \vec{r} \cdot \vec{B} \right)\vec{r} \right)} + \dfrac{1}{2}\ I\oint_{\partial S}{\vec{r} \times d\vec{r} \times \vec{B}}\]

Il primo termine è nullo, essendo una circuitazione di una forma differenziale esatta, quindi:

\[\oint_{\partial S}{d\left( \left( \vec{r} \cdot \vec{B} \right)\vec{r} \right)} = \vec{0}\]

Per cui si ha:

\[\vec{N} = \dfrac{1}{2}\ I\oint_{\partial S}{\vec{r} \times d\vec{r} \times \vec{B}}\]

Siccome il campo è costante lungo tutto il percorso di integrazione, può essere portato all'esterno del simbolo di integrale:

\[\vec{N} = \dfrac{1}{2}\ I\left( \oint_{\partial S}{\vec{r} \times d\vec{r}} \right) \times \vec{B}\]

Si definisce il \textit{momento magnetico} della spira:

\[\vec{\mu} = \dfrac{1}{2}\ I\oint_{\partial S}{\vec{r} \times d\vec{r}} = IS{\hat{\imath}}_{n}\]

La coppia \(\vec{N}\) può, in definitiva, essere espressa come:

\[\vec{N} = \vec{\mu} \times \vec{B}\]

Nel caso più generale, il momento magnetico di una distribuzione continua di corrente in un volume \(V\) è dato da:

\[\vec{\mu} = \dfrac{1}{2}\int_{V}{\vec{r} \times \vec{J}dV}\]

\subsection{Momento magnetico di un anello rotante}\label{momento-magnetico-di-un-anello-rotante}

Si vuole calcolare il momento magnetico di un anello sottile di raggio \(R\), massa \(M\) e carica \(q\) distribuita uniformemente, che ruota con velocità angolare \(\omega\) attorno ad un asse ortogonale al piano in cui giace l'anello e passante per il centro.

\begin{figure}[ht]
\centering
\includegraphics[width=1.34709in,height=2.1703in,alt={P1332\#yIS1}]{media/3_Elettromagnetismo/image21.pdf}\caption{Anello rotante nel campo magnetico}
\end{figure}

L’anello, ruotando, genera una corrente la cui intensità media è:

\[I = \dfrac{q}{T}\]

Dove \(T\) è il periodo dell'oscillazione dato da:

\[T = \dfrac{2\pi}{\omega}\]

Per cui la corrente è data da:

\[I = \dfrac{q}{T} = q\dfrac{\omega}{2\pi}\]

L'anello è assimilabile a una spira percorsa da corrente, dunque, il suo momento magnetico è dato da:

\[\vec{\mu} = IS{\hat{\imath}}_{n} = q\dfrac{\omega}{2\pi}\pi R^{2}{\hat{\imath}}_{n}\]

Semplificando, si ottiene il momento magnetico:

\[\vec{\mu} = \dfrac{1}{2}q\omega R^{2}{\hat{\imath}}_{n}\]

\subsection{Momento magnetico per un guscio sferico}\label{momento-magnetico-per-un-guscio-sferico}

Si vuole calcolare il momento magnetico di uno strato sferico sottile di raggio \(R\), massa \(M\) e carica \(Q\) distribuita uniformemente, che ruota con velocità angolare \(\omega\) attorno ad un asse passante per il centro della sfera.

\begin{figure}[ht]
\centering
\includegraphics[width=2.625in,height=1.991in,alt={P1346\#yIS1}]{media/3_Elettromagnetismo/image22.pdf}\caption{Guscio sferico}
\end{figure}

Si pone l'asse \(z\) coincidente con l'asse di rotazione, per cui \(\vec{\omega}=\omega\,\hat{z}\). Si suddivide la superficie sferica in anelli infinitesimi, ciascuno giacente su un piano ortogonale all'asse \(z\). In corrispondenza dell'angolo polare \(\vartheta\) l'anello ha:
\[
\text{raggio: } r=R\sin\vartheta,\qquad
\text{lunghezza: } 2\pi R\sin\vartheta,\qquad
\text{larghezza (sulla sfera): } R\,d\vartheta.
\]

L'elemento di area dell'anello è quindi:
\[
dS = (2\pi R\sin\vartheta)(R\,d\vartheta)=2\pi R^{2}\sin\vartheta\,d\vartheta.
\]

Se \(\sigma\) è la densità superficiale di carica uniforme, la carica dell'anello è \(dq=\sigma\,dS\) e la corrente associata alla sua rotazione è:
\[
dI=\dfrac{dq}{T}=\dfrac{\sigma\,dS}{T}=\dfrac{\omega\sigma}{2\pi}\,2\pi R^{2}\sin\vartheta\,d\vartheta
=\omega\sigma R^{2}\sin\vartheta\,d\vartheta,
\]
con \(T=2\pi/\omega\) periodo di rotazione della sfera. L'area della spira (anello) è \(S=\pi (R\sin\vartheta)^{2}\), dunque il momento magnetico elementare vale:
\[
d\vec{\mu}=S\,dI\,\hat{z}
= \pi R^{2}\sin^{2}\vartheta\;\omega\sigma R^{2}\sin\vartheta\,d\vartheta\;\hat{z}
= \pi\omega\sigma R^{4}\sin^{3}\vartheta\,d\vartheta\;\hat{z}.
\]

Per le ipotesi fatte sul sistema di riferimento, risulta che \({\hat{\imath}}_{n} = {\hat{\imath}}_{z}\). Infatti, il momento magnetico è diretto come \(\vec{\omega}\). Svolgendo i prodotti, si ha:

\[d\vec{\mu} = \pi\omega\sigma R^{4}\sin^{3}\vartheta\ d\vartheta{\hat{\imath}}_{z}\]

Il momento magnetico è ottenuto integrando l'equazione ottenuta su tutti i possibili valori assunti da \(\vartheta\), ovvero:

\[\vec{\mu} = \int_{0}^{\pi}{\pi\omega\sigma R^{4}\sin^{3}\vartheta\ d\vartheta{\hat{\imath}}_{z}} = \pi\omega\sigma R^{4}\int_{0}^{\pi}{\sin^{3}\vartheta\ d\vartheta}{\hat{\imath}}_{z}\]

Si risolve l'integrale. Per le relazioni trigonometriche è possibile scrivere:

\[\int_{0}^{\pi}{\sin^{3}\vartheta d\vartheta} = \int_{0}^{\pi}{\left( \dfrac{3\sin\vartheta - \sin{3\vartheta}}{4} \right)d\vartheta} = \dfrac{1}{4}\left( 3\int_{0}^{\pi}{\sin\vartheta d\vartheta} - \int_{0}^{\pi}{\sin{3\vartheta}d\vartheta} \right)\]

Dove:

\[3\int_{0}^{\pi}{\sin\vartheta d\vartheta} = 3\left\lbrack - \cos\vartheta \right\rbrack_{0}^{\pi} = 3\left( - \cos\pi + \cos 0 \right) = 3(1 + 1) = 6\]

\[\int_{0}^{\pi}{\sin{3\vartheta}d\vartheta} = - \dfrac{1}{3}\left\lbrack \cos{3\vartheta} \right\rbrack_{0}^{\pi} = - \dfrac{1}{3}\left( \cos{3\pi} - \cos 0 \right) = - \dfrac{1}{3}( - 1 - 1) = \dfrac{2}{3}\]

Nel complesso, l'integrale è dato da:

\[\int_{0}^{\pi}{\sin^{3}\vartheta d\vartheta} = \dfrac{1}{4}\left( 6 - \dfrac{2}{3} \right) = \dfrac{1}{4}\left( \dfrac{18 - 2}{3} \right) = \dfrac{1}{4}\dfrac{16}{3} = \dfrac{4}{3}\]

Il momento magnetico è dato da:

\[\vec{\mu} = \dfrac{4}{3}\ \pi\omega\sigma R^{4}{\hat{\imath}}_{z}\]

La superficie totale della sfera è data da:

\[S = 4\pi R^{2}\]

Il prodotto della densità superficiale di carica \(\sigma\) per la superficie \(S\) restituisce la carica globale \(Q\). Il momento magnetico può essere scritto come:

\[\vec{\mu} = \dfrac{1}{3}\ \omega QR^{2}{\hat{\imath}}_{z}\]

%\input{Image/Ring}

\subsection{Momento magnetico per una sfera}\label{momento-magnetico-per-una-sfera}

Si vuole calcolare il momento magnetico di una sfera raggio \(R\), massa \(M\) e carica \(Q\) distribuita uniformemente, che ruota con velocità angolare \(\omega\) attorno ad un asse passante per il centro della sfera.

\begin{figure}[ht]
\centering
\includegraphics[width=1.375in,height=1.73838in,alt={P1372\#yIS1}]{media/3_Elettromagnetismo/image23.pdf}\caption{Sfera rotante}
\end{figure}

Si utilizzano le coordinate sferiche \(r\), \(\vartheta\), \(\varphi\). Si assume come polo il centro \(O\) della sfera carica e come asse polare il diametro parallelo alla velocità angolare\(\omega\). Dove:

\[r \in \left[ 0; R\right],\ \vartheta \in \left[ 0;\pi\right],\ \varphi \in \left[ 0;2\pi\right]\]

Si indica con \(\rho\) la densità di carica volumetrica. Il volumetto \(dV\) contiene una carica \(dq\):

\[dq = \rho dV\]

Un elemento di volume $dV$ contribuisce alla densità di corrente volumetrica $\vec{J}$, data da:

\[\vec{J} = \rho \vec{v}\]

Poiché la sfera ruota con velocità angolare \(\vec{\omega}\) e la carica ha densità \(\rho\), la densità di corrente volumetrica è:

\[\vec{J} = \rho \vec{v} = \rho \left( \vec{\omega} \times \vec{r}\right)\]

per calcolare il momento magnetico di un corpo di volume \(V\) con densità di corrente \(J\), si usa la formula generale:

\[\vec{\mu} = \dfrac{1}{2}\int_{V}{\vec{r} \times \vec{J}dV}\]

Sostituendo la densità di corrente dell'elemento di volume infinitesimo nella definizione del momento magnetico si ha:

\[\vec{\mu} = \dfrac{1}{2}\int_{V}{\vec{r} \times \left(\rho \left( \vec{\omega} \times \vec{r}\right)\right)dV}\]

Nell'ipotesi di densità di carica costante in tutto il volume, è possibile portare all'esterno del simbolo di integrale \(\rho\):


\[\vec{\mu} = \dfrac{1}{2}\rho\int_{V}{\vec{r} \times \left( \vec{\omega} \times \vec{r}\right)dV}\]

Si considera il triplo prodotto vettoriale all'interno dell'integrale e si utilizza l'identità nota:

\[
\vec{a} \times (\vec{b} \times \vec{c}) = (\vec{a} \cdot \vec{c})\vec{b} - (\vec{a} \cdot \vec{b})\vec{c}
\]

Ponendo \(\vec{a} = \vec{r}\), \(\vec{b} = \vec{\omega}\), e \(\vec{c} = \vec{r}\), l'espressione \(\vec{r} \times \vec{\omega} \times \vec{r} \) diventa:

\[
\vec{r} \times \vec{\omega} \times \vec{r} = (\vec{r} \cdot \vec{r})\vec{\omega} - (\vec{r} \cdot \vec{\omega})\vec{r}
\]

Ricordando che \(\vec{r} \cdot \vec{r} = r^2\), l'ultima relazione può essere scritta come:

\[
\vec{r} \times \vec{\omega} \times \vec{r} = r^{2}\vec{\omega}-\left(\vec{r}\cdot\vec{\omega}\right)\vec{r}
\]

Per ipotesi, la rotazione avviene lungo la direzione \(\hat{\imath}_{z}\), per cui \(\vec{\omega}=\omega \hat{\imath}_{z}\). Inoltre, è possibile esprimere il vettore posizione in coordinate sferiche:

\[
\vec{r} = r (\sin\vartheta \cos\varphi\,\hat{\imath}_x + \sin\vartheta \sin\varphi\,\hat{\imath}_y + \cos\vartheta\,\hat{\imath}_z)
\]

Il prodotto scalare tra il vettore posizione e la velocità angolare avviene solamente tre le componenti lungo \(\hat{\imath}_z\):

\[
\vec{r}\cdot \vec{\omega} = r (\sin\vartheta \cos\varphi\,\hat{\imath}_x + \sin\vartheta \sin\varphi\,\hat{\imath}_y + \cos\vartheta\,\hat{\imath}_z) \cdot \omega\hat{\imath}_z = r\omega\cos\vartheta\
\]

Il prodotto vettoriale triplo si scrive come:

\[
\vec{r} \times \vec{\omega} \times \vec{r} = {r}^2\omega\hat{\imath}_z - \left(\omega r \cos\vartheta\right)\left(r (\sin\vartheta \cos\varphi\,\hat{\imath}_x + \sin\vartheta \sin\varphi\,\hat{\imath}_y + \cos\vartheta\,\hat{\imath}_z)\right)
\]

Svolgendo i prodotti, si ricava:

\[
\vec{r} \times \vec{\omega} \times \vec{r} = {r}^2\omega\hat{\imath}_z - r^{2}\omega\left( \sin{\vartheta}\cos{\varphi}\,\cos\vartheta\hat{\imath}_{x} + \sin\vartheta \sin\varphi\,\cos\vartheta\hat{\imath}_y+\cos^2\vartheta\hat{\imath}_z\right)
\]

Sostituendo questo risultato nell'espressione vettoriale per il momento magnetico, si ottiene:

\[
\vec{\mu} = \dfrac{1}{2}\rho\int_{V}{ \left({r}^2\omega\hat{\imath}_z - r^{2}\omega\left( \sin{\vartheta}\cos{\varphi}\,\cos\vartheta\hat{\imath}_{x} + \sin\vartheta \sin\varphi\,\cos\vartheta\hat{\imath}_y+\cos^2\vartheta\hat{\imath}_z\right)\right)dV}
\]

Si raccoglie \(\omega\) che, essendo costante rispetto la posizione, può essere portata fuori dal simbolo di integrale:

\[
\vec{\mu} = \dfrac{1}{2}\rho\omega \int_{V}{ \left({(r}^2\hat{\imath}_z - r^{2}\left( \sin{\vartheta}\cos{\varphi}\,\cos\vartheta\hat{\imath}_{x} + \sin\vartheta \sin\varphi\,\cos\vartheta\hat{\imath}_y+\cos^2\vartheta\hat{\imath}_z\right)\right)dV}
\]

Si divide l'integrale in base alle componenti spaziali delle coordinate sferiche:

\[
\vec{\mu} = \dfrac{1}{2}\rho\omega \left(\int_{V}{ \left(r^2\,\hat{\imath}_z - r\cos^{2}{\vartheta} \right)\hat{\imath}_z\,dV} - \int_{V}{\left( r^{2} \sin{\vartheta}\cos{\varphi}\,\cos\vartheta\hat{\imath}_{x} \right) \,dV} - \int_{V}{\left( r^2sin\vartheta \sin\varphi\,\cos\vartheta\hat{\imath}_y\right) \,dV}\right)
\]

In coordinate sferiche, l'elemento di volume infinitesimo è il prodotto delle tre lunghezze elementari:

\[
dV = 
\underbrace{dr}_{\text{spessore radiale}}\,
\underbrace{(r\,d\vartheta)}_{\text{altezza meridiana}}\,
\underbrace{(r\sin\vartheta\,d\varphi)}_{\text{lunghezza azimutale}}
= r^{2}\sin\vartheta\,dr\,d\vartheta\,d\varphi
\]

Si risolve l'integrale contenente la componente lungo la direzione \(\hat{\imath}_x\). Utilizzando le coordinate sferiche e integrando su tutto il volume, si ottiene:

\[
\int_{V}{\left( r^{2} \sin{\vartheta}\cos{\varphi}\,\cos\vartheta \right) \,dV}=\int_{0}^{R}{\int_{0}^{\pi}{\int_{0}^{2\pi}{r^{2} \sin{\vartheta}\cos{\varphi}\,\cos\vartheta \left(r^{2}\sin\vartheta\right)d\varphi}d\vartheta}dr}=
\]

Per la linearità dell'operatore integrale è possibile scrivere:

\[
=\int_{0}^{R}{r^{4}\int_{0}^{\pi}{\sin^{2}{\vartheta}\cos{\vartheta}\int_{0}^{2\pi}{cos{\varphi}\,d\varphi}d\vartheta}dr}
\]

Nella relazione individuata il \(cos{\varphi}\) è integrato sul suo periodo, pertanto il suo risultato è nullo. Di conseguenza anche l'integrale complessivo è nullo:

\[
\int_{V}{\left( r^{2} \sin{\vartheta}\cos{\varphi}\,\cos\vartheta \right) \,dV}=\int_{0}^{R}{r^{4}\int_{0}^{\pi}{\sin^{2}\cos{\vartheta}\int_{0}^{2\pi}{cos{\varphi}\,d\varphi}d\vartheta}dr}=0
\]

Per l'integrazione lungo la componente \(\hat{\imath}_y\) vale un discorso analogo, dunque, anche questo contributo è nullo. Le componenti $\mu_x$ e $\mu_y$ del momento magnetico si annullano a causa della simmetria di rotazione della sfera attorno all'asse $\hat{\imath}_{z}$, espressa matematicamente dall'integrazione su $\varphi$ (l'angolo azimutale) che annulla i termini $\cos\varphi$ e $\sin\varphi$. Il momento magnetico totale è quindi solo in direzione $\hat{\imath}_z$:

\[
\vec{\mu} = \mu_z\,\hat{\imath}_z
\]

Al fine di valutare l'espressione del momento magnetico, bisogna risolvere l'integrale lungo l'asse di rotazione:

\[
\mu_{z} = \dfrac{1}{2}\rho\omega\int_{V}{\left(r^{2} - r^{2}\cos^2{\vartheta} \right) dV} = \dfrac{1}{2}\rho\omega\int_{V}{r^{2}\left(1 -\cos^2{\vartheta} \right) dV}
\]

Si utilizza l'identità trigonometrica \(1 - \cos^2{\vartheta} = \sin^2{\vartheta}\) e si sostituisce l'espressione del volumetto elementare \(dV = r^{2}\sin\vartheta drd\vartheta d\varphi\):

\[
\mu_{z} = \dfrac{1}{2}\rho\omega\int_{V}{r^{2}\sin^2{\vartheta} dV}=\dfrac{1}{2}\rho\omega\int_{0}^{R}{\int_{0}^{\pi}{\int_{0}^{2\pi}{r^{2}\sin^2{\vartheta} \left(r^{2}\sin\vartheta drd\vartheta d\varphi\right)}}}
\]

Riorganizzando i termini e separando gli integrali, si ricava:

\[
\mu_{z} = \dfrac{1}{2}\rho\omega \left(\int_{0}^{2\pi} d\varphi\right) \left(\int_{0}^{\pi} \sin^3{\vartheta} d\vartheta\right) \left(\int_{0}^{R} r^{4} dr\right)
\]

L'integrale sulla coordinata azimutale è:

\[
\int_{0}^{2\pi} d\varphi = 2\pi
\]

L'integrale sul raggio, è invece dato da:

\[
\int_{0}^{R} r^{4} dr = \left[ \dfrac{r^{5}}{5} \right]_{0}^{R} = \dfrac{R^{5}}{5}
\]

L'integrale sulla coordinata polare è:

\[\int_{0}^{\pi}{\sin^{3}\vartheta d\vartheta} = \int_{0}^{\pi}{\left( \dfrac{3\sin\vartheta - \sin{3\vartheta}}{4} \right)d\vartheta} = \dfrac{1}{4}\left( 3\int_{0}^{\pi}{\sin\vartheta d\vartheta} - \int_{0}^{\pi}{\sin{3\vartheta}d\vartheta} \right)\]

\[\int_{0}^{\pi}{\sin^{3}\vartheta d\vartheta} = \int_{0}^{\pi}{\left( \dfrac{3\sin\vartheta - \sin{3\vartheta}}{4} \right)d\vartheta} = \dfrac{1}{4}\left( 3\int_{0}^{\pi}{\sin\vartheta d\vartheta} - \int_{0}^{\pi}{\sin{3\vartheta}d\vartheta} \right)\]

Dove:

\[3\int_{0}^{\pi}{\sin\vartheta d\vartheta} = 3\left\lbrack - \cos\vartheta \right\rbrack_{0}^{\pi} = 3\left( - \cos\pi + \cos 0 \right) = 3(1 + 1) = 6\]

\[\int_{0}^{\pi}{\sin{3\vartheta}d\vartheta} = - \dfrac{1}{3}\left\lbrack \cos{3\vartheta} \right\rbrack_{0}^{\pi} = - \dfrac{1}{3}\left( \cos{3\pi} - \cos 0 \right) = - \dfrac{1}{3}( - 1 - 1) = \dfrac{2}{3}\]

Nel complesso, l'integrale è dato da:

\[
\int_{0}^{\pi} \sin^3{\vartheta} d\vartheta = \dfrac{4}{3}
\]


Di conseguenza, il momento magnetico della sfera è dato da:

\[
\mu_{z} = \dfrac{1}{2}\rho\omega \left( 2\pi \right) \left( \dfrac{4}{3} \right) \left( \dfrac{R^{5}}{5} \right) = \dfrac{4\pi}{15}\rho\omega R^5
\]

La carica totale \(Q\) è data dal prodotto della densità volumetrica \(\rho\) per il volume della sfera \(V=4\pi R^{3}/3\):

\[
Q=\rho V\Leftrightarrow \rho = \dfrac{Q}{V}
\]

Sostituendo il volume della sfera, si ottiene:
\[
Q=\rho V\Leftrightarrow \rho = \dfrac{Q}{V}=\dfrac{Q}{\dfrac{4}{3}\pi R^3}=\dfrac{3Q}{4\pi R^3}
\]

Sostituendo l'espressione di \(\rho\) nell'espressione per il momento magnetico lungo l'asse di rotazione, si ottiene:

\[
\mu_{z} = \dfrac{4\pi}{15}\omega R^5 \left( \dfrac{3Q}{4\pi R^3} \right)
\]

Semplificando:

\[
\mu_{z} = \dfrac{1}{5} Q \omega R^2
\]

Dato che il momento magnetico della sfera rotante è diretto lungo l'asse di rotazione, si ha:
\[
\vec{\mu} = \dfrac{1}{5} Q \omega R^{2}\hat{\imath}_z
\]

\section{Forza su piccola spira in un campo disomogeneo}\label{forza-su-piccola-spira-in-un-campo-disomogeneo}

La forza agente su una spira elementare, percorsa da corrente, immersa in un campo magnetico generico può essere espressa come:

\[\vec{F} = \vec{\nabla}\left( \vec{\mu} \cdot \vec{B} \right)\]

Dove \(\vec{\mu}\) è il momento magnetico della spira. Per un atomo il momento magnetico è costante, dunque, può essere portato all'esterno del simbolo di gradiente, producendo l'operatore:

\[
(\vec{\mu} \cdot \vec{\nabla}) = \mu_x\dfrac{\partial}{\partial x} + \mu_y\dfrac{\partial}{\partial y} + \mu_z\dfrac{\partial}{\partial z}
\]

L'espressione per la forza si riduce a:

\[
\vec{F} = \left(\vec{\mu} \cdot \vec{\nabla}\right) \vec{B}
\]

L'equazione, scritta in forma estesa, è:

\[
\begin{pmatrix}
F_{x} \\
F_{y} \\
F_{z}
\end{pmatrix} = \begin{pmatrix}
     \mu_x\dfrac{\partial}{\partial x} + \mu_y\dfrac{\partial}{\partial y} + \mu_z\dfrac{\partial}{\partial z}
\end{pmatrix}\begin{pmatrix}
B_{x} \\
B_{y} \\
B_{z}
\end{pmatrix} = \begin{pmatrix}
 \mu_x\dfrac{\partial B_x}{\partial x} + \mu_y\dfrac{\partial B_x}{\partial y} + \mu_z\dfrac{\partial B_x}{\partial z} \\
 \mu_x\dfrac{\partial B_y}{\partial x} + \mu_y\dfrac{\partial B_y}{\partial y} + \mu_z\dfrac{\partial B_y}{\partial z} \\
 \mu_x\dfrac{\partial B_z}{\partial x} + \mu_y\dfrac{\partial B_z}{\partial y} + \mu_z\dfrac{\partial B_z}{\partial z}
\end{pmatrix}
\]

Si suppone che il campo sia diretto solamente lungo \(z\), direzione lungo cui è variabile. In altre parole, il campo è dato da:

\[\vec{B} = B_{z}(z){\hat{\imath}}_{z}\]

La forza, in questo contesto, può essere espressa come:
\[
\begin{pmatrix}
F_{x} \\
F_{y} \\
F_{z}
\end{pmatrix} = \begin{pmatrix}
      \mu_x\dfrac{\partial}{\partial x} + \mu_y\dfrac{\partial}{\partial y} + \mu_z\dfrac{\partial}{\partial z}
\end{pmatrix}\begin{pmatrix}
0 \\
0 \\
B_{z}\left(z\right)
\end{pmatrix} = \begin{pmatrix}
0 \\
0 \\
 \mu_z\dfrac{\partial B_z}{\partial z}
\end{pmatrix}
\]

La componente non nulla della forza è solamente quella lungo \(z\), data da:

\[F_{z} = \mu_{z}\dfrac{\partial B_{z}}{\partial z}\]

La forza che agisce sulla spira dipende dalla proiezione del momento magnetico \(\vec{\mu}\) lungo \(z\) e dal gradiente del campo magnetico \(\vec{B}\) lungo \(z\).

Questa relazione mostra che un dipolo magnetico immerso in un campo non uniforme subisce una forza che tende a spostarlo verso le regioni di campo più intenso se il momento magnetico è positivo, \(\mu_z>0\), o meno intenso se \(\mu_z<0\).

\subsection{Esperimento di Stern-Gerlach}\label{esperimento-di-stern-gerlach}

L'esperimento di Stern-Gerlach fornisce una prova sperimentale della quantizzazione del momento magnetico dell'atomo.

Nel modello classico l'elettrone era visto come un pianeta orbitante intorno al nucleo, dunque, caratterizzato da un momento angolare dato dalla rivoluzione dell'elettrone intorno al nucleo e intorno al proprio asse.

\begin{figure}[ht]
\centering
\includegraphics[width=3.21429in,height=2.53278in,alt={P1423\#yIS1}]{media/3_Elettromagnetismo/image24.pdf}\caption{Modello planetario dell'atomo}
\end{figure}

È noto, inoltre, che la massa del nucleo è di molti ordini di grandezza maggiore rispetto quella dell'elettrone; dunque, è possibile ritenere il nucleo fermo rispetto all'elettrone.

Il moto dell'elettrone può essere considerato circolare con raggio \(r\). L'elettrone viaggia con velocità \(\vec{v}\). Per definizione la corrente è:

\[I = \dfrac{\mathrm{\Delta}q}{\mathrm{\Delta}t}\]

Dove, la carica coincide con quella dell'elettrone, mentre \(\mathrm{\Delta}t\) coincide con il periodo di rivoluzione della particella carica:

\[I = \dfrac{\mathrm{\Delta}q}{\mathrm{\Delta}t} = \dfrac{e}{2\pi\dfrac{r}{v}} = \dfrac{ev}{2\pi r}\]

Data la piccola corrente generata, esiste un momento magnetico \(\vec{\mu}\) ortogonale al piano sul quale l'elettrone esegue la sua orbita, con verso diretto in modo da vedere la corrente ruotare in senso antiorario. Per convenzione sulla corrente, l'elettrone deve ruotare in senso orario.

Il momento magnetico è dato da:

\[\vec{\mu} = IS{\hat{\imath}}_{n}\]

Si considera il modulo, si sostituisce l'espressione della corrente prodotta dall'elettrone e la superficie della spira descritta :

\[\mu = IS = - \dfrac{ev}{2\pi r}\pi r^{2} = - \dfrac{evr}{2}\]

Dove il segno meno è dovuto alla convenzione sulle correnti.

Sull'elettrone agisce un momento angolare \(\vec{L}\), dovuto all'orbita circolare descritto dall'elettrone, anch'esso ortogonale al piano dell'orbita, dato da:

\[\vec{L} = m\vec{r} \times \vec{v}\]

Si considera il modulo del momento angolare:

\[L = mrv\]

Si moltiplicano ambo i membri per \(e/2\):

\[\dfrac{e}{2}L = \dfrac{e}{2}mrv\]

Sostituendo l'espressione del momento magnetico, si ha:

\[\dfrac{e}{2}L = m\mu\]

Ricavano il momento magnetico in funzione del momento angolare, si ha:

\[\mu = \dfrac{e}{2m}L\]

In generale, se la particella ha carica \(q\) negativa, risulta:

\[\vec{\mu} = - \dfrac{q}{2m}\vec{L}\]

Dove \(\vec{\mu}\) è opposto a \(\vec{L}\) per la convenzione sulle correnti.

Sebbene l'ultima equazione sia stata ricavata nell'ambito della fisica classica, è valida anche in meccanica quantistica, in cui il punto di vista è completamente diverso.

Sebbene l'esperimento venga oggi usato per dimostrare lo spin, al tempo l'obiettivo era dimostrare l'esistenza del momento magnetico intrinseco e la sua quantizzazione spaziale. I due fisici eseguirono l'esperimento sugli atomi di argento, emessi da una sorgente. Questi atomi venivano deflessi da un campo magnetico non omogeneo, variabile lungo \(z\). L'argento è stato scelto proprio perché il suo momento angolare totale, derivante dagli elettroni di valenza, è dovuto essenzialmente a un singolo elettrone $s$-orbitale, semplificando l'interpretazione dei risultati.

La meccanica classica prevede che momento magnetico degli atomi di argento sia distribuito statisticamente in tutte le direzioni. Sugli atomi di argento agisce una forza data da:

\[F_{i} = {\mu_{i}}_{z}\dfrac{\partial B_{z}}{\partial z}\ \]

Dunque, ogni atomo subisce una deflessione dovuta alla proiezione del suo momento magnetico lungo l'asse \(z\) e dal gradiente del campo magnetico lungo lo stesso asse.

Nella teoria classica, dato che ogni atomo possiede un momento magnetico orientato casualmente, Stern e Gerlach si aspettavano di ottenere una linea retta compresa tra un massimo e minimo. Tutte le posizioni compresi tra questi due valori hanno tutti la stessa probabilità.

Tuttavia, i due scienziati rilevarono solo due punti di arrivo. I due dedussero che gli orientamenti dei momenti magnetici degli atomi non disposti in modo casuale ma in maniera quantizzata.

Questo risultato non può essere previsto dalla meccanica classica, ma viene spiegato dalla meccanica quantistica.

\begin{figure}[ht]
\centering
\includegraphics[width=5.47196in,height=2.05093in,alt={P1456\#yIS1}]{media/3_Elettromagnetismo/image25.pdf}\caption{Esperimento di Stern-Gerlach}
\end{figure}

\subsection{Concetto di spin}\label{concetto-di-spin}

Il concetto di spin è inglobato nella meccanica quantistica e può essere visualizzato come la rotazione dell'elettrone intorno al suo asse. Lo spin determina un momento angolare e un momento magnetico. I due parametri sono antiparalleli e legati dal rapporto giromagnetico ($\gamma$).

\[\vec{\mu} = - \dfrac{q}{2m}\vec{L} = - \gamma_{e}\vec{L}\]

Il concetto di elettrone rotante non può essere considerato valido poiché, nel contesto della meccanica quantistica, l'elettrone è privo di estensione superficiale. Lo spin è una proprietà intrinseca della particella, non legata a rotazione spaziale.

In meccanica quantistica, ogni particella elementare o composta che possieda un momento angolare intrinseco (spin $\vec{S}$) o orbitale ($\vec{L}$) è associata a un momento magnetico $\vec{\mu}$. Il momento magnetico intrinseco ($\vec{\mu}_s$) di una particella è sempre proporzionale al suo spin ($\vec{S}$):

\[
\vec{\mu}_s = g \dfrac{q}{2m} \vec{S}
\]
Dove:
\begin{itemize}
    \item $q$ e $m$ sono, rispettivamente, la carica e la massa della particella (ad esempio, elettrone, protone, muone, ecc.);
    \item $\vec{S}$ è il momento angolare di spin della particella;
    \item $g$ è il fattore giromagnetico ($g$-factor) o fattore di Landé, un numero adimensionale che indica di quanto il momento magnetico devia dal valore classico ($q/2m$)
\end{itemize}

Per l'elettrone, la formula è:
\[
\vec{\mu}_{e} = - g_{e} \dfrac{e}{2m_{e}} \vec{S}
\]

Il fattore $e/2m_{e}$ è l'unità naturale del magnetismo per l'elettrone ed è chiamato magnetone di Bohr ($\mu_B$). Il fattore $g_e$ per l'elettrone libero è $g_e \approx 2.0023$ (la piccola deviazione da $g=2$ è spiegata dall'Elettrodinamica Quantistica, QED).

Per i nucleoni e nuclei, si usa la massa del protone ($m_p$) come riferimento per l'unità magnetica, poiché i momenti magnetici nucleari sono molto più piccoli:

\[
\vec{\mu}_{N} = g_{N} \dfrac{e}{2m_{p}} \vec{I}
\]
Dove:
\begin{itemize}
    \item \(e/2m_{p}\) è il magnetone nucleare ($\mu_N$);
    \item $\vec{I}$ è il momento angolare di spin nucleare;
    \item $g_N$ è il fattore $g$ nucleare, che è caratteristico di ciascun nucleo.
\end{itemize}

Per un atomo in uno stato quantico definito, il momento magnetico totale ($\vec{\mu}_{tot}$) è proporzionale al momento angolare totale ($\vec{J} = \vec{L} + \vec{S}$):

\[
\vec{\mu}_{tot} = - g \dfrac{e}{2m_{e}} \vec{J}
\]
In questo caso, il fattore $g$ è il fattore di Landé ($g$) dell'atomo, il cui valore dipende da come si accoppiano $\vec{L}$ e $\vec{S}$ (ad esempio, tramite l'accoppiamento $LS$)

In sintesi, la relazione $\vec{\mu} = g q/2m \vec{S}$ è la formula unificatrice in meccanica quantistica, con il fattore $g$ che incorpora la natura specifica della particella o del sistema considerato.

\section{Campo magnetico prodotto da un momento magnetico}\label{campo-magnetico-prodotto-da-un-momento-magnetico}

Si vuole determinare il campo creato da un momento magnetico di una piccola spira percorsa da corrente. A tale scopo si considerano le equazioni di Maxwell per il campo induzione magnetica nel vuoto:

\[\begin{cases}
\vec{\nabla} \cdot \vec{B} = 0 \\
\vec{\nabla} \times \vec{B} = \mu_{0}\left( \vec{J} + \varepsilon_{0}\dfrac{\partial\vec{E}}{\partial t} \right)
\end{cases}
\]

Se la lunghezza d'onda del campo incidente \(\lambda\) è molto maggiore della dimensione lineare dell'oggetto, è possibile ritenere il campo elettromagnetico lentamente variabile sulla superficie della spira, dunque:

\[\dfrac{\partial\vec{E}}{\partial t} \simeq 0\]

È possibile scrivere:

\[\begin{cases}
\vec{\nabla} \cdot \vec{B} = 0 \\
\vec{\nabla} \times \vec{B} = \mu_{0}\vec{J}
\end{cases}
\]

Siccome il campo induzione magnetica è solenoidale, è possibile definire un potenziale vettore, tale che:

\[\vec{B} = \vec{\nabla} \times \vec{A}\]

Si sostituisce la definizione del potenziale vettore nella seconda equazione:

\[\vec{\nabla} \times \vec{B} = \mu_{0}\vec{J} \Leftrightarrow \vec{\nabla} \times \vec{\nabla} \times \vec{A}\]

Il rotore del rotore può essere scritto come:

\[\vec{\nabla} \times \vec{\nabla} \times = \vec{\nabla}\left( \vec{\nabla} \cdot \  \right) - \nabla^{2}\ \]

Dunque, si ottiene:

\[\vec{\nabla}\left( \vec{\nabla} \cdot \vec{A} \right) - \nabla^{2}\vec{A} = \mu_{0}\vec{J}\]

Il potenziale vettore non è univocamente definito, dunque, è possibile imporre la condizione, detta gauge di Coulomb:

\[\vec{\nabla} \cdot \vec{A} = 0\]

Si ottiene che il laplaciano del campo vettore è dato da:

\[\nabla^{2}\vec{A} = - \mu_{0}\vec{J}\]

Si dimostra che la soluzione è del tipo:

\[\vec{A} = \dfrac{\mu_{0}}{4\pi}\int_{V}\dfrac{\vec{J}\left( {\vec{r}}' \right)}{\left| \vec{r} - {\vec{r}}' \right|}dV'\]

Per una piccola spira, lontano da essa, si utilizza lo sviluppo in serie di multipoli per $1/\left| \vec{r} - {\vec{r}}' \right|$ e si considera solo il termine di dipolo magnetico. Si dimostra che il potenziale vettore in un punto di osservazione $\vec{R}$ è dato da::

\[\vec{A}\left( \vec{r} \right) = \dfrac{\mu_{0}}{4\pi}\dfrac{\vec{\mu} \times \vec{R}}{R^{3}}\]

Dove \(\vec{R}\) è il vettore che congiunge il centro della piccola spira col punto di osservazione.

\begin{figure}[ht]
\centering
\includegraphics[width=2.69167in,height=2.03333in,alt={P1496\#yIS1}]{media/3_Elettromagnetismo/image26.pdf}\caption{Campo prodotto da una spira elementare}
\end{figure}

L'iterazione di una spira con un campo magnetico \(\vec{B}\) esterno è descritta dall'energia potenziale \(U\), data da:

\[U = - \vec{\mu} \cdot \vec{B} = -\mu B\cos\beta\]

Con \(\beta\) angolo formato dal campo magnetico e il momento magnetico. L'energia potenziale è nulla quando il campo magnetico è ortogonale al momento magnetico.

Il momento magnetico immerso in un campo magnetico subisce l'effetto di una coppia data da:

\[\vec{\tau} = \vec{\mu} \times \vec{B}\]

Con \(\vec{\tau}\) momento torcente.

\section{Moto del momento magnetico in campo magnetico}\label{moto-del-momento-magnetico-in-campo-magnetico}

Si considera una particella, come un elettrone, un atomo o un nucleo, immerso in un campo magnetico \(\vec{B}\). Per semplicità si analizza il sistema mediante una descrizione classica e non relativistica.

Ogni atomo o particella subatomica, immerso in un campo magnetico subisce un momento torcente \(\vec{\tau}\) dato da:

\[\vec{\tau} = \vec{\mu} \times \vec{B}\]

Il momento angolare rispetta la seconda legge di Newton:

\[\dfrac{d\vec{L}}{dt} = \vec{\tau}\]

Sostituendo l'espressione per il momento angolare in funzione del momento magnetico si ha:

\[\dfrac{d\vec{L}}{dt} = \vec{\mu} \times \vec{B}\]

Il momento magnetico è legato al momento angolare da un fattore di proporzionalità \(\gamma\):

\[\vec{\mu} = \gamma\vec{L}\]

In caso di elettroni, la costante di proporzionalità coincide con il rapporto giromagnetico:

\[\gamma = - \gamma_{e}\]

In caso di atomo con:

\[\gamma = - g\dfrac{q}{2m}\]

Mentre per un nucleo con:

\[\gamma = g\dfrac{q}{2m_{p}}\]

La seconda legge di Newton può essere scritta come:

\[\dfrac{d\vec{L}}{dt} = \vec{\mu} \times \vec{B} =  \gamma\vec{L} \times \vec{B}\]

\(\gamma\) è positivo se momento angolare e momento magnetico sono paralleli, negativo se antiparalleli.

Dall'equazione ottenuta si nota che la derivata del momento angolare deve essere perpendicolare al momento angolare stesso. Ne discende che il modulo di \(\vec{L}\) è costante. In altre parole, la punta del vettore momento angolare \(\vec{L}\) giace su una sfera e percorre una traiettoria circolare nel piano perpendicolare a \(\vec{B}\). Tale modo è detto precessione del momento magnetico o del momento angolare.

\begin{figure}[ht]
\centering
\includegraphics[width=2.30833in,height=2.58576in,alt={P1524\#yIS1}]{media/3_Elettromagnetismo/image27.pdf}\caption{Moto di precessione del momento magnetico}
\end{figure}

Confrontando l'equazione ottenuta per il momento angolare:

\[\dfrac{d\vec{L}}{dt} = \gamma\vec{L} \times \vec{B} = - \gamma\vec{B} \times \vec{L}\]

Con la relazione generale che lega la rotazione di un vettore alla velocità angolare:

\[d\vec{v} = \vec{\Omega} \times \vec{v}dt\]

è evidente che la velocità del moto di precessione è data da:

\[\vec{\Omega} = - \gamma\vec{B}\]

Con questa definizione, è possibile scrivere:

\[\dfrac{d\vec{L}}{dt} = \vec{\Omega} \times \vec{L}\]

\section{Diamagnetismo}\label{diamagnetismo}

I materiali diamagnetici sono caratterizzati da atomi che, se immersi in un campo magnetico \(\vec{B}\), sviluppano un momento magnetico aggiuntivo tale che il momento magnetico risultante si oppone al campo applicato.

Si suppone di applicare lentamente un campo magnetico a un atomo di materiale diamagnetico. Il suo nucleo, avendo una massa molto maggiore dell'elettrone può essere ritenuto fermo, mentre l'elettrone ruota interno al nucleo. Questo movimento può essere assimilato a una spira di raggio \(r\) percorsa da corrente centrata sul nucleo.

\begin{figure}[ht]
\centering
\includegraphics[width=2.05833in,height=1.91724in,alt={P1537\#yIS1}]{media/3_Elettromagnetismo/image28.pdf}\caption{Nucleo immerso in un campo magnetico}
\end{figure}

Il campo magnetico \(\vec{B}\), variabile nel tempo e nello spazio, si concatena con la spira, provocando la generazione di una forza elettromagnetica data dalla legge di Faraday:

\[\oint_{C}{\vec{E} \cdot d\vec{C}} = - \dfrac{\partial}{\partial t}\int_{S}{\vec{B} \cdot d\vec{S}}\]

Dove \(S\) è la superficie della spira e \(\partial S = C\) il suo contorno. Se il campo \(\vec{B}\) è costante sulla superficie della spira può essere portato fuori dal simbolo di integrale:

\[\vec{E} \cdot \oint_{C}{d\vec{C}} = - \dfrac{\partial}{\partial t}\vec{B} \cdot \int_{S}{d\vec{S}}\]

Ne discende che il campo elettrico è anch'esso uniforme sulla spira e parallelo, in ogni punto, a \(d\vec{C}\). Per una circonferenza risulta:

\[E2\pi r = \pi r^{2}\left( - \dfrac{\partial B}{\partial t} \right)\]

Dato che \(B\) è una funzione solo del tempo, la derivata parziale si riduce a una totale. Semplificando \(\pi r\) si ottiene:

\[E = - \dfrac{r}{2}\dfrac{dB}{dt}\]

Il campo elettrico produce un momento torcente sull'elettrone, dato dalla relazione:

\[\vec{\tau} = q\vec{r} \times \vec{E}\]

Dato che il campo elettrico è ortogonale al raggio, essendo ortogonale anche al campo magnetico \(\vec{B}\), il modulo del momento torcente, esplicitando anche la carica dell'elettrone, può essere espresso come:

\[\tau = - erE\]

Sostituendo il campo elettrico prima determinato, si ha:

\[\tau = e\dfrac{r^{2}}{2}\dfrac{dB}{dt}\]

Per la seconda legge di Newton \(dL\backslash dt = \tau\), si ottiene:

\[\dfrac{dL}{dt} = e\dfrac{r^{2}}{2}\dfrac{dB}{dt}\]

Si integra tra \(t_{0}\), tempo di applicazione del campo magnetico, e \(t_{1}\) tempo in cui il campo \(B\) raggiunge il suo valore massimo:

\[\int_{t_{0}}^{t_{1}}{\dfrac{dL}{dt}dt} = e\dfrac{r^{2}}{2}\int_{t_{0}}^{t_{1}}{\dfrac{dB}{dt}dt}\]

Risolvendo si ha:

\[L\left( t_{1} \right) - L\left( t_{0} \right) = e\dfrac{r^{2}}{2}\left\lbrack B\left( t_{1} \right) - B\left( t_{0} \right) \right\rbrack\]

Inizialmente il campo è nullo \(B\left( t_{0} \right) = 0\), per cui la differenza di momento angolare è data da:

\[\mathrm{\Delta}L = e\dfrac{r^{2}}{2}B\]

Dove \(B\) è il massimo valore del campo.

Il momento magnetico dell'elettrone intorno al nucleo è antiparallelo al momento angolare. Le due quantità sono legate dal rapporto giromagnetico:

\[\mathrm{\Delta}\mu = - \dfrac{e}{2m}\mathrm{\Delta}L\]

Sostituendo l'espressione della differenza del momento angolare, si ottiene:

\[\mathrm{\Delta}\mu = - \dfrac{e^{2}r^{2}}{4m}B\]

In linea di principio l'orbita descritta dall'elettrone dovrebbe variare per l'applicazione del campo magnetico.

\begin{figure}[ht]
\centering
\includegraphics[width=2.30915in,height=2.64167in,alt={P1568\#yIS1}]{media/3_Elettromagnetismo/image29.pdf}\caption{Elettrone in equilibrio sull'orbita}
\end{figure}

Un elettrone orbitante è mantenuto, infatti, in equilibrio dalla forza centripeta (\(F_{c}\)) e dell'iterazione coulombiana con il nucleo (\(F_{e}\)):

\[F_{e} = F_{c}\]

Sostituendo le relative espressioni nell'ipotesi che il nucleo sia composto da un solo elettrone, si ha:

\[m\dfrac{v^{2}}{r} = \dfrac{1}{4\pi\varepsilon_{0}}\dfrac{e^{2}}{r^{2}}\]

La variazione di velocità indotta dal campo elettrico dopo l'introduzione del campo \(dB\) è data dalla forza che il campo esercita:

\[F = \dfrac{dp}{dt} = m\dfrac{dv}{dt}\]

La forza può essere espressa in termini di campo elettrico. Dalla definizione di campo elettrico, per l'elettrone risulta:

\[E = \dfrac{F}{q} \Leftrightarrow F = - eE\]

Dunque, la variazione di quantità di moto può essere espressa come:

\[- eE = m\dfrac{dv}{dt}\]

Il campo elettrico, per la legge di Faraday, è legato al campo magnetico dalla relazione:

\[E = - \dfrac{r}{2}\dfrac{d\vec{B}}{dt}\]

Dunque, si ha:

\[- eE = m\dfrac{dv}{dt} \Leftrightarrow e\dfrac{r}{2}\dfrac{d\vec{B}}{dt} = m\dfrac{dv}{dt}\]

Si analizza la sola variazione di velocità subita dall'elettrone:

\[dv = \dfrac{er}{2m}dB\]

A causa della variazione di velocità, anche l'accelerazione centripeta \(\alpha\) varia. Ricorrendo allo sviluppo in serie di Taylor, trascurando gli ordini superiori al primo, si ha:

\[\alpha + d\alpha = \dfrac{v^{2}}{r} + d\left( \dfrac{v^{2}}{r} \right) = \dfrac{v^{2}}{r} + \dfrac{2vdv}{r}\  + o\left( \dfrac{1}{r^{2}}dr \right) + o\left( dv^{2} \right)\]

Dopo l'applicazione del campo magnetico, bisogna considerare anche la forza di Lorentz nel bilancio delle forze sull'elettrone. La forza di Lorentz e quella elettrostatica sono concordi, dunque, affinché l'elettrone sia in equilibrio, la velocità deve aumentare per incrementare la forza centripeta:

\[\dfrac{mv^{2}}{r} + \dfrac{2mvdv}{r} = \dfrac{1}{4\pi\varepsilon_{0}}\dfrac{e^{2}}{r^{2}} + evdB\]

Si è visto che:

\[dv = \dfrac{er}{2m}dB\]

Sostituendo nel bilancio, si ha:

\[\dfrac{mv^{2}}{r} + \dfrac{2mv}{r}\dfrac{er}{2m}dB = \dfrac{1}{4\pi\varepsilon_{0}}\dfrac{e^{2}}{r^{2}} + evdB \Leftrightarrow \dfrac{mv^{2}}{r} + vedB = \dfrac{1}{4\pi\varepsilon_{0}}\dfrac{e^{2}}{r^{2}} + evdB\]

Per cui si ottiene l'equazione dell'equilibrio prima dell'applicazione del campo esterno:

\[\dfrac{mv^{2}}{r} = \dfrac{1}{4\pi\varepsilon_{0}}\dfrac{e^{2}}{r^{2}}\]

Dopo l'applicazione del campo magnetico, l'equilibrio dell'elettrone è ottenuto mediante una sola variazione della velocità di rotazione. La variazione del raggio è influente per ordini superiori, dunque, in prima analisi, è perfettamente trascurabile. La cancellazione dei termini di ordine superiore ($\pm evdB$) dimostra che, se si ignora la variazione di raggio, la variazione di velocità indotta ($2mvdv/r$) è esattamente ciò che è richiesto dalla forza aggiuntiva di Lorentz ($evdB$) per mantenere l'equilibrio (cioè, $F_{centripeta} = F_{Lorentz} + F_{Coulomb}$).

Per i materiali diamagnetici la permeabilità relativa è circa unitaria, \(\mu_{r} \simeq 1\), con valori leggermente inferiori all'unità in modo da opporsi al campo applicato. Inoltre, essendo un fenomeno legato all'orbita elettronica, il diamagnetismo è presente in tutti i materiali.

\section{Paramagnetismo}\label{paramagnetismo}

A differenza del diamagnetismo sempre presente in ogni materiale, il paramagnetismo è presente solamente in quei materiali i cui atomi o molecole hanno un momento magnetico permanente ($\vec{\mu} \neq 0$), dovuto alla presenza di elettroni spaiati negli orbitali atomici.

Per il principio di esclusione di Pauli, gli elettroni negli orbitali atomici si dispongono con spin antiparallelo, così da cancellare gli effetti degli spin stessi. Negli atomi in cui sono presenti elettroni spaiati, ovvero con orbitali atomici non completamente riempiti, la cancellazione degli spin non avviene, dunque, lo spin netto dell'atomo è diverso da zero. 

A livello macroscopico, gli spin dei vari costituenti del materiale si sommano, dando luogo alla magnetizzazione macroscopica. Quest'ultima è influenzata dalla presenza o meno di un campo magnetico.

In assenza di un campo magnetico esterno, il momento magnetico di ogni atomo è orientato in modo casuale a causa dell'agitazione termica, dunque, la magnetizzazione macroscopica media ($\vec{M}$) del materiale è nulla.

Applicato un campo magnetico esterno, invece, i singoli momenti magnetici (spin) subiscono un momento torcente che tende ad allinearli nella direzione del campo. Essi iniziano a precedere attorno alla direzione del campo con la frequenza di Larmor:

\[\omega = \gamma B\]

\begin{figure}[ht]
\centering
\includegraphics[width=3.81482in,height=2.1701in,alt={P1604\#yIS1}]{media/3_Elettromagnetismo/image30.pdf}\caption{Spin nei materiali paramagnetici}
\end{figure}

Il paramagnetismo è presente solo in alcune sostanze ed è caratterizzato da una permeabilità magnetica relativa \(\mu_{r} \simeq 1\), lievemente maggiore dell'unità.

Si osservi che il vettore di magnetizzazione ($M$) è definito come la somma vettoriale dei momenti magnetici per unità di volume:

\[
\vec{M} = \dfrac{1}{V}\sum_{i}\vec{\mu}_i
\]

L'interazione del campo esterno con i momenti magnetici determina la magnetizzazione macroscopica $M$ del materiale, che in equilibrio termico è generalmente debole a causa del disorientamento termico.
\begin{center}
\vfill
    \chapter{Meccanica quantistica}
    \label{blx:Quantistica\therefsection}
\vfill

\minitoc
\newpage
\end{center}
\justify

\section{Crisi della Fisica Classica e Nascita della Meccanica Quantistica}
\label{crisi-meccanica-quantistica}

All'inizio del Novecento, il modello matematico basato sulla fisica classica (Meccanica Newtoniana ed Elettrodinamica di Maxwell) si rivelò inadeguato a descrivere con successo numerosi fenomeni osservati su scala microscopica. Queste \textbf{incongruenze} tra teoria ed esperimento segnarono l'inizio di una profonda revisione dei fondamenti della fisica.

La fisica classica non era in grado di spiegare in modo soddisfacente i seguenti fenomeni, che richiedevano l'introduzione della \textbf{quantizzazione}:
\begin{itemize}
\item
    \textbf{La Stabilità dell'Atomo di Rutherford}: Secondo l'elettrodinamica classica, un elettrone in orbita attorno al nucleo (in moto accelerato) dovrebbe emettere continuamente radiazione elettromagnetica, perdendo energia e spiraleggiando verso il nucleo nel giro di una frazione di secondo. Al contrario, gli atomi sono \textbf{strutture stabili};
\item
    \textbf{La Distribuzione Spettrale della Radiazione del Corpo Nero}: La teoria classica (legge di Rayleigh-Jeans) prevedeva che l'energia emessa da un corpo nero divergesse alle alte frequenze, portando a una quantità di energia infinita (la cosiddetta \textbf{catastrofe ultravioletta} o \textbf{Rayleigh-Jeans catastrophe}), in netto contrasto con i dati sperimentali. Questo problema fu risolto da Planck (1900) postulando che l'energia fosse emessa in pacchetti discreti ($E=h\nu$);
\item
    \textbf{L'Effetto Fotoelettrico}: La luce incidente su una superficie metallica può liberare elettroni solo se la sua frequenza supera un valore soglia specifico, indipendentemente dall'intensità della luce stessa. La teoria ondulatoria classica non poteva spiegare questa dipendenza dalla frequenza. Il fenomeno fu spiegato da Einstein (1905) introducendo il concetto di \textbf{fotone} (quanto di luce).
\end{itemize}

Parallelamente o immediatamente dopo la risoluzione di questi problemi, una serie di esperimenti fornirono ulteriori e definitive prove a sostegno della necessità di una nuova teoria quantistica:

\begin{itemize}
\item
    \textbf{Spettri di Emissione e Assorbimento Atomico}: Gli atomi emettono o assorbono luce solo a \textbf{frequenze discrete} e ben definite, anziché su uno spettro continuo. Questo risultato fu la prova diretta che l'energia degli elettroni all'interno dell'atomo è \textbf{quantizzata} in livelli specifici (modello di Bohr, 1913);
\item
    \textbf{Esperimento di Stern-Gerlach (1922)}: Un fascio di atomi (come l'argento) fatto passare attraverso un campo magnetico non uniforme si divide in un numero finito e discreto di componenti (storicamente, due). Ciò dimostrò la \textbf{quantizzazione spaziale} del momento angolare e, successivamente, l'esistenza dello \textbf{spin} (momento angolare intrinseco) dell'elettrone, che non ha alcun analogo classico;
\item
    \textbf{Diffrazione di Elettroni e Doppio Slit (Davisson-Germer, 1927)}: L'osservazione che particelle come gli elettroni producano figure di interferenza tipiche delle onde confermò la \textbf{dualità onda-particella} proposta da De Broglie (1924), evidenziando che le particelle microscopiche si comportano come onde in determinate circostanze.
\end{itemize}

Queste difficoltà e scoperte portarono alla necessità di rivedere i fondamenti della fisica. Nacque così la \textbf{Meccanica Quantistica}, una nuova teoria che descrive il comportamento della materia e dell'energia su scala microscopica, introducendo concetti fondamentali come la \textbf{quantizzazione}, la \textbf{probabilità} e la \textbf{dualità onda-particella} \cite{messiah1961quantum, wichmann1971quantistica, feynman1965vol3}.

\subsection{Stabilità dell'Atomo e Collasso Elettronico (Critica al Modello di Rutherford)}\label{stabilituxe0-dellelettrone}

Il problema della stabilità atomica rappresenta una delle incongruenze più acute tra teoria classica e osservazione.
Secondo l'elettrodinamica classica, formalizzata dalle equazioni di Maxwell, una carica elettrica in \textbf{moto accelerato} (come l'elettrone in orbita attorno al nucleo nel modello planetario di Rutherford) dovrebbe emettere continuamente energia sotto forma di \textbf{radiazione elettromagnetica}. 

Questo rilascio continuo di energia implicherebbe una rapida \textbf{perdita di energia cinetica} da parte dell'elettrone e una conseguente riduzione del raggio orbitale, portando al \textbf{collasso} dell'elettrone sul nucleo in un tempo stimato di circa $10^{-11}$ secondi.

Tuttavia, l'evidenza sperimentale dimostra che gli atomi sono strutture intrinsecamente \textbf{stabili}, che esistono indefinitamente nel tempo senza collassare. Questa lampante discrepanza richiese l'abbandono dei principi classici per la descrizione della materia a livello microscopico.

La prima risoluzione teorica di questa crisi fu proposta da \textbf{Niels Bohr nel 1913}, attraverso l'introduzione di \textbf{postulati quantistici}: l'elettrone può occupare solo determinate \textbf{orbite stazionarie} (o stati quantici) in cui, per definizione, \textbf{non emette alcuna radiazione}. L'emissione o l'assorbimento di energia avviene solo durante la transizione tra questi livelli discreti, sotto forma di quanti di energia (fotoni), in accordo con la relazione di Planck ($E = h\nu$).

\subsection{Radiazione del Corpo Nero e la Quantizzazione di Planck}\label{corpo-nero}

Il \textbf{corpo nero} è un oggetto ideale che assorbe completamente tutta la radiazione elettromagnetica incidente, senza rifletterla né trasmetterla. Di conseguenza, per la conservazione dell'energia, quando riscaldato, emette radiazione termica a ogni lunghezza d'onda, e la distribuzione di questa energia è funzione esclusiva della sua temperatura.

Sebbene il corpo nero sia un'astrazione teorica, può essere approssimato sperimentalmente da una cavità con pareti interne nere e una piccola apertura. La radiazione che entra ha una probabilità estremamente bassa di uscire, simulando il comportamento ideale.

\begin{figure}[ht]
\centering
\resizebox{0.5\textwidth}{!}{%
\begin{tikzpicture}[scale=1.3]

  \shade[top color=black!60,bottom color=black!80,shading angle=10] % background
    (7:1) arc (7:355:1);
  
  \fill[thick,black,postaction=decorate, % rough inner surface
    decoration={markings,mark=between positions 0.55 and 1 step 0.03 with {
                  \node[transform shape,inner sep=1pt]
                  (hit\pgfkeysvalueof{/pgf/decoration/mark info/sequence number}) {};
    }}]
    (7:1) arc (7:353:1) --++ (-7:-0.18)
    decorate[decoration={random steps,segment length=2,amplitude=1pt}]
        {arc (-7:-353:0.82)} -- cycle;

% Raggi che entrano e colpiscono la circonferenza
\draw[red,thick,->] (2.3,0) -- (0.9,0); % ingresso
\draw[red,thick,->] (0.9,0) -- (-0.1,-0.8); % primo rimbalzo
\draw[red,thick,->] (-0.1,-0.8) -- (0.6,0.5); % secondo rimbalzo
\draw[red,thick,->] (0.6,0.5) -- (-0.8,-0.1); % terzo rimbalzo
\draw[red,thick,->] (-0.8,-0.1) -- (-0.2,-0.51); % quarto rimbalzo

% Etichetta raggio
\node[red] at (3.5,0) {Raggio incidente};

\end{tikzpicture}
}
\caption{Modello pratico di corpo nero, come una cavità con una piccola apertura}
\label{fig:4_CorpoNero}
\end{figure}


Secondo la fisica classica, in particolare la teoria basata sulla legge di \textbf{Rayleigh-Jeans}, l'intensità della radiazione emessa per unità di frequenza ($I(\nu, T)$) dovrebbe aumentare indefinitamente con l'aumento della frequenza ($\nu$). Questa previsione teorica, che prevedeva una divergenza dell'energia alle alte frequenze, è nota come \textbf{catastrofe ultravioletta} e contraddiceva clamorosamente i risultati sperimentali.

\begin{figure}[ht]
\centering
\resizebox{0.5\textwidth}{!}{%

% BLACK BODY - 3000, 4000, 5000K, Wien's displacement law
\begin{tikzpicture}
% redraw axis on top
\makeatletter \newcommand{\pgfplotsdrawaxis}{\pgfplots@draw@axis} \makeatother
\pgfplotsset{axis line on top/.style={after end axis/.append code={\pgfplotsdrawaxis}}
}

% CUSTOM COLORS
% See https://tikz.net/blackbody_color/
\definecolor{1000K}{rgb}{1,0.0337,0}
\definecolor{2000K}{rgb}{1,0.2647,0.0033}
\definecolor{3000K}{rgb}{1,0.4870,0.1411}
\definecolor{4000K}{rgb}{1,0.6636,0.3583}
\definecolor{5000K}{rgb}{1,0.7992,0.6045}
\definecolor{6000K}{rgb}{1,0.9019,0.8473}
\definecolor{8000K}{rgb}{0.7874,0.8187,1}
\definecolor{10000K}{rgb}{0.6268,0.7039,1}
\pgfdeclareverticalshading{rainbow}{100bp}{
  color(0bp)=(red); color(25bp)=(red); color(35bp)=(yellow);
  color(45bp)=(green); color(55bp)=(cyan); color(65bp)=(blue);
  color(75bp)=(violet); color(100bp)=(violet)
}
\colorlet{myred}{red!70!black}
\colorlet{mygreen}{green!70!black}
\colorlet{mydarkgreen}{green!55!black}

% PLANCK & RAYLEIGH-JEANS
% 2hc^2/lambda^5 = 2 * 6.62607015e-34 * 299792458^2
%                = 1.191042972e-16
%    W.m -> kW.nm: 1.191042972e26
%  hc/k lambda T = 6.62607015e-34*299792458/(1.38064852e-23)
%                = 0.01438777378
%         m -> nm: 0.01438777378e9
% 2ckT/lambda^4  = 2 * 299792458 * 1.38064852e-23
%                = 8.278160269e-15
%    W.m -> kW.nm: 8.278160269e18
\pgfmathdeclarefunction{planck}{2}{%
  \pgfmathparse{1.191042972e26/(#1^5)/(exp(0.01439e9/(#1*#2))-1)}%
}
\pgfmathdeclarefunction{rayleighjeans}{2}{%
  \pgfmathparse{8.278160269e18*#2/(#1^4)}%
}
\pgfmathdeclarefunction{wien}{2}{%
  \pgfmathparse{1.191042972e26/(#1^5)*exp(-0.01439e9/(#1*#2))}%
}
\pgfmathdeclarefunction{lampeak}{1}{% % Wien's displacement law
  \pgfmathparse{2.898e6/#1}%
}
  \message{^^JBlack body, Wien's displacement law}
  \def\N{60}
  \def\xmax{3100}
  \def\ymax{1.36e10}
  \def\tick#1#2{\draw[thick] (#1+.01*\ymax) -- (#1-.01*\ymax) node[below=-.5pt,scale=0.75] {#2};}
  \begin{axis}[
      every axis plot/.style={
        very thick,mark=none,samples=\N,domain=5:\xmax,smooth},
      xmin=(-.05*\xmax), xmax=(1.05*\xmax),
      ymin=(-.08*\ymax), ymax=(1.08*\ymax),
      restrict y to domain=0:\ymax,
      axis lines=middle,
      axis line style=thick,
      %enlargelimits=upper, % extend the axes a bit to the right and top
      tick style={black,thick},
      ticklabel style={scale=0.8},
      %xtick style={draw=none},xticklabels=none,
      max space between ticks=26,
      xlabel={Wavelength $\lambda$ [nm]},
      ylabel={Power $P$ [kW/sr\,m$^2$\,nm]},
      xlabel style={at={(rel axis cs:0.5,0)},below=-1pt,font=\small},
      ylabel style={at={(rel axis cs:-0.11,0.5)},rotate=90},
      width=9cm, height=7cm,
      %clip=false
      tick scale binop=\times,
      every y tick scale label/.style={at={(rel axis cs:0,1)},anchor=south}]
    ]
    
    % RAINBOW
    \shade[shading=rainbow,shading angle=90,opacity=0.5] (380,0) rectangle (740,\ymax);
    \node[above=-1pt,scale=0.8] at (200,\ymax) {\strut UV}; % 10 - 400 nm
    \node[above=-1pt,scale=0.8] at (570,\ymax) {\strut optical}; % 380 - 740 nm
    \node[above=-1pt,scale=0.8] at (920,\ymax) {\strut IR}; % 740 - 1050 nm
    
    % PLANCK
    \addplot[red]    {planck(x,3000)};
    \addplot[orange] {planck(x,4000)};
    \addplot[blue,samples=3*\N] {planck(x,5000)};
    \addplot[dashed,thick,blue,domain=1000:3500] {rayleighjeans(x,5000)};
    
    % MAXIMUM (Wien's displacement law)
    \addplot[mydarkgreen,thick,variable=T,domain=2200:4000,samples=40]
      ({lampeak(T)},{planck(lampeak(T),T)});
    \addplot[mydarkgreen,thick,variable=T,domain=4000:5200,samples=100]
      ({lampeak(T)},{planck(lampeak(T),T)});
    \fill[mydarkgreen!80!black] ({lampeak(3000)},{planck(lampeak(3000),3000)}) circle(1.5pt);
    \fill[mydarkgreen!80!black] ({lampeak(4000)},{planck(lampeak(4000),4000)}) circle(1.5pt);
    \fill[mydarkgreen!80!black] ({lampeak(5000)},{planck(lampeak(5000),5000)}) circle(1.5pt);
    
    % LABELS
    \node[above=0pt,scale=0.75,red] at (1150,{planck(1150,3000)}) {\SI{3000}{K}};
    \node[above right=-1pt,scale=0.75,orange!80!black] at (740,{planck(740,4000)}) {\SI{4000}{K}};
    \node[above right=-1pt,scale=0.75,blue] at (800,{planck(800,5000)}) {\SI{5000}{K}};
    \node[above right=-1pt,scale=0.75,blue] at (1500,{rayleighjeans(1500,5000)}) {\SI{5000}{K} Rayleigh-Jeans};
    \node[above right=-1pt,scale=0.75,blue] at (1600,{rayleighjeans(1430,5000)}) {Ultraviolet Catastrophe};
    
  \end{axis}
\end{tikzpicture}

}
\caption{Confronto tra la previsione classica (Rayleigh-Jeans) e i dati sperimentali che mostrano la catastrofe ultravioletta}
\label{fig:4_CatastrofeUV}
\end{figure}

In realtà, lo spettro di emissione del corpo nero dipende strettamente dalla temperatura, presentando un andamento continuo e finito: l'intensità raggiunge un massimo a una specifica lunghezza d'onda, che si sposta verso valori più corti all'aumentare della temperatura (Legge dello spostamento di Wien), per poi decrescere.

La risoluzione teorica di questo problema fondamentale fu proposta da \textbf{Max Planck nel 1900}, segnando la nascita della meccanica quantistica. Planck ipotizzò che l'energia scambiata tra gli oscillatori atomici delle pareti della cavità e la radiazione non fosse continua, ma avvenisse in pacchetti discreti o \textit{quanti} di energia.

L'energia emessa o assorbita è quindi un multiplo intero ($n$) di una quantità elementare, proporzionale alla frequenza ($\nu$) della radiazione:

\[
E = h\nu
\]

dove $E$ è l'energia del quanto (per $n=1$), $\nu$ è la frequenza della radiazione, e $h$ è la \textbf{costante di Planck}. Questa ipotesi permise a Planck di derivare una formula che descriveva correttamente l'intero spettro osservato.

\subsection{Effetto fotoelettrico}\label{effetto-fotoelettrico}

L'effetto fotoelettrico è un fenomeno di interazione tra la radiazione elettromagnetica e la materia, caratterizzato dall'emissione di elettroni (detti \textbf{fotoelettroni}) da una superficie, generalmente metallica, quando irradiata da energia elettromagnetica. 

Sperimentalmente, l'effetto si osserva ponendo due elettrodi metallici all'interno di un'ampolla sotto vuoto spinto. Quando l'elettrodo (catodo) viene irradiato, gli elettroni espulsi generano una \textbf{corrente elettrica} rilevabile sull'elettrodo opposto (anodo).

\begin{figure}[ht]
\centering
\includegraphics[width=2.98611in,height=2.19407in,alt={P1636\#yIS1}]{media/4_Quantiatica/image33.pdf}\caption{Circuito per l'effetto fotoelettrico}
\end{figure}

Gli elettroni emessi dall'elettrodo irradiato, dotati di energia cinetica non nulla, vengono accelerati dal campo elettrico generato da una batteria, raggiungendo il secondo elettrodo collegato al polo positivo. In questo modo, si chiude il circuito elettrico e si rileva una corrente.

L'energia cinetica massima ($K_{\max}$) dei fotoelettroni può essere misurata applicando una \textbf{tensione di arresto} ($V_s$): una differenza di potenziale opposta e sufficiente ad annullare la corrente, impedendo anche agli elettroni più energetici di raggiungere l'anodo. L'energia cinetica massima è quindi data da $K_{\max} = e V_s$.

Si applica una differenza di potenziale tale da bloccare la circolazione di corrente nel circuito, nonostante l'irraggiamento dell'elettrodo. In questa situazione, un elettrone, espulso per effetto fotoelettrico con una certa energia cinetica, si trova in un campo elettrico che lo decelera fino a farlo tornare sull'elettrodo che l'ha prodotto. Nota la tensione applicata, si riesce a determinare l'energia dell'elettrone espulso.

Secondo la teoria ondulatoria classica, l'energia trasferita agli elettroni e, di conseguenza, la loro energia cinetica massima ($K_{\max}$), avrebbe dovuto:
\begin{enumerate}
    \item Dipendere dall'\textbf{intensità} della radiazione (una luce più brillante avrebbe dovuto liberare elettroni più energetici).
    \item Manifestarsi dopo un certo \textbf{ritardo temporale} per accumulare l'energia necessaria.
\end{enumerate}
L'esperimento smentì entrambe le previsioni: l'emissione è \textbf{istantanea} e $K_{\max}$ dipende esclusivamente dalla \textbf{frequenza} ($\nu$) della radiazione incidente, e non dalla sua intensità. Inoltre, esiste una \textbf{frequenza di soglia} ($\nu_0$) al di sotto della quale l'emissione non avviene affatto, anche ad alta intensità.


Albert Einstein risolse questa incongruenza nel 1905, estendendo l'ipotesi di Planck: la luce non si comporta solo come un'onda, ma è composta da particelle discrete, i \textbf{fotoni} (quanti di luce), ciascuno con energia:

\[
E = h\nu
\]

dove $h$ è la costante di Planck e $\nu$ è la frequenza. L'emissione di un elettrone avviene in un processo di interazione \textbf{uno-a-uno} tra un singolo fotone e un singolo elettrone.

L'equazione fondamentale che descrive l'effetto fotoelettrico è:

$$K_{\max} = E - \phi = h\nu - \phi$$

dove $K_{\max}$ è l'energia cinetica massima dell'elettrone espulso, $h\nu$ è l'energia del fotone incidente, e $\mathbf{\phi}$ (o $W_0$) è la \textbf{funzione lavoro} (l'energia minima necessaria per estrarre l'elettrone dalla superficie metallica).

Questa spiegazione, che attribuiva alla luce una \textbf{natura corpuscolare} (dualità onda-particella), valse ad Einstein il Premio Nobel per la Fisica nel 1921.

\subsection{L'Esperimento della Doppia Fenditura e la Dualità Onda-Particella}\label{double-slit}

Mentre la crisi della fisica classica era iniziata con la natura quantizzata della luce (fotoni, 1905), l'esperimento della \textbf{doppia fenditura} (o \textbf{double slit}) ha esteso il dualismo alla \textbf{materia}, rivelando la natura ondulatoria di particelle come gli elettroni. Questo risultato ha messo definitivamente in crisi la visione strettamente corpuscolare della materia.

Nell'esperimento cruciale, un fascio di elettroni (o altre particelle) viene inviato contro una parete dotata di due fenditure parallele. Sullo schermo di rilevazione posto dietro le fenditure, anziché osservare le due bande nette previste per le particelle classiche, si forma una \textbf{figura di interferenza}: una sequenza di frange chiare (interferenza costruttiva) e scure (interferenza distruttiva), un comportamento tipico ed esclusivo dei fenomeni ondulatori.

\begin{figure}[ht]
\centering
\includegraphics[width=3.02778in,height=2.29079in,alt={P1647\#yIS1}]{media/4_Quantiatica/image34.pdf}\caption{Figura di interferenza della doppia fenditura}
\end{figure}

Il comportamento degli elettroni in questo esperimento può essere interpretato attraverso il principio di Huygens dei fenomeni ondulatori, secondo cui ogni punto \(d\Sigma\) di un fronte d'onda \(\Sigma\) può essere considerato come sorgente secondaria di onde sferiche. La perturbazione risultante in un punto dello spazio è data dalla sovrapposizione di tutte le onde secondarie che vi giungono.

Il risultato è sorprendente perché, anche inviando gli elettroni \textbf{uno alla volta}, il pattern di interferenza si accumula gradualmente, suggerendo che ogni particella interferisca in qualche modo \textbf{con sé stessa}.

Gli elettroni sono particelle dotate di massa, infatti, in altri esperimenti, come l'effetto fotoelettrici, gli elettroni interagiscono con i fotoni come particelle, attraverso urti elastici. Questi due comportamenti apparentemente contraddittori (particella in urti elastici come nell'effetto fotoelettrico; onda nella doppia fenditura) hanno portato alla formulazione del concetto di \textbf{dualismo onda-corpuscolo}.

Nel 1924, \textbf{Louis de Broglie} propose l'ipotesi che a ogni particella materiale, con quantità di moto $p$, fosse associata un'onda, definita \textbf{onda di materia}, la cui lunghezza d'onda ($\lambda$) è inversamente proporzionale alla quantità di moto stessa:

\[
\lambda = \dfrac{h}{p}
\]

dove $h$ è la costante di Planck. Questa ipotesi teorica fu confermata sperimentalmente nel 1927 dagli esperimenti di \textbf{Davisson e Germer} e da \textbf{G.P. Thomson} sulla diffrazione di elettroni, che rappresenta uno dei pilastri fondamentali della meccanica quantistica.

\subsection{Esperimento di Stern e Gerlach: La Quantizzazione dello Spin}\label{esperimento-di-stern-e-gerlach}

L'esperimento di Stern e Gerlach, condotto nel 1922, ha fornito una delle prove sperimentali più dirette e convincenti della \textbf{quantizzazione del momento angolare} e ha rivelato l'esistenza di una proprietà intrinseca delle particelle: lo \textbf{spin}.

Nel setup sperimentale, un fascio di atomi d'argento (il cui momento angolare deriva quasi interamente dall'unico elettrone nel guscio esterno) viene fatto passare attraverso un \textbf{campo magnetico fortemente non uniforme}. 

Secondo la fisica classica, il momento magnetico degli atomi, orientandosi in modo continuo rispetto al campo, avrebbe dovuto produrre un'unica linea o una \textbf{distribuzione continua} sullo schermo di rilevazione.

Tuttavia, ciò che si osserva è una \textbf{separazione netta} del fascio in \textbf{due sole componenti distinte}. Questo risultato indica inequivocabilmente che il momento magnetico degli atomi, e quindi il loro momento angolare totale $L$, non può assumere valori arbitrari (come previsto dalla classica), ma solo un insieme finito e discreto di orientamenti spaziali.

In particolare, per gli atomi d'argento, la separazione in due componenti dimostra la quantizzazione dello \textbf{spin} dell'elettrone, per cui la sua proiezione lungo la direzione del campo può assumere solo due valori: $m_s = +1/2$ (spin "su") e $m_s = -1/2$ (spin "giù").

Questo esperimento ha dimostrato che:

\begin{itemize}
\item Il momento angolare (e il momento magnetico associato) è \textbf{quantizzato} spazialmente (quantizzazione spaziale).
\item Lo \textbf{spin} è una proprietà quantistica intrinseca e fondamentale delle particelle subatomiche, priva di analogo classico.
\item Le misure di grandezze quantistiche non danno risultati continui, ma solo \textbf{valori discreti} (autovalori), confermando il carattere non deterministico e non continuo della realtà microscopica.
\end{itemize}

L'esperimento di Stern-Gerlach è un pilastro della meccanica quantistica, poiché mostra in modo diretto la discrezione degli stati quantici.

\subsection{Spettri di Assorbimento e di Emissione: Evidenza della Quantizzazione Atomica}\label{spettro-di-assorbimento}

Lo \textbf{spettro di assorbimento} di un elemento chimico è l'insieme delle radiazioni elettromagnetiche assorbite quando la sostanza viene esposta a una sorgente luminosa a spettro continuo. Le sostanze assorbono energia solamente a determinate frequenze specifiche, dando origine a uno \textbf{spettro a righe scure} caratteristico.

Nei primi anni del Novecento, l'incapacità della meccanica classica di prevedere questi spettri a righe rappresentava un'ulteriore prova della sua inadeguatezza. Secondo la fisica classica, un atomo dovrebbe assorbire o emettere radiazioni su tutte le frequenze. Invece, l'osservazione che l'assorbimento avvenisse solo a frequenze discrete, indipendentemente dall'intensità, suggeriva l'esistenza di un meccanismo interno \textbf{quantizzato}.

Lo spettro risultante, osservato a valle dell'esposizione, è dato dalla radiazione incidente privata delle lunghezze d'onda assorbite dal materiale. In altre parole, lo spettro presenta delle bande scure, corrispondenti alle frequenze assorbite.

\begin{figure}[ht]
\centering
\includegraphics[width=3.34722in,height=2.81542in,alt={P1658\#yIS1}]{media/4_Quantiatica/image35.pdf}\caption{Spettro di assorbimento e di emissione}
\end{figure}

Già prima della formulazione teorica, la regolarità degli spettri atomici fu descritta empiricamente. In particolare, per l'atomo di idrogeno, \textbf{Johannes Rydberg} propose una formula per calcolare le lunghezze d'onda delle righe spettrali (sia di assorbimento che di emissione):

\[
\dfrac{1}{\lambda} = R_{H}\left( \dfrac{1}{n_{1}^{2}} - \dfrac{1}{n_{2}^{2}} \right)
\]

dove $R_H$ è la costante di Rydberg e $n_1, n_2$ sono \textbf{numeri interi positivi} ($n_1, n_2 \in \mathbb{N}$, con $n_2 > n_1$). La dipendenza da numeri interi era un indizio potentissimo dell'esistenza di \textbf{livelli energetici discreti} all'interno dell'atomo.

Sebbene la formula di Rydberg fosse predittiva, non ne spiegava il fondamento fisico. La spiegazione teorica arrivò nel 1913 con il \textbf{modello atomico di Bohr}. Bohr postulò che:

\begin{itemize}
    \item L'energia dell'elettrone è \textbf{quantizzata} e confinata in livelli specifici ($E_n$);
    \item L'assorbimento (o l'emissione) di radiazione avviene solo quando l'elettrone compie una \textbf{transizione quantica} tra due di questi livelli discreti, con la frequenza del fotone emesso/assorbito data dalla differenza di energia tra i livelli: $\nu = (E_2 - E_1)/h$.
\end{itemize}

Il modello di Bohr fornì la prima giustificazione fisica della formula di Rydberg, legando i numeri interi $n_1$ e $n_2$ ai numeri quantici dei livelli energetici stazionari dell'atomo.

\section{Teoria della meccanica quantistica}\label{teoria-della-meccanica-quantistica}

La Meccanica Quantistica (MQ) costituisce il quadro teorico fondamentale per la descrizione del comportamento della materia e dell'energia su scala microscopica. In fisica teorica, la MQ viene sviluppata attraverso due principali formulazioni:

\begin{itemize}
    \item La MQ \textbf{non relativistica}: sufficiente per descrivere la struttura degli atomi, le molecole e la dinamica delle particelle che si muovono a velocità ben inferiori a quella della luce ($v \ll c$). Questa modellazione si basa sull'equazione di Schrödinger.
    \item La MQ \textbf{relativistica}: necessaria per trattare particelle che si muovono a velocità prossime a $c$ e per le interazioni ad alta energia, un campo noto come \textbf{Teoria Quantistica dei Campi} (QFT).
\end{itemize}

La teoria quantistica relativistica più celebre e di successo è l'\textbf{Elettrodinamica Quantistica} (QED, \textit{Quantum Electrodynamics}). Essa descrive con straordinaria precisione le interazioni elettromagnetiche, modellando il comportamento dei fotoni (mediatori del campo) e le loro interazioni con le particelle cariche, come gli elettroni. 

\begin{figure}[h!]
    \centering
    \resizebox{1\textwidth}{!}{
        \begin{tikzpicture}[>=stealth, thick]

%--- Direct Channel ---
% Fermions in ingresso
\node[left] (a1) at (-2,1) {$e^-$};
\node[left] (a2) at (-2,-1) {$e^-$};

% Fermions in uscita
\node[right] (b1) at (2,1) {$e^-$};
\node[right] (b2) at (2,-1) {$e^-$};

% Vertici
\node (v1) at (0,1) {};
\node (v2) at (0,-1) {};

% Linee fermioniche
\draw[->] (a1) -- (v1);
\draw[->] (a2) -- (v2);
\draw[->] (v1) -- (b1);
\draw[->] (v2) -- (b2);

% Fotone
\draw[decorate, decoration={snake}] (v1) -- node[right] {$\gamma$} (v2);

% Titolo
\node at (0,-2.2) {Direct Channel};

\end{tikzpicture}
\hspace{2cm}
\begin{tikzpicture}[>=stealth, thick]

%--- Exchange Channel ---
% Fermions in ingresso
\node[left] (c1) at (-2,1) {$e^-$};
\node[left] (c2) at (-2,-1) {$e^-$};

% Fermions in uscita
\node[right] (d1) at (2,1) {$e^-$};
\node[right] (d2) at (2,-1) {$e^-$};

% Vertici
\node (v3) at (0,1) {};
\node (v4) at (0,-1) {};

% Linee fermioniche con scambio
\draw[->] (c1) -- (v3);
\draw[->] (c2) -- (v4);
\draw[->] (v3) -- (d2);
\draw[->] (v4) -- (d1);

% Fotone
\draw[decorate, decoration={snake}] (v3) -- node[right] {$\gamma$} (v4);

% Titolo
\node at (0,-2.2) {Exchange Channel};

\end{tikzpicture}

    }
    \caption{Diagrammi di Feynman per lo scattering elettrone-elettrone (Møller). Il diagramma a sinistra rappresenta il canale diretto, e il diagramma a destra rappresenta il canale di scambio.}
    \label{fig:electron_electron_scattering_arranged}
\end{figure}

Una delle previsioni cruciali della formulazione relativistica della MQ (in particolare, l'equazione di Dirac del 1928, fondamento della QED) è l'esistenza delle \textbf{antiparticelle}, entità con carica e numeri quantici opposti rispetto alla materia ordinaria. Un esempio è il \textbf{positrone} ($e^{+}$), l'antiparticella dell'elettrone, scoperto sperimentalmente nel 1932.

La Meccanica Quantistica non relativistica si è sviluppata a partire dai contributi pionieristici di:
\begin{itemize}
    \item \textbf{Max Planck} (1900): con l'ipotesi della quantizzazione dell'energia ($E=h\nu$) per spiegare la radiazione del corpo nero.
    \item \textbf{Albert Einstein} (1905): con l'interpretazione quantistica dell'effetto fotoelettrico, introducendo il concetto di fotone.
    \item \textbf{Louis de Broglie} (1924): con la proposta della dualità onda-corpuscolo per la materia.
\end{itemize}

Questi contributi hanno posto le basi per la formulazione moderna della MQ, una teoria che abbandona il determinismo classico per descrivere il comportamento delle particelle microscopiche in termini \textbf{probabilistici}, introducendo concetti come la \textbf{funzione d’onda} ($\Psi$), il \textbf{principio di indeterminazione} (Heisenberg) e la \textbf{quantizzazione degli stati energetici} \cite{dirac1930principles, messiah1961quantum, wichmann1971quantistica, feynman1965vol3, landau1975quantistica_rel}.

\subsection{Quantizzazione della materia}\label{quantizzazione-della-materia}

L'ipotesi fondamentale della meccanica quantistica è che le grandezze fisiche a livello microscopico siano \textbf{quantizzate}: esse non variano in modo continuo, ma possono assumere solo \textbf{valori discreti} (autovalori), spesso multipli interi di una quantità fondamentale. Questo principio, inizialmente suggerito da Planck per l'energia della radiazione e da Einstein per il fotone, è cruciale per descrivere la struttura della materia.

Gli elettroni all'interno di un atomo occupano \textbf{livelli energetici stazionari} ($E_1, E_2, \ldots$) che rappresentano stati quantici definiti. Poiché gli elettroni possono occupare solo questi livelli quantizzati, non sono soggetti al collasso sul nucleo (come previsto dalla fisica classica), garantendo la \textbf{stabilità dell'atomo}.

Un fotone incidente sull'atomo può essere assorbito dall'elettrone solo se la sua energia è \textbf{esattamente uguale} alla differenza di energia tra due livelli ammissibili ($E_f$ e $E_i$):

\[
E_{\text{fotone}} = E_f - E_i
\]

L'energia del fotone è data dalla relazione di Planck:

\[
E = h\nu
\]

dove $E$ è l'energia del fotone, $\nu$ è la sua frequenza, e $h$ è la costante di Planck. Da questa condizione deriva il fenomeno degli \textbf{spettri a righe}: l'atomo assorbe o emette radiazione solo alle frequenze $\nu$ ben definite, come dimostrato dalla formula empirica di Rydberg.

Se il fotone viene assorbito, l'elettrone subisce una transizione (eccitazione) a un livello energetico superiore ($E_f$). Successivamente, l'elettrone può ritornare a un livello inferiore (rilassamento), emettendo un fotone con un'energia pari alla differenza energetica tra i due stati.

\subsection{Dualismo Onda-Particella (Ipotesi di De Broglie)}\label{dualismo-onda-particella} 
 
L'ipotesi di \textbf{Louis de Broglie}, formulata nel 1924, ha esteso il dualismo onda-corpuscolo (già riconosciuto per la luce) alla \textbf{materia}. Secondo questa ipotesi, a ogni particella materiale, come l'elettrone, è associata un'onda (l'\textbf{onda di materia}), descritta da una \textbf{funzione d'onda} $\Psi(\vec{r}, t)$, che racchiude le informazioni sullo stato quantico della particella. 
 
Le relazioni di De Broglie stabiliscono il legame quantitativo tra le proprietà corpuscolari (Energia $E$, Quantità di moto $p$) e le proprietà ondulatorie (Frequenza $\nu$, Lunghezza d'onda $\lambda$). 
 
L'energia ($E$) della particella è legata alla frequenza ($\nu$) e alla pulsazione ($\omega$) dell'onda associata dalla relazione di Planck: 

\[
E = h\nu
\]

dove \(E\) è l'energia associata all'onda, \(\nu\) è la frequenza, e \(h\) è la costante di Planck. La pulsazione è legata alla frequenza da:

\[
\omega = 2\pi\nu
\]

Sostituendo nella formula dell'energia, si ottiene:

\[
E = \dfrac{h}{2\pi}\omega = \hslash\omega
\]

dove:

\[
\hslash = \dfrac{h}{2\pi}
\]

è la \textbf{costante di Planck ridotta}. 

Alla particella è anche associata una quantità di moto \(p\). Per l'ipotesi di de Broglie, la quantità di moto ($p$) è legata al \textbf{numero d'onda} ($k$) e alla lunghezza d'onda ($\lambda$): 

\[
p = \hslash k
\]

dove \(k\) è il \textbf{numero d'onda} definito come:

\[
k = \dfrac{2\pi}{\lambda}
\]

Sostituendo la definizione di $\hslash$ e $k$, si ottiene la relazione fondamentale che quantifica il dualismo:

\[
\lambda = \dfrac{h}{p}
\]

Questa formula dimostra che maggiore è la quantità di moto della particella, minore è la lunghezza d'onda associata, spiegando perché gli effetti ondulatori non siano osservabili su scala macroscopica. 
 
Mentre le relazioni di De Broglie si applicano a tutte le particelle, il loro contesto energetico è definito dalla relatività. Per una particella con massa a riposo $m_0$, la relazione completa tra energia e quantità di moto è: 

\[
E^2 = (pc)^2 + (m_0c^2)^2
\]

In \textbf{regime non relativistico} ($v \ll c$), l'energia cinetica domina, e la relazione $\lambda = h/p$ è l'approssimazione valida. Il \textbf{fotone} ($m_0=0$) rappresenta il limite estremo, dove la relazione $p = h/\lambda$ è ottenuta direttamente dalle leggi dell'elettromagnetismo e di Planck ($p = E/c = h\nu/c$). 
 
Queste relazioni confermano in modo definitivo il \textbf{dualismo onda-corpuscolo} alla base della meccanica quantistica. 

\subsection{Funzione d'onda associata alla particella}\label{funzione-donda-associata-alla-particella}

L'ipotesi di de Broglie stabilisce che a ogni particella (elettrone, protone, ecc.) è associata una \textbf{funzione d'onda} $\Psi(\vec{r}, t)$, che ne descrive lo stato quantico. Nel caso più semplice di una \textbf{particella libera} in moto (ad esempio, lungo l'asse $x$), la funzione d'onda assume la forma di un'onda piana, espressa mediante un esponenziale complesso:

\[
\Psi(x,t) = A \exp\left( j(kx - \omega t) \right)
\]

dove $A$ è l'ampiezza dell'onda, \(k\) è il numero d'onda, \(\omega\) è la pulsazione, e \(j\) è l'unità immaginaria (\(j^2 = -1\)). Una qualsiasi forma d'onda può essere ottenuta come sovrapposizione di infinite onde piane.

I parametri \(k\) (numero d'onda) e \(\omega\) (pulsazione) possono essere espressi in funzione delle grandezze fisiche della particella:

\[
k = \dfrac{p}{\hslash} \quad \text{e} \quad \omega = \dfrac{E}{\hslash}
\]

Sostituendo i parametri ondulatori ($k$ e $\omega$) con le grandezze fisiche della particella (quantità di moto $p$ ed Energia $E$) tramite la relazione di De Broglie ($k = p/\hslash$ e $\omega = E/\hslash$), si ottiene:

\[
\Psi(x,t) = A \exp\left( \dfrac{j}{\hslash}\left(px - Et\right) \right)
\]

Nel caso tridimensionale, l'espressione è generalizzata utilizzando il prodotto scalare tra il vettore quantità di moto $\vec{p}$ e il vettore posizione $\vec{r}$:

\[
\Psi(\vec{r},t) = A \exp\left( \dfrac{j}{\hslash} \left(\vec{p} \cdot \vec{r} - Et\right) \right)
\]

Secondo l'\textbf{interpretazione di Born}, la funzione d'onda $\Psi(\vec{r}, t)$ non ha un significato fisico diretto, ma il suo modulo quadro, $|\Psi(\vec{r}, t)|^2$, rappresenta la \textbf{densità di probabilità} $P(\vec{r}, t)$ di trovare la particella nella posizione $\vec{r}$ al tempo $t$:

\[
P(\vec{r},t) = \left| \Psi(\vec{r},t) \right|^{2}
\]

Nel caso specifico dell'onda piana, la probabilità risulta uniforme in tutto lo spazio ($|\Psi|^2 = |A|^2 = \text{costante}$). Tuttavia, tale funzione si estende all'infinito e non è normalizzabile, pertanto non rappresenta una situazione fisica realistica di una particella localizzata. Per descrivere particelle localizzate, si utilizza il concetto di \textbf{pacchetti d’onda}, ottenuti come sovrapposizione di infinite onde piane con diverse lunghezze d'onda (e quindi diverse quantità di moto).

In questo contesto probabilistico, il concetto classico di orbita (inteso come traiettoria deterministica) non è più applicabile. La regione dello spazio in cui la probabilità di trovare l'elettrone è massima viene definita \textbf{orbitale} (per gli stati legati negli atomi).

\section{Principio di indeterminazione di Heisenberg}\label{principio-di-indeterminazione-di-heisenberg}

Il principio di indeterminazione di Heisenberg è coerente con l’interpretazione probabilistica della meccanica quantistica proposta da Born. Secondo questo principio, non è possibile conoscere con precisione e simultaneamente la posizione e la quantità di moto di una particella. Indicando con \({\Delta}x\) l’incertezza sulla posizione e con \({\Delta}p\) quella sulla quantità di moto, si ha:

\[
{\Delta}x {\Delta}p \geq \dfrac{\hslash}{2}
\]

Questo significa che, se la posizione viene determinata con estrema precisione (\({\Delta}x \rightarrow 0\)), l’incertezza sulla quantità di moto deve aumentare (\({\Delta}p \rightarrow \infty\)) per mantenere valida la disuguaglianza. Analogamente, maggiore è la precisione sulla misura della quantità di moto meno precisa è la conoscenza della posizione, ovvero \({\Delta}p \rightarrow 0\) allora \({\Delta}x \rightarrow \infty\).

Il principio può essere esteso anche alla coppia energia-tempo, come:

\[
{\Delta}E {\Delta}t \geq \dfrac{\hslash}{2}
\]

In questo contesto, $\Delta t$ rappresenta l'intervallo di tempo durante il quale l'energia del sistema è definita. Questo significa che una misura precisa dell’energia ($\Delta E \rightarrow 0$) richiede un intervallo di tempo $\Delta t$ sufficientemente lungo (tendente a infinito), mentre una misura molto rapida ($\Delta t$ piccolo) comporta una maggiore incertezza sull’energia ($\Delta E$ grande).

Questo principio rappresenta un limite fondamentale alla conoscenza dello stato di una particella e riflette la natura intrinsecamente probabilistica della meccanica quantistica.

L’indeterminazione può essere interpretata come una conseguenza del processo di misura. Per determinare la posizione di una particella, ad esempio un elettrone, è necessario interagire con essa, ad esempio mediante fotoni. Questa interazione modifica inevitabilmente lo stato della particella. Nel mondo macroscopico, tale effetto è trascurabile, poiché l’energia dei fotoni è molto inferiore rispetto a quella degli oggetti con cui interagiscono, e quindi non ne altera significativamente lo stato.

\section{Equazione di Schrödinger}\label{equazione-di-schruxf6dinger}

La meccanica ondulatoria di Schrödinger è una teoria quantistica non relativistica, valida per particelle che si muovono a velocità molto inferiori a quella della luce \(c\). In questa descrizione si trascurano fenomeni come la creazione e l'annichilazione di particelle, poiché le energie coinvolte in tali processi sono troppo elevate per essere trattate nel contesto non relativistico della meccanica ondulatoria. Si osservi che il fotone è l'unica particella che può essere creata e distrutta con semplicità, tramite i fenomeni di emissione e assorbimento. Questi fenomeni sono descritti dalla meccanica ondulatoria.

La teoria si applica principalmente agli stati stazionari delle particelle e rappresenta la base della meccanica quantistica non relativistica. Da essa sono state sviluppate estensioni relativistiche, come l’equazione di Klein-Gordon e quella di Dirac, quest’ultima capace di prevedere l’esistenza del positrone \cite{dirac1930principles, landau1975quantistica_rel}.

La funzione d’onda associata a una particella libera può essere espressa come:

\[
\Psi(\vec{r},t) = \exp\left( \dfrac{j}{\hslash}(\vec{p} \cdot \vec{r} - Et) \right)
\]

La funzione d’onda deve soddisfare un’equazione differenziale che descriva la sua evoluzione nel tempo. Per l'ipotesi di Schrödinger, la funzione d'onda contiene tutte le informazioni necessarie a definire il moto della particella, per questo motivo l'equazione che permette di ricavare la funzione d'onda deve essere un'equazione differenziale, contenente la sua derivata temporale al primo ordine. In questo modo è possibile ricavare la funzione d'onda in ogni istante temporale, noto l'istante iniziale.

Inoltre, la teoria di Schrödinger non considera gli effetti relativistici. Un primo tentativo di includere la relatività fu formulato da Klein-Gordon che, appunto, considerarono l'equazione con una derivata temporale al secondo ordine.

Dirac, infine, scrisse un'equazione relativistica corretta da cui fu possibile prevedere l'esistenza del positrone \cite{dirac1930principles}.

Per ricavare l'equazione di Schrödinger, si applica il gradiente alla funzione d'onda piana:

\[
\vec{\nabla}\Psi\left(\vec{r},t \right) = \dfrac{\partial\Psi}{\partial\vec{r}} = \dfrac{\partial}{\partial\vec{r}}\exp\left( \dfrac{j}{\hslash}\left( \vec{p} \cdot \vec{r} - Et \right) \right) = \dfrac{j}{\hslash}\vec{p}\exp\left( \dfrac{j}{\hslash}\left( \vec{p} \cdot \vec{r} - Et \right) \right)
\]

Il gradiente può essere riscritto come:

\[
\vec{\nabla}\Psi\left( \vec{r},t \right) = \dfrac{j}{\hslash}\vec{p}\,\Psi\left( \vec{r},t \right)
\]

Per valutare il laplaciano della funzione d'onda, si applica la divergenza al gradiente di \(\Psi\):

\[
\vec{\nabla} \cdot \vec{\nabla}\Psi\left( \vec{r},t \right) = \nabla^{2}\Psi\left( \vec{r},t \right) = \dfrac{j}{\hslash}\left( \vec{\nabla} \cdot \left(\vec{p}\Psi\left( \vec{r},t \right)\right) \right) = 
\]

Per le proprietà dell'operatore divergenza è possibile scrivere:

\[
 = \dfrac{j}{\hslash}\left(\Psi\left( \vec{r},t \right) \vec{\nabla} \cdot \vec{p} + \vec{p}\cdot \vec{\nabla}\Psi\left( \vec{r},t \right) 
 \right) = 
\]

Poiché il vettore quantità di moto per una particella libera è costante, la sua divergenza è nulla (\(\vec{\nabla} \cdot \vec{p} = 0\)), per cui risulta:

\[
\vec{\nabla} \cdot \vec{\nabla}\Psi\left( \vec{r},t \right) = \dfrac{j}{\hslash}\vec{p}\cdot\vec{\nabla}\,\Psi\left( \vec{r},t \right)
\]

Il laplaciano della funzione d'onda si scrive come:

\[
\nabla^{2}\Psi = \dfrac{j}{\hslash}\vec{p}\cdot\vec{\nabla}\,\Psi\left( \vec{r},t \right) = \dfrac{j}{\hslash}\vec{p}\cdot\left( \dfrac{j}{\hslash}\vec{p}\,\Psi\left( \vec{r},t \right) \right) = \left( \dfrac{j}{\hslash} \right)^{2}\Psi\left( \vec{r},t \right) \vec{p}\cdot\vec{p}
\]

Per le proprietà del prodotto scalare e delle potenze dell'unità immaginaria, si scrive:

\[
\nabla^{2}\Psi = - \left( \dfrac{p}{\hslash} \right)^{2}\Psi  
\]

La derivata temporale di \(\Psi\) è, invece:

\[\dfrac{\partial\Psi}{\partial t} = \dfrac{\partial}{\partial t}\exp\left( \dfrac{j}{\hslash}\left( \vec{p} \cdot \vec{r} - Et \right) \right) = - \dfrac{j}{\hslash}E\exp\left( \dfrac{j}{\hslash}\left( \vec{p} \cdot \vec{r} - Et \right) \right)
\]

Ovvero:

\[\dfrac{\partial\Psi}{\partial t} = - \dfrac{j}{\hslash}E\Psi\]

L'equazione di Schrödinger è ottenuta confrontando le due quantità ottenute:

\[
\begin{cases}
 \nabla^{2}\Psi = - \left( \dfrac{p}{\hslash} \right)^{2}\Psi \\
 \dfrac{\partial\Psi}{\partial t} = - \dfrac{j}{\hslash}E\Psi
\end{cases} 
\]

Si isola \(\Psi\) per entrambe le equazione:

\[
\begin{cases}
\Psi = - \dfrac{\hslash^{2}}{p^{2}}\nabla^{2}\Psi \\
\Psi = - \dfrac{\hslash}{jE}\dfrac{\partial\Psi}{\partial t}
\end{cases} \Leftrightarrow \begin{cases}
 \Psi = - \dfrac{\hslash^{2}}{p^{2}}\nabla^{2}\Psi \\
 \Psi = j\dfrac{\hslash}{E}\dfrac{\partial\Psi}{\partial t}
\end{cases}
\]

Uguagliando i secondi membri delle due equazioni si ottiene:

\begin{equation}
    - \dfrac{\hslash^{2}}{p^{2}}\nabla^{2}\Psi = j\dfrac{\hslash}{E}\dfrac{\partial\Psi}{\partial t} 
    \label{eq:eq1}
\end{equation}



Per una particella libera, l’energia totale o hamiltoniana  coincide con l'energia cinetica:

\[
H = E = T = \dfrac{1}{2}mv^{2} = \dfrac{p^{2}}{2m}
\]

Dall'ultima uguaglianza si isola il termine \(p^{2}\):

\[
p^{2} = 2mE
\]

Sostituendo questo risultato nell'equazione differenziale ottenuta, si ha;

\[
- \dfrac{\hslash^{2}}{2mE}\nabla^{2}\Psi = j\dfrac{\hslash}{E}\dfrac{\partial\Psi}{\partial t}
\]

Si semplifica \(\hslash\) ed \(E\), ottenendo l'equazione:

\[
\dfrac{\hslash}{2m}\nabla^{2}\Psi + j\dfrac{\partial\Psi}{\partial t} = 0
\]

Questa relazione rappresenta la forma standard dell’equazione di Schrödinger per una particella libera. 

Nel caso in cui la particella si trovi in un campo di potenziale,  dipendente dalla posizione e dal tempo \(U(\vec{r},t)\), l’equazione si modifica includendo il termine di energia potenziale al secondo membro:

\[
j\hslash\dfrac{\partial\Psi}{\partial t} = \left( -\dfrac{\hslash^{2}}{2m}\nabla^{2} + U(\vec{r},t) \right)\Psi
\]

Questa equazione descrive l’evoluzione temporale della funzione d’onda \(\Psi(\vec{r},t)\), che contiene tutte le informazioni sullo stato quantico della particella. La sua soluzione consente di determinare la probabilità di trovare la particella in una certa posizione e in un certo istante.

\section{Operatori in meccanica quantistica}\label{operatori-in-meccanica-quantistica}
In meccanica quantistica, a ogni grandezza fisica osservabile è associato un operatore matematico \(\hat{f}\) che agisce sullo spazio delle funzioni d'onda:

\[
\hat{f} : \Psi \rightarrow \varphi
\]

L'operatore \(\hat{f}\) è un'applicazione lineare che trasforma una funzione d'onda \(\Psi\) in un'altra funzione d'onda \(\varphi\), ovvero porta una particella da uno stato quantico iniziale a uno finale. I valori misurabili della grandezza fisica corrispondono agli autovalori dell'operatore \(\hat{f}\), ottenuti risolvendo l'equazione agli autovalori:

\[
\hat{f} \Psi = f \Psi
\]

In questo contesto, un’osservabile è una grandezza fisica misurabile, rappresentata da un operatore lineare, in genere complesso, che agisce sullo spazio degli stati quantici.

Essendo \(\hat{f}\) un operatore lineare, vale la proprietà:

\[
\hat{f}\left( c_{1}\Psi_{1} + c_{2}\Psi_{2} \right) = c_{1}\hat{f}\left( \Psi_{1} \right) + c_{2}\hat{f}\left( \Psi_{2} \right)
\]

In meccanica quantistica, la quantità di moto (o momento lineare) è associata all'operatore:

\[
\hat{\vec{p}} = -j\hslash\vec{\nabla}
\]

che, in coordinate cartesiane, si scrive:

\[
\hat{\vec{p}} = -j\hslash
\begin{pmatrix}
 \dfrac{\partial}{\partial x} \\
 \dfrac{\partial}{\partial y} \\
 \dfrac{\partial}{\partial z}
\end{pmatrix}
\]

L'energia di una particella libera è descritta dall'operatore hamiltoniano:

\[
\hat{H} = \dfrac{1}{2m} \hat{\vec{p}} \cdot \hat{\vec{p}} = \dfrac{1}{2m} (-j\hslash\vec{\nabla}) \cdot (-j\hslash\vec{\nabla}) = -\dfrac{\hslash^{2}}{2m}\nabla^{2}
\]

che, in coordinate cartesiane, si scrive:

\[
\hat{H} = -\dfrac{\hslash^{2}}{2m}\left( \dfrac{\partial^{2}}{\partial x^{2}} + \dfrac{\partial^{2}}{\partial y^{2}} + \dfrac{\partial^{2}}{\partial z^{2}} \right)
\]

Se la particella è immersa in un campo di potenziale \(U\left(\vec{r},t\right)\), l'operatore hamiltoniano, rappresentante l'energia totale della particella, si generalizza come:

\[
\hat{H} =  -\dfrac{\hslash^{2}}{2m}\nabla^{2} + U\left(\vec{r},t\right)
\]

Il momento angolare è associato all'operatore:

\[
\vec{\hat{L}} = \vec{r} \times \hat{\vec{p}} = \vec{r} \times \left( -j\hslash\vec{\nabla} \right) = -j\hslash\, \vec{r} \times \vec{\nabla}
\]

\begin{table}[ht]
    \centering
    \caption{Operatori Fondamentali in Meccanica Quantistica}
    \label{tab:operatori-quantistici}
    \begin{tabular}{|l|c|c|}
        \hline
        \textbf{Osservabile} & \textbf{Simbolo Classico} & \textbf{Operatore Quantistico ($\hat{f}$)} \\
        \hline
        Posizione & $\vec{r}$ & $\hat{\vec{r}} = \vec{r}$ \\
        \hline
        Quantità di Moto & $\vec{p}$ & $\hat{\vec{p}} = -j\hslash\vec{\nabla}$ \\
        \hline
        Energia Cinetica & $T = \dfrac{p^2}{2m}$ & $\hat{T} = -\dfrac{\hslash^{2}}{2m}\nabla^{2}$ \\
        \hline
        Energia Potenziale & $U(\vec{r},t)$ & $\hat{U} = U(\vec{r},t)$ \\
        \hline
        Energia Totale (Hamiltoniana) & $H = T + U$ & $\hat{H} = -\dfrac{\hslash^{2}}{2m}\nabla^{2} + U(\vec{r},t)$ \\
        \hline
        Momento Angolare & $\vec{L} = \vec{r} \times \vec{p}$ & $\vec{\hat{L}} = -j\hslash\, (\vec{r} \times \vec{\nabla})$ \\
        \hline
    \end{tabular}
\end{table}

Utilizzando la definizione dell'operatore hamiltoniano, è possibile riscrivere in forma compatta l'equazione di Schrödinger per una particella immersa in un campo potenziale variabile nel tempo e con la posizione:

\[
j\hslash\dfrac{\partial \Psi}{\partial t} = -\dfrac{\hslash^2}{2m} \nabla^2 \Psi + U(\vec{r},t)\Psi
\]

Poiché \(\hat{H} = -\hslash^{2}/{2m}\,\nabla^{2} + U\left(\vec{r},t\right)\), si può scrivere:

\[
\left( U\left( \vec{r},t \right) -\dfrac{\hslash^{2}}{2m}\nabla^{2}\right) \Psi = j\hslash\dfrac{\partial\Psi}{\partial t}
\]

Si ottiene così la forma operativa dell’equazione di Schrödinger:

\[
\hat{H}\Psi = j\hslash\dfrac{\partial\Psi}{\partial t}
\]

oppure, portando tutti i termini da un lato:

\[
\hat{H}\Psi - j\hslash\dfrac{\partial\Psi}{\partial t} = 0
\]

\section{Equazione di Schrödinger per stati stazionari}\label{equazione-di-schruxf6dinger-per-stati-stazionari}

La funzione d'onda \(\Psi\left( \vec{r},t \right)\), espressa come un'onda piana, può essere scritta come:

\[
\Psi\left( \vec{r},t \right) = \exp\left( \dfrac{j}{\hslash}\left( \vec{p} \cdot \vec{r} - Et \right) \right) = \exp\left( \dfrac{j}{\hslash}\vec{p} \cdot \vec{r} \right)\exp\left( - \dfrac{j}{\hslash}Et \right)
\]

Si definisce \(\phi\left( \vec{r} \right)\) come la parte della funzione d'onda dipendente dalla posizione:

\[
\phi\left( \vec{r} \right) = \exp\left( \dfrac{j}{\hslash}\vec{p} \cdot \vec{r} \right)
\]

L'equazione di Schrödinger si scrive quindi come:

\[
\Psi\left( \vec{r},t \right) = \phi\left( \vec{r} \right)\exp\left( - \dfrac{j}{\hslash}Et \right)
\]

Se l'energia del sistema è costante, l'hamiltoniana non dipende dal tempo. Si applica questo operatore alla funzione d'onda:

\[
\hat{H}\Psi = \hat{H}\left( \phi\left( \vec{r} \right)\exp\left( - \dfrac{j}{\hslash}Et \right) \right)
\]

Per definizione dell'operatore hamiltoniano, si ha:

\[
\hat{H}\Psi = \left( -\dfrac{\hslash^{2}}{2m}\nabla^{2} + V(\vec{r}) \right)\left( \phi\left( \vec{r} \right)\exp\left( - \dfrac{j}{\hslash}Et \right) \right)
\]

Poiché $V(\vec{r})$ è un operatore di moltiplicazione e l'esponenziale temporale non dipende dalla posizione (dunque, può essere portato all'esterno del Laplaciano) si ottiene:

\[
\hat{H}\Psi = - \dfrac{\hslash^{2}}{2m}\exp\left( - \dfrac{j}{\hslash}Et \right)\nabla^{2}\phi\left( \vec{r} \right) + V(\vec{r})\phi\left( \vec{r} \right) \exp\left( - \dfrac{j}{\hslash}Et \right)
\]

Raccogliendo opportunamente, si ha:

\[
\hat{H}\Psi = \exp\left( - \dfrac{j}{\hslash}Et \right)\left( -\dfrac{\hslash^{2}}{2m}\nabla^{2} + V(\vec{r}) \right)\phi\left( \vec{r} \right)
\]

Per definizione dell'operatore hamiltoniano, si ricava:

\[
\hat{H}\Psi = \exp\left( - \dfrac{j}{\hslash}Et \right)\hat{H}\phi\left( \vec{r} \right)
\]

Si calcola ora la derivata temporale di \(\Psi\):

\[
\dfrac{\partial\Psi}{\partial t} = \dfrac{\partial}{\partial t}\left( \phi\left( \vec{r} \right)\exp\left( - \dfrac{j}{\hslash}Et \right) \right)
\]

Poiché \(\phi\left( \vec{r} \right)\) non dipende dal tempo, ma solamente dalla posizione, può essere portata all'esterno del simbolo di derivata:

\[
\dfrac{\partial\Psi}{\partial t} = \phi\left( \vec{r} \right)\dfrac{\partial}{\partial t}\exp\left( - \dfrac{j}{\hslash}Et \right) = - \dfrac{j}{\hslash}E\phi\left( \vec{r} \right)\exp\left( - \dfrac{j}{\hslash}Et \right)
\]

Si considera l'equazione di Schrödinger in termini di hamiltoniana, \(\hat{H}\Psi - j\hslash\partial\Psi/\partial t = 0\). Si è visto che:

\[
\begin{cases}
\dfrac{\partial\Psi}{\partial t} = - \dfrac{j}{\hslash}E\phi\left( \vec{r} \right)\exp\left( - \dfrac{j}{\hslash}Et \right) \\
\hat{H}\Psi = \exp\left( - \dfrac{j}{\hslash}Et \right)\hat{H}\phi\left( \vec{r} \right)
\end{cases} 
\]

Sostituendo:

\[
\exp\left( - \dfrac{j}{\hslash}Et \right)\hat{H}\phi\left( \vec{r} \right) - j\hslash\left( - \dfrac{j}{\hslash}E\phi\left( \vec{r} \right)\exp\left( - \dfrac{j}{\hslash}Et \right) \right) = 0
\]

Svolgendo i prodotti, si ottiene:

\[
\exp\left( - \dfrac{j}{\hslash}Et \right)\hat{H}\phi\left( \vec{r} \right) - E\phi\left( \vec{r} \right)\exp\left( - \dfrac{j}{\hslash}Et \right) = 0
\]

Semplificando il termine esponenziale, si ottiene:

\[
\hat{H}\phi\left( \vec{r} \right) = E\phi\left( \vec{r} \right)
\]

Si ottiene un'equazioni agli autovettori e autovalori; infatti, è possibile scrivere:

\[\left( \hat{H} - E \right)\phi\left( \vec{r} \right) = 0\]

Dove \(E\) è l'energia totale del sistema supposta costante. L'energia \(E\) rappresenta gli autovalori dell'operatore hamiltoniano \(\hat{H}\). Questo risultato è coerente con gli esperimenti, poiché gli autovalori sono, in genere, un'infinità numerabile e, dunque, discreti.

La meccanica ondulatoria i Schrödinger prevede la quantizzazione dell'energia degli orbitali atomici. La soluzione dell'equazione agli autovalori permette di ottenere i livelli energetici del sistema e le autofunzioni \(\phi\left( \vec{r} \right)\), il cui modulo quadro rappresenta la probabilità che la particella del sistema si trovi un quel livello energetico.

Sia \(V\left( \vec{r} \right)\) l'energia potenziale a cui la particella è soggetta, ad esempio un elettrone attratto dal nucleo. L'operatore hamiltoniano si scrive:

\[
\hat{H} = - \dfrac{\hslash^{2}}{2m}\nabla^{2} + V\left( \vec{r} \right)
\]

Con \({\hat{E}}_{c}\) operatore energia cinetica:

\[
{\hat{E}}_{c} = - \dfrac{\hslash^{2}}{2m}\nabla^{2}
\]

L'operatore hamiltoniano è dato da:

\[
\hat{H} = \hat{E} = - \dfrac{\hslash^{2}}{2m}\nabla^{2} + V\left( \vec{r} \right)
\]

Nel caso dell'elettrone attratto dal nucleo, il potenziale è di tipo coulombiano:

\[
V\left( \vec{r} \right) = -\dfrac{1}{4\pi\varepsilon_{0}}\dfrac{Ze^{2}}{r}
\]

Dove \(Z\) è il numero atomico, ovvero il numero di protoni nel nucleo.

L'equazione agli autovettori dell'hamiltoniano si scrive come:


\[
\hat{H}\phi\left( \vec{r} \right) - \ E\phi\left( \vec{r} \right) = 0 \Leftrightarrow - \dfrac{\hslash^{2}}{2m}\nabla^{2}\phi + V\left( \vec{r} \right)\phi = E\phi
\]

Moltiplicando per \(- 1\), si ha:

\[
\dfrac{\hslash^{2}}{2m}\nabla^{2}\phi - V\left( \vec{r} \right)\phi + E\phi = 0
\]

Raccogliendo, si ha:

\[
\left( \dfrac{\hslash^{2}}{2m}\nabla^{2} + \left( E - V\left( \vec{r} \right) \right) \right) \phi = 0
\]

In generale, è possibile scrivere un'equazione agli autovalori per ogni operatore. I corrispondenti autovalori sono i valori che quell'operatore può assumere.

Ad esempio, gli autovalori del momento angolare sono i possibili valori del momento angolare ottenuti risolvendo l'equazione agli autovalori:

\[
\hat{L}\phi = L\phi
\]

In presenza di un campo magnetico, l'energia totale dell'elettrone o particella è data da:

\[
E = E_{c} + V\left( \vec{r} \right)
\]

Dove il potenziale è dato da:

\[
V\left( \vec{r} \right) = \vec{\mu} \cdot \vec{B}
\]

Il momento magnetico è legato al momento angolare dal fattore giromagnetico:

\[
\hat{\vec{\mu}} = - \gamma\hat{\vec{L}}
\]

Dunque, il potenziale può essere espresso come:

\[
V\left( \vec{r} \right) = - \gamma\hat{\vec{L}} \cdot \vec{B}
\]

In definitiva, l'operatore hamiltoniano si scrive:

\[
\hat{H} = - \dfrac{\hslash^{2}}{2m}\nabla^{2} - \gamma\hat{\vec{L}} \cdot \vec{B}
\]

\subsection{Buco di potenziale}\label{buco-di-potenziale}

Secondo la meccanica classica, l'energia cinetica di una particella è:

\[
T = \dfrac{1}{2}mv^{2}
\]

Mentre l'energia totale è data da:

\[
E = T + V
\]

Con \(V\) energia potenziale. Dalla relazione per l'energia totale è possibile valutare l'energia cinetica in funzione di quella totale e potenziale:

\[
T = E - V
\]

Nella teoria classica, la particella è in moto se l'energia cinetica è positiva, ovvero:

\[
T > 0 \Leftrightarrow E > V
\]

Pertanto, in ambito classico, il moto della particella può avvenire solo nelle regioni in cui \(E > V\).

Si considera ora una buca di potenziale unidimensionale definita da:

\[
V(x) = \begin{cases}
0 & 0 < x < a \\
\infty & x \leq 0 \text{ oppure } x \geq a
\end{cases}
\]

\begin{figure}[ht]
\centering
\includegraphics[width=2.67857in,height=0.86783in,alt={P1856C2T1\#yIS1}]{media/4_Quantiatica/image36.pdf}\caption{Buca di potenziale}
\end{figure}

La particella può muoversi solo all'interno della regione di spazio \(0 < x < a\), dunque, la funzione d'onda \(\phi\left( \vec{r} \right)\) è nulla all'esterno della buca di potenziale, in cui \(V \rightarrow \infty\).


L'equazione di Schrödinger ha soluzioni non nulle solo dove il potenziale è finito. In termini di hamiltoniana:

\[
\hat{H}\phi(x) = E\phi(x)
\]

Poiché la particella è libera all'interno della buca, l'operatore hamiltoniano nel caso generale è:

\[
\hat{H} = - \dfrac{\hslash^{2}}{2m}\nabla^{2}
\]

Il moto avviene solamente lungo l'asse \(x\), per cui l'operatore è:

\[
\hat{H} = - \dfrac{\hslash^{2}}{2m}\dfrac{d^{2}}{d x^{2}}
\]

L'equazione di Schrödinger per la buca di potenziale e, quindi:

\[
- \dfrac{\hslash^{2}}{2m}\dfrac{d^{2}\phi}{dx^{2}} = E\phi
\Leftrightarrow
\dfrac{d^{2}\phi}{dx^{2}} + \dfrac{2mE}{\hslash^{2}}\phi = 0
\]

L'equazione ottenuta coincide con l'oscillatore armonico, la cui soluzione è del tipo:

\[\phi(x) = A\cos\left( \sqrt{\dfrac{2mE}{\hslash^{2}}}x \right) + B\sin\left( \sqrt{\dfrac{2mE}{\hslash^{2}}}x \right)\]

Dove \(A\) e \(C\) sono due costanti dipendenti dalle condizioni al contorno, ottenute considerando la funzione d'onda continua nei punti \(x = a\) e \(x = b\):

\[
\begin{cases}
\phi(a) = 0 \\
\phi(0) = 0
\end{cases} 
\]

Si applica la prima condizione \(\phi(a) = 0 \):

\[
\phi(a) = A\cos\left( \sqrt{\dfrac{2mE}{\hslash^{2}}}0 \right) + B\sin\left( \sqrt{\dfrac{2mE}{\hslash^{2}}}0 \right) = 0
\]

Da cui risulta:

\[
\phi(0) = A = 0
\]

Si applica la seconda condizione al contorno:

\[
\phi(a) = B\sin\left( \sqrt{\dfrac{2mE}{\hslash^{2}}}a \right) = 0
\]

Le soluzioni dell'equazione:

\[
\sin\left( \sqrt{\dfrac{2mE}{\hslash^{2}}}a \right) = 0
\]

sono un'infinità numerabile:

\[
\sqrt{\dfrac{2mE_{n}}{\hslash^{2}}}a = n\pi,\ n\in\mathbb{N}
\]

Si ricava \(E_{n}\):

\[
\dfrac{a}{\hslash}\sqrt{2mE_{n}} = n\pi \Leftrightarrow \sqrt{2mE_{n}} = \dfrac{\hslash^{2}}{a^{2}}n\pi,\ n\in \mathbb{N}
\]

Si eleva al quadrato e si divide ambo i membro per \(2m\):

\[
E_{n} = \dfrac{\hslash^{2}}{2ma^{2}}n^{2}\pi^{2},\ n\in \mathbb{N}
\]

Gli \(E_{n}\) sono gli autovalori possibili dell'operatore hamiltoniano e, di conseguenza, i possibili livelli energetici che può assumere la particella in una buca di potenziale monodimensionale. Come si nota, i livelli energetici sono quantizzati, in accordo con le previsioni sperimentali.

Si considera, ora, il caso tridimensionale, ovvero la particella si muove in una scatola di potenziale. La funzione potenziale è data da:

\[
V(x,y,z) = \begin{cases}
0 & 0 < x < a,\ 0 < y < b,0 < z < c\  \\
\infty & altrove
\end{cases} 
\]

L'equazione di Schrödinger fornisce valori non nulli solamente nella regione di spazio in cui il potenziale è finito \(\left[ 0;a\right] \times \left[ 0;b\right] \times \left[0;c\right]\). In termini di hamiltoniano, costante nel tempo, si ha:

\[
\hat{H}\phi\left( \vec{r} \right) - E\phi\left( \vec{r} \right) = 0
\]

Esplicitando l'operatore hamiltoniano, si ottiene:

\[
- \dfrac{\hslash^{2}}{2m}\nabla^{2}\ \phi\left( \vec{r} \right) - E\phi\left( \vec{r} \right) = 0
\]

Ricavando il laplaciano di \(\phi\):

\[
\nabla^{2}\ \phi + \dfrac{2m}{\hslash^{2}}E\phi = 0
\]

In coordinate cartesiane, l'equazione è data:

\[
\dfrac{\partial^{2}\phi}{\partial x^{2}} + \dfrac{\partial^{2}\phi}{\partial y^{2}} + \dfrac{\partial^{2}\phi}{\partial z^{2}} + \dfrac{2m}{\hslash^{2}}E\phi = 0
\]

Si applica il metodo delle variabile separabili, secondo cui la soluzione è del tipo:

\[
\phi\left( \vec{r} \right) = \alpha(x)\beta(y)\gamma(z)
\]

Si sostituisce questa espressione nell'equazione di Schrödinger in coordinate cartesiane:

\[
\beta\gamma\dfrac{\partial^{2}\alpha}{\partial x^{2}} + \alpha\gamma\dfrac{\partial^{2}\beta}{\partial y^{2}} + \alpha\beta\dfrac{\partial^{2}\gamma}{\partial z^{2}} + \dfrac{2mE}{\hslash^{2}}\alpha\beta\gamma = 0
\]

Si divide per \(\phi\left( \vec{r} \right) = \alpha(x)\beta(y)\gamma(z)\):

\[\dfrac{1}{\alpha}\ \dfrac{\partial^{2}\alpha}{\partial x^{2}} + \dfrac{1}{\beta}\ \dfrac{\partial^{2}\beta}{\partial y^{2}} + \dfrac{1}{\gamma}\dfrac{\partial^{2}\gamma}{\partial z^{2}} + \dfrac{2mE}{\hslash^{2}} = 0\]

I tre termini dipendono solamente da una variabile spaziale, dunque, è possibile scrivere tre equazioni diverse:

\[
\begin{cases}
\dfrac{\partial^{2}\alpha}{\partial x^{2}} + k_{x}^{2}\alpha = 0 \\
\dfrac{\partial^{2}\beta}{\partial y^{2}} + k_{y}^{2}\beta = 0 \\
\dfrac{\partial^{2}\gamma}{\partial z^{2}} + k_{z}^{2}\gamma = 0
\end{cases}
\]

La soluzione delle equazioni è del tipo:

\[
f(q) = c\exp\left( - jk_{q}q \right),\ \ q = x,y,z
\]

Dunque, la integrale generale è del tipo:

\[
\phi(x,y,z) = \left( A_x \cos(k_x x) + B_x \sin(k_x x) \right) \cdot \left( A_y \cos(k_y y) + B_y \sin(k_y y) \right) \cdot \left( A_z \cos(k_z z) + B_z \sin(k_z z) \right)
\]

Dove:
\[
k_x = \sqrt{2m\dfrac{E_x}{\hslash^2}},\ k_y = \sqrt{2m\dfrac{E_y}{\hslash^2}}\, k_z = \sqrt{2m\dfrac{E_z}{\hslash^2}}
\]
mentre $A_i$ e $B_i$ sono costanti determinate dalle condizioni al contorno.

Le condizione da imporre riguardano la continuità della funzione d'onda ai bordi della buca di potenziale:

\[
\begin{cases}
\phi(0,y,z) = 0 \\
\phi(a,y,z) = 0 \\
\phi(x,0,z) = 0 \\
\phi(x,b,z) = 0 \\
\phi(x,y,0) = 0 \\
\phi(x,y,c) = 0
\end{cases} 
\]

Le condizioni al contorno portano a un'equazione del tipo:

\[
\sin\left( ak_{x} \right) = 0
\]

La cui soluzioni sono:

\[
k_{x} = n_{x}\dfrac{\pi}{a},\ \ n_{x}\in \mathbb{N}
\]

Da cui si ottengono gli autovalori lungo \(x\) dell'equazione di Schrödinger:

\[
E_{x} = k_{x}^{2}\dfrac{\hslash}{2m} = n_{x}^{2}\dfrac{\pi^{2}}{a^{2}}\dfrac{\hslash^{2}}{2m},\ \ n_{x}\in\mathbb{ N}
\]

Analogo risultato lo si ottiene per \(E_{y}\):

\[
E_{y} = n_{y}^{2}\dfrac{\pi^{2}}{b^{2}}\dfrac{\hslash^{2}}{2m},\ \ n_{y}\in\mathbb{ N}
\]

ed \(E_{z}\):

\[E_{z} = n_{z}^{2}\dfrac{\pi^{2}}{c^{2}}\dfrac{\hslash^{2}}{2m},\ \ n_{z}\in\mathbb{N}\]

La somma dei tre autovalori deve essere uguale all'energia totale:

\[
E_{n} = E_{x} + E_{z} + E_{z}
\]

Sostituendo i valori ottenuti si ottengono gli autovalori dell'operatore hamiltoniano per la geometria considerata:

\[
E_{n} = \left( \dfrac{n_{x}^{2}}{a^{2}} + \dfrac{n_{y}^{2}}{b^{2}} + \dfrac{n_{z}^{2}}{c^{2}} \right)\dfrac{\pi^{2}\hslash^{2}}{2m},\ \ n_{x},n_{y},n_{z}\in \mathbb{N}
\]

Allo stesso modo è possibile ottenere i livelli energetici per un atomo qualsiasi, come quello di idrogeno. In questo caso, il potenziale in cui è immerso l'elettrone è di tipo coulombiano:

\[V(r) = \dfrac{1}{4\pi\varepsilon_{0}}\dfrac{e^{2}}{r}\]

La risoluzione dell'equazione di Schrödinger in coordinate sferiche fornisce i livelli energetici dell'atomo.

Le previsioni teoriche, ovvero livelli energetici ricavati dalla risoluzione dell'equazione di Schrödinger, coincidono con l'energia dei livelli energetici dell'atomo. Ne discende che, mediante l'equazione di Schrödinger è possibile ottenere una previsione teorica anche per gli spettri di assorbimento. Infatti, nota l'energia degli orbitali, è nota anche l'energia, \(E = h\nu\), che il fotone deve possedere affinché sia assorbito dall'elettrone. L'energia del fotone deve essere maggiore della differenza dell'energia dei due livelli energetici coinvolti.

\subsection{Gradino di potenziale}\label{gradino-di-potenziale}

In meccanica classica il moto di una particella non può avvenire nella regione di spazio in cui la sua energia \(E\) è minore  del potenziale \(V\) che insiste in quella regione di spazio (\(E < V\)). In altre parole, se la particella non ha energia sufficiente, non riesce a superare il gradino di potenziale.

Si consideri una particella elementare, come un elettrone, con energia \(E\), lanciata verso un gradino di potenziale \(V_0\), definita come:

\[
V(x) = \begin{cases}
V_{0} & x > 0 \\
0 & x < 0
\end{cases} 
\]

\begin{figure}[ht]
\centering
\includegraphics[width=5.10848in,height=0.63328in,alt={P1933\#yIS1}]{media/4_Quantiatica/image37.pdf}\caption{Gradino di potenziale}
\end{figure}

È possibile scrivere l'equazione di Schrödinger per le due regioni dello spazio, in base al potenziale \(V(x)\). Essendo il moto monodimensionale, risulta:

\[
\begin{cases}
 \dfrac{\partial^{2}\phi}{\partial x^{2}} + \dfrac{2mE}{\hslash^{2}}\phi = 0 & x < 0 \\
 \dfrac{\partial^{2}\phi}{\partial x^{2}} + \dfrac{2m}{\hslash^{2}}\left( E - V_{0} \right)\phi = 0 & x > 0
\end{cases}
\]

La prima equazione presenta una soluzione del tipo:

\[
\phi(x) = A\exp\left( j\dfrac{\sqrt{2mE}}{\hslash}x \right) + C\exp\left( - j\dfrac{\sqrt{2mE}}{\hslash}x \right)
\]

All'interfaccia del gradino di potenziale, si genera un'onda riflessa che prosegue in verso retrogrado. Dunque, Nella regione di spazio \(x < 0\) vi sono due onde: una progressiva (o incidente) e una regressiva (o riflessa). Nell'equazione, $A$ è il coefficiente dell'onda incidente e $C$ quello della riflessa.

\begin{figure}[ht]
\centering
\includegraphics[width=5.25in,height=0.81494in,alt={P1940\#yIS1}]{media/4_Quantiatica/image38.pdf}\caption{Onda riflessa e trasmessa all'interfaccia}
\end{figure}

Nella regione \(x > 0\) supponendo che \(E < V_{0}\), l'esponenziale dell'onda deve essere reale e negativo, ovvero:

\[
\phi(x) = B\exp\left( - \dfrac{\sqrt{2m(V_{0} - E)}}{\hslash}x \right)
\]

Applicando la condizione di continuità della funzione d'onda all'interfaccia \(x = 0\), risulta che:

\[
\phi\left( 0^{-} \right) = \phi\left( 0^{+} \right)
\]

Ovvero:

\[
A + C = B
\]

Esiste, dunque, una probabilità non nulla di trovare la particella oltre il gradino di potenziale. Tuttavia, poiché l'onda nella regione di spazio \(x>0\) presenta un esponenziale reale, la probabilità di trovare la particella elementare oltre il gradino di potenziale decresce rapidamente con la distanza. Nonostante ciò, trovare la particella oltre il gradino di potenziale è un evento possibile, soprattutto in prossimità dell'interfaccia. 

\begin{figure}[ht]
\centering
\includegraphics[width=3.77117in,height=2.41667in,alt={P1949\#yIS1}]{media/4_Quantiatica/image39.pdf}\caption{Probabilità di rilevare la particella oltre il gradino di potenziale}
\end{figure}

L'effetto del gradino di potenziale è sfruttato nei dispositivi a semiconduttore.

\subsection{Effetto tunnel}\label{effetto-tunnel}

Si suppone che il potenziale sia confinato in una regione dello spazio, ovvero del tipo:

\[V(x) = \begin{cases}
0 & x < 0 \\
V_{0} & 0 \leq x \leq a \\
0 & x > 0
\end{cases} 
\]

\begin{figure}[ht]
\centering
\includegraphics[width=6.52905in,height=0.84849in,alt={P1955\#yIS1}]{media/4_Quantiatica/image40.pdf}\caption{Impulso di tensione}
\end{figure}

Si suppone che la particella provenga da sinistra, ovvero proceda nel verso delle \(x\) crescenti. Nella prima regione (\(x < 0\)), l'equazione di Schrödinger stazionaria è:

\[
\dfrac{d^2\phi}{dx^2} + \dfrac{2mE}{\hslash^2}\phi = 0
\]

La cui soluzione prevede due onde: una progressiva e una regressiva a causa dei fenomeni di riflessione:

\[
\phi_{x < 0}(x) = I\exp\left( j\dfrac{\sqrt{2mE}}{\hslash}x \right) + R\exp\left( - j\dfrac{\sqrt{2mE}}{\hslash}x \right)
\]

dove \(I\) è l'ampiezza dell'onda incidente e \(R\) quella dell'onda riflessa.

Nella regione intermedia (\(0 < x < a\)), dove il potenziale è costante e maggiore dell'energia della particella (\(V_0 > E\)), l'equazione diventa:

\[
\dfrac{d^{2}\phi}{d x^{2}} + \dfrac{2m}{\hslash^{2}}\left( E - V_{0} \right)\phi = 0,\ 0 < x < a
\]

Per la presenza dell'interfaccia successiva, per ottenere la soluzione completa, è necessario prevedere la presenza di due onde: una progressiva e una regressiva:

\[
\phi_{0 < x < a}(x) = \ A\exp\left( \dfrac{\sqrt{2m(V_0 - E)}}{\hslash}x \right) + B\exp\left( - \dfrac{\sqrt{2m(V_0 - E)}}{\hslash}x \right)
\]

Dove gli esponenziali sono reali a causa della condizione \(V_{0} > E\).

Nella regione \(x > a\), invece, si ha un'unica onda poiché non vi sono fenomeni di riflessione. L'equazione stazionaria è la stessa della regione per \(x < 0\):

\[\dfrac{d^{2}\phi}{d x^{2}} + \dfrac{2mE}{\hslash^{2}}\phi = 0,\ \ x > a\]

Dove la soluzione è:

\[
\phi_{x > a}(x) = S\exp\left( j\dfrac{\sqrt{2mE}}{\hslash}x \right)
\]

Oltre l'impulso di tensione di ampiezza \(V_{0}\) maggiore di \(E\) della particella, esiste una probabilità non nulla di trovare la particella. Tale fenomeno è noto come effetto tunnel e rappresenta uno dei risultati più controintuitivi e utilizzati della meccanica quantistica.

\begin{figure}[ht]
\centering
\includegraphics[width=4.07292in,height=1.87026in,alt={P1971\#yIS1}]{media/4_Quantiatica/image41.pdf}\caption{Effetto tunnel}
\end{figure}

L'effetto tunnel ha importanti applicazioni, tra cui il Microscopio a Scansione a Effetto Tunnel (STM) e i diodi tunnel.

%fenomeno della \textbf{scintillazione}, utilizzato per ridurre la dose di radiazione somministrata al paziente durante esami radiologici.

\section{Meccanica quantistica con notazione di Dirac}\label{meccanica-quantistica-con-notazione-di-dirac}

La \textbf{meccanica ondulatoria}, sviluppata principalmente da Schrödinger, e la \textbf{meccanica matriciale}, introdotta da Heisenberg, sono due formulazioni della meccanica quantistica elaborate nello stesso periodo storico. Dirac ha dimostrato l'equivalenza tra le due teorie attraverso un approccio algebrico basato sugli \textbf{spazi di Hilbert}, una generalizzazione dello spazio euclideo.

Ogni sistema microscopico è descritto, in ogni istante, da uno \textbf{stato quantico} rappresentato da un vettore nello spazio di Hilbert, indicato con \(\left| \varphi \right\rangle\), dove \(\varphi\) rappresenta lo stato del sistema.

Uno spazio di Hilbert \(H = \left( \mathbb{H}, \left\langle \cdot \middle| \cdot \right\rangle \right)\) è uno spazio vettoriale \textbf{complesso} dotato di un \textbf{prodotto interno sesquilineare} e \textbf{positivo definito}. Una forma sesquilineare è una funzione \(B: V \times V \rightarrow F\) lineare nel primo argomento e coniugata lineare nel secondo.  Essendo uno spazio lineare, vale il \textbf{principio di sovrapposizione}. Le principali proprietà dello spazio di Hilbert sono:

\begin{itemize}
  \item \textbf{Prodotto interno}: esiste un prodotto interno \(\left\langle \cdot \middle| \cdot \right\rangle\) tale che, detta \(d\) la distanza indotta dal prodotto interno, lo spazio metrico \(\left( \mathbb{H}, d \right)\) è \textbf{completo}, ovvero ogni successione di Cauchy converge in \(\mathbb{H}\).
  \item \textbf{Norma}: si può definire una norma associata al prodotto interno:
  \[
  \left\| \vec{v} \right\| = \sqrt{\left\langle \vec{v} \middle| \vec{v} \right\rangle}
  \]
  \item \textbf{Distanza}: la distanza tra due vettori è definita come:
  \[
  d\left( \vec{u}, \vec{v} \right) = \sqrt{\left\langle \vec{u} - \vec{v} \middle| \vec{u} - \vec{v} \right\rangle}
  \]
\end{itemize}

Le grandezze fisiche misurabili in un esperimento sono dette \textbf{osservabili} e corrispondono a \textbf{operatori hermitiani} (o della meccanica quantistica) che agiscono sullo spazio di Hilbert. Un operatore quantistico, come \(\hat{A}\), agisce su uno stato \(\left| \varphi \right\rangle\) per generare un nuovo stato \(\left| b \right\rangle\):

\[
\hat{A} \left| \varphi \right\rangle = \left| b \right\rangle
\]

L'operazione di misura \textbf{perturba} lo stato del sistema microscopico. In altre parole, la misura \textbf{modifica} lo stato quantico. Questo fenomeno è alla base del \textbf{principio di indeterminazione di Heisenberg}, secondo il quale non è possibile conoscere simultaneamente con precisione due grandezze coniugate (come energia e intervallo temporale). Il principio si esprime come:

\[
\Delta E \Delta t \geq \dfrac{\hslash}{2}
\]

Per calcolare l'energia di una particella, si applica l'\textbf{operatore hamiltoniano} \(\hat{H}\) allo stato \(\left| \varphi \right\rangle\):

\[
\hat{H} \left| \varphi \right\rangle = E \left| \varphi \right\rangle
\]

L'operatore hamiltoniano cambia lo stato del sistema. In questo caso, \(E\) è un \textbf{autovalore} dell'operatore \(\hat{H}\). Il sistema può assumere solo i valori energetici corrispondenti agli autovalori dell'operatore applicato.

Analogamente, per misurare il \textbf{momento angolare} si applica l'operatore \(\hat{L}\):

\[
\hat{L} \left| \varphi \right\rangle = L \left| \varphi \right\rangle
\]

Il momento angolare è \textbf{quantizzato}, quindi il sistema può assumere solo i valori corrispondenti agli autovalori dell'operatore \(\hat{L}\).

\subsection{Autovalori dell'operatore hamiltoniano}\label{autovalori-delloperatore-hamiltoniano}

Per un elettrone legato a un nucleo atomico, gli autovalori dell'operatore hamiltoniano rappresentano i possibili \textbf{livelli energetici} degli orbitali \(s\), \(p\), \(d\) e \(f\). Ogni orbitale può contenere al massimo \textbf{due elettroni con spin opposto}, secondo il \textbf{principio di esclusione di Pauli}, che stabilisce che due fermioni non possono occupare lo stesso stato quantico simultaneamente.

Questi livelli energetici sono discreti e quantizzati, e corrispondono alle soluzioni dell'equazione di Schrödinger per l'atomo. La struttura elettronica degli atomi è quindi determinata dagli autovalori dell'hamiltoniano, che definiscono le energie consentite per ciascun elettrone.

\subsection{Risultato dell'operatore di misura}\label{risultato-delloperatore-di-misura}

In generale, il risultato di una misura deve essere un \textbf{autovalore} dell'operatore di misura. Il valore della grandezza fisica in esame, in altre parole, deve essere una soluzione dell'equazione:

\[
\hat{A} \left| a \right\rangle = a_{n} \left| a \right\rangle
\]

Una misura può dare come esito uno qualsiasi degli autovalori \(a_{n}\) dell'operatore \(\hat{A}\). L'autovettore associato all'autovalore è detto \textbf{autostato}. Se lo stato iniziale del sistema è un autostato dell'operatore, la misura restituirà con certezza l'autovalore corrispondente. In altre parole, se un sistema si trova in un autostato \(\left| a_{n} \right\rangle\) con autovalore \(a_{n}\), allora il risultato della misura sarà proprio \(a_{n}\).

Se, invece, il sistema si trova in uno stato qualsiasi \(\left| \varphi \right\rangle\), questo può essere espresso come combinazione lineare degli autostati dell'operatore:

\[
\left| \varphi \right\rangle = \sum_{n} \varphi_{n} \left| a_{n} \right\rangle
\]

Gli autostati \(\left| a_{n} \right\rangle\) di un operatore di misura \(\hat{A}\)  formano una \textbf{base ortonormale} dello spazio di Hilbert, quindi ogni stato può essere scritto come loro combinazione lineare. Applicando l'operatore \(\hat{A}\) allo stato \(\left| \varphi \right\rangle\):

\[
\hat{A} \left| \varphi \right\rangle = \hat{A} \sum_{n} \varphi_{n} \left| a_{n} \right\rangle
\]

Per linearità dell'operatore sommatoria, è possibile scrivere:

\[
\hat{A}\left| \varphi \right\rangle = \sum_{n}^{}{\varphi_{n}\hat{A}\left| a_{n} \right\rangle}
\]

Dato che \(\left| a_{n} \right\rangle\) è autovettore dell'operatore applicato, \(\hat{A}\left| a_{n} \right\rangle = a_{n}\left| a_{n} \right\rangle\), risulta:

\[
\hat{A}\left| \varphi \right\rangle = \ \sum_{n}^{}{\varphi_{n}a_{n}\left| a_{n} \right\rangle}
\]

Ovvero, si è espresso lo stato \(\hat{A}\left| \varphi \right\rangle\) come combinazione degli autovettori dell'operatore \(\hat{A}\).

In meccanica quantistica non è possibile prevedere con certezza il risultato della misura, ma è possibile calcolare la \textbf{probabilità} che il sistema transiti dallo stato iniziale \(\varphi\) nello stato \(\left| a_{n} \right\rangle\), attraverso l'operatore di misura \(\hat{A}\). Questa è data dal modulo quadro del prodotto scalare:

\[
\left| \left\langle a_{n} \middle| \varphi \right\rangle \right|^2
\]

La notazione di Dirac (o bra-ket) distingue tra il vettore di stato nello spazio di Hilbert e il suo coniugato hermitiano:
\begin{itemize}
    \item Ket: La notazione $\left| \varphi \right\rangle$ è detta ket e rappresenta il vettore di stato del sistema nello spazio di Hilbert ($\mathbb{H}$).
    \item Bra: La notazione $\left\langle \varphi \right|$ è detta bra e rappresenta il vettore duale o coniugato hermitiano del ket $\left| \varphi \right\rangle$. Il bra appartiene allo spazio duale di $\mathbb{H}$ ($\mathbb{H}^*$).
\end{itemize}

Il prodotto scalare tra i due vettori è il bra-ket (la parola "bra-ket" deriva dalla contrazione delle due notazioni, $\left\langle \text{bra} \middle| \text{ket} \right\rangle$):

\[
\left\langle a_{n} \middle| \varphi \right\rangle
\]

Questo prodotto è un numero complesso che proietta lo stato iniziale $\left| \varphi \right\rangle$ sull'autostato $\left| a_{n} \right\rangle$, fornendo l'ampiezza di probabilità del risultato $a_n$.

Dirac assunse che il \textbf{valor medio} (o \textbf{valore di aspettazione}) di una grandezza fisica, associata all'operatore \(\hat{A}\), per un sistema nello stato \(\left| \varphi \right\rangle\), come:

\[
\left\langle \hat{A} \right\rangle = \left\langle \varphi \right| \hat{A} \left| \varphi \right\rangle
\]

Ad esempio, l'energia media \(\left\langle E \right\rangle\) si ottiene applicando l'operatore hamiltoniano:

\[
\left\langle E \right\rangle = \left\langle \varphi \right| \hat{H} \left| \varphi \right\rangle
\]

In generale, il vettore \(\varphi\) può essere espresso come combinazione lineare degli autostati dell'operatore di misura \(\hat{A}\):

\[
\left| \varphi \right\rangle = \sum_{n} \varphi_{n} \left| a_{n} \right\rangle
\]

Il valor medio dell'operatore di misura è, dunque:

\[
\left\langle \varphi \right|\hat{A}\left| \varphi \right\rangle = \left\langle \varphi \right|\hat{A}\sum_{n}^{}{\varphi_{n}\left| a_{n} \right\rangle
}
\]

Per la linearità si ha:

\[
\left\langle \varphi \right|\hat{A}\left| \varphi \right\rangle = \left\langle \varphi \right|\hat{A}\sum_{n}^{}{\varphi_{n}\left| a_{n} \right\rangle} = \left\langle \varphi \right|\sum_{n}^{}{\varphi_{n}\hat{A}\left| a_{n} \right\rangle}
\]

Anche il vettore bra può essere espresso mediante gli autovettori dell'operatore \(\hat{A}\), tuttavia, si rende necessario l'uso del complesso coniugato, in modo che il prodotto scalare sia reale:

\[
\left\langle \varphi \right| = \sum_{k} \varphi_{k}^{*} \left\langle a_{k} \right|
\]

Sostituendo nell'espresso per il valor medio si ha:

\[
\left\langle \varphi \right|\hat{A}\left| \varphi \right\rangle = \sum_{k}^{}{\varphi_{k}^{*}\left\langle a_{k} \right|}\sum_{n}^{}{\varphi_{n}\hat{A}\left| a_{n} \right\rangle} = \sum_{n}^{}{\sum_{k}^{}{\varphi_{k}^{*}\varphi_{n}\left\langle a_{k} \right|\hat{A}\left| a_{n} \right\rangle}}
\]

Poiché \(\hat{A} \left| a_{n} \right\rangle = a_{n} \left| a_{n} \right\rangle\), ovvero \(\left| a_{n} \right\rangle\) sono gli autostati dell'operatore \(\hat{A}\), si ha:

\[
\left\langle \varphi \right|\hat{A}\left| \varphi \right\rangle  = \sum_{n}^{}{\sum_{k}^{}{\varphi_{k}^{*}\varphi_{n}\left\langle a_{k} \right|\hat{A}\left| a_{n} \right\rangle}} = \sum_{n}^{}{\sum_{k}^{}{\varphi_{k}^{*}\varphi_{n}a_{n}\left\langle a_{k} \middle| a_{n} \right\rangle}}
\]

Si scelgono gli autostati dell'operatore \(\hat{A}\) in modo che siano ortonormali, ovvero sono ortogonali tra loro, mentre la loro norma è unitaria. Per cui risulta:

\[
\left\langle a_{k} \middle| a_{n} \right\rangle = \delta_{kn} =  \begin{cases}
1,\ \  & k = n \\
0,\ \  & k \neq n
\end{cases}
\]

Dunque, il valor medio si esprime come:

\[
\left\langle \varphi \right|\hat{A}\left| \varphi \right\rangle = \sum_{n}^{}{\left| \varphi_{n} \right|^{2}a_{n}}
\]

La probabilità che la misura dia come risultato \(a_{n}\) è quindi \(\left| \varphi_{n} \right|^2\).

Se la misura ha dato come risultato \(a_{k}\), il sistema si trova nello stato \(\left| a_{k} \right\rangle\). 

Se la misura ha dato come risultato \(a_{k}\), il sistema, dopo la misura, si trova nello stato corrispondente all'autovalore \(a_{k}\), ovvero nell'autostato \(\left| a_{k} \right\rangle\). Di conseguenza, lo stato precedente alla misura non è più conoscibile, ma quello successivo sì.

Se si ripete la misura in un tempo sufficientemente breve, il risultato sarà nuovamente \(a_{k}\), poiché il sistema si trova già in un autostato dell'operatore \(\hat{A}\). Tuttavia, se il tempo tra le due misure è troppo lungo, il sistema potrebbe decadere in uno stato diverso, rendendo il risultato imprevedibile.

\subsection{Evoluzione libera degli stati}\label{evoluzione-libera-degli-stati}

Se il sistema microscopico non viene perturbato, evolve liberamente secondo l'equazione differenziale deterministica di Schrödinger:

\[
\hat{H}\left| \varphi \right\rangle = j\hslash\dfrac{d}{dt}\left| \varphi \right\rangle
\]

Dove \(\hat{H}\) è l'hamiltoniana del sistema.
A seguito della misura dell'operatore \(\hat{H}\), il sistema microscopico collassa in un autostato \(\left| \varphi_{n} \right\rangle\) con autovalore \(E_{n}\), che rappresenta l'energia dello stato:

\[
\hat{H}\left| \varphi_{n} \right\rangle = E_{n}\left| \varphi_{n} \right\rangle
\]

L'equazione di Schrödinger si scrive:

\[
j\hslash\dfrac{d}{dt}\left| \varphi_{n} \right\rangle = \hat{H}\left| \varphi_{n} \right\rangle = E_{n}\left| \varphi_{n} \right\rangle
\]

La soluzione di questa equazione differenziale è:

\[
\left| \varphi_{n}(t) \right\rangle = \left| \varphi_{n}\left( t_{0} \right) \right\rangle\exp\left( -j\dfrac{E_{n}}{\hslash}\left( t - t_{0} \right) \right)
\]

Dove \(t_{0}\) è l'istante di misura.

Per uno stato generico \(\left| \varphi \right\rangle\), è possibile generalizzare questo risultato sviluppando lo stato iniziale come combinazione lineare degli autostati dell'operatore hamiltoniano, con coefficienti \(c_n = \left\langle \varphi_{n}(t_0) \middle| \varphi(t_0) \right\rangle\):

\[
\left| \varphi(t_0) \right\rangle = \sum_{n}^{}{c_{n}\left| \varphi_{n}\left( t_{0} \right) \right\rangle}
\]

L'evoluzione dello stato generico è data dall'evoluzione di ciascun autostato componente:

\[
\left| \varphi(t) \right\rangle = \sum_{n}^{}{c_{n}\left| \varphi_{n}\left( t_{0} \right) \right\rangle\exp\left\lbrack -j\dfrac{E_{n}}{\hslash}\left( t - t_{0} \right) \right\rbrack}
\]


È possibile giungere al \textbf{principio di indeterminazione di Heisenberg} osservando che, nello spazio vettoriale degli stati, gli operatori non soddisfano in generale la proprietà algebrica commutativa. Dati due operatori \(\hat{A}\) e \(\hat{B}\), si ha:

\[
\hat{A}\hat{B} \neq \hat{B}\hat{A}
\]

Ciò implica che la misura consecutiva di due grandezze fisiche diverse su un sistema microscopico non porta allo stesso risultato, indipendentemente dall'ordine con cui si effettuano le misure. Di conseguenza, non è possibile determinare simultaneamente quantità di moto e posizione, in accordo con il principio di Heisenberg:

\[
{\Delta}p{\Delta}x \geq \dfrac{\hslash}{2}
\]

In altre parole, una misura di posizione altera la misura della quantità di moto e viceversa. Questo effetto è legato alla necessità di perturbare il sistema per osservare una grandezza. Se si misura la posizione, non è più possibile ottenere altre informazioni sullo stato iniziale del sistema. Nota la posizione della particella, si dice che la \textbf{funzione d'onda collassa}, poiché la probabilità di ritrovare la particella in altre posizioni si annulla.

Nel contesto probabilistico della meccanica quantistica, il concetto deterministico di \textbf{traiettoria} perde di significato.

\subsection{Notazione di Dirac per lo spin}\label{notazione-di-dirac-per-lo-spin}

L'esperimento di Stern e Gerlach ha evidenziato che il momento angolare delle particelle microscopiche deve essere quantizzato. Affinché i risultati sperimentali siano in accordo con le previsioni teoriche, è necessario ammettere l'esistenza dell'operatore spin \(\hat{S}\), rappresentante il \textbf{momento angolare intrinseco} di una particella.

Nel caso di un protone o di un elettrone in un campo magnetico, gli stati possibili dello spin sono due, indicati con \(+\) e \(-\), in base alle due possibili orientazioni lungo la direzione \(z\).

Siano \(\left| + \right\rangle\) e \(\left| - \right\rangle\) i due stati dello spin, e \(\hat{S}_z\) l'operatore di spin lungo l'asse \(z\). Applicando la misura del momento magnetico \(\hat{S}_z\) a uno stato \(\left| \varphi \right\rangle\), si ha:


\[
\hat{S}_z \left| \varphi \right\rangle = s_z \left| \varphi \right\rangle
\]

dove \(s_z\) sono gli autovalori dell'operatore \(\hat{S}_z\), che possono assumere solo due valori:

\[
+ \dfrac{\hslash}{2}, \quad - \dfrac{\hslash}{2}
\]


I vettori \(\left| + \right\rangle\) e \(\left| - \right\rangle\) sono gli autovettori dell'operatore \({\hat{S}}_{z}\), dunque, risulta:

\[
\hat{S}_z \left| + \right\rangle = + \dfrac{\hslash}{2} \left| + \right\rangle, \quad
\hat{S}_z \left| - \right\rangle = - \dfrac{\hslash}{2} \left| - \right\rangle
\]

\begin{figure}[ht]
\centering
\includegraphics[width=2.72257in,height=2.79167in,alt={P2063\#yIS1}]{media/4_Quantiatica/image42.pdf}\caption{Orientazione dello spin}
\end{figure}

Un qualsiasi stato può essere espresso come combinazione lineare dei due autostati dell'operatore \({\hat{S}}_{z}\):

\[
\left| \varphi \right\rangle = \varphi_{+}\left| + \right\rangle + \varphi_{-}\left| - \right\rangle
\]

L'operazione di misura dello spin \({\hat{S}}_{z}\) lungo l'asse \(z\) è fondamentale poiché le proiezioni del momento magnetico della particella lungo gli altri assi si dimostra essere dipendente da \({\hat{S}}_{z}\).

Sia \(\left\langle \varphi \right|\) lo stato iniziale e \(\left| \psi \right\rangle\) lo stato finale. La quantità \(\left\langle \varphi \right| \hat{S}_z \left| \psi \right\rangle\) rappresenta la misura della proiezione del momento magnetico lungo \(z\).

Sia \(\left\langle \varphi \right|\) lo stato iniziale dello spin della particella e \(\left| \psi \right\rangle\) lo stato finale. La quantità \(\left\langle \varphi \right|{\hat{S}}_{z}\left| \psi \right\rangle\) indica rappresenta la misura della proiezione del momento magnetico lungo \(z\) di una particella inizialmente nello stato \(\left\langle \varphi \right|\). A fine della misura, la particella si porta nello stato \(\left| \psi \right\rangle\). Se lo stato iniziale coincide con l'autostato \(\left\langle + \right|\) e anche quello finale \(\left| + \right\rangle\), allora risulta:

\[
\left\langle + \right|{\hat{S}}_{z}\left| + \right\rangle = \left\langle + \right|\left( + \dfrac{\hslash}{2} \right)\left| + \right\rangle = + \dfrac{\hslash}{2}\left\langle + \middle| + \right\rangle
\]

Siccome gli autovettori sono ortonormali, risulta:

\[
\left\langle + \middle| + \right\rangle = 1
\]

Quindi, in definitiva, si ottiene:

\[
\left\langle + \right|{\hat{S}}_{z}\left| + \right\rangle = + \dfrac{\hslash}{2}
\]

Se lo stato iniziale coincide con l'autostato \(\left\langle - \right|\) e anche quello finale \(\left| - \right\rangle\), poiché gli autovettori sono ortonormali, si ha:

\[
\left\langle - \right|{\hat{S}}_{z}\left| - \right\rangle = \left\langle - \right| - \dfrac{\hslash}{2}\left| - \right\rangle = - \dfrac{\hslash}{2}\left\langle - \middle| - \right\rangle = - \dfrac{\hslash}{2}
\]

Se lo stato iniziale coincide con l'autostato \(\left\langle + \right|\) e anche quello finale \(\left| - \right\rangle\), risulta:

\[
\left\langle + \right|{\hat{S}}_{z}\left| - \right\rangle = \left\langle + \right| - \dfrac{\hslash}{2}\left| - \right\rangle = - \dfrac{\hslash}{2}\left\langle + \middle| - \right\rangle = 0
\]

Se lo stato iniziale coincide con l'autostato \(\left\langle - \right|\) e anche quello finale \(\left| + \right\rangle\), risulta:

\[
\left\langle - \right|{\hat{S}}_{z}\left| + \right\rangle = \left\langle - \right| + \dfrac{\hslash}{2}\left| + \right\rangle = + \dfrac{\hslash}{2}\left\langle - \middle| + \right\rangle = 0
\]

Siccome i vettori della base dell'operatore \({\hat{S}}_{z}\) sono due, è possibile associare il vettore \((1,0)^{T}\) all'autostato \(\left| + \right\rangle\) e \((0,1)^{T}\) all'autostato \(\left| - \right\rangle\). Dato che sono possibili quattro combinazioni, tra stato iniziale e finale dopo la misura, l'operatore \({\hat{S}}_{z}\) ha dimensione finita. È possibile associare una matrice alla trasformazione:

\[
\begin{pmatrix}
\left\langle + \right|{\hat{S}}_{z}\left| + \right\rangle & \left\langle + \right|{\hat{S}}_{z}\left| - \right\rangle \\
\left\langle - \right|{\hat{S}}_{z}\left| + \right\rangle & \left\langle - \right|{\hat{S}}_{z}\left| - \right\rangle
\end{pmatrix} = \begin{pmatrix}
 + \dfrac{\hslash}{2} & 0 \\
0 & - \dfrac{\hslash}{2}
\end{pmatrix} = \dfrac{\hslash}{2}\begin{pmatrix}
1 & 0 \\
0 & - 1
\end{pmatrix}
\]

La matrice:

\[
\boldsymbol{\sigma}_{z} = \begin{pmatrix}
1 & 0 \\
0 & - 1
\end{pmatrix}
\]

È detta matrice di Pauli e riassume i valori dello spin lungo l'asse \(z\) e consente di scrivere l'operatore di misura del momento magnetico come:

\[{\hat{S}}_{z} = \dfrac{\hslash}{2}{\boldsymbol{\sigma}}_{z}\]

Tramite la matrice di Pauli è possibile, inoltre, calcolare lo stato finale di una particella. Ad esempio, se lo stato iniziale di una particella è \(\left\langle + \right|\), lo stato finale può essere espresso come:

\[
{\hat{S}}_{z}\left| + \right\rangle = \dfrac{\hslash}{2}{\boldsymbol{\sigma}}_{z}\left| + \right\rangle = \dfrac{\hslash}{2}\begin{pmatrix}
1 & 0 \\
0 & - 1
\end{pmatrix}\left( \begin{array}{r}
1 \\
0
\end{array} \right) = \dfrac{\hslash}{2}\left( \begin{array}{r}
1 \\
0
\end{array} \right) = \dfrac{\hslash}{2}\left| + \right\rangle
\]

È possibile definire le matrici di Pauli anche per la misura del momento magnetico intrinseco lungo gli altri assi, come:

\[
\hat{S}_x = \dfrac{\hslash}{2}
\begin{pmatrix}
0 & 1 \\
1 & 0
\end{pmatrix}, \quad
\hat{S}_y = \dfrac{\hslash}{2}
\begin{pmatrix}
0 & -j \\
j & 0
\end{pmatrix}
\]

Se una particella si trova in uno stato \(\left| + \right\rangle\), autovettore di \({\hat{S}}_{z}\), allora lungo \(x\) si ha uno stato:

\[
{\hat{S}}_{x}\left| + \right\rangle = \dfrac{\hslash}{2}\begin{pmatrix}
0 & 1 \\
1 & 0
\end{pmatrix}\left( \begin{array}{r}
1 \\
0
\end{array} \right) = \dfrac{\hslash}{2}\left( \begin{array}{r}
0 \\
1
\end{array} \right) = \dfrac{\hslash}{2}\left| - \right\rangle
\]

Se, invece, lo stato iniziale è \(\left| - \right\rangle\), lungo l'asse \(x\), si ha:

\[{
\hat{S}}_{x}\left| - \right\rangle = \dfrac{\hslash}{2}\begin{pmatrix}
0 & 1 \\
1 & 0
\end{pmatrix}\left( \begin{array}{r}
0 \\
1
\end{array} \right) = \dfrac{\hslash}{2}\left( \begin{array}{r}
1 \\
0
\end{array} \right) = \dfrac{\hslash}{2}\left| + \right\rangle
\]

I vettori \(\left| + \right\rangle\) e \(\left| - \right\rangle\) non sono una base per l'operatore \({\hat{S}}_{x}\) poiché, l'applicazione di questo operatore a uno dei due vettori, non restituisce un vettore parallelo.

Analogamente, è possibile ripetere lo stesso discorso per l'operatore \({\hat{S}}_{y}\):

\[{\hat{S}}_{y}\left| + \right\rangle = \dfrac{\hslash}{2}\begin{pmatrix}
0 & - j \\
j & 0
\end{pmatrix}\left( \begin{array}{r}
1 \\
0
\end{array} \right) = j\dfrac{\hslash}{2}\left( \begin{array}{r}
0 \\
1
\end{array} \right) = j\dfrac{\hslash}{2}\left| - \right\rangle\]

\[{\hat{S}}_{y}\left| - \right\rangle = \dfrac{\hslash}{2}\begin{pmatrix}
0 & - j \\
j & 0
\end{pmatrix}\left( \begin{array}{r}
0 \\
1
\end{array} \right) = - j\dfrac{\hslash}{2}\left( \begin{array}{r}
1 \\
0
\end{array} \right) = - j\dfrac{\hslash}{2}\left| + \right\rangle\]

Anche per \({\hat{S}}_{y}\), i vettori \(\left| + \right\rangle\) e \(\left| - \right\rangle\) non sono una base per questo operatore.

Posizionando un campo magnetico \(\vec{B}\) diretto lungo l'asse \(z\) e misurando \(\hat{S}_z\), si ottiene un valore certo di \(S_z\) (\(\pm \hslash/2\)). Tuttavia, a causa della \textbf{non-commutatività} degli operatori di spin, ovvero:

\[
[\hat{S}_i, \hat{S}_j] \neq 0 \quad \text{per } i \neq j,
\]

lo stato risultante \(\left| + \right\rangle\) o \(\left| - \right\rangle\) non è un autostato di \(\hat{S}_x\) o \(\hat{S}_y\). Di conseguenza, non è possibile conoscere simultaneamente le componenti dello spin lungo gli assi \(x\) e \(y\): la misura di \(S_z\) rende le componenti \(S_x\) e \(S_y\) completamente \textbf{indeterminate}, con valore di aspettazione nullo e massima incertezza:

\[
\Delta S_x = \Delta S_y = \dfrac{\hslash}{2}
\]

Non è dunque possibile ricostruire tridimensionalmente il vettore momento intrinseco con precisione illimitata.

\subsection{Split dei livelli energetici}\label{split-dei-livelli-energetici}

Considerando un nucleo immerso in un campo magnetico \(\vec{B}\) diretto lungo l'asse \(z\), l'operatore hamiltoniano totale \(\hat{H}\) si compone dell'hamiltoniano imperturbato \(\hat{H}^{(0)}\) (energia cinetica e potenziale) e dell'energia potenziale magnetica \(\hat{U}\):

\[
\hat{H} = \hat{H}^{(0)} + \hat{U} = \left( \dfrac{{\hat{p}}^{2}}{2m} + \hat{V} \right) - \hat{\vec{\mu}} \cdot \vec{B}
\]

dove \(\hat{V}\) è l'energia potenziale coulombiana che agisce sul nucleo, \(\hat{\vec{\mu}}\) è l'operatore momento magnetico intrinseco e \(\hat{U}\) è l'energia potenziale magnetica:

\[
\hat{U} = - \hat{\mu} \cdot \vec{B}
\]

\begin{figure}[ht]
\centering
\includegraphics[width=1.94783in,height=2.39583in,alt={P2108\#yIS1}]{media/4_Quantiatica/image43.pdf}
\caption{Nucleo immerso in un campo magnetico}
\end{figure}

Proiettando lungo \(z\), il momento magnetico è legato all'operatore di spin dal rapporto giromagnetico \(\gamma\): \(\hat{\mu}_z = \gamma \hat{S}_z\). Poiché il campo magnetico ha solo componente lungo \(\hat{\imath}_z\), ovvero \(\vec{B} = B \hat{\imath}_z\), l'hamiltoniano si scrive:

\[
\hat{H} = \left( \dfrac{{\hat{p}}^{2}}{2m} + \hat{V} \right) - \gamma{\hat{S}}_{z}B
\]

Sia \(E_n^{(0)}\) l'autovalore di \(\hat{H}^{(0)}\) per uno stato orbitale \(\left| \varphi_n \right\rangle\), tale che:

\[
\left( \dfrac{{\hat{p}}^{2}}{2m} + \hat{V} \right)\left| \varphi_{n} \right\rangle = E_{n}^{(0)}\left| \varphi_{n} \right\rangle
\]

Assumendo che lo stato totale sia separabile in una parte orbitale e una di spin, \(\left| \psi_{n,s} \right\rangle = \left| \varphi_n \right\rangle \otimes \left| s_z \right\rangle\), l'applicazione dell'hamiltoniano produce:

\[
\hat{H}\left| \psi_{n,s} \right\rangle = E_{n}^{(0)}\left| \psi_{n,s} \right\rangle - \gamma B{\hat{S}}_{z}\left| \psi_{n,s} \right\rangle
\]

L'operatore \(\hat{S}_z\) possiede solo due autovalori per lo spin \(1/2\):

\[
s_z = \pm \dfrac{\hslash}{2}
\]

Sostituendo tali autovalori, l'energia totale si divide in due livelli:

\[
E = E_n^{(0)} - \gamma B s_z =
\begin{cases}
E_{\text{low}} = E_n^{(0)} - \dfrac{\hslash}{2} \gamma B & \text{per } s_z = +\dfrac{\hslash}{2} \text{ (Spin Up, } \left| + \right\rangle \text{)} \\
E_{\text{high}} = E_n^{(0)} + \dfrac{\hslash}{2} \gamma B & \text{per } s_z = -\dfrac{\hslash}{2} \text{ (Spin Down, } \left| - \right\rangle \text{)}
\end{cases}
\]

Per effetto del campo magnetico diretto lungo \(z\), ogni autovalore dell'energia si divide in due livelli energetici che differiscono di:

\[
\Delta E = E_{\text{high}} - E_{\text{low}} = \hslash \gamma B
\]

Questo fenomeno è noto come \textbf{split dei livelli energetici} o \textbf{effetto Zeeman nucleare}.

\begin{figure}[ht]
\centering
\includegraphics[width=5.59819in,height=1.7684in,alt={P2123\#yIS1}]{media/4_Quantiatica/image44.pdf}\caption{Split di livelli energetici di un nucleo immerso in un campo magnetico}
\end{figure}

L'hamiltoniano di un nucleo in un campo magnetico diretto lungo l'asse \(z\) contiene i termini \(\pm \dfrac{\hslash}{2} \gamma B\). Gli spin si orientano quindi in direzione parallela o antiparallela rispetto al campo magnetico applicato.

Si può trascurare il termine \(\hat{p}^{2}/2m\) poiché il nucleo, avendo massa elevata, può essere considerato fermo. Se il nucleo è in equilibrio, anche il potenziale coulombiano può essere trascurato. L'hamiltoniano si riduce quindi a:

\[
\hat{H}\left| \varphi_{n} \right\rangle = - \gamma B{\hat{S}}_{z}\left| \varphi_{n} \right\rangle
\]

Se lo stato del nucleo è \(\left| + \right\rangle\), allora:

\[
\hat{H}\left| + \right\rangle = - \gamma B{\hat{S}}_{z}\left| + \right\rangle = - \dfrac{\hslash}{2}\gamma B
\]

Per $\gamma > 0$ come nel protone, lo spin è parallelo al campo magnetico poiché il nucleo possiede la minima energia. Al contrario, se lo stato è \(\left| - \right\rangle\), risulta:

\[
\hat{H}\left| - \right\rangle = - \gamma B{\hat{S}}_{z}\left| - \right\rangle = \dfrac{\hslash}{2}\gamma B
\]

 Per $\gamma > 0$, lo spin è antiparallelo al campo magnetico poiché il suo nucleo possiede energia massima.

\subsection{Energia media di un nucleo in base allo spin}\label{energia-media-di-un-nucleo-in-base-allo-spin}

Uno stato qualsiasi del sistema microscopico è descritto dal vettore ket \(\left| \psi \right\rangle\), che può essere decomposto come combinazione lineare degli autostati dell'operatore spin \(\hat{S}_z\):

\[
\hat{H} \left| \psi \right\rangle = C_{+} \hat{H} \left| + \right\rangle + C_{-} \hat{H} \left| - \right\rangle
\]

Si calcola l'energia del sistema applicando l'operatore hamiltoniano:

\[
\hat{H}\left| \psi \right\rangle = \hat{H}\left( C_{+}\left| + \right\rangle + C_{-}\left| - \right\rangle \right) = C_{+}\hat{H}\left| + \right\rangle + C_{-}\hat{H}\left| - \right\rangle
\]

Si valuta il valor medio dell'energia del nucleo, mediante la definizione \(\left\langle \psi \right|\hat{H}\left| \psi \right\rangle\). Sostituendo l'equazione ottenuta per \(\hat{H}\left| \psi \right\rangle\), si ha:

\[
\left\langle \psi \right|\hat{H}\left| \psi \right\rangle = \left\langle \psi \right|\left( C_{+}\hat{H}\left| + \right\rangle + C_{-}\hat{H}\left| - \right\rangle \right) = C_{+}\left\langle \psi \right|\hat{H}\left| + \right\rangle + C_{-}\left\langle \psi \right|\hat{H}\left| - \right\rangle
\]

L'autostato bra \(\left\langle \psi \right|\) può essere espresso come combinazione lineare degli autostati dell'operatore \({\hat{S}}_{z}\), mediante coefficienti complessi e coniugati:

\[
\left\langle \psi \right| = C_{+}^{*}\left\langle + \right| + C_{-}^{*}\left\langle - \right|
\]

Sostituendo nella relazione per \(\left\langle \psi \right|\hat{H}\left| \psi \right\rangle\), si ha:

\[
\left\langle \psi \right|\hat{H}\left| \psi \right\rangle = C_{+}\left( C_{+}^{*}\left\langle + \right| + C_{-}^{*}\left\langle - \right| \right)\hat{H}\left| + \right\rangle + C_{-}\left( C_{+}^{*}\left\langle + \right| + C_{-}^{*}\left\langle - \right| \right)\hat{H}\left| - \right\rangle
\]

Svolgendo i prodotti, si ottiene:

\[
\left\langle \psi \right|\hat{H}\left| \psi \right\rangle = C_{+}C_{+}^{*}\left\langle + \right|\hat{H}\left| + \right\rangle + C_{+}C_{-}^{*}\left\langle - \right|\hat{H}\left| + \right\rangle + C_{-}C_{+}^{*}\left\langle + \right|\hat{H}\left| - \right\rangle + C_{-}C_{-}^{*}\left\langle - \right|\hat{H}\left| - \right\rangle
\]

Dove \(C_{+}C_{+}^{*} = \left| C_{+} \right|^{2}\) e \(C_{-}C_{-}^{*} = \left| C_{-} \right|^{2}\), dunque, è possibile scrive:

\[
\left\langle \psi \right|\hat{H}\left| \psi \right\rangle = \left| C_{+} \right|^{2}\left\langle + \right|\hat{H}\left| + \right\rangle + C_{+}C_{-}^{*}\left\langle - \right|\hat{H}\left| + \right\rangle + C_{-}C_{+}^{*}\left\langle + \right|\hat{H}\left| - \right\rangle + \left| C_{-} \right|^{2}\left\langle - \right|\hat{H}\left| - \right\rangle
\]

Per un nucleo immesso in un campo magnetico di ampiezza \(B_{0}\) diretto lungo \(z\), l'operatore hamiltoniano, trascurano la velocità del nucleo e il campo coulombiano, può essere espresso come:

\[
\hat{H} = - \gamma B_0 \hat{S}_z
\]

Il valor medio dell'energia del nucleo, può essere espressa come:

\[
\left\langle \psi \right|\hat{H}\left| \psi \right\rangle = - \gamma B_{0}\left( \left| C_{+} \right|^{2}\left\langle + \right|{\hat{S}}_{z}\left| + \right\rangle + C_{+}C_{-}^{*}\left\langle - \right|{\hat{S}}_{z}\left| + \right\rangle + C_{-}C_{+}^{*}\left\langle + \right|{\hat{S}}_{z}\left| - \right\rangle + \left| C_{-} \right|^{2}\left\langle - \right|{\hat{S}}_{z}\left| - \right\rangle \right)
\]

Dalla definizione di matrice di Pauli, risulta:

\[{\hat{S}}_{z}\left| + \right\rangle = \dfrac{\hslash}{2}{\boldsymbol{\sigma}}_{z}\left| + \right\rangle = \dfrac{\hslash}{2}\begin{pmatrix}
1 & 0 \\
0 & - 1
\end{pmatrix}\left( \begin{array}{r}
1 \\
0
\end{array} \right) = \dfrac{\hslash}{2}\left( \begin{array}{r}
1 \\
0
\end{array} \right) = \dfrac{\hslash}{2}\left| + \right\rangle\]

\[{\hat{S}}_{z}\left| - \right\rangle = \dfrac{\hslash}{2}{\boldsymbol{\sigma}}_{z}\left| - \right\rangle = \dfrac{\hslash}{2}\begin{pmatrix}
1 & 0 \\
0 & - 1
\end{pmatrix}\left( \begin{array}{r}
0 \\
1
\end{array} \right) = \dfrac{\hslash}{2}\left( \begin{array}{r}
0 \\
 - 1
\end{array} \right) = - \dfrac{\hslash}{2}\left| - \right\rangle\]

Inoltre, i vettori \(\left| + \right\rangle\) e \(\left| - \right\rangle\) sono ortonormali, dunque, risulta che:

\[\left\langle + \right|{\hat{S}}_{z}\left| + \right\rangle = \left\langle + \right|\dfrac{\hslash}{2}\left| + \right\rangle = \dfrac{\hslash}{2}\left\langle + \middle| + \right\rangle = \dfrac{\hslash}{2}\]

\[\left\langle - \right|{\hat{S}}_{z}\left| - \right\rangle = - \left\langle - \right|\dfrac{\hslash}{2}\left| - \right\rangle = - \dfrac{\hslash}{2}\left\langle - \middle| - \right\rangle = - \dfrac{\hslash}{2}\]

\[\left\langle - \right|{\hat{S}}_{z}\left| + \right\rangle = \left\langle - \right|\dfrac{\hslash}{2}\left| + \right\rangle = \dfrac{\hslash}{2}\left\langle - \middle| + \right\rangle = 0\]

\[\left\langle + \right|{\hat{S}}_{z}\left| - \right\rangle = - \left\langle + \right|\dfrac{\hslash}{2}\left| - \right\rangle = - \dfrac{\hslash}{2}\left\langle + \middle| - \right\rangle = 0\]

Si ottiene:

\[\left\langle \psi \right|\hat{H}\left| \psi \right\rangle = - \gamma B_{0}\left( \dfrac{\hslash}{2}\left| C_{+} \right|^{2} - \dfrac{\hslash}{2}\left| C_{-} \right|^{2} \right) = - \gamma B_{0}\dfrac{\hslash}{2}\left( \left| C_{+} \right|^{2} - \left| C_{-} \right|^{2} \right)\]

Il parametro \(\left| C_{+} \right|^{2}\) rappresenta la probabilità che il nucleo sia nello stato \(\left| + \right\rangle\) alla fine del processo di misura. Analogamente, \(\left| C_{-} \right|^{2}\) rappresenta la probabilità che il nucleo sia nello stato \(\left| - \right\rangle\).

In un insieme di \(N\) nuclei (come in un campione NMR), se si assume che la probabilità di trovare un singolo nucleo negli autostati \(|+\rangle\) o \(|-\rangle\) sia data dalle popolazioni termiche di equilibrio, il numero atteso di nuclei in ciascuno stato è, per lo stato \(|+\rangle\):

\[
N_{+} = N P_{+} = N\left| C_{+} \right|^{2}
\]

per lo stato \(|-\rangle\), invece:

\[
N_{-} = N P_{-} = N\left| C_{-} \right|^{2}
\]

Sia \(\Delta N = N_{+} - N_{-}\) la differenza di popolazione tra gli spin nello stato \(\left| + \right\rangle\) e quelli nello stato \(\left| - \right\rangle\):

\[
\Delta N = N_{+} - N_{-} = N\left| C_{+} \right|^{2} - N\left| C_{-} \right|^{2} = N \left( \left| C_{+} \right|^{2} - \left| C_{-} \right|^{2} \right)
\]

Dividendo per \(N\), si ottiene l'eccesso di popolazione normalizzato:

\[
\dfrac{\Delta N}{N} = \left| C_{+} \right|^{2} - \left| C_{-} \right|^{2}
\]

Sostituendo questa relazione nell'espressione del valor medio dell'energia \(\left\langle \psi \right|\hat{H}\left| \psi \right\rangle\), si ottiene:

\[
\left\langle \psi \right|\hat{H}\left| \psi \right\rangle = - \gamma B_{0}\dfrac{\hslash}{2}\dfrac{\Delta N}{N}
\]

L'energia media dei nuclei è data dalla differenza di nuclei nello stato \(\left| + \right\rangle\) rispetto a quelli nello stato \(\left| - \right\rangle\), rapportato al numero totale dei nuclei.

\subsection{Evoluzione temporale dello stato spin up}\label{evoluzione-temporale-dello-stato-left-mathbf-rightrangle}

A seguito della misura, il sistema evolve secondo una legge deterministica. Lo stato quantico ha un'evoluzione temporale descritta da:

\[
\left| \varphi_{n}(t) \right\rangle = \left| \varphi_{n}\left( t_{0} \right) \right\rangle\exp\left( -j\dfrac{E_{n}}{\hslash}\left( t - t_{0} \right) \right)
\]

Se il sistema si trova nello stato \(\left| \mathbf{+} \right\rangle\), l'energia associata è:

\[
E^{+} = - \dfrac{\hslash}{2}\gamma B_{0}
\]

Per cui, l'evoluzione temporale dello stato è:

\[
\left| + (t) \right\rangle = \left| + \left( t_{0} \right) \right\rangle\exp\left( - j \dfrac{1}{\hslash} \left( - \dfrac{\hslash}{2}\gamma B_{0} \right) \left( t - t_{0} \right) \right) = \left| + \left( t_{0} \right) \right\rangle\exp\left( + j\dfrac{\gamma B_{0}}{2}\left( t - t_{0} \right) \right)
\]

Analogamente, nello stato \(\left| \mathbf{-} \right\rangle\) il sistema possiede un energia data da:

\[E^{-} = \dfrac{\hslash}{2}\gamma B_{0}\]

Per cui, l'andamento dello stato è:

\[
\left| - (t) \right\rangle = \left| - \left( t_{0} \right) \right\rangle\exp\left( - j \dfrac{1}{\hslash} \left( + \dfrac{\hslash}{2}\gamma B_{0} \right) \left( t - t_{0} \right) \right) = \left| - \left( t_{0} \right) \right\rangle\exp\left( - j\dfrac{\gamma B_{0}}{2}\left( t - t_{0} \right) \right)
\]

Un qualsiasi stato \(\left| \psi \right\rangle\) del sistema può essere espresso come combinazione lineare degli autostati dell'operatore \({\hat{S}}_{z}\):

\[
\left| \psi(t) \right\rangle = C_{+}\left| + \left( t_{0} \right) \right\rangle\exp\left( + j\dfrac{\gamma B_{0}}{2}\left( t - t_{0} \right) \right) + C_{-}\left| - \left( t_{0} \right) \right\rangle\exp\left( - j\dfrac{\gamma B_{0}}{2}\left( t - t_{0} \right) \right)
\]

\subsection{Valore medio del momento magnetico sull'asse longitudinale}\label{valore-medio-del-momento-magnetico-su-mathbfz}

L'operatore momento angolare su \(z\), \({\hat{\mu}}_{z}\), è legato all'operatore di spin \({\hat{S}}_{z}\), dalla relazione:

\[{\hat{\mu}}_{z} = \gamma{\hat{S}}_{z}\]

È possibile calcolare il valor medio di tale operatore mediante la definizione:

\[\left\langle \psi \right|{\hat{\mu}}_{z}\left| \psi \right\rangle = \left\langle \psi \right|\gamma{\hat{S}}_{z}\left| \psi \right\rangle = \gamma\left\langle \psi \right|{\hat{S}}_{z}\left| \psi \right\rangle\]

È possibile esprimere gli stati come sovrapposizione degli autostati dell'operatore spin:

\[\left\langle \psi \right| = C_{+}^{*}\left\langle + \right| + C_{-}^{*}\left\langle - \right|\]

\[\left| \psi \right\rangle = C_{+}\left| + \right\rangle + C_{-}\left| - \right\rangle\]

Sostituendo queste due relazioni nel valor medio, si ottiene:

\[\left\langle \psi \right|{\hat{\mu}}_{z}\left| \psi \right\rangle = \left( C_{+}^{*}\left\langle + \right| + C_{-}^{*}\left\langle - \right| \right){\hat{S}}_{z}\left( C_{+}\left| + \right\rangle + C_{-}\left| - \right\rangle \right)\]

Si svolgono i prodotti:

\[\left\langle \psi \right|{\hat{\mu}}_{z}\left| \psi \right\rangle = \gamma\left( {C_{+}C}_{+}^{*}\left\langle + \right|{\hat{S}}_{z}\left| + \right\rangle + C_{+}C_{-}^{*}\left\langle - \right|{\hat{S}}_{z}\left| + \right\rangle + C_{-}C_{+}^{*}\left\langle + \right|{\hat{S}}_{z}\left| - \right\rangle + C_{-}C_{-}^{*}\left\langle - \right|{\hat{S}}_{z}\left| - \right\rangle \right)\]

Dove:
\[
\begin{cases}
    \left\langle + \right|{\hat{S}}_{z}\left| + \right\rangle = \dfrac{\hslash}{2}\left\langle + \middle| + \right\rangle = \dfrac{\hslash}{2} \\
    \left\langle - \right|{\hat{S}}_{z}\left| - \right\rangle = - \dfrac{\hslash}{2}\left\langle - \middle| - \right\rangle = - \dfrac{\hslash}{2} \\
     \left\langle + \right|{\hat{S}}_{z}\left| - \right\rangle = - \dfrac{\hslash}{2}\left\langle + \middle| - \right\rangle = 0 \\
     \left\langle - \right|{\hat{S}}_{z}\left| + \right\rangle = \dfrac{\hslash}{2}\left\langle - \middle| + \right\rangle = 0    
\end{cases}
\]

Per cui si ha:

\[\left\langle \psi \right|{\hat{\mu}}_{z}\left| \psi \right\rangle = \gamma\dfrac{\hslash}{2}\left( \left| C_{+} \right|^{2} - \left| C_{-} \right|^{2} \right)\]

\subsection{Andamento temporale dello stato momento magnetico lungo l'asse trasversale}\label{andamento-temporale-dello-stato-momento-magnetico-lungo-mathbfx}

Si vuole calcolare l'energia media nel tempo dell'operatore momento magnetico lungo \(x\), \({\hat{\mu}}_{x}\). Siccome il momento magnetico possiede solo due autostati, può assumere solamente due valori. Lo stato energetico dei due autostati \(\left| + (t) \right\rangle\) e \(\left| - (t) \right\rangle\) sono rispettivamente:

\[E^{-} = \dfrac{\hslash}{2}\gamma B_{0},\ \ E^{+} = - \dfrac{\hslash}{2}\gamma B_{0}\]

L'evoluzione temporale dei due autostati stazionari è:

\[\left| + (t) \right\rangle = \left| + \left( t_{0} \right) \right\rangle\exp\left\lbrack - j\dfrac{E^{+}}{\hslash}\left( t - t_{0} \right) \right\rbrack\]

\[\left| - (t) \right\rangle = \left| - \left( t_{0} \right) \right\rangle\exp\left\lbrack - j\dfrac{E^{-}}{\hslash}\left( t - t_{0} \right) \right\rbrack\]

Sia \(\left| \psi(t) \right\rangle\) un qualsiasi stato, esso può essere ottenuto come combinazione lineare dei due autostati stazionari:

\[\left| \psi\left( t_{0} \right) \right\rangle = C_{+}\left| + \left( t_{0} \right) \right\rangle + C_{-}\left| - \left( t_{0} \right) \right\rangle\]

Dunque, l'evoluzione temporale è:

\[\left| \psi(t) \right\rangle = C_{+}\left| + \left( t_{0} \right) \right\rangle\exp\left\lbrack - j\dfrac{E^{+}}{\hslash}\left( t - t_{0} \right) \right\rbrack + C_{-}\left| - \left( t_{0} \right) \right\rangle\exp\left\lbrack - j\dfrac{E^{-}}{\hslash}\left( t - t_{0} \right) \right\rbrack\]

Si valuta l'energia media dell'operatore \({\hat{\mu}}_{x}\), sostituendo la combinazione per \(\left| \psi(t) \right\rangle\), si ha:

\[\left\langle \psi(t) \right|{\hat{\mu}}_{x}\left| \psi(t) \right\rangle = \left\langle \psi(t) \right|{\hat{\mu}}_{x}C_{+}\left| + \left( t_{0} \right) \right\rangle\exp\left\lbrack - j\dfrac{E^{+}}{\hslash}\left( t - t_{0} \right) \right\rbrack + \left\langle \psi(t) \right|{\hat{\mu}}_{x}C_{-}\left| - \left( t_{0} \right) \right\rangle\exp\left\lbrack - j\dfrac{E^{-}}{\hslash}\left( t - t_{0} \right) \right\rbrack\]

L'operatore momento magnetico lungo \(x\) può essere espresso in termini dell'operatore momento angolare intrinseco lungo lo stesso asse, \({\hat{S}}_{x}\):

\[{\hat{\mu}}_{x} = \gamma{\hat{S}}_{x}\]

Mediante la matrice di Pauli è possibile valutare il risultato dell'applicazione di \({\hat{S}}_{x}\) agli autostati \(\left| + \right\rangle\) e \(\left| - \right\rangle\):

\[{\hat{\mu}}_{x}\left| + \right\rangle = \gamma{\hat{S}}_{x}\left| + \right\rangle = \gamma\dfrac{\hslash}{2}\begin{pmatrix}
0 & 1 \\
1 & 0
\end{pmatrix}\left( \begin{array}{r}
1 \\
0
\end{array} \right) = \gamma\dfrac{\hslash}{2}\left( \begin{array}{r}
0 \\
1
\end{array} \right) = \gamma\dfrac{\hslash}{2}\left| - \right\rangle\]

\[{\hat{\mu}}_{x}\left| - \right\rangle = \gamma{\hat{S}}_{x}\left| - \right\rangle = \gamma\dfrac{\hslash}{2}\begin{pmatrix}
0 & 1 \\
1 & 0
\end{pmatrix}\left( \begin{array}{r}
0 \\
1
\end{array} \right) = \gamma\dfrac{\hslash}{2}\left( \begin{array}{r}
1 \\
0
\end{array} \right) = \gamma\dfrac{\hslash}{2}\left| + \right\rangle\]

Per cui, l'energia media può essere scritta come:

\[\left\langle \psi(t) \right|{\hat{\mu}}_{x}\left| \psi(t) \right\rangle = \gamma\dfrac{\hslash}{2}\left\{ C_{+}\left\langle \psi(t) \middle| - \left( t_{0} \right) \right\rangle\exp\left\lbrack - j\dfrac{E^{+}}{\hslash}\left( t - t_{0} \right) \right\rbrack + C_{-}\left\langle \psi(t) \middle| + \left( t_{0} \right) \right\rangle\exp\left\lbrack - j\dfrac{E^{-}}{\hslash}\left( t - t_{0} \right) \right\rbrack \right\}\]

Il vettore bra può essere espresso come combinazione lineare degli autostati di \({\hat{S}}_{z}\):

\[\left\langle \psi(t) \right| = C_{+}^{*}\left\langle + (t) \right| + C_{-}^{*}\left\langle - (t) \right| = C_{+}^{*}\left\langle + \left( t_{0} \right) \right|\exp\left\lbrack j\dfrac{E^{+}}{\hslash}\left( t - t_{0} \right) \right\rbrack + C_{-}^{*}\left\langle - \left( t_{0} \right) \right|\exp\left\lbrack j\dfrac{E^{-}}{\hslash}\left( t - t_{0} \right) \right\rbrack\]

Sostituendo questa relazione nel calcolo dell'energia media, si ottengono i risultati:

\[C_{+}C_{+}^{*}\left\langle + \left( t_{0} \right) \middle| - \left( t_{0} \right) \right\rangle\exp\left\lbrack - j\dfrac{E^{+}}{\hslash}\left( t - t_{0} \right) \right\rbrack\exp\left\lbrack j\dfrac{E^{+}}{\hslash}\left( t - t_{0} \right) \right\rbrack = 0\]

\[C_{+}C_{-}^{*}\left\langle - \left( t_{0} \right) \middle| - \left( t_{0} \right) \right\rangle\exp\left\lbrack - j\dfrac{E^{+}}{\hslash}\left( t - t_{0} \right) \right\rbrack\exp\left\lbrack j\dfrac{E^{-}}{\hslash}\left( t - t_{0} \right) \right\rbrack = C_{+}C_{-}^{*}\exp\left\lbrack - j\dfrac{E^{+} - E^{-}}{\hslash}\left( t - t_{0} \right) \right\rbrack\]

\[C_{-}C_{+}^{*}\left\langle - \left( t_{0} \right) \middle| - \left( t_{0} \right) \right\rangle\exp\left\lbrack - j\dfrac{E^{-}}{\hslash}\left( t - t_{0} \right) \right\rbrack\exp\left\lbrack j\dfrac{E^{+}}{\hslash}\left( t - t_{0} \right) \right\rbrack = C_{-}C_{+}^{*}\exp\left\lbrack j\dfrac{E^{+} - E^{-}}{\hslash}\left( t - t_{0} \right) \right\rbrack\]

\[C_{-}^{*}C_{+}^{*}\left\langle - \left( t_{0} \right) \middle| + \left( t_{0} \right) \right\rangle\exp\left\lbrack - j\dfrac{E^{-}}{\hslash}\left( t - t_{0} \right) \right\rbrack\exp\left\lbrack j\dfrac{E^{-}}{\hslash}\left( t - t_{0} \right) \right\rbrack = 0\]

L'energia media del momento magnetico lungo \(x\) è, dunque:

\[\left\langle \psi(t) \right|{\hat{\mu}}_{x}\left| \psi(t) \right\rangle = \gamma\dfrac{\hslash}{2}\left\{ C_{+}C_{-}^{*}\exp\left\lbrack - j\dfrac{E^{+} - E^{-}}{\hslash}\left( t - t_{0} \right) \right\rbrack + C_{-}C_{+}^{*}\exp\left\lbrack j\dfrac{E^{+} - E^{-}}{\hslash}\left( t - t_{0} \right) \right\rbrack \right\}\]

Si pone \(C_{+}C_{-}^{*} = A\exp{j\beta}\), dove compaiono esplicitamente modulo, \(A = \left| C_{+}C_{-}^{*} \right|\), e fase, \(\angle C_{+}C_{-}^{*}\), del prodotto \(C_{+}C_{-}^{*}\). La relazione per il valor medio del momento magnetico lungo \(x\) può essere scritta tenendo conto delle proprietà dei numeri complessi: la somma di un numero complesso e del suo coniugato restituisce il doppio della parte reale del numero stesso. Questa relazione può essere scritta anche come:

\[\left\langle \psi(t) \right|{\hat{\mu}}_{x}\left| \psi(t) \right\rangle = \gamma\hslash Re\left\{ C_{-}C_{+}^{*}\exp\left\lbrack j\dfrac{E^{+} - E^{-}}{\hslash}\left( t - t_{0} \right) \right\rbrack \right\}\]

Si pone:

\[
\omega_0 = \dfrac{E^{-} - E^{+}}{\hslash} 
\]

Sostituendo le espressioni per le due energie, \(E^{-} = \hslash/2\gamma B_{0}\) e \(E^{+} = - \hslash/2\gamma B_{0}\), si ottiene:

\[
\omega_0 = \dfrac{E^{-} - E^{+}}{\hslash} = \dfrac{1}{\hslash}\left( \dfrac{\hslash}{2}\gamma B_{0} - \left( - \dfrac{\hslash}{2}\gamma B_{0} \right) \right) = \gamma B_0
\]

\(\omega_{0}\) è detta pulsazione di Larmor, quantità positiva se \(\gamma>0\).

In definitiva, il valor medio del momento magnetico lungo \(x\) è:

\[\left\langle \psi(t) \right|{\hat{\mu}}_{x}\left| \psi(t) \right\rangle = \gamma\hslash A\cos\left( \beta - \omega\left( t - t_{0} \right) \right)\]

\subsection{Transizioni di stato}\label{transizioni-di-stato}

Gli autostati dell'operatore di misura \(\hat{A}\) presentano energie costanti, rappresentante dagli autovalori dell'operatore di misura applicato.

Finché il sistema si trova in un autostato stazionario, ovvero il sistema non è perturbato, l'evoluzione temporale avviene in maniera determinista, secondo l'equazione:

\[\left| \psi(t) \right\rangle = \left| \psi\left( t_{0} \right) \right\rangle\exp\left( - j\dfrac{E_{n}}{\hslash}\left( t - t_{0} \right) \right)\]

In seguito a una misura, mediante l'applicazione dell'operatore $\hat{A}$), lo stato del sistema collassa in un autostato di $\hat{A}$, e la probabilità che lo stato $\left| \psi \right\rangle$ sia trovato nell'autostato $\left| \varphi \right\rangle$ è data da $|\langle \varphi | \psi \rangle|^2$.

Il passaggio da uno stato energetico all'altro, ovvero la transizione di stato, può avvenire in modo guidato a seguito di una sollecitazione elettromagnetica di un'opportuna frequenza. L'efficacia di tale transizione è determinata dall'elemento di matrice di transizione, come ad esempio $\langle \varphi | \hat{\mu}_x | \psi \rangle$ per l'interazione magnetica trasversale.

Se il sistema quantico si trova in un campo elettromagnetico oscillante, è possibile sviluppare una soluzione dell'equazione di Schrödinger tempo dipendente:

\[j\hslash\dfrac{\partial\psi}{\partial t} = \hat{H}\psi\]

I metodi approssimata per poter risolvere questa equazione sono abbastanza complessi. Questi metodi sono basati sull'assunzione che la soluzione approssimata al primo ordine, ottenuta considerando un operatore hamiltoniano senza perturbazione \({\hat{H}}_{0}\) a cui si aggiunge un termine del primo ordine dello sviluppo della perturbazione, \({\hat{H}}_{1}(t)\). Questo termine aggiuntivo rappresenta la probabilità di transizione dallo stato a energia \(l\) verso quello a energia \(k\). In simboli è possibile scrivere:

\[H_{kl} = \left\langle k \right|{\hat{H}}_{1}\left| l \right\rangle\]

I termini che esprimono le probabilità di transizione per unità di tempo, al tempo \(t\), da un livello energetico all'altro sono esprimibili secondo la \textbf{meccanica quantistica perturbativa nel tempo} (\textit{time-dependent perturbation theory}):

\[
W_{a \rightarrow b} =\dfrac{1}{\hslash^{2}}\left|\int_{-\infty}^{+\infty}H_{ab}\left(\tau\right)\,\exp\left(
j\dfrac{E_{b} - E_{a}}{\hslash}\,\tau\right)d\tau\right|^{2}
\]

dove:
\begin{itemize}
    \item \( W_{a \rightarrow b} \): probabilità di transizione (proporzionale alla probabilità che il sistema passi dallo stato \( a \) allo stato \( b \));
    \item \( H_{ab}(\tau) = \langle b | H'(\tau) | a \rangle \): elemento di matrice del termine perturbativo tra gli stati \( |a\rangle \) e \( |b\rangle \).
    \item \( E_a, E_b \): energie degli stati iniziale e finale del sistema non perturbato;
    \item \( \hslash \): costante di Planck ridotta, che lega energia e frequenza quantistica (\( E = \hslash \omega \)).
    \item l'esponenziale: fattore di fase oscillante dovuto alla differenza di energia tra i due stati.
\end{itemize}

La probabilità di transizione tra lo stato \(a\) a quello \(b\) dipende dalla trasformata di Fourier dell'hamiltoniana associata alla radiazione oscillante. \({\Delta}E = E_{b} - E_{a}\) è la differenza di energia tra i due stati. Se la perturbazione \({\hat{H}}_{1}\) è costituita da un campo di frequenze, con spettro concentrato intorno alla frequenza:

\[\omega_{ab} = \dfrac{E_{b} - E_{a}}{\hslash}\]

La probabilità \(W_{a \rightarrow b}\) risulta essere molto più alta rispetto al caso in cui lo spettro della perturbazione non è concentrato intorno a questa frequenza ma è disperso.


\begin{figure}[ht]
\centering
\includegraphics[width=\linewidth]{media/4_Quantiatica/image45.pdf}\caption{Spettro concentrato in alto e più disperso in basso}
\end{figure}

La probabilità i transizione inversa, \(W_{b \rightarrow a}\) assume lo stesso valore, a causa della presenza del modulo quadro:

\[
W_{b \rightarrow a} = \mathbf{\dfrac{1}{\hslash^2}}\left| \int_{- \infty}^{+ \infty}{H_{ba}(\tau)\exp\left( j\dfrac{E_{a} - E_{b}}{\hslash}\tau \right)d\tau} \right|^{2}
\]

L'elemento di matrice dovrebbe è $H_{ba} = \langle a | \hat{H}_1 | b \rangle$, ma poiché $|H_{ba}| = |H_{ab}|$, l'uso di $H_{ab}$ è accettabile se si intende $|H_{ab}|^2$. Tuttavia, per rigore, si usa l'elemento di matrice corretto per la transizione $b \to a$)

Se \(E_{a} > E_{b}\), è possibile avere le transizioni in entrambi i versi, ovvero, un sistema può sia passare dal livello energetico inferiore \(E_{b}\) a quello superiore \(E_{a}\) che il viceversa. Al primo passaggio è associato l'assorbimento di un fotone, con frequenza data dalla legge di Plank-Einstein:

\[h\nu = \left| E_{a} - E_{b} \right|\]

Il passaggio inverso avviene mediante l'emissione, detta stimolata, di un fotone con frequenza uguale a quella del fotone assorbito.

\begin{figure}[ht]
\centering
\includegraphics[width=6.69306in,height=3.00486in,alt={P2265\#yIS1}]{media/4_Quantiatica/image46.pdf}\caption{Transizioni tra livelli energetici mediante assorbimento ed emissione di fotoni}
\end{figure}

\begin{center}
\vfill
    \chapter{Meccanica statistica}
    \label{blx:refsection\therefsection}
\vfill

\minitoc
\newpage
\end{center}
\justify


\section{Cenni di meccanica statistica}
\label{cenni-di-meccanica-statistica}

La Meccanica Statistica è il fondamentale ponte teorico che collega il comportamento delle singole particelle a livello microscopico con le proprietà misurabili e osservabili dei sistemi a livello macroscopico (termodinamico). Essa utilizza i principi della probabilità e della statistica per trattare sistemi composti da un numero enorme (\(N \approx 10^{23}\)) di costituenti (atomi, molecole o particelle quantistiche), superando l'impossibilità pratica di applicare le leggi della meccanica classica o quantistica a ogni singola entità.

Il cuore della meccanica statistica risiede nell'idea di \textbf{ensemble statistico} e nel principio che il valore di equilibrio di una grandezza macroscopica \(M\) è dato dalla media sui microstati permessi (o dalla media temporale, secondo l'ipotesi ergodica). Formalmente:

\[
\left\langle M \right\rangle = \sum_{\text{microstati } i} M_i P_i
\]

dove \(P_i\) è la probabilità che il sistema si trovi nel microstato \(i\). Le probabilità \(P_i\) sono ricavate dalla \textbf{funzione di partizione} \(Z\), che funge da generatrice di tutte le grandezze termodinamiche.

La meccanica statistica consente di dedurre tutte le grandezze termodinamiche (macroscopiche) partendo dalle proprietà microscopiche (livelli energetici e Hamiltoniana del sistema). Le grandezze fondamentali che essa riesce a esprimere in termini microscopici includono:

\begin{table}[h!]
\centering
\caption{Grandezze macroscopiche derivate dalla meccanica statistica}
\label{tab:mecc_stat}
\scriptsize
\begin{tabular}{|l|c|c|>{\centering\arraybackslash}m{4.5cm}|}
\hline
\textbf{Grandezza} & \textbf{Simbolo} & \textbf{Relazione Statistica} & \textbf{Descrizione Microscopica} \\
\hline \hline
\textbf{Energia Interna} & \(U\) (o \(E\)) & \(\displaystyle U = \langle H \rangle = -\left(\frac{\partial \ln Z}{\partial \beta}\right)_{V, N}\) & Media dell'energia totale su tutti i microstati. \\
\hline
\textbf{Pressione} & \(P\) & \(\displaystyle P = \frac{1}{\beta}\left(\frac{\partial \ln Z}{\partial V}\right)_{T, N}\) & Legata all'impulso trasferito dagli urti delle particelle contro le pareti. \\
\hline
\textbf{Temperatura} & \(T\) & \(\displaystyle T = \frac{1}{k_B \beta}\) & Misura dell'agitazione termica media delle particelle. \\
\hline
\textbf{Entropia} & \(S\) & \(S = k_B \ln W\) & Logaritmo del numero \(W\) di microstati compatibili (disordine molecolare). \\
\hline
\textbf{Capacità Termica} & \(C_V\) & \(\displaystyle C_V = \left(\frac{\partial U}{\partial T}\right)_{V}\) & Misura della variazione dell'energia interna (media) con la temperatura. \\
\hline
\end{tabular}
\end{table}

In sintesi, la meccanica statistica traduce i parametri microscopici (posizioni, quantità di moto e interazioni delle \(N\) particelle) in grandezze macroscopiche, fornendo una giustificazione teorica per le leggi fenomenologiche della Termodinamica.

\section{Possibili stati di un sistema in base allo spin}\label{possibili-stati-di-un-sistema-in-base-allo-spin}

Si considerano due sistemi separati da un setto. Il sistema a destra possiede inizialmente due particelle, mentre quello a sinistra una sola. Si suppone di rimuovere il setto in modo da far interagire tra loro le tre particelle. Si osservano fenomeni diffusivi.

\begin{figure}[ht]
\centering
\includegraphics[width=4.88764in,height=2.22619in]{media/5_MecStatistica/image47.pdf}
\caption{Sistemi separati da un setto}
\end{figure}

Una particella può trovarsi in due stati possibili: a destra o a sinistra. Le possibili configurazioni del sistema complessivo, privato del setto, sono \(2^{N}\), dove \(2\) è il numero degli stati e \(N\) il numero delle particelle. Nel caso in esame si hanno i seguenti casi:


\begin{longtable}[]{@{}ccc@{}}
\caption{Possibili configurazioni del sistema con tre particelle}\tabularnewline
\toprule
Particella \(1\) & Particella \(2\) & Particella \(3\) \\
\midrule
\endfirsthead
\toprule
Particella \(1\) & Particella \(2\) & Particella \(3\) \\
\midrule
\endhead
S & S & S \\
S & S & D \\
S & D & S \\
S & D & D \\
D & S & S \\
D & S & D \\
D & D & S \\
D & D & D \\
\bottomrule
\end{longtable}

Di tutte le possibili configurazioni, una prevede che tutte le particelle si trovino a sinistra, una che tutte siano a destra, mentre tre prevedono due particelle a sinistra e una a destra. Analogamente, vi sono tre casi in cui due particelle sono a destra e una a sinistra.

Tutte le configurazioni elencate sono possibili, tuttavia quelle intermedie hanno maggiore probabilità di verificarsi. Infatti, la probabilità che tutte siano a destra o tutte a sinistra è:

\[
\frac{1}{8} = 0.125 = 12.5\%
\]

La probabilità di avere due particelle a sinistra e una a destra, o viceversa, è:

\[
\frac{3}{8} = 0.375 = 37.5\%
\]

\begin{figure}[ht]
\centering
\includegraphics[width=4.08333in,height=2.63031in,alt={P2317\#yIS1}]{media/5_MecStatistica/image48.pdf}\caption{Probabilità delle varie configurazioni}
\end{figure}

Si considera un sistema di \(N\) particelle. Per la sua descrizione sono presenti tre gradi di libertà per ciascuna particella, uno per ogni coordinata cartesiana. Se le particelle si muovono secondo le leggi della meccanica classica, la lagrangiana del sistema totale si scrive come:

\[
L = \frac{1}{2}\sum_{i = 1}^{N}{m_{i}v_{i}^{2}} - U\left( x_{1},y_{1},z_{1},\ldots,x_{N},y_{N},z_{N} \right)
\]

dove \(U\) rappresenta l'energia di interazione tra le particelle del sistema. Se si considera un grammo di una sostanza pura, esso contiene un numero di Avogadro di particelle:

\[
N_{A} = 6.022 \cdot 10^{23}\ \text{mol}^{-1}
\]

Risulta, dunque, molto complesso, se non impossibile, scrivere le equazioni del moto e risolverle tutte. A tale scopo è necessario conoscere le condizioni iniziali delle particelle.

Per studiare tali sistemi, si ricorre a una descrizione probabilistica, come nel caso delle tre particelle. Con un grammo di sostanza, vi sono \(2^{N_{A}}\) combinazioni possibili, ognuna con la propria probabilità di verificarsi.

Si considera ora un sistema composto da \(N\) spin non interagenti, ciascuno dei quali può assumere solamente due stati possibili: \(\left| + \right\rangle\) e \(\left| - \right\rangle\).

Si suppone che ogni spin occupi una posizione fissa nello spazio. Lo stato complessivo del sistema macroscopico è descritto da una sequenza di stati \(\left| + \right\rangle\) e \(\left| - \right\rangle\), che identificano lo stato di ciascuno spin.

Il sistema può assumere \(2^{N}\) configurazioni diverse. Si vuole determinare il numero massimo di combinazioni con \(k\) spin nello stato \(\left| + \right\rangle\). Il numero delle combinazioni con \(k\) spin nello stato \(\left| + \right\rangle\) coincide con il coefficiente binomiale:

\[
N_{k} = \binom{N}{k} = \frac{N!}{k!(N - k)!}
\]

La probabilità di avere \(k\) spin nello stato \(\left| + \right\rangle\) è ottenuta dividendo questo risultato per il numero totale delle configurazioni ammissibili del sistema \(2^{N}\):

\[
P(k\text{ spin nello stato } \left| + \right\rangle) = \frac{1}{2^{N}} \binom{N}{k} = \frac{1}{2^{k}2^{N - k}}\binom{N}{k}
\]


Il numero degli spin nello stato \(\left| + \right\rangle\) equivale a una distribuzione binomiale con probabilità di \(1/2\). Per cui la media è:

\[
m = \frac{N}{2}
\]

La varianza della distribuzione binomiale è:

\[
\sigma^{2} = Npq = N \cdot \frac{1}{2} \cdot \frac{1}{2} = \frac{N}{4}
\]

Dove \(p\) è la probabilità del caso voluto e \(q = 1 - p\) la probabilità del caso sfavorevole. 

All'aumentare del numero di particelle, il numero delle configurazioni intermedie aumenta, così come la loro probabilità di comparsa. Inoltre, la distribuzione binomiale tende a una distribuzione gaussiana, in cui il picco rappresenta la configurazione più probabile. Le configurazioni estreme, come tutti gli spin nello stato \(\left| + \right\rangle\) o \(\left| - \right\rangle\), hanno probabilità praticamente nulla.

Le fluttuazioni della relative a una determinata configurazione sono date da \(\sigma/m\), con \(m\) media. Sostituendo la relazione per media e varianza si ha:

\[
\frac{\sigma}{m} = \frac{\sqrt{N}/2}{N/2} = \frac{\sqrt{N}}{2}\frac{2}{N} = \frac{1}{\sqrt{N}}
\]

Se il numero delle particelle è dell'ordine del numero di Avogadro, la fluttuazione è:

\[
\frac{\sigma}{m} \simeq \frac{1}{\sqrt{10^{23}}} \simeq 10^{-11} \ll 1
\]

La distribuzione delle configurazioni ha una varianza molto ridotta, quindi la curva gaussiana è molto concentrata intorno al valore medio.


\begin{figure}[ht]
\centering
\includegraphics[width=3.66791in,height=2.66284in,alt={P2343\#yIS1}]{media/5_MecStatistica/image49.pdf}\caption{Variazione della distribuzione delle possibili configurazioni al variare del numero di spin}
\end{figure}

Se il sistema di \(N\) spin potesse essere osservato in tutti gli istanti di tempo, si vedrebbe quasi sempre una configurazione in cui \(k = N/2\), ovvero la metà degli spin è nello stato \(\left| + \right\rangle\) e l'altra metà nello stato \(\left| - \right\rangle\). In definitiva, è molto più probabile trovare il sistema in una configurazione con probabilità \(p = 1/2\), piuttosto che in una configurazione lontana da quella maggiormente favorevole.

\section{Approssimazione di Stirling}\label{approssimazione-di-Stirling}

Si considera un numero intero \(N\), sufficientemente grande, e si valuta la quantità:

\[
\ln{N!} = \ln{\prod_{n = 1}^{N}n}
\]

Per le proprietà dei logaritmi, si ha:

\[
\ln{\prod_{n = 1}^{N}n} = \sum_{n = 1}^{N}{\ln n}
\]

Per \(N \gg 1\), si può approssimare la somma con un integrale:

\[
\sum_{n = 1}^{N}{\ln n} \simeq \int_{1}^{N}{\ln x\,dx}
\]

Integrando per parti, si ha;


\[
\int_{1}^{N}{\ln xdx} = \left[ x\ln x \right]_{1}^{N} - \int_{1}^{N}{dx} = N\ln N - N + 1
\]

Per \(N \gg 1\), il termine costante può essere trascurato:

\[
\ln{N!} \simeq N\ln N - N
\]

Maggiore è il numero delle particelle, più l'approssimazione di Stirling risulta accurata. L'uguaglianza diventa esattamente valida nel limite \(N \rightarrow \infty\).

\subsection{Probabilità con approssimazione di Stirling}\label{probabilituxe0-con-approssimazione-di-Stirling}

L'approssimazione di Stirling può essere utilizzata per stimare la probabilità che un sistema composto da \(N\) particelle assuma una determinata configurazione, quando \(N \gg 1\) e \(k\) è dell'ordine di \(N/2\). Il coefficiente binomiale è:

\[
P\left( k\ \text{spin su}\ N\ \text{nello stato}\ \left| + \right\rangle \right) = \frac{1}{2^{N}} \binom{N}{k} = \frac{1}{2^{N}} \frac{N!}{k!(N - k)!}
\]

Applicando il logaritmo e le sue proprietà:

\[
\ln P = \ln\left( \frac{1}{2^{N}}\frac{N!}{k!(N - k)!} \right) = \ln N! - \ln k! - \ln(N - k)! - N\ln 2
\]

Si applica l'approssimazione di Stirling (\(\ln M! \simeq M\ln M - M\)):

\[
\ln P \simeq \underbrace{(N\ln N - N)}_{\ln N!} - \underbrace{(k\ln k - k)}_{\ln k!} - \underbrace{((N - k)\ln(N - k) - (N - k))}_{\ln(N-k)!} - N\ln 2
\]

Riordinando i termini è possibile scrivere:

\[
\ln{P} \simeq N\ln{N} - N -k\ln{k} + k - \left(N-k\right)\ln{\left(N-k\right)} + N - k-N\ln{2}
\]

I termini lineari in $N$ e $k$ si cancellano, quindi:

\[
\ln P \simeq N\ln N - \big[k\ln k + (N-k)\ln(N-k)\big] - N\ln 2.
\]

Si applicano le proprietà dei logaritmi ai termini $N\ln N - N\ln 2$:

\[
\ln P \simeq N\ln\!\frac{N}{2} - \left[k\ln k + (N-k)\ln(N-k)\right]
\]

Si definisce lo scarto $k$ dalla media $\mu = N/2$ di una quantità $\Delta s \ll N/2$:

\[
k = \frac{N}{2} + \Delta s \quad \text{e} \quad N - k = \frac{N}{2} - \Delta s
\]

Si può scrivere \(N - k\) come:

\[
N - k = N - \frac{N}{2} - \Delta s = \frac{N}{2} - \Delta s = \frac{N}{2}\left( 1 - \frac{\Delta s}{\frac{N}{2}} \right)
\]

Si applica il logaritmo a entrambi i membri:

\[
\ln(N - k) = \ln\left( \frac{N}{2}\left( 1 - \frac{{\Delta}s}{\frac{N}{2}} \right) \right) = \ln\frac{N}{2} + \ln\left( 1 - \frac{{\Delta}s}{\frac{N}{2}} \right)
\]

Siccome  \(\Delta s \ll N/2\), è possibile sviluppare in serie di Taylor il secondo logaritmo al secondo membro:

\[
\ln\left( 1 - \frac{2\Delta s}{N} \right) \simeq - \frac{2\Delta s}{N} - \frac{1}{2}\left(\frac{2\Delta s}{N}\right)^2
\]

Per cui si ottiene:

\[
\ln(N - k) = \ln\left(\frac{N}{2} - \Delta s\right) = \ln\frac{N}{2} + \ln\left(1 - \frac{2\Delta s}{N}\right) \simeq \ln\frac{N}{2} - \frac{2\Delta s}{N} - \frac{1}{2}\left(\frac{2\Delta s}{N}\right)^2
\]

Riscrivendo l'equazione si ottiene:

\[
\ln (N - k) \simeq \ln\frac{N}{2} - \frac{2\Delta s}{N} - \frac{2\Delta s^{2}}{N^{2}}
\]

Analogamente, si considera \(\ln\left( k \right)\) e si applica la relazione \(k=N/2+\Delta s\). Per le proprietà dei logaritmi si ha:

\[
\ln k = \ln\left(\frac{N}{2} + \Delta s\right) = \ln\frac{N}{2} + \ln\left(1 + \frac{2\Delta s}{N}\right) \simeq \ln\frac{N}{2} + \frac{2\Delta s}{N} - \frac{1}{2}\left(\frac{2\Delta s}{N}\right)^2
\]

Scrivendo diversamente, si ottiene:

\[
\ln k \simeq \ln\frac{N}{2} + \frac{2\Delta s}{N} - \frac{2\Delta s^{2}}{N^{2}}
\]

Si considerano, uno alla volta, i gruppi presenti dell'equazione per \(\ln P\):


\[
\ln P \simeq - \left[ k\ln k + (N-k)\ln(N-k) \right] + N\ln\frac{N}{2}
\]

Si parte dai termini contenenti il parametro \(k\), ovvero \(k\ln{k}\):

\[
\begin{aligned}
k\ln k 
&= \left(\frac{N}{2}+\Delta s\right)
   \left[\ln\!\frac{N}{2} + \frac{2\Delta s}{N} - \frac{1}{2}\!\left(\frac{2\Delta s}{N}\right)^2\right] \\
&= \frac{N}{2}\ln\!\frac{N}{2} + \Delta s\ln\!\frac{N}{2}
   + \frac{N}{2}\cdot\frac{2\Delta s}{N} + \Delta s\cdot\frac{2\Delta s}{N}
   - \frac{N}{2}\cdot\frac{1}{2}\!\left(\frac{4\Delta s^2}{N^2}\right)
\end{aligned}
\]

Eseguendo le somme e semplificando si ottiene si ottiene:

\[
k\ln k \simeq 
\frac{N}{2}\ln\!\frac{N}{2} + \Delta s\ln\!\frac{N}{2} + \Delta s + \frac{\Delta s^2}{N}
\]

Si considera ora la restante parte:

\[
\begin{aligned}
(N - k)\ln(N - k)
&= \left(\frac{N}{2} - \Delta s\right)
\left[
\ln\!\frac{N}{2}
- \frac{2\Delta s}{N}
- \frac{1}{2}\!\left(\frac{2\Delta s}{N}\right)^{2}
\right] \\
& \simeq \frac{N}{2}\ln\!\frac{N}{2}
- \Delta s\ln\!\frac{N}{2}
- \frac{N}{2}\cdot\frac{2\Delta s}{N}
+ \Delta s\cdot\frac{2\Delta s}{N}
- \frac{N}{2}\cdot\frac{1}{2}\!\left(\frac{4\Delta s^{2}}{N^{2}}\right)
\end{aligned}
\]

Svolgendo i seguenti passaggi:
\[
\frac{N}{2}\cdot\frac{2\Delta s}{N} = \Delta s,
\qquad
\frac{N}{2}\cdot\frac{1}{2}\!\left(\frac{4\Delta s^{2}}{N^{2}}\right) = \frac{\Delta s^{2}}{N}.
\]

si ottiene:

\[
\begin{aligned}
(N - k)\ln(N - k)\simeq \frac{N}{2}\ln\!\frac{N}{2}
- \Delta s\ln\!\frac{N}{2}
- \Delta s
+ \frac{\Delta s^{2}}{N}
\end{aligned}
\]

Si sostituiscono questi risultati nell'espressione per \(\ln P\):

\[
\ln P \simeq N\ln\!\frac{N}{2} - \left[k\ln k + (N-k)\ln(N-k)\right]
\]

Si sostituiscono le relative espressioni in \(k\ln k + (N-k)\ln(N-k)\):

\[
\begin{aligned}
k\ln k + (N-k)\ln(N-k)
&\simeq \frac{N}{2}\ln\!\frac{N}{2}
- \Delta s\ln\!\frac{N}{2}
- \Delta s
+ \frac{\Delta s^{2}}{N} + \\
& + N\ln\!\frac{N}{2} + \frac{2\Delta s^2}{N} + \frac{N}{2}\ln\!\frac{N}{2} + \Delta s\ln\!\frac{N}{2} + \Delta s + \frac{\Delta s^2}{N}
\end{aligned}
\]

Sommando i termini analoghi si ha:

\[
k\ln k + (N-k)\ln(N-k)\simeq N\ln\!\frac{N}{2} + \frac{2\Delta s^2}{N}.
\]

Si sostituisce tale risultato nell'espressione per la probabilità:

\[
\ln P \simeq 
N\ln\!\frac{N}{2} - 
\left[N\ln\!\frac{N}{2} + \frac{2\Delta s^2}{N}\right]
= -\frac{2\Delta s^2}{N}
\]

Dalla definizione di \({\Delta}s\) si ha che \(k = N/2 + {\Delta}s \Leftrightarrow {\Delta}s = k - N/2\). È possibile scrivere:

\[
\ln P \simeq - \frac{\left( k - \frac{N}{2} \right)^{2}}{\frac{N}{2}}
\]

Questa relazione può essere arrangiata in modo da evidenziare la varianza della distribuzione binomiale \(\sigma = N/4\), moltiplicando e dividendo per \(2\) il secondo membro:

\[
\ln P \simeq - \frac{1}{2}\frac{\left( k - \frac{N}{2} \right)^{2}}{\frac{N}{4}}
\]

Si applica l'esponenziale, in modo da ricavare la probabilità \(P\):

\[
P\left( k\ \text{spin su}\ N\ \text{nello stato}\ \left| + \right\rangle \right) \simeq \exp\left( - \frac{1}{2} \frac{(k - \frac{N}{2})^2}{\sigma^2} \right)
\]

La distribuzione ottenuta è di tipo gaussiano, con media \(\mu = N/2\) e varianza  \(\sigma^{2} = N/4\). Affinché la relazione sia effettivamente una gaussiana, è necessario introdurre un termine di normalizzazione \(A\);

\[
P\left( k\ \text{spin su}\ N\ \text{nello stato}\ \left| + \right\rangle \right) \simeq \exp\left( - \frac{1}{2} \frac{(k - \frac{N}{2})^2}{\sigma^2} \right)
\]

Nei passaggi precedenti, la costante di normalizzazione non è comparsa a causa delle approssimazioni introdotte da Stirling e dallo sviluppo in serie di Taylor. La costante di normalizzazione è:
\[
A = \frac{1}{\sqrt{2\pi \sigma^2}} = \frac{1}{\sqrt{2\pi (N/4)}} = \frac{2}{\sqrt{2\pi N}}
\]

Quindi la distribuzione che \(k\) spin siano nello stato up è:

\[
P\left( k\ \text{spin su}\ N\ \text{nello stato}\ \left| + \right\rangle \right) \simeq \frac{2}{\sqrt{2\pi N}} \exp\left( - \frac{1}{2} \frac{(k - \frac{N}{2})^2}{\sigma^2} \right)
\]

\section{Stati ammissibili}\label{stati-ammissibili}

Si considera un sistema composto da \(N\) particelle immerse in un potenziale. Un esempio di tale configurazione è un sistema di \(N\) spin immersi in un campo magnetico \(\vec{B}\). In questa situazione, gli spin sono soggetti a un'energia potenziale:

\[
U = - \vec{\mu} \cdot \vec{B}
\]

Se il campo magnetico ha solo componente lungo l'asse \(z\), ovvero \(\vec{B} = B_{0} \hat{i}_{z}\), l'energia potenziale diventa:

\[
U = \mp \mu_{z} B_{0}
\]

dove \(\mu_{z}\) è la proiezione del momento magnetico intrinseco sull'asse \(z\).

\begin{figure}[ht]
\centering
\includegraphics[width=1.7273in,height=1.75in,alt={P2413\#yIS1}]{media/5_MecStatistica/image50.pdf}\caption{Spin immerso in un campo magnetico}
\end{figure}

Secondo la meccanica quantistica, le particelle possono assumere solo due livelli energetici, \(\pm \mu_{z} B_{0}\), corrispondenti agli autovalori dell'operatore hamiltoniano in presenza di un campo magnetico:

\[
\hat{H} = - \gamma B_{0} \hat{S}_{z}
\]

L'operatore \(\hat{S}_{z}\) presenta due autovalori: \(\pm \hslash / 2\), che corrispondono a due possibili momenti magnetici. Il momento magnetico parallelo al campo ha energia minima \(- \gamma \hslash B_0 / 2\), mentre quello antiparallelo ha energia massima \(+ \gamma \hslash B_0 / 2\). I due stati sono indicati con \(\left| + \right\rangle\) e \(\left| - \right\rangle\).

Supponendo che gli spin siano non interagenti, l'energia totale del sistema è data dalla somma delle energie di ciascuna particella:

\[
U_{\text{tot}} = \sum_{n = 1}^{N} \left[ - \mu_{z} B_{0} s(n) \right] = \mathbf{- \mu_{z} B_{0} \sum_{n = 1}^{N} s(n)}
\]

dove \(s(n) = 1\) se lo spin si trova nello stato \(\left| + \right\rangle\), invece, \(s(n) = - 1\) se lo spin si trova nello stato \(\left| - \right\rangle\).

Se il sistema è isolato, non può scambiare energia con l'ambiente, quindi la sua energia totale rimane costante nel tempo. Il sistema può trovarsi in qualsiasi configurazione con \(k\) spin nello stato \(\left| + \right\rangle\), purché l'energia totale sia invariata. La somma si scrive come:

\[
\sum s(n) = k - (N-k) = 2k - N
\]

Nel caso in esame l'energia può essere espressa come multiplo della differenza tra il numero di spin nello stato \(\left| + \right\rangle\) e quelli nello stato \(\left| - \right\rangle\):

\[
{U_{\text{tot}} = - \mu_{z} B_{0} \sum_{n = 1}^{N} s(n)} \Rightarrow U_{\text{tot} = - \left[ k - (N - k) \right] \mu_{z} B_{0} = - (2k - N) \mu_{z} B_{0}}
\]

Il numero delle possibili configurazioni che il sistema può assumere con \(k\) spin nello stato \(\left| + \right\rangle\) è dato da:

\[
\binom{N}{k} = \frac{N!}{k!(N - k)!}
\]

Nell'ipotesi di spin non interagenti, il sistema è statico e permane nello stato iniziale. Tuttavia, nella realtà gli spin interagiscono tra loro, seppur per tempi brevi, scambiandosi energia. Così, uno spin con energia maggiore può trasferire energia a uno con energia inferiore, causando un cambio di stato per entrambi.

Si suppone che gli spin siano debolmente interagenti, ovvero che scambino una quantità di energia trascurabile rispetto all'energia totale del sistema. A causa di queste interazioni, il sistema non permane nello stato iniziale, ma evolve attraverso configurazioni diverse, purché l'energia totale rimanga costante. Le configurazioni che condividono la stessa energia totale sono dette \textbf{stati ammissibili}. In pratica, il sistema transita continuamente tra stati ammissibili, ciascuno dei quali ha uguale probabilità. Tutti gli stati ammissibili corrispondenti allo stesso stato macroscopico osservato sono equiprobabili; pertanto, osservando il sistema, esso può trovarsi in una qualsiasi di queste configurazioni.

\section{Entropia}\label{entropia}

Si considerano due sistemi \(\mathbb{S}_{1}\) e \(\mathbb{S}_{2}\), composti rispettivamente da \(N_{1}\) e \(N_{2}\) particelle e con energia \(U_{1}\) e \(U_{2}\). I due sistemi sono messi in contatto così che possano scambiare energia ma non materia; ovvero, i due sistemi sono messi in contatto termico ma non diffusivo.

\begin{figure}[ht]
\centering
\includegraphics[width=2.37891in,height=1.5625in,alt={P2432\#yIS1}]{media/5_MecStatistica/image51.pdf}\caption{Sistemi posti in contatto}
\end{figure}

Aver posto in contatto i due sistemi, si ottiene un sistema \(\mathbb{S}\) con \(N = N_{1} + N_{2}\) particelle e con energia totale \(U = U_{1} + U_{2}\). Si suppone che il sistema complessivo \(S\) sia isolato dall'ambiente esterno, così da non poter scambiare energia con l'ambiente.

Si considera il sistema \(\mathbb{S}_{1}\), la cui energia dipende dallo stato delle sue particelle. Sia \(s_{1}\) la variabile che enumera le configurazioni possibili per il sistema \(\mathbb{S}_{1}\), come, ad esempio, il numero degli spin nello stato \(\left| + \right\rangle\). L'energia del sistema considerato dipende dalla configurazione del sistema:

\[U_{1} = U_{1}\left( s_{1} \right)\]

La variabile \(s_{1}\) non identifica un'unica configurazione ma un insieme di configurazioni, aventi tutte la stessa energia totale \(U_{1}\left( s_{1} \right)\).

La numerosità degli stati ammissibili \(g\) è una funzione del numero di particelle e dell'energia del sistema:

\[g_{1}\left( U_{1}\left( s_{1} \right),N_{1} \right)\]

Analogo discorso vale per il sistema \(\mathbb{S}_{2}\), la cui numerosità degli stati ammissibili è:

\[g_{2}\left( U_{s}\left( s_{2} \right),N_{2} \right)\]

Ponendo in contatto i due sistemi l'energia totale, costante nel tempo, è data da:

\[U = U_{1}\left( s_{1} \right) + U_{2}\left( s_{2} \right)\]

È possibile descrivere l'energia del secondo sistema in funzione dell'energia totale e dell'energia del primo sistema:

\[U_{2}\left( s_{2} \right) = U - U_{1}\left( s_{1} \right)\]

La numerosità degli stati nel sistema \(\mathbb{S}_{2}\) può essere scritta come:

\[g_{2}\left( U - U_{1}\left( s_{1} \right),N_{2} \right)\]

Da questo ragionamento si evince che, aumentando l'energia del sistema \(U_{1}\), aumenta il numero delle sue configurazioni ammissibili; mentre il sistema \(\mathbb{S}_{2}\) riduce la sua energia, dunque, la numerosità dei suoi stati ammissibili si riduce.

Si fissa il valore di \(s_{1}\) del sistema \(\mathbb{S}_{1}\), in questo modo anche l'energia \(U_{1}\) è fissata. Il numero di configurazioni possibili per il sistema complessivo \(\mathbb{S =}\mathbb{S}_{1} + \mathbb{S}_{2}\) è dato dal prodotto delle numerosità:

\[g(U,N) = g_{1}\left( U_{1}\left( s_{1} \right),N_{1} \right)g_{2}\left( U - U_{1}\left( s_{1} \right),N_{2} \right)\]

Dove \(N = N_{1} + N_{2}\). Per ogni stato ammissibile del primo del primo sistema, il secondo può trovarsi in uno qualunque dei suoi stati ammissibili, il cui numero dipende dall'energia del sistema \(\mathbb{S}_{1}\). In altre parole, al variare dell'enumerazione \(s_{1}\) del primo sistema, varia anche la numerosità degli stati ammissibili del sistema totale. Esiste, di conseguenza, un massimo della funzione \(g(U,N)\), ottenuto ponendo uguale a \(0\) la derivata di \(g\) rispetto all'energia:

\[\frac{\partial g}{\partial U} = \frac{\partial g}{\partial U_{1}} = \frac{\partial}{\partial U_{1}}\left[ g_{1}\left( U_{1}\left( s_{1} \right),N_{1} \right)g_{2}\left( U - U_{1}\left( s_{1} \right),N_{2} \right) \right)\]

Per le proprietà delle derivate si ha:

\[\frac{\partial g}{\partial U_{1}} = \frac{\partial g_{1}}{\partial U_{1}}g_{2} - g_{1}\frac{\partial g_{2}}{\partial U_{1}}\]

Dove il meno è legato alla dipendenza di \(g_{2}\) da \(- U_{1}\). In condizione di massimo, deve risultare:

\[\frac{\partial g}{\partial U} = 0\]

Ovvero:

\[\frac{\partial g_{1}}{\partial U_{1}}g_{2} - g_{1}\frac{\partial g_{2}}{\partial U_{1}} = 0 \Leftrightarrow \frac{\partial g_{1}}{\partial U_{1}}g_{2} = g_{1}\frac{\partial g_{2}}{\partial U_{1}}\]

Si divide per \(g_{1}g_{2}\):

\[\frac{1}{g_{1}}\frac{\partial g_{1}}{\partial U_{1}} = \frac{1}{g_{2}}\frac{\partial g_{2}}{\partial U_{1}}\]

È noto che:

\[\frac{\partial}{\partial U_{1}}\left( \ln g_{1} \right) = \frac{1}{g_{1}}\frac{\partial g_{1}}{\partial U_{1}}\]

Dunque, è possibile scrivere:

\[\frac{\partial}{\partial U_{1}}\left( \ln g_{1} \right) = \frac{\partial}{\partial U_{1}}\left( \ln g_{2} \right)\]

Si definisce entropia di un sistema \(\sigma\), quantità adimensionale, come il logaritmo della molteplicità degli stati aventi tutti la stessa energia:

\[S = \sigma = \ln g\]

Tramite il concetto di entropia è possibile riscrivere la relazione precedente come:

\[\left. \ \frac{\partial\sigma}{\partial U} \right|_{1} = \left. \ \frac{\partial\sigma}{\partial U} \right|_{2}\]

L'entropia nel sistema \(\mathbb{S}_{1}\) è uguale a quella del sistema \(\mathbb{S}_{2}\) una volta raggiunto l'equilibrio termodinamico, rappresentato dalla configurazione più probabile, coincidente con il massimo della numerosità.

La legge zero della termodinamica classica afferma che, se due corpi sono posti a contatto, dopo un certo tempo, raggiungono la stessa temperatura. Con il ragionamento effettuato si è visto che due sistemi, messi in contatto termico ma non diffusivo, raggiungono la stessa variazione di entropia, rispetto l'energia \(U_{1}\). Da questo risultato si definisce la temperatura come:

\[\frac{\partial\sigma}{\partial U} = \frac{1}{k_{B}T}\]

Dove \(k_{B}\) è la costante di Boltzmann, utile affinché l'equazione sia valida dal punto di vista dimensionale:

\[k_{B} = 1.38 \cdot 10^{- 23}JK^{- 1}\]

L'entropia è legata al disordine del sistema: più un sistema ha un alto numero di configurazioni ammissibili e più è disordinato, poiché aumenta la sua numerosità.

\section{Distribuzione di Boltzmann}\label{distribuzione-di-boltzmann}

Si suppone che un sistema \(\mathbb{S}_{1}\) sia molto più piccolo del sistema \(\mathbb{S}_{2}\). I due sistemi sono posti in contatto termico tra loro. Il sistema \(\mathbb{S}_{2}\) è detto serbatoio termico poiché, ponendolo in contatto col piccolo sistema, la sua energia \(U\) varia di una quantità trascurabile, idealmente nulla.

\begin{figure}[ht]
\centering
\includegraphics[width=4.23666in,height=2.7381in,alt={P2477\#yIS1}]{media/5_MecStatistica/image52.pdf}\caption{Sistema molto più piccolo dell'altro messi in contatto termico}
\end{figure}

Si vuole collegare le variazioni di entropia del piccolo sistema con quelle del grande sistema, raggiunto l'equilibrio termico.

L'entropia è collegata alla probabilità di trovare il sistema \(\mathbb{S}_{1}\) in uno stato specifico \(s_{1}\), a cui corrisponde un'energia \(\varepsilon_{1}^{\alpha} \ll U\), mediante la numerosità degli stati. Per valutare questa probabilità si scrive la numerosità del sistema totale:

\[g\left( \varepsilon_{1}^{\alpha},N \right) = g_{1}\left( \varepsilon_{1}^{\alpha},N_{1} \right)g_{2}\left( U - \varepsilon_{1}^{\alpha},N_{2} \right)\]

Fissata l'energia e la configurazione del piccolo sistema (come lo stato di uno spin immerso nell'ambiente) risulta che:

\[g_{1}\left( \varepsilon_{1}^{\alpha},N_{1} \right) = 1\]

Le possibili configurazioni ammissibili dipendono solamente dal sistema \(\mathbb{S}_{2}\), poiché, appunto, la configurazione di \(\mathbb{S}_{1}\) è fissata:

\[g\left( \varepsilon_{1}^{\alpha},N \right) = g_{2}\left( U - \varepsilon_{1}^{\alpha},N_{2} \right)\]

Si considera un secondo valore di energia del sistema \(\mathbb{S}_{1}\), \(\varepsilon_{1}^{\beta}\). Fissato lo stato del sistema \(\mathbb{S}_{1}\), la numerosità degli stati ammissibili dal sistema complessivo è data da:

\[g\left( \varepsilon_{1}^{\beta},N \right) = g_{2}\left( U - \varepsilon_{1}^{\beta},N_{2} \right)\]

La numerosità è legata alla probabilità mediante un fattore di normalizzazione, dato dal numero totale delle configurazioni possibili del sistema, indipendente dall'energia. Calcolando il rapporto tra le due numerosità degli stati del sistema globale, si ottiene il rapporto tra la probabilità che il sistema \(\mathbb{S}_{2}\) sia nello stato \(g_{2}\left( U - \varepsilon_{1}^{\alpha} \right)\) e la probabilità che sia nello stato \(g_{2}\left( U - \varepsilon_{1}^{\beta} \right)\). In altre parole, il fattore di normalizzazione si elide:

\[\frac{P\left( \varepsilon_{1}^{\alpha} \right)}{P\left( \varepsilon_{1}^{\beta} \right)} = \frac{g_{2}\left( U - \varepsilon_{1}^{\alpha},N_{2} \right)}{g_{2}\left( U - \varepsilon_{1}^{\beta},N_{2} \right)}\]

L'entropia è legata alla numerosità degli stati mediante logaritmo:

\[\sigma = \ln g = \ln g_{2}\]

Applicando l'esponenziale si ha:

\[g_{2} = e^{\sigma}\]

Sostituendo questo risultato nel rapporto tra le probabilità si ottiene:

\[\frac{P\left( \varepsilon_{1}^{\alpha} \right)}{P\left( \varepsilon_{1}^{\beta} \right)} = \frac{g_{2}\left( U - \varepsilon_{1}^{\alpha},N_{2} \right)}{g_{2}\left( U - \varepsilon_{1}^{\beta},N_{2} \right)} = \frac{\exp\left[ \sigma\left( U - \varepsilon_{1}^{\alpha},N_{2} \right) \right)}{\exp\left[ \sigma\left( U - \varepsilon_{1}^{\beta},N_{2} \right) \right)}\]

Siccome \(\varepsilon_{1}^{\alpha},\varepsilon_{1}^{\beta} \ll U\), è possibile sviluppare in serie di Taylor, arrestato al primo ordine, la funzione entropia:

\[\sigma\left( U - \varepsilon_{1}^{i},N_{2} \right) \simeq \sigma(U) - \varepsilon_{1}^{i}\frac{\partial\sigma}{\partial U},\ \ i = \alpha,\beta\]

Per definizione di temperatura:

\[\frac{\partial\sigma}{\partial U} = \frac{1}{k_{B}T}\]

Lo sviluppo si scrive come:

\[\sigma\left( U - \varepsilon_{1}^{i},N_{2} \right) \simeq \sigma(U) - \varepsilon_{1}^{i}\frac{1}{k_{B}T},\ \ i = \alpha,\beta\]

Si sostituisce questo risultato nel rapporto tra le due probabilità:

\[\frac{P\left( \varepsilon_{1}^{\alpha} \right)}{P\left( \varepsilon_{1}^{\beta} \right)} = \frac{\exp\left[ \sigma\left( U - \varepsilon_{1}^{\alpha},N_{2} \right) \right)}{\exp\left[ \sigma\left( U - \varepsilon_{1}^{\beta},N_{2} \right) \right)} \simeq \frac{\exp\left[ \sigma(U) \right)\exp\left[ - \frac{\varepsilon_{1}^{\alpha}}{k_{B}T} \right)}{\exp\left[ \sigma(U) \right)\exp\left[ - \frac{\varepsilon_{1}^{\beta}}{k_{B}T} \right)}\]

Dato che l'entropia è la stessa per entrambe le configurazione, si semplifica \(\exp\left[ \sigma(U) \right)\):

\[\frac{P\left( \varepsilon_{1}^{\alpha} \right)}{P\left( \varepsilon_{1}^{\beta} \right)} \simeq \frac{\exp\left( - \frac{\varepsilon_{1}^{\alpha}}{k_{B}T} \right)}{\exp\left( - \frac{\varepsilon_{1}^{\beta}}{k_{B}T} \right)}\]

Dato che il rapporto tra le due probabilità \(P\left( \varepsilon_{1}^{\alpha} \right)\) e \(P\left( \varepsilon_{1}^{\beta} \right)\), restituisce il rapporto tra due esponenziali, la probabilità generica \(P(\varepsilon)\) deve essere proporzionale all'esponenziale:

\[P(\varepsilon) \propto \exp\left( - \frac{\varepsilon}{k_{B}T} \right)\]

Il termine esponenziale al secondo membro, \(\exp\left( - \frac{\varepsilon}{k_{B}T} \right)\), è detto fattore di Boltzmann.

Per ottenere la dipendenza esatta della probabilità dall'energia del piccolo sistema, \(\varepsilon\), si introduce un fattore di normalizzazione \(Z\), tale che:

\[P(\varepsilon) = \frac{1}{Z}\exp\left( - \frac{\varepsilon}{k_{B}T} \right)\]

\(Z\) è scelto in modo tale che la somma di tutte le probabilità, al variare dell'energia del sistema \(\mathbb{S}_{1}\) sia unitaria, ovvero:

\[Z:\sum_{n}^{}{P\left( \varepsilon_{n} \right)} = 1\]

Sostituendo l'espressione per la probabilità, si ha:

\[\frac{1}{Z}\sum_{n}^{}{\exp\left( - \frac{\varepsilon_{n}}{k_{B}T} \right)} = 1\]

Da cui:

\[Z = \sum_{n}^{}{\exp\left( - \frac{\varepsilon_{n}}{k_{B}T} \right)}\]

\(Z\) è noto come fattore di ripartizione e funge, come detto, da costante di normalizzazione. La probabilità di una configurazione del piccolo sistema con energia \(\varepsilon\) è:

\[P(\varepsilon) = \frac{\exp\left( - \frac{\varepsilon}{k_{B}T} \right)}{\sum_{n}^{}{\exp\left( - \frac{\varepsilon_{n}}{k_{B}T} \right)}}\]

\subsection{Magnetizzazione macroscopica}\label{magnetizzazione-macroscopica}

Mediante la formula di Boltzmann è possibile determinare la magnetizzazione macroscopica di un volumetto di materiale, contenente un numero di Avogadro di particelle, immerso in un campo magnetico diretto lungo \(z\).

Ogni spin presente nel volumetto possiede uno dei due livelli energetici:

\[\varepsilon_{1,2} = \pm \gamma\frac{\hslash}{2}B_{0}\]

\begin{figure}[ht]
\centering
\includegraphics[width=2.71424in,height=2.52778in,alt={P2523\#yIS1}]{media/5_MecStatistica/image53.pdf}\caption{Volume elementare immerso in un campo magnetico}
\end{figure}

Usando il fattore il fattore di Boltzmann, la probabilità che le particelle si trovino nello stato \(\left| + \right\rangle\) o \(\left| - \right\rangle\) sono date da:

\[P\left( \varepsilon_{1,2} \right) = \frac{1}{Z}\exp\left( - \frac{\varepsilon_{1,2}}{k_{B}T} \right)\]

Dove:

\[Z = \sum_{n}^{}{\exp\left( - \frac{\varepsilon_{n}}{k_{B}T} \right)} = \exp\left( \frac{\gamma\hslash B_{0}}{2k_{B}T} \right) + \exp\left( - \frac{\gamma\hslash B_{0}}{2k_{B}T} \right)\]

Il fattore di Boltzmann è dato dalla somma di due elementi poiché gli spin possono assumere solamente due livelli energetici.

Sia \(N\) il numero degli spin nell'unità di volume. Il momento di magnetizzazione è dato da:

\[M = N\gamma\frac{\hslash}{2}\left( P^{+} - P^{-} \right)\]

Dove \(P^{+}\) è il totale degli spin nello stato \(\left| + \right\rangle\), mentre \(P^{-}\) nello stato \(\left| - \right\rangle\). In altre parole, il vettore di magnetizzazione \(M\) è dato dal netto degli spin nello stato parallelo rispetto a quelli nello stato antiparallelo, rispetto al campo applicato.

Per la distribuzione di Boltzmann, risulta che:

\[P^{+} = \frac{\exp\left( \frac{\gamma\hslash B_{0}}{2k_{B}T} \right)}{\exp\left( \frac{\gamma\hslash B_{0}}{2k_{B}T} \right) + \exp\left( - \frac{\gamma\hslash B_{0}}{2k_{B}T} \right)},\ \ P^{-} = \frac{\exp\left( - \frac{\gamma\hslash B_{0}}{2k_{B}T} \right)}{\exp\left( \frac{\gamma\hslash B_{0}}{2k_{B}T} \right) + \exp\left( - \frac{\gamma\hslash B_{0}}{2k_{B}T} \right)}\]

Il vettore di magnetizzazione, dunque, è dato da:

\[M = N\gamma\frac{\hslash}{2}\left[ \frac{\exp\left( \frac{\gamma\hslash B_{0}}{2k_{B}T} \right) - \exp\left( - \frac{\gamma\hslash B_{0}}{2k_{B}T} \right)}{\exp\left( \frac{\gamma\hslash B_{0}}{2k_{B}T} \right) + \exp\left( - \frac{\gamma\hslash B_{0}}{2k_{B}T} \right)} \right)\]

Il secondo membro coincide con la tangente iperbolica, per cui:

\[M = N\gamma\frac{\hslash}{2}\tanh\left( \frac{\gamma\hslash B_{0}}{2k_{B}T} \right)\]

Con opportune temperature, risulta che:

\[\frac{\gamma\hslash B_{0}}{2k_{B}T} \ll 1 \Leftrightarrow \frac{\gamma\hslash B_{0}}{2} \ll k_{B}T\]

Il momento di magnetizzazione può essere scritto come:

\[M = N\gamma\frac{\hslash}{2}\tanh\left( \frac{\gamma\hslash B_{0}}{2k_{B}T} \right) \simeq \ N\gamma\frac{\hslash}{2}\frac{\gamma\hslash B_{0}}{2k_{B}T}\]

Da cui si ricava la legge di Curie:

\[M \simeq \ N\frac{\gamma^{2}\hslash^{2}}{4k_{B}T}B_{0}\]

Da questa relazione è possibile ricavare un'espressione per la suscettività magnetica \(\chi_{m}\):

\[\vec{M} = \chi_{m}\vec{H}\]

La magnetizzazione netta dipende dal campo applicato e dall'inverso della temperatura; gli altri parametri sono costanti, dunque, non è possibile agire su di essi. In risonanza magnetica, la temperatura non può essere resa piccola a piacere per non raffreddare eccessivamente il paziente. Si utilizzano, per tale motivo, campi magnetici molto elevati, dell'ordine di \(1.5\ T\) in diagnosti, \(3\ T\) in terapia e \(7\ T\) in ricerca.

\subsection{Legge di Planck}\label{legge-di-planck}

Si considera una cavità metallica rettangolare, con un piccolo foro sul centro di un lato. Questo oggetto approssima il comportamento del corpo nero. Dal foro fuoriesce una radiazione elettromagnetica, la cui distribuzione energetica è dipendente dalla temperatura della cavità.

\begin{figure}[ht]
\centering
\includegraphics[width=1.525in,height=2.12413in,alt={P2550\#yIS1}]{media/5_MecStatistica/image54.pdf}\caption{Approssimazione del corpo nero}
\end{figure}

Mediatane la meccanica classica non è possibile spiegare il comportamento dello spettro di emissione del corpo nero. Plank ipotizzò che la radiazione elettromagnetica avesse una natura quantizzata su livello energetico, data da:

\[E = nh\nu,\ \ n\mathbb{\in N}\]

Il comportamento del corpo nero può essere spiegato e descritto mediante l'uso della meccanica quantistica e statistica.

Nella cavità esistono dei modi di oscillazione del campo magnetico. Sia \(\omega\) la pulsazione di uno di questi modi.

L'ipotesi di Planck afferma che l'energia è quantizzata, dunque, può assumere valori multipli di una quantità fondamentale \(\hslash\omega\). I livelli energetici dei fotoni di questo modo sono dati da:

\[E_{s} = s\hslash\omega,\ \ s\mathbb{\in N}\]

Si considera un singolo fotone, visto come particella immessa in contatto con un sistema molto più grande costituito dai restanti fotoni. La probabilità che un fotone si trovi al livello energetico \(E_{s} = s\hslash\omega\), è data da:

\[P\left( E_{s} \right) = \frac{1}{Z}\exp\left( - \frac{s\hslash\omega}{k_{B}T} \right)\]

Dove il fattore di ripartizione \(Z\), nell'ipotesi che vi siano infiniti fotoni nella cavità, è dato da:

\[Z = \sum_{s = 0}^{\infty}{\exp\left( - \frac{s\hslash\omega}{k_{B}T} \right)}\]

Il fattore di ripartizione \(Z\) è una serie geometrica con ragione minore dell'unità, dunque, convergente:

\[Z = \sum_{s = 0}^{\infty}{\exp\left( - \frac{s\hslash\omega}{k_{B}T} \right)} = \frac{1}{1 - \exp\left( - \frac{\hslash\omega}{k_{B}T} \right)}\]

La probabilità che un fotone si trovi nel livello energetico \(s\) è:

\[P\left( E_{s} \right) = \frac{1}{Z}\exp\left( - \frac{s\hslash\omega}{k_{B}T} \right) = \exp\left( - \frac{s\hslash\omega}{k_{B}T} \right)\left[ 1 - \exp\left( - \frac{\hslash\omega}{k_{B}T} \right) \right)\]

Si calcola il valor medio dell'energia del modo con pulsazione \(\omega\):

\[\left\langle E \right\rangle = \sum_{s = 0}^{\infty}{E_{s}P\left( E_{s} \right)}\]

Dove \(E_{s} = s\hslash\omega\). Sostituendo le espressioni per\(E_{s}\) e \(P\left( E_{s} \right)\), si ricava:

\[\left\langle E \right\rangle = \sum_{s = 0}^{\infty}{E_{s}P\left( E_{s} \right)} = \sum_{s = 0}^{\infty}{s\hslash\omega\exp\left( - \frac{s\hslash\omega}{k_{B}T} \right)\left[ 1 - \exp\left( - \frac{\hslash\omega}{k_{B}T} \right) \right)} =\]

Per la linearità della sommatoria, si ha:

\[\left\langle E \right\rangle = \ \hslash\omega\left[ 1 - \exp\left( - \frac{\hslash\omega}{k_{B}T} \right) \right)\sum_{s = 0}^{\infty}{s\exp\left( - \frac{s\hslash\omega}{k_{B}T} \right)}\]

Si considera la quantità:

\[\sum_{s = 0}^{\infty}{s\exp( - xs)},\ \ x = \frac{s\hslash\omega}{k_{B}T}\]

È possibile scrivere che:

\[s\exp( - xs) = - \frac{d}{dx}\exp( - xs)\]

Per cui è possibile scrivere:

\[\left\langle E \right\rangle = \ \hslash\omega\left[ 1 - \exp\left( - \frac{\hslash\omega}{k_{B}T} \right) \right)\sum_{s = 0}^{\infty}\left[ - \frac{d}{dx}\exp\left( - \frac{s\hslash\omega}{k_{B}T} \right) \right)\]

Per la linearità della sommatoria e della derivata si ha:

\[\left\langle E \right\rangle = - \ \hslash\omega\left[ 1 - \exp\left( - \frac{\hslash\omega}{k_{B}T} \right) \right)\frac{d}{dx}\sum_{s = 0}^{\infty}{\exp( - xs)}\]

La serie risulta essere convergente in quanto serie geometrica con ragione in modulo minore dell'unità. Il valor medio può essere scritto come:

\[\left\langle E \right\rangle = - \ \hslash\omega\left[ 1 - \exp\left( - \frac{\hslash\omega}{k_{B}T} \right) \right)\frac{d}{dx}\left[ \frac{1}{1 - \exp( - x)} \right)\]

Svolgendo la derivata, si ottiene:

\[\left\langle E \right\rangle = - \ \hslash\omega\left[ 1 - \exp\left( - \frac{\hslash\omega}{k_{B}T} \right) \right)\frac{- \exp\left( - \frac{\hslash\omega}{k_{B}T} \right)}{\left[ 1 - \exp\left( - \frac{\hslash\omega}{k_{B}T} \right) \right)^{2}}\]

Semplificando:

\[\left\langle E \right\rangle = \ \hslash\omega\frac{\exp\left( - \frac{\hslash\omega}{k_{B}T} \right)}{\left[ 1 - \exp\left( - \frac{\hslash\omega}{k_{B}T} \right) \right)}\]

È possibile raccogliere il termine esponenziale \(\exp\left( - \frac{\hslash\omega}{k_{B}T} \right)\):

\[\left\langle E \right\rangle = \ \frac{\hslash\omega}{\left[ \exp\left( \frac{\hslash\omega}{k_{B}T} \right) - 1 \right)}\]

Tale equazione è la legge di Planck e rappresenta una prima logge in cui si applica la quantizzazione della materia, unita alle teoria di Boltzmann.

\subsection{Rumore termico o di Johnson-Nyquist}\label{rumore-termico-o-di-johnson-nyquist}

Il rumore termico o di Johnson-Nyquist è sempre presente nei componenti elettronici, come resistori o dispositivi a semiconduttore.

Si considera una linea di trasmissione chiusa alle estremità da due resistori uguali alle impedenza caratteristica della linea. In altre parole, le resistenze sono adattate alla linea. Sulla linea di trasmissione, cioè, viaggiano solamente onde progressive o modi, a causa dell'adattamento (\(\Gamma = 0\)), generate dalle resistenze, la cui energia dipende dalla temperatura a cui si trovano, in accordo con la legge di Planck. Poiché il sistema è adattato, la potenza trasferita sul carico è data da:

\[P = \frac{\left\langle V^{2} \right\rangle}{4R} = \left\langle I^{2} \right\rangle R\]

\begin{figure}[ht]
\centering
\includegraphics[width=4.11381in,height=1.43842in,alt={P2593\#yIS1}]{media/5_MecStatistica/image55.pdf}\caption{Linea di trasmissione adattata}
\end{figure}

Per ogni pulsazione del modo, \(\omega\), esistono due onde viaggianti in direzione opposte. Sia \(L\) la lunghezza della linea e \(c\) la velocità di propagazione del segnale sulla linea. Il tempo impiegato dall'onda per percorrere l'intera linea di trasmissione è:

\[{\Delta}t = \frac{L}{c}\]

I modi presenti sulla linea presentano delle frequenze multiple intere di una quantità \(\delta f\) data da:

\[\delta f = \frac{1}{{\Delta}t} = \frac{c}{L}\]

Le frequenze dei modi sono multiple di \(\delta f\)\emph{,} per cui in un certo intervallo di frequenze \({\Delta}f\) è presente un numero di modi dato da:

\[n_{modi} = \frac{{\Delta}f}{\delta f} = \frac{L}{c}{\Delta}f\]

Noto il numero di modi presenti nell'intervallo frequenziale \({\Delta}f\), è possibile valutare l'energia complessivamente presente sulla linea di trasmissione:

\[E = 2\frac{{\Delta}f}{\delta f}\left\langle E \right\rangle\]

Dove il \(2\) è dovuto alle due onde trasmesse dai due resistori.

Ogni singolo modo può essere considerato come un fotone immerso in un sistema, composto dagli altri fotoni, per cui è possibile applicare la legge di Planck per il calcolo dell'energia media:

\[\left\langle E \right\rangle = \ \frac{\hslash\omega}{\left[ \exp\left( \frac{\hslash\omega}{k_{B}T} \right) - 1 \right)}\]

Da cui:

\[E = 2\frac{{\Delta}f}{\delta f}\frac{\hslash\omega}{\left[ \exp\left( \frac{\hslash\omega}{k_{B}T} \right) - 1 \right)}\]

Per le normali frequenze utilizzate nella pratica elettrotecnica, risulta che:

\[\hslash\omega \ll k_{B}T \Leftrightarrow \frac{\hslash\omega}{k_{B}T} \ll 1\]

In questa ipotesi, è possibile approssimare l'esponenziale in serie di Taylor, arrestato al primo ordine:

\[\exp\left( \frac{\hslash\omega}{k_{B}T} \right) \simeq 1 + \frac{\hslash\omega}{k_{B}T}\]

Con questa approssimazione, l'energia media è:

\[\left\langle E \right\rangle = \ \frac{\hslash\omega}{\left[ \exp\left( \frac{\hslash\omega}{k_{B}T} \right) - 1 \right)} \simeq \ \frac{\hslash\omega}{\left[ 1 + \frac{\hslash\omega}{k_{B}T} - 1 \right)} = k_{B}T\]

L'energia complessiva, di conseguenza, è:

\[E = 2\frac{{\Delta}f}{\delta f}k_{B}T\]

A questa energia corrisponde una potenza data, per definizione, da:

\[P = \frac{E}{{\Delta}t} = 2\frac{1}{{\Delta}t}\frac{{\Delta}f}{\delta f}k_{B}T\]

Ma \({\Delta}t\) è l'inverso di \(\delta f\), dunque, il prodotto dei due termini è unitario. La potenza è, quindi:

\[P = \frac{E}{{\Delta}t} = 2{\Delta}fk_{B}T\]

Su un singolo resistore, si ritrova metà potenza, ovvero:

\[P_{R} = \frac{P}{2} = {\Delta}fk_{B}T\]

Per l'adattamento in potenza, la potenza trasferita al carico è:

\[P = \frac{\left\langle V^{2} \right\rangle}{4R}\]

Sostituendo l'espressione appena determinata per la potenza, è possibile ricavare \(\left\langle V^{2} \right\rangle\):

\[{\Delta}fk_{B}T = \frac{\left\langle V^{2} \right\rangle}{4R} \Leftrightarrow \left\langle V^{2} \right\rangle = 4R{\Delta}fk_{B}T\]

\(\left\langle V^{2} \right\rangle\) corrisponde al valor quadratico medio della tensione su un resistore, dovuto all'agitazione termica dei suoi portatori di carica, nella banda \({\Delta}f\). Il rumore termico presenta la stessa ampiezza in tutto il range frequenziale \({\Delta}f\) per cui può essere modellato come un rumore bianco.

Non approssimando l'esponenziale, il valor quadratico medio della tensione su un resistore è dato da:

\[\left\langle V^{2} \right\rangle = 4R{\Delta}f\ \frac{\hslash\omega}{\left[ \exp\left( \frac{\hslash\omega}{k_{B}T} \right) - 1 \right)}\]

La frequenza di taglio è:

\[f_{0} = \frac{k_{B}T}{2\pi\hslash}\]

Questa frequenza, a temperatura ambiente, è data da:

\[f_{0} = \frac{k_{B}T}{2\pi\hslash} = \frac{1.38 \cdot 10^{- 23}\frac{J}{K} \cdot 290\ K}{6.63 \cdot 10^{- 34}\ J \cdot s} = 6.05\ THz\]

Questa frequenza, come detto precedentemente, è molto maggiore di quelle ottenibili con l'attuale strumentazione elettronica. Per questo motivo si ricorre alla relazione approssimata.

\section{Distribuzione di Gibbs}\label{distribuzione-di-gibbs}

Si considerano due sistemi contenenti, rispettivamente, \(N_{1}\) e \(N_{2}\) particelle. I due sistemi sono posti in contatto si termico che diffusivo, ovvero possono scambiare sia materia che energia.

Il sistema totale ha energia data dalla somme delle singole energie iniziali dei due sistemi. \(U = U_{1} + U_{2}\). Si suppone, infine, che il sistema complessivo sia isolato dall'ambiente, così da conservare la propria energia.

\begin{figure}[ht]
\centering
\includegraphics[width=4.71575in,height=2.30556in,alt={P2637\#yIS1}]{media/5_MecStatistica/image56.pdf}\caption{Sistemi posti in contatto termico e diffusivo}
\end{figure}

Prima del contatto, i due sistemi possedevano un'entropia, rispettivamente, uguali a \(\sigma_{1}\) e \(\sigma_{2}\). Dopo il contatto, per ragioni probabilistiche, l'entropia dovrà essere massima.

Si suppone che il sistema \(\mathbb{S}_{1}\) sia molto più piccolo del sistema \(\mathbb{S}_{2}\), approssimabile come un serbatoio termico. Si vuole valutare la probabilità che il sistema \(\mathbb{S}_{1}\) sia in uno stato caratterizzato da \(N_{a}\) particelle ed energia \(U_{a}\).

Fissata la configurazione del sistema \(\mathbb{S}_{1}\), la probabilità che si verifichi questa condizione dipende dalle configurazioni ammissibili del sistema \(\mathbb{S}_{2}\), ovvero:

\[P\left( N_{a},U_{a} \right) = g_{2}\left( N_{2},U_{2} \right)\]

Siccome il numero delle particelle è costante, così come l'energia, è possibile scrivere:

\[N = N_{a} + N_{2} \Leftrightarrow N_{2} = N - N_{a},\ \ U = U_{a} + U_{2} \Leftrightarrow U_{2} = U - U_{a}\]

La numerosità degli stati ammissibili dal sistema \(\mathbb{S}_{2}\) può essere scritta come:

\[P\left( N_{a},U_{a} \right) \propto g_{2}\left( N - N_{a},U - U_{a} \right)\]

La probabilità che il sistema \(\mathbb{S}_{1}\) assuma un'altra configurazione, caratterizzata da un numero \(N_{b}\) di particelle da un'energia uguale a \(U_{b}\), è data da:

\[P\left( N_{b},U_{b} \right) \propto g_{2}\left( N - N_{b},U - U_{b} \right)\]

Il rapporto tra le due probabilità si scrive come:

\[\frac{P\left( N_{a},U_{a} \right)}{P\left( N_{b},U_{b} \right)} = \frac{g_{2}\left( N - N_{a},U - U_{a} \right)}{g_{2}\left( N - N_{b},U - U_{b} \right)}\]

Si considera il logaritmo di tale rapporto:

\[\log\left[ \frac{P\left( N_{a},U_{a} \right)}{P\left( N_{b},U_{b} \right)} \right) = \log\left[ \frac{g_{2}\left( N - N_{a},U - U_{a} \right)}{g_{2}\left( N - N_{b},U - U_{b} \right)} \right) = \log{g_{2}\left( N - N_{a},U - U_{a} \right)} - \log{g_{2}\left( N - N_{b},U - U_{b} \right)}\]

Per definizione di entropia:

\[\sigma(N,U) = \log{g(N,U)}\]

Il logaritmo del rapporto può essere espresso come:

\[\log\left[ \frac{P\left( N_{a},U_{a} \right)}{P\left( N_{b},U_{b} \right)} \right) = \sigma\left( N - N_{a},U - U_{a} \right) - \sigma\left( N - N_{b},U - U_{b} \right)\]

Siccome il sistema \(\mathbb{S}_{2}\) è molto più grande del sistema \(\mathbb{S}_{1}\), è possibile concludere che:

\[U \gg U_{a},U_{b},\ \ N \gg N_{a},N_{b}\]

È possibile sviluppare in serie di Taylor l'entropia, arrestando lo sviluppo al primo ordine:

\[\sigma\left( N - N_{i},U - U_{i} \right) = \ \sigma(N,U) - \left. \ \frac{\partial\sigma}{\partial N} \right|_{N}N_{i} - \left. \ \frac{\partial\sigma}{\partial U} \right|_{U}U_{i},\ \ i = a,b\]

Con questa approssimazione si ottiene:

\[\log\left[ \frac{P\left( N_{a},U_{a} \right)}{P\left( N_{b},U_{b} \right)} \right) = \sigma(N,U) - \left. \ \frac{\partial\sigma}{\partial N} \right|_{N}N_{a} - \left. \ \frac{\partial\sigma}{\partial U} \right|_{U}U_{a} - \sigma(N,U) + \left. \ \frac{\partial\sigma}{\partial N} \right|_{N}N_{b} - \left. \ \frac{\partial\sigma}{\partial U} \right|_{U}U_{b}\]

Elidendo \(\sigma(N,U)\) e raccogliendo:

\[\log\left[ \frac{P\left( N_{a},U_{a} \right)}{P\left( N_{b},U_{b} \right)} \right) = \left( N_{b} - N_{a} \right)\left. \ \frac{\partial\sigma}{\partial N} \right|_{N} + \left( U_{b} - U_{a} \right)\left. \ \frac{\partial\sigma}{\partial U} \right|_{U}\]

Si definisce la temperatura come:

\[\left. \ \frac{\partial\sigma}{\partial U} \right|_{U} = \frac{1}{k_{B}T}\]

Si definisce potenziale chimico \(\mu\) come il fattore di proporzionalità tra la derivata dell'entropia rispetto al numero di particelle e la temperatura:

\[\left. \ \frac{\partial\sigma}{\partial N} \right|_{N} = - \frac{\mu}{k_{B}T}\]

Con queste definizioni, si ha:

\[\log\left[ \frac{P\left( N_{a},U_{a} \right)}{P\left( N_{b},U_{b} \right)} \right) = - \left( N_{b} - N_{a} \right)\frac{\mu}{k_{B}T} + \left( U_{b} - U_{a} \right)\frac{1}{k_{B}T} = \left( N_{a} - N_{b} \right)\frac{\mu}{k_{B}T} - \left( U_{a} - U_{b} \right)\frac{1}{k_{B}T}\]

Si applica l'esponenziale, si ha:

\[\frac{P\left( N_{a},U_{a} \right)}{P\left( N_{b},U_{b} \right)} = \exp\left[ \left( N_{a} - N_{b} \right)\frac{\mu}{k_{B}T} \right)\exp\left[ - \left( U_{a} - U_{b} \right)\frac{1}{k_{B}T} \right) = \frac{\exp\left[ \left( N_{a} - N_{b} \right)\frac{\mu}{k_{B}T} \right)}{\exp\left[ - \left( U_{a} - U_{b} \right)\frac{1}{k_{B}T} \right)}\]

Quindi, la probabilità che il sistema \(\mathbb{S}_{1}\) si trovi in uno stato con energia \(U_{a}\) e con un numero di particelle \(N_{a}\) è proporzionale a:

\[P\left( N_{a},U_{a} \right) \propto \exp\left( \frac{\mu N_{a} - U_{a}}{k_{B}T} \right)\]

L'esponenziale nella relazione individuata è detto fattore di Gibbs. La probabilità esatta si esprime introducendo un fattore di normalizzazione \(Z\), tale che:

\[\frac{1}{Z}\sum_{N}^{}{\sum_{U}^{}{P(N,U)}} = 1\]

Sostituendo la relazione per la probabilità, si ricava:

\[\frac{1}{Z}\sum_{N}^{}{\sum_{U}^{}{\exp\left( \frac{\mu N - U}{k_{B}T} \right)}} = 1 \Leftrightarrow Z = \sum_{N}^{}{\sum_{U}^{}{\exp\left( \frac{\mu N - U}{k_{B}T} \right)}}\]

Il fattore di partizione \(Z\) così ottenuto è detto somma di Gibbs.

La probabilità che il piccolo sistema \(\mathbb{S}_{1}\) si trovi in uno stato \(\left( U_{a},N_{a} \right)\) è, dunque, data da:

\[P\left( N_{a},U_{a} \right) = \frac{\exp\left( \frac{\mu N_{a} - U_{a}}{k_{B}T} \right)}{\sum_{N}^{}{\sum_{U}^{}{\exp\left( \frac{\mu N - U}{k_{B}T} \right)}}}\]

La quantità \(\mu\) rappresenta la capacità di scambio diffusivo tra sistemi messi in contatto diffusivo e termico.

\section{Distribuzione di Fermi-Dirac}\label{distribuzione-di-fermi-dirac}

I fermioni sono particelle elementari con spin frazionario, ovvero uguale a \(\pm 1/2\). Due fermioni non possono occupare lo stesso livello energetico per il principio di esclusione di Pauli.

Si considera un piccolo sistema costituito da fermioni, come protoni o elettroni. Siano \(N = 0\) e \(N = 1\) gli stati ammissibili per un fermione, corrispondenti ai livelli energetici \(U = 0\) e \(U = \varepsilon\).

Un fermione o non è presente nel livello energetico, nel caso \(U = 0\) e \(N = 0\), o c'è, nel caso \(U = \varepsilon\) e \(N = 1\). la somma di Gibbs è data da:

\[Z = \sum_{N}^{}{\sum_{U}^{}{\exp\left( \frac{\mu N - U}{k_{B}T} \right)}} = \exp\left( \frac{\mu 0 - 0}{k_{B}T} \right) + \exp\left( \frac{\mu \cdot 1 - \varepsilon}{k_{B}T} \right) = 1 + \exp\left( \frac{\mu - \varepsilon}{k_{B}T} \right)\]

Si vuole determinare il numero medio di particelle del piccolo sistema \(\mathbb{S}_{1}\) con energia \(U = \varepsilon\). Per definizione di media, si ha:

\[\left\langle N(\varepsilon) \right\rangle = \frac{1}{Z}\sum_{N}^{}{\sum_{U}^{}{P(N,U)}} = \frac{\exp\left( \frac{\mu - \varepsilon}{k_{B}T} \right)}{1 + \exp\left( \frac{\mu - \varepsilon}{k_{B}T} \right)}\]

La somma si riduce a un solo elemento, in quanto solo un fermione può trovarsi in quel livello energetico. Raccogliendo il fattore di Gibbs, si ottiene:

\[\left\langle N(\varepsilon) \right\rangle = \frac{1}{1 + \exp\left( \frac{\mu - \varepsilon}{k_{B}T} \right)}\]

La distribuzione di Fermi-Dirac permette di conoscere il numero medio dei fermioni in un determinato livello energetico. La statistica è utilizzata nel campo dei semiconduttori o cristalli scintillatori al fine di valutare il numero di elettroni in un determinato livello energetico.

\subsection{Distribuzione di Bose-Einstein}\label{distribuzione-di-bose-einstein}

I bosoni sono particelle con spin nullo o intero. Un livello energetico può essere occupato da un numero qualsiasi di bosoni, dunque, queste particelle non rispettano il principio di esclusione di Pauli. Il fotone è uno dei bosoni più importante; esso presenta uno spin unitario.

Si considera un piccolo sistema di bosoni. Questo sistema può contenere un numero qualsiasi \(N\) di particelle, ciascuna con energia \(\varepsilon\). La somma di Gibbs, nel caso di energia fissata a \(U = \varepsilon\), si esprime come:

\[Z = \sum_{N}^{}{\sum_{U}^{}{\exp\left( \frac{\mu N - U}{k_{B}T} \right)}} = \sum_{N}^{}{\exp\left( \frac{N\mu - N\varepsilon}{k_{B}T} \right)} = \sum_{N}^{}\left[ \exp\left( \frac{\mu - \varepsilon}{k_{B}T} \right) \right)^{N}\]

La somma è convergente, in quanto serie geometrica con ragione, in modulo, minore dell'unità:

\[Z = \frac{1}{1 - \exp\left( \frac{\mu - \varepsilon}{k_{B}T} \right)}\]

Il numero medio delle particelle del sistema che si trovano nel livello energetico \(\varepsilon\) è:

\[\left\langle N(\varepsilon) \right\rangle = \frac{1}{Z}\sum_{N}^{}{\sum_{U}^{}{NP(N,U)}} = \left[ 1 - \exp\left( \frac{\mu - \varepsilon}{k_{B}T} \right) \right)\sum_{N}^{}{\sum_{U}^{}{N\exp\left( \frac{\mu N - U}{k_{B}T} \right)}}\]

Avendo fissata l'energia, risulta:

\[\left\langle N(\varepsilon) \right\rangle = \left[ 1 - \exp\left( \frac{\mu - \varepsilon}{k_{B}T} \right) \right)\sum_{N}^{}{N\exp\left( \frac{N\mu - N\varepsilon}{k_{B}T} \right)} = \left[ 1 - \exp\left( \frac{\mu - \varepsilon}{k_{B}T} \right) \right)\sum_{N}^{}{N\left[ \exp\left( \frac{\mu - \varepsilon}{k_{B}T} \right) \right)^{N}}\]

Si pone:

\[X = \exp\left( \frac{\mu - \varepsilon}{k_{B}T} \right)\]

Il numero medio di bosoni nel livello energetico \(\varepsilon\) può essere scritto come.

\[\left\langle N(\varepsilon) \right\rangle = (1 - X)\sum_{N}^{}{NX^{N}}\]

Si considera la quantità:

\[Z = \sum_{N}^{}X^{N}\]

Si deriva rispetto a \(X\):

\[\frac{dZ}{dX} = \frac{d}{dX}\sum_{N}^{}X^{N} = \sum_{N}^{}{\frac{d}{dX}\left( X^{N} \right)} = \sum_{N}^{}{NX^{N - 1}} = \sum_{N}^{}{NX^{N}X^{- 1}} =\]

Per la linearità dell'operatore sommatoria, si scrive:

\[\frac{dZ}{dX} = \frac{1}{X}\sum_{N}^{}{NX^{N}}\]

Da cui si ha:

\[\sum_{N}^{}{NX^{N}} = X\frac{dZ}{dX} = X\frac{d}{dX}\left[ \frac{1}{1 - X} \right) = \frac{X}{(1 - X)^{2}}\]

Il numero medio dei bosoni può essere scritto come:

\[\left\langle N(\varepsilon) \right\rangle = (1 - X)\sum_{N}^{}{NX^{N}} = (1 - X)\frac{X}{(1 - X)^{2}} = \frac{X}{1 - X} = \frac{1}{\frac{1}{X} - 1}\]

Sostituendo il valore di \(X\) si ottiene il numero medio dei bosoni nel livello energetico \(\varepsilon\):

\[\left\langle N(\varepsilon) \right\rangle = \frac{1}{\exp\left( - \frac{\mu - \varepsilon}{k_{B}T} \right) - 1}\]

Le distribuzioni di Bose-Einstein, Fermi-Dirac e Gibbs a temperatura ambiente convergono alla distribuzione di Boltzmann. Le statistiche sono sempre valide e descrivono il comportamento dei fermioni e bosoni.



\medskip

\newpage
\printbibheading[
heading=bibintoc,
title={Bibliografia}
]
\bibbysection[heading=bibbysubsect]

\end{document}
