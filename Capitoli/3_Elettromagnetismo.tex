\begin{center}
\vfill
    \chapter{Elettromagnetismo}
    \label{blx:refsection\therefsection}
\vfill

\minitoc
\newpage
\end{center}
\justify


\section{Cenni di elettromagnetismo}\label{cenni-di-elettromagnetismo}

Le \textbf{equazioni di Maxwell}, formalizzate verso la metà dell'Ottocento dal fisico James Clerk Maxwell, non solo sintetizzarono le leggi dell'elettricità e del magnetismo note all'epoca (come le leggi di Faraday e Ampère), ma predissero l'esistenza delle onde elettromagnetiche e dimostrarono che la luce stessa è un'onda elettromagnetica \cite{landau1975campi, purcell1985elettromagnetismo, feynman1964vol2, halliday2004fisica2}. Queste equazioni sono ancora valide in molte applicazioni pratiche moderne, con poche modifiche rispetto alla formulazione originale.

Il concetto di \textbf{spazio-tempo} (nella teoria \textbf{Relatività Ristretta}) fu introdotto da Albert Einstein nel 1905 proprio per risolvere il conflitto tra la meccanica newtoniana (che era invariante solo rispetto alle trasformazioni Galileiane) e le equazioni di Maxwell (che sono invece invarianti rispetto alle trasformazioni di Lorentz).

Il campo elettromagnetico è descritto in modo formale dai due campi vettoriali: il \textbf{campo elettrico} ($\vec{E}$) e il \textbf{campo di induzione magnetica} ($\vec{B}$). Questi due campi non sono entità separate e indipendenti, ma sono le manifestazioni di un unico tensore elettromagnetico.

Il campo magnetico ($\vec{B}$) è essenzialmente una manifestazione relativistica del campo elettrico ($\vec{E}$). Ossia, il campo magnetico osservato in un sistema di riferimento $K$ è in parte il risultato della trasformazione del campo elettrico e delle cariche in moto osservate in un altro sistema $K'$

Quando una carica $q$ si muove in una regione di spazio, essa è soggetta alla Forza di Lorentz ($\vec{F} = q(\vec{E} + \vec{v} \times \vec{B})$) \cite{landau1975campi, purcell1985elettromagnetismo, feynman1964vol2, halliday2004fisica2}. Se si considera un sistema di riferimento $K'$ solidale con la carica ($\vec{v}' = 0$), su di essa agisce solo la forza elettrica ($\vec{F}' = q\vec{E}'$). Pertanto, in tale sistema, qualsiasi forza percepita dalla carica deve essere attribuita al solo campo elettrico trasformato $\vec{E}'$, mentre il termine magnetico è nullo.

\section{Equazioni di Maxwell nel vuoto in forma locale}\label{equazioni-di-maxwell-nel-vuoto-in-forma-locale}

Sebbene sia possibile descrivere l'interazione elettromagnetica mediante l'uso del solo campo elettrico, per semplicità si ammette l'esistenza anche del campo magnetico. In quest'ottica, il campo elettrico e il campo di induzione magnetica sono due manifestazioni dello stesso campo: il campo elettromagnetico, descritto, in assenza di mezzo materiale (nel vuoto), dalle equazioni differenziali in forma locale:

\[\begin{cases}
 \vec{\nabla} \cdot \vec{E} = \dfrac{\rho}{\varepsilon_{0}} & \text{(Legge di Gauss per l'Elettricità)} \\
\vec{\nabla} \cdot \vec{B} = 0 & \text{(Legge di Gauss per il Magnetismo)} \\
 \vec{\nabla} \times \vec{E} = - \dfrac{\partial\vec{B}}{\partial t} & \text{(Legge di Faraday-Neumann-Lenz)} \\
 \vec{\nabla} \times \vec{B} = \mu_{0}\left( \vec{J} + \varepsilon_{0}\dfrac{\partial\vec{E}}{\partial t} \right) & \text{(Teorema di Ampère-Maxwell)}
\end{cases}
\]

La prima equazione è la legge di Gauss per il campo elettrico e afferma che la divergenza del campo elettrico è proporzionale alla densità di carica $\rho$ presente nel volume di riferimento, rapportata alla costante dielettrica nel vuoto \(\varepsilon_{0}\).

La seconda equazione è la legge di Gauss per il magnetismo, secondo cui la divergenza del campo di induzione magnetica $\vec{B}$ è nulla. Questa legge implica che le linee di campo magnetico non hanno sorgenti o pozzi (non esistono monopoli magnetici) e che il campo $\vec{B}$ è solenoidale. Dal punto di vista fisico, questa legge implica che le linee di campo magnetico entrano ed escono dal volumetto \(d\Omega\) su cui si calcola la divergenza, che, di conseguenza, è nulla \figurename~\ref{fig:CampSolenoidale}.

La terza equazione è la legge di Faraday-Neumann-Lenz e stabilisce che un campo di induzione magnetica variabile nel tempo ($\partial\vec{B}/\partial t$) induce un campo elettrico rotazionale ($\vec{\nabla} \times \vec{E}$). In altre parole, la variazione del flusso del campo magnetico attraverso un circuito elettrico genera una forza elettromotrice indotta e, di conseguenza, una corrente elettrica indotta

\begin{figure}[ht]
\centering
\includegraphics[width=1.90691in,height=1.76389in,alt={P1154\#yIS1}]{media/3_Elettromagnetismo/image18.pdf}\caption{Flusso campo magnetico attraverso superficie chiusa}\label{fig:CampSolenoidale}
\end{figure}

La quarta equazione è il teorema di Ampère-Maxwell, secondo cui il rotore del campo $\vec{B}$ è determinato dalla densità di corrente di conduzione ($\vec{J}$) più la corrente di spostamento ($\varepsilon_{0}\partial\vec{E}/\partial t$). La corrente di spostamento fu introdotta da Maxwell per garantire la conservazione della carica. Secondo questo teorema, dal punto di vista globale, un campo magnetico indotto intorno a un circuito chiuso qualsiasi è proporzionale alla corrente elettrica concatenata \(\mu_{0}\vec{J}\) al circuito più la corrente di spostamento (\(\mu_{0}\varepsilon_{0}\partial\vec{E}/\partial t\)) attraverso la superficie chiusa.

\subsection{Equazione di continuità della carica}
Dalle equazioni di Maxwell è possibile ricavare la legge di continuità della carica. A tale scopo si applica la divergenza all'equazione di Ampere-Maxwell:

\[
\vec{\nabla} \cdot \vec{\nabla} \times \vec{B} = \mu_{0}\vec{\nabla} \cdot \vec{J} + \mu_{0}\varepsilon_{0}\vec{\nabla} \cdot \dfrac{\partial\vec{E}}{\partial t}
\]

Il primo membro è la divergenza di un rotore che, per qualsiasi campo vettoriale derivabile due volte, è sempre nulla, ovvero \(\vec{\nabla} \cdot \vec{\nabla} \times \vec{B} = 0\). Si ottiene:

\[\mu_{0}\vec{\nabla} \cdot \vec{J} + \mu_{0}\varepsilon_{0}\vec{\nabla} \cdot \dfrac{\partial\vec{E}}{\partial t} = 0\]

È possibile scambiare il simbolo di derivata temporale con quello di divergenza:

\[\mu_{0}\vec{\nabla} \cdot \vec{J} + \mu_{0}\varepsilon_{0}\dfrac{\partial}{\partial t}\left( \vec{\nabla} \cdot \vec{E} \right) = 0\]

La divergenza del campo elettrico è data dalla prima equazione di Maxwell. Semplificando anche \(\mu_{0}\), si ottiene:

\[\vec{\nabla} \cdot \vec{J} + \varepsilon_{0}\dfrac{\partial}{\partial t}\left( \dfrac{\rho}{\varepsilon_{0}} \right) = 0\]

Dunque, si è ottenuta la legge di conservazione della carica:

\[\vec{\nabla} \cdot \vec{J} + \dfrac{\partial\rho}{\partial t} = 0\]

Tale equazione afferma che, se la carica in un volume varia nel tempo, deve esistere una corrente in ingresso o in uscita dal volume che sostenga tale variazione di carica.

\subsection{Risoluzione delle equazioni di Maxwell omogenee}\label{risoluzione-delle-equazioni-maxwell-omogenee}

Si considera il caso di assenza di sorgenti, ovvero, si vuole risolvere le equazioni di Maxwell lontano dalle sorgenti di campo \(\vec{J}\) e \(\rho\)

\[\begin{cases}
 \vec{\nabla} \cdot \vec{E} = \dfrac{\rho}{\varepsilon_{0}} \\
\vec{\nabla} \cdot \vec{B} = 0 \\
\vec{\nabla} \times \vec{E} = - \dfrac{\partial\vec{B}}{\partial t}  \\
\vec{\nabla} \times \vec{B} = \mu_{0}\left( \vec{J} + \varepsilon_{0}\dfrac{\partial\vec{E}}{\partial t} \right)
\end{cases}
\]

Si vuole risolvere il sistema di equazioni in modo da determinare il campo induzione magnetica \(\vec{B}\). A tale scopo si applica il rotore all'equazione di Ampere-Maxwell:

\[\vec{\nabla} \times \vec{\nabla} \times \vec{B} = \mu_{0}\varepsilon_{0}\vec{\nabla} \times \dfrac{\partial\vec{E}}{\partial t}\]

Il rotore del rotore può essere scritto come:

\[\vec{\nabla} \times \vec{\nabla} \times = \vec{\nabla}\left( \vec{\nabla} \cdot \  \right) - \nabla^{2}\ \]

Dove \(\nabla^{2}\) è l'operatore laplaciano che, scritto in coordinate cartesiane, è dato da:

\[\nabla^{2} = \Delta = \dfrac{\partial^{2}}{\partial x^{2}} + \dfrac{\partial^{2}}{\partial y^{2}} + \dfrac{\partial^{2}}{\partial z^{2}}\]

L'equazione di Ampere-Maxwell si può scrivere, invertendo l'operatore derivata temporale col rotore, come:

\[\vec{\nabla} \times \vec{\nabla} \times \vec{B} = \mu_{0}\varepsilon_{0}\vec{\nabla} \times \dfrac{\partial\vec{E}}{\partial t} \Leftrightarrow \vec{\nabla}\left( \vec{\nabla} \cdot \ \vec{B} \right) - \nabla^{2}\vec{B} = \mu_{0}\varepsilon_{0}\dfrac{\partial}{\partial t}\left( \vec{\nabla} \times \vec{E} \right)\]

La divergenza del campo induzione magnetica è nulla, inoltre, sostituendo la terza equazione di Maxwell si ha:

\[- \nabla^{2}\vec{B} = \mu_{0}\varepsilon_{0}\dfrac{\partial}{\partial t}\left( - \dfrac{\partial\vec{B}}{\partial t} \right) \Leftrightarrow - \nabla^{2}\vec{B} = - \ \mu_{0}\varepsilon_{0}\dfrac{\partial^{2}\vec{B}}{\partial t^{2}}\]

L'equazione:

\[\nabla^{2}\vec{B} - \ \mu_{0}\varepsilon_{0}\dfrac{\partial^{2}\vec{B}}{\partial t^{2}} = 0\]

È detta equazione delle onde o di d'Alembert. Si definisce operatore di d'Alembert come:

\[\square = \nabla^{2} - \dfrac{1}{c^{2}}\dfrac{\partial^{2}}{\partial t^{2}}\]

Dove \(c\) è la velocità di propagazione dell'onda. In questo caso, \(c\) coincide con la velocità della luce nel vuoto. Dalle equazioni di Maxwell discende che:

\[c = \dfrac{1}{\sqrt{\mu_{0}\varepsilon_{0}}} \simeq 3.00 \cdot 10^{8}\ m/s\]

\subsection{Risoluzione delle equazioni di Maxwell in presenza di sorgenti}\label{risoluzione-delle-equazioni-di-maxwell}

Si vuole risolvere le equazioni di Maxwell in prossimità delle sorgenti \(\vec{J}\) e \(\rho\). In questo modo è possibile ricostruire l'intero campo elettromagnetico generato dalle sorgenti.

\[\begin{cases}
 \vec{\nabla} \cdot \vec{E} = \dfrac{\rho}{\varepsilon_{0}} \\
\vec{\nabla} \cdot \vec{B} = 0 \\
\vec{\nabla} \times \vec{E} = - \dfrac{\partial\vec{B}}{\partial t}  \\
\vec{\nabla} \times \vec{B} = \mu_{0}\left( \vec{J} + \varepsilon_{0}\dfrac{\partial\vec{E}}{\partial t} \right)
\end{cases}
\]

Dall'equazione di Gauss per il magnetismo, risulta che il campo induzione magnetica è solenoidale, quindi, è possibile definire un potenziale vettore \(\vec{A}\) tale che:

\[\vec{\nabla} \times \vec{A} = \vec{B}\]

Si sostituisce tale definizione nella terza equazione di Maxwell:

\[\vec{\nabla} \times \vec{E} = - \dfrac{\partial\vec{B}}{\partial t} = - \dfrac{\partial}{\partial t}\left( \vec{\nabla} \times \vec{A} \right)\]

L'operatore \(\vec{\nabla}\) è indipendente dalla derivata temporale, dunque, è lecita la loro inversione:

\[\vec{\nabla} \times \vec{E} = - \vec{\nabla} \times \dfrac{\partial\vec{A}}{\partial t}\]

Portando tutto al primo membro e applicando la linearità dell'operatore rotore, è possibile scrivere:

\[\vec{\nabla} \times \left( \vec{E} + \dfrac{\partial\vec{A}}{\partial t} \right) = \vec{0}\]

Il campo \(\vec{E} + \partial\vec{A}/\partial t\) è irrotazionale, per cui, è possibile definire un potenziale scalare \(\phi\) tale che:

\[\vec{E} + \dfrac{\partial\vec{A}}{\partial t} = - \vec{\nabla}\phi\]

Il potenziale scalare $\phi$ è generalmente definito con un segno negativo. Nell'elettrostatica, il potenziale scalare concide con la tensione $\vec{E} = -\vec{\nabla}V$.

In presenza di un campo magnetostatico, la derivata temporale di \(\vec{B}\) è nulla, per cui il potenziale scalare coincide con il potenziale elettrico:

\[\vec{E} = - \vec{\nabla}\phi\]

La forza elettrica ($\vec{F}$) deve essere opposta al gradiente del potenziale scalare ($\phi$) perché il potenziale $\phi$ rappresenta l'energia potenziale elettrica per unità di carica (ossia, la tensione o voltaggio).In fisica, una forza conservativa (come la forza elettrica) tende sempre a muovere un oggetto nella direzione in cui l'energia potenziale diminuisce più rapidamente. Il gradiente ($\vec{\nabla}\phi$) punta, per definizione, nella direzione in cui il potenziale aumenta più rapidamente. Per questo motivo, la relazione fondamentale tra campo elettrico ($\vec{E}$) e potenziale elettrico ($\phi$) deve includere un segno negativo

 potenziali vettore $\vec{A}$ e scalare $\phi$ non sono univocamente determinati. Per tale motivo, si adotta una condizione di Gauge per disaccoppiare le equazioni differenziali. La Gauge di Lorentz, data da:

\[\vec{\nabla} \cdot \vec{A} + \mu_0\varepsilon_{0}\dfrac{\partial\phi}{\partial t} = 0\]

è invariante quando si passa da un sistema di riferimento inerziale a un altro tramite le trasformazioni di Lorentz,il che è un requisito fondamentale della Relatività Ristretta. Per tale motivo si parla di condizione covariante in relatività.

Sotto la condizione di Gauge di Lorentz, le Equazioni di Maxwell si riducono a due equazioni d'onda disaccoppiate per $\vec{A}$ e $\phi$ con termini sorgente. La soluzione di queste equazioni porta ai potenziali ritardati.

Si dimostra che, per una sorgente puntiforme, il potenziale vettore, nel dominio della frequenza, è dato da:

\[\vec{A} = \dfrac{\mu_{0}}{4\pi}\dfrac{\vec{J}}{\left| \vec{r} - {\vec{r}}_{0} \right|}e^{- jk\left| \vec{r} - {\vec{r}}_{0} \right|}\]

Dove \({\vec{r}}_{0}\) è la posizione in cui sono posizionate le sorgenti, mentre \(\vec{r}\) è il punto nello spazio in cui si valuta il campo. La quantità:

\[g\left( \vec{r} \right) = \dfrac{1}{4\pi}\dfrac{1}{\left| \vec{r} \right|}e^{- jk\left| \vec{r} \right|}\]

È la funzione di Green o risposta impulsiva del mezzo, lineare e isotropo, visto come un sistema ingresso-uscita. Si definisce il numero d'onda (o costante di propagazione nel vuoto) come:

\[k = \omega\sqrt{\mu_{0}\varepsilon_{0}}=\dfrac{\omega}{c}\]

\subsection{Equazioni di Maxwell nel vuoto in forma globale}\label{equazioni-di-maxwell-nel-vuoto-in-forma-globale}

Le equazioni di Maxwell in forma locale non possono essere utilizzate in presenza di brusche variazioni della superficie dei volumi sui quali calcolare i campi. In questo caso si ricorre alla forma globale o integrale:

\[\begin{cases}
 \oiint_{S}{\vec{E} \cdot d\vec{S}\ } = \dfrac{Q}{\varepsilon_{0}} \\
 \oiint_{S}{\vec{B} \cdot d\vec{S}\ } = 0  \\
 \oint_{\partial S}{\vec{E} \cdot d\vec{s}} = - \dfrac{\partial}{\partial t}\int_{\Sigma}{\vec{B} \cdot d\vec{\Sigma}}  \\
 \oint_{\partial S}{\vec{B} \cdot d\vec{s}} = \mu_{0}I + \mu_{0}\varepsilon_{0}\dfrac{\partial}{\partial t}\int_{\Sigma}{\vec{E} \cdot d\vec{\Sigma}}
\end{cases}
\]

Dove \(S\) è una superficie all'interno della quale si vuole valutare il campo, mentre \(\Sigma\) è una superficie aperta avente per contorno \(\partial S\). \(Q\) è la quantità di carica contenuta in \(S\) e \(I\) la corrente che attraversa \(\Sigma\).

\section{Equazioni di Maxwell nel mezzo}\label{equazioni-di-maxwell-nel-mezzo}

In presenza di un mezzo materiale si introducono i vettori di induzione elettrica \(\vec{D}\) e di campo magnetico \(\vec{H}\). Il vettore \(\vec{B}\) è detto vettore di induzione magnetica, mentre \(\vec{E}\) rappresenta il campo elettrico.

I campi di induzione descrivono il comportamento del materiale a seguito dell'applicazione dei rispettivi campi. Con l'introduzione dei campi \(\vec{E}\), \(\vec{D}\), \(\vec{H}\) e \(\vec{B}\), le equazioni di Maxwell in forma locale si scrivono come:

\[
\begin{cases}
\vec{\nabla} \cdot \vec{D} = \rho_{lib} \\
\vec{\nabla} \cdot \vec{B} = 0 \\
\vec{\nabla} \times \vec{E} = - \dfrac{\partial \vec{B}}{\partial t} \\
\vec{\nabla} \times \vec{H} = \vec{J}_{lib} + \dfrac{\partial \vec{D}}{\partial t}
\end{cases}
\]

Dove \(\rho_{lib}\) e \(\vec{J}_{lib}\) sono le sorgenti libere ovvero le cariche e correnti che possono muoversi liberamente all’interno o all’esterno di un materiale, non vincolate alla struttura atomica o molecolare del mezzo. Queste sorgenti possono essere controllate direttamente, ad esempio, applicando una tensione o facendo passare una corrente in un conduttore. In contrapposizione, ci sono le sorgenti vincolate (o di legame), che derivano dal comportamento microscopico delle molecole e atomi del materiale. I vettori di polarizzazione e di magnetizzazione dipendono da tali sorgenti.

Il vettore di induzione elettrica è definito come:

\[
\vec{D} = \varepsilon_{0}\vec{E} + \vec{P}
\]

dove \(\vec{P}\) è il vettore di polarizzazione; esso descrive come i dipoli elettrici contenuti nel materiale si orientano sotto l'effetto del campo elettrico.

L'induzione magnetica è invece definita come:

\[
\vec{B} = \mu_{0}(\vec{H} + \vec{M})
\]

dove \(\vec{M}\) è il vettore di magnetizzazione e descrive come i dipoli magnetici nel materialesi orientano sotto l’azione del campo magnetico applicato H \(\vec{H}\).

Per i mezzi lineari e isotropi esistono due costanti, dette suscettibilità elettrica \(\chi_{e}\) e suscettibilità magnetica \(\chi_{m}\), che rappresentano i coefficienti di proporzionalità, rispettivamente, tra i campi di polarizzazione e magnetizzazione e i relativi campi applicati:

\[
\vec{P} = \varepsilon_{0}\chi_{e}\vec{E}, \qquad \vec{M} = \chi_{m}\vec{H}
\]

Sostituendo il vettore di polarizzazione nella definizione del vettore di induzione elettrica si ottiene:

\[
\vec{D} = \varepsilon_{0}\vec{E} + \vec{P} = \varepsilon_{0}\vec{E} + \varepsilon_{0}\chi_{e}\vec{E} = \varepsilon_{0}\left( 1 + \chi_{e} \right)\vec{E}
\]

Si definisce la costante dielettrica del mezzo, per un materiale lineare e isotropo, come:

\[
\varepsilon = \varepsilon_{0}\left( 1 + \chi_{e} \right)
\]

Con questa definizione si ha:

\[
\vec{D} = \varepsilon\vec{E}
\]

Analogamente, per il campo di induzione magnetica:

\[
\vec{B} = \mu_{0}\vec{H} + \mu_{0}\vec{M} = \mu_{0}\vec{H} + \mu_{0}\chi_{m}\vec{H} = \mu_{0}\left( 1 + \chi_{m} \right)\vec{H}
\]

Si definisce la permeabilità magnetica del mezzo, per un materiale lineare e isotropo, come:

\[
\mu = \mu_{0}\left( 1 + \chi_{m} \right)
\]

Il campo di induzione magnetica si scrive quindi come:

\[
\vec{B} = \mu\vec{H}
\]

\begin{figure}[ht]
\centering
\includegraphics[width=3.19044in,height=2.4711in,alt={P1240\#yIS1}]{media/3_Elettromagnetismo/image19.pdf}
\caption{Dipolo elettrico e magnetico.}
\end{figure}

Infine, la densità di corrente totale si suddivide in una componente libera, associata alle cariche mobili, e in una componente vincolata, dovuta alla polarizzazione e alla magnetizzazione del materiale. Si scrive dunque:

\[
\vec{J} = \vec{J}_{lib} + \vec{J}_{vinc}
\]

La densità di corrente vincolata (o di legame) è data da:

\[
\vec{J}_{vinc} = \dfrac{\partial \vec{P}}{\partial t} + \vec{\nabla} \times \vec{M}
\]

Per un mezzo lineare e isotropo, sostituendo le relazioni costitutive si ottiene:

\[
\vec{J}_{vinc} = \varepsilon_0 \chi_e \dfrac{\partial \vec{E}}{\partial t} + \left(\vec{\nabla} \times \left(\chi_{m} \vec{H}\right)\right)
\]

Se il mezzo è omogeneo, $\chi_m$ è una costante spaziale. In questa ipotesi, la densità di corrente vincolata si scrive come:

\[
\vec{J}_{vinc} = \varepsilon_0 \chi_e \dfrac{\partial \vec{E}}{\partial t} + \chi_{m}\left(\vec{\nabla} \times \vec{H}\right)
\]

La densità di corrente totale \(\vec{J}\) dipende dunque sia dal campo elettrico sia dal campo magnetico. Nella quarta equazione di Maxwell, tuttavia, la densità di corrente si riferisce solo alla parte libera:

\[
\vec{\nabla} \times \vec{H} = \vec{J}_{lib} + \dfrac{\partial \vec{D}}{\partial t}
\]

In definitiva, i campi fondamentali che descrivono le forze elettromagnetiche sono \(\vec{E}\) e \(\vec{B}\), mentre \(\vec{D}\) e \(\vec{H}\) sono campi ausiliari introdotti per semplificare le equazioni di Maxwell in presenza di un mezzo materiale (come un dielettrico o un magnete), inglobando negli stessi gli effetti microscopici di polarizzazione (\(\vec{P}\)) e magnetizzazione (\(\vec{M}\)).

\section{Forza di Lorentz}\label{forza-di-lorentz}

I campi elettrico \(\vec{E}\) e di induzione magnetica \(\vec{B}\) esercitano una forza su una particella di carica \(q\) che si muove con velocità \(\vec{v}\) in una regione di spazio contenente tali campi. L’espressione della forza totale è:

\[
\vec{F} = q\vec{E} + q\vec{v} \times \vec{B}
\]

Questa interazione è nota come \textbf{forza di Lorentz}. Essa si compone di due contributi distinti:

\begin{itemize}
    \item \textbf{Forza elettrica}: \(\vec{F}_{E} = q\vec{E}\).
    Questa forza è parallela o antiparallela al campo elettrico, a seconda del segno della carica \(q\), ed è indipendente dalla velocità \(\vec{v}\) della particella.
    
    \item \textbf{Forza magnetica}: \(\vec{F}_{B} = q(\vec{v} \times \vec{B})\).
    Questa forza è sempre perpendicolare sia alla velocità \(\vec{v}\) sia al campo di induzione magnetica \(\vec{B}\). Poiché la direzione della forza è perpendicolare allo spostamento della particella, la forza magnetica \textit{non compie lavoro} sulla carica.
\end{itemize}

\section{Parallelismo tra campo elettrico e magnetico}\label{parallelismo-tra-campo-magnetico-ed-elettrico}

Le equazioni di Maxwell possono essere rese formalmente simmetriche introducendo le \textit{cariche magnetiche} e le \textit{correnti magnetiche}, concetti ipotetici non osservati sperimentalmente ma utili per evidenziare l’analogia tra fenomeni elettrici e magnetici.

Si indichi con \(U\) l’energia potenziale di un dipolo elettrico \(\vec{d}\) o magnetico \(\vec{\mu}\), immerso rispettivamente in un campo elettrico o magnetico.  
Conoscendo \(U\), è possibile determinare la forza \(\vec{F}\) e il momento torcente \(\vec{N}\) agente sul dipolo.

Per il campo elettrico:

\[
\begin{cases}
 U = - \vec{d} \cdot \vec{E} \\
 \vec{F} = - \vec{\nabla}U = \vec{\nabla}\left( \vec{d} \cdot \vec{E} \right) \\
 \vec{N} = \vec{d} \times \vec{E} \\
 \phi(\vec{r}) = \dfrac{1}{4\pi\varepsilon_0}\int \dfrac{\rho(\vec{r}')}{R}\,dV' \\
 \vec{E} = - \vec{\nabla}\phi \\
 \nabla^{2}\phi = -\dfrac{\rho}{\varepsilon_{0}}
\end{cases}
\]

Dove:
\begin{itemize}
    \item \(\vec{d}\) è il \textbf{momento di dipolo elettrico}, vettore diretto dalla carica negativa a quella positiva, proporzionale al modulo della carica e alla distanza di separazione;
    \item \(\vec{E}\) è il \textbf{campo elettrico}, che rappresenta la forza per unità di carica esercitata nello spazio;
    \item \(\phi\) è il \textbf{potenziale elettrico scalare}, da cui il campo deriva come gradiente negativo;
    \item \(\rho\) è la \textbf{densità di carica elettrica}, sorgente del campo elettrico;
    \item \(\varepsilon_0\) è la \textbf{costante dielettrica del vuoto}, che stabilisce la relazione tra campo elettrico e densità di carica.
\end{itemize}

Per il campo magnetico risulta:

\[
\begin{cases}
 U = - \vec{\mu} \cdot \vec{B} \\
 \vec{F} = - \vec{\nabla}U = \vec{\nabla}\left( \vec{\mu} \cdot \vec{B} \right) \\
 \vec{N} = \vec{\mu} \times \vec{B} \\
 \vec{A}(\vec{r}) = \dfrac{\mu_0}{4\pi}\int \dfrac{\vec{J}(\vec{r}')}{R}\,dV' \\
 \vec{B} = \vec{\nabla} \times \vec{A} \\
 \nabla^{2}\vec{A} = -\mu_{0}\vec{J}
\end{cases}
\]

Dove:
\begin{itemize}
    \item \(\vec{\mu}\) è il \textbf{momento di dipolo magnetico}, legato al moto circolare di cariche (correnti) e orientato secondo la regola della mano destra rispetto al verso della corrente;
    \item \(\vec{B}\) è il \textbf{campo magnetico}, che descrive l’azione sui dipoli magnetici e sulle particelle cariche in movimento;
    \item \(\vec{A}\) è il \textbf{potenziale vettore magnetico}, da cui si ricava il campo magnetico mediante il rotore;
    \item \(\vec{J}\) è la \textbf{densità di corrente elettrica}, sorgente del campo magnetico;
    \item \(\mu_0\) è la \textbf{permeabilità magnetica del vuoto}, che regola l’intensità del campo magnetico generato dalle correnti.
\end{itemize}

Queste relazioni mostrano il parallelismo formale tra le grandezze elettriche \((\vec{d}, \vec{E}, \phi, \rho, \varepsilon_0)\) e le corrispondenti magnetiche \((\vec{\mu}, \vec{B}, \vec{A}, \vec{J}, \mu_0)\).

\begin{table}[h!]
\centering
\begin{tabular}{@{} c c l @{}}
\toprule
\textbf{Elettrico} & \textbf{Magnetico} & \textbf{Descrizione} \\ 
\midrule
$\vec{d}$ & $\vec{\mu}$ & Momento di dipolo (separazione di cariche o corrente circolare) \\
$\vec{E}$ & $\vec{B}$ & Campo vettoriale (forza per unità di carica o effetto su dipoli) \\
$\phi$ & $\vec{A}$ & Potenziale (scalare o vettoriale) da cui deriva il campo \\ 
$\rho$ & $\vec{J}$ & Sorgente del campo (cariche o correnti) \\
$\varepsilon_0$ & $\mu_0$ & Costanti fondamentali del vuoto \\
\bottomrule
\end{tabular}
\caption{Confronto tra le grandezze elettriche e magnetiche e loro significato fisico.}
\label{tab:parallelo-elettromagnetico}
\end{table}

Questa analogia sottolinea la profonda simmetria formale tra elettricità e magnetismo, unificate nella teoria elettromagnetica di Maxwell.

\section{Forza e coppia su una piccola spira}\label{forza-e-coppia-su-una-piccola-spira}

Si vuole determinare la forza totale agente su una piccola spira di superficie \(S\), percorsa da una corrente \(I\), immersa in un campo di induzione magnetica uniforme.

\begin{figure}[ht]
\centering
\includegraphics[width=0.35\textwidth]{media/3_Elettromagnetismo/image20.pdf}
\caption{Spira percorsa da corrente in un campo magnetico.}
\end{figure}

Il momento magnetico dipolare è dato da:
\[
\vec{\mu} = I S\,\hat{\imath}_n
\]
dove \(\hat{\imath}_n\) è la normale alla superficie \(S\), orientata in modo da vedere la corrente ruotare in senso antiorario.

Si consideri un tratto elementare \(d\vec{s}\) della spira; su di esso agisce la forza di Lorentz, poiché le cariche al suo interno sono messe in moto dalla corrente elettrica:
\[
d\vec{F} = I\,d\vec{s} \times \vec{B}
\]
Integrando su tutta la lunghezza della spira si ottiene:
\[
\vec{F} = \oint_{\partial S} I\,d\vec{s} \times \vec{B}
\]
Poiché la corrente e il campo \(\vec{B}\) sono costanti, possono essere portati fuori dal segno di integrale:
\[
\vec{F} = I\left( \oint_{\partial S} d\vec{s} \right) \times \vec{B}
\]
Siccome la spira è in equilibrio, la somma di tutti i contributi elementari è nulla; di conseguenza la forza risultante è nulla:
\[
\vec{F} = \vec{0}
\]
ovvero:
\[
I\left( \oint_{\partial S} d\vec{s} \right) \times \vec{B} = \vec{0}
 \Longleftrightarrow
\left( \oint_{\partial S} d\vec{s} \right) \times \vec{B} = \vec{0}
\]

Anche se la forza netta agente sulla spira è nulla, possono comunque esistere momenti torcenti. Poiché la forza agente sulla spira elementare è nulla, il momento delle forze risulta indipendente dal polo scelto. La coppia agente sull’elemento infinitesimo è data da:
\[
d\vec{N} = \vec{r} \times d\vec{F}
\]
La coppia, o momento torcente, può essere denotata anche con \(\vec{\tau}\). Integrando lungo la spira si ottiene:
\[
\vec{N} = \oint_{\partial S} \vec{r} \times d\vec{F}
\]
Sostituendo la forza di Lorentz \(d\vec{F} = I\,d\vec{s} \times \vec{B}\), si ha:
\[
\vec{N} = I \oint_{\partial S} \vec{r} \times (d\vec{s} \times \vec{B})
\]
Dati tre vettori, è valida l’identità:
\[
\vec{a} \times (\vec{b} \times \vec{c}) = (\vec{a}\cdot\vec{c})\,\vec{b} - (\vec{a}\cdot\vec{b})\,\vec{c}
\]
Applicando tale relazione, l'espressione del momento si scrive come:
\[
\vec{N} = I \oint_{\partial S} \big[(\vec{r}\cdot\vec{B})\,d\vec{s} - \vec{B}\,(\vec{r}\cdot d\vec{s})\big]
\]
Poiché il vettore \(\vec{r}\) appartiene alla spira, risulta \(d\vec{s} = d\vec{r}\), e dunque:
\[
\vec{N} = I \oint_{\partial S} \big[(\vec{r}\cdot\vec{B})\,d\vec{r} - \vec{B}\,(\vec{r}\cdot d\vec{r})\big]
\]
Il termine \(\vec{r}\cdot d\vec{r}\) può essere riscritto come:
\[
\vec{r}\cdot d\vec{r} = \dfrac{1}{2}\,d(\vec{r}\cdot\vec{r})
\]
per cui:

\[\vec{N} = I\oint_{\partial S}{\left( \vec{r} \cdot \vec{B} \right)d\vec{r}} - I\oint_{\partial S}{\vec{B}\left( \vec{r} \cdot d\vec{r} \right)} = I\oint_{\partial S}{\left( \vec{r} \cdot \vec{B} \right)d\vec{r}} - \dfrac{1}{2}I\vec{B}\oint_{\partial S}{d\left( \vec{r} \cdot \vec{r} \right)}\]

L'ultimo termine è l'integrale esteso a una linea chiusa di una forma differenziale esatta, dunque, è nullo:

\[\oint_{\partial S}{d\left( \vec{r} \cdot \vec{r} \right)} = 0\]

Pertanto, la coppia risulta:

\[\vec{N} = I\oint_{\partial S}{\left( \vec{r} \cdot \vec{B} \right)d\vec{r}}\]

Per risolvere tale integrale si considera la quantità \(d\left( \left( \vec{r} \cdot \vec{B} \right)\vec{r} \right)\); si applicano le proprietà del differenziale:

\[d\left( \left( \vec{r} \cdot \vec{B} \right)\vec{r} \right) = \left( d\vec{r} \cdot \vec{B} \right)\vec{r} + \left( \vec{r} \cdot \vec{B} \right)d\vec{r} + \left( \vec{r} \cdot d\vec{B} \right)\vec{r}\]

Dato che il campo è uniforme, \(d\vec{B} = 0\). Per cui:

\[d\left( \left( \vec{r} \cdot \vec{B} \right)\vec{r} \right) = \left( d\vec{r} \cdot \vec{B} \right)\vec{r} + \left( \vec{r} \cdot \vec{B} \right)d\vec{r}\]

Si ricava la quantità \(\left( \vec{r} \cdot \vec{B} \right)d\vec{r}\), presente nell'espressione della coppia:

\[\left( \vec{r} \cdot \vec{B} \right)d\vec{r} = d\left( \left( \vec{r} \cdot \vec{B} \right)\vec{r} \right) - \left( d\vec{r} \cdot \vec{B} \right)\vec{r}\]

Per applicare nuovamente la proprietà sul prodotto vettore di tre vettori, si aggiunge e sottrae \(\left( \vec{r} \cdot \vec{B} \right)d\vec{r}\) al secondo membro:

\[\left( \vec{r} \cdot \vec{B} \right)d\vec{r} = d\left( \left( \vec{r} \cdot \vec{B} \right)\vec{r} \right) - \left( d\vec{r} \cdot \vec{B} \right)\vec{r} - \left( \vec{r} \cdot \vec{B} \right)d\vec{r} + \left( \vec{r} \cdot \vec{B} \right)d\vec{r}\]

Dove:

\[\left( \vec{r} \cdot \vec{B} \right)d\vec{r} - \left( d\vec{r} \cdot \vec{B} \right)\vec{r} = \vec{r} \times d\vec{r} \times \vec{B}\]

Per cui si ha:

\[\left( \vec{r} \cdot \vec{B} \right)d\vec{r} = d\left( \left( \vec{r} \cdot \vec{B} \right)\vec{r} \right) + \vec{r} \times d\vec{r} \times \vec{B} - \left( \vec{r} \cdot \vec{B} \right)d\vec{r} \Leftrightarrow 2\left( \vec{r} \cdot \vec{B} \right)d\vec{r} = d\left( \left( \vec{r} \cdot \vec{B} \right)\vec{r} \right) + \vec{r} \times d\vec{r} \times \vec{B}\]

In definitiva, si ha:

\[\left( \vec{r} \cdot \vec{B} \right)d\vec{r} = \dfrac{1}{2}d\left( \left( \vec{r} \cdot \vec{B} \right)\vec{r} \right) + \dfrac{1}{2}\vec{r} \times d\vec{r} \times \vec{B}\]

Sostituendo tale risultato nell’espressione della coppia si ha::

\[\vec{N} = I\oint_{\partial S}{\left( \vec{r} \cdot \vec{B} \right)d\vec{r}} = \dfrac{1}{2}\ I\oint_{\partial S}{d\left( \left( \vec{r} \cdot \vec{B} \right)\vec{r} \right)} + \dfrac{1}{2}\ I\oint_{\partial S}{\vec{r} \times d\vec{r} \times \vec{B}}\]

Il primo termine è nullo, essendo una circuitazione di una forma differenziale esatta, quindi:

\[\oint_{\partial S}{d\left( \left( \vec{r} \cdot \vec{B} \right)\vec{r} \right)} = \vec{0}\]

Per cui si ha:

\[\vec{N} = \dfrac{1}{2}\ I\oint_{\partial S}{\vec{r} \times d\vec{r} \times \vec{B}}\]

Siccome il campo è costante lungo tutto il percorso di integrazione, può essere portato all'esterno del simbolo di integrale:

\[\vec{N} = \dfrac{1}{2}\ I\left( \oint_{\partial S}{\vec{r} \times d\vec{r}} \right) \times \vec{B}\]

Si definisce il \textit{momento magnetico} della spira:

\[\vec{\mu} = \dfrac{1}{2}\ I\oint_{\partial S}{\vec{r} \times d\vec{r}} = IS{\hat{\imath}}_{n}\]

La coppia \(\vec{N}\) può, in definitiva, essere espressa come:

\[\vec{N} = \vec{\mu} \times \vec{B}\]

Nel caso più generale, il momento magnetico di una distribuzione continua di corrente in un volume \(V\) è dato da:

\[\vec{\mu} = \dfrac{1}{2}\int_{V}{\vec{r} \times \vec{J}dV}\]

\subsection{Momento magnetico di un anello rotante}\label{momento-magnetico-di-un-anello-rotante}

Si vuole calcolare il momento magnetico di un anello sottile di raggio \(R\), massa \(M\) e carica \(q\) distribuita uniformemente, che ruota con velocità angolare \(\omega\) attorno ad un asse ortogonale al piano in cui giace l'anello e passante per il centro.

\begin{figure}[ht]
\centering
\includegraphics[width=1.34709in,height=2.1703in,alt={P1332\#yIS1}]{media/3_Elettromagnetismo/image21.pdf}\caption{Anello rotante nel campo magnetico}
\end{figure}

L’anello, ruotando, genera una corrente la cui intensità media è:

\[I = \dfrac{q}{T}\]

Dove \(T\) è il periodo dell'oscillazione dato da:

\[T = \dfrac{2\pi}{\omega}\]

Per cui la corrente è data da:

\[I = \dfrac{q}{T} = q\dfrac{\omega}{2\pi}\]

L'anello è assimilabile a una spira percorsa da corrente, dunque, il suo momento magnetico è dato da:

\[\vec{\mu} = IS{\hat{\imath}}_{n} = q\dfrac{\omega}{2\pi}\pi R^{2}{\hat{\imath}}_{n}\]

Semplificando, si ottiene il momento magnetico:

\[\vec{\mu} = \dfrac{1}{2}q\omega R^{2}{\hat{\imath}}_{n}\]

\subsection{Momento magnetico per un guscio sferico}\label{momento-magnetico-per-un-guscio-sferico}

Si vuole calcolare il momento magnetico di uno strato sferico sottile di raggio \(R\), massa \(M\) e carica \(Q\) distribuita uniformemente, che ruota con velocità angolare \(\omega\) attorno ad un asse passante per il centro della sfera.

\begin{figure}[ht]
\centering
\includegraphics[width=2.625in,height=1.991in,alt={P1346\#yIS1}]{media/3_Elettromagnetismo/image22.pdf}\caption{Guscio sferico}
\end{figure}

Si pone l'asse \(z\) coincidente con l'asse di rotazione, per cui \(\vec{\omega}=\omega\,\hat{z}\). Si suddivide la superficie sferica in anelli infinitesimi, ciascuno giacente su un piano ortogonale all'asse \(z\). In corrispondenza dell'angolo polare \(\vartheta\) l'anello ha:
\[
\text{raggio: } r=R\sin\vartheta,\qquad
\text{lunghezza: } 2\pi R\sin\vartheta,\qquad
\text{larghezza (sulla sfera): } R\,d\vartheta.
\]

L'elemento di area dell'anello è quindi:
\[
dS = (2\pi R\sin\vartheta)(R\,d\vartheta)=2\pi R^{2}\sin\vartheta\,d\vartheta.
\]

Se \(\sigma\) è la densità superficiale di carica uniforme, la carica dell'anello è \(dq=\sigma\,dS\) e la corrente associata alla sua rotazione è:
\[
dI=\dfrac{dq}{T}=\dfrac{\sigma\,dS}{T}=\dfrac{\omega\sigma}{2\pi}\,2\pi R^{2}\sin\vartheta\,d\vartheta
=\omega\sigma R^{2}\sin\vartheta\,d\vartheta,
\]
con \(T=2\pi/\omega\) periodo di rotazione della sfera. L'area della spira (anello) è \(S=\pi (R\sin\vartheta)^{2}\), dunque il momento magnetico elementare vale:
\[
d\vec{\mu}=S\,dI\,\hat{z}
= \pi R^{2}\sin^{2}\vartheta\;\omega\sigma R^{2}\sin\vartheta\,d\vartheta\;\hat{z}
= \pi\omega\sigma R^{4}\sin^{3}\vartheta\,d\vartheta\;\hat{z}.
\]

Per le ipotesi fatte sul sistema di riferimento, risulta che \({\hat{\imath}}_{n} = {\hat{\imath}}_{z}\). Infatti, il momento magnetico è diretto come \(\vec{\omega}\). Svolgendo i prodotti, si ha:

\[d\vec{\mu} = \pi\omega\sigma R^{4}\sin^{3}\vartheta\ d\vartheta{\hat{\imath}}_{z}\]

Il momento magnetico è ottenuto integrando l'equazione ottenuta su tutti i possibili valori assunti da \(\vartheta\), ovvero:

\[\vec{\mu} = \int_{0}^{\pi}{\pi\omega\sigma R^{4}\sin^{3}\vartheta\ d\vartheta{\hat{\imath}}_{z}} = \pi\omega\sigma R^{4}\int_{0}^{\pi}{\sin^{3}\vartheta\ d\vartheta}{\hat{\imath}}_{z}\]

Si risolve l'integrale. Per le relazioni trigonometriche è possibile scrivere:

\[\int_{0}^{\pi}{\sin^{3}\vartheta d\vartheta} = \int_{0}^{\pi}{\left( \dfrac{3\sin\vartheta - \sin{3\vartheta}}{4} \right)d\vartheta} = \dfrac{1}{4}\left( 3\int_{0}^{\pi}{\sin\vartheta d\vartheta} - \int_{0}^{\pi}{\sin{3\vartheta}d\vartheta} \right)\]

Dove:

\[3\int_{0}^{\pi}{\sin\vartheta d\vartheta} = 3\left\lbrack - \cos\vartheta \right\rbrack_{0}^{\pi} = 3\left( - \cos\pi + \cos 0 \right) = 3(1 + 1) = 6\]

\[\int_{0}^{\pi}{\sin{3\vartheta}d\vartheta} = - \dfrac{1}{3}\left\lbrack \cos{3\vartheta} \right\rbrack_{0}^{\pi} = - \dfrac{1}{3}\left( \cos{3\pi} - \cos 0 \right) = - \dfrac{1}{3}( - 1 - 1) = \dfrac{2}{3}\]

Nel complesso, l'integrale è dato da:

\[\int_{0}^{\pi}{\sin^{3}\vartheta d\vartheta} = \dfrac{1}{4}\left( 6 - \dfrac{2}{3} \right) = \dfrac{1}{4}\left( \dfrac{18 - 2}{3} \right) = \dfrac{1}{4}\dfrac{16}{3} = \dfrac{4}{3}\]

Il momento magnetico è dato da:

\[\vec{\mu} = \dfrac{4}{3}\ \pi\omega\sigma R^{4}{\hat{\imath}}_{z}\]

La superficie totale della sfera è data da:

\[S = 4\pi R^{2}\]

Il prodotto della densità superficiale di carica \(\sigma\) per la superficie \(S\) restituisce la carica globale \(Q\). Il momento magnetico può essere scritto come:

\[\vec{\mu} = \dfrac{1}{3}\ \omega QR^{2}{\hat{\imath}}_{z}\]

%\input{Image/Ring}

\subsection{Momento magnetico per una sfera}\label{momento-magnetico-per-una-sfera}

Si vuole calcolare il momento magnetico di una sfera raggio \(R\), massa \(M\) e carica \(Q\) distribuita uniformemente, che ruota con velocità angolare \(\omega\) attorno ad un asse passante per il centro della sfera.

\begin{figure}[ht]
\centering
\includegraphics[width=1.375in,height=1.73838in,alt={P1372\#yIS1}]{media/3_Elettromagnetismo/image23.pdf}\caption{Sfera rotante}
\end{figure}

Si utilizzano le coordinate sferiche \(r\), \(\vartheta\), \(\varphi\). Si assume come polo il centro \(O\) della sfera carica e come asse polare il diametro parallelo alla velocità angolare\(\omega\). Dove:

\[r \in \left[ 0; R\right],\ \vartheta \in \left[ 0;\pi\right],\ \varphi \in \left[ 0;2\pi\right]\]

Si indica con \(\rho\) la densità di carica volumetrica. Il volumetto \(dV\) contiene una carica \(dq\):

\[dq = \rho dV\]

Un elemento di volume $dV$ contribuisce alla densità di corrente volumetrica $\vec{J}$, data da:

\[\vec{J} = \rho \vec{v}\]

Poiché la sfera ruota con velocità angolare \(\vec{\omega}\) e la carica ha densità \(\rho\), la densità di corrente volumetrica è:

\[\vec{J} = \rho \vec{v} = \rho \left( \vec{\omega} \times \vec{r}\right)\]

per calcolare il momento magnetico di un corpo di volume \(V\) con densità di corrente \(J\), si usa la formula generale:

\[\vec{\mu} = \dfrac{1}{2}\int_{V}{\vec{r} \times \vec{J}dV}\]

Sostituendo la densità di corrente dell'elemento di volume infinitesimo nella definizione del momento magnetico si ha:

\[\vec{\mu} = \dfrac{1}{2}\int_{V}{\vec{r} \times \left(\rho \left( \vec{\omega} \times \vec{r}\right)\right)dV}\]

Nell'ipotesi di densità di carica costante in tutto il volume, è possibile portare all'esterno del simbolo di integrale \(\rho\):


\[\vec{\mu} = \dfrac{1}{2}\rho\int_{V}{\vec{r} \times \left( \vec{\omega} \times \vec{r}\right)dV}\]

Si considera il triplo prodotto vettoriale all'interno dell'integrale e si utilizza l'identità nota:

\[
\vec{a} \times (\vec{b} \times \vec{c}) = (\vec{a} \cdot \vec{c})\vec{b} - (\vec{a} \cdot \vec{b})\vec{c}
\]

Ponendo \(\vec{a} = \vec{r}\), \(\vec{b} = \vec{\omega}\), e \(\vec{c} = \vec{r}\), l'espressione \(\vec{r} \times \vec{\omega} \times \vec{r} \) diventa:

\[
\vec{r} \times \vec{\omega} \times \vec{r} = (\vec{r} \cdot \vec{r})\vec{\omega} - (\vec{r} \cdot \vec{\omega})\vec{r}
\]

Ricordando che \(\vec{r} \cdot \vec{r} = r^2\), l'ultima relazione può essere scritta come:

\[
\vec{r} \times \vec{\omega} \times \vec{r} = r^{2}\vec{\omega}-\left(\vec{r}\cdot\vec{\omega}\right)\vec{r}
\]

Per ipotesi, la rotazione avviene lungo la direzione \(\hat{\imath}_{z}\), per cui \(\vec{\omega}=\omega \hat{\imath}_{z}\). Inoltre, è possibile esprimere il vettore posizione in coordinate sferiche:

\[
\vec{r} = r (\sin\vartheta \cos\varphi\,\hat{\imath}_x + \sin\vartheta \sin\varphi\,\hat{\imath}_y + \cos\vartheta\,\hat{\imath}_z)
\]

Il prodotto scalare tra il vettore posizione e la velocità angolare avviene solamente tre le componenti lungo \(\hat{\imath}_z\):

\[
\vec{r}\cdot \vec{\omega} = r (\sin\vartheta \cos\varphi\,\hat{\imath}_x + \sin\vartheta \sin\varphi\,\hat{\imath}_y + \cos\vartheta\,\hat{\imath}_z) \cdot \omega\hat{\imath}_z = r\omega\cos\vartheta\
\]

Il prodotto vettoriale triplo si scrive come:

\[
\vec{r} \times \vec{\omega} \times \vec{r} = {r}^2\omega\hat{\imath}_z - \left(\omega r \cos\vartheta\right)\left(r (\sin\vartheta \cos\varphi\,\hat{\imath}_x + \sin\vartheta \sin\varphi\,\hat{\imath}_y + \cos\vartheta\,\hat{\imath}_z)\right)
\]

Svolgendo i prodotti, si ricava:

\[
\vec{r} \times \vec{\omega} \times \vec{r} = {r}^2\omega\hat{\imath}_z - r^{2}\omega\left( \sin{\vartheta}\cos{\varphi}\,\cos\vartheta\hat{\imath}_{x} + \sin\vartheta \sin\varphi\,\cos\vartheta\hat{\imath}_y+\cos^2\vartheta\hat{\imath}_z\right)
\]

Sostituendo questo risultato nell'espressione vettoriale per il momento magnetico, si ottiene:

\[
\vec{\mu} = \dfrac{1}{2}\rho\int_{V}{ \left({r}^2\omega\hat{\imath}_z - r^{2}\omega\left( \sin{\vartheta}\cos{\varphi}\,\cos\vartheta\hat{\imath}_{x} + \sin\vartheta \sin\varphi\,\cos\vartheta\hat{\imath}_y+\cos^2\vartheta\hat{\imath}_z\right)\right)dV}
\]

Si raccoglie \(\omega\) che, essendo costante rispetto la posizione, può essere portata fuori dal simbolo di integrale:

\[
\vec{\mu} = \dfrac{1}{2}\rho\omega \int_{V}{ \left({(r}^2\hat{\imath}_z - r^{2}\left( \sin{\vartheta}\cos{\varphi}\,\cos\vartheta\hat{\imath}_{x} + \sin\vartheta \sin\varphi\,\cos\vartheta\hat{\imath}_y+\cos^2\vartheta\hat{\imath}_z\right)\right)dV}
\]

Si divide l'integrale in base alle componenti spaziali delle coordinate sferiche:

\[
\vec{\mu} = \dfrac{1}{2}\rho\omega \left(\int_{V}{ \left(r^2\,\hat{\imath}_z - r\cos^{2}{\vartheta} \right)\hat{\imath}_z\,dV} - \int_{V}{\left( r^{2} \sin{\vartheta}\cos{\varphi}\,\cos\vartheta\hat{\imath}_{x} \right) \,dV} - \int_{V}{\left( r^2sin\vartheta \sin\varphi\,\cos\vartheta\hat{\imath}_y\right) \,dV}\right)
\]

In coordinate sferiche, l'elemento di volume infinitesimo è il prodotto delle tre lunghezze elementari:

\[
dV = 
\underbrace{dr}_{\text{spessore radiale}}\,
\underbrace{(r\,d\vartheta)}_{\text{altezza meridiana}}\,
\underbrace{(r\sin\vartheta\,d\varphi)}_{\text{lunghezza azimutale}}
= r^{2}\sin\vartheta\,dr\,d\vartheta\,d\varphi
\]

Si risolve l'integrale contenente la componente lungo la direzione \(\hat{\imath}_x\). Utilizzando le coordinate sferiche e integrando su tutto il volume, si ottiene:

\[
\int_{V}{\left( r^{2} \sin{\vartheta}\cos{\varphi}\,\cos\vartheta \right) \,dV}=\int_{0}^{R}{\int_{0}^{\pi}{\int_{0}^{2\pi}{r^{2} \sin{\vartheta}\cos{\varphi}\,\cos\vartheta \left(r^{2}\sin\vartheta\right)d\varphi}d\vartheta}dr}=
\]

Per la linearità dell'operatore integrale è possibile scrivere:

\[
=\int_{0}^{R}{r^{4}\int_{0}^{\pi}{\sin^{2}{\vartheta}\cos{\vartheta}\int_{0}^{2\pi}{cos{\varphi}\,d\varphi}d\vartheta}dr}
\]

Nella relazione individuata il \(cos{\varphi}\) è integrato sul suo periodo, pertanto il suo risultato è nullo. Di conseguenza anche l'integrale complessivo è nullo:

\[
\int_{V}{\left( r^{2} \sin{\vartheta}\cos{\varphi}\,\cos\vartheta \right) \,dV}=\int_{0}^{R}{r^{4}\int_{0}^{\pi}{\sin^{2}\cos{\vartheta}\int_{0}^{2\pi}{cos{\varphi}\,d\varphi}d\vartheta}dr}=0
\]

Per l'integrazione lungo la componente \(\hat{\imath}_y\) vale un discorso analogo, dunque, anche questo contributo è nullo. Le componenti $\mu_x$ e $\mu_y$ del momento magnetico si annullano a causa della simmetria di rotazione della sfera attorno all'asse $\hat{\imath}_{z}$, espressa matematicamente dall'integrazione su $\varphi$ (l'angolo azimutale) che annulla i termini $\cos\varphi$ e $\sin\varphi$. Il momento magnetico totale è quindi solo in direzione $\hat{\imath}_z$:

\[
\vec{\mu} = \mu_z\,\hat{\imath}_z
\]

Al fine di valutare l'espressione del momento magnetico, bisogna risolvere l'integrale lungo l'asse di rotazione:

\[
\mu_{z} = \dfrac{1}{2}\rho\omega\int_{V}{\left(r^{2} - r^{2}\cos^2{\vartheta} \right) dV} = \dfrac{1}{2}\rho\omega\int_{V}{r^{2}\left(1 -\cos^2{\vartheta} \right) dV}
\]

Si utilizza l'identità trigonometrica \(1 - \cos^2{\vartheta} = \sin^2{\vartheta}\) e si sostituisce l'espressione del volumetto elementare \(dV = r^{2}\sin\vartheta drd\vartheta d\varphi\):

\[
\mu_{z} = \dfrac{1}{2}\rho\omega\int_{V}{r^{2}\sin^2{\vartheta} dV}=\dfrac{1}{2}\rho\omega\int_{0}^{R}{\int_{0}^{\pi}{\int_{0}^{2\pi}{r^{2}\sin^2{\vartheta} \left(r^{2}\sin\vartheta drd\vartheta d\varphi\right)}}}
\]

Riorganizzando i termini e separando gli integrali, si ricava:

\[
\mu_{z} = \dfrac{1}{2}\rho\omega \left(\int_{0}^{2\pi} d\varphi\right) \left(\int_{0}^{\pi} \sin^3{\vartheta} d\vartheta\right) \left(\int_{0}^{R} r^{4} dr\right)
\]

L'integrale sulla coordinata azimutale è:

\[
\int_{0}^{2\pi} d\varphi = 2\pi
\]

L'integrale sul raggio, è invece dato da:

\[
\int_{0}^{R} r^{4} dr = \left[ \dfrac{r^{5}}{5} \right]_{0}^{R} = \dfrac{R^{5}}{5}
\]

L'integrale sulla coordinata polare è:

\[\int_{0}^{\pi}{\sin^{3}\vartheta d\vartheta} = \int_{0}^{\pi}{\left( \dfrac{3\sin\vartheta - \sin{3\vartheta}}{4} \right)d\vartheta} = \dfrac{1}{4}\left( 3\int_{0}^{\pi}{\sin\vartheta d\vartheta} - \int_{0}^{\pi}{\sin{3\vartheta}d\vartheta} \right)\]

\[\int_{0}^{\pi}{\sin^{3}\vartheta d\vartheta} = \int_{0}^{\pi}{\left( \dfrac{3\sin\vartheta - \sin{3\vartheta}}{4} \right)d\vartheta} = \dfrac{1}{4}\left( 3\int_{0}^{\pi}{\sin\vartheta d\vartheta} - \int_{0}^{\pi}{\sin{3\vartheta}d\vartheta} \right)\]

Dove:

\[3\int_{0}^{\pi}{\sin\vartheta d\vartheta} = 3\left\lbrack - \cos\vartheta \right\rbrack_{0}^{\pi} = 3\left( - \cos\pi + \cos 0 \right) = 3(1 + 1) = 6\]

\[\int_{0}^{\pi}{\sin{3\vartheta}d\vartheta} = - \dfrac{1}{3}\left\lbrack \cos{3\vartheta} \right\rbrack_{0}^{\pi} = - \dfrac{1}{3}\left( \cos{3\pi} - \cos 0 \right) = - \dfrac{1}{3}( - 1 - 1) = \dfrac{2}{3}\]

Nel complesso, l'integrale è dato da:

\[
\int_{0}^{\pi} \sin^3{\vartheta} d\vartheta = \dfrac{4}{3}
\]


Di conseguenza, il momento magnetico della sfera è dato da:

\[
\mu_{z} = \dfrac{1}{2}\rho\omega \left( 2\pi \right) \left( \dfrac{4}{3} \right) \left( \dfrac{R^{5}}{5} \right) = \dfrac{4\pi}{15}\rho\omega R^5
\]

La carica totale \(Q\) è data dal prodotto della densità volumetrica \(\rho\) per il volume della sfera \(V=4\pi R^{3}/3\):

\[
Q=\rho V\Leftrightarrow \rho = \dfrac{Q}{V}
\]

Sostituendo il volume della sfera, si ottiene:
\[
Q=\rho V\Leftrightarrow \rho = \dfrac{Q}{V}=\dfrac{Q}{\dfrac{4}{3}\pi R^3}=\dfrac{3Q}{4\pi R^3}
\]

Sostituendo l'espressione di \(\rho\) nell'espressione per il momento magnetico lungo l'asse di rotazione, si ottiene:

\[
\mu_{z} = \dfrac{4\pi}{15}\omega R^5 \left( \dfrac{3Q}{4\pi R^3} \right)
\]

Semplificando:

\[
\mu_{z} = \dfrac{1}{5} Q \omega R^2
\]

Dato che il momento magnetico della sfera rotante è diretto lungo l'asse di rotazione, si ha:
\[
\vec{\mu} = \dfrac{1}{5} Q \omega R^{2}\hat{\imath}_z
\]

\section{Forza su piccola spira in un campo disomogeneo}\label{forza-su-piccola-spira-in-un-campo-disomogeneo}

La forza agente su una spira elementare, percorsa da corrente, immersa in un campo magnetico generico può essere espressa come:

\[\vec{F} = \vec{\nabla}\left( \vec{\mu} \cdot \vec{B} \right)\]

Dove \(\vec{\mu}\) è il momento magnetico della spira. Per un atomo il momento magnetico è costante, dunque, può essere portato all'esterno del simbolo di gradiente, producendo l'operatore:

\[
(\vec{\mu} \cdot \vec{\nabla}) = \mu_x\dfrac{\partial}{\partial x} + \mu_y\dfrac{\partial}{\partial y} + \mu_z\dfrac{\partial}{\partial z}
\]

L'espressione per la forza si riduce a:

\[
\vec{F} = \left(\vec{\mu} \cdot \vec{\nabla}\right) \vec{B}
\]

L'equazione, scritta in forma estesa, è:

\[
\begin{pmatrix}
F_{x} \\
F_{y} \\
F_{z}
\end{pmatrix} = \begin{pmatrix}
     \mu_x\dfrac{\partial}{\partial x} + \mu_y\dfrac{\partial}{\partial y} + \mu_z\dfrac{\partial}{\partial z}
\end{pmatrix}\begin{pmatrix}
B_{x} \\
B_{y} \\
B_{z}
\end{pmatrix} = \begin{pmatrix}
 \mu_x\dfrac{\partial B_x}{\partial x} + \mu_y\dfrac{\partial B_x}{\partial y} + \mu_z\dfrac{\partial B_x}{\partial z} \\
 \mu_x\dfrac{\partial B_y}{\partial x} + \mu_y\dfrac{\partial B_y}{\partial y} + \mu_z\dfrac{\partial B_y}{\partial z} \\
 \mu_x\dfrac{\partial B_z}{\partial x} + \mu_y\dfrac{\partial B_z}{\partial y} + \mu_z\dfrac{\partial B_z}{\partial z}
\end{pmatrix}
\]

Si suppone che il campo sia diretto solamente lungo \(z\), direzione lungo cui è variabile. In altre parole, il campo è dato da:

\[\vec{B} = B_{z}(z){\hat{\imath}}_{z}\]

La forza, in questo contesto, può essere espressa come:
\[
\begin{pmatrix}
F_{x} \\
F_{y} \\
F_{z}
\end{pmatrix} = \begin{pmatrix}
      \mu_x\dfrac{\partial}{\partial x} + \mu_y\dfrac{\partial}{\partial y} + \mu_z\dfrac{\partial}{\partial z}
\end{pmatrix}\begin{pmatrix}
0 \\
0 \\
B_{z}\left(z\right)
\end{pmatrix} = \begin{pmatrix}
0 \\
0 \\
 \mu_z\dfrac{\partial B_z}{\partial z}
\end{pmatrix}
\]

La componente non nulla della forza è solamente quella lungo \(z\), data da:

\[F_{z} = \mu_{z}\dfrac{\partial B_{z}}{\partial z}\]

La forza che agisce sulla spira dipende dalla proiezione del momento magnetico \(\vec{\mu}\) lungo \(z\) e dal gradiente del campo magnetico \(\vec{B}\) lungo \(z\).

Questa relazione mostra che un dipolo magnetico immerso in un campo non uniforme subisce una forza che tende a spostarlo verso le regioni di campo più intenso se il momento magnetico è positivo, \(\mu_z>0\), o meno intenso se \(\mu_z<0\).

\subsection{Esperimento di Stern-Gerlach}\label{esperimento-di-stern-gerlach}

L'esperimento di Stern-Gerlach fornisce una prova sperimentale della quantizzazione del momento magnetico dell'atomo.

Nel modello classico l'elettrone era visto come un pianeta orbitante intorno al nucleo, dunque, caratterizzato da un momento angolare dato dalla rivoluzione dell'elettrone intorno al nucleo e intorno al proprio asse.

\begin{figure}[ht]
\centering
\includegraphics[width=3.21429in,height=2.53278in,alt={P1423\#yIS1}]{media/3_Elettromagnetismo/image24.pdf}\caption{Modello planetario dell'atomo}
\end{figure}

È noto, inoltre, che la massa del nucleo è di molti ordini di grandezza maggiore rispetto quella dell'elettrone; dunque, è possibile ritenere il nucleo fermo rispetto all'elettrone.

Il moto dell'elettrone può essere considerato circolare con raggio \(r\). L'elettrone viaggia con velocità \(\vec{v}\). Per definizione la corrente è:

\[I = \dfrac{\mathrm{\Delta}q}{\mathrm{\Delta}t}\]

Dove, la carica coincide con quella dell'elettrone, mentre \(\mathrm{\Delta}t\) coincide con il periodo di rivoluzione della particella carica:

\[I = \dfrac{\mathrm{\Delta}q}{\mathrm{\Delta}t} = \dfrac{e}{2\pi\dfrac{r}{v}} = \dfrac{ev}{2\pi r}\]

Data la piccola corrente generata, esiste un momento magnetico \(\vec{\mu}\) ortogonale al piano sul quale l'elettrone esegue la sua orbita, con verso diretto in modo da vedere la corrente ruotare in senso antiorario. Per convenzione sulla corrente, l'elettrone deve ruotare in senso orario.

Il momento magnetico è dato da:

\[\vec{\mu} = IS{\hat{\imath}}_{n}\]

Si considera il modulo, si sostituisce l'espressione della corrente prodotta dall'elettrone e la superficie della spira descritta :

\[\mu = IS = - \dfrac{ev}{2\pi r}\pi r^{2} = - \dfrac{evr}{2}\]

Dove il segno meno è dovuto alla convenzione sulle correnti.

Sull'elettrone agisce un momento angolare \(\vec{L}\), dovuto all'orbita circolare descritto dall'elettrone, anch'esso ortogonale al piano dell'orbita, dato da:

\[\vec{L} = m\vec{r} \times \vec{v}\]

Si considera il modulo del momento angolare:

\[L = mrv\]

Si moltiplicano ambo i membri per \(e/2\):

\[\dfrac{e}{2}L = \dfrac{e}{2}mrv\]

Sostituendo l'espressione del momento magnetico, si ha:

\[\dfrac{e}{2}L = m\mu\]

Ricavano il momento magnetico in funzione del momento angolare, si ha:

\[\mu = \dfrac{e}{2m}L\]

In generale, se la particella ha carica \(q\) negativa, risulta:

\[\vec{\mu} = - \dfrac{q}{2m}\vec{L}\]

Dove \(\vec{\mu}\) è opposto a \(\vec{L}\) per la convenzione sulle correnti.

Sebbene l'ultima equazione sia stata ricavata nell'ambito della fisica classica, è valida anche in meccanica quantistica, in cui il punto di vista è completamente diverso.

Sebbene l'esperimento venga oggi usato per dimostrare lo spin, al tempo l'obiettivo era dimostrare l'esistenza del momento magnetico intrinseco e la sua quantizzazione spaziale. I due fisici eseguirono l'esperimento sugli atomi di argento, emessi da una sorgente. Questi atomi venivano deflessi da un campo magnetico non omogeneo, variabile lungo \(z\). L'argento è stato scelto proprio perché il suo momento angolare totale, derivante dagli elettroni di valenza, è dovuto essenzialmente a un singolo elettrone $s$-orbitale, semplificando l'interpretazione dei risultati.

La meccanica classica prevede che momento magnetico degli atomi di argento sia distribuito statisticamente in tutte le direzioni. Sugli atomi di argento agisce una forza data da:

\[F_{i} = {\mu_{i}}_{z}\dfrac{\partial B_{z}}{\partial z}\ \]

Dunque, ogni atomo subisce una deflessione dovuta alla proiezione del suo momento magnetico lungo l'asse \(z\) e dal gradiente del campo magnetico lungo lo stesso asse.

Nella teoria classica, dato che ogni atomo possiede un momento magnetico orientato casualmente, Stern e Gerlach si aspettavano di ottenere una linea retta compresa tra un massimo e minimo. Tutte le posizioni compresi tra questi due valori hanno tutti la stessa probabilità.

Tuttavia, i due scienziati rilevarono solo due punti di arrivo. I due dedussero che gli orientamenti dei momenti magnetici degli atomi non disposti in modo casuale ma in maniera quantizzata.

Questo risultato non può essere previsto dalla meccanica classica, ma viene spiegato dalla meccanica quantistica.

\begin{figure}[ht]
\centering
\includegraphics[width=5.47196in,height=2.05093in,alt={P1456\#yIS1}]{media/3_Elettromagnetismo/image25.pdf}\caption{Esperimento di Stern-Gerlach}
\end{figure}

\subsection{Concetto di spin}\label{concetto-di-spin}

Il concetto di spin è inglobato nella meccanica quantistica e può essere visualizzato come la rotazione dell'elettrone intorno al suo asse. Lo spin determina un momento angolare e un momento magnetico. I due parametri sono antiparalleli e legati dal rapporto giromagnetico ($\gamma$).

\[\vec{\mu} = - \dfrac{q}{2m}\vec{L} = - \gamma_{e}\vec{L}\]

Il concetto di elettrone rotante non può essere considerato valido poiché, nel contesto della meccanica quantistica, l'elettrone è privo di estensione superficiale. Lo spin è una proprietà intrinseca della particella, non legata a rotazione spaziale.

In meccanica quantistica, ogni particella elementare o composta che possieda un momento angolare intrinseco (spin $\vec{S}$) o orbitale ($\vec{L}$) è associata a un momento magnetico $\vec{\mu}$. Il momento magnetico intrinseco ($\vec{\mu}_s$) di una particella è sempre proporzionale al suo spin ($\vec{S}$):

\[
\vec{\mu}_s = g \dfrac{q}{2m} \vec{S}
\]
Dove:
\begin{itemize}
    \item $q$ e $m$ sono, rispettivamente, la carica e la massa della particella (ad esempio, elettrone, protone, muone, ecc.);
    \item $\vec{S}$ è il momento angolare di spin della particella;
    \item $g$ è il fattore giromagnetico ($g$-factor) o fattore di Landé, un numero adimensionale che indica di quanto il momento magnetico devia dal valore classico ($q/2m$)
\end{itemize}

Per l'elettrone, la formula è:
\[
\vec{\mu}_{e} = - g_{e} \dfrac{e}{2m_{e}} \vec{S}
\]

Il fattore $e/2m_{e}$ è l'unità naturale del magnetismo per l'elettrone ed è chiamato magnetone di Bohr ($\mu_B$). Il fattore $g_e$ per l'elettrone libero è $g_e \approx 2.0023$ (la piccola deviazione da $g=2$ è spiegata dall'Elettrodinamica Quantistica, QED).

Per i nucleoni e nuclei, si usa la massa del protone ($m_p$) come riferimento per l'unità magnetica, poiché i momenti magnetici nucleari sono molto più piccoli:

\[
\vec{\mu}_{N} = g_{N} \dfrac{e}{2m_{p}} \vec{I}
\]
Dove:
\begin{itemize}
    \item \(e/2m_{p}\) è il magnetone nucleare ($\mu_N$);
    \item $\vec{I}$ è il momento angolare di spin nucleare;
    \item $g_N$ è il fattore $g$ nucleare, che è caratteristico di ciascun nucleo.
\end{itemize}

Per un atomo in uno stato quantico definito, il momento magnetico totale ($\vec{\mu}_{tot}$) è proporzionale al momento angolare totale ($\vec{J} = \vec{L} + \vec{S}$):

\[
\vec{\mu}_{tot} = - g \dfrac{e}{2m_{e}} \vec{J}
\]
In questo caso, il fattore $g$ è il fattore di Landé ($g$) dell'atomo, il cui valore dipende da come si accoppiano $\vec{L}$ e $\vec{S}$ (ad esempio, tramite l'accoppiamento $LS$)

In sintesi, la relazione $\vec{\mu} = g q/2m \vec{S}$ è la formula unificatrice in meccanica quantistica, con il fattore $g$ che incorpora la natura specifica della particella o del sistema considerato.

\section{Campo magnetico prodotto da un momento magnetico}\label{campo-magnetico-prodotto-da-un-momento-magnetico}

Si vuole determinare il campo creato da un momento magnetico di una piccola spira percorsa da corrente. A tale scopo si considerano le equazioni di Maxwell per il campo induzione magnetica nel vuoto:

\[\begin{cases}
\vec{\nabla} \cdot \vec{B} = 0 \\
\vec{\nabla} \times \vec{B} = \mu_{0}\left( \vec{J} + \varepsilon_{0}\dfrac{\partial\vec{E}}{\partial t} \right)
\end{cases}
\]

Se la lunghezza d'onda del campo incidente \(\lambda\) è molto maggiore della dimensione lineare dell'oggetto, è possibile ritenere il campo elettromagnetico lentamente variabile sulla superficie della spira, dunque:

\[\dfrac{\partial\vec{E}}{\partial t} \simeq 0\]

È possibile scrivere:

\[\begin{cases}
\vec{\nabla} \cdot \vec{B} = 0 \\
\vec{\nabla} \times \vec{B} = \mu_{0}\vec{J}
\end{cases}
\]

Siccome il campo induzione magnetica è solenoidale, è possibile definire un potenziale vettore, tale che:

\[\vec{B} = \vec{\nabla} \times \vec{A}\]

Si sostituisce la definizione del potenziale vettore nella seconda equazione:

\[\vec{\nabla} \times \vec{B} = \mu_{0}\vec{J} \Leftrightarrow \vec{\nabla} \times \vec{\nabla} \times \vec{A}\]

Il rotore del rotore può essere scritto come:

\[\vec{\nabla} \times \vec{\nabla} \times = \vec{\nabla}\left( \vec{\nabla} \cdot \  \right) - \nabla^{2}\ \]

Dunque, si ottiene:

\[\vec{\nabla}\left( \vec{\nabla} \cdot \vec{A} \right) - \nabla^{2}\vec{A} = \mu_{0}\vec{J}\]

Il potenziale vettore non è univocamente definito, dunque, è possibile imporre la condizione, detta gauge di Coulomb:

\[\vec{\nabla} \cdot \vec{A} = 0\]

Si ottiene che il laplaciano del campo vettore è dato da:

\[\nabla^{2}\vec{A} = - \mu_{0}\vec{J}\]

Si dimostra che la soluzione è del tipo:

\[\vec{A} = \dfrac{\mu_{0}}{4\pi}\int_{V}\dfrac{\vec{J}\left( {\vec{r}}' \right)}{\left| \vec{r} - {\vec{r}}' \right|}dV'\]

Per una piccola spira, lontano da essa, si utilizza lo sviluppo in serie di multipoli per $1/\left| \vec{r} - {\vec{r}}' \right|$ e si considera solo il termine di dipolo magnetico. Si dimostra che il potenziale vettore in un punto di osservazione $\vec{R}$ è dato da::

\[\vec{A}\left( \vec{r} \right) = \dfrac{\mu_{0}}{4\pi}\dfrac{\vec{\mu} \times \vec{R}}{R^{3}}\]

Dove \(\vec{R}\) è il vettore che congiunge il centro della piccola spira col punto di osservazione.

\begin{figure}[ht]
\centering
\includegraphics[width=2.69167in,height=2.03333in,alt={P1496\#yIS1}]{media/3_Elettromagnetismo/image26.pdf}\caption{Campo prodotto da una spira elementare}
\end{figure}

L'iterazione di una spira con un campo magnetico \(\vec{B}\) esterno è descritta dall'energia potenziale \(U\), data da:

\[U = - \vec{\mu} \cdot \vec{B} = -\mu B\cos\beta\]

Con \(\beta\) angolo formato dal campo magnetico e il momento magnetico. L'energia potenziale è nulla quando il campo magnetico è ortogonale al momento magnetico.

Il momento magnetico immerso in un campo magnetico subisce l'effetto di una coppia data da:

\[\vec{\tau} = \vec{\mu} \times \vec{B}\]

Con \(\vec{\tau}\) momento torcente.

\section{Moto del momento magnetico in campo magnetico}\label{moto-del-momento-magnetico-in-campo-magnetico}

Si considera una particella, come un elettrone, un atomo o un nucleo, immerso in un campo magnetico \(\vec{B}\). Per semplicità si analizza il sistema mediante una descrizione classica e non relativistica.

Ogni atomo o particella subatomica, immerso in un campo magnetico subisce un momento torcente \(\vec{\tau}\) dato da:

\[\vec{\tau} = \vec{\mu} \times \vec{B}\]

Il momento angolare rispetta la seconda legge di Newton:

\[\dfrac{d\vec{L}}{dt} = \vec{\tau}\]

Sostituendo l'espressione per il momento angolare in funzione del momento magnetico si ha:

\[\dfrac{d\vec{L}}{dt} = \vec{\mu} \times \vec{B}\]

Il momento magnetico è legato al momento angolare da un fattore di proporzionalità \(\gamma\):

\[\vec{\mu} = \gamma\vec{L}\]

In caso di elettroni, la costante di proporzionalità coincide con il rapporto giromagnetico:

\[\gamma = - \gamma_{e}\]

In caso di atomo con:

\[\gamma = - g\dfrac{q}{2m}\]

Mentre per un nucleo con:

\[\gamma = g\dfrac{q}{2m_{p}}\]

La seconda legge di Newton può essere scritta come:

\[\dfrac{d\vec{L}}{dt} = \vec{\mu} \times \vec{B} =  \gamma\vec{L} \times \vec{B}\]

\(\gamma\) è positivo se momento angolare e momento magnetico sono paralleli, negativo se antiparalleli.

Dall'equazione ottenuta si nota che la derivata del momento angolare deve essere perpendicolare al momento angolare stesso. Ne discende che il modulo di \(\vec{L}\) è costante. In altre parole, la punta del vettore momento angolare \(\vec{L}\) giace su una sfera e percorre una traiettoria circolare nel piano perpendicolare a \(\vec{B}\). Tale modo è detto precessione del momento magnetico o del momento angolare.

\begin{figure}[ht]
\centering
\includegraphics[width=2.30833in,height=2.58576in,alt={P1524\#yIS1}]{media/3_Elettromagnetismo/image27.pdf}\caption{Moto di precessione del momento magnetico}
\end{figure}

Confrontando l'equazione ottenuta per il momento angolare:

\[\dfrac{d\vec{L}}{dt} = \gamma\vec{L} \times \vec{B} = - \gamma\vec{B} \times \vec{L}\]

Con la relazione generale che lega la rotazione di un vettore alla velocità angolare:

\[d\vec{v} = \vec{\Omega} \times \vec{v}dt\]

è evidente che la velocità del moto di precessione è data da:

\[\vec{\Omega} = - \gamma\vec{B}\]

Con questa definizione, è possibile scrivere:

\[\dfrac{d\vec{L}}{dt} = \vec{\Omega} \times \vec{L}\]

\section{Diamagnetismo}\label{diamagnetismo}

I materiali diamagnetici sono caratterizzati da atomi che, se immersi in un campo magnetico \(\vec{B}\), sviluppano un momento magnetico aggiuntivo tale che il momento magnetico risultante si oppone al campo applicato.

Si suppone di applicare lentamente un campo magnetico a un atomo di materiale diamagnetico. Il suo nucleo, avendo una massa molto maggiore dell'elettrone può essere ritenuto fermo, mentre l'elettrone ruota interno al nucleo. Questo movimento può essere assimilato a una spira di raggio \(r\) percorsa da corrente centrata sul nucleo.

\begin{figure}[ht]
\centering
\includegraphics[width=2.05833in,height=1.91724in,alt={P1537\#yIS1}]{media/3_Elettromagnetismo/image28.pdf}\caption{Nucleo immerso in un campo magnetico}
\end{figure}

Il campo magnetico \(\vec{B}\), variabile nel tempo e nello spazio, si concatena con la spira, provocando la generazione di una forza elettromagnetica data dalla legge di Faraday:

\[\oint_{C}{\vec{E} \cdot d\vec{C}} = - \dfrac{\partial}{\partial t}\int_{S}{\vec{B} \cdot d\vec{S}}\]

Dove \(S\) è la superficie della spira e \(\partial S = C\) il suo contorno. Se il campo \(\vec{B}\) è costante sulla superficie della spira può essere portato fuori dal simbolo di integrale:

\[\vec{E} \cdot \oint_{C}{d\vec{C}} = - \dfrac{\partial}{\partial t}\vec{B} \cdot \int_{S}{d\vec{S}}\]

Ne discende che il campo elettrico è anch'esso uniforme sulla spira e parallelo, in ogni punto, a \(d\vec{C}\). Per una circonferenza risulta:

\[E2\pi r = \pi r^{2}\left( - \dfrac{\partial B}{\partial t} \right)\]

Dato che \(B\) è una funzione solo del tempo, la derivata parziale si riduce a una totale. Semplificando \(\pi r\) si ottiene:

\[E = - \dfrac{r}{2}\dfrac{dB}{dt}\]

Il campo elettrico produce un momento torcente sull'elettrone, dato dalla relazione:

\[\vec{\tau} = q\vec{r} \times \vec{E}\]

Dato che il campo elettrico è ortogonale al raggio, essendo ortogonale anche al campo magnetico \(\vec{B}\), il modulo del momento torcente, esplicitando anche la carica dell'elettrone, può essere espresso come:

\[\tau = - erE\]

Sostituendo il campo elettrico prima determinato, si ha:

\[\tau = e\dfrac{r^{2}}{2}\dfrac{dB}{dt}\]

Per la seconda legge di Newton \(dL\backslash dt = \tau\), si ottiene:

\[\dfrac{dL}{dt} = e\dfrac{r^{2}}{2}\dfrac{dB}{dt}\]

Si integra tra \(t_{0}\), tempo di applicazione del campo magnetico, e \(t_{1}\) tempo in cui il campo \(B\) raggiunge il suo valore massimo:

\[\int_{t_{0}}^{t_{1}}{\dfrac{dL}{dt}dt} = e\dfrac{r^{2}}{2}\int_{t_{0}}^{t_{1}}{\dfrac{dB}{dt}dt}\]

Risolvendo si ha:

\[L\left( t_{1} \right) - L\left( t_{0} \right) = e\dfrac{r^{2}}{2}\left\lbrack B\left( t_{1} \right) - B\left( t_{0} \right) \right\rbrack\]

Inizialmente il campo è nullo \(B\left( t_{0} \right) = 0\), per cui la differenza di momento angolare è data da:

\[\mathrm{\Delta}L = e\dfrac{r^{2}}{2}B\]

Dove \(B\) è il massimo valore del campo.

Il momento magnetico dell'elettrone intorno al nucleo è antiparallelo al momento angolare. Le due quantità sono legate dal rapporto giromagnetico:

\[\mathrm{\Delta}\mu = - \dfrac{e}{2m}\mathrm{\Delta}L\]

Sostituendo l'espressione della differenza del momento angolare, si ottiene:

\[\mathrm{\Delta}\mu = - \dfrac{e^{2}r^{2}}{4m}B\]

In linea di principio l'orbita descritta dall'elettrone dovrebbe variare per l'applicazione del campo magnetico.

\begin{figure}[ht]
\centering
\includegraphics[width=2.30915in,height=2.64167in,alt={P1568\#yIS1}]{media/3_Elettromagnetismo/image29.pdf}\caption{Elettrone in equilibrio sull'orbita}
\end{figure}

Un elettrone orbitante è mantenuto, infatti, in equilibrio dalla forza centripeta (\(F_{c}\)) e dell'iterazione coulombiana con il nucleo (\(F_{e}\)):

\[F_{e} = F_{c}\]

Sostituendo le relative espressioni nell'ipotesi che il nucleo sia composto da un solo elettrone, si ha:

\[m\dfrac{v^{2}}{r} = \dfrac{1}{4\pi\varepsilon_{0}}\dfrac{e^{2}}{r^{2}}\]

La variazione di velocità indotta dal campo elettrico dopo l'introduzione del campo \(dB\) è data dalla forza che il campo esercita:

\[F = \dfrac{dp}{dt} = m\dfrac{dv}{dt}\]

La forza può essere espressa in termini di campo elettrico. Dalla definizione di campo elettrico, per l'elettrone risulta:

\[E = \dfrac{F}{q} \Leftrightarrow F = - eE\]

Dunque, la variazione di quantità di moto può essere espressa come:

\[- eE = m\dfrac{dv}{dt}\]

Il campo elettrico, per la legge di Faraday, è legato al campo magnetico dalla relazione:

\[E = - \dfrac{r}{2}\dfrac{d\vec{B}}{dt}\]

Dunque, si ha:

\[- eE = m\dfrac{dv}{dt} \Leftrightarrow e\dfrac{r}{2}\dfrac{d\vec{B}}{dt} = m\dfrac{dv}{dt}\]

Si analizza la sola variazione di velocità subita dall'elettrone:

\[dv = \dfrac{er}{2m}dB\]

A causa della variazione di velocità, anche l'accelerazione centripeta \(\alpha\) varia. Ricorrendo allo sviluppo in serie di Taylor, trascurando gli ordini superiori al primo, si ha:

\[\alpha + d\alpha = \dfrac{v^{2}}{r} + d\left( \dfrac{v^{2}}{r} \right) = \dfrac{v^{2}}{r} + \dfrac{2vdv}{r}\  + o\left( \dfrac{1}{r^{2}}dr \right) + o\left( dv^{2} \right)\]

Dopo l'applicazione del campo magnetico, bisogna considerare anche la forza di Lorentz nel bilancio delle forze sull'elettrone. La forza di Lorentz e quella elettrostatica sono concordi, dunque, affinché l'elettrone sia in equilibrio, la velocità deve aumentare per incrementare la forza centripeta:

\[\dfrac{mv^{2}}{r} + \dfrac{2mvdv}{r} = \dfrac{1}{4\pi\varepsilon_{0}}\dfrac{e^{2}}{r^{2}} + evdB\]

Si è visto che:

\[dv = \dfrac{er}{2m}dB\]

Sostituendo nel bilancio, si ha:

\[\dfrac{mv^{2}}{r} + \dfrac{2mv}{r}\dfrac{er}{2m}dB = \dfrac{1}{4\pi\varepsilon_{0}}\dfrac{e^{2}}{r^{2}} + evdB \Leftrightarrow \dfrac{mv^{2}}{r} + vedB = \dfrac{1}{4\pi\varepsilon_{0}}\dfrac{e^{2}}{r^{2}} + evdB\]

Per cui si ottiene l'equazione dell'equilibrio prima dell'applicazione del campo esterno:

\[\dfrac{mv^{2}}{r} = \dfrac{1}{4\pi\varepsilon_{0}}\dfrac{e^{2}}{r^{2}}\]

Dopo l'applicazione del campo magnetico, l'equilibrio dell'elettrone è ottenuto mediante una sola variazione della velocità di rotazione. La variazione del raggio è influente per ordini superiori, dunque, in prima analisi, è perfettamente trascurabile. La cancellazione dei termini di ordine superiore ($\pm evdB$) dimostra che, se si ignora la variazione di raggio, la variazione di velocità indotta ($2mvdv/r$) è esattamente ciò che è richiesto dalla forza aggiuntiva di Lorentz ($evdB$) per mantenere l'equilibrio (cioè, $F_{centripeta} = F_{Lorentz} + F_{Coulomb}$).

Per i materiali diamagnetici la permeabilità relativa è circa unitaria, \(\mu_{r} \simeq 1\), con valori leggermente inferiori all'unità in modo da opporsi al campo applicato. Inoltre, essendo un fenomeno legato all'orbita elettronica, il diamagnetismo è presente in tutti i materiali.

\section{Paramagnetismo}\label{paramagnetismo}

A differenza del diamagnetismo sempre presente in ogni materiale, il paramagnetismo è presente solamente in quei materiali i cui atomi o molecole hanno un momento magnetico permanente ($\vec{\mu} \neq 0$), dovuto alla presenza di elettroni spaiati negli orbitali atomici.

Per il principio di esclusione di Pauli, gli elettroni negli orbitali atomici si dispongono con spin antiparallelo, così da cancellare gli effetti degli spin stessi. Negli atomi in cui sono presenti elettroni spaiati, ovvero con orbitali atomici non completamente riempiti, la cancellazione degli spin non avviene, dunque, lo spin netto dell'atomo è diverso da zero. 

A livello macroscopico, gli spin dei vari costituenti del materiale si sommano, dando luogo alla magnetizzazione macroscopica. Quest'ultima è influenzata dalla presenza o meno di un campo magnetico.

In assenza di un campo magnetico esterno, il momento magnetico di ogni atomo è orientato in modo casuale a causa dell'agitazione termica, dunque, la magnetizzazione macroscopica media ($\vec{M}$) del materiale è nulla.

Applicato un campo magnetico esterno, invece, i singoli momenti magnetici (spin) subiscono un momento torcente che tende ad allinearli nella direzione del campo. Essi iniziano a precedere attorno alla direzione del campo con la frequenza di Larmor:

\[\omega = \gamma B\]

\begin{figure}[ht]
\centering
\includegraphics[width=3.81482in,height=2.1701in,alt={P1604\#yIS1}]{media/3_Elettromagnetismo/image30.pdf}\caption{Spin nei materiali paramagnetici}
\end{figure}

Il paramagnetismo è presente solo in alcune sostanze ed è caratterizzato da una permeabilità magnetica relativa \(\mu_{r} \simeq 1\), lievemente maggiore dell'unità.

Si osservi che il vettore di magnetizzazione ($M$) è definito come la somma vettoriale dei momenti magnetici per unità di volume:

\[
\vec{M} = \dfrac{1}{V}\sum_{i}\vec{\mu}_i
\]

L'interazione del campo esterno con i momenti magnetici determina la magnetizzazione macroscopica $M$ del materiale, che in equilibrio termico è generalmente debole a causa del disorientamento termico.