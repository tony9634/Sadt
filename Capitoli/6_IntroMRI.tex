\section{Introduzione alla risonanza magnetica}\label{introduzione-alla-risonanza-magnetica}

\subsection{Diverse metodiche di imaging}\label{diverse-metodiche-di-imaging}

Dal punto di vista estetico, i macchinari risonanza magnetica, tomografia computerizzata (TC) e PET presentano una forma a spirale. Inoltre, tutte queste metodiche di imaging prevedono un gantry, dove è presente la circuiteria necessaria per la produzione dello stimolo da fornire al paziente e la sua ricezione. Si osservi che nella PET, il segnale di stimolo è generato all'interno del paziente stesso, mediante l'iniezione di un mezzo di contrasto.

\begin{figure}
\centering
\includegraphics[width=4.35482in,height=2.05952in,alt={P2729\#yIS1}]{media/6_IntroMRI/image57.pdf}\caption{Figura .: Tipica struttura di una strumentazione di imaging}
\end{figure}

Il paziente viene posto su un tavolo in moto longitudinale, detto tavolo porta-paziente. Sebbene le applicazioni, dal punto di vista esterno, possano sembrare uguali, esse si basano su principi fisici diversi:

\begin{itemize}
\item
  La TC si basa sull'assorbimento, da parte del corpo umano, della radiazione a raggi X. L'assorbimento di questa radiazione è, essenzialmente, legato alla densità del tessuto biologico attraversato. Così, l'osso assorbe una percentuale molto maggiore dei tessuti moli. In particolare, quest'ultimi assorbono una percentuale di raggi X simile tra loro. Mediante TC si ottengono delle immagini con un alto contenuto di informazioni morfologiche dei tessuti;
\item
  La risonanza magnetica si basa sulla risonanza dei nuclei di idrogeno, contenuti nel corpo umano, per effetto di un campo magnetico esterno. La maggior parte dell'idrogeno nel corpo umano è contenuto negli atomi di acqua (\(H_{2}O\)), dunque, è possibile discriminare i tessuti biologici in base al suo contenuto acquoso. Con questa metodica, l'osso, essendo un tessuto duro, contiene una quantità di acqua molto più bassa dei tessuti molli, dunque, è poco visibile;
\item
  La PET si basa sull'emissione di fotoni \(\gamma\) da parte di un radiofarmaco iniettato nel paziente. La distribuzione di radiofarmaco fornisce informazione sul comportamento metabolico dei tessuti. Dunque, mentre la TC fornisce immagini con informazioni morfologiche, la PET fornisce immagini funzionali poiché, appunto, indicative delle attività metaboliche di un tessuto.
\end{itemize}

La strumentazione moderna prevede una combinazione della CT e della PET, ottenendo la PETT-CT, in modo da ottenere immagini con un alto contenuto informativo sia morfologico che funzionale.

La risonanza magnetica può fornire immagini sia morfologiche che funzioni. Queste ultime presentano la maggior applicazione nello studio dell'encefalo: in base alle aree attivate da un determinato stimolo, si genera una mappa a colori, indicante il funzionamento della corteccia cerebrale.

La MRI (\emph{magnetic resonance imaging}) funzionale è utilizzata per la diagnosi di malattie neurodegenerative. Questo tipo di studio non può essere eseguito mediante CT poiché la scatola cranica assorbe la maggior parte della radiazione X incidente. Ne risulta, dunque, un'immagine dell'encefalo poco definita.

La CT è essenzialmente utilizzata per ottenere informazioni di natura morfologica sugli organi interni in breve tempo. Le immagini CT presentano una buona risoluzione nel discriminare le ossa; tuttavia, la risoluzione peggiora con i tessuti molli.

Utilizzando una radiazione X, la CT possiede una pericolosità intrinseca, dovuta ai possibili effetti cancerogeni della radiazione ionizzante. I danni biologici si verificano con un certo andamento statistico, quindi, sono state prodotte delle specifiche normative che regolano la dose assorbita sia dal paziente che dal tecnico radiologo.

La tomografia computerizzata è la metodica di imaging più veloce poiché, mediante le attuali tecnologie, permette di ottenere delle scansioni \emph{total body} in pochi secondi. Per tale motivo la CT è la metodica di analisi più utilizzata per eseguire la diagnosi di una patologia in tempo breve, seppur la risoluzione grossolana per i tessuti molli.

La risonanza magnetica, a differenza delle altre metodiche, non utilizza radiazioni ionizzanti ma campi elettromagnetici statici, di grande intensità, e a radiofrequenza, dell'ordine dei \(mT\), Ai campi magnetici non sono associati particolari effetti nocivi, tuttavia, la loro intensità in risonanza magnetica è normata da norme nazionali e internazionali.

L'utilizzo del campo elettromagnetico consente di eseguire l'esame di risonanza magnetica ripetutamente. Alla risonanza magnetica, inoltre, è associata una bassa invasività e una grande flessibilità, poiché permette di ottenere sia immagini morfologiche che funzionali, ovvero, è possibile visualizzare un tessuto biologico in base alla sua composizione biochimica.

A differenza della CT, tuttavia, l'esame con risonanza magnetica richiede un lungo tempo di esecuzione, che si aggira intorno ai \(30\) minuti fino a un'ora. Ciò provoca anche un senso di disagio per il paziente, il quale deve restare immobile per un lungo periodo di tempo. Tuttavia, a causa dei normali movimento del paziente, si producono artefatti da movimento durante l'esecuzione dell'esame radiologico. Infine, la strumentazione usata per la risonanza magnetica è più complessa di quella della CT, poiché deve produrre campi statici e a radiofrequenza con determinate caratteristiche. Ne discende che la risonanza magnetica è più costosa della CT.

La PET condivide con la risonanza magnetica i lunghi tempo di analisi, infatti, al fine di ottenere delle immagini \emph{total body} sono richiesti di \(45\) minuti a un'ora. Storicamente, questa metodologia diagnostica nasce per eseguire lo studio metabolico del cervello. Attualmente, l'applicazione più frequente dalla PET è lo studio del comportamento metabolico dei tumori, al fine di evidenziare il suo stadio e la presenza di metastasi. Infatti, in presenza di diagnosi di tumore, è necessario eseguire la PET almeno una volta all'anno, al fine di rilevare precocemente la presenza di nuove metastasi, formatesi per problemi legati alla recidiva tumorale.

La PET si basa sull'emissione di radiazione \(\gamma\) da parte di radionuclidi eccitati. La diversa distribuzione del radio-metabolita permette di ottenere immagini indicanti la funzionalità degli organi interessati.

L'unione della PET con la CT permette di ottenere immagini con informazioni morfologiche e funzionali. La sola informazione funzionale, generalmente, risulta essere di difficile compressione poiché in questo tipo di immagini non sono evidenziate le strutture anatomiche che emettono quella data quantità di tracciante.

Lo svantaggio principale associato alla PET è la presenza di radiazione \(\gamma\) emessa dai radionuclidi. Il paziente, dunque, è una fonte di radiazioni \(\gamma\) finché il radiofarmaco non ha esaurito la sua radioattività, generalmente entro le \(24\) ore. In questo intervallo temporale, il paziente non può entrare in contatto con donne incinte e bambini.

\subsection{Storia della risonanza magnetica}\label{storia-della-risonanza-magnetica}

Con l'esperimento di Stern e Gerlach si dimostrò che il momento magnetico atomico è quantizzato, ovvero può assumere solamente due valori a cui corrispondono due determinati livelli energetici.

Negli anni '30 il fisico Felix Bloch sviluppò delle equazioni fenomenologiche, basate su una descrizione intermedia tra fisica classica e meccanica quantistica, per descrivere il comportamento degli spin immersi in un campo magnetico.

Grazie agli studi di Bloch, negli anni '70 il chimico Paul Christian Lauterbur mise appunto una tecnica per ottenere immagini di sezioni di un corpo mediante campi elettromagnetici a radiofrequenza. La prima risonanza magnetica fu eseguita su un limone. Da questa prima applicazione sono stati sviluppato dei macchinari commerciali, attualmente utilizzati in molti ambiti della diagnostica medica. A Lauterbur fu attribuita l'idea secondo la quale i gradienti del campo magnetico consentono l'individuazione dell'origine delle onde radio emesse dai nuclei dell'oggetto in esame, ottenendo, così, immagini bidimensionali.

\subsection{Introduzione al principio di risonanza magnetica}\label{introduzione-al-principio-di-risonanza-magnetica}

Il principio fisico su cui si basa la risonanza magnetica è descritto dalle equazioni di Bloch: un gran numero di protoni di idrogeno, immersi in un campo magnetico \(\overset{\underline{}}{B}\), producono una magnetizzazione netta \(\overset{\underline{}}{M}\), misurabile, a temperatura ambiente e all'equilibrio termodinamico, mediante la legge di Curie:

\[M \simeq \frac{N}{V}\frac{\gamma^{2}\hslash^{2}}{4k_{B}T}B_{0}\]

Dove \(M\) è il momento magnetico per unità di volume, il quale contiene \(N\) particelle con spin. La grandezza \(N/V\) è detta densità protonica ed è indicata con \(\rho\).

Dalla legge i Curie si evince che la magnetizzazione dipende dal campo magnetico applicato, dalla temperatura, dal rapporto giromagnetico e dal numero di spin presenti nel volume elementare sotto analisi. Ne discende che per aumentare il valore della magnetizzazione è possibile:

\begin{itemize}
\item
  Incrementare il valore del campo magnetico utilizzato. Tipicamente il valore utilizzato del campo magnetico è \(1.5\ T\). Questo valore è regolato da norme internazionali;
\item
  La temperatura non può essere ridotta a piacere poiché non è possibile raffreddare eccessivamente il paziente. Nella sala della risonanza magnetica la temperatura è mantenuta costante, intorno ai \(25\ {^\circ}C\), al fine di evitare fluttuazioni della magnetizzazione;
\item
  È possibile scegliere la sostanza di cui si vuole calcolare la magnetizzazione, selezionando quella con un rapporto giromagnetico maggiore. In linea di principio è possibile determinare immagini di risonanza magnetica anche degli elettroni, che presentano un rapporto giromagnetico \(\gamma_{e} = - 1.76 \cdot 10^{11}\ rad/Ts\), mentre quello del protone è \(\gamma_{p} = 2.68 \cdot 10^{8}\ rad/Ts\). Il rapporto di \(\gamma_{e}\) e \(\gamma_{p}\) è:
\end{itemize}

\[\gamma = \frac{\left| \gamma_{e} \right|}{\gamma_{p}} = \frac{1.76 \cdot 10^{11}\ \frac{rad}{Ts}}{2.68 \cdot 10^{8}\ \frac{rad}{Ts}} = 658\]

Il rapporto giromagnetico dell'elettrone è molto maggiore di quel del protone, ovvero del nucleo di idrogeno, quindi, il vettore magnetizzazione degli elettroni, all'equilibrio termodinamico, ha intensità maggiore rispetto a quello dei nuclei di idrogeno. Si osservi, tuttavia, che il fattore giromagnetico è presente nell'espressione della frequenza di precessione di Larmor, ovvero la frequenza con cui gli spin ruotano intorno all'asse individuato dal campo magnetico:

\[\omega_{0} = \gamma B_{0}\]

La frequenza del campo magnetico applicato aumenta al crescere del rapporto giromagnetico, in particolare, con un campo di \(1.5\ T\), per un elettrone, si ha:

\[f_{0} = 2\pi\omega_{0} = 2\pi 1.76 \cdot 10^{11}\ \frac{rad}{Ts}1.5\ T = 42\ GHz\]

Il campo irradiato dall'elettrone è, dunque, dell'ordine della decina di \(GHz\). Ciò determina una maggiore energia associata all'onda, che si deposita nei tessuti biologici. In generale, più l'onda si avvicina allo spettro dei raggi X, maggiore è il loro contenuto energetico e maggiori sono i possibili effetti biologici. Per tale motivo le radiofrequenze adoperate sono ottimizzate per il protone.

\begin{itemize}
\item
  Si potrebbe pensare di utilizzare un atomo o un composto che risuoni alle radiofrequenze ma con un fattore giromagnetico più alto; tuttavia, l'idrogeno ha una concentrare di \(88\ M = 88\ mol/V\), molto maggiore degli altri composti che presentano una concentrazione molare dell'ordine dei \(\mu M\) o \(mM\). L'uso della risonanza dei protoni consente di ottenere il giusto compromesso tra energia depositata nel paziente, dunque effetto biologico, e un elevato numero di spin per unità di volume, ottimizzando, di conseguenza, il valore della magnetizzazione all'equilibrio;
\item
  L'unico parametro che può essere modificato per aumentare la magnetizzazione è il campo statico esterno applicato, detto principale.
\end{itemize}

Il vettore di magnetizzazione è valutato su un volumetto elementare contenente un numero di Avogadro \(N_{A}\) di particelle. Ne discende che in risonanza magnetica il paziente può essere considerato come un insieme di volumetti elementari, ognuno dei quali possiede il proprio vettore di magnetizzazione \(d\overset{\underline{}}{M}\). La ricostruzione del momento magnetico permettere di eseguire l'imaging del corpo umano. Si osservi che non tutti i volumetti considerati possiedono lo stesso numero di particelle. In media, è possibile ritenere che il numero delle particelle sia pressocché lo stesso se si considerano volumetti elementari con dimensione lineare di \(1\ mm\).

Giunti all'equilibrio termodinamico, il vettore magnetizzazione non è direttamente misurabile, poiché non produce alcun segnale variabile nel tempo da captare con apposite antenne. Per ottenere un'immagine tomografica è necessario perturbare l'equilibrio termodinamico e registrare il segnale emesso dal corpo del paziente durante il ritorno all'equilibrio dei vettori di magnetizzazione di ogni singolo volumetto elementare in cui è scomponibile il paziente.

I protoni, ovvero i nuclei degli atomi di idrogeno, non sono presenti solamente nell'acqua ma sono legati anche ad altre molecole biologiche come proteine, acidi nuclei e lipidi. I nuclei di idrogeno contenuti in queste molecole non sono soggetti allo stesso campo magnetico principale imposto dall'esterno. Ciò è dovuto all'effetto di schermatura prodotto dalla molecola. Di conseguenza, gli spin di questi nuclei compieranno delle oscillazioni a frequenza diversa da quelle degli altri atomi di idrogeno. Eccitando opportunamente un tessuto, è possibile discriminare i suoi vari costituenti sulla base delle caratteristiche biochimiche.

Dalla meccanica quantistica è noto che la transizione tra due stati \(\left| \psi \right\rangle\) e \(\left| \varphi \right\rangle\) possiede un andamento nel piano \(xy\) dato da:

\[\gamma\hslash\cos\left( \beta - \omega_{0}t \right)\]

Dove \(\omega_{0} = \gamma B_{0}\) è detta pulsazione di Larmor.

Le previsioni della meccanica quantistica possono essere descritte in maniera più semplice considerando gli spin orientati in modo casuale. Quando si applica un campo magnetico, gli spin si orientano nella direzione del campo, mentre nel piano \(xy\), traverso all'asse del campo, si instaura un moto di precessione con pulsazione angolare \(\omega_{0} = \gamma B_{0}\). In altre parole, gli spin ruotano intorno all'asse \(z\), individuato dal campo magnetico, in senso orario con frequenza \(2\pi\omega_{0}\).

\begin{figure}
\centering
\includegraphics[width=2.4729in,height=1.38542in,alt={P2775\#yIS1}]{media/6_IntroMRI/image58.pdf}
\caption{Figura .: Orientamento degli spin a causa del campo}
\end{figure}

Questa assunzione, sebbene non sia esatta, consente di descrivere in modo semplice il comportamento degli spin immersi in un campo magnetico, ottenendo gli stessi risultati della meccanica quantistica.

\begin{figure}
\centering
\includegraphics[width=2.60417in,height=2.14214in,alt={P2778\#yIS1}]{media/6_IntroMRI/image59.pdf}\caption{Figura .: Moto di precessione}
\end{figure}

Si suppone di applicare un campo magnetico nella direzione \(z\), convenzionalmente coincidente con l'asse maggiore del tavolo porta-paziente. Si suppone di applicare uno stimolo \(B_{1}\) tale da ruotare il vettore di magnetizzazione all'equilibrio sull'asse \(y\).

\begin{figure}
\centering
\includegraphics[width=6.125in,height=2.40972in,alt={P2781\#yIS1}]{media/6_IntroMRI/image60.pdf}\caption{Figura .: Rotazione del vettore magnetizzazione a opera di uno stimolo esterno}
\end{figure}

Il vettore magnetizzazione globale, somma di tanti momenti magnetici intrinseci, presenta un andamento più complesso di un normale vettore; infatti, le componenti trasversali evolvono con una tempistica diversa dalle componenti longitudinali.

Rimosso lo stimolo, il vettore magnetizzazione torna all'equilibrio termodinamico, emettendo un segnale dato da:

\[s = - \frac{d}{dt}\int_{V}^{}{\overset{\underline{}}{M} \cdot {\overset{\underline{}}{B}}_{RF}dV}\]

Dove \({\overset{\underline{}}{B}}_{RF}\) è il campo che sarebbe erogato dall'antenna ricevente se percorsa da una certa corrente. Trascurando questo termine, il segnale registrato è proporzionale alla devirata del vettore magnetizzazione:

\[s \propto \frac{dM}{dt}\]

Generalmente l'eccitamento è di tipo sinusoidale, dunque:

\[\frac{dM}{dt} = \omega_{0}M = \gamma B_{0}M\]

Per la legge di Curie, scritta in termini di densità protonica:

\[M \simeq \frac{N}{V}\frac{\gamma^{2}\hslash^{2}}{4k_{B}T}B_{0} = \rho\frac{\gamma^{2}\hslash^{2}}{4k_{B}T}B_{0}\]

Il segnale registrato è proporzionale a:

\[s \propto \ \gamma B_{0}M = \rho\frac{\gamma^{3}\hslash^{2}}{4k_{B}T}B_{0}^{2}\]

Tralasciando i termini costante:

\[s \propto \ \rho\frac{\gamma^{3}}{T}B_{0}^{2}\]

\(\rho\) rappresenta il numero di protoni di idrogeno presenti nel volumetto considerato. Dato che il segnale dipende anche da \(B_{0}^{2}\), è fondamentale che il campo \(B_{0}\) sia molto elevato, in modo da ottenere un buon rapporto segnale-rumore o \emph{signal-to-noise ratio} (SNR).

\subsection{Risonanza magnetica come tecnica spettroscopica}\label{risonanza-magnetica-come-tecnica-spettroscopica}

La risonanza magnetica nasce negli anni '50-'60 con applicazioni spettroscopiche. Questa metodica, ancora oggi molto utilizzata, permette di valutare la composizione chimica del materiale irraggiato dal campo magnetico. In campo medico, la spettroscopia è utilizzata per valutare lo stato metabolico di un tessuto.

A causa dell'effetto della schermatura delle macromolecole, i nuclei di idrogeno non appartenenti all'acqua subiscono un campo magnetico diverso da quello esterno. Si assistono, dunque, a più moti di precessione con frequenza di Larmor diverse.

\begin{figure}
\centering
\includegraphics[width=3.67692in,height=3.00926in,alt={P2800\#yIS1}]{media/6_IntroMRI/image61.pdf}\caption{Figura .: Moti di precessione con frequenze diverse}
\end{figure}

L'applicazione dello stimolo non ruoterà tutti i momenti magnetici allo stesso modo. Dunque, il ritorno all'equilibrio produce dei campi magnetici variabili nel tempo, i quali, a loro volta, inducono sulle antenne riceventi delle fem. con differenti contenuti frequenziali.

Il segnale prelevato è proporzionale alla somma dei veri momenti magnetici che procedono introno all'asse \(z\), con diversa frequenza di Larmor, poiché è diverso il campo principale percepito:

\[s \propto \sum_{k}^{}{M(k)\omega(k)\exp\left( - j\omega_{0}kt \right)}\]

Si ha, ovvero, una somma di \(N\) segnali con proprie frequenze, in base alla molecola a cui il nucleo di idrogeno è legato.

Ogni tessuto biologico è caratterizzato da una certa composizione chimica, quindi, determinando la contrazione dei costituenti, come le proteine, è possibile valutare lo stato di salute di un tessuto e la sua attività metabolica.

Nei tessuti, tuttavia, sono presenti un gran numero di molecole, oltre alle proteine o sostanze di interesse come acqua, acidi nucleici ed ecc. Per cui, al fine di aumentare il numero di molecole di cui si vuole effettuare l'imaging è necessario aumentare le dimensioni del volumetto elementare \(dV\).

Nel caso delle immagini funzionali, di conseguenza, l'aumento del SNR comporta un peggioramento della risoluzione spaziale. In questo modo, è possibile distinguere il contenuto metabolico del tessuto dalla restante parte di acqua e altri costituenti.

Per ottenere le immagini, infine, oltre al campo magnetico costante di grande intensità, si applica un campo stazionario variabile lungo una direzione \(x\) o \(y\). Si instaura così un gradiente di campo magnetico che determina la variazione, per ogni punto \(\overset{\underline{}}{r}\) del paziente, la frequenza di precessione con cui si muovono gli spin varia con la posizione:

\[\omega\left( \overset{\underline{}}{r} \right) = \gamma\overset{\underline{}}{B}\left( \overset{\underline{}}{r} \right)\]

Se il gradiente è posizionato lungo \(z\), il campo magnetico è del tipo:

\[B = B_{0} + G_{z}z\]

Con questa soluzione, la frequenza di precessione è data da:

\[f(z) = \frac{\omega(z)}{2\pi} = \frac{\gamma}{2\pi}\left( B_{0} + G_{z}z \right)\]

Si definisce \(\overline{\gamma} = \gamma/2\pi\). Questa quantità, per il nucleo di idrogeno è:

\[\overline{\gamma} = \frac{\gamma}{2\pi} = \frac{2.68 \cdot 10^{8}\ \frac{rad}{Ts}}{2\pi} = 42.6\frac{MHz}{T}\]

Con un campo statico principale di \(1.5\ T\) si ha una frequenza di precessione data da:

\[f_{0} = \overline{\gamma}B_{0} = 42.6\frac{MHz}{T}1.5\ T \simeq 64\ MHz\]

Per un campo di \(3\ T\), risulta invece:

\[f_{0} = \overline{\gamma}B_{0} = 42.6\frac{MHz}{T}3\ T \simeq 128\ MHz\]

Queste frequenze rientrano nello spettro delle onde radio, in particolare, nella banda di frequenze normalmente utilizzate nella trasmissione FM.

L'uso dei gradienti i campo permette di variare la frequenza di precessione con la posizione. In questo modo, è possibile selezionare una sola fetta del corpo del paziente di cui si vuole eseguire l'imaging. Nello specifico, una volta raggiunto l'equilibrio termodinamico, si applica uno stimolo che ribalta il vettore di magnetizzazione che procede a una determinata frequenza.

\subsection{Momento di precessione}\label{momento-di-precessione}

Si considera un singolo spin immerso in un campo magnetico diretto lungo l'asse \(z\). All'equilibrio termico questo spin si allinea lungo l'asse individuato dal campo magnetico.

Dal punto di vista classico, lo spin immerso nel campo magnetico subisce una torsione meccanica, data da:

\[\overset{\underline{}}{\tau} = \overset{\underline{}}{\mu} \times \overset{\underline{}}{B}\]

Dove:

\[\overset{\underline{}}{\tau} = \frac{d\overset{\underline{}}{L}}{dt}\]

\(\overset{\underline{}}{L}\) è il momento angolare, legato al momento magnetico \(\overset{\underline{}}{\mu}\) dal fattore giromagnetico \(\gamma\):

\[\overset{\underline{}}{\mu} = \gamma\overset{\underline{}}{L} \Leftrightarrow \overset{\underline{}}{L} = \frac{1}{\gamma}\overset{\underline{}}{\mu}\]

Sostituendo i due risultati nell'equazione differenziale si ha:

\[\overset{\underline{}}{\mu} \times \overset{\underline{}}{B} = \frac{d}{dt}\left( \frac{1}{\gamma}\overset{\underline{}}{\mu} \right) \Leftrightarrow \frac{d\overset{\underline{}}{\mu}}{dt} = \gamma\overset{\underline{}}{\mu} \times \overset{\underline{}}{B}\]

La descrizione classica, rappresentata dall'equazione differenziale appena individuata, non è precisa, tuttavia, presenta gli stessi risultati della meccanica quantistica. Quest'ultima teoria si basa sui livelli energetici \(\left| + \right\rangle\) e \(\left| - \right\rangle\). Si dimostra che la proiezione dello spin lungo l'asse \(x\) è del tipo:

\[\mu_{x} \propto \cos\left( \omega_{0}t \right)\]

Dove \(\omega_{0} = \gamma B_{0}\) è la frequenza di precessione di Larmor.

L'equazione differenziale:

\[\frac{d\overset{\underline{}}{\mu}}{dt} = \gamma\overset{\underline{}}{\mu} \times \overset{\underline{}}{B}\]

Presenta la stessa soluzione della meccanica quantistica ma con una descrizione semplificata; per tale motivo, si ricorre alla descrizione classica.

Si considera il prodotto scalare tra \(\overset{\underline{}}{\mu}\) e la sua derivata:

\[\overset{\underline{}}{\mu} \cdot \frac{d\overset{\underline{}}{\mu}}{dt} = \gamma\overset{\underline{}}{\mu} \cdot \left( \overset{\underline{}}{\mu} \times \overset{\underline{}}{B} \right)\]

Il vettore \(\overset{\underline{}}{\mu} \times \overset{\underline{}}{B}\) è ortogonale sia al vettore \(\overset{\underline{}}{\mu}\) che \(\overset{\underline{}}{B}\), dunque, il prodotto scalare è nullo:

\[\overset{\underline{}}{\mu} \cdot \frac{d\overset{\underline{}}{\mu}}{dt} = \gamma\overset{\underline{}}{\mu} \cdot \left( \overset{\underline{}}{\mu} \times \overset{\underline{}}{B} \right) = 0\]

Il prodotto scalare tra \(\overset{\underline{}}{\mu}\) e la sua derivata può essere scritto come:

\[\overset{\underline{}}{\mu} \cdot \frac{d\overset{\underline{}}{\mu}}{dt} = \frac{1}{2}\frac{d}{dt}\left( \overset{\underline{}}{\mu} \cdot \overset{\underline{}}{\mu} \right) = \frac{1}{2}\frac{d}{dt}\left| \overset{\underline{}}{\mu} \right|^{2} = 0\]

La derivata del modulo quadro è nulla, dunque, il modulo del momento magnetico intrinseco è costante nel tempo:

\[\left| \overset{\underline{}}{\mu} \right| = const\]

Nonostante il suo modulo sia costante, la fase del momento magnetico decresce costantemente nel tempo, infatti, risulta:

\[\frac{d\varphi}{dt} = - \omega\]

Il momento magnetico precede nel piano \(xy\) in senso orario. La fase può essere scritta come:

\[\varphi(t) = \varphi_{0} - \omega t\]

Dove \(\varphi_{0}\) è la fase iniziale.

\begin{figure}
\centering
\includegraphics[width=2.36664in,height=1.64815in,alt={P2852\#yIS1}]{media/6_IntroMRI/image62.pdf}\caption{Figura .: Verso di rotazione del moto di precessione}
\end{figure}

Si risolve l'equazione differenziale per il momento magnetico. A tale scopo si scompone il prodotto vettoriale \(\overset{\underline{}}{\mu} \times \overset{\underline{}}{B}\) lungo gli assi:

\[\overset{\underline{}}{\mu} \times \overset{\underline{}}{B} = \left| \begin{matrix}
{\widehat{i}}_{x} & {\widehat{i}}_{y} & {\widehat{i}}_{z} \\
\mu_{x} & \mu_{y} & \mu_{z} \\
0 & 0 & B_{0}
\end{matrix} \right| = B_{0}\left( \mu_{y}{\widehat{i}}_{x} - \mu_{x}{\widehat{i}}_{y} \right)\]

L'equazione differenziale, si scrive come:

\[\frac{d\overset{\underline{}}{\mu}}{dt} = \gamma B_{0}\left( \mu_{y}{\widehat{i}}_{x} - \mu_{x}{\widehat{i}}_{y} \right)\]

Scomponendo lungo gli assi, si ha:

\[\left\{ \begin{matrix}
\frac{d\mu_{x}}{dt} = \gamma B_{0}\mu_{y} \\
\frac{d\mu_{y}}{dt} = - \gamma B_{0}\mu_{x} \\
\frac{d\mu_{z}}{dt} = 0
\end{matrix} \right.\ \]

Si deriva la seconda equazione rispetto al tempo:

\[\frac{d^{2}\mu_{y}}{dt^{2}} = - \gamma B_{0}\frac{d\mu_{x}}{dt}\]

Sostituendo la prima equazione, si ha:

\[\frac{d^{2}\mu_{y}}{dt^{2}} = - \gamma^{2}B_{0}^{2}\mu_{y} \Leftrightarrow \frac{d^{2}\mu_{y}}{dt^{2}} + \gamma^{2}B_{0}^{2}\mu_{y} = 0\]

Passando al polinomio associato si ha:

\[\lambda^{2} + \gamma^{2}B_{0}^{2} = 0 \Leftrightarrow \lambda = \pm j\gamma B_{0}\]

Ponendo, \(\omega_{0} = \gamma B_{0}\), la soluzione \(\mu_{y}\) è del tipo:

\[\mu_{x}(t) = A\cos\left( \omega_{0}t \right) + B\sin\left( \omega_{0}t \right)\]

Dove \(A\) e \(B\) sono due costanti di integrazione, ottenute applicando le condizioni iniziali.

Nota l'espressione per \(\mu_{x}(t)\), è possibile ottenere quella di \(\mu_{y}(t)\); dalla prima equazione differenziale, infatti, risulta:

\[\frac{d\mu_{x}}{dt} = \gamma B_{0}\mu_{y} \Leftrightarrow \mu_{y} = \frac{1}{\omega_{0}}\frac{d\mu_{x}}{dt} = \frac{1}{\omega_{0}}\frac{d}{dt}\left\lbrack A\cos\left( \omega_{0}t \right) + B\sin\left( \omega_{0}t \right) \right\rbrack\]

Svolgendo la derivata si ha:

\[\mu_{y}(t) = - A\cos\left( \omega_{0}t \right) + B\sin\left( \omega_{0}t \right)\]

Lungo \(z\), la derivata del momento magnetico è nulla, dunque, \(\mu_{z}\) è costante. Si suppone di conoscere lo stato iniziale del momento:

\[\left\{ \begin{matrix}
\left. \ \mu_{x}(t) \right|_{t = 0} = \mu_{x}(0) \\
\left. \ \mu_{y}(t) \right|_{t = 0} = \mu_{y}(0) \\
\left. \ \mu_{z}(t) \right|_{t = 0} = \mu_{z}(0)
\end{matrix} \right.\ \]

Al fine di ricavare le due costanti di integrazione \(A\) e \(B\), si usano le prime due equazioni:

\[\left\{ \begin{matrix}
\left\lbrack A\cos\left( \omega_{0}t \right) + B\sin\left( \omega_{0}t \right) \right\rbrack_{t = 0} = \mu_{x}(0) \\
\left\lbrack - A\cos\left( \omega_{0}t \right) + B\cos\left( \omega_{0}t \right) \right\rbrack_{t = 0} = \mu_{y}(0)
\end{matrix} \right.\  \Leftrightarrow \left\{ \begin{matrix}
A = \mu_{x}(0) \\
B = \mu_{y}(0)
\end{matrix} \right.\ \]

La soluzione dell'equazione differenziale è, dunque:

\[\left\{ \begin{matrix}
\mu_{x}(t) = \mu_{x}(0)\cos\left( \omega_{0}t \right) + \mu_{y}(0)\sin\left( \omega_{0}t \right) \\
\mu_{y}(t) = - \mu_{x}(0)\cos\left( \omega_{0}t \right) + \mu_{y}(0)\sin\left( \omega_{0}t \right) \\
\mu_{z}(t) = \mu_{z}(0)
\end{matrix} \right.\ \]

È possibile scrivere la soluzione dell'equazione differenziale:

\[\frac{d\overset{\underline{}}{\mu}}{dt} = \gamma\overset{\underline{}}{\mu} \times \overset{\underline{}}{B}\]

in forma compatta introducendo la matrice di rotazione intorno all'asse \(z\):

\[{\overset{\underline{}}{\overset{\underline{}}{R}}}_{z}\left( \omega_{0}t \right) = \begin{pmatrix}
\cos\left( \omega_{0}t \right) & \sin\left( \omega_{0}t \right) & 0 \\
 - \sin\left( \omega_{0}t \right) & \cos\left( \omega_{0}t \right) & 0 \\
0 & 0 & 1
\end{pmatrix}\]

Con questa posizione, il momento magnetico \(\overset{\underline{}}{\mu}\) in funzione del tempo è dato da:

\[\overset{\underline{}}{\mu}(t) = {\overset{\underline{}}{\overset{\underline{}}{R}}}_{z}\left( \omega_{0}t \right)\overset{\underline{}}{\mu}(0)\]

\subsubsection{Rappresentazione complessa}\label{rappresentazione-complessa}

Dato che gli spin dei protoni eseguono un moto di precessione intorno all'asse \(z\) individuato dal campo magnetico, è utile introdurre una notazione utilizzante il piano complesso, in modo da considerare anche il movimento degli spin nel piano trasverso al campo magnetico. Si definisce la grandezza fasoriale \(\mu_{\bot}(t)\) come:

\[\mu_{\bot}(t) = \mu_{x}(t) + j\mu_{y}(t)\]

Il fasore permette di descrivere il movimento nel piano trasverso del momento magnetico nel tempo.

La derivata temporale di un fasore si scrive come:

\[\frac{d\mu_{\bot}}{dt} = - j\omega_{o}\mu_{\bot}\]

Dove \(\omega_{o} = \gamma B_{0}\) è la frequenza di precessione di Larmor. Per definizione di \(\mu_{\bot}(t)\), si ha:

\[\frac{d\mu_{\bot}}{dt} = - j\omega_{o}\mu_{\bot} = - j\omega_{o}\mu_{x}(t) - j\omega_{o}\left\lbrack j\mu_{y}(t) \right\rbrack = - j\omega_{o}\mu_{x}(t) - j\omega_{o}\mu_{y}(t)\]

Inoltre, deve risultare:

\[\frac{d\mu_{\bot}}{dt} = \frac{d\mu_{x}}{dt} + j\frac{d\mu_{y}}{dt}\]

Confrontando le due espressioni risulta che:

\[\frac{d\mu_{x}}{dt} = - j\omega_{o}\mu_{x}(t),\ \ \frac{d\mu_{y}}{dt} = - j\omega_{o}\mu_{y}(t)\]

La notazione fasoriale è più semplice rispetto al sistema di equazioni differenziali. Si integra per variabili separabili l'equazione:

\[\frac{d\mu_{\bot}}{dt} = - j\omega_{o}\mu_{\bot} \Leftrightarrow \frac{1}{\mu_{\bot}}d\mu_{\bot} = - j\omega_{o}\ dt\]

Da cui si ottiene:

\[\mu_{\bot}(t) = \mu_{\bot}(0)\exp\left( - j\omega_{0}t \right)\]

Questa soluzione coincide con quella nel dominio del tempo del vettore momento magnetico, ristretto al piano \(xy\). Il termine esponenziale, \(\exp\left( - j\omega_{0}t \right)\), rappresenta una rotazione in senso orario nel piano complesso o, equivalentemente, intorno all'asse \(z\) individuato dal campo principale. La rotazione nel piano \(xy\) è dovuto alla rotazione del momento magnetico intrinseco dello spin sul piano trasverso al campo applicato.

La fase del fasore è strettamente correlata con la pulsazione con cui il vettore momento magnetico intrinseco ruota nel piano trasverso. La conoscenza della posizione, ovvero della fase, è fondamentale per ricostruire l'immagine di risonanza magnetica. Per cui, data l'importanza della fase, è molto utile introdurre una notazione complessa per tale quantità, in modo da esplicitarla.

In generale, un numero complesso può essere espresso esplicitando modulo e fase:

\[\mu_{\bot}(t) = \left| \mu_{\bot}(t) \right|\exp\left\lbrack - j\varphi(t) \right\rbrack\]

Il modulo del fasore è costante nel tempo, infatti, nel dominio del tempo si è dimostrato che:

\[\overset{\underline{}}{\mu} \cdot \frac{d\overset{\underline{}}{\mu}}{dt} = 0\]

La fase, invece, è data da:

\[\angle\mu_{\bot} = \varphi(t) + \angle\mu(0) = - j\omega_{0}t + \varphi_{0}\]

Dove \(\varphi_{0} = \angle\mu(0)\).

L'evoluzione del fasore \(\mu_{\bot}\) può essere scritto come:

\[\mu_{\bot}(t) = \left| \mu_{\bot}(0) \right|\exp\left\lbrack - j\left( \omega_{0}t + \varphi_{0} \right) \right\rbrack\]

A partire dalla fase \(\varphi_{0}\) è possibile determinare l'evoluzione della posizione dello spin, nota la pulsazione di Larmor \(\omega_{0} = \gamma B_{0}\).

\subsubsection{\texorpdfstring{Sistema di riferimento rotante con velocità angolare \(\mathbf{\omega}\)}{Sistema di riferimento rotante con velocità angolare \textbackslash mathbf\{\textbackslash omega\}}}\label{sistema-di-riferimento-rotante-con-velocituxe0-angolare-mathbfomega}

Applicando un campo magnetico lungo un asse di un sistema di riferimento fisso, gli spin si orientano lungo l'asse individuato dal campo principale, \(z\), compiendo un moto di precessione all'equilibrio termodinamico.

Al fine di ottenere un segnale misurabile è necessario perturbare l'equilibrio termodinamica, mediante un campo elettromagnetico esterno con opportuna frequenza.

Per descrivere al meglio la perturbazione, si introduce un sistema di riferimento rotante con pulsazione \(\omega\) in senso orario interno all'asse \(z\).

Sia \(\left( x',y',z' \right)\) il sistema di riferimento rotante. Nel sistema di riferimento fisso del laboratorio, il sistema rotante presenta una velocità angolare:

\[\overset{\underline{}}{\Omega} = - \omega\widehat{z}\]

Dove il segno meno è dovuto alla rotazione oraria. La velocità di rotazione può variare nel tempo.

Si vuole determinare una relazione che leghi un vettore nel sistema fisso alle componenti del vettore rotante. Ogni vettore \(\overset{\underline{}}{c}\), istantaneamente ruotato con velocità angolare \(\overset{\underline{}}{\Omega}\), presenta una variazione temporale rispetto al sistema di riferimento fisso del laboratorio, descritta dalla relazione:

\[\frac{d\overset{\underline{}}{c}}{dt} = \overset{\underline{}}{\Omega} \times \overset{\underline{}}{c}\]

Il vettore \(\overset{\underline{}}{c}\) può essere espresso come somma della componente parallela all'asse \(z\) e della componente trasversale:

\[\overset{\underline{}}{c} = {\overset{\underline{}}{c}}_{\|} + {\overset{\underline{}}{c}}_{\bot}\]

Il vettore \(\overset{\underline{}}{\Omega}\) ha solo componente lungo \(z\), per cui il prodotto vettorare tra \(\overset{\underline{}}{\Omega}\) e la componente parallela all'asse \(z\) di \(\overset{\underline{}}{c}\), \({\overset{\underline{}}{c}}_{\|}\), è nullo:

\[\overset{\underline{}}{\Omega} \times {\overset{\underline{}}{c}}_{\|} = \overset{\underline{}}{0}\]

Se il vettore \(\overset{\underline{}}{c}\) è fermo nel sistema di riferimento fisso, la sua derivata temporale è nulla:

\[\frac{d\overset{\underline{}}{c}}{dt} = \overset{\underline{}}{0}\]

Nel sistema di riferimento rotante, il vettore \(\overset{\underline{}}{c}\) non è fermo; infatti, questo sistema vede il vettore ruotare, per cui risulta:

\[\left( \frac{d\overset{\underline{}}{c}}{dt} \right)' = \overset{\underline{}}{\Omega} \times {\overset{\underline{}}{c}}' \neq \overset{\underline{}}{0}\]

Dove \(\left( d\overset{\underline{}}{c}/dt \right)'\) indica la derivata del vettore \(\overset{\underline{}}{c}\) nel sistema di riferimento rotante \(x'y'\).

Nel sistema di riferimento fisso, il vettore \(\overset{\underline{}}{c}\) ha delle componenti:

\[\overset{\underline{}}{c}(t) = c_{x}(t){\widehat{i}}_{x} + c_{y}(t){\widehat{i}}_{y} + c_{z}(t){\widehat{i}}_{z}\]

Il tempo è considerato invariante nei due sistemi di riferimento poiché le velocità considerate sono molto minore della velocità della luce. Nel sistema rotante, invece, il vettore \(\overset{\underline{}}{c}\) possiede delle componenti diversi:

\[{\overset{\underline{}}{c}}'(t) = c_{x'}(t){\widehat{i}}_{x'} + c_{y'}(t){\widehat{i}}_{y'} + c_{z'}(t){\widehat{i}}_{z'}\]

Dato che i due sistemi possiedono l'asse \(z\) in comune, deve risultare \({\widehat{i}}_{z} \equiv {\widehat{i}}_{z'}\), per cui:

\[{\overset{\underline{}}{c}}'(t) = c_{x'}(t){\widehat{i}}_{x'} + c_{y'}(t){\widehat{i}}_{y'} + c_{z}(t){\widehat{i}}_{z}\]

Si osserva che non è necessario distinguere \(\overset{\underline{}}{c}(t)\) nel sistema fisso e \({\overset{\underline{}}{c}}'(t)\) nel sistema rotante in quanto il vettore è lo stesso per entrambi i sistemi di riferimento.

Le componenti del vettore nel sistema fisso dipendono da quelle nel sistema rotante e viceversa. Si calcola la derivata rispetto al tempo del vettore \(\overset{\underline{}}{c}(t)\) nel sistema fisso di riferimento:

\[\frac{d\overset{\underline{}}{c}}{dt} = \frac{d}{dt}\left\lbrack c_{x}(t){\widehat{i}}_{x} + c_{y}(t){\widehat{i}}_{y} + c_{z}(t){\widehat{i}}_{z} \right\rbrack = \frac{dc_{x}}{dt}{\widehat{i}}_{x} + c_{x}\frac{d{\widehat{i}}_{x}}{dt} + \frac{dc_{y}}{dt}{\widehat{i}}_{y} + c_{y}\frac{d{\widehat{i}}_{y}}{dt} + \frac{dc_{z}}{dt}{\widehat{i}}_{z} + c_{z}\frac{d{\widehat{i}}_{z}}{dt}\]

Siccome il sistema di riferimento è fisso, le derivate dei suoi versori sono nulle:

\[\frac{d{\widehat{i}}_{x}}{dt} = 0,\ \ \frac{d{\widehat{i}}_{y}}{dt} = 0,\ \ \frac{d{\widehat{i}}_{z}}{dt} = 0\]

Si ottiene:

\[\frac{d\overset{\underline{}}{c}}{dt} = \frac{dc_{x}}{dt}{\widehat{i}}_{x} + \frac{dc_{y}}{dt}{\widehat{i}}_{y} + \frac{dc_{z}}{dt}{\widehat{i}}_{z}\]

La derivata del sistema rotante è, invece:

\[\frac{d\overset{\underline{}}{c}}{dt} = \frac{dc_{x'}}{dt}{\widehat{i}}_{x'} + c_{x'}\frac{d{\widehat{i}}_{x'}}{dt} + \frac{dc_{y'}}{dt}{\widehat{i}}_{y'} + c_{y'}\frac{d{\widehat{i}}_{y'}}{dt} + \frac{dc_{z'}}{dt}{\widehat{i}}_{z'} + c_{z'}\frac{d{\widehat{i}}_{z'}}{dt}\]

La relazione \(d\overset{\underline{}}{c}/dt = \overset{\underline{}}{\Omega} \times \overset{\underline{}}{c}\) è valida per un generico vettore \(\overset{\underline{}}{c}\), dunque, è valida anche per i versori degli assi coordinati:

\[\frac{d{\widehat{i}}_{x'}}{dt} = \overset{\underline{}}{\Omega} \times {\widehat{i}}_{x'},\ \ \frac{d{\widehat{i}}_{y'}}{dt} = \overset{\underline{}}{\Omega} \times {\widehat{i}}_{y'},\ \ \frac{d{\widehat{i}}_{z'}}{dt} = \overset{\underline{}}{\Omega} \times {\widehat{i}}_{z'}\]

Per cui la derivata nel sistema rotante può essere scritta sostituendo queste relazioni:

\[\frac{d\overset{\underline{}}{c}}{dt} = \frac{dc_{x'}}{dt}{\widehat{i}}_{x'} + c_{x'}\overset{\underline{}}{\Omega} \times {\widehat{i}}_{x'} + \frac{dc_{y'}}{dt}{\widehat{i}}_{y'} + c_{y'}\overset{\underline{}}{\Omega} \times {\widehat{i}}_{y'} + \frac{dc_{z'}}{dt}{\widehat{i}}_{z'} + c_{z'}\overset{\underline{}}{\Omega} \times {\widehat{i}}_{z'} = \frac{dc_{x'}}{dt}{\widehat{i}}_{x'} + \frac{dc_{y'}}{dt}{\widehat{i}}_{y'} + \frac{dc_{z'}}{dt}{\widehat{i}}_{z'} + \overset{\underline{}}{\Omega} \times \left( c_{x'}{\widehat{i}}_{x'} + c_{y'}{\widehat{i}}_{y'} + c_{z'}{\widehat{i}}_{z'} \right)\]

Dove \({\overset{\underline{}}{c}}' = c_{x'}{\widehat{i}}_{x'} + c_{y'}{\widehat{i}}_{y'} + c_{z'}{\widehat{i}}_{z'}\) è il vettore nel sistema di riferimento fisso. Si indica con:

\[\left( \frac{d\overset{\underline{}}{c}}{dt} \right)' = \frac{dc_{x'}}{dt}{\widehat{i}}_{x'} + \frac{dc_{y'}}{dt}{\widehat{i}}_{y'} + \frac{dc_{z'}}{dt}{\widehat{i}}_{z'}\]

La derivata del vettore \(\overset{\underline{}}{c}\) nel sistema di riferimento rotante, i cui versori sono fissi. La derivata nel sistema di riferimento rotante può essere scritta come:

\[\frac{d\overset{\underline{}}{c}}{dt} = \left( \frac{d\overset{\underline{}}{c}}{dt} \right)' + \overset{\underline{}}{\Omega} \times {\overset{\underline{}}{c}}'\]

La derivata del vettore \(\overset{\underline{}}{c}\) nel sistema rotante è data dalla variazione nel tempo del vettore \(\overset{\underline{}}{c}\) nel sistema rotante a cui si aggiunge un termine che tiene conto della rotazione del sistema.

Si considera, come vettore \(\overset{\underline{}}{c}\), il momento magnetico di uno spin immerso in un campo magnetico \(\overset{\underline{}}{B}\) uniforme e diretto lungo \(z\):

\[\frac{d\overset{\underline{}}{\mu}}{dt} = \left( \frac{d\overset{\underline{}}{\mu}}{dt} \right)' + \overset{\underline{}}{\Omega} \times \overset{\underline{}}{\mu}\]

La variazione del momento magnetico è legata al campo magnetico dalla relazione:

\[\frac{d\overset{\underline{}}{\mu}}{dt} = \gamma\overset{\underline{}}{\mu} \times \overset{\underline{}}{B}\]

Sostituendo questo risultato nella relazione del sistema rotante, si ha:

\[\gamma\overset{\underline{}}{\mu} \times \overset{\underline{}}{B} = \left( \frac{d\overset{\underline{}}{\mu}}{dt} \right)' + \overset{\underline{}}{\Omega} \times \overset{\underline{}}{\mu}\]

Ricavando la derivata nel sistema rotante si ha:

\[\left( \frac{d\overset{\underline{}}{\mu}}{dt} \right)' = \gamma\overset{\underline{}}{\mu} \times \overset{\underline{}}{B} - \overset{\underline{}}{\Omega} \times \overset{\underline{}}{\mu}\]

È valido il risultato \(- \overset{\underline{}}{\Omega} \times \overset{\underline{}}{\mu} = \overset{\underline{}}{\mu} \times \overset{\underline{}}{\Omega}\). Sostituendo si ha:

\[\left( \frac{d\overset{\underline{}}{\mu}}{dt} \right)' = \gamma\overset{\underline{}}{\mu} \times \overset{\underline{}}{B} + \overset{\underline{}}{\mu} \times \overset{\underline{}}{\Omega} = \gamma\overset{\underline{}}{\mu} \times \left( \overset{\underline{}}{B} + \frac{1}{\gamma}\overset{\underline{}}{\Omega} \right)\]

Si definisce campo magnetico effettivo o efficace come:

\[{\overset{\underline{}}{B}}_{eff} = \overset{\underline{}}{B} + \frac{1}{\gamma}\overset{\underline{}}{\Omega}\]

L'equazione per il momento magnetico nel sistema rotante assume la forma:

\[\left( \frac{d\overset{\underline{}}{\mu}}{dt} \right)' = \gamma\overset{\underline{}}{\mu} \times {\overset{\underline{}}{B}}_{eff}\]

Si ottiene la stessa forma dell'equazione ricavata per il sistema fisso, a patto di sostituire il campo magnetico principale \(\overset{\underline{}}{B}\) applicato, il campo effettivamente visto nel sistema rotante \({\overset{\underline{}}{B}}_{eff}\), il quale prevede che, al campo magnetico applicato nel sistema rotante, si aggiunge un campo fittizio, dovuta alla rotazione \(\overset{\underline{}}{\Omega}\) del sistema. Si suppone che il sistema ruoti intorno all'asse \(z\) in senso orario con velocità \(\omega\):

\[\overset{\underline{}}{\Omega} = - \omega{\widehat{i}}_{z}\]

In queste condizioni, il campo efficace è:

\[{\overset{\underline{}}{B}}_{eff} = B_{0}{\widehat{i}}_{z} - \frac{1}{\gamma}\omega{\widehat{i}}_{z}\]

Se la frequenza con cui ruota il sistema coincide con quella di Larmor, ovvero \(\omega = \omega_{0} = \gamma B_{0}\), allora:

\[{\overset{\underline{}}{B}}_{eff} = B_{0}{\widehat{i}}_{z} - \frac{1}{\gamma}\omega_{0}{\widehat{i}}_{z} = B_{0}{\widehat{i}}_{z} - \frac{1}{\gamma}\gamma B_{0}{\widehat{i}}_{z} = \overset{\underline{}}{0}\]

Se la velocità di rotazione è uguale alla frequenza di Larmor, il momento magnetico \(\overset{\underline{}}{\mu}\) appare fisso nel sistema di riferimento rotante. Ne discende che questo sistema di riferimento è solidale con gli spin.

L'introduzione del sistema rotante permette di descrivere in modo semplice il segnale registrato. Nella pratica non è semplice produrre un campo con la stessa frequenza con cui risuonano i protoni, quindi, il campo non è mai nullo.

\subsubsection{Rotazione del sistema di riferimento per un campo a radiofrequenza}\label{rotazione-del-sistema-di-riferimento-per-un-campo-a-radiofrequenza}

Si considera un protone allineato al campo magnetico esterno, diretto lungo l'asse \(z\). Il protone risuona o precede alla frequenza di Larmor nel piano trasverso. Al fine di eccitare lo spin per eseguire la misura, bisogna stimarlo mediante un campo elettromagnetico alla frequenza di Larmor. Questo campo magnetico, essendo oscillante, possiede una polarizzazione.

Si suppone che la polarizzazione del campo eccitante sia lineare, ovvero che il campo sia diretto solamente lungo un asse, ad esempio \(x\). Il campo possiede, dunque, un andamento del tipo:

\[{\overset{\underline{}}{B}}_{x}(t) = b_{x}(t)\cos\left( \omega_{0}t \right){\widehat{i}}_{x}\]

Nell'espressione vi è la dipendenza addizionale dell'addizionale dell'ampiezza delle oscillazioni \(b_{x}(t)\), poiché il segnale emanato è, solitamente, un pacchetto di sinusoidi.

L'uso del sistema rotante rende la descrizione del campo magnetico a cui è soggetto lo spin più semplice. Infatti, in questo sistema il campo può essere descritto semplicemente osservando che i versori del sistema rotante sono legati a quelli del sistema fisso dalle relazioni:

\[\left\{ \begin{matrix}
{\widehat{i}}_{x'} = {\widehat{i}}_{x}\cos\left( \omega_{0}t \right) - {\widehat{i}}_{y}\sin\left( \omega_{0}t \right) \\
{\widehat{i}}_{y'} = {\widehat{i}}_{x}\sin\left( \omega_{0}t \right) + {\widehat{i}}_{y}\cos\left( \omega_{0}t \right)
\end{matrix} \right.\ \]

Si scrive il sistema in forma matriciale, in modo poter ricavare i versori del sistema fisso in funzione di quelli del sistema rotante.

\[\begin{pmatrix}
{\widehat{i}}_{x'} \\
{\widehat{i}}_{y'}
\end{pmatrix} = \begin{pmatrix}
\cos\left( \omega_{0}t \right) & - \sin\left( \omega_{0}t \right) \\
\sin\left( \omega_{0}t \right) & \cos\left( \omega_{0}t \right)
\end{pmatrix}\begin{pmatrix}
{\widehat{i}}_{x} \\
{\widehat{i}}_{y}
\end{pmatrix}\]

Dove:

\[R_{z}(\omega t) = \begin{pmatrix}
\cos\left( \omega_{0}t \right) & - \sin\left( \omega_{0}t \right) \\
\sin\left( \omega_{0}t \right) & \cos\left( \omega_{0}t \right)
\end{pmatrix}\]

È la matrice di rotazione intorno all'asse \(z\). Il suo determinante è unitario:

\[\det{R_{z}\left( \omega_{0}t \right)} = \left| \begin{matrix}
\cos\left( \omega_{0}t \right) & - \sin\left( \omega_{0}t \right) \\
\sin\left( \omega_{0}t \right) & \cos\left( \omega_{0}t \right)
\end{matrix} \right| = \cos^{2}\left( \omega_{0}t \right) + \sin^{2}\left( \omega_{0}t \right) = 1\]

\(R_{z}\left( \omega_{0}t \right)\) è invertibile:

\[R_{z}^{- 1}\left( \omega_{0}t \right) = \begin{pmatrix}
\cos\left( \omega_{0}t \right) & \sin\left( \omega_{0}t \right) \\
 - \sin\left( \omega_{0}t \right) & \cos\left( \omega_{0}t \right)
\end{pmatrix}\]

Per cui la relazione matriciale invertite è:

\[\begin{pmatrix}
{\widehat{i}}_{x} \\
{\widehat{i}}_{y}
\end{pmatrix} = \begin{pmatrix}
\cos\left( \omega_{0}t \right) & \sin\left( \omega_{0}t \right) \\
 - \sin\left( \omega_{0}t \right) & \cos\left( \omega_{0}t \right)
\end{pmatrix}\begin{pmatrix}
{\widehat{i}}_{x'} \\
{\widehat{i}}_{y'}
\end{pmatrix}\]

Il versore relativo all'asse \(x\) del sistema fisso, in funzione dei versori del sistema rotante, è:

\[{\widehat{i}}_{x} = {\widehat{i}}_{x'}\cos\left( \omega_{0}t \right) + {\widehat{i}}_{y'}\sin\left( \omega_{0}t \right)\]

Il campo magnetico polarizzato linearmente nel sistema rotante può essere scritto come:

\[{\overset{\underline{}}{B}}_{x}(t) = b_{x}(t)\cos\left( \omega_{0}t \right){\widehat{i}}_{x} = b_{x}(t)\cos\left( \omega_{0}t \right)\left\lbrack {\widehat{i}}_{x'}\cos\left( \omega_{0}t \right) + {\widehat{i}}_{y'}\sin\left( \omega_{0}t \right) \right\rbrack = b_{x}(t)\cos^{2}\left( \omega_{0}t \right){\widehat{i}}_{x'} + b_{x}(t)\cos\left( \omega_{0}t \right)\sin\left( \omega_{0}t \right){\widehat{i}}_{y'}\]

Per le formule di duplicazione e addizione, si ottiene:

\[{\overset{\underline{}}{B}}_{x}(t) = b_{x}(t)\left\lbrack \frac{1}{2}\cos\left( 2\omega_{0}t \right) + \frac{1}{2} \right\rbrack{\widehat{i}}_{x'} + \frac{1}{2}b_{x}(t)\sin\left( 2\omega_{0}t \right){\widehat{i}}_{y'}\]

Nel sistema rotante il campo magnetico lineare può essere scritto come somma di una costante con due campi a frequenza doppia di quella impostata:

\[{\overset{\underline{}}{B}}_{x}(t) = \frac{1}{2}b_{x}(t){\widehat{i}}_{x'} + b_{x}(t)\left\lbrack \frac{1}{2}\cos\left( 2\omega_{0}t \right){\widehat{i}}_{x'} + \frac{1}{2}\sin\left( 2\omega_{0}t \right){\widehat{i}}_{y'} \right\rbrack\]

Si calcola il valor medio del campo magnetico su un intervallo di tempo sufficientemente lungo, come un periodo dell'onda a radiofrequenza \(T\):

\[\left\langle {\overset{\underline{}}{B}}_{x}(t) \right\rangle = \frac{1}{T}\int_{T}^{}{{\overset{\underline{}}{B}}_{x}(t)dt} = \frac{1}{2T}\int_{T}^{}{\left| \left\{ b_{x}(t){\widehat{i}}_{x'} + b_{x}(t)\left\lbrack \cos\left( 2\omega_{0}t \right){\widehat{i}}_{x'} + \sin\left( 2\omega_{0}t \right){\widehat{i}}_{y'} \right\rbrack \right\} \right|dt} = \frac{1}{2}\left\langle b_{x}(t) \right\rangle\]

Gli altri termini sono nulli poiché termini sinusoidali integrati su un intervallo temporale uguale al doppio del periodo di oscillazione. L'applicazione del valor medio implica che solo metà della polarizzazione lineare è utilizzata per ruotare gli spin intorno all'asse \(x\). Per un tempo sufficientemente lungo il campo può essere considerato a media costante.

In molte applicazioni, per rendere più precisa la ricostruzione delle immagini, si instaura un campo magnetico con polarizzazione circolare, ottenuto sovrapponendo due campi lineari di ugual intensità, in quadratura e diretti lungo due assi ortogonali.

\begin{figure}
\centering
\includegraphics[width=2.96296in,height=1.78624in,alt={P3006\#yIS1}]{media/6_IntroMRI/image63.pdf}\caption{Figura .: Sovrapposizione di onde ortogonali}
\end{figure}

A tale scopo si posizionano due antenne ortogonali tra loro, ognuna delle quali eroga un campo magnetico lineare. Affinché la polarizzazione sia circolare è necessario che i due campi siano in quadratura tra loro. Il campo totale, polarizzato circolarmente, è espresso, nel sistema di riferimento fisso, come:

\[\overset{\underline{}}{B}(t) = B_{1}\left\lbrack \cos\left( \omega_{0}t \right){\widehat{i}}_{x} - \sin\left( \omega_{0}t \right){\widehat{i}}_{x} \right\rbrack\]

Si ricava l'espressione del campo magnetico polarizzato circolarmente nel sistema di riferimento rotante alla frequenza di Larmor. A tale scopo, si considerano le relazioni tra i versori dei due sistemi di riferimento:

\[\begin{pmatrix}
{\widehat{i}}_{x} \\
{\widehat{i}}_{y}
\end{pmatrix} = \begin{pmatrix}
\cos\left( \omega_{0}t \right) & \sin\left( \omega_{0}t \right) \\
 - \sin\left( \omega_{0}t \right) & \cos\left( \omega_{0}t \right)
\end{pmatrix}\begin{pmatrix}
{\widehat{i}}_{x'} \\
{\widehat{i}}_{y'}
\end{pmatrix} \Leftrightarrow \left\{ \begin{matrix}
{\widehat{i}}_{x} = \cos\left( \omega_{0}t \right){\widehat{i}}_{x'} + \sin\left( \omega_{0}t \right){\widehat{i}}_{y'} \\
{\widehat{i}}_{y} = - \sin\left( \omega_{0}t \right){\widehat{i}}_{x'} + \cos\left( \omega_{0}t \right){\widehat{i}}_{y'}
\end{matrix} \right.\ \]

Sostituendo queste relazioni nell'espressione del campo magnetico circolare si ha:

\[\overset{\underline{}}{B}(t) = B_{1}\left\lbrack \cos\left( \omega_{0}t \right){\widehat{i}}_{x} - \sin\left( \omega_{0}t \right){\widehat{i}}_{x} \right\rbrack = B_{1}\left\{ \cos\left( \omega_{0}t \right)\left\lbrack \cos\left( \omega_{0}t \right){\widehat{i}}_{x'} + \sin\left( \omega_{0}t \right){\widehat{i}}_{y'} \right\rbrack - \sin\left( \omega_{0}t \right)\left\lbrack - \sin\left( \omega_{0}t \right){\widehat{i}}_{x'} + \cos\left( \omega_{0}t \right){\widehat{i}}_{y'} \right\rbrack \right\} = B_{1}\left\{ \cos^{2}\left( \omega_{0}t \right){\widehat{i}}_{x'} + \cos\left( \omega_{0}t \right)\sin\left( \omega_{0}t \right){\widehat{i}}_{y'} + \sin^{2}\left( \omega_{0}t \right){\widehat{i}}_{x'} - \sin\left( \omega_{0}t \right)\cos\left( \omega_{0}t \right){\widehat{i}}_{y'} \right\} = B_{1}\left\lbrack \cos^{2}\left( \omega_{0}t \right){\widehat{i}}_{x'} + \sin^{2}\left( \omega_{0}t \right){\widehat{i}}_{x'} \right\rbrack = B_{1}{\widehat{i}}_{x'}\]

Per ogni istante temporale, il campo magnetico a polarizzazione circolare con frequenza uguale a quella del sistema rotante è orientato lungo l'asse \(x'\); inoltre, a differenza del caso lineare, la relazione è valida per ogni istante di tempo poiché non è ottenuta mediante operazione di media.

Riassumendo, quando si applica l'impulso a radiofrequenza con polarizzazione circolare, gli spin ruotano intorno all'asse \(x'\), lungo cui iniziano un moto di precessione.

È possibile dimostrare che si ha un minor dispendio energetico per generare un campo a polarizzazione circolare. A causa di ciò, corredati ad altri fattori riguardanti il rapporto segnale/rumore e la necessità di omogeneizzare il campo a radiofrequenza, i campi rotanti a polarizzazione circolare sono molto usati in risonanza magnetica.

\paragraph{Condizione di risonanza}\label{condizione-di-risonanza}

Si applica un campo a radiofrequenza con polarizzazione lineare o circolare; nel primo caso bisogna considerare quantità medie, mentre nel secondo si hanno relazioni esatte dal punto di vista teorico. Nel sistema rotante il campo magnetico subito da uno spin è detto campo effettivo.

Prima dell'applicazione dell'impulso, l'equazione che descrive il comportamento dello spin, dal punto di vista classico, nel sistema rotante è:

\[\left( \frac{d\overset{\underline{}}{\mu}}{dt} \right)' = \gamma\overset{\underline{}}{\mu} \times {\overset{\underline{}}{B}}_{eff}\]

Dove:

\[{\overset{\underline{}}{B}}_{eff} = B_{0}{\widehat{i}}_{z} - \frac{1}{\gamma}\omega{\widehat{i}}_{z}\]

\(B_{0}{\widehat{i}}_{z}\) è il campo statico esterno o principale applicato.

Si applica, ora, un campo a radiofrequenza. Se il sistema di riferimento ruota con la stessa pulsazione \(\omega_{0}\) del campo magnetico applicato, al campo effettivo, \({\overset{\underline{}}{B}}_{eff}\), va aggiunto un termine costante, diretto lungo \({\widehat{i}}_{x'}\):

\[\left( \frac{d\overset{\underline{}}{\mu}}{dt} \right)' = \gamma\overset{\underline{}}{\mu} \times \left( B_{0}{\widehat{i}}_{z} - \frac{1}{\gamma}\omega{\widehat{i}}_{z} + B_{1}{\widehat{i}}_{x'} \right)\]

I due sistemi di riferimento possiedono gli assi \(z\) paralleli, per cui \({\widehat{i}}_{z} = {\widehat{i}}_{z'}\).

Si dice condizione di risonanza se la frequenza dell'impulso a radiofrequenza è uguale alla frequenza di Larmor degli spin contenuti in uno strato di tessuto:

\[\omega = \omega_{0} = \gamma B_{0}\]

In questo caso, risulta:

\[\left( \frac{d\overset{\underline{}}{\mu}}{dt} \right)' = \gamma B_{1}\overset{\underline{}}{\mu} \times {\widehat{i}}_{x'}\]

Si assiste, dunque, a una processione degli spin intorno all'asse \({\widehat{i}}_{x}'\)

\begin{figure}
\centering
\includegraphics[width=4.4166in,height=3.61111in,alt={P3032\#yIS1}]{media/6_IntroMRI/image64.pdf}\caption{Figura .: Precessione nel sistema rotante in condizione di risonanza}
\end{figure}

Il termine risonanza, in questa tecnica di imaging, è legato al fatto che, per ottenere l'immagine, il campo magnetico applicato è sintonizzato con la frequenza degli spin in moto di precessione intorno al campo magnetico applicato.

L'introduzione del sistema rotante consente di descrivere in modo semplice il comportamento degli spin, mediante un moto di precessione intorno all'asse \(x'\). Nel sistema di riferimento fisso del laboratorio, invece, il contributo del campo a radiofrequenza si somma a quello stato e omogeneo, diretto lungo \({\widehat{i}}_{z}\). Si dimostra che la combinazione dei due campi produce un modo elicoidale, il cui raggio aumento all'avvicinarsi del piano \(xy\).

\begin{figure}
\centering
\includegraphics[width=4.13528in,height=2.63095in,alt={P3036\#yIS1}]{media/6_IntroMRI/image65.pdf}\caption{Figura .: Moto elicoidale nel sistema fisso}
\end{figure}

Nel sistema rotante con pulsazione \(\omega\), è possibile scrivere il campo efficace come:

\[{\overset{\underline{}}{B}}_{eff} = B_{0}{\widehat{i}}_{z} - \frac{1}{\gamma}\omega{\widehat{i}}_{z} + B_{1}{\widehat{i}}_{x'}\]

Con pulsazione di Larmor:

\[\omega = \gamma B_{0} \Leftrightarrow B_{0} = \frac{1}{\gamma}\omega_{0}\]

La pulsazione del campo a radiofrequenza applicato è data da:

\[\omega_{1} = \gamma B_{1} \Leftrightarrow B_{1} = \frac{1}{\gamma}\omega_{1}\]

Per cui è possibile scrivere:

\[{\overset{\underline{}}{B}}_{eff} = \frac{1}{\gamma}\left\lbrack \left( \omega_{0} - \omega \right){\widehat{i}}_{z} + \omega_{1}{\widehat{i}}_{x'} \right\rbrack\]

In generale, avere un campo a radiofrequenza con stessa frequenza di Larmor, con cui precedono gli spin, è abbastanza complesso per la schermatura che alcune molecole compiono nei confronti dei campi magnetici. Inoltre, in linea di principio la frequenza con cui ruota il sistema rotante, in senso orario, è scelta arbitrariamente rispetto sia al campo a radiofrequenza che alla frequenza di precessione degli spin. Nel caso generale, si ha la stessa complessità della descrizione del sistema fisso. I vantaggi del sistema rotante si mostrano quando \(\omega_{1} = \omega_{0} = \omega\), poiché il vettore momento magnetico ruota intorno all'asse \(x'\).

Generalmente gli impulsi a radiofrequenza hanno ampiezza di qualche \(mT\) o \(\mu T\). Si suppone, infatti, di applicare un impulso a radiofrequenza per un tempo \(\tau = 1\ ms\), così da far ruotare di \(\pi/2\) il campo momento magnetico. L'ampiezza del campo a radiofrequenza necessaria a tale scopo è data da:

\[\mathrm{\Delta}\vartheta = \gamma B_{1}\tau\]

Dove \(\gamma = 42.6\ MHz/T\); si ottiene:

\[B_{1} = \frac{\mathrm{\Delta}\vartheta}{\gamma\tau} = \frac{\frac{\pi}{2}}{42.6\ \frac{MHz}{T}1\ ms} \simeq 5.9\ \mu T\]

Il ribaltamento degli spin a opera del campo magnetico a radiofrequenza è detto in gergo to flip.

La maggior difficoltà pratica e tecnica della risonanza magnetica consiste nella produzione del campo magnetico principale, costante nel tempo ed omogeno nello spazio.

I campi a radiofrequenza con frequenza di Larmor riescono a flippare gli spin anche avendo un intensità molto minore del campo principale; tuttavia, più i campo a radiofrequenza si allontana dalla frequenza di precessione di Larmor e più il campo effettivo nel sistema rotante tende a quello stato, ovvero, è minore il numero degli spin ruotati di \(\mathrm{\Delta}\vartheta\). Le frequenze degli impulsi devono essere compatibili con i tempi di rilassamento del corpo in esame.

\paragraph{Calcolo del campo magnetico a radiofrequenza}\label{calcolo-del-campo-magnetico-a-radiofrequenza}

L'uso del sistema rotante intorno all'asse \({\widehat{i}}_{z}\) permette di descrivere un campo a polarizzazione circolare in modo molto semplice. Inoltre, se la frequenza del campo a radiofrequenza e quella del sistema rotante sono uguali a quella di Larmor, è possibile scrivere la relazione:

\[\left( \frac{d\overset{\underline{}}{\mu}}{dt} \right)' = \gamma B_{1}\overset{\underline{}}{\mu} \times {\widehat{i}}_{x'}\]

Dove \({\overset{\underline{}}{B}}_{1} = B_{1}{\widehat{i}}_{x'}\) è il campo magnetico polarizzato circolarmente visto nel sistema rotante. La rotazione intorno all'asse \({\widehat{i}}_{z}\), nel sistema fisso, si scrive come:

\[\overset{\underline{}}{\mu}(t) = {\overset{\underline{}}{\overset{\underline{}}{R}}}_{z}(\omega t)\overset{\underline{}}{\mu}(0)\]

Dove \({\overset{\underline{}}{\overset{\underline{}}{R}}}_{z}\) è la matrice di rotazione intorno all'asse \({\widehat{i}}_{z}\):

\[{\overset{\underline{}}{\overset{\underline{}}{R}}}_{z} = \begin{pmatrix}
\cos\left( \omega_{0}t \right) & - \sin\left( \omega_{0}t \right) & 0 \\
\sin\left( \omega_{0}t \right) & \cos\left( \omega_{0}t \right) & 0 \\
0 & 0 & 1
\end{pmatrix}\]

Invece di ruotare il sistema lungo l'asse \({\widehat{i}}_{z}\), si suppone che la rotazione avvenga intorno all'asse \({\widehat{i}}_{y}\). In questo caso, la soluzione si scrive come:

\[\overset{\underline{}}{\mu}(t) = {\overset{\underline{}}{\overset{\underline{}}{R}}}_{y}(\omega t)\overset{\underline{}}{\mu}(0)\]

Dove:

\[{\overset{\underline{}}{\overset{\underline{}}{R}}}_{y} = \begin{pmatrix}
\cos\left( \omega_{0}t \right) & 0 & - \sin\left( \omega_{0}t \right) \\
0 & 1 & 0 \\
\sin\left( \omega_{0}t \right) & 0 & \cos\left( \omega_{0}t \right)
\end{pmatrix}\]

Con rotazione del sistema attorno all'asse \({\widehat{i}}_{x}\), la soluzione è:

\[\overset{\underline{}}{\mu}(t) = {\overset{\underline{}}{\overset{\underline{}}{R}}}_{x}(\omega t)\overset{\underline{}}{\mu}(0)\]

Con:

\[{\overset{\underline{}}{\overset{\underline{}}{R}}}_{z} = \begin{pmatrix}
1 & 0 & 0 \\
0 & \cos\left( \omega_{0}t \right) & - \sin\left( \omega_{0}t \right) \\
0 & \sin\left( \omega_{0}t \right) & \cos\left( \omega_{0}t \right)
\end{pmatrix}\]

L'angolo di cui ruota il sistema di riferimento rotante rispetto a quello fisso, fissato un istante temporale \(\tau\), è dato da:

\[\phi = \omega_{1}\tau\]

\(\omega_{1}\) è legato al campo a radiofrequenza applicato dalla relazione:

\[\omega_{1} = \gamma B_{1}\]

Con \(B_{1}\) costante. L'angolo di cui ruota il sistema può essere scritto come:

\[\phi = \gamma B_{1}\tau\]

Nel caso generale, in cui l'ampiezza del campo a radiofrequenza cambia nel tempo, ovvero \(B_{1} = B_{1}(t)\), la fase all'istante fissato \(\tau\), è ottenuto come integrale temporale, infatti:

\[\frac{d\phi}{dt} = \omega_{1} \Leftrightarrow \phi = \int_{t_{0}}^{\tau}{\omega_{1}dt}\]

Per il legame tra pulsazione del sistema rotante e campo a radiofrequenza applicato, si ha:

\[\phi = \int_{t_{0}}^{\tau}{\gamma B_{1}dt}\]

Questa generalizzazione è necessaria nel caso in cui il campo polarizzato circolarmente sia applicato per un intervallo temporale finito e abbia ampiezza variabile nel tempo. L'uso dell'integrale, quindi, permette di ottenere una descrizione più generale, applicabile anche a campi di ampiezza variabile nel tempo.

Si suppone di applicare un campo a radiofrequenza \(B_{1}\), diretto lungo \({\widehat{i}}_{x'}\), tale da far ruotare gli spin di un angolo \(\vartheta\) rispetto all'asse \({\widehat{i}}_{x'}\). Si applica, non appena viene interrotta la trasmissione del primo campo, un secondo impulso a radiofrequenza, diretto lungo l'asse \({\widehat{i}}_{y'}\). La descrizione del moto di precessione, nel sistema di riferimento fisso, in questo scenario, è dato da:

\[\overset{\underline{}}{\mu}(t) = {\overset{\underline{}}{\overset{\underline{}}{R}}}_{y}{\overset{\underline{}}{\overset{\underline{}}{R}}}_{x}(\omega t)\overset{\underline{}}{\mu}(0)\]

\subsection{Vettore di magnetizzazione}\label{vettore-di-magnetizzazione}

Per ottenere un'immagine di una sezione del corpo umano mediante il fenomeno della risonanza magnetica, si suddivide il corpo del paziente in volumetti elementi contenenti un numero di Avogadro di protoni che precedono alla frequenza di Larmor.

Dato l'elevato numero di spin si introduce il vettore di magnetizzazione su unità di volume \(\overset{\underline{}}{M}\).

Si posiziona un volumetto elementare contenente un numero di Avogadro \(N_{A}\) di protoni, in un campo magnetico diretto lungo \({\widehat{i}}_{z}\). Gli spin nel volumetto si allineano rispetto all'asse del campo magnetico esterno.

\begin{figure}
\centering
\includegraphics[width=2.89286in,height=2.63294in,alt={P3086\#yIS1}]{media/6_IntroMRI/image66.pdf}\caption{Figura .: Volumetto elementare di \(N_{A}\) di protoni in un campo magnetico diretto lungo \({\widehat{i}}_{z}\)}
\end{figure}

Secondo la meccanica quantistica, in realtà, gli spin saltano su due livelli energetici: \(\left\langle - \right|\), di energia \(+ \hslash/2\) e \(\left| + \right\rangle\), di energia \(- \hslash/2\). Dal punto di vista macroscopico, ogni volumetto elementare presenta un vettore di magnetizzazione \(\overset{\underline{}}{M}\), dato dalla somma dei singoli momenti magnetici associati ai protoni, normalizzato il volume elementare:

\[\overset{\underline{}}{M} = \frac{1}{V}\sum_{i = 1}^{N_{A}}{\overset{\underline{}}{\mu}}_{i}\]

L'insieme degli spin nel volumetto elementare, dato che risuonano alla stessa frequenza, è chiamato spin \emph{isochromat} poiché può essere considerato come un insieme o un dominio di spin con stessa frequenza.

Trascurando le iterazioni dei protoni con il loro ambiente, è possibile supporre che ogni spin precede intorno al campo applicato secondo l'equazione:

\[\frac{d{\overset{\underline{}}{\mu}}_{i}}{dt} = \gamma{\overset{\underline{}}{\mu}}_{i} \times {\overset{\underline{}}{B}}_{0}\]

Sommando i vari contributi di ogni spin del volumetto e dividendo per il volume elementare si ha:

\[\frac{1}{V}\sum_{i = 1}^{N_{A}}\frac{d{\overset{\underline{}}{\mu}}_{i}}{dt} = \frac{1}{V}\sum_{i = 1}^{N_{A}}{\gamma{\overset{\underline{}}{\mu}}_{i} \times {\overset{\underline{}}{B}}_{0}}\]

Per la linearità dell'operatore somma e derivata è possibile scrivere:

\[\sum_{i = 1}^{N_{A}}{\frac{d}{dt}\left( \frac{1}{V}{\overset{\underline{}}{\mu}}_{i} \right)} = \sum_{i = 1}^{N_{A}}{\gamma\frac{1}{V}{\overset{\underline{}}{\mu}}_{i} \times {\overset{\underline{}}{B}}_{0}} \Leftrightarrow \frac{d}{dt}\left( \sum_{i = 1}^{N_{A}}{\frac{1}{V}{\overset{\underline{}}{\mu}}_{i}} \right) = \gamma\left( \sum_{i = 1}^{N_{A}}{\frac{1}{V}{\overset{\underline{}}{\mu}}_{i}} \right) \times {\overset{\underline{}}{B}}_{0}\]

Per definizione del vettore di magnetizzazione per unità di volume è possibile scrivere:

\[\frac{d\overset{\underline{}}{M}}{dt} = \gamma\overset{\underline{}}{M} \times {\overset{\underline{}}{B}}_{0}\]

È molto conveniente analizzare la magnetizzazione del volumetto e, quindi, l'equazione differenziale, in termini del vettore di magnetizzazione parallelo all'asse individuato dal campo magnetico principale e la componente trasversale. In altre parole, si indica:

\[\overset{\underline{}}{M} = {\overset{\underline{}}{M}}_{\|} + {\overset{\underline{}}{M}}_{\bot}\]

Dove, in coordinate cartesiane e con un campo magnetico esterno diretto lungo \({\widehat{i}}_{z}\):

\[{\overset{\underline{}}{M}}_{\|} = M_{z}{\widehat{i}}_{z}\]

\[{\overset{\underline{}}{M}}_{\bot} = M_{x}{\widehat{i}}_{x} + = M_{y}{\widehat{i}}_{y}\]

L'equazione differenziale può essere scritta scomponendo il vettore di magnetizzazione:

\[\frac{d}{dt}\left( {\overset{\underline{}}{M}}_{\|} + {\overset{\underline{}}{M}}_{\bot} \right) = \gamma\left( {\overset{\underline{}}{M}}_{\|} + {\overset{\underline{}}{M}}_{\bot} \right) \times B_{o}{\widehat{i}}_{Z}\]

Per linearità:

\[\frac{d{\overset{\underline{}}{M}}_{\|}}{dt} + \frac{d{\overset{\underline{}}{M}}_{\bot}}{dt} = \gamma B_{o}{\overset{\underline{}}{M}}_{\|} \times {\widehat{i}}_{Z} + \gamma B_{o}{\overset{\underline{}}{M}}_{\bot} \times {\widehat{i}}_{Z}\]

Per definizione di componente parallela del vettore di magnetizzazione, il termine \({\overset{\underline{}}{M}}_{\|} \times {\widehat{i}}_{Z}\) si annulla:

\[{\overset{\underline{}}{M}}_{\|} \times {\widehat{i}}_{Z} = M_{z}{\widehat{i}}_{Z} \times {\widehat{i}}_{Z} = 0\]

Per cui:

\[\frac{d{\overset{\underline{}}{M}}_{\|}}{dt} + \frac{d{\overset{\underline{}}{M}}_{\bot}}{dt} = \gamma B_{o}{\overset{\underline{}}{M}}_{\bot} \times {\widehat{i}}_{Z}\]

Scrivendo le due equazioni per le proiezioni si ha:

\[\left\{ \begin{matrix}
\frac{dM_{\|}}{dt} = 0 \\
\frac{d{\overset{\underline{}}{M}}_{\bot}}{dt} = \gamma B_{o}{\overset{\underline{}}{M}}_{\bot} \times {\widehat{i}}_{Z}
\end{matrix} \right.\ \]

Questa modellazione non considera i fenomeni di interazione tra gli spin e il reticolo o tra i veri spin del volumetto. In particolare, delle equazioni risulta che la dinamica longitudinale, evolve in maniera indipendente da quella trasversale. La prima è descritta dal tempo di rilassamento longitudinale \(T_{1}\) mentre la seconda dal tempo di rilassamento trasversale \(T_{2}\). Come risultato delle due evoluzioni, il modulo del vettore magnetizzazione non è costante nel tempo.

Per comprendere tale concetto si suppone di porre un volumetto elementare in un campo magnetico diretto lungo l'asse \({\widehat{i}}_{z}\). Raggiunto l'equilibrio termodinamico, il vettore di magnetizzazione si raggiunge il regime, con modulo \(M_{0}\) e diretto come il campo magnetico esterno. Si suppone di applicare un impulso elettromagnetico tale da far ruotare il vettore di magnetizzazione. Durante il ritorno alle condizioni di equilibrio termodinamico, la componente longitudinali del vettore di magnetizzazione è più lenta nel raggiungere il valore di regime \(M_{0}\), rispetto alla componente trasversale, che deve raggiungere il valore iniziale, ovvero nullo.

Questo comportamento è descritto dalle equazioni di Bloch, basate su una descrizione classica della materia. La complessità del moto del vettore di magnetizzazione è dovuta al fatto che tale vettore è composto da circa \(10^{23}\) momenti magnetici, ognuno dei quali con una proprio andamento, influenzato dall'ambiente.

Si osservi che a temperature ambiente, il vettore di magnetizzazione è dato dalla legge di Curie:

\[M_{0} \simeq \frac{\gamma^{2}\hslash^{2}}{4k_{B}T}B_{0}\frac{N_{A}}{V}\]

Il rapporto tra numero di spin, uguale al numero di Avogadro per un volumetto elementare, e il volume del corpo è detto densità protonica ed è indicato con \(\rho\):

\[\rho = \frac{N_{A}}{V}\]

Per cui:

\[M_{0} \simeq \frac{\gamma^{2}\hslash^{2}}{4k_{B}T}B_{0}\rho\]

Le iterazioni degli spin con l'ambiente dipendono dal tessuto analizzato e dal suo stato di salute. Note le tempistiche con cui evolve il vettore di magnetizzazione macroscopico, nel ritorno alla condizione di regime, è possibile ricostruire l'immagine del tessuto.

\subsubsection{Iterazione spin-reticolo}\label{iterazione-spin-reticolo}

In normali condizioni, gli spin non sono indipendenti tra loro ma interagiscono sia con il materiale in cui sono contenuti sia, nello stesso istante, con gli altri spin del reticolo. Ciò determina che l'equazione per la componente longitudinale:

\[\frac{dM_{\|}}{dt} = 0\]

è errata poiché non tiene conto di tali fenomeni; infatti, momenti magnetici dei protoni tendono ad allinearsi col campo esterno, tuttavia, le iterazioni con l'ambiente circostante non consentono al vettore magnetizzazione di essere diretto esattamente lungo l'asse del campo principale, \({\widehat{i}}_{z}\).

Ogni spin è inclinato rispetto l'asse verticale di un angolo \(\vartheta\), inoltre, ogni spin interagisce con gli altri prossimi. Nello specifico, l'angolo \(\vartheta\) è l'inclinazione del vettore che congiunge il centro del sistema di riferimento con lo spin rispetto l'asse individuato dal campo magnetico esterno.

Nell'analisi della componente longitudinale è possibile trascurare gli effetti di repulsione dei protoni dovute alle interazioni coulombiane. In altre parole, si ritiene che la distanza tra due protoni, detta distanza internucleare, sia molto maggiore del raggio gi azione delle forze elettriche.

\begin{figure}
\centering
\includegraphics[width=5.08796in,height=4.76042in,alt={P3130\#yIS1}]{media/6_IntroMRI/image67.pdf}\caption{Figura .: Spin con diverse inclinazioni}
\end{figure}

Ogni singolo spin può essere descritto come un dipolo magnetico elementare, il cui potenziale vettore si dimostra essere:

\[\overset{\underline{}}{A} = \frac{1}{\left| \overset{\underline{}}{r} \right|}\overset{\underline{}}{\mu} \times \overset{\underline{}}{r}\]

Dove \(\overset{\underline{}}{r}\) è il vettore che congiunge due spin.

\begin{figure}
\centering
\includegraphics[width=3.92856in,height=2.17593in,alt={P3135\#yIS1}]{media/6_IntroMRI/image68.pdf}\caption{Figura .: Vettore distanza tra due spin}
\end{figure}

Dal punto di vista classico, il campo prodotto da uno spin è dato da:

\[\overset{\underline{}}{B} = \overset{\underline{}}{\nabla} \times \overset{\underline{}}{A} = \frac{1}{\left| \overset{\underline{}}{r} \right|}\overset{\underline{}}{\nabla} \times \left( \overset{\underline{}}{\mu} \times \overset{\underline{}}{r} \right)\]

Si dimostra che la soluzione a tale equazione è data da:

\[\left\{ \begin{matrix}
B_{x} = 3\mu\frac{\sin\vartheta\cos\vartheta}{r^{3}} \\
B_{y} = 0 \\
B_{z} = \mu\frac{3\cos^{2}\vartheta - 1}{r^{3}}
\end{matrix} \right.\ \]

La componente lungo \({\widehat{i}}_{y}\) è nulla poiché l'andamento delle linee di campo magnetico di uno spin hanno simmetria cilindrica.

\begin{figure}
\centering
\includegraphics[width=2.20513in,height=2.44444in,alt={P3142\#yIS1}]{media/6_IntroMRI/image69.pdf}\caption{Figura .: Simmetria cilindrica del campo prodotto da uno spin}
\end{figure}

Per il principio di sovrapposizione è possibile che uno spin influenzi quelli vicini e viceversa, uno spin è influenzato da quelli vicini.

La componente lungo \({\widehat{i}}_{z}\) è legata all'energia del sistema, infatti, in meccanica classica, si prova che l'energia di un momento magnetico \(\overset{\underline{}}{\mu}\) immerso in un campo magnetico \({\overset{\underline{}}{B}}_{0}\) è data da:

\[U = - \overset{\underline{}}{\mu} \cdot {\overset{\underline{}}{B}}_{0}\]

Con un campo con solo componente verticale ovvero \({\overset{\underline{}}{B}}_{0} = B_{0}{\widehat{i}}_{z}\), si ha:

\[U = - \mu_{z}B_{0}\]

Ne discende che la componente verticale, lungo \({\widehat{i}}_{z}\), è coinvolta negli scambi energetici tra dipoli elementari.

L'interpretazione della componente lungo \({\widehat{i}}_{x}\) del campo prodotto da uno spin \(B_{x}\), è più complessa in quanto bisogna tener conto che i dipoli non sono fermi nello spazio ma precedono intorno all'asse individuato dal campo esterno, uno ente a influenzare questa componente.

Nello specifico, a causa dell'agitazione termica, i dipoli non sono fissi nello spazio ma in moto casuale. L'agitazione termica è molto forte nei liquidi, di cui è prevalentemente composto il corpo umano, e nei gas.

A causa dell'agitazione termica, i dipoli si muovono reciprocamente tra loro in modo continuo. Le iterazioni reciproche dei campo magnetici degli spin sono molto complesse da descrivere; infatti, la teoria del rilassamento è molto sofistica e può essere affrontata solamente adoperando la meccanica quantistica. In letteratura, esistono delle trattazioni riguardati solamente le sostanze pure. A oggi, non esistono analisi basate sulla meccanica quantistica in grado di descrivere il fenomeno del rilassamento nel corpo umano per l'elevato numero di molecole di cui è composto, come acqua, sali disciolti, proteine, lipidi ed ecc.

Nell'analisi della risonanza magnetica, si forniscono delle leggi che descrivono in modo approssimativo, ma allo stesso modo abbastanza fedelmente, i meccanismi di rilassamento all'interno dei tessi umani,

A causa delle iterazioni tra i vari spin, un protone non è soggetto a un campo magnetico uguale a quello statico applicato, ma prossimo a esso. Ogni spin, per le influenze degli altri, ruoterà attorno all'asse individuato dal campo magnetico principale con propria frequenza di Larmor.

Quando è applicato un campo magnetico a frequenza \(\omega_{0} = \gamma B_{0}\), è probabile che avvengano delle transizioni degli spin dallo stato \(\left| + \right\rangle\) allo stato \(\left| - \right\rangle\) e viceversa. Dunque, un dipolo magnetico, soggetto a una radiazione elettromagnetica a frequenza \(\omega_{0}\), prossima a quella con cui precede, può essere indotto a una transizione verso l'alto o verso il basso:

\begin{itemize}
\item
  Nella transizione dal livello energetico \(\left| + \right\rangle\) a \(\left| - \right\rangle\), il protone emette un fotone di energia \(\hslash\gamma B_{0} = \hslash\omega_{0}\). Tale fenomeno è noto come emissione stimolata;
\item
  Nella transizione dal livello energetico \(\left| - \right\rangle\) a \(\left| + \right\rangle\), protone assorbe un foto di energia \(\hslash\omega_{0}\).
\end{itemize}

Il passaggio dallo stato \(\left| + \right\rangle\) a \(\left| - \right\rangle\) produce l'emissione di un fotone che può stimolare un altro spin vicino, il quale transita a sua volta da \(\left| - \right\rangle\) a \(\left| + \right\rangle\). Le iterazioni energetiche a livello microscopico sono, quindi, molto complesse a causa dell'elevato numero di spin contenuti in un volumetto elementare.

Per analizzare il comportamento di un singolo spin con l'intera massa che lo circonda, è possibile ricorrere alla statistica di Boltzmann, considerando lo spin a contatto con il reticolo, o \emph{lattice}, come un piccolo sistema a contatto con un altro avente un numero di spin molto maggiore.

Sia \(N^{+}\) il numero di protoni nel livello energetico \(E^{+} = \hslash\gamma B_{0}/2\) e \(N^{-}\) il numero di protoni nel livello energetico \(E^{+} = - \hslash\gamma B_{0}/2\). Sia \(W_{+ \rightarrow -}\) la probabilità di transizione di uno spin dallo stato \(\left| + \right\rangle\) a \(\left| - \right\rangle\), mentre \(W_{- \rightarrow +}\) la probabilità opposta, ovvero che uno spin transiti da uno stato \(\left| - \right\rangle\) a \(\left| + \right\rangle\).

\begin{figure}
\centering
\includegraphics[width=4.84649in,height=1.12963in,alt={P3161\#yIS1}]{media/6_IntroMRI/image70.pdf}\caption{Figura .: Probabilità di transizioni}
\end{figure}

In meccanica quantistica, le due probabilità si esprimono come:

\[W_{+ \rightarrow -} = \left\langle + \right|\widehat{\mu}\left| - \right\rangle\]

\[W_{- \rightarrow +} = \left\langle - \right|\widehat{\mu}\left| + \right\rangle\]

Queste quantità possono essere stimate a patto di conoscere l'hamiltoniano \(\widehat{H}\) del sistema. Dato il gran numero di particelle presente nel volumetto elementare, la trattazione con la meccanica quantistica, attraverso l'operatore hamiltoniano, è molto complessa. Per tale ragione si ricorre alla statistica di Boltzmann.

All'equilibrio termodinamico, il numero delle transizioni dallo stato \(\left| + \right\rangle\) allo stato \(\left| - \right\rangle\) deve essere uguale al numero delle transizioni da \(\left| - \right\rangle\) a \(\left| + \right\rangle\), in quanto l'energia del sistema si conserva. Si instaura così un equilibrio dinamico nel sistema. Per tale motivo è possibile scrivere l'uguaglianza:

\[N^{+}W_{+ \rightarrow -} = N^{-}W_{- \rightarrow +}\]

Dove \(N^{+}W_{+ \rightarrow -}\) è il numero dei protoni che dallo stato \(\left| + \right\rangle\) transitano allo stato \(\left| - \right\rangle\) e \(N^{-}W_{- \rightarrow +}\) è il numero dei protoni che eseguono la transizione opposta. Le transizioni devono essere tali da mantenere la popolazione degli spin costante.

Si scrive l'equazione all'equilibrio termodinamico come:

\[N^{+}W_{+ \rightarrow -} = N^{-}W_{- \rightarrow +} \Leftrightarrow \frac{W_{+ \rightarrow -}}{W_{- \rightarrow +}} = \frac{N^{-}}{N^{+}}\]

Per la statistica di Boltzmann, il rapporto tra le due probabilità è dato dal rapporto degli esponenziali:

\[\frac{W_{+ \rightarrow -}}{W_{- \rightarrow +}} = \frac{\exp\left( \frac{E^{+}}{k_{B}T} \right)}{\exp\left( \frac{E^{-}}{k_{B}T} \right)} = \exp\left( \frac{E^{+} - E^{-}}{k_{B}T} \right)\]

Sebbene non sia possibile valutare numericamente le due probabilità, è possibile valutarne il rapporto mediante la meccanica statistica.

Si scrive l'equazione differenziale che collega l'evoluzione temporale del numero di protoni che si trovano nello stato \(\left| + \right\rangle\). La variazione nel tempo del numero di spin nello stato energetico \(\left| + \right\rangle\) dipende dal numero di protoni che dallo stato \(\left| - \right\rangle\) passano allo stato \(\left| + \right\rangle\), a cui va sottratto il numero di protoni che transitano dallo stato \(\left| + \right\rangle\) a quello \(\left| - \right\rangle\) nell'intervallo temporale infinitesimo \(dt\):

\[\frac{dN^{+}}{dt} = N^{-}W_{- \rightarrow +} - N^{+}W_{+ \rightarrow -}\]

Da questa espressione si evince che le quantità \(W_{- \rightarrow +}\) e \(W_{+ \rightarrow -}\) sono dimensionalmente omogenee con l'inverso di un tempo:

\[\left\lbrack W_{- \rightarrow +} \right\rbrack = \left\lbrack \frac{1}{s} \right\rbrack\]

In altre parole, \(W_{- \rightarrow +}\) e \(W_{+ \rightarrow -}\) sono delle probabilità per unità di tempo.

Analogamente per la popolazione di spin nello stato \(\left| - \right\rangle\): la variazione del numero di spin nell'unità di tempo è data dalla popolazione di spin nello stato \(\left| + \right\rangle\) che transita nello stato \(\left| - \right\rangle\) a cui va sottratto il numero di spin che esegue la transizione opposta:

\[\frac{dN^{-}}{dt} = N^{+}W_{+ \rightarrow -} - N^{-}W_{- \rightarrow +}\]

Si scrivono le equazioni in termini di differenza di popolazioni \(\mathrm{\Delta}N\):

\[\mathrm{\Delta}N = N^{+} - N^{-}\]

Il parametro \(\mathrm{\Delta}N\) è fondamentale poiché la magnetizzazione macroscopica del volumetto elementare \(\overset{\underline{}}{M}\) è proporzionale alla differenza di popolazione di spin nello stato \(\left| + \right\rangle\) rispetto a quelle nello stato \(\left| - \right\rangle\):

\[\left| \overset{\underline{}}{M} \right| \propto \mathrm{\Delta}N = N^{+} - N^{-}\]

La magnetizzazione è, in ultima analisi, legata al netto di spin paralleli al campo magnetico principale, rispetto a quelli antiparalleli. Da ciò si evince la convenienza nel ragionare in termini di evoluzione della differenza di popolazione, piuttosto che mediante la variazione delle singole popolazioni.

Dalle due equazioni differenziali, che descrivono l'evoluzione temporale di una sola popolazione, sottraendo membro a membro si ha:

\[\left\{ \begin{matrix}
\frac{dN^{+}}{dt} = N^{-}W_{- \rightarrow +} - N^{+}W_{+ \rightarrow -} \\
\frac{dN^{-}}{dt} = N^{+}W_{+ \rightarrow -} - N^{-}W_{- \rightarrow +}
\end{matrix} \right.\ \]

\[\frac{d}{dt}\left( N^{+} - N^{-} \right) = N^{-}W_{- \rightarrow +} - N^{+}W_{+ \rightarrow -} - N^{+}W_{+ \rightarrow -} + N^{-}W_{- \rightarrow +}\]

Da cui:

\[\frac{d\mathrm{\Delta}N}{dt} = 2N^{-}W_{- \rightarrow +} - 2N^{+}W_{+ \rightarrow -}\]

Sia \(N\) il numero totale degli spin contenuto nel volumetto elementare. Siccome il volume non scambia materia con l'esterno, il numero degli spin è costante:

\[N = N^{+} + N^{-} = const\]

Per definizione:

\[\mathrm{\Delta}N = N^{+} - N^{-}\]

Sommando membro a membro sia ha:

\[N + \mathrm{\Delta}N = 2N^{+} \Leftrightarrow N^{+} = \frac{1}{2}(N + \mathrm{\Delta}N)\]

Invece, sottraendo membro a membro, si ha:

\[N - \mathrm{\Delta}N = 2N^{+} \Leftrightarrow N^{-} = \frac{1}{2}(N - \mathrm{\Delta}N)\]

L'equazione differenziale:

\[\frac{d}{dt}\left( N^{+} - N^{-} \right) = 2N^{-}W_{- \rightarrow +} - 2N^{+}W_{+ \rightarrow -}\]

Può essere scritta come:

\[\frac{d\mathrm{\Delta}N}{dt} = 2\frac{1}{2}(N - \mathrm{\Delta}N)W_{- \rightarrow +} - 2\frac{1}{2}(N + \mathrm{\Delta}N)W_{+ \rightarrow -} = (N - \mathrm{\Delta}N)W_{- \rightarrow +} - (N + \mathrm{\Delta}N)W_{+ \rightarrow -}\]

Raccogliendo \(N\) e \(\mathrm{\Delta}N\) al secondo membro si ha:

\[\frac{d\mathrm{\Delta}N}{dt} = N\left( W_{- \rightarrow +} - W_{+ \rightarrow -} \right) - \mathrm{\Delta}N\left( W_{- \rightarrow +} + W_{+ \rightarrow -} \right)\]

Queste equazioni devono essere sempre valide poiché non sono state proposte ipotesi particolari semplificative. Se le equazioni sono sempre valide, lo sono anche all'equilibrio termodinamico, condizione in cui non vi è nessuna variazione di \(\mathrm{\Delta}N\) dato che le due popolazioni presentano lo stesso numero di spin, nonostante la variazione della configurazione del sistema.

Dal punto di vista matematico, all'equilibrio termodinamico, risulta:

\[\frac{d\mathrm{\Delta}N}{dt} = 0\]

Dunque, \(\mathrm{\Delta}N = \mathrm{\Delta}N_{0} = const\). Siano \(N_{0}^{+}\) il numero di spin nello stato \(\left| + \right\rangle\) e \(N_{0}^{-}\) il numero di spin nello stato \(\left| - \right\rangle\) all'equilibrio termodinamico, risulta:

\[\mathrm{\Delta}N_{0} = N_{0}^{+} - N_{o}^{-}\]

\(\mathrm{\Delta}N_{0}\) è la differenza di spin nello stato \(\left| + \right\rangle\) rispetto a quelli nello stato \(\left| - \right\rangle\) all'equilibrio termodinamico; ciò produce la magnetizzazione macroscopica \(\overset{\underline{}}{M}\).

All'equilibrio termodinamico, l'equazione:

\[\frac{d\mathrm{\Delta}N}{dt} = N\left( W_{- \rightarrow +} - W_{+ \rightarrow -} \right) - \mathrm{\Delta}N\left( W_{- \rightarrow +} + W_{+ \rightarrow -} \right)\]

Diventa:

\[N\left( W_{- \rightarrow +} - W_{+ \rightarrow -} \right) - \mathrm{\Delta}N_{0}\left( W_{- \rightarrow +} + W_{+ \rightarrow -} \right) = 0\]

Si ricava \(\mathrm{\Delta}N_{0}\):

\[\mathrm{\Delta}N_{0} = N\frac{W_{- \rightarrow +} - W_{+ \rightarrow -}}{W_{- \rightarrow +} + W_{+ \rightarrow -}}\]

Sapendo che il valore all'equilibrio termodinamico del netto di spin è \(\mathrm{\Delta}N_{0}\), è possibile ricavare l'andamento temporale di \(\mathrm{\Delta}N\). Infatti, siccome \(\mathrm{\Delta}N_{0}\) è costante, è possibile sottrarre la sua derivata al primo membro dell'equazione:

\[\frac{d\mathrm{\Delta}N}{dt} = N\left( W_{- \rightarrow +} - W_{+ \rightarrow -} \right) - \mathrm{\Delta}N\left( W_{- \rightarrow +} + W_{+ \rightarrow -} \right)\]

Ottenendo:

\[\frac{d}{dt}\left( \mathrm{\Delta}N - \mathrm{\Delta}N_{0} \right) = N\left( W_{- \rightarrow +} - W_{+ \rightarrow -} \right) - \mathrm{\Delta}N\left( W_{- \rightarrow +} + W_{+ \rightarrow -} \right)\]

Al secondo membro si raccoglie il termine \(\left( W_{- \rightarrow +} + W_{+ \rightarrow -} \right)\), ottenendo:

\[\frac{d}{dt}\left( \mathrm{\Delta}N - \mathrm{\Delta}N_{0} \right) = \left( N\frac{W_{- \rightarrow +} - W_{+ \rightarrow -}}{W_{- \rightarrow +} + W_{+ \rightarrow -}} - \mathrm{\Delta}N \right)\left( W_{- \rightarrow +} + W_{+ \rightarrow -} \right)\]

All'equilibrio termodinamico, si ha:

\[\mathrm{\Delta}N_{0} = N\frac{W_{- \rightarrow +} - W_{+ \rightarrow -}}{W_{- \rightarrow +} + W_{+ \rightarrow -}}\]

Per cui, l'equazione differenziale può essere scritta come:

\[\frac{d}{dt}\left( \mathrm{\Delta}N - \mathrm{\Delta}N_{0} \right) = \left( \mathrm{\Delta}N_{0} - \mathrm{\Delta}N \right)\left( W_{- \rightarrow +} + W_{+ \rightarrow -} \right)\]

Dato che \(W_{- \rightarrow +}\) e \(W_{+ \rightarrow -}\) sono omogenee con l'inverso di un tempo, si pone:

\[\frac{1}{T_{1}} = W_{- \rightarrow +} + W_{+ \rightarrow -}\]

Con questa posizione, l'equazione differenziale si può scrivere:

\[\frac{d}{dt}\left( \mathrm{\Delta}N - \mathrm{\Delta}N_{0} \right) = \frac{1}{T_{1}}\left( \mathrm{\Delta}N_{0} - \mathrm{\Delta}N \right)\]

Al fine di avere la stessa quantità, si raccoglie un segno meno al secondo membro:

\[\frac{d}{dt}\left( \mathrm{\Delta}N - \mathrm{\Delta}N_{0} \right) = - \frac{1}{T_{1}}\left( \mathrm{\Delta}N - \mathrm{\Delta}N_{0} \right)\]

L'evoluzione temporale di \(\mathrm{\Delta}N\) dipende dalla costante di tempo \(T_{1}\), data dalla somma delle probabilità di transizione da uno stato all'altro.

La relazione individuata è molto approssimata, poiché le probabilità non sono note a priori, ma permette di descrivere l'evoluzione temporale del netto degli spin nello stato parallelo nel tempo. Si osserva che la soluzione dell'equazione è di tipo esponenziale crescente, con costante di tempo \(T_{1}\), il valore di regime è \(\mathrm{\Delta}N_{0}\).

Generalizzando, in un sistema di spin immerso in un campo magnetico, la componente longitudinale del vettore di magnetizzazione tende a raggiungere il valore di regime, dato dalla legge di Curie, con una costante di tempo \(\tau = T_{1}\).

\begin{figure}
\centering
\includegraphics[width=2.2315in,height=1.68519in,alt={P3237\#yIS1}]{media/6_IntroMRI/image71.pdf}\caption{Figura .: Evoluzione temporale del netto di vettore di magnetizzazione per effetto del campo principale}
\end{figure}

Disattivando il campo magnetico principale dopo che il sistema ha raggiuto l'equilibrio termodinamico, la componente longitudinale del vettore di magnetizzazione si annulla. Il decadimento è di tipo esponenziale con costante di tempo sempre uguale a \(T_{1}\).

\begin{figure}
\centering
\includegraphics[width=3.31667in,height=2.56742in,alt={P3240\#yIS1}]{media/6_IntroMRI/image72.pdf}\caption{Figura .: Evoluzione del vettore di magnetizzazione alla rimozione del campo principale}
\end{figure}

Il tempo \(T_{1}\) interessa gli scambi tra i vari spin contenuti nel materiale ed è legato alla componente longitudinale, lungo \({\widehat{i}}_{z}\). L'equazione di Bloch \(dM_{z}/dt = 0\) viene corretta introducendo l'andamento temporale, dettato dalla costante di tempo \(T_{1}\):

\[\frac{dM_{z}}{dt} = \frac{1}{T_{1}}\left( M_{0} - M_{z}\  \right)\]

Tale equazione è detta I equazione di Bloch. Inoltre, il tempo \(T_{1}\) è detto tempo di rilassamento longitudinale o spin-reticolo o, in letteratura anglosassone, \emph{spin-lattice}. Dal nome si evince che il tempo \(T_{1}\) tiene conto delle iterazioni di uno spin a contatto con l'intero reticolo, visto come un serbatoio termico, in cui è inserito. Generalmente, \(T_{1}\) è maggiore nei liquidi rispetto ai solidi.

La soluzione della prima equazione di Bloch è del tipo:

\[M_{z}(t) = M_{z}(0)\exp\left( - \frac{t}{T_{1}} \right) + M_{0}\left\lbrack 1 - \exp\left( - \frac{t}{T_{1}} \right) \right\rbrack\]

L'equazione di Bloch è fondamentale per studiare come il vettore di magnetizzazione raggiunga nuovamente l'equilibrio termodinamico, a valle di una perturbazione.

La costante di tempo \(T_{1}\) è un parametro essenziale per la caratterizzazione dei tessuti, così da eseguire l'\emph{imaging}. Tessuti diversi sono caratterizzati da un tempo di rilassamento longitudinale specifico.

\subsubsection{Iterazioni spin-spin}\label{iterazioni-spin-spin}

Un importante meccanismo, che determina il decadimento delle componenti trasversali della magnetizzazione, è la generazioni di campi locali prodotti dagli spin. Questi campi si combinano col campo principale applicando, modificando localmente il campo in cui sono immersi i campi vicini.

Si considera un singolo spin, il campo prodotto da questo spin è indicato con \({\overset{\underline{}}{B}}_{loc}\). Gli altri spin genereranno dei campi diversi. Tutte le componenti verticali possiedono intensità diverse, non predicibili in quanto non è possibile ricostruire il moto di ogni singolo spin del volumetto considerato.

Per studiare il campo locale visto da uno spin si utilizza il valor quadratico medio:

\[B_{RMS} = \sqrt{\left\langle \left| {\overset{\underline{}}{B}}_{loc} \right| \right\rangle^{2}}\]

Dunque, vi è una componente magnetica del campo valor quadratico medio lungo \({\widehat{i}}_{z}\). Questa componente si sovrappone al campo magnetico principale \(B_{0}{\widehat{i}}_{z}\), applicato dall'esterno; ciò determina una variazione della frequenza con cui i singoli spin precedono intorno all'asse \({\widehat{i}}_{z}\).

Data l'agitazione termina, gli spin sono in moto relativo, quindi, la loro posizione non è statica ma dinamica. Ne consegue che il campo locale prodotto da uno spin varia nel tempo, poiché, appunto, varia la distribuzione locale degli spin. Per tale motivo il campo valor quadratico medio è una funzione del tempo:

\[B_{RMS}(t) = \sqrt{\left\langle \left| {\overset{\underline{}}{B}}_{loc}(t) \right| \right\rangle^{2}}\]

Sia \(t_{1}\) un primo istante di osservazione; il campo locale di uno spin vale \({\overset{\underline{}}{B}}_{loc}\left( t_{1} \right)\). In un secondo istante temporale \(t_{2}\), successivo al primo, il campo locale sarà diverso, a causa del modi termici, per cui:

\[{\overset{\underline{}}{B}}_{loc}\left( t_{1} \right) \neq {\overset{\underline{}}{B}}_{loc}\left( t_{2} \right)\]

Questo risultato è dovuto alla diversa configurazione degli spin tra gli istanti \(t_{1}\) e \(t_{2}\).

Per ogni spin, la frequenza di precessione intorno all'asse \({\widehat{i}}_{z}\) è diversa dagli altri ed è data da:

\[\mathrm{\Delta}\omega(t) = \gamma{\overset{\underline{}}{B}}_{loc}(t)\]

Dove \(\mathrm{\Delta}\omega = \omega_{0} - \omega_{loc}(t)\). La differenza tra le frequenze di precessione è, dunque, una funzione del tempo. Dati due istanti temporale, risulta:

\[\mathrm{\Delta}\omega\left( t_{1} \right) = \gamma{\overset{\underline{}}{B}}_{loc}\left( t_{1} \right) \neq \mathrm{\Delta}\omega\left( t_{2} \right) = \gamma{\overset{\underline{}}{B}}_{loc}\left( t_{2} \right)\]

Ovviamente \(\mathrm{\Delta}\omega\left( t_{1} \right) \neq \mathrm{\Delta}\omega\left( t_{2} \right)\) poiché i campi locali nei due istanti temporali sono diversi.

Si vuole studiare il fenomeno del campo locale nel sistema di riferimento locale. Tralasciando la frequenza principale \(\omega_{0}\), restano solamente le variazioni \(\mathrm{\Delta}\omega\) dovute alle iterazioni tra uno spin e i protoni locali. Queste variazioni, nel sistema rotante danno luogo a delle variazioni di fase:

\[\vartheta = \int_{t_{0}}^{t_{1}}{\mathrm{\Delta}\omega\left( t' \right)dt'}\]

Può capitare che in un istante la fase cresca più rapidamente, altre volte meno rapidamente. In ogni caso la rotazione degli spin avviene sempre, complessivamente, in senso orario. Nel sistema rotante, in particolare, le variazioni di fase possono essere sia positive che negative; di conseguenza, la proiezione dello spin sul piano \(x' - y'\) non è fissa ma varia nel tempo, secondo la sua variazione di fase legate al campo locale.

\begin{figure}
\centering
\includegraphics[width=4.22917in,height=1.49311in,alt={P3268\#yIS1}]{media/6_IntroMRI/image73.pdf}\caption{Figura .: Variazione della fase a causa del campo locale}
\end{figure}

La proiezione di tutti gli spine nel piano \(x' - y'\) è del tutto casuale, poiché casuale è il campo locale che ogni spin percepisce. È possibile, quindi, concludere che le proiezioni degli spin sono tutte distribuite uniformemente nel piano \(x' - y'\). Di conseguenza, l'effetto medio, percepito a livello macroscopico nel volumetto contenente un numero di Avogadro di particelle, è l'annullamento delle componenti trasverse del vettore di magnetizzazione. Infatti, le componente trasversa, essendo casuali, si elidono a vicenda; ovvero, la risultate media della componente trasversa del vettore di magnetizzazione è nulla, dato l'elevato numero di spin considerato.

\begin{figure}
\centering
\includegraphics[width=4.40341in,height=3.58333in,alt={P3271\#yIS1}]{media/6_IntroMRI/image74.pdf}\caption{Figura .: Eliminazione delle componenti trasverse degli spin causa distribuzione uniforme}
\end{figure}

Le componenti longitudinali del campo magnetico prodotto da uno spin induce uno sfasamento e defasamento degli spin vicini, poiché procedono con frequenze diverse e casuali. Ne discende che la somma delle componenti trasversali degli spin mediamente è nulla. Da notare che lo sfasamento è legato alla fase diversa per ogni singolo spin.

Il tempo con cui la componente trasversale media va zero può essere stimato, studiando le iterazioni tra due spin. Si suppone che a un certo instante di tempo le fasi dei due spin siano le stesse.

In generale, la differenza di fase è data da:

\[\mathrm{\Delta}\phi = \phi_{1} - \phi_{2} = \left( \phi_{0}^{1} - \mathrm{\Delta}\omega_{{loc}_{1}}t \right) - \left( \phi_{0}^{2} - \mathrm{\Delta}\omega_{{loc}_{2}}t \right)\]

Dove \(\mathrm{\Delta}\omega_{{loc}_{1}}\) e \(\mathrm{\Delta}\omega_{{loc}_{2}}\) sono le frequenze di precessione relative rispetto a quelle del campo magnetico principale, legati ai campi locali, rispettivamente del primo spin, \(B_{{loc}_{1}}\) e del secondo spin \(B_{{loc}_{2}}\).

Se le fasi inziali dei due spin sono le stesse, ovvero \(\phi_{0}^{1} = \phi_{0}^{2}\), la differenza di fase si scrive come:

\[\mathrm{\Delta}\phi = - \left( \mathrm{\Delta}\omega_{{loc}_{1}} - \mathrm{\Delta}\omega_{{loc}_{2}} \right)t\]

Le frequenze di precessione relative rispetto a quelle del campo magnetico principale possono essere scritte in funzione del campo locale:

\[\mathrm{\Delta}\omega_{{loc}_{i}} = \gamma B_{{loc}_{i}},\ \ i = 1,2\]

Per cui la differenza di fase può essere scritta come:

\[\mathrm{\Delta}\phi = - \left( \gamma B_{{loc}_{1}} - \gamma B_{{loc}_{2}} \right)t\]

Si indica la differenza di campi locali come:

\[\mathrm{\Delta}B_{loc} = B_{{loc}_{1}} - B_{{loc}_{2}}\]

Si ottiene:

\[\mathrm{\Delta}\phi = - \gamma\mathrm{\Delta}B_{loc}t\]

Per avere una risultate netta nulla nel piano trasverso, le due proiezioni sul piano \(x' - y'\) devono essere uguali e oppure, dunque, la differenza di fase deve essere \(\mathrm{\Delta}\phi = \pi\). È, ora, possibile ottenere il tempo con cui due spin annullano le proprie componenti trasversali. Si indica tale tempo con \(\tau\):

\[\mathrm{\Delta}\phi = \pi = - \gamma\mathrm{\Delta}B_{loc}\tau\]

Da cui:

\[\tau = - \frac{\pi}{\gamma\mathrm{\Delta}B_{loc}}\]

Le differenze dei campi indotti hanno un'intensità proporzionale al fattore \(3\mu/r^{3}\):

\[B_{loc} \propto \frac{3\mu}{r^{3}}\]

Il tempo di defasamento, al limite (ignorando le costanti), è dato da:

\[\tau\sim\frac{\pi}{\gamma\frac{\mu}{r^{3}}} = \frac{\pi r^{3}}{\mu\gamma}\]

Da questa relazione, noti i valori medi del momento magnetico e della distanza interatomica, è possibile ottenere una stima del tempo di defasamento più o meno concorde ai dati sperimentali. La stima di basa su una descrizione non esatta del fenomeno, in quanto fondata della meccanica classica; tuttavia, è usata poiché permette di ottenere una buona stima dei risultati sperimentali.

Il tempo di defasamento \(\tau\) è indicato con \(T_{2}\) ed è detto tempo di rilassamento traversarsele. \(T_{2}\) rappresenta la costante di tempo con cui la componente trasversale del vettore di magnetizzazione, \({\overset{\underline{}}{M}}_{\bot}\), tende a zero:

\[M_{x} \rightarrow 0,\ \ M_{y} \rightarrow 0\]

\(T_{2}\) è generalmente minore nei solidi rispetto ai liquidi.

\subsection{Equazioni di Bloch}\label{equazioni-di-bloch}

All'equazione che descrive la magnetizzazione macroscopica di un volumetto elementare:

\[\frac{d\overset{\underline{}}{M}}{dt} = \gamma\overset{\underline{}}{M} \times {\overset{\underline{}}{B}}_{0}\]

vanno aggiunti due termini che tengono conto dei fenomeni di rilassamento. Il primo termine è legato agli scambi energetici tra uno spin e il reticolo, visto come un serbatoio termico, il secondo legato alle iterazioni locali tra spin:

\[\frac{d\overset{\underline{}}{M}}{dt} = \gamma\overset{\underline{}}{M} \times {\overset{\underline{}}{B}}_{0} + \frac{1}{T_{1}}\left( M_{0} - M_{z}\  \right){\widehat{i}}_{z} - \frac{1}{T_{2}}{\overset{\underline{}}{M}}_{\bot}\]

La precedente equazione è detta di Bloch. Il segno meno nel termine \({\overset{\underline{}}{M}}_{\bot}/T_{2}\) è dovuto al tendere a zero della componente traversale del vettore di magnetizzazione con un tempo \(T_{2}\).

Le equazioni di Bloch forniscono una descrizione fenomenologica del vettore di magnetizzazione, poiché non descrive i fenomeni fisici ma forniscono dei risultati concordi alle osservazioni sperimentali. Inoltre, le equazioni di Bloch non possono essere dedotte utilizzando la meccanica quantistica.

In letteratura anglosassone si definisce rilassività o relaxivity longitudinale e trasversale come:

\[R_{1} = \frac{1}{T_{1}},\ \ R_{2} = \frac{1}{T_{2}}\]

L'equazione vettoriale di Bloch può essere anche scritta in termini di relaxivity

\[\frac{d\overset{\underline{}}{M}}{dt} = \gamma\overset{\underline{}}{M} \times {\overset{\underline{}}{B}}_{0} + R_{1}\left( M_{0} - M_{z}\  \right){\widehat{i}}_{z} - R_{2}{\overset{\underline{}}{M}}_{\bot}\]

I tempi di rilassamento permettono di caratterizzare tessuti diversi e di identificare lo stato di salute di un tessuto stesso.

I tempi di rilassamento sono abbastanza variabili a seconda del tessuto; in particolare, tra i vari tessuti molli vi è una grande variabilità dei tempi di rilassamento. Ciò consente di discriminare con gradi di grigio diversi i tempi di rilassamento i diversi tessuti.

Le differenze di tempo di rilassamento longitudinale \(T_{1}\) non possono essere determinate sulla base della teoria quantistica, a causa dell'elevata complessità biochimiche dei tessuti umani.

In media, la composizione dei vari tessuti è simile tra i diversi individui, quindi, in un paziente i tempo di rilassamento sono grossomodo simili a quelli medi. Mediante delle opportune sequenze di acquisizione è possibile ottenere informazioni sui tempi di rilassamento, caratterizzando così il tessuto.

\begin{longtable}[]{@{}
  >{\centering\arraybackslash}p{(\linewidth - 4\tabcolsep) * \real{0.3333}}
  >{\centering\arraybackslash}p{(\linewidth - 4\tabcolsep) * \real{0.3333}}
  >{\centering\arraybackslash}p{(\linewidth - 4\tabcolsep) * \real{0.3334}}@{}}
\caption{Tabella 6.1: Tempi di rilassamento di vari tessuti molli con campo principale di \(1.5\ T\) e temperatura di \(37\ {^\circ}C\)}\tabularnewline
\toprule\noalign{}
\begin{minipage}[b]{\linewidth}\centering
Tessuto
\end{minipage} & \begin{minipage}[b]{\linewidth}\centering
Tempo di rilassamento longitudinale \(T_{1}\)
\end{minipage} & \begin{minipage}[b]{\linewidth}\centering
Tempo di rilassamento trasverale \(T_{2}\)
\end{minipage} \\
\midrule\noalign{}
\endfirsthead
\toprule\noalign{}
\begin{minipage}[b]{\linewidth}\centering
Tessuto
\end{minipage} & \begin{minipage}[b]{\linewidth}\centering
Tempo di rilassamento longitudinale \(T_{1}\)
\end{minipage} & \begin{minipage}[b]{\linewidth}\centering
Tempo di rilassamento trasverale \(T_{2}\)
\end{minipage} \\
\midrule\noalign{}
\endhead
\bottomrule\noalign{}
\endlastfoot
Materia grigia & \(\sim 950\ ms\) & \(\sim 100\ ms\) \\
Tessuto muscolare & \(\sim 900\ ms\) & \(\sim 50\ ms\) \\
Grasso & \(\sim 250\ ms\) & \(\sim 60\ ms\) \\
Sangue & \(\sim 1200\ ms\) & \(\sim 100\ ms\) \\
Materia bianca & \(\sim 600\ ms\) & \(\sim 80ms\) \\
Fluido cerebrospinale (CSF) & \(\sim 4500\ ms\) & \(\sim 2200\ ms\) \\
\end{longtable}

Generalmente risulta che il tempo di rilassamento longitudinale è maggiore di quello trasversale, \(T_{1} > T_{2}\), quindi, l'evoluzione longitudinale evolve più lentamente di quella trasversale. Analogamente per le relaxivity risulta \(R_{2} > R_{1}\).

\subsubsection{Risoluzione dell'equazione di Bloch}\label{risoluzione-dellequazione-di-bloch}

Si è visto che l'equazione vettoriale di Bloch, comprendente i tempi di rilassamento longitudinale e traversale è:

\[\frac{d\overset{\underline{}}{M}}{dt} = \gamma\overset{\underline{}}{M} \times {\overset{\underline{}}{B}}_{0} + \frac{1}{T_{1}}\left( M_{0} - M_{z}\  \right){\widehat{i}}_{z} - \frac{1}{T_{2}}{\overset{\underline{}}{M}}_{\bot}\]

Questa equazione descrive da un punto di vista classico l'andamento del vettore di magnetizzazione \(\overset{\underline{}}{M}\) nel tempo.

Si suppone di forzare il sistema di spin mediante un campo magnetico principale \(B_{0}\) diretto lungo \({\widehat{i}}_{z}\). Si scrive il prodotto vettorale tra il vettore di magnetizzazione e il campo esterno applicato:

\[\overset{\underline{}}{M} \times {\overset{\underline{}}{B}}_{0} = \left| \begin{matrix}
{\widehat{i}}_{x} & {\widehat{i}}_{y} & {\widehat{i}}_{z} \\
M_{x} & M_{y} & M_{z} \\
0 & 0 & B_{0}
\end{matrix} \right| = B_{0}\left( M_{y}{\widehat{i}}_{x} - M_{x}{\widehat{i}}_{y} \right)\]

Si scrive l'equazione di Bloch esplicitando le componenti dei vettori coinvolti:

\[\frac{d}{dt}\left( M_{x}{\widehat{i}}_{x} + M_{y}{\widehat{i}}_{y} + M_{z}{\widehat{i}}_{z} \right) = \gamma B_{0}\left( M_{y}{\widehat{i}}_{x} - M_{x}{\widehat{i}}_{y} \right) + \frac{1}{T_{1}}\left( M_{0} - M_{z}\  \right){\widehat{i}}_{z} - \frac{1}{T_{2}}\left( M_{x}{\widehat{i}}_{x} + M_{y}{\widehat{i}}_{y} \right)\]

Per effetto del campo magnetico principale, il vettore di magnetizzazione si sposta da una configurazione iniziale \(\overset{\underline{}}{M}\left( t_{0} \right)\) a quella finale, descritta dall'equazione di Bloch.

Si proietta quest'ultima lungo gli assi:

\[\left\{ \begin{matrix}
\frac{dM_{x}}{dt} = \gamma B_{0}M_{y} - \frac{1}{T_{2}}M_{x} \\
\frac{dM_{y}}{dt} = - \gamma B_{0}M_{x} - \frac{1}{T_{2}}M_{y} \\
\frac{dM_{z}}{dt} = \frac{1}{T_{1}}\left( M_{0} - M_{z}\  \right)
\end{matrix} \right.\ \]

Com'è facile osservare, l'evoluzione della componente longitudinale, \(M_{z}\), è indipendente da quelle trasversali \(M_{x}\) e \(M_{y}\); infatti, le relative equazioni non sono accoppiate dopo la proiezione sugli assi. Le componenti trasversa sono, invece, legate tra loro.

Si risolve l'equazione relativa alla componente longitudinale del vettore \(\overset{\underline{}}{M}\):

\[\frac{dM_{z}}{dt} = \frac{1}{T_{1}}\left( M_{0} - M_{z}\  \right)\]

Si passa all'equazione omogenea associata:

\[\frac{dM_{z}}{dt} = \frac{1}{T_{1}}M_{z}\]

La soluzione di questa equazione è del tipo:

\[M_{z}(t) = k\exp\left( - \frac{t}{T_{1}} \right)\]

Dove \(k\) è una costante dipendente dalle condizioni iniziali del sistema.

Alla soluzione dell'omogenea, va associata una soluzione particolare. Siccome il forzamento è costante una possibile soluzione particolare è del tipo:

\[M_{z}(t) = c = const\]

Si sostituisce tale soluzione nell'equazione differenziale:

\[\left. \ \frac{dM_{z}}{dt} \right|_{M_{z} = c} = \frac{1}{T_{1}}\left. \ \left( M_{0} - M_{z}\  \right) \right|_{M_{z} = c} \Leftrightarrow \frac{dc}{dt} = \frac{1}{T_{1}}\left( M_{0} - c \right)\]

La derivata di una costante è nulla, per cui:

\[\frac{1}{T_{1}}\left( M_{0} - c \right) = 0 \Leftrightarrow c = M_{0}\]

L'integrale generale dell'equazione differenziale per la componente longitudinale è:

\[M_{z}(t) = k\exp\left( - \frac{t}{T_{1}} \right) + M_{0}\]

Si suppone che il vettore di magnetizzazione sia noto all'istante \(t = t_{0}\). Applicando la condizione iniziale è possibile determinare il valore di \(k\):

\[M_{z}\left( t_{0} \right) = k\exp\left( - \frac{t_{0}}{T_{1}} \right) + M_{0} \Leftrightarrow k = \left\lbrack M_{z}\left( t_{0} \right) - M_{0} \right\rbrack\exp\left( \frac{t_{0}}{T_{1}} \right)\]

La soluzione è, dunque:

\[M_{z}(t) = \left\lbrack M_{z}\left( t_{0} \right) - M_{0} \right\rbrack\exp\left( \frac{t_{0}}{T_{1}} \right)\exp\left( - \frac{t}{T_{1}} \right) + M_{0}\]

Esplicitando la componente transitoria e quella di regime, si ha:

\[M_{z}(t) = M_{0}\left\lbrack 1 - \exp\left( \frac{t_{0} - t}{T_{1}} \right) \right\rbrack + M_{z}\left( t_{0} \right)\exp\left( \frac{t_{0} - t}{T_{1}} \right)\]

Ovviamente, per tempi sufficientemente lunghi:

\[M_{z}\left( t_{0} \right)\exp\left( \frac{t_{0} - t}{T_{1}} \right) \rightarrow 0,\ \ t \rightarrow \infty\]

Se \(t_{0} = 0\), risulta:

\[M_{z}(t) = M_{0}\left\lbrack 1 - \exp\left( - \frac{t}{T_{1}} \right) \right\rbrack + M_{z}(0)\exp\left( - \frac{t}{T_{1}} \right)\]

Supponendo che il valore iniziale della magnetizzazione sia nullo, la soluzione diventa:

\[M_{z}(t) = M_{0}\left\lbrack 1 - \exp\left( - \frac{t}{T_{1}} \right) \right\rbrack\]

Applicando il campo magnetico principale, da una condizione di equilibrio termodinamico, la componente longitudinale tende al valore di regime con andamento esponenziale e costante di tempo \(T_{1}\). Il valore di regime è dato dall'equazione di Curie:

\[M_{0} \simeq \ \frac{N}{V}\frac{\gamma^{2}\hslash^{2}}{4k_{B}T}B_{0}\]

\begin{figure}
\centering
\includegraphics[width=4.32769in,height=3.10307in,alt={P3386\#yIS1}]{media/6_IntroMRI/image75.pdf}\caption{Tabella 6.2: Andamento della soluzione longitudinale dell'equazione di Bloch}
\end{figure}

È possibile ritenere che la componente longitudinale raggiunga il regime dopo un tempo uguale a \(4 \div 5\) volte la costante di tempo \(T_{1}\). Generalmente il tempo di rilassamento longitudinale è dell'ordine del secondo, per cui si ritiene che il transitorio abbia una durata di circa \(5\ s\).

Nella pratica, una volta immesso il paziente nel gantry della risonanza magnetica è necessario aspettare un giusto tempo prima di eseguire l'imaging, affinché gli spin dei protoni contenuti nel corpo del paziente raggiungano l'equilibrio termodinamico. L'esame, quindi, non può iniziale istantaneamente ma è necessario che il vettore di magnetizzazione del paziente raggiunga l'equilibrio termodinamico col campo. In caso contrario, si ottengono delle immagini con risultati falsati. In un tempo di \(5 \div 10\ s\) il tecnico radiologo entra nella sala di comando.

Per quanto riguarda le componenti trasversali, le equazioni che descrivono la loro evoluzione sono:

\[\left\{ \begin{matrix}
\frac{dM_{x}}{dt} = \gamma B_{0}M_{y} - \frac{1}{T_{2}}M_{x} \\
\frac{dM_{y}}{dt} = - \gamma B_{0}M_{x} - \frac{1}{T_{2}}M_{y}
\end{matrix} \right.\ \]

Dove \(\omega_{0} = \ \gamma B_{0}\); per cui il sistema può essere scritto come:

\[\left\{ \begin{matrix}
\frac{dM_{x}}{dt} = \omega_{0}M_{y} - \frac{1}{T_{2}}M_{x} \\
\frac{dM_{y}}{dt} = - \omega_{0}M_{x} - \frac{1}{T_{2}}M_{y}
\end{matrix} \right.\ \]

Si pone il sistema in forma matriciale:

\[\frac{d}{dt}\left( \begin{array}{r}
M_{x} \\
M_{y}
\end{array} \right) = \begin{pmatrix}
 - \frac{1}{T_{2}} & \omega_{0} \\
 - \omega_{0} & - \frac{1}{T_{2}}
\end{pmatrix}\left( \begin{array}{r}
M_{x} \\
M_{y}
\end{array} \right)\]

La soluzione di questa equazione è del tipo:

\[{\overset{\underline{}}{M}}_{\bot} = \overset{\underline{}}{k}\exp\left( \lambda\overset{\underline{}}{\overset{\underline{}}{I}}t \right)\]

Sostituendo tale equazione nell'equazione differenziale si ha:

\[\frac{d}{dt}\left\lbrack \overset{\underline{}}{k}\exp\left( \lambda\overset{\underline{}}{\overset{\underline{}}{I}}t \right) \right\rbrack = \begin{pmatrix}
 - \frac{1}{T_{2}} & \omega_{0} \\
 - \omega_{0} & - \frac{1}{T_{2}}
\end{pmatrix}\overset{\underline{}}{k}\exp\left( \lambda\overset{\underline{}}{\overset{\underline{}}{I}}t \right) \Leftrightarrow \lambda\overset{\underline{}}{\overset{\underline{}}{I}}\overset{\underline{}}{k}\exp\left( \lambda\overset{\underline{}}{\overset{\underline{}}{I}}t \right) = \begin{pmatrix}
 - \frac{1}{T_{2}} & \omega_{0} \\
 - \omega_{0} & - \frac{1}{T_{2}}
\end{pmatrix}\overset{\underline{}}{k}\exp\left( \lambda\overset{\underline{}}{\overset{\underline{}}{I}}t \right)\]

Moltiplicando per la matrice inversa a \(\overset{\underline{}}{k}\exp\left( \lambda\overset{\underline{}}{\overset{\underline{}}{I}}t \right)\) si ottiene:

\[\lambda\overset{\underline{}}{\overset{\underline{}}{I}} = \begin{pmatrix}
 - \frac{1}{T_{2}} & \omega_{0} \\
 - \omega_{0} & - \frac{1}{T_{2}}
\end{pmatrix} \Leftrightarrow \begin{pmatrix}
 - \frac{1}{T_{2}} - \lambda & \omega_{0} \\
 - \omega_{0} & - \frac{1}{T_{2}} - \lambda
\end{pmatrix} = \overset{\underline{}}{\overset{\underline{}}{0}}\]

Si pone il determinate della matrice individuata a zero, al fine di identificare i suoi autovettori. Conoscendo gli autovalori è possibile individuare anche il tipo di moto a cui sono soggette le componenti trasverse:

\[\det\begin{pmatrix}
 - \frac{1}{T_{2}} - \lambda & \omega_{0} \\
 - \omega_{0} & - \frac{1}{T_{2}} - \lambda
\end{pmatrix} = 0 \Leftrightarrow \left( - \frac{1}{T_{2}} - \lambda \right)\left( - \frac{1}{T_{2}} - \lambda \right) + \omega_{0}^{2} = 0 \Leftrightarrow \left( \frac{1}{T_{2}} + \lambda \right)^{2} + \omega_{0}^{2} = 0\]

Svolgendo i prodotti si ha:

\[\frac{1}{T_{2}^{2}} + \frac{2\lambda}{T_{2}} + \lambda^{2} + \omega_{0}^{2} = 0 \Leftrightarrow \lambda^{2} + \frac{2\lambda}{T_{2}} + \omega_{0}^{2} + \frac{1}{T_{2}^{2}} = 0\]

Si valuta il delta dell'equazione:

\[\frac{\Delta}{4} = \frac{1}{T_{2}^{2}} - \left( \omega_{0}^{2} + \frac{1}{T_{2}^{2}} \right) = \frac{1}{T_{2}^{2}} - \omega_{0}^{2} - \frac{1}{T_{2}^{2}} = - \omega_{0}^{2}\]

Il determinate è negativo per cui l'equazione non ammette soluzioni reali. Ne consegue che l'evoluzione delle componenti trasverse sono di tipo oscillanti smorzate.

Gli autovalori della matrice dei coefficienti sono:

\[\lambda_{1,2} = - \frac{1}{T_{2}} \pm j\omega_{0} \Leftrightarrow \lambda_{1} = - \frac{1}{T_{2}} - j\omega_{0},\ \ \lambda_{2} = - \frac{1}{T_{2}} + j\omega_{0}\]

Le soluzioni delle componenti trasverse sono del tipo:

\[\left\{ \begin{matrix}
M_{x}(t) = k_{1,x}\exp\left\lbrack \left( - \frac{1}{T_{2}} - j\omega_{0} \right)t \right\rbrack + k_{2,x}\exp\left\lbrack \left( - \frac{1}{T_{2}} + j\omega_{0} \right)t \right\rbrack \\
M_{y}(t) = k_{1,y}\exp\left\lbrack \left( - \frac{1}{T_{2}} - j\omega_{0} \right)t \right\rbrack + k_{2,y}\exp\left\lbrack \left( - \frac{1}{T_{2}} + j\omega_{0} \right)t \right\rbrack
\end{matrix} \right.\ \]

Per le proprietà degli esponenziali, è possibile scrivere:

\[\left\{ \begin{matrix}
M_{x}(t) = k_{1,x}\exp\left( - \frac{t}{T_{2}} \right)\exp\left( - j\omega_{0}t \right) + k_{2,x}\exp\left( - \frac{t}{T_{2}} \right)\exp\left( j\omega_{0}t \right) \\
M_{y}(t) = k_{1,y}\exp\left( - \frac{t}{T_{2}} \right)\exp\left( - j\omega_{0}t \right) + k_{2,y}\exp\left( - \frac{t}{T_{2}} \right)\exp\left( j\omega_{0}t \right)
\end{matrix} \right.\ \]

Raccogliendo il termine dipendente da \(T_{2}\), si ottiene:

\[\left\{ \begin{matrix}
M_{x}(t) = \left\lbrack k_{1,x}\exp\left( - j\omega_{0}t \right) + k_{2,x}\exp\left( j\omega_{0}t \right) \right\rbrack\exp\left( - \frac{t}{T_{2}} \right) \\
M_{y}(t) = \left\lbrack k_{1,y}\exp\left( - j\omega_{0}t \right) + k_{2,y}\exp\left( j\omega_{0}t \right) \right\rbrack\exp\left( - \frac{t}{T_{2}} \right)
\end{matrix} \right.\ \]

Per individuare le costanti di integrazione bisognerebbe sostituire le equazioni individuate nel sistema di equazioni differenziali per le componenti trasverse e applicare le condizioni al contorno.

Per semplificare la trattazione si nota che la dipendenza da \(T_{2}\) è espressa mediante un fattore esponenziale moltiplicativo, quindi, è possibile scrivere le soluzioni delle componenti trasversali come:

\[M_{x}(t) = m_{x}(t)\exp\left( - \frac{t}{T_{2}} \right),\ \ M_{y}(t) = m_{y}(t)\exp\left( - \frac{t}{T_{2}} \right)\]

Le due equazioni differenziali si scrivono come:

\[\left\{ \begin{matrix}
\frac{dM_{x}}{dt} = \omega_{0}M_{y} - \frac{1}{T_{2}}M_{x} \\
\frac{dM_{y}}{dt} = - \omega_{0}M_{x} - \frac{1}{T_{2}}M_{y}
\end{matrix} \right.\  \Leftrightarrow \left\{ \begin{matrix}
\frac{d}{dt}\left\lbrack m_{x}\exp\left( - \frac{t}{T_{2}} \right) \right\rbrack = \omega_{0}m_{y}\exp\left( - \frac{t}{T_{2}} \right) - \frac{1}{T_{2}}m_{x}\exp\left( - \frac{t}{T_{2}} \right) \\
\frac{d}{dt}\left\lbrack m_{y}\exp\left( - \frac{t}{T_{2}} \right) \right\rbrack = - \omega_{0}m_{x}\exp\left( - \frac{t}{T_{2}} \right) - \frac{1}{T_{2}}m_{y}\exp\left( - \frac{t}{T_{2}} \right)
\end{matrix} \right.\ \]

Svolgendo le derivate, si ha:

\[\left\{ \begin{matrix}
\frac{dm_{x}}{dt}\exp\left( - \frac{t}{T_{2}} \right) - \frac{1}{T_{2}}m_{x}\exp\left( - \frac{t}{T_{2}} \right) = \omega_{0}m_{y}\exp\left( - \frac{t}{T_{2}} \right) - \frac{1}{T_{2}}m_{x}\exp\left( - \frac{t}{T_{2}} \right) \\
\frac{dm_{y}}{dt}\exp\left( - \frac{t}{T_{2}} \right) - \frac{1}{T_{2}}m_{y}\exp\left( - \frac{t}{T_{2}} \right) = - \omega_{0}m_{x}\exp\left( - \frac{t}{T_{2}} \right) - \frac{1}{T_{2}}m_{y}\exp\left( - \frac{t}{T_{2}} \right)
\end{matrix} \right.\ \]

Semplificando i termini comuni al primo e al secondo membro e il termine esponenziale dipendente da \(T_{2}\), si ottiene un semplice sistema di equazioni differenziali, la cui soluzione è nota:

\[\left\{ \begin{matrix}
\frac{dm_{x}}{dt} = \omega_{0}m_{y} \\
\frac{dm_{y}}{dt} = - \omega_{0}m_{x}
\end{matrix} \right.\ \]

Si deriva la prima equazione rispetto al tempo:

\[\frac{d^{2}m_{x}}{dt} = \omega_{0}\frac{dm_{y}}{dt}\]

Sostituendo la seconda equazione, si ottiene:

\[\frac{d^{2}m_{x}}{dt} = - \omega_{0}^{2}m_{x}\]

L'equazione differenziale ha come soluzione:

\[m_{x} = C_{1}\exp\left( \lambda_{1}t \right) + C_{2}\exp\left( \lambda_{2}t \right) = C_{1}\exp\left( - j\omega_{0}t \right) + C_{2}\exp\left( j\omega_{0}t \right)\]

Nota \(m_{x}\) è possibile ricavare l'equazione per \(m_{y}\) dalla prima relazione del sistema:

\[\frac{dm_{x}}{dt} = \omega_{0}m_{y} \Leftrightarrow m_{y} = \frac{1}{\omega_{0}}\frac{dm_{x}}{dt} = \frac{1}{\omega_{0}}\frac{d}{dt}\left\lbrack C_{1}\exp\left( - j\omega_{0}t \right) + C_{2}\exp\left( j\omega_{0}t \right) \right\rbrack\]

Svolgendo l'operazione di derivata si ottiene:

\[m_{y} = \frac{1}{\omega_{0}}\left\lbrack - j\omega_{0}C_{1}\exp\left( - j\omega_{0}t \right) + j\omega_{0}C_{2}\exp\left( j\omega_{0}t \right) \right\rbrack = - jC_{1}\exp\left( - j\omega_{0}t \right) + jC_{2}\exp\left( j\omega_{0}t \right)\]

In definitiva, si è ottenuto:

\[\left\{ \begin{matrix}
m_{x} = C_{1}\exp\left( - j\omega_{0}t \right) + C_{2}\exp\left( j\omega_{0}t \right) \\
m_{y} = - jC_{1}\exp\left( - j\omega_{0}t \right) + jC_{2}\exp\left( j\omega_{0}t \right)
\end{matrix} \right.\ \]

Tornando alle componenti trasverse si ha:

\[\left\{ \begin{matrix}
M_{x}(t) = \left\lbrack C_{1}\exp\left( - j\omega_{0}t \right) + C_{2}\exp\left( j\omega_{0}t \right) \right\rbrack\exp\left( - \frac{t}{T_{2}} \right) \\
M_{y}(t) = \left\lbrack - jC_{1}\exp\left( - j\omega_{0}t \right) + jC_{2}\exp\left( j\omega_{0}t \right) \right\rbrack\exp\left( - \frac{t}{T_{2}} \right)
\end{matrix} \right.\ \]

Si suppone che il vettore di magnetizzazione sia noto all'istante \(t_{0}\). In questo modo è possibile individuare le costanti di integrazione:

\[\left\{ \begin{matrix}
M_{x}\left( t_{0} \right) = \left\lbrack C_{1}\exp\left( - j\omega_{0}t_{0} \right) + C_{2}\exp\left( j\omega_{0}t_{0} \right) \right\rbrack\exp\left( - \frac{t_{0}}{T_{2}} \right) \\
M_{y}\left( t_{0} \right) = \left\lbrack - jC_{1}\exp\left( - j\omega_{0}t_{0} \right) + jC_{2}\exp\left( j\omega_{0}t_{0} \right) \right\rbrack\exp\left( - \frac{t_{0}}{T_{2}} \right)
\end{matrix} \right.\ \]

Si divide per il termine esponenziale per entrambe le equazioni:

\[\left\{ \begin{matrix}
M_{x}\left( t_{0} \right)\exp\left( \frac{t_{0}}{T_{2}} \right) = C_{1}\exp\left( - j\omega_{0}t_{0} \right) + C_{2}\exp\left( j\omega_{0}t_{0} \right) \\
M_{y}\left( t_{0} \right)\exp\left( \frac{t_{0}}{T_{2}} \right) = - jC_{1}\exp\left( - j\omega_{0}t_{0} \right) + jC_{2}\exp\left( j\omega_{0}t_{0} \right)
\end{matrix} \right.\ \]

Si divide per l'unità immaginaria nella seconda equazione:

\[\left\{ \begin{matrix}
M_{x}\left( t_{0} \right)\exp\left( \frac{t_{0}}{T_{2}} \right) = C_{1}\exp\left( - j\omega_{0}t_{0} \right) + C_{2}\exp\left( j\omega_{0}t_{0} \right) \\
 - jM_{y}\left( t_{0} \right)\exp\left( \frac{t_{0}}{T_{2}} \right) = - C_{1}\exp\left( - j\omega_{0}t_{0} \right) + C_{2}\exp\left( j\omega_{0}t_{0} \right)
\end{matrix} \right.\ \]

Sommando membro a membro si ottiene:

\[M_{x}\left( t_{0} \right)\exp\left( \frac{t_{0}}{T_{2}} \right) - jM_{y}\left( t_{0} \right)\exp\left( \frac{t_{0}}{T_{2}} \right) = 2C_{2}\exp\left( j\omega_{0}t_{0} \right)\]

Da cui si ricava \(C_{2}\)

\[C_{2} = \frac{1}{2}\left\lbrack M_{x}\left( t_{0} \right)\exp\left( \frac{t_{0}}{T_{2}} \right) - jM_{y}\left( t_{0} \right)\exp\left( \frac{t_{0}}{T_{2}} \right) \right\rbrack\exp\left( - j\omega_{0}t_{0} \right)\]

Sottraendo membro a membro si ricava \(C_{1}\):

\[M_{x}\left( t_{0} \right)\exp\left( \frac{t_{0}}{T_{2}} \right) + jM_{y}\left( t_{0} \right)\exp\left( \frac{t_{0}}{T_{2}} \right) = 2C_{1}\exp\left( - j\omega_{0}t_{0} \right)\]

\[C_{1} = \frac{1}{2}\left\lbrack M_{x}\left( t_{0} \right)\exp\left( \frac{t_{0}}{T_{2}} \right) + jM_{y}\left( t_{0} \right)\exp\left( \frac{t_{0}}{T_{2}} \right) \right\rbrack\exp\left( j\omega_{0}t_{0} \right)\]

La soluzione lungo \({\widehat{i}}_{x}\) è:

\[M_{x}(t) = \left\lbrack C_{1}\exp\left( - j\omega_{0}t \right) + C_{2}\exp\left( j\omega_{0}t \right) \right\rbrack\exp\left( - \frac{t}{T_{2}} \right)\]

Dove:

\[C_{1}\exp\left( - j\omega_{0}t \right) = \frac{1}{2}\left\lbrack M_{x}\left( t_{0} \right)\exp\left( \frac{t_{0}}{T_{2}} \right) + jM_{y}\left( t_{0} \right)\exp\left( \frac{t_{0}}{T_{2}} \right) \right\rbrack\exp\left( j\omega_{0}t_{0} \right)\exp\left( - j\omega_{0}t \right) =\]

\[= \left\lbrack \frac{1}{2}M_{x}\left( t_{0} \right) + j\frac{1}{2}M_{y}\left( t_{0} \right) \right\rbrack\exp\left( \frac{t_{0}}{T_{2}} \right)\exp\left\lbrack - j\omega_{0}\left( t - t_{0} \right) \right\rbrack = \left\lbrack \frac{1}{2}M_{x}\left( t_{0} \right) - \frac{1}{2j}M_{y}\left( t_{0} \right) \right\rbrack\exp\left( \frac{t_{0}}{T_{2}} \right)\exp\left\lbrack - j\omega_{0}\left( t - t_{0} \right) \right\rbrack\]

\[C_{2}\exp\left( j\omega_{0}t \right) = \frac{1}{2}\left\lbrack M_{x}\left( t_{0} \right)\exp\left( \frac{t_{0}}{T_{2}} \right) - jM_{y}\left( t_{0} \right)\exp\left( \frac{t_{0}}{T_{2}} \right) \right\rbrack\exp\left( - j\omega_{0}t_{0} \right)\exp\left( j\omega_{0}t \right) = \left\lbrack \frac{1}{2}M_{x}\left( t_{0} \right) - j\frac{1}{2}M_{y}\left( t_{0} \right) \right\rbrack\exp\left( - j\omega_{0}t_{0} \right)\exp\left\lbrack j\omega_{0}\left( t - t_{0} \right) \right\rbrack = \left\lbrack \frac{1}{2}M_{x}\left( t_{0} \right) + \frac{1}{2j}M_{y}\left( t_{0} \right) \right\rbrack\exp\left( - j\omega_{0}t_{0} \right)\exp\left\lbrack j\omega_{0}\left( t - t_{0} \right) \right\rbrack\]

Quindi, la soluzione lungo \({\widehat{i}}_{x}\) si scrive come:

\[M_{x}(t) = \left\{ \left\lbrack \frac{1}{2}M_{x}\left( t_{0} \right) - \frac{1}{2j}M_{y}\left( t_{0} \right) \right\rbrack\exp\left( \frac{t_{0}}{T_{2}} \right)\exp\left\lbrack - j\omega_{0}\left( t - t_{0} \right) \right\rbrack + \left\lbrack \frac{1}{2}M_{x}\left( t_{0} \right) - j\frac{1}{2}M_{y}\left( t_{0} \right) \right\rbrack\exp\left( - j\omega_{0}t_{0} \right)\exp\left\lbrack j\omega_{0}\left( t - t_{0} \right) \right\rbrack \right\}\exp\left( - \frac{t}{T_{2}} \right)\]

Raccogliendo opportunamente si ha:

\[M_{x}(t) = \left\{ M_{x}\left( t_{0} \right)\frac{\exp\left\lbrack j\omega_{0}\left( t - t_{0} \right) \right\rbrack + \exp\left\lbrack - j\omega_{0}\left( t - t_{0} \right) \right\rbrack}{2} + M_{y}\left( t_{0} \right)\frac{\exp\left\lbrack j\omega_{0}\left( t - t_{0} \right) \right\rbrack - \exp\left\lbrack - j\omega_{0}\left( t - t_{0} \right) \right\rbrack}{2j} \right\}\exp\left( - \frac{t - t_{0}}{T_{2}} \right)\]

Per le relazioni di Eulero, risulta:

\[M_{x}(t) = \left\{ M_{x}\left( t_{0} \right)\cos\left\lbrack \omega_{0}\left( t - t_{0} \right) \right\rbrack + M_{y}\left( t_{0} \right)\sin{\cos\left\lbrack \omega_{0}\left( t - t_{0} \right) \right\rbrack} \right\}\exp\left( - \frac{t - t_{0}}{T_{2}} \right)\]

Analogamente, per la componente lungo \({\widehat{i}}_{y}\) si ha:

\[M_{y}(t) = \left\lbrack - jC_{1}\exp\left( - j\omega_{0}t \right) + jC_{2}\exp\left( j\omega_{0}t \right) \right\rbrack\exp\left( - \frac{t}{T_{2}} \right)\]

Dove:

\[- jC_{1}\exp\left( - j\omega_{0}t \right) = - j\frac{1}{2}\left\lbrack M_{x}\left( t_{0} \right)\exp\left( \frac{t_{0}}{T_{2}} \right) + jM_{y}\left( t_{0} \right)\exp\left( \frac{t_{0}}{T_{2}} \right) \right\rbrack\exp\left( j\omega_{0}t_{0} \right)\exp\left( - j\omega_{0}t \right) = \left\lbrack - j\frac{1}{2}M_{x}\left( t_{0} \right) + \frac{1}{2}M_{y}\left( t_{0} \right) \right\rbrack\exp\left( \frac{t_{0}}{T_{2}} \right)\exp\left\lbrack - j\omega_{0}\left( t - t_{0} \right) \right\rbrack = \left\lbrack \frac{1}{2j}M_{x}\left( t_{0} \right) + \frac{1}{2}M_{y}\left( t_{0} \right) \right\rbrack\exp\left( \frac{t_{0}}{T_{2}} \right)\exp\left\lbrack - j\omega_{0}\left( t - t_{0} \right) \right\rbrack\]

\[jC_{2}\exp\left( j\omega_{0}t \right) = j\frac{1}{2}\left\lbrack M_{x}\left( t_{0} \right)\exp\left( \frac{t_{0}}{T_{2}} \right) - jM_{y}\left( t_{0} \right)\exp\left( \frac{t_{0}}{T_{2}} \right) \right\rbrack\exp\left( - j\omega_{0}t_{0} \right)\exp\left( j\omega_{0}t \right) = = \left\lbrack j\frac{1}{2}M_{x}\left( t_{0} \right) + \frac{1}{2}M_{y}\left( t_{0} \right) \right\rbrack\exp\left( \frac{t_{0}}{T_{2}} \right)\exp\left\lbrack j\omega_{0}\left( t - t_{0} \right) \right\rbrack =\]

\[= \left\lbrack - \frac{1}{2j}M_{x}\left( t_{0} \right) + \frac{1}{2}M_{y}\left( t_{0} \right) \right\rbrack\exp\left( \frac{t_{0}}{T_{2}} \right)\exp\left\lbrack j\omega_{0}\left( t - t_{0} \right) \right\rbrack\]

Quindi, la soluzione lungo \({\widehat{i}}_{y}\) si scrive come:

\[M_{y}(t) = \left\{ \left\lbrack \frac{1}{2j}M_{x}\left( t_{0} \right) + \frac{1}{2}M_{y}\left( t_{0} \right) \right\rbrack\exp\left( \frac{t_{0}}{T_{2}} \right)\exp\left\lbrack - j\omega_{0}\left( t - t_{0} \right) \right\rbrack + \left\lbrack - \frac{1}{2j}M_{x}\left( t_{0} \right) + \frac{1}{2}M_{y}\left( t_{0} \right) \right\rbrack\exp\left( \frac{t_{0}}{T_{2}} \right)\exp\left\lbrack j\omega_{0}\left( t - t_{0} \right) \right\rbrack \right\}\exp\left( - \frac{t}{T_{2}} \right)\]

Raccogliendo opportunamente si ha:

\[M_{y}(t) = \left\{ - M_{x}\left( t_{0} \right)\frac{\exp\left\lbrack j\omega_{0}\left( t - t_{0} \right) \right\rbrack - \exp\left\lbrack - j\omega_{0}\left( t - t_{0} \right) \right\rbrack}{2j} + M_{y}\left( t_{0} \right)\frac{\exp\left\lbrack j\omega_{0}\left( t - t_{0} \right) \right\rbrack + \exp\left\lbrack - j\omega_{0}\left( t - t_{0} \right) \right\rbrack}{2} \right\}\exp\left( - \frac{t - t_{0}}{T_{2}} \right)\]

Per l'equivalenza di Eulero risulta:

\[M_{y}(t) = \left\{ - M_{x}\left( t_{0} \right)\sin\left\lbrack \omega_{0}\left( t - t_{0} \right) \right\rbrack + M_{y}\left( t_{0} \right)\cos\left\lbrack \omega_{0}\left( t - t_{0} \right) \right\rbrack \right\}\exp\left( - \frac{t - t_{0}}{T_{2}} \right)\]

In definitiva, le componenti trasverse evolvono secondo le seguenti equazioni:

\[\left\{ \begin{matrix}
M_{x}(t) = \left\{ M_{x}\left( t_{0} \right)\cos\left\lbrack \omega_{0}\left( t - t_{0} \right) \right\rbrack + M_{y}\left( t_{0} \right)\sin\left\lbrack \omega_{0}\left( t - t_{0} \right) \right\rbrack \right\}\exp\left( - \frac{t - t_{0}}{T_{2}} \right) \\
M_{y}(t) = \left\{ - M_{x}\left( t_{0} \right)\sin\left\lbrack \omega_{0}\left( t - t_{0} \right) \right\rbrack + M_{y}\left( t_{0} \right)\cos\left\lbrack \omega_{0}\left( t - t_{0} \right) \right\rbrack \right\}\exp\left( - \frac{t - t_{0}}{T_{2}} \right)
\end{matrix} \right.\ \]

Se l'istante iniziale coincide con l'origine dei tempi, \(t_{0} = 0\), risulta:

\[\left\{ \begin{matrix}
M_{x}(t) = \left\lbrack M_{x}(0)\cos\left( \omega_{0}t \right) + M_{y}(0)\sin\left( \omega_{0}t \right) \right\rbrack\exp\left( - \frac{t}{T_{2}} \right) \\
M_{y}(t) = \left\lbrack - M_{x}(0)\sin\left( \omega_{0}t \right) + M_{y}(0)\cos\left( \omega_{0}t \right) \right\rbrack\exp\left( - \frac{t - t_{0}}{T_{2}} \right)
\end{matrix} \right.\ \]

Per un tempo sufficientemente lungo, circa \(4 \div 5\) volte \(T_{2}\) è possibile ritenere la risposta transitoria esaurita, per cui le componenti trasversali sono nulle:

\[M_{x}(t) \rightarrow 0,\ \ t \rightarrow \infty\]

\[M_{y}(t) \rightarrow 0,\ \ t \rightarrow \infty\]

Siccome il tempo di rilassamento trasversale è dell'ordine di \(100\ ms\), il tempo necessario affinché le componenti trasversali del vettore di magnetizzazione si annullino è dell'ordine di \(500\ ms\), ovvero un ordine di grandezza inferiore rispetto al tempo che la componente longitudinale impiega per raggiungere il regime.

È possibile scrivere la soluzione delle componenti trasversali in forma complessa, introducendo il fasore \(M_{+}\), definito come:

\[M_{+}(t) = M_{x}(t) + jM_{y}(t) = \left\lbrack M_{x}(0)\cos\left( \omega_{0}t \right) + M_{y}(0)\sin\left( \omega_{0}t \right) - jM_{x}(0)\sin\left( \omega_{0}t \right) + jM_{y}(0)\cos\left( \omega_{0}t \right) \right\rbrack\exp\left( - \frac{t}{T_{2}} \right)\]

Raccogliendo si ottiene:

\[M_{+}(t) = \left\{ M_{x}(0)\left\lbrack \cos\left( \omega_{0}t \right) - j\sin\left( \omega_{0}t \right) \right\rbrack + jM_{y}(0)\left\lbrack \cos\left( \omega_{0}t \right) - j\sin\left( \omega_{0}t \right) \right\rbrack \right\}\exp\left( - \frac{t}{T_{2}} \right) = \left\{ \left\lbrack \cos\left( \omega_{0}t \right) - j\sin\left( \omega_{0}t \right) \right\rbrack\left\lbrack M_{x}(0) + jM_{y}(0) \right\rbrack \right\}\exp\left( - \frac{t}{T_{2}} \right)\]

Dove:

\[M_{+}(0) = M_{x}(0) + jM_{y}(0)\]

Inoltre, per l'identità di Eulero e le proprietà delle funzioni trigonometriche:

\[\cos\left( \omega_{0}t \right) - j\sin\left( \omega_{0}t \right) = j\sin\left( - \omega_{0}t \right) + \cos\left( - \omega_{0}t \right) = \exp\left( - j\omega_{0}t \right)\]

Per cui il fasore si scrive come:

\[M_{+}(t) = M_{+}(0)\exp\left( - j\omega_{0}t \right)\exp\left( - \frac{t}{T_{2}} \right) = M_{+}(0)\exp\left\lbrack - \left( j\omega_{0}t + \frac{t}{T_{2}} \right) \right\rbrack\ \]

Tale relazione fornisce l'evoluzione temporale del vettore di magnetizzazione nel piano complesso. Il moto nel piano trasverso avviene con pulsazione \(\omega_{0}\) e con ampiezza che decade esponenzialmente con constante ti tempo \(T_{2}\). In circa \(500\ ms\) la componente trasversa del vettore di magnetizzazione si annulla.

\begin{figure}
\centering
\includegraphics[width=2.7037in,height=2.1196in,alt={P3496\#yIS1}]{media/6_IntroMRI/image76.pdf}\caption{Figura .: Andamento del vettore di magnetizzazione nel piano trasverso}
\end{figure}

Il vettore di magnetizzazione \(\overset{\underline{}}{M}\), in definitiva, evolve secondo un movimento rotatorio decrescente nel piano trasverso con constante di tempo \(T_{2}\) e con andamento esponenziale crescente, e costante di tempo \(T_{1}\), lungo la direzione longitudinale. Le componenti del vettore di magnetizzazione sono:

\[\left\{ \begin{matrix}
M_{x}(t) = \left\lbrack M_{x}(0)\cos\left( \omega_{0}t \right) + M_{y}(0)\sin\left( \omega_{0}t \right) \right\rbrack\exp\left( - \frac{t}{T_{2}} \right) \\
M_{y}(t) = \left\lbrack - M_{x}(0)\sin\left( \omega_{0}t \right) + M_{y}(0)\cos\left( \omega_{0}t \right) \right\rbrack\exp\left( - \frac{t - t_{0}}{T_{2}} \right) \\
M_{z}(t) = M_{0}\left\lbrack 1 - \exp\left( - \frac{t}{T_{1}} \right) \right\rbrack
\end{matrix} \right.\ \]

La composizione dei due moti determina che il vettore di magnetizzazione ha un modulo variabile nel tempo.

Si considera come condizione iniziale il vettore di magnetizzazione a valle di un ribaltamento nel piano trasverso a opera di un impulso a radiofrequenza. Al tempo \(t = 0\), il vettore di magnetizzazione è:

\[\overset{\underline{}}{M}(0) = M_{x}(0){\widehat{i}}_{x} + M_{y}(0){\widehat{i}}_{y}\]

La componente lungo \({\widehat{i}}_{z}\) parte dal valore nullo e si porta al valore di regime in un tempo di \(5 \div 10\ s\), mentre quelle trasversali vanno a zero in un tempo di \(0.5 \div 1\ s\).

La curva descritta dal vettore di magnetizzazione tende a raggiungere il valore di regime \(M_{0}\) sull'asse delle \({\widehat{i}}_{z}\) mediante un andamento a spirale, convergente sull'asse del campo principale. Il modo elicoidale a raggio variabile lo si ritrova dopo una perturbazione ed è il moto in cui il vettore \(\overset{\underline{}}{M}\) torna all'equilibrio termodinamico per effetto del campo principale \(B_{0}\). I tempi di rilassamento, quindi, interessano il ritorno all'equilibrio.

\begin{figure}
\centering
\includegraphics[width=3.1141in,height=2.12733in,alt={P3505\#yIS1}]{media/6_IntroMRI/image77.pdf}\caption{Figura .: Evoluzione temporale del vettore di magnetizzazione a valle di un ribaltamento sul piano trasverso}
\end{figure}

\subsubsection{Vettore magnetizzazione durante una perturbazione}\label{vettore-magnetizzazione-durante-una-perturbazione}

Si vuole analizzare l'evoluzione temporale del vettore di magnetizzazione durante l'applicazione di un impulso a radiofrequenza, che perturba l'equilibrio del sistema. La trattazione viene eseguita nel sistema rotante in modo da poter essere facilmente comprensibile.

Il campo efficace, visto da uno spin nel sistema di riferimento rotante al quale è applicato un impulso a radiofrequenza diretto lungo l'asse \({\widehat{i}}_{x'}\), è:

\[{\overset{\underline{}}{B}}_{eff} = \left( B_{0} - \frac{\omega}{\gamma} \right){\widehat{i}}_{z} + B_{1}{\widehat{i}}_{x'}\]

Dove \(\omega\) è la velocità di rotazione del sistema.

L'equazione di Bloch, scritte nel sistema di riferimento rotante, è analoga a quella scritta nel sistema fisso del laboratorio:

\[\left( \frac{d\overset{\underline{}}{M}}{dt} \right)' = \gamma\overset{\underline{}}{M} \times {\overset{\underline{}}{B}}_{eff} + \frac{1}{T_{1}}\left( M_{0} - M_{z}\  \right){\widehat{i}}_{z} - \frac{1}{T_{2}}{\overset{\underline{}}{M}}_{\bot}\]

Questa equazione è stata ottenuta senza applicare nessuna ipotesi sul sistema di riferimento, quindi, è valida sia in sistemi inerziali che non inerziali. In particolare, la componente longitudinale del vettore di magnetizzazione evolve in maniera indipendente da quelle trasversali, quindi, il modo lungo l'asse \({\widehat{i}}_{z}\) si conserva.

Si proietta l'equazione vettoriale di Bloch lungo gli assi, svolgendo il prodotto vettoriale:

\[\overset{\underline{}}{M} \times {\overset{\underline{}}{B}}_{eff} = \left| \begin{matrix}
{\widehat{i}}_{x'} & {\widehat{i}}_{y'} & {\widehat{i}}_{z}\  \\
M_{x'} & M_{y'} & M_{z} \\
B_{1} & 0 & B_{0} - \frac{\omega}{\gamma}
\end{matrix} \right| = M_{y'}\left( B_{0} - \frac{\omega}{\gamma} \right){\widehat{i}}_{x'} + M_{z}B_{1}{\widehat{i}}_{y'} - M_{y'}B_{1}{\widehat{i}}_{z} - M_{x'}\left( B_{0} - \frac{\omega}{\gamma} \right){\widehat{i}}_{y'}\]

Raccogliendo i termini, si ottiene:

\[\overset{\underline{}}{M} \times {\overset{\underline{}}{B}}_{eff} = M_{y'}\left( B_{0} - \frac{\omega}{\gamma} \right){\widehat{i}}_{x'} + M_{z}B_{1}{\widehat{i}}_{y'} - M_{y'}B_{1}{\widehat{i}}_{z} - M_{x'}\left( B_{0} - \frac{\omega}{\gamma} \right){\widehat{i}}_{y'} = M_{y'}\left( B_{0} - \frac{\omega}{\gamma} \right){\widehat{i}}_{x'} - \left\lbrack M_{x'}\left( B_{0} - \frac{\omega}{\gamma} \right) - M_{z}B_{1} \right\rbrack{\widehat{i}}_{y'} - M_{y'}B_{1}{\widehat{i}}_{z}\]

L'equazione vettoriale di Bloch si scrive:

\[\left( \frac{d\overset{\underline{}}{M}}{dt} \right)' = \gamma\overset{\underline{}}{M} \times {\overset{\underline{}}{B}}_{eff} + \frac{1}{T_{1}}\left( M_{0} - M_{z}\  \right){\widehat{i}}_{z} - \frac{1}{T_{2}}{\overset{\underline{}}{M}}_{\bot} = \gamma\left\{ M_{y'}\left( B_{0} - \frac{\omega}{\gamma} \right){\widehat{i}}_{x'} - \left\lbrack M_{x'}\left( B_{0} - \frac{\omega}{\gamma} \right) - M_{z}B_{1} \right\rbrack{\widehat{i}}_{y'} - M_{y'}B_{1}{\widehat{i}}_{z} \right\} + \frac{1}{T_{1}}\left( M_{0} - M_{z}\  \right){\widehat{i}}_{z} - \frac{1}{T_{2}}\left( M_{x'}{\widehat{i}}_{x'} + M_{y'}{\widehat{i}}_{y'} \right)\]

Scomponendo lungo gli assi del sistema rotante si ottiene:

\[\left\{ \begin{matrix}
\left( \frac{dM_{x'}}{dt} \right)' = \gamma M_{y'}\left( B_{0} - \frac{\omega}{\gamma} \right) - \frac{1}{T_{2}}M_{x'} \\
\left( \frac{dM_{y'}}{dt} \right)' = \gamma M_{z}B_{1} - \gamma M_{x'}\left( B_{0} - \frac{\omega}{\gamma} \right) - \frac{1}{T_{2}}M_{y'} \\
\left( \frac{dM_{z}}{dt} \right)' = \frac{1}{T_{1}}\left( M_{0} - M_{z}\  \right) - \gamma M_{y'}B_{1}
\end{matrix} \right.\ \]

Dove \(\omega_{1} = \gamma B_{1}\) e \(\omega_{0} = \gamma B_{0}\):

\[\left\{ \begin{matrix}
\left( \frac{dM_{x'}}{dt} \right)' = \left( \omega_{0} - \omega \right)M_{y'} - \frac{1}{T_{2}}M_{x'} \\
\left( \frac{dM_{y'}}{dt} \right)' = \omega_{1}M_{z} - \left( \omega_{0} - \omega \right)M_{x'} - \frac{1}{T_{2}}M_{y'} \\
\left( \frac{dM_{z}}{dt} \right)' = \frac{1}{T_{1}}\left( M_{0} - M_{z}\  \right) - \omega_{1}M_{y'}
\end{matrix} \right.\ \]

Nelle equazioni \(\omega_{0}\) è la frequenza di Larmor, \(\omega_{1}\) è la frequenza del campo a radiofrequenza, mentre \(\omega\) è la frequenza con cui ruota il sistema di riferimento.

Si definisce \(\mathrm{\Delta}\omega = \omega_{0} - \omega\) e rappresenta la deviazione dalla condizione ideale. Questa deviazione è legata alle disomogeneità del campo o alla variazioni delle frequenze dell'impulso utilizzato. Affinché il vettore di magnetizzazione ruoti dell'angolo desiderato, è necessario che l'impulso a radiofrequenza abbia una durata di qualche \(ms\). Nel dettaglio, il periodo di applicazione del campo a radiofrequenza, che genera la processione intorno a \({\widehat{i}}_{x'}\), è di qualche millisecondo ed è legato alla pulsazione \(\omega_{1}\) dalla relazione:

\[\omega_{1} = \frac{2\pi}{T}\]

Dato che il tempo di rilassamento longitudinale \(T_{1}\) è dell'ordine dei secondi, risulta che:

\[T \ll T_{1} \Leftrightarrow \frac{1}{T} \gg \frac{1}{T_{1}}\]

A meno di un fattore \(2\pi\), risulta:

\[\frac{2\pi}{T} = \omega_{1} \gg \frac{1}{T_{1}}\]

Analogo discorso vale per il tempo di rilassamento trasversale \(T_{2}\), dell'ordine dei \(500\ ms\). Rispetto alla pulsazione \(\omega_{1}\), i termini che evolvono con costanti di tempo \(T_{1}\) e \(T_{2}\) possono essere trascurati, in quanto molto più lenti. L'evoluzione del vettore di magnetizzazione, in ultima analisi, non dipende dai tempi di rilassamento:

\[\left\{ \begin{matrix}
\left( \frac{dM_{x'}}{dt} \right)' = \left( \omega_{0} - \omega \right)M_{y'} \\
\left( \frac{dM_{y'}}{dt} \right)' = \omega_{1}M_{z} - \left( \omega_{0} - \omega \right)M_{x'} \\
\left( \frac{dM_{z}}{dt} \right)' = - \omega_{1}M_{y'}
\end{matrix} \right.\ \]

Se il sistema ruota con una pulsazione angolare molto prossima a quella di Larmor risulta:

\[\omega_{0} \simeq \omega \Leftrightarrow \mathrm{\Delta}\omega = \omega_{0} - \omega \simeq 0\]

È possibile, quindi, trascurare i termini contenenti \(\mathrm{\Delta}\omega\):

\[\left\{ \begin{matrix}
\left( \frac{dM_{x'}}{dt} \right)' = 0 \\
\left( \frac{dM_{y'}}{dt} \right)' = \omega_{1}M_{z} \\
\left( \frac{dM_{z}}{dt} \right)' = - \omega_{1}M_{y'}
\end{matrix} \right.\ \]

Le equazioni individuate suggeriscono un moto di precessione intorno l'asse \({\widehat{i}}_{x'}\).

Per le sequenze applicate normalmente nella pratica, in definitiva, si ritiene che la rotazione del vettore di magnetizzazione avvenga senza l'influenza dei tempi di rilassamento, poiché l'evoluzione temporale legata all'impulso a radiofrequenza è molto più veloce di quella legata ai fenomeni di rilassamento. Si approssima, inoltre, la frequenza di risonanza con quella del campo a radiofrequenza. Anche in presenza di tali approssimazioni, i risultati ottenuti sono attendibili, nel senso che concordi ai risultati sperimentali. Il ritorno all'equilibrio è, invece, caratterizzato dai tempi di rilassamento del tessuto.

L'evoluzione del vettore di magnetizzazione, legato all'applicazione dell'impulso a radiofrequenza, si compone di due fasi:

\begin{itemize}
\item
  Raggiunto l'equilibrio termodinamico, il vettore di magnetizzazione è ribaltato lungo uno degli assi \(x'\) o \(y'\), utilizzando un impulso a radiofrequenza a polarizzazione lineare o circolare. Come detto precedentemente, il ribaltamento è descritto trascurando gli effetti del rilassamento, poiché i tempi \(T_{1}\) e \(T_{2}\) sono molto più lunghi della durata dell'impulso a radiofrequenza;
\end{itemize}

\begin{longtable}[]{@{}
  >{\raggedright\arraybackslash}p{(\linewidth - 2\tabcolsep) * \real{0.5000}}
  >{\raggedright\arraybackslash}p{(\linewidth - 2\tabcolsep) * \real{0.5000}}@{}}
\caption{Figura .: L'applicazione dell'impulso RF porta a una precessione intorno all'asse \(y'\)}\tabularnewline
\toprule\noalign{}
\begin{minipage}[b]{\linewidth}\centering
\includegraphics[width=3.17495in,height=3.19514in,alt={P3542C1T5\#yIS1}]{media/6_IntroMRI/image78.pdf}\end{minipage} & \begin{minipage}[b]{\linewidth}\centering
\includegraphics[width=2.90572in,height=3.19514in,alt={P3543C2T5\#yIS1}]{media/6_IntroMRI/image79.pdf}\end{minipage} \\
\midrule\noalign{}
\endfirsthead
\toprule\noalign{}
\begin{minipage}[b]{\linewidth}\centering
\includegraphics[width=3.17495in,height=3.19514in,alt={P3542C1T5\#yIS1}]{media/6_IntroMRI/image78.pdf}\end{minipage} & \begin{minipage}[b]{\linewidth}\centering
\includegraphics[width=2.90572in,height=3.19514in,alt={P3543C2T5\#yIS1}]{media/6_IntroMRI/image79.pdf}\end{minipage} \\
\midrule\noalign{}
\endhead
\bottomrule\noalign{}
\endlastfoot
\end{longtable}

\begin{itemize}
\item
  Il fenomeno di recupero della magnetizzazione è caratterizzato dal ritorno all'equilibrio della magnetizzazione e avviene con costante di tempo \(T_{1}\), tempo di rilassamento per la componente longitudinale e costante di tempo \(T_{2}\) per la componente trasversale. I tempi di evoluzione del vettore di magnetizzazione dono paragonabili a quelli di rilassamento per cui non possono essere trascurati.
\end{itemize}

\subsubsection{Sequenza FID}\label{sequenza-fid}

Per ottenere una misura del vettore di magnetizzazione, è necessario perturbare l'equilibrio raggiunto dagli spin contenuti nei tessuti del paziente. Dal segnale registrato è possibile, in seguito, ricavare le informazioni sui tempi di rilassamento \emph{spin-lattice}, \(T_{1}\), e spin-spin \(T_{2}\), così da caratterizzare completamente il tessuto.

La sequenza più semplice da applicare per perturbare il sistema è detta FID (\emph{Free Induction Decay}) in cui si applica una radiazione a radiofrequenza, indicata nei diagrammi con una \(rect\) o con un pacchetto di onde sinusoidali, a frequenza \(\omega \simeq \omega_{0}\).

Successivamente, si registra il segnale \(s\) dovuto al ritorno all'equilibrio del vettore magnetizzazione. Le antenne in ricezione iniziano ad acquisire il segnale subito dopo la fine dell'impulso.

Il segnale registrato è proporzionale alla componente trasversa del campo, dunque, le antenne ricevono una sinusoide, a frequenza \(\omega_{0}\) smorzata con un decadimento di tipo esponenziale e costante di tempo \(T_{2}\). Adoperando la notazione complessa, il segnale registrato dall'antenna è proporzionale al fasore \(M_{+}(t)\):

\[M_{+}(t) = M_{+}(0)\exp\left( - \frac{t}{T_{2}} \right)\exp\left( - j\omega_{0}t \right)\]

Dove \(M_{+}(0)\) è proporzionale al vettore di magnetizzazione all'equilibrio termodinamico \(M_{0}\), che a sua volta dipende dalla densità protonica \(\rho\), indicante il numero di protoni nel volumetto \(V\).

Siccome non vi è nessun gradiente di campo, ogni spin all'interno del volume del paziente precede alla frequenza di Larmor \(\omega_{0}\), dunque, l'applicazione del campo rotante a frequenza prossima a quella di Larmor eccita tutti gli spin nel corpo del paziente. In questo caso, il volume-paziente non è considerato in tanti volumetti elementari ma è assunto essere un unico volume.

La sequenza FID non fornisce informazioni sulla composizione chimica dei tessuti, ovvero non fornisce i tempi di rilassamento di ogni singolo volumetto. In altre parole, la sequenza FID non fornire informazioni utili all'\emph{imaging} ma permette di correggere degli errori introdotti dalle disomogeneità del campo magnetico principale. Infatti, le misure eseguite con la sequenza FID possono fornire un'idea su quanto il campo magnetico principale si discosta dall'andamento ideale, ovvero uniforme in tutto lo spazio contenente il paziente.

Il tempo di rilassamento trasversale \(T_{2}\) è determinato dall'iterazioni spin-spin, che ha l'effetto di alterare localmente il campo magnetico visto da uno spin. Se il campo magnetico esterno è disomogeneo, il campo locale visto dagli spin varia con la posizione, dunque, si aggiunge un ulteriore fonte di disturbo.

In generale, i fornitori di risonanze magnetiche garantiscono che l'omogeneità del campo nel gantry abbia una valore nominale variabile di una parte per milione (ppm), all'interno di una sfera centrata nel gantry e raggio dell'ordine dei \(20\ cm\).

\begin{figure}
\centering
\includegraphics[width=3.73125in,height=2.04747in,alt={P3558\#yIS1}]{media/6_IntroMRI/image80.pdf}\caption{Figura .: Regione di spazio del gantry in cui il campo è omogeneo}
\end{figure}

Se il valore nominale del campo è di \(1.5\ T\), la variazione di campo all'interno della sfera è dell'ordine di \(10^{- 6}\ \), ovvero una variazione massima di \(1.5\ \mu T\). Queste disomogeneità di campo si aggiungono ai campi locali visti dai singoli spin. Ciò porta a un'ulteriore riduzione del tempo di rilassamento trasversale \(T_{2}\).

Si introduce il tempo \(T_{2}^{*}\) legato sia alla disomogeneità del campo principale, sia all'interazione spin-spin. Il primo fenomeno è quantificato da un tempo \(T_{2}'\), legato alla tecnologia costruttiva con cui si produce il campo magnetico, che, fondamentalmente, definisce le disomogeneità di \(B_{0}\).

Il tempo \(T_{2}^{*}\) è definito come:

\[\frac{1}{T_{2}^{*}} = \frac{1}{T_{2}} + \frac{1}{T_{2}'}\]

Con la sequenza FID è possibile ottenere delle informazioni su \(T_{2}^{*}\) e, di conseguenza, su quanto il campo magnetico principale varia nello spazio. Noto il tempo \(T_{2}^{*}\), mediante appositi algoritmi, è possibile correggere le immagini, le quali mostrano una maggiore affidabilità nella stima dei tempi di rilassamento \(T_{1}\) e \(T_{2}\).
