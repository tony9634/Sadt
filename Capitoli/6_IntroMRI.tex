\begin{center}
\vfill
    \chapter{Introduzione alla risonanza magnetica}
    \label{blx:refsection\therefsection}
\vfill

\minitoc
\newpage
\end{center}
\justify


\section{Diverse metodiche di imaging}\label{diverse-metodiche-di-imaging}

Dal punto di vista estetico, i macchinari per risonanza magnetica (MRI), tomografia computerizzata (CT) e tomografia a emissione di positroni (PET) presentano una struttura a spirale. Tutte queste metodiche prevedono un \textbf{gantry}, che ospita la circuiteria necessaria per la generazione dello stimolo e la sua ricezione. Nella PET, lo stimolo è generato internamente al paziente tramite l’iniezione di un radiofarmaco.

\begin{figure}[ht]
\centering
\includegraphics[width=4.35482in,height=2.05952in,alt={P2729\#yIS1}]{media/6_IntroMRI/image57.pdf}\caption{Tipica struttura di una strumentazione di imaging}
\end{figure}

Il paziente viene posizionato su un tavolo mobile longitudinale, detto tavolo \textbf{porta-paziente}. Sebbene le applicazioni esterne possano sembrare simili, le metodiche si basano su principi fisici differenti:

\begin{itemize}
\item La \textbf{TC} si basa sull’assorbimento dei raggi X da parte del corpo umano. L’assorbimento dipende dalla densità del tessuto attraversato: l’osso assorbe molto più dei tessuti molli, che tra loro presentano assorbimenti simili. La TC fornisce immagini con elevato contenuto morfologico;
\item La \textbf{risonanza magnetica} sfrutta la risonanza dei nuclei di idrogeno in presenza di un campo magnetico esterno. Poiché l’idrogeno è prevalentemente presente nell’acqua (\ce{H2O}), è possibile discriminare i tessuti in base al contenuto acquoso. L’osso, essendo povero d’acqua, risulta poco visibile. Più in dettaglio la porzione corticale (dura) dell'osso, avendo bassa densità di protoni mobili, è meno visibile, mentre il midollo osseo (grasso e acqua) è ben visualizzato;
\item La \textbf{PET} si basa sull’emissione di fotoni \(\gamma\) da parte di un radiofarmaco iniettato nel paziente. La distribuzione del radiofarmaco fornisce informazioni sul metabolismo dei tessuti, quindi la PET produce immagini funzionali.
\end{itemize}

La strumentazione moderna combina PET e CT in un’unica macchina, detta \textbf{PET-CT}, per ottenere immagini con contenuto sia morfologico che funzionale.

La risonanza magnetica può fornire immagini sia morfologiche che funzionali. Queste ultime sono utilizzate prevalentemente  nello studio dell'encefalo: in base alle aree attivate da un determinato stimolo, si genera una mappa a colori, indicante il funzionamento della corteccia cerebrale.

La MRI (o \textit{magnetic resonance imaging}) funzionale (fMRI) è utilizzata per la diagnosi di malattie neurodegenerative. Questo tipo di studio non può essere eseguito mediante CT poiché la scatola cranica assorbe la maggior parte della radiazione X incidente. Ne risulta, dunque, un'immagine dell'encefalo poco definita.

La \textbf{CT} è utilizzata per ottenere rapidamente immagini morfologiche degli organi interni. Questa tecnologia è particolarmente efficace nella visualizzazione delle ossa, ma meno nei tessuti molli. Poiché utilizza radiazioni ionizzanti, comporta un rischio biologico dovuta ai possibili effetti cancerogeni della radiazione ionizzante.  danni biologici si verificano con un certo andamento statistico, quindi, sono state prodotte delle specifiche normative che regolano la dose assorbita sia dal paziente che dal tecnico radiologo

La CT è la metodica più veloce: consente scansioni \textit{total body} in pochi secondi, ed è quindi molto usata per diagnosi rapide, nonostante la risoluzione limitata nei tessuti molli.

La \textbf{risonanza magnetica}, al contrario, non utilizza radiazioni ionizzanti, ma campi elettromagnetici statici e a radiofrequenza (dell'ordine dei \(mT\)). Questi campi non sono associati a effetti nocivi significativi, ma sono regolati da normative nazionali e internazionali, relative all'energia elettromagnetica depositata nei tessuti.

L'utilizzo del campo elettromagnetico consente di eseguire l'esame di risonanza magnetica ripetutamente. Alla risonanza magnetica, inoltre, è associata una bassa invasività e una grande flessibilità, poiché permette di ottenere sia immagini morfologiche che funzionali, ovvero, è possibile visualizzare un tessuto biologico in base alla sua composizione biochimica.

A differenza della CT, tuttavia, l'esame con risonanza magnetica richiede un lungo tempo di esecuzione, che si aggira intorno ai \(30\) minuti fino a un'ora. Ciò provoca anche un senso di disagio per il paziente, il quale deve restare immobile per un lungo periodo di tempo. Inoltre, a causa dei normali movimento del paziente, si producono artefatti da movimento durante l'esecuzione dell'esame radiologico. Infine, la strumentazione usata per la risonanza magnetica è più complessa di quella della CT, poiché deve produrre campi statici e a radiofrequenza con determinate caratteristiche. Ne discende che la risonanza magnetica è più costosa della CT.

La PET condivide con la risonanza magnetica i lunghi tempo di analisi, infatti, al fine di ottenere delle immagini \textit{total body} sono richiesti di \(45\) minuti a un'ora. Storicamente, questa metodologia diagnostica nasce per eseguire lo studio metabolico del cervello, ma attualmente, è usata soprattutto per analizzare il comportamento metabolico dei tumori, al fine di evidenziare il suo stadio e la presenza di metastasi. Infatti, in caso di diagnosi di tumore, è necessario eseguire la PET almeno una volta all'anno, al fine di rilevare precocemente la presenza di nuove metastasi, formatesi per problemi legati alla recidiva tumorale.

La PET si basa sull'emissione di radiazione \(\gamma\) da parte di radionuclidi eccitati. La distribuzione del radiofarmaco permette di ottenere immagini indicanti la funzionalità degli organi interessati.

La combinazione PET-CT consente di associare l’informazione funzionale a quella morfologica. Le sole immagini funzionali sono difficili da interpretare, poiché non evidenziano le strutture anatomiche, che emettono quella data quantità di tracciante.

Lo svantaggio principale della PET è la radioattività del paziente, che emette radiazione \(\gamma\) fino al decadimento del radiofarmaco (circa 24 ore). Durante questo periodo, il paziente deve evitare il contatto con donne incinte e bambini.

\section{Storia della risonanza magnetica}\label{storia-della-risonanza-magnetica}

Con l’esperimento di Stern e Gerlach si dimostrò che il momento magnetico atomico è quantizzato, ovvero può assumere solamente due valori, corrispondenti a due distinti livelli energetici.

Negli anni ’30, il fisico Felix Bloch sviluppò delle equazioni fenomenologiche per descrivere il comportamento degli spin immersi in un campo magnetico. Queste equazioni rappresentano un approccio intermedio tra la fisica classica e la meccanica quantistica.

Grazie agli studi di Bloch, negli anni ’70 il chimico Paul Christian Lauterbur mise a punto una tecnica per ottenere immagini di sezioni corporee mediante campi elettromagnetici a radiofrequenza. La prima immagine ottenuta con la risonanza magnetica fu quella di un limone.

Da questa prima applicazione furono sviluppati i primi macchinari commerciali, oggi ampiamente utilizzati nella diagnostica medica. A Lauterbur si deve l’intuizione secondo cui l’introduzione di gradienti di campo magnetico consente di localizzare l’origine delle onde radio emesse dai nuclei dell’oggetto in esame, permettendo così la ricostruzione di immagini bidimensionali.

\section{Introduzione al principio di risonanza magnetica}\label{introduzione-al-principio-di-risonanza-magnetica}

Il principio fisico alla base della risonanza magnetica è descritto dalle equazioni di Bloch: un gran numero di protoni di idrogeno, immersi in un campo magnetico \(\vec{B}\), produce una magnetizzazione netta \(\vec{M}\), misurabile.  A temperatura ambiente e all’equilibrio termodinamico,  la magnetizzazione netta segue la legge di Curie:

\[
M \simeq \dfrac{N}{V} \dfrac{\gamma^2 \hslash^2}{4k_B T} B_0
\]

dove \(M\) è il momento magnetico per unità di volume, il quale contiene \(N\) particelle con spin, \(N/V = \rho\) è la densità protonica, \(\gamma\) è il rapporto giromagnetico, \(k_B\) la costante di Boltzmann, \(T\) la temperatura e \(B_0\) il campo magnetico statico applicato.

Dalla legge i Curie si evince che la magnetizzazione dipende dal campo magnetico applicato, dalla temperatura, dal rapporto giromagnetico e dal numero di spin presenti nel volume elementare sotto analisi. Ne discende che per aumentare il valore della magnetizzazione è possibile:

\begin{itemize}
\item \textbf{Aumentare il campo magnetico} \(B_0\): tipicamente si utilizzano campi da \(1.5\,T\), valore regolato da normative internazionali;
\item \textbf{Ridurre la temperatura}: non è praticabile in ambito clinico, poiché il paziente non può essere raffreddato eccessivamente. Nella sala della risonanza magnetica la temperatura è mantenuta costante, intorno ai \(25\ {^\circ}C\), al fine di evitare fluttuazioni della magnetizzazione;
\item
  \textbf{Scegliere sostanze con \(\gamma\) maggiore}: In linea di principio è possibile determinare immagini di risonanza magnetica anche degli elettroni, che presentano un rapporto giromagnetico \(\gamma_{e} = - 1.76 \cdot 10^{11}\ rad/Ts\), mentre quello del protone è \(\gamma_{p} = 2.68 \cdot 10^{8}\ rad/Ts\). Il rapporto di \(\gamma_{e}\) e \(\gamma_{p}\) è:

\[
\gamma = \dfrac{|\gamma_e|}{\gamma_p} = \dfrac{1.76 \cdot 10^{11}}{2.68 \cdot 10^8} \simeq 658
\]

Il rapporto giromagnetico dell'elettrone è molto maggiore di quel del protone, ovvero del nucleo di idrogeno, quindi, il vettore magnetizzazione degli elettroni, all'equilibrio termodinamico, ha intensità maggiore rispetto a quello dei nuclei di idrogeno. Si osservi, tuttavia, che il fattore giromagnetico è presente nell'espressione della frequenza di precessione di Larmor, ovvero la frequenza con cui gli spin ruotano intorno all'asse individuato dal campo magnetico:

\[
\omega_{0} = \gamma B_{0}
\]

La frequenza del campo magnetico applicato aumenta al crescere del rapporto giromagnetico, in particolare, con un campo di \(1.5\ T\), per un elettrone, si ha:

\[
f_{0} = \dfrac{\omega_{0}}{2\pi} = \dfrac{\gamma_{e}}{2\pi} B_{0} = \dfrac{ 1.76 \cdot 10^{11} \dfrac{rad}{Ts}\{2\pi rad}1,5T = 42\ GHz
\]

Il campo irradiato dall'elettrone è, dunque, dell'ordine della decina di \(GHz\). Ciò determina una maggiore energia associata all'onda, che si deposita nei tessuti biologici e quindi potenziali effetti biologici più rilevanti.. In generale, più l'onda si avvicina allo spettro dei raggi X, maggiore è il loro contenuto energetico e maggiori sono i possibili effetti biologici. Per tale motivo le radiofrequenze adoperate sono ottimizzate per il protone.

  \item \textbf{Utilizzare nuclei abbondanti}: l'idrogeno ha una concentrare di circa \(88\ M = 88\ mol/V\), molto maggiore degli altri composti organici che presentano una concentrazione molare dell'ordine dei \(\mu M\) o \(mM\). L'uso della risonanza dei protoni consente  il giusto compromesso tra energia depositata nel paziente, dunque effetto biologico, e un elevato numero di spin per unità di volume, ottimizzando, di conseguenza, il valore della magnetizzazione all'equilibrio;
\end{itemize}

Il vettore di magnetizzazione è valutato su un volumetto elementare contenente un numero di Avogadro \(N_{A}\) di particelle. Ne discende che in risonanza magnetica il paziente può essere considerato come un insieme di volumetti elementari, ognuno dei quali possiede il proprio vettore di magnetizzazione \(d\vec{M}\). La ricostruzione del momento magnetico permettere di eseguire l'imaging del corpo umano. Si osservi che non tutti i volumetti considerati possiedono lo stesso numero di particelle. In media, è possibile ritenere che il numero delle particelle sia pressocché lo stesso se si considerano volumetti elementari con dimensione lineare di \(1\ mm\).

Giunti all'equilibrio termodinamico, il vettore magnetizzazione non è direttamente misurabile, poiché non produce alcun segnale variabile nel tempo da captare con apposite antenne. Per ottenere un'immagine tomografica è necessario perturbare l'equilibrio termodinamico e registrare il segnale emesso dal corpo del paziente durante il ritorno all'equilibrio dei vettori di magnetizzazione di ogni singolo volumetto elementare in cui è scomponibile il paziente.

I protoni, ovvero i nuclei degli atomi di idrogeno, non sono presenti solamente nell'acqua ma sono legati anche ad altre molecole biologiche come proteine, acidi nuclei e lipidi. I nuclei di idrogeno contenuti in queste molecole non sono soggetti allo stesso campo magnetico principale imposto dall'esterno. Ciò è dovuto all'effetto di schermatura prodotto dalla molecola. Di conseguenza, gli spin di questi nuclei compieranno delle oscillazioni a frequenza diversa da quelle degli altri atomi di idrogeno. Eccitando opportunamente un tessuto, è possibile discriminare i suoi vari costituenti sulla base delle caratteristiche biochimiche.

Dalla meccanica quantistica, la transizione tra due stati \(|\psi\rangle\) e \(|\varphi\rangle\) ha un andamento nel piano \(xy\) descritto da:

\[
\gamma\hslash\cos\left( \beta - \omega_{0}t \right)
\]

Dove \(\omega_{0} = \gamma B_{0}\) è detta pulsazione di Larmor.

Le previsioni della meccanica quantistica possono essere descritte in maniera più semplice considerando gli spin orientati in modo casuale. Quando si applica un campo magnetico, gli spin, inizialmente orientati in modo casuale, si allineano lungo la direzione del campo, mentre nel piano \(xy\), traverso all'asse del campo, si instaura un moto di precessione con pulsazione angolare \(\omega_{0} = \gamma B_{0}\). In altre parole, gli spin ruotano intorno all'asse \(z\), individuato dal campo magnetico, in senso orario con frequenza \(2\pi\omega_{0}\).

\begin{figure}[ht]
\centering
\includegraphics[width=2.4729in,height=1.38542in,alt={P2775\#yIS1}]{media/6_IntroMRI/image58.pdf}
\caption{Orientamento degli spin a causa del campo}
\end{figure}

Questa assunzione, sebbene non sia esatta, consente di descrivere in modo semplice il comportamento degli spin immersi in un campo magnetico, ottenendo gli stessi risultati della meccanica quantistica.

\begin{figure}[ht]
\centering
\includegraphics[width=2.60417in,height=2.14214in,alt={P2778\#yIS1}]{media/6_IntroMRI/image59.pdf}\caption{Moto di precessione}
\end{figure}

Si suppone di applicare un campo magnetico nella direzione \(z\), convenzionalmente coincidente con l'asse maggiore del tavolo porta-paziente. Applicando uno stimolo \(B_{1}\) trasversale, si può ruotare il vettore \(\vec{M}\) sull’asse \(y\).

\begin{figure}[ht]
\centering
\includegraphics[width=6.125in,height=2.40972in,alt={P2781\#yIS1}]{media/6_IntroMRI/image60.pdf}\caption{Rotazione del vettore magnetizzazione a opera di uno stimolo esterno}
\end{figure}

Il vettore magnetizzazione globale, somma di tanti momenti magnetici intrinseci, presenta un andamento più complesso di un normale vettore; infatti, le componenti trasversali evolvono con una tempistica diversa dalle componenti longitudinali. Rimosso lo stimolo, il vettore magnetizzazione torna all'equilibrio termodinamico, emettendo un segnale dato da:

\[
s = - \dfrac{d}{dt} \int_V \vec{M} \cdot \vec{B}_{RF} \, dV
\]

dove \({\vec{B}}_{RF}\) è il campo che sarebbe erogato dall'antenna ricevente se percorsa da una certa corrente. Trascurando questo termine, il segnale registrato è proporzionale alla devirata del vettore magnetizzazione:

\[
s \propto \dfrac{dM}{dt}
\]

Generalmente l'eccitamento è di tipo sinusoidale, dunque:

\[
\dfrac{dM}{dt} = \omega_0 M = \gamma B_0 M
\]

Per la legge di Curie, scritta in termini di densità protonica:

\[
M \simeq \dfrac{N}{V}\dfrac{\gamma^{2}\hslash^{2}}{4k_{B}T}B_{0} = \rho\dfrac{\gamma^{2}\hslash^{2}}{4k_{B}T}B_{0}
\]

Il segnale registrato, sostituendo la legge di Curie, è proporzionale a:

\[
s \propto \ \gamma B_{0}M = \rho\dfrac{\gamma^{3}\hslash^{2}}{4k_{B}T}B_{0}^{2}
\]

Tralasciando i termini costanti:

\[
s \propto \rho \dfrac{\gamma^3}{T} B_0^2
\]

Il segnale dipende dalla densità protonica \(\rho\), dal rapporto giromagnetico \(\gamma\), dalla temperatura \(T\) e dal quadrato del campo magnetico \(B_0\). Per ottenere un buon rapporto segnale-rumore (o \textit{signal-to-noise ratio}, SNR), è fondamentale utilizzare campi magnetici elevati.

\section{Risonanza magnetica come tecnica spettroscopica}\label{risonanza-magnetica-come-tecnica-spettroscopica}

La risonanza magnetica nasce negli anni '50–'60 con applicazioni spettroscopiche. Questa metodica, ancora oggi molto utilizzata, consente di valutare la composizione chimica del materiale irraggiato dal campo magnetico. In ambito medico, la spettroscopia è impiegata per analizzare lo stato metabolico di un tessuto.

A causa dell’effetto di schermatura delle macromolecole, i nuclei di idrogeno non appartenenti all’acqua percepiscono un campo magnetico diverso da quello esterno. Si osservano, quindi, moti di precessione con frequenze di Larmor differenti.

\begin{figure}[ht]
\centering
\includegraphics[width=3.67692in,height=3.00926in,alt={P2800\#yIS1}]{media/6_IntroMRI/image61.pdf}\caption{Moti di precessione con frequenze diverse}
\end{figure}

L’applicazione dello stimolo non ruota tutti i momenti magnetici allo stesso modo. Il ritorno all’equilibrio produce campi magnetici variabili nel tempo, che inducono nelle antenne riceventi delle fem con contenuti frequenziali differenti.

Il segnale registrato è proporzionale alla somma dei momenti magnetici che precessano attorno all’asse \(z\), ciascuno con una propria frequenza di Larmor:

\[
s \propto \sum_k M(k)\omega(k)\exp\left( -j\omega_0 k t \right)
\]

Si ha quindi una somma di \(N\) segnali con frequenze diverse, ciascuna associata a una molecola diversa a cui il nucleo di idrogeno è legato.

Ogni tessuto biologico è caratterizzato da una specifica composizione chimica. Determinando la concentrazione dei costituenti, come le proteine, è possibile valutare lo stato di salute e l’attività metabolica del tessuto.

Poiché nei tessuti sono presenti molte molecole (acqua, proteine, acidi nucleici, lipidi, ecc.), per aumentare il numero di molecole analizzabili è necessario aumentare le dimensioni del volumetto elementare \(dV\). Tuttavia, questo comporta una riduzione della risoluzione spaziale e un aumento del SNR. Con questa scelta è possibile distinguere il contenuto metabolico del tessuto dalla restante parte di acqua e altri costituenti.

Per ottenere le immagini, oltre al campo magnetico statico di grande intensità, si applica un campo stazionario variabile lungo una direzione \(x\), \(y\) o \(z\). Si instaura così un gradiente di campo magnetico, che determina una variazione spaziale della frequenza di precessione:

\[
\omega\left( \vec{r} \right) = \gamma\vec{B}\left( \vec{r} \right)
\]

Se il gradiente è lungo \(z\), il campo magnetico è:

\[
B(z) = B_0 + G_z z
\]

Con questa soluzione, la frequenza di precessione è data da:

\[
f(z) = \dfrac{\omega(z)}{2\pi} = \dfrac{\gamma}{2\pi}\left( B_{0} + G_{z}z \right)
\]

Si definisce:

\[
\gbar = \dfrac{\gamma}{2\pi}
\]

Questa quantità, per il nucleo di idrogeno è:

\[\gbar  = \dfrac{\gamma}{2\pi} = \dfrac{2.68 \cdot 10^{8}\ \dfrac{rad}{Ts}}{2\pi} = 42.6\dfrac{MHz}{T}\]

Con un campo statico principale di \(1.5\ T\) si ha una frequenza di precessione data da:

\[
f_{0} = \gbar B_{0} = 42.6\dfrac{MHz}{T}1.5\ T \simeq 64\ MHz
\]

Per un campo di \(3\ T\), risulta invece:

\[
f_{0} = \gbar B_{0} = 42.6\dfrac{MHz}{T}3\ T \simeq 128\ MHz
\]

Queste frequenze rientrano nello spettro delle onde radio, in particolare, nella banda di frequenze normalmente utilizzate nella trasmissione FM.


L’uso dei gradienti di campo consente di variare la frequenza di precessione con la posizione. In questo modo, è possibile selezionare una singola fetta del corpo da analizzare. Una volta raggiunto l’equilibrio termodinamico, si applica uno stimolo che ribalta il vettore di magnetizzazione corrispondente alla frequenza desiderata.

\section{Momento di precessione}\label{momento-di-precessione}

Si consideri un singolo spin immerso in un campo magnetico diretto lungo l’asse \(z\). All’equilibrio termico, lo spin si allinea lungo la direzione del campo magnetico.

Dal punto di vista classico, lo spin considerato subisce una torsione meccanica:

\[
\vec{\tau} = \vec{\mu} \times \vec{B}
\]

Dove:

\[
\vec{\tau} = \dfrac{d\vec{L}}{dt}
\]

dove \(\vec{L}\) è il momento angolare, legato al momento magnetico \(\vec{\mu}\) tramite il rapporto giromagnetico \(\gamma\):

\[
\vec{\mu} = \gamma\vec{L} \Leftrightarrow \vec{L} = \dfrac{1}{\gamma}\vec{\mu}
\]

Sostituendo i due risultati nell'equazione differenziale si ha:

\[
\vec{\mu} \times \vec{B} = \dfrac{d}{dt}\left( \dfrac{1}{\gamma}\vec{\mu} \right) \Leftrightarrow \dfrac{d\vec{\mu}}{dt} = \gamma\vec{\mu} \times \vec{B0}
\]

Questa equazione differenziale, ricavata nel contesto della meccanica classica, sebbene approssimata, fornisce gli stessi risultati della meccanica quantistica, che descrive la transizione tra stati \(|+\rangle\) e \(|-\rangle\). Si dimostra che la proiezione dello spin lungo l'asse \(x\) è del tipo:

\[
\mu_{x} \propto \cos\left( \omega_{0}t \right)
\]

dove \(\omega_{0} = \gamma B_{0}\) è la frequenza di precessione di Larmor.

L'equazione differenziale:

\[
\dfrac{d\vec{\mu}}{dt} = \gamma\vec{\mu} \times \vec{B}
\]

Presenta la stessa soluzione della meccanica quantistica ma con una descrizione semplificata; per tale motivo, si ricorre alla descrizione classica.

Si considera il prodotto scalare tra \(\vec{\mu}\) e la sua derivata:

\[
\vec{\mu} \cdot \dfrac{d\vec{\mu}}{dt} = \gamma\vec{\mu} \cdot \left( \vec{\mu} \times \vec{B} \right)
\]

Il vettore \(\vec{\mu} \times \vec{B}\) è ortogonale sia al vettore \(\vec{\mu}\) che \(\vec{B}\), dunque, il prodotto scalare è nullo:

\[
\vec{\mu} \cdot \dfrac{d\vec{\mu}}{dt} = \gamma\vec{\mu} \cdot \left( \vec{\mu} \times \vec{B} \right) = 0
\]

Il prodotto scalare tra \(\vec{\mu}\) e la sua derivata può essere scritto come:

\[
\vec{\mu} \cdot \dfrac{d\vec{\mu}}{dt} = \dfrac{1}{2}\dfrac{d}{dt}\left( \vec{\mu} \cdot \vec{\mu} \right) = \dfrac{1}{2}\dfrac{d}{dt}\left| \vec{\mu} \right|^{2} = 0
\]

La derivata del modulo quadro è nulla, dunque, il modulo del momento magnetico intrinseco è costante nel tempo:

\[
\left| \vec{\mu} \right| = const
\]

Nonostante il suo modulo sia costante, la fase del momento magnetico decresce costantemente nel tempo, infatti, risulta:

\[
\dfrac{d\varphi}{dt} = - \omega
\]

Il momento magnetico precede nel piano \(xy\) in senso orario. La fase può essere scritta come:

\[
\varphi\left(t\right) = \varphi_{0} - \omega t
\]

dove \(\varphi_{0}\) è la fase iniziale.

\begin{figure}[ht]
\centering
\includegraphics[width=2.36664in,height=1.64815in,alt={P2852\#yIS1}]{media/6_IntroMRI/image62.pdf}\caption{Verso di rotazione del moto di precessione}
\end{figure}

Si risolve l'equazione differenziale per il momento magnetico. A tale scopo si scompone il prodotto vettoriale \(\vec{\mu} \times \vec{B}\) lungo gli assi:

\[
\vec{\mu} \times \vec{B} = \left| \begin{matrix}
{\hat{\imath}}_{x} & {\hat{\imath}}_{y} & {\hat{\imath}}_{z} \\
\mu_{x} & \mu_{y} & \mu_{z} \\
0 & 0 & B_{0}
\end{matrix} \right| = B_{0}\left( \mu_{y}{\hat{\imath}}_{x} - \mu_{x}{\hat{\imath}}_{y} \right)
\]

L'equazione differenziale scomposta lungo gli assi, si scrive come:

\[
\dfrac{d\vec{\mu}}{dt} = \gamma B_{0}\left( \mu_{y}{\hat{\imath}}_{x} - \mu_{x}{\hat{\imath}}_{y} \right)
\]

Proiettando l'equazione differenziale lungo gli assi, si ottiene:

\[
\begin{cases}
\dfrac{d\mu_{x}}{dt} = \gamma B_{0}\mu_{y} \\
\dfrac{d\mu_{y}}{dt} = - \gamma B_{0}\mu_{x} \\
\dfrac{d\mu_{z}}{dt} = 0
\end{cases}
\]

Si deriva la seconda equazione rispetto al tempo:

\[
\dfrac{d^{2}\mu_{y}}{dt^{2}} = - \gamma B_{0}\dfrac{d\mu_{x}}{dt}
\]

e la si sostituisce nella prima:

\[
\dfrac{d^{2}\mu_{y}}{dt^{2}} = - \gamma^{2}B_{0}^{2}\mu_{y} \Leftrightarrow \dfrac{d^{2}\mu_{y}}{dt^{2}} + \gamma^{2}B_{0}^{2}\mu_{y} = 0
\]

Passando al polinomio associato si ha:

\[
\lambda^{2} + \gamma^{2}B_{0}^{2} = 0 \Leftrightarrow \lambda = \pm j\gamma B_{0}
\]

Ponendo\(\omega_{0} = \gamma B_{0}\), la soluzione \(\mu_{y}\) è del tipo:

\[
\mu_{x}\left(t\right) = A\cos\left( \omega_{0}t \right) + B\sin\left( \omega_{0}t \right)
\]

dove \(A\) e \(B\) sono due costanti di integrazione, ottenute applicando le condizioni iniziali.

Nota l'espressione per \(\mu_{x}\left(t\right)\), è possibile ottenere quella di \(\mu_{y}\left(t\right)\); dalla prima equazione differenziale, infatti, risulta:

\[
\dfrac{d\mu_{x}}{dt} = \gamma B_{0}\mu_{y} \Leftrightarrow \mu_{y} = \dfrac{1}{\omega_{0}}\dfrac{d\mu_{x}}{dt} = \dfrac{1}{\omega_{0}}\dfrac{d}{dt}\left\lbrack A\cos\left( \omega_{0}t \right) + B\sin\left( \omega_{0}t \right) \right\rbrack
\]

Svolgendo la derivata si ha:

\[
\mu_{y}\left(t\right) =- A\sin\left( \omega_{0}t \right) + B\cos\left( \omega_{0}t \right)
\]


Lungo \(z\), la derivata del momento magnetico è nulla, dunque, \(\mu_{z}\) è costante. Si suppone di conoscere lo stato iniziale del momento:

\[
\begin{cases}
\left. \mu_{x}\left(t\right) \right|_{t = 0} = \mu_{x}\left(0\right) \\
\left. \mu_{y}\left(t\right) \right|_{t = 0} = \mu_{y}\left(0\right) \\
\left. \mu_{z}\left(t\right) \right|_{t = 0} = \mu_{z}\left(0\right)
\end{cases}
\]

Al fine di ricavare le due costanti di integrazione \(A\) e \(B\), si usano le prime due equazioni:

\[
\begin{cases}
\left. A\cos\left( \omega_{0}t \right) + B\sin\left( \omega_{0}t \right) \right|{t = 0} = \mu_{x}\left(0\right) \\
\left\lbrack - A\sin\left( \omega_{0}t \right) + B\cos\left( \omega_{0}t \right) \right\rbrack_{t = 0} = \mu_{y}\left(0\right)
\end{cases} \Leftrightarrow \begin{cases}
A = \mu_{x}\left(0\right) \\
B = \mu_{y}\left(0\right)
\end{cases}
\]

La soluzione dell'equazione differenziale è, dunque:

\[
\begin{cases}
\mu_{x}\left(t\right) = \mu_{x}\left(0\right)\cos\left( \omega_{0}t \right) + \mu_{y}\left(0\right)\sin\left( \omega_{0}t \right) \\
\mu_{y}\left(t\right) = - \mu_{x}\left(0\right)\sin\left( \omega_{0}t \right) + \mu_{y}\left(0\right)\cos\left( \omega_{0}t \right) \\
\mu_{z}\left(t\right) = \mu_{z}\left(0\right)
\end{cases}
\]

È possibile scrivere la soluzione dell'equazione differenziale:

\[
\dfrac{d\vec{\mu}}{dt} = \gamma\vec{\mu} \times \vec{B}
\]

in forma compatta introducendo la matrice di rotazione intorno all'asse \(z\):

\[
{\mathbf{R}}_{z}\left( \omega_{0}t \right) = \begin{pmatrix}
\cos\left( \omega_{0}t \right) & \sin\left( \omega_{0}t \right) & 0 \\
 - \sin\left( \omega_{0}t \right) & \cos\left( \omega_{0}t \right) & 0 \\
0 & 0 & 1
\end{pmatrix}
\]

Con questa posizione, il momento magnetico \(\vec{\mu}\) in funzione del tempo è dato da:

\[\vec{\mu}\left(t\right) = {\mathbf{R}}_{z}\left( \omega_{0}t \right)\vec{\mu}\left(0\right)\]

\subsection{Rappresentazione complessa}\label{rappresentazione-complessa}


Poiché gli spin dei protoni eseguono un moto di precessione attorno all’asse \(z\), individuato dal campo magnetico, è utile introdurre una notazione nel piano complesso per descrivere il comportamento nel piano trasverso al campo. Si definisce il fasore \(\mu_{\perp}\left(t\right)\) come:

\[
\mu_{\perp}\left(t\right) = \mu_{x}\left(t\right) + j\mu_{y}\left(t\right)
\]

Il fasore permette di descrivere il movimento nel piano trasverso del momento magnetico nel tempo.

La derivata temporale di un fasore si scrive come:

\[
\dfrac{d\mu_{\perp}}{dt} = - j\omega_{0}\mu_{\perp}
\]

dove \(\omega_{0} = \gamma B_{0}\) è la frequenza di precessione di Larmor. Per definizione di \(\mu_{\perp}\left(t\right)\), si ha:

\[
\dfrac{d\mu_{\perp}}{dt} = - j\omega_{0}\mu_{\perp} = - j\omega_{0}\mu_{x}\left(t\right) - j\omega_{0}\left\lbrack j\mu_{y}\left(t\right) \right\rbrack 
\]

Inoltre, per la linearità dell'operazione derivata, deve risultare:

\[
\dfrac{d\mu_{\perp}}{dt} = \dfrac{d\mu_{x}}{dt} + j\dfrac{d\mu_{y}}{dt}
\]

Confrontando le due espressioni risulta che:

\[
\begin{cases} \dfrac{d\mu_{x}}{dt} = \omega_{0}\mu_{y}, & \text{(Parte Reale)} \\ \dfrac{d\mu_{y}}{dt} = - \omega_{0}\mu_{x}, & \text{(Parte Immaginaria)} \end{cases}
\]


La notazione fasoriale semplifica la descrizione del moto rispetto al sistema di equazioni differenziali. Integrando per variabili separabili:

\[
\dfrac{d\mu_{\perp}}{dt} = - j\omega_{0}\mu_{\perp} \Leftrightarrow \dfrac{1}{\mu_{\perp}}d\mu_{\perp} = - j\omega_{0}\ dt
\]

Da cui si ottiene:

\[
\mu_{\perp}\left(t\right) = \mu_{\perp}\left(0\right)\exp\left( - j\omega_{0}t \right)
\]

Questa soluzione coincide con quella del vettore momento magnetico, ristretto al piano \(xy\), nel dominio del tempo. Il termine esponenziale, \(\exp\left( - j\omega_{0}t \right)\), rappresenta una rotazione in senso orario nel piano complesso o, equivalentemente, intorno all'asse \(z\) individuato dal campo principale. La rotazione nel piano \(xy\) è dovuto alla rotazione del momento magnetico intrinseco dello spin sul piano trasverso al campo applicato.

La fase del fasore è strettamente correlata alla pulsazione con cui il vettore momento magnetico intrinseco ruota nel piano trasverso. La conoscenza della posizione, ovvero della fase, è fondamentale per ricostruire l'immagine di risonanza magnetica. Per cui, data l'importanza della fase, è molto utile introdurre una notazione complessa per tale quantità, in modo da esplicitarla.

In generale, un numero complesso può essere espresso esplicitando modulo e fase:

\[
\mu_{\perp}\left(t\right) = \left| \mu_{\perp}\left(t\right) \right|\exp\left( - j\varphi\left(t\right) \right)
\]

Il modulo del fasore è costante nel tempo, infatti, nel dominio del tempo si è dimostrato che:

\[
\vec{\mu} \cdot \dfrac{d\vec{\mu}}{dt} = 0
\]

La fase, invece, è data da:

\[
\angle\mu_{\perp} = \varphi\left(t\right) + \angle\mu\left(0\right) = - j\omega_{0}t + \varphi_{0}
\]

Dove \(\varphi_0 = \angle \mu_{\perp}\left(0\right)\) è la fase iniziale. L'evoluzione del fasore \(\mu_{\perp}\) può essere scritto come:

\[
\mu_{\perp}\left(t\right) = \left| \mu_{\perp}\left(0\right)\right| \exp\left( - j\left( \omega_{0}t + \varphi_{0} \right) \right)  
\]

Conoscendo la fase iniziale \(\varphi_0\) e la pulsazione di Larmor \(\omega_0 = \gamma B_0\), è possibile determinare l’evoluzione della posizione dello spin nel piano trasverso.

\section[Sistema di riferimento rotante con velocità angolare omega]{Sistema di riferimento rotante con velocità angolare $\boldsymbol{\omega}$}
\label{sistema-di-riferimento-rotante-con-velocita-angolare-omega}

Applicando un campo magnetico lungo un asse di un sistema di riferimento fisso, gli spin si orientano lungo quell'asse, generalmente coincidente con \(z\), compiendo un moto di precessione in condizioni di equilibrio termodinamico. Per ottenere un segnale misurabile, è necessario perturbare tale equilibrio mediante un campo elettromagnetico esterno alla frequenza opportuna.

Per descrivere la perturbazione, si introduce un sistema di riferimento rotante con pulsazione \(\omega\), che ruota in senso orario attorno all’asse \(z\). Sia \((x', y', z')\) il sistema rotante. Nel sistema fisso del laboratorio, la velocità angolare del sistema rotante è:

\[
\vec{\Omega} = -\omega \hat{\mathbf{z}}
\]

Il segno negativo indica la rotazione oraria. La pulsazione \(\omega\) può variare nel tempo.

Si vuole determinare una relazione che leghi un vettore nel sistema fisso alle componenti del vettore nel sistema rotante. Ogni vettore \(\vec{c}\), che ruota con velocità angolare \(\vec{\Omega}\), presenta una variazione temporale rispetto al sistema di riferimento fisso del laboratorio, descritta dalla relazione:

\[
\dfrac{d\vec{c}}{dt} = \vec{\Omega} \times \vec{c}
\]

Il vettore \(\vec{c}\) può essere espresso come somma della componente parallela all'asse \(z\) e della componente trasversale:

\[
\vec{c} = \vec{c}_{\parallel} + \vec{c}_{\perp}
\]

Poiché \(\vec{\Omega}\) ha solo componente lungo \(z\), il prodotto vettoriale tra \(\vec{\Omega}\) e la componente parallela all'asse \(z\) di \(\vec{c}\), \(\vec{c}_{\parallel}\), è nullo:

\[
\vec{\Omega} \times \vec{c}_{\parallel} = \vec{0}
\]

Se il vettore \(\vec{c}\) è costante nel sistema di riferimento fisso, la sua derivata temporale è nulla:

\[
\dfrac{d\vec{c}}{dt} = \vec{0}
\]

Nel sistema di riferimento rotante, il vettore \(\vec{c}\) non è costante; infatti, questo sistema vede il vettore ruotare, per cui risulta:

\[
\left( \dfrac{d\vec{c}}{dt} \right)' = \vec{\Omega} \times {\vec{c}}' \neq \vec{0}
\]

dove \(\left( d\vec{c}/dt \right)'\) indica la derivata del vettore \(\vec{c}\) nel sistema di riferimento rotante \(\left(x', y', z'\right)\).

Nel sistema di riferimento fisso, il vettore \(\vec{c}\left(t\right)\) ha componenti:

\[
\vec{c}\left(t\right) = c_{x}\left(t\right){\hat{\imath}}_{x} + c_{y}\left(t\right){\hat{\imath}}_{y} + c_{z}\left(t\right){\hat{\imath}}_{z}
\]

Il tempo è considerato invariante nei due sistemi di riferimento, poiché le velocità considerate sono molto inferiori alla velocità della luce. 

Nel sistema rotante, invece, il vettore \(\vec{c}\) possiede componenti diverse:

\[
{\vec{c}}'\left(t\right) = c_{x'}\left(t\right){\hat{\imath}}_{x'} + c_{y'}\left(t\right){\hat{\imath}}_{y'} + c_{z'}\left(t\right){\hat{\imath}}_{z'}
\]

Dato che i due sistemi possiedono l'asse \(z\) in comune, deve risultare \({\hat{\imath}}_{z} \equiv {\hat{\imath}}_{z'}\), per cui:

\[
{\vec{c}}'\left(t\right) = c_{x'}\left(t\right){\hat{\imath}}_{x'} + c_{y'}\left(t\right){\hat{\imath}}_{y'} + c_{z}\left(t\right){\hat{\imath}}_{z}
\]

Si osserva che non è necessario distinguere \(\vec{c}\left(t\right)\) nel sistema fisso e \({\vec{c}}'\left(t\right)\) nel sistema rotante in quanto il vettore è lo stesso per entrambi i sistemi di riferimento.

Le componenti del vettore nel sistema fisso dipendono da quelle nel sistema rotante e viceversa. Si calcola la derivata rispetto al tempo del vettore \(\vec{c}\left(t\right)\) nel sistema fisso di riferimento:

\[
\dfrac{d\vec{c}}{dt} = \dfrac{d}{dt}\left( c_{x}\left(t\right){\hat{\imath}}_{x} + c_{y}\left(t\right){\hat{\imath}}_{y} + c_{z}\left(t\right){\hat{\imath}}_{z} \right) = \dfrac{dc_{x}}{dt}{\hat{\imath}}_{x} + c_{x}\dfrac{d{\hat{\imath}}_{x}}{dt} + \dfrac{dc_{y}}{dt}{\hat{\imath}}_{y} + c_{y}\dfrac{d{\hat{\imath}}_{y}}{dt} + \dfrac{dc_{z}}{dt}{\hat{\imath}}_{z} + c_{z}\dfrac{d{\hat{\imath}}_{z}}{dt}
\]

Poiché il sistema di riferimento è fisso, le derivate dei suoi versori sono nulle:

\[
\dfrac{d{\hat{\imath}}_{x}}{dt} = 0,\ \ \dfrac{d{\hat{\imath}}_{y}}{dt} = 0,\ \ \dfrac{d{\hat{\imath}}_{z}}{dt} = 0
\]

Si ottiene:

\[
\dfrac{d\vec{c}}{dt} = \dfrac{dc_{x}}{dt}{\hat{\imath}}_{x} + \dfrac{dc_{y}}{dt}{\hat{\imath}}_{y} + \dfrac{dc_{z}}{dt}{\hat{\imath}}_{z}
\]

Nel sistema rotante, invece, la derivata è:

\[
\dfrac{d\vec{c}}{dt} = \dfrac{dc_{x'}}{dt}{\hat{\imath}}_{x'} + c_{x'}\dfrac{d{\hat{\imath}}_{x'}}{dt} + \dfrac{dc_{y'}}{dt}{\hat{\imath}}_{y'} + c_{y'}\dfrac{d{\hat{\imath}}_{y'}}{dt} + \dfrac{dc_{z'}}{dt}{\hat{\imath}}_{z'} + c_{z'}\dfrac{d{\hat{\imath}}_{z'}}{dt}
\]

La relazione \(d\vec{c}/dt = \vec{\Omega} \times \vec{c}\) è valida per un generico vettore \(\vec{c}\), dunque, è valida anche per i versori degli assi coordinati:

\[
\dfrac{d{\hat{\imath}}_{x'}}{dt} = \vec{\Omega} \times {\hat{\imath}}_{x'},\ \ \dfrac{d{\hat{\imath}}_{y'}}{dt} = \vec{\Omega} \times {\hat{\imath}}_{y'},\ \ \dfrac{d{\hat{\imath}}_{z'}}{dt} = \vec{\Omega} \times {\hat{\imath}}_{z'}
\]

Sostituendo queste relazioni, la derivata nel sistema rotante diventa:

\begin{align}
\dfrac{d\vec{c}}{dt} &=
\dfrac{dc_{x'}}{dt}{\hat{\imath}}_{x'} + c_{x'}\vec{\Omega} \times {\hat{\imath}}_{x'}
 + \dfrac{dc_{y'}}{dt}{\hat{\imath}}_{y'} + c_{y'}\vec{\Omega} \times {\hat{\imath}}_{y'}  + \dfrac{dc_{z'}}{dt}{\hat{\imath}}_{z'} + c_{z'}\vec{\Omega} \times {\hat{\imath}}_{z'} \nonumber\\[6pt]
&= \dfrac{dc_{x'}}{dt}{\hat{\imath}}_{x'} +
\dfrac{dc_{y'}}{dt}{\hat{\imath}}_{y'} +
\dfrac{dc_{z'}}{dt}{\hat{\imath}}_{z'} +  \vec{\Omega} \times
\left(c_{x'}{\hat{\imath}}_{x'} + c_{y'}{\hat{\imath}}_{y'} + c_{z'}{\hat{\imath}}_{z'}\right)
\end{align}

dove \({\vec{c}}' = c_{x'}{\hat{\imath}}_{x'} + c_{y'}{\hat{\imath}}_{y'} + c_{z'}{\hat{\imath}}_{z'}\) è il vettore nel sistema di riferimento fisso. Si indica con:

\[
\left( \dfrac{d\vec{c}}{dt} \right)' = \dfrac{dc_{x'}}{dt}{\hat{\imath}}_{x'} + \dfrac{dc_{y'}}{dt}{\hat{\imath}}_{y'} + \dfrac{dc_{z'}}{dt}{\hat{\imath}}_{z'}
\]

La derivata del vettore \(\vec{c}\) nel sistema di riferimento rotante, i cui versori sono fissi. La derivata nel sistema di riferimento rotante può essere scritta come:

\[
\dfrac{d\vec{c}}{dt} = \left( \dfrac{d\vec{c}}{dt} \right)' + \vec{\Omega} \times {\vec{c}}'
\]

Questa relazione mostra che la variazione temporale di un vettore in un sistema rotante è data dalla somma della derivata nel sistema rotante e del termine di rotazione.

Si considera, come vettore \(\vec{c}\), il momento magnetico \(\vec{\mu}\) di uno spin immerso in un campo magnetico uniforme \(\vec{B}\), diretto lungo l'asse \(z\):

\[
\dfrac{d\vec{\mu}}{dt} = \left( \dfrac{d\vec{\mu}}{dt} \right)' + \vec{\Omega} \times \vec{\mu}
\]

La variazione del momento magnetico è legata al campo magnetico dalla relazione:

\[
\dfrac{d\vec{\mu}}{dt} = \gamma\vec{\mu} \times \vec{B}
\]

Sostituendo questa espressione nella relazione del sistema rotante, si ottiene:

\[
\gamma\vec{\mu} \times \vec{B} = \left( \dfrac{d\vec{\mu}}{dt} \right)' + \vec{\Omega} \times \vec{\mu}
\]

Ricavando la derivata nel sistema rotante si ha:

\[
\left( \dfrac{d\vec{\mu}}{dt} \right)' = \gamma\vec{\mu} \times \vec{B} - \vec{\Omega} \times \vec{\mu}
\]

Poiché è valido il risultato \(-\vec{\Omega} \times \vec{\mu} = \vec{\mu} \times \vec{\Omega}\), si può riscrivere:

\[
\left( \dfrac{d\vec{\mu}}{dt} \right)' = \gamma\vec{\mu} \times \vec{B} + \vec{\mu} \times \vec{\Omega} = \gamma\vec{\mu} \times \left( \vec{B} + \dfrac{1}{\gamma}\vec{\Omega} \right)
\]

Si definisce \textbf{campo magnetico effettivo} o \textbf{efficace} come:

\[
\vec{B}_{\text{eff}} = \vec{B} + \dfrac{1}{\gamma}\vec{\Omega}
\]

L'equazione per il momento magnetico nel sistema rotante assume la forma:

\[
\left( \dfrac{d\vec{\mu}}{dt} \right)' = \gamma\vec{\mu} \times {\vec{B}_{\text{eff}}}
\]

Si ottiene la stessa forma dell'equazione nel sistema fisso, a patto di sostituire al campo magnetico applicato \(\vec{B}\) il campo magnetico efficace \(\vec{B}_{\text{eff}}\), che include un termine fittizio dovuto alla rotazione del sistema. Il campo efficace è quello che lo spin avverte nel sistema di riferimento rotante.

Si suppone che il sistema ruoti intorno all'asse \(z\) in senso orario con velocità angolare \(\omega\):

\[
\vec{\Omega} = - \omega{\hat{\imath}}_{z}
\]

In queste condizioni, il campo efficace è:

\[
{\vec{B}_{\text{eff}}} = B_{0}{\hat{\imath}}_{z} - \dfrac{1}{\gamma}\omega{\hat{\imath}}_{z}
\]

Se la frequenza con cui ruota il sistema coincide con quella di Larmor, ovvero \(\omega = \omega_{0} = \gamma B_{0}\), allora:

\[
{\vec{B}_{\text{eff}}} = B_{0}{\hat{\imath}}_{z} - \dfrac{1}{\gamma}\omega_{0}{\hat{\imath}}_{z} = B_{0}{\hat{\imath}}_{z} - \dfrac{1}{\gamma}\gamma B_{0}{\hat{\imath}}_{z} = \vec{0}
\]


In questo caso, il momento magnetico \(\vec{\mu}\) appare fisso nel sistema di riferimento rotante. Ne discende che tale sistema è solidale con gli spin.

L'introduzione del sistema rotante permette di descrivere in modo semplice il segnale registrato. Nella pratica, tuttavia, non è semplice produrre un campo con la stessa frequenza con cui risuonano i protoni, quindi il campo efficace non è mai nullo.

\subsection{Rotazione del sistema di riferimento per un campo a radiofrequenza}\label{rotazione-del-sistema-di-riferimento-per-un-campo-a-radiofrequenza}

Si considera un protone allineato al campo magnetico esterno, diretto lungo l'asse \(z\). Il protone risuona, ovvero compie un moto di precessione alla frequenza di Larmor nel piano trasverso. Per eccitare lo spin e poter effettuare una misura, è necessario stimolarlo mediante un campo elettromagnetico alla frequenza di Larmor. Questo campo magnetico, essendo oscillante, possiede una polarizzazione.

Si suppone che la polarizzazione del campo eccitante sia lineare, ovvero che il campo sia diretto solamente lungo un asse, ad esempio \(x\). Il campo ha quindi un andamento del tipo:


\[
{\vec{B}}_{x}\left(t\right) = b_{x}\left(t\right)\cos\left( \omega_{0}t \right){\hat{\imath}}_{x}
\]

In questa espressione, \(b_{x}\left(t\right)\) rappresenta l'ampiezza modulata del segnale, che solitamente è un pacchetto di sinusoidi.

L'uso del sistema rotante semplifica la descrizione del campo magnetico percepito dallo spin. In questo sistema, i versori sono legati a quelli del sistema fisso dalle relazioni:

\[
\begin{cases}
{\hat{\imath}}_{x'} = {\hat{\imath}}_{x}\cos\left( \omega_{0}t \right) - {\hat{\imath}}_{y}\sin\left( \omega_{0}t \right) \\
{\hat{\imath}}_{y'} = {\hat{\imath}}_{x}\sin\left( \omega_{0}t \right) + {\hat{\imath}}_{y}\cos\left( \omega_{0}t \right)
\end{cases} 
\]

Si scrive il sistema in forma matriciale, in modo poter ricavare i versori del sistema fisso in funzione di quelli del sistema rotante.

\[\begin{pmatrix}
{\hat{\imath}}_{x'} \\
{\hat{\imath}}_{y'}
\end{pmatrix} = \begin{pmatrix}
\cos\left( \omega_{0}t \right) & - \sin\left( \omega_{0}t \right) \\
\sin\left( \omega_{0}t \right) & \cos\left( \omega_{0}t \right)
\end{pmatrix}\begin{pmatrix}
{\hat{\imath}}_{x} \\
{\hat{\imath}}_{y}
\end{pmatrix}\]

dove \(\mathbf{R}_{z}\) è la matrice di rotazione intorno all'asse \(z\):

\[
\mathbf{R}_{z}\left( \omega_{0}t \right) = \begin{pmatrix}
\cos\left( \omega_{0}t \right) & - \sin\left( \omega_{0}t \right) \\
\sin\left( \omega_{0}t \right) & \cos\left( \omega_{0}t \right)
\end{pmatrix}
\]

Il suo determinante è unitario:

\[
\mathbf{R}_{z}\left( \omega_{0}t \right) = \begin{vmatrix}
\cos\left( \omega_{0}t \right) & - \sin\left( \omega_{0}t \right) \\
\sin\left( \omega_{0}t \right) & \cos\left( \omega_{0}t \right)
\end{vmatrix} = \cos^{2}\left( \omega_{0}t \right) + \sin^{2}\left( \omega_{0}t \right) = 1
\]

La matrice \(\mathbf{R}_{z}\left( \omega_{0}t \right)\) è invertibile, con inversa:

\[
\mathbf{R}_{z}^{- 1}\left( \omega_{0}t \right) = \begin{pmatrix}
\cos\left( \omega_{0}t \right) & \sin\left( \omega_{0}t \right) \\
 - \sin\left( \omega_{0}t \right) & \cos\left( \omega_{0}t \right)
\end{pmatrix}
\]

La relazione inversa tra i versori è quindi:

\[
\begin{pmatrix}
{\hat{\imath}}_{x} \\
{\hat{\imath}}_{y}
\end{pmatrix} = \begin{pmatrix}
\cos\left( \omega_{0}t \right) & \sin\left( \omega_{0}t \right) \\
 - \sin\left( \omega_{0}t \right) & \cos\left( \omega_{0}t \right)
\end{pmatrix}\begin{pmatrix}
{\hat{\imath}}_{x'} \\
{\hat{\imath}}_{y'}
\end{pmatrix}
\]

Il versore relativo all'asse \(x\) del sistema fisso, in funzione dei versori del sistema rotante, è:

\[{
\hat{\imath}}_{x} = {\hat{\imath}}_{x'}\cos\left( \omega_{0}t \right) + {\hat{\imath}}_{y'}\sin\left( \omega_{0}t \right)
\]

Il campo magnetico polarizzato linearmente, nel sistema rotante, può essere scritto come:

\[
\begin{aligned}
\vec{B}_{x}\!\left(t\right) 
&= b_{x}\!\left(t\right)\cos\!\left(\omega_{0}t\right)\,\hat{\imath}_{x} \\
&= b_{x}\!\left(t\right)\cos\!\left(\omega_{0}t\right)
\left[\hat{\imath}_{x'}\cos\!\left(\omega_{0}t\right)+ \hat{\imath}_{y'}\sin\!\left(\omega_{0}t\right)\right] \\
&= b_{x}\!\left(t\right)\cos^{2}\!\left(\omega_{0}t\right)\,\hat{\imath}_{x'}+b_{x}\!\left(t\right)\cos\!\left(\omega_{0}t\right)\sin\!\left(\omega_{0}t\right)\,\hat{\imath}_{y'}
\end{aligned}
\]

Utilizzando le formule di duplicazione, si ottiene:

\[
{\vec{B}}_{x}\left(t\right) = b_{x}\left(t\right)\left\lbrack \dfrac{1}{2}\cos\left( 2\omega_{0}t \right) + \dfrac{1}{2} \right\rbrack{\hat{\imath}}_{x'} + \dfrac{1}{2}b_{x}\left(t\right)\sin\left( 2\omega_{0}t \right){\hat{\imath}}_{y'}
\]

Nel sistema rotante, il campo magnetico lineare può essere visto come somma di una componente costante e due componenti oscillanti a frequenza doppia:

\[
{\vec{B}}_{x}\left(t\right) = \dfrac{1}{2}b_{x}\left(t\right){\hat{\imath}}_{x'} + b_{x}\left(t\right)\left\lbrack \dfrac{1}{2}\cos\left( 2\omega_{0}t \right){\hat{\imath}}_{x'} + \dfrac{1}{2}\sin\left( 2\omega_{0}t \right){\hat{\imath}}_{y'} \right\rbrack
\]

Si calcola il valor medio del campo magnetico su un intervallo di tempo sufficientemente lungo, come un periodo dell'onda a radiofrequenza, \(T\):

\[
\left\langle {\vec{B}}_{x}\left(t\right) \right\rangle = \dfrac{1}{T}\int_{T}^{}{{\vec{B}}_{x}\left(t\right)dt} = \dfrac{1}{2T}\int_{T}^{}{\left| \left\{ b_{x}\left(t\right){\hat{\imath}}_{x'} + b_{x}\left(t\right)\left\lbrack \cos\left( 2\omega_{0}t \right){\hat{\imath}}_{x'} + \sin\left( 2\omega_{0}t \right){\hat{\imath}}_{y'} \right\rbrack \right\} \right|dt} = \dfrac{1}{2}\left\langle b_{x}\left(t\right) \right\rangle
\]

Gli altri termini sono nulli poiché termini sinusoidali integrati su un intervallo temporale uguale al doppio del periodo di oscillazione. L'applicazione del valor medio implica che solo metà della polarizzazione lineare è utilizzata per ruotare gli spin intorno all'asse \(x\). Per un tempo sufficientemente lungo il campo può essere considerato a media costante.

In molte applicazioni, per migliorare la precisione nella ricostruzione delle immagini, si utilizza un campo magnetico con polarizzazione circolare, ottenuto sovrapponendo due campi lineari di uguale intensità, in quadratura e diretti lungo assi ortogonali.

\begin{figure}[ht]
\centering
\includegraphics[width=2.96296in,height=1.78624in,alt={P3006\#yIS1}]{media/6_IntroMRI/image63.pdf}\caption{Sovrapposizione di onde ortogonali}
\end{figure}


A tale scopo si posizionano due antenne ortogonali tra loro, ciascuna delle quali genera un campo magnetico lineare. Affinché la polarizzazione risultante sia circolare, è necessario che i due campi siano in quadratura tra loro. Il campo totale, polarizzato circolarmente, è espresso nel sistema di riferimento fisso come:

\[
\vec{B}\left(t\right) = B_{1}\left\lbrack \cos\left( \omega_{0}t \right){\hat{\imath}}_{x} - \sin\left( \omega_{0}t \right){\hat{\imath}}_{y} \right\rbrack
\]

Si ricava ora l'espressione del campo magnetico polarizzato circolarmente nel sistema di riferimento rotante alla frequenza di Larmor. A tale scopo, si considerano le relazioni tra i versori dei due sistemi:

\[\begin{pmatrix}
{\hat{\imath}}_{x} \\
{\hat{\imath}}_{y}
\end{pmatrix} = \begin{pmatrix}
\cos\left( \omega_{0}t \right) & \sin\left( \omega_{0}t \right) \\
 - \sin\left( \omega_{0}t \right) & \cos\left( \omega_{0}t \right)
\end{pmatrix}\begin{pmatrix}
{\hat{\imath}}_{x'} \\
{\hat{\imath}}_{y'}
\end{pmatrix} \Leftrightarrow \begin{cases}
{\hat{\imath}}_{x} = \cos\left( \omega_{0}t \right){\hat{\imath}}_{x'} + \sin\left( \omega_{0}t \right){\hat{\imath}}_{y'} \\
{\hat{\imath}}_{y} = - \sin\left( \omega_{0}t \right){\hat{\imath}}_{x'} + \cos\left( \omega_{0}t \right){\hat{\imath}}_{y'}
\end{cases} 
\]

Sostituendo queste relazioni nell'espressione del campo magnetico circolare si ha:

\[
\begin{aligned}
\vec{B}\!\left(t\right)
&= B_{1}\!\left[\cos\!\left(\omega_{0}t\right)\hat{\imath}_{x}
- \sin\!\left(\omega_{0}t\right)\hat{\imath}_{y}
\right] \\
&= B_{1}\!\left\{\cos\!\left(\omega_{0}t\right)\left[ \cos\!\left(\omega_{0}t\right)\hat{\imath}_{x'} + \sin\!\left(\omega_{0}t\right)\hat{\imath}_{y'}\right] - \sin\!\left(\omega_{0}t\right)\left[-\sin\!\left(\omega_{0}t\right)\hat{\imath}_{x'}+ \cos\!\left(\omega_{0}t\right)\hat{\imath}_{y'} \right] \right\} \\[6pt]
&= B_{1}\!\left\{
\cos^{2}\!\left(\omega_{0}t\right)\hat{\imath}_{x'} + \cos\!\left(\omega_{0}t\right)\sin\!\left(\omega_{0}t\right)\hat{\imath}_{y'} + \sin^{2}\!\left(\omega_{0}t\right)\hat{\imath}_{x'} - \sin\!\left(\omega_{0}t\right)\cos\!\left(\omega_{0}t\right)\hat{\imath}_{y'}\right\} \\
&= B_{1}\!\left[\cos^{2}\!\left(\omega_{0}t\right)\hat{\imath}_{x'}+ \sin^{2}\!\left(\omega_{0}t\right)\hat{\imath}_{x'}\right] \\
&= B_{1}\,\hat{\imath}_{x'}
\end{aligned}
\]

Per ogni istante temporale, il campo magnetico a polarizzazione circolare con frequenza uguale a quella del sistema rotante è orientato lungo l'asse \(x'\). A differenza del caso lineare, questa relazione è valida per ogni istante di tempo, poiché non è ottenuta mediante operazione di media.

Riassumendo, quando si applica un impulso a radiofrequenza con polarizzazione circolare, gli spin ruotano intorno all'asse \(x'\), lungo il quale iniziano un moto di precessione.

È possibile dimostrare che la generazione di un campo a polarizzazione circolare comporta un minor dispendio energetico. Per questo motivo, insieme ad altri fattori come il rapporto segnale/rumore e la necessità di omogeneizzare il campo a radiofrequenza, i campi rotanti a polarizzazione circolare sono ampiamente utilizzati in risonanza magnetica.

\subsection{Condizione di risonanza}\label{condizione-di-risonanza}

Si applica un campo a radiofrequenza con polarizzazione lineare o circolare; nel primo caso è necessario considerare quantità medie, mentre nel secondo si ottengono relazioni esatte dal punto di vista teorico. Nel sistema rotante, il campo magnetico percepito dallo spin è detto \textbf{campo effettivo}.

Prima dell'applicazione dell'impulso, l'equazione che descrive il comportamento dello spin, dal punto di vista classico, nel sistema rotante è:

\[
\left( \dfrac{d\vec{\mu}}{dt} \right)' = \gamma\vec{\mu} \times {\vec{B}_{\text{eff}}}
\]

dove:

\[
{\vec{B}_{\text{eff}}} = B_{0}{\hat{\imath}}_{z} - \dfrac{1}{\gamma}\omega{\hat{\imath}}_{z}
\]

Il termine \(B_0 \hat{\mathbf{z}}\) rappresenta il campo statico esterno o principale applicato.

Si applica ora un campo a radiofrequenza. Se il sistema di riferimento ruota con la stessa pulsazione \(\omega_0\) del campo magnetico applicato, al campo effettivo \(\vec{B}_{\text{eff}}\) va aggiunto un termine costante diretto lungo \(\hat{\mathbf{x}}'\):

\[
\left( \dfrac{d\vec{\mu}}{dt} \right)' = \gamma\vec{\mu} \times \left( B_{0}{\hat{\imath}}_{z} - \dfrac{1}{\gamma}\omega{\hat{\imath}}_{z} + B_{1}{\hat{\imath}}_{x'} \right)
\]

Poiché i due sistemi di riferimento condividono l'asse \(z\), si ha \(\hat{\mathbf{z}} = \hat{\mathbf{z}}'\).

Si parla di \textbf{condizione di risonanza} quando la frequenza dell'impulso a radiofrequenza è uguale alla frequenza di Larmor degli spin contenuti in uno strato di tessuto:

\[
\omega = \omega_{0} = \gamma B_{0}
\]

In tal caso, l'equazione diventa:

\[
\left( \dfrac{d\vec{\mu}}{dt} \right)' = \gamma B_{1}\vec{\mu} \times {\hat{\imath}}_{x'}
\]

Si osserva quindi una precessione degli spin intorno all'asse \(\hat{\mathbf{x}}'\).

\begin{figure}[ht]
\centering
\includegraphics[width=4.4166in,height=3.61111in,alt={P3032\#yIS1}]{media/6_IntroMRI/image64.pdf}\caption{Precessione nel sistema rotante in condizione di risonanza}
\end{figure}

Il termine \textbf{risonanza}, in questa tecnica di imaging, si riferisce al fatto che, per ottenere l'immagine, il campo magnetico applicato è sintonizzato con la frequenza di precessione degli spin dei tessuti di interesse.

L'introduzione del sistema rotante consente di descrivere in modo semplice il comportamento degli spin, mediante un moto di precessione intorno all'asse \(x'\). Nel sistema di riferimento fisso del laboratorio, invece, il contributo del campo a radiofrequenza si somma a quello statico e omogeneo, diretto lungo \(\hat{\mathbf{z}}\). Si dimostra che la combinazione dei due campi produce un moto elicoidale, il cui raggio aumenta all'avvicinarsi al piano \(xy\).

\begin{figure}[ht]
\centering
\includegraphics[width=4.13528in,height=2.63095in,alt={P3036\#yIS1}]{media/6_IntroMRI/image65.pdf}\caption{Moto elicoidale nel sistema fisso}
\end{figure}

Nel sistema rotante con pulsazione \(\omega\), il campo efficace può essere scritto come:

\[
{\vec{B}_{\text{eff}}} = B_{0}{\hat{\imath}}_{z} - \dfrac{1}{\gamma}\omega{\hat{\imath}}_{z} + B_{1}{\hat{\imath}}_{x'}
\]

Con pulsazione di Larmor:

\[
\omega = \gamma B_{0} \Leftrightarrow B_{0} = \dfrac{1}{\gamma}\omega_{0}
\]

La pulsazione del campo a radiofrequenza applicato è:

\[
\omega_{1} = \gamma B_{1} \Leftrightarrow B_{1} = \dfrac{1}{\gamma}\omega_{1}
\]

Pertanto, il campo efficace può essere espresso come:

\[
{\vec{B}_{\text{eff}}} = \dfrac{1}{\gamma}\left( \left( \omega_{0} - \omega \right){\hat{\imath}}_{z} + \omega_{1}{\hat{\imath}}_{x'} \right)
\]

In generale, ottenere un campo a radiofrequenza con frequenza esattamente pari a quella di Larmor è complesso, a causa della schermatura che alcune molecole esercitano sui campi magnetici. Inoltre, la frequenza di rotazione del sistema rotante può essere scelta arbitrariamente rispetto sia al campo a radiofrequenza che alla frequenza di precessione degli spin. In tal caso, la descrizione del sistema rotante è altrettanto complessa quanto quella del sistema fisso. I vantaggi del sistema rotante si manifestano quando \(\omega_1 = \omega_0 = \omega\), poiché il vettore momento magnetico ruota intorno all'asse \(x'\).

Generalmente, gli impulsi a radiofrequenza hanno ampiezza dell'ordine di qualche \(mT\) o \(\mu T\). Si suppone, ad esempio, di applicare un impulso a radiofrequenza per un tempo \(\tau = 1\ ms\), tale da ruotare il momento magnetico di un angolo \(\pi/2\). L'ampiezza del campo a radiofrequenza necessaria è data da:


\[
\Delta \vartheta = \gamma B_{1}\tau
\]

dove \(\gamma = 42.6\ MHz/T\). Si ottiene:

\[
B_{1} = \dfrac{\Delta \vartheta}{\gamma\tau} = \dfrac{\dfrac{\pi}{2}}{42.6\ \dfrac{MHz}{T}1\ ms} \simeq 5.9\ \mu T
\]

Il ribaltamento degli spin a opera del campo magnetico a radiofrequenza è detto in gergo \textit{to flip}.

La principale difficoltà tecnica della risonanza magnetica consiste nella produzione del campo magnetico principale, costante nel tempo e omogeneo nello spazio.

I campi a radiofrequenza con frequenza di Larmor riescono a flippare gli spin anche con intensità molto minore rispetto al campo principale. Tuttavia, quanto più la frequenza del campo a radiofrequenza si discosta dalla frequenza di Larmor, tanto più il campo efficace nel sistema rotante tende a quello statico, e minore è il numero di spin ruotati di \(\Delta \vartheta\). Le frequenze degli impulsi devono essere compatibili con i tempi di rilassamento del corpo in esame.

\subsection{Calcolo del campo magnetico a radiofrequenza}\label{calcolo-del-campo-magnetico-a-radiofrequenza}

L'uso del sistema rotante intorno all'asse \(\hat{\mathbf{z}}\) permette di descrivere un campo a polarizzazione circolare in modo molto semplice. Inoltre, se la frequenza del campo a radiofrequenza e quella del sistema rotante coincidono con la frequenza di Larmor, è possibile scrivere la relazione:

\[
\left( \dfrac{d\vec{\mu}}{dt} \right)' = \gamma B_{1}\vec{\mu} \times {\hat{\imath}}_{x'}
\]

dove \(\vec{B}_1 = B_1 \hat{\mathbf{x}}'\) è il campo magnetico polarizzato circolarmente visto nel sistema rotante.

La rotazione intorno all'asse \(\hat{\mathbf{z}}\), nel sistema fisso, si scrive come:

\[
\vec{\mu}\left(t\right) = {\mathbf{R}}_{z}\left(\omega t\right)\vec{\mu}\left(0\right)
\]

dove \(\mathbf{R}_z\) è la matrice di rotazione intorno all'asse \(\hat{\mathbf{z}}\):

\[
\mathbf{R}_{z} = 
\begin{pmatrix}
\cos\!\left(\omega_{0}t\right) & -\sin\!\left(\omega_{0}t\right) & 0 \\[4pt]
\sin\!\left(\omega_{0}t\right) & \cos\!\left(\omega_{0}t\right) & 0 \\[4pt]
0 & 0 & 1
\end{pmatrix}
\]

Se invece la rotazione avviene intorno all'asse \(\hat{\mathbf{y}}\), la soluzione è:

\[
\vec{\mu}\left(t\right) = {\mathbf{R}}_{y}\left(\omega t\right)\vec{\mu}\left(0\right)
\]

con:

\[
{\mathbf{R}}_{y} = \begin{pmatrix}
\cos\left( \omega_{0}t \right) & 0 & - \sin\left( \omega_{0}t \right) \\
0 & 1 & 0 \\
\sin\left( \omega_{0}t \right) & 0 & \cos\left( \omega_{0}t \right)
\end{pmatrix}
\]

Nel caso di rotazione attorno all'asse \(\hat{\mathbf{x}}\), la soluzione è:

\[
\vec{\mu}\left(t\right) = {\mathbf{R}}_{x}\left(\omega t\right)\vec{\mu}\left(0\right)
\]

Con:

\[
{\mathbf{R}}_{x} = \begin{pmatrix}
1 & 0 & 0 \\
0 & \cos\left( \omega_{0}t \right) & - \sin\left( \omega_{0}t \right) \\
0 & \sin\left( \omega_{0}t \right) & \cos\left( \omega_{0}t \right)
\end{pmatrix}
\]

L'angolo di rotazione del sistema di riferimento rotante rispetto a quello fisso, fissato un istante temporale \(\tau\), è dato da:

\[
\phi = \omega_{1}\tau
\]

dove \(\omega_1\) è legato al campo a radiofrequenza applicato dalla relazione:

\[
\omega_{1} = \gamma B_{1}
\]

Questa relazione è valida se \(B_{1}\) è costante. L'angolo di cui ruota il sistema può essere scritto come:

\[
\phi = \gamma B_{1}\tau
\]

Nel caso generale in cui l'ampiezza del campo a radiofrequenza varia nel tempo, ovvero \(B_1 = B_1\left(t\right)\), la fase all'istante \(\tau\) è ottenuta come integrale temporale:

\[
\dfrac{d\phi}{dt} = \omega_{1} \Leftrightarrow \phi = \int_{t_{0}}^{\tau}{\omega_{1}dt}
\]

Per il legame tra pulsazione del sistema rotante e campo a radiofrequenza applicato, si ha:

\[
\phi = \int_{t_{0}}^{\tau}{\gamma B_{1}dt}
\]


Questa generalizzazione è necessaria quando il campo polarizzato circolarmente è applicato per un intervallo temporale finito e ha ampiezza variabile. L'uso dell'integrale consente una descrizione più generale, applicabile anche a campi non costanti.


Si suppone di applicare un campo a radiofrequenza \(B_1\), diretto lungo \(\hat{\mathbf{x}}'\), tale da far ruotare gli spin di un angolo \(\vartheta\) rispetto a \(\hat{\mathbf{x}}'\). Successivamente, si applica un secondo impulso a radiofrequenza, diretto lungo \(\hat{\mathbf{y}}'\), che ruota gli spin di un angolo \(\varphi\). La descrizione del vettore di magnetizzazione nel sistema rotante, dopo i due impulsi, è data dalla composizione delle due rotazioni:

\[
\vec{\mu}\left(t\right) = {\mathbf{R}}_{y}\left(\varphi\right){\mathbf{R}}_{x}(\vartheta)\vec{\mu}\left(0\right)
\]

Per descrivere il moto di precessione nel sistema fisso, dopo un tempo \(t\) di precessione libera, si moltiplica per la matrice di rotazione lungo \(z\):

\[
\vec{\mu}\left(t\right) = \mathbf{R}_{z}\left(\omega_0 t\right)\left({\mathbf{R}}_{y}\left(\varphi\right){\mathbf{R}}_{x}(\vartheta)\vec{\mu}\left(0\right)\right)
\]

\section{Vettore di magnetizzazione}\label{vettore-di-magnetizzazione}

Per ottenere un'immagine di una sezione del corpo umano mediante il fenomeno della risonanza magnetica, si suddivide il corpo del paziente in volumetti elementari contenenti un numero di Avogadro di protoni che precessano alla frequenza di Larmor. Dato l'elevato numero di spin, si introduce il vettore di magnetizzazione per unità di volume \(\vec{M}\).

Si considera un volumetto elementare contenente un numero di Avogadro \(N_A\) di protoni, immerso in un campo magnetico diretto lungo \(\hat{\mathbf{z}}\). Gli spin nel volumetto si allineano rispetto all'asse del campo magnetico esterno.

\begin{figure}[ht]
\centering
\includegraphics[width=2.89286in,height=2.63294in,alt={P3086\#yIS1}]{media/6_IntroMRI/image66.pdf}\caption{Volumetto elementare di \(N_{A}\) di protoni in un campo magnetico diretto lungo \({\hat{\imath}}_{z}\)}
\end{figure}

Secondo la meccanica quantistica, gli spin occupano due livelli energetici: \( \left| - \right\rangle\), di energia \(+\hslash\omega_0/2\), e \( \left| + \right\rangle \), di energia \(-\hslash\omega_0/2\). Dal punto di vista macroscopico, ogni volumetto elementare presenta un vettore di magnetizzazione \(\vec{M}\), dato dalla somma dei singoli momenti magnetici dei protoni, normalizzata per il volume:

\[
\vec{M} = \dfrac{1}{V}\sum_{i = 1}^{N_{A}}{\vec{\mu}}_{\imath}
\]

L'insieme degli spin nel volumetto, che precessano alla stessa frequenza, è detto \textit{isochromat} o \textbf{isocromati}, poiché può essere considerato come un dominio (o insieme) di spin con frequenza uniforme.

Trascurando le interazioni tra i protoni e l'ambiente, ogni spin precessa intorno al campo applicato secondo:

\[
\dfrac{d{\vec{\mu}}_{\imath}}{dt} = \gamma{\vec{\mu}}_{\imath} \times {\vec{B}}_{0}
\]

Sommando i contributi di tutti gli spin e dividendo per il volume si ottiene:

\[
\dfrac{1}{V}\sum_{i = 1}^{N_{A}}\dfrac{d{\vec{\mu}}_{\imath}}{dt} = \dfrac{1}{V}\sum_{i = 1}^{N_{A}}{\gamma{\vec{\mu}}_{\imath} \times {\vec{B}}_{0}}
\]

Per la linearità dell'operatore somma e derivata è possibile scrivere:

\[
\sum_{i = 1}^{N_{A}}{\dfrac{d}{dt}\left( \dfrac{1}{V}{\vec{\mu}}_{\imath} \right)} = \sum_{i = 1}^{N_{A}}{\gamma\dfrac{1}{V}{\vec{\mu}}_{\imath} \times {\vec{B}}_{0}} \Leftrightarrow \dfrac{d}{dt}\left( \sum_{i = 1}^{N_{A}}{\dfrac{1}{V}{\vec{\mu}}_{\imath}} \right) = \gamma\left( \sum_{i = 1}^{N_{A}}{\dfrac{1}{V}{\vec{\mu}}_{\imath}} \right) \times {\vec{B}}_{0}
\]

Per definizione del vettore di magnetizzazione per unità di volume è possibile scrivere:

\[
\dfrac{d\vec{M}}{dt} = \gamma\vec{M} \times {\vec{B}}_{0}
\]

È utile analizzare la magnetizzazione scomponendola in una componente parallela all'asse del campo magnetico principale e una componente trasversale:

\[
\vec{M} = {\vec{M}}_{\parallel} + {\vec{M}}_{\perp}
\]

Dove, in coordinate cartesiane e con un campo magnetico esterno diretto lungo \({\hat{\imath}}_{z}\):

\[
\vec{M}_{\parallel} = M_z \hat{\imath}_{z}, \quad \vec{M}_{\perp} = M_x \hat{\imath}_{x} + M_y \hat{\imath}_{y}
\]

L'equazione differenziale può essere scritta scomponendo il vettore di magnetizzazione:

\[
\dfrac{d}{dt}\left( {\vec{M}}_{\parallel} + {\vec{M}}_{\perp} \right) = \gamma\left( {\vec{M}}_{\parallel} + {\vec{M}}_{\perp} \right) \times B_{0}{\hat{\imath}}_{z}
\]

Per linearità:

\[
\dfrac{d{\vec{M}}_{\parallel}}{dt} + \dfrac{d{\vec{M}}_{\perp}}{dt} = \gamma B_{0}{\vec{M}}_{\parallel} \times {\hat{\imath}}_{z} + \gamma B_{0}{\vec{M}}_{\perp} \times {\hat{\imath}}_{z}
\]

Per definizione di componente parallela del vettore di magnetizzazione, il termine \({\vec{M}}_{\parallel} \times {\hat{\imath}}_{z}\) si annulla:

\[
{\vec{M}}_{\parallel} \times {\hat{\imath}}_{z} = M_{z}{\hat{\imath}}_{z} \times {\hat{\imath}}_{z} = 0
\]

Per cui:

\[
\dfrac{d{\vec{M}}_{\parallel}}{dt} + \dfrac{d{\vec{M}}_{\perp}}{dt} = \gamma B_{0}{\vec{M}}_{\perp} \times {\hat{\imath}}_{z}
\]

Scrivendo le due equazioni per le proiezioni si ha:

\[
\begin{cases}
\dfrac{dM_{\parallel}}{dt} = 0 \\
\dfrac{d{\vec{M}}_{\perp}}{dt} = \gamma B_{0}{\vec{M}}_{\perp} \times {\hat{\imath}}_{z}
\end{cases}
\]

Questa modellazione non considera le interazioni tra spin e ambiente (reticolo) o tra spin stessi. In particolare, delle equazioni risulta che la dinamica longitudinale, evolve in maniera indipendente da quella trasversale. La prima è descritta dal tempo di rilassamento longitudinale \(T_{1}\) mentre la seconda dal tempo di rilassamento trasversale \(T_{2}\). Come risultato delle due evoluzioni, il modulo del vettore magnetizzazione non è costante nel tempo.

Per chiarire questo concetto, si consideri un volumetto in un campo magnetico lungo \(\hat{\mathbf{z}}\). All'equilibrio termodinamico, il vettore di magnetizzazione ha modulo \(M_0\) ed è diretto lungo il campo. Se si applica un impulso elettromagnetico che ruota \(\vec{M}\), durante il ritorno all'equilibrio la componente longitudinale \(M_z\) cresce lentamente verso \(M_0\), mentre la componente trasversale \(M_{\perp}\) decade rapidamente verso zero.

Questo comportamento è descritto dalle \textbf{equazioni di Bloch}, basate su una descrizione classica della materia. La complessità del moto di \(\vec{M}\) deriva dal fatto che esso rappresenta la somma di circa \(10^{23}\) momenti magnetici, ciascuno con un'evoluzione influenzata dall'ambiente.

A temperature ambiente, il vettore di magnetizzazione è dato dalla legge di Curie:

\[
M_{0} \simeq \dfrac{\gamma^{2}\hslash^{2}}{4k_{B}T}B_{0}\dfrac{N_{A}}{V}
\]

Il rapporto tra il numero di spin (pari a \(N_A\) per volumetto) e il volume del corpo è detto \textbf{densità protonica} \(\rho\):

\[
\rho = \dfrac{N_{A}}{V}
\]

Per cui:

\[
M_{0} \simeq \dfrac{\gamma^{2}\hslash^{2}}{4k_{B}T}B_{0}\rho
\]

Le interazioni degli spin con l'ambiente dipendono dal tipo di tessuto e dal suo stato fisiologico. Conoscendo le tempistiche con cui evolve il vettore di magnetizzazione durante il ritorno all'equilibrio, è possibile ricostruire l'immagine del tessuto.

\subsection{Iterazione spin-reticolo}\label{iterazione-spin-reticolo}

In condizioni normali, gli spin non sono indipendenti tra loro, ma interagiscono sia con il materiale in cui sono contenuti, sia simultaneamente con gli altri spin del reticolo. Di conseguenza, l'equazione per la componente longitudinale:

\[
\dfrac{dM_{\parallel}}{dt} = 0
\]

risulta errata, poiché non tiene conto di tali fenomeni. Infatti, i momenti magnetici dei protoni tendono ad allinearsi con il campo esterno; tuttavia, le interazioni con l'ambiente circostante impediscono al vettore di magnetizzazione di essere diretto esattamente lungo l'asse del campo principale, \({\hat{\imath}}_{z}\).

Ogni spin è inclinato rispetto all'asse verticale di un angolo \(\vartheta\) e interagisce con gli spin vicini. In particolare, l'angolo \(\vartheta\) rappresenta l'inclinazione del vettore che congiunge il centro del sistema di riferimento con lo spin, rispetto all'asse individuato dal campo magnetico esterno.

Nell'analisi della componente longitudinale è possibile trascurare gli effetti di repulsione tra protoni dovuti alle interazioni coulombiane. In altre parole, si assume che la distanza tra due protoni, detta distanza internucleare, sia molto maggiore del raggio d'azione delle forze elettriche.

\begin{figure}[ht]
\centering
\includegraphics[width=5.08796in,height=4.76042in,alt={P3130\#yIS1}]{media/6_IntroMRI/image67.pdf}\caption{Spin con diverse inclinazioni}
\end{figure}

Ogni singolo spin può essere descritto come un dipolo magnetico elementare, il cui potenziale vettore è dato da:

\[
\vec{A} = \dfrac{1}{\left| \vec{r} \right|}\vec{\mu} \times \vec{r}
\]

dove \(\vec{r}\) è il vettore che congiunge due spin.

\begin{figure}[ht]
\centering
\includegraphics[width=3.92856in,height=2.17593in,alt={P3135\#yIS1}]{media/6_IntroMRI/image68.pdf}\caption{Vettore distanza tra due spin}
\end{figure}

Dal punto di vista classico, il campo prodotto da uno spin è dato da:

\[
\vec{B} = \vec{\nabla} \times \vec{A} = \dfrac{1}{\left| \vec{r} \right|}\vec{\nabla} \times \left( \vec{\mu} \times \vec{r} \right)
\]

Si dimostra che la soluzione a tale equazione è data da:

\[
\begin{cases}
B_{x} = 3\mu\dfrac{\sin\vartheta\cos\vartheta}{r^{3}} \\
B_{y} = 0 \\
B_{z} = \mu\dfrac{3\cos^{2}\vartheta - 1}{r^{3}}
\end{cases} 
\]

La componente lungo \({\hat{\imath}}_{y}\) è nulla, poiché le linee di campo magnetico di uno spin presentano simmetria cilindrica.

\begin{figure}[ht]
\centering
\includegraphics[width=2.20513in,height=2.44444in,alt={P3142\#yIS1}]{media/6_IntroMRI/image69.pdf}\caption{Simmetria cilindrica del campo prodotto da uno spin}
\end{figure}

Per il principio di sovrapposizione, uno spin può influenzare quelli vicini e, viceversa, essere influenzato da essi. La componente lungo \({\hat{\imath}}_{z}\) è legata all'energia del sistema. In meccanica classica, l'energia di un momento magnetico \(\vec{\mu}\) immerso in un campo magnetico \({\vec{B}}_{0}\) è data da:

\[
U = - \vec{\mu} \cdot {\vec{B}}_{0}
\]

Con un campo avente solo componente verticale, ovvero \({\vec{B}}_{0} = B_{0}{\hat{\imath}}_{z}\), si ha:

\[
U = - \mu_{z}B_{0}
\]

Ne consegue che la componente verticale, lungo \({\hat{\imath}}_{z}\), è coinvolta negli scambi energetici tra dipoli elementari.

L'interpretazione della componente lungo \({\hat{\imath}}_{x}\) del campo prodotto da uno spin, \(B_{x}\), è più complessa, poiché bisogna considerare che i dipoli non sono fermi nello spazio, ma precessano attorno all'asse individuato dal campo esterno, influenzando tale componente. In particolare, a causa dell'agitazione termica, i dipoli sono in moto casuale. L'agitazione termica è molto intensa nei liquidi, come quelli che compongono prevalentemente il corpo umano, e nei gas.

A causa di tale agitazione, i dipoli si muovono reciprocamente in modo continuo. Le interazioni reciproche tra i campi magnetici degli spin sono molto complesse da descrivere; infatti, la teoria del rilassamento è altamente sofisticata e può essere affrontata solo mediante la meccanica quantistica. In letteratura esistono trattazioni relative esclusivamente a sostanze pure. Attualmente, non esistono modelli quantistici in grado di descrivere il rilassamento nel corpo umano, a causa dell'elevato numero di molecole coinvolte, come acqua, sali disciolti, proteine, lipidi, ecc.

Nell'analisi della risonanza magnetica, si adottano leggi che descrivono in modo approssimativo, ma sufficientemente fedele, i meccanismi di rilassamento nei tessuti umani.

A causa delle interazioni tra i vari spin, un protone non è soggetto a un campo magnetico identico a quello statico applicato, ma a uno prossimo. Ogni spin, influenzato dagli altri, ruota attorno all'asse individuato dal campo magnetico principale con una propria frequenza di Larmor.

Quando viene applicato un campo magnetico a frequenza \(\omega_{0} = \gamma B_{0}\), possono avvenire transizioni degli spin dallo stato \(\left| + \right\rangle\) allo stato \(\left| - \right\rangle\) e viceversa. Un dipolo magnetico, soggetto a una radiazione elettromagnetica a frequenza \(\omega_{0}\), prossima a quella di precessione, può essere indotto a una transizione verso l'alto o verso il basso:

\begin{itemize}
\item
  Nella transizione dal livello energetico \(\left| + \right\rangle\) a \(\left| - \right\rangle\), il protone emette un fotone di energia \(\hslash\gamma B_{0} = \hslash\omega_{0}\). Questo fenomeno è noto come \textit{emissione stimolata};
\item
  Nella transizione dal livello energetico \(\left| - \right\rangle\) a \(\left| + \right\rangle\), il protone assorbe un fotone di energia \(\hslash\omega_{0}\).
\end{itemize}

Il passaggio dallo stato \(\left| + \right\rangle\) a \(\left| - \right\rangle\) produce l'emissione di un fotone che può stimolare un altro spin vicino, inducendolo a sua volta a transitare da \(\left| - \right\rangle\) a \(\left| + \right\rangle\). Le interazioni energetiche a livello microscopico sono quindi molto complesse, a causa dell'elevato numero di spin contenuti in un volumetto elementare.

Per analizzare il comportamento di un singolo spin rispetto alla massa circostante, è possibile ricorrere alla statistica di Boltzmann, considerando lo spin a contatto con il reticolo, o \textit{lattice}, come un piccolo sistema in interazione con un altro contenente un numero di spin molto maggiore.

Sia \(N^{+}\) il numero di protoni nel livello energetico \(E^{+} = \hslash\gamma B_{0}/2\) e \(N^{-}\) il numero di protoni nel livello energetico \(E^{-} = - \hslash\gamma B_{0}/2\). Sia \(W_{+ \rightarrow -}\) la probabilità di transizione dallo stato \(\left| + \right\rangle\) a \(\left| - \right\rangle\), e \(W_{- \rightarrow +}\) la probabilità opposta.

\begin{figure}[ht]
\centering
\includegraphics[width=4.84649in,height=1.12963in,alt={P3161\#yIS1}]{media/6_IntroMRI/image70.pdf}\caption{Probabilità di transizioni}
\end{figure}

In meccanica quantistica, le due probabilità si esprimono come:

\[
\begin{cases}
W_{+ \rightarrow -} = \left\langle + \right|\hat{\mu}\left| - \right\rangle \\
W_{- \rightarrow +} = \left\langle - \right|\hat{\mu}\left| + \right\rangle
\end{cases}
\]

Queste quantità possono essere stimate conoscendo l'hamiltoniano \(\hat{H}\) del sistema. Tuttavia, data la complessità dovuta al gran numero di particelle nel volumetto elementare, si preferisce ricorrere alla statistica di Boltzmann.

All'equilibrio termodinamico, il numero delle transizioni dallo stato \(\left| + \right\rangle\) allo stato \(\left| - \right\rangle\) deve essere uguale al numero delle transizioni da \(\left| - \right\rangle\) a \(\left| + \right\rangle\), in quanto l'energia del sistema si conserva. Si instaura così un equilibrio dinamico nel sistema. Per tale motivo è possibile scrivere l'uguaglianza:

\[
N^{+}W_{+ \rightarrow -} = N^{-}W_{- \rightarrow +}
\]

dove \(N^{+}W_{+ \rightarrow -}\) è il numero dei protoni che dallo stato \(\left| + \right\rangle\) transitano allo stato \(\left| - \right\rangle\) e \(N^{-}W_{- \rightarrow +}\) è il numero dei protoni che eseguono la transizione opposta. Le transizioni devono essere tali da mantenere la popolazione degli spin costante.

Si scrive l'equazione all'equilibrio termodinamico come:

\[
N^{+}W_{+ \rightarrow -} = N^{-}W_{- \rightarrow +} \Leftrightarrow \dfrac{W_{+ \rightarrow -}}{W_{- \rightarrow +}} = \dfrac{N^{-}}{N^{+}}
\]

Per la statistica di Boltzmann, il rapporto tra le due probabilità è dato dal rapporto degli esponenziali:

\[
\dfrac{W_{+ \rightarrow -}}{W_{- \rightarrow +}} = \dfrac{\exp\left( \dfrac{E^{+}}{k_{B}T} \right)}{\exp\left( \dfrac{E^{-}}{k_{B}T} \right)} = \exp\left( \dfrac{E^{+} - E^{-}}{k_{B}T} \right)
\]

Sebbene non sia possibile valutare numericamente le due probabilità, è possibile valutarne il rapporto mediante la meccanica statistica.

Si scrive l'equazione differenziale che collega l'evoluzione temporale del numero di protoni che si trovano nello stato \(\left| + \right\rangle\). La variazione nel tempo del numero di spin nello stato energetico \(\left| + \right\rangle\) dipende dal numero di protoni che dallo stato \(\left| - \right\rangle\) passano allo stato \(\left| + \right\rangle\), a cui va sottratto il numero di protoni che transitano dallo stato \(\left| + \right\rangle\) a quello \(\left| - \right\rangle\) nell'intervallo temporale infinitesimo \(dt\):

\[
\dfrac{dN^{+}}{dt} = N^{-}W_{- \rightarrow +} - N^{+}W_{+ \rightarrow -}
\]

Da questa espressione si evince che le quantità \(W_{- \rightarrow +}\) e \(W_{+ \rightarrow -}\) sono dimensionalmente omogenee con l'inverso di un tempo:

\[
\left\lbrack W_{- \rightarrow +} \right\rbrack = \left\lbrack \dfrac{1}{s} \right\rbrack
\]

In altre parole, \(W_{- \rightarrow +}\) e \(W_{+ \rightarrow -}\) sono delle probabilità per unità di tempo.

Analogamente per la popolazione di spin nello stato \(\left| - \right\rangle\): la variazione del numero di spin nell'unità di tempo è data dalla popolazione di spin nello stato \(\left| + \right\rangle\) che transita nello stato \(\left| - \right\rangle\) a cui va sottratto il numero di spin che esegue la transizione opposta:

\[
\dfrac{dN^{-}}{dt} = N^{+}W_{+ \rightarrow -} - N^{-}W_{- \rightarrow +}
\]

Si scrivono le equazioni in termini di differenza di popolazioni \(\Delta N\):

\[
\Delta N = N^{+} - N^{-}
\]

Il parametro \(\Delta N\) è fondamentale poiché la magnetizzazione macroscopica del volumetto elementare \(\vec{M}\) è proporzionale alla differenza di popolazione di spin nello stato \textit{up} (o \(\left| + \right\rangle\)) rispetto a quelle nello stato \textit{down} (o \(\left| - \right\rangle\)):

\[
\left| \vec{M} \right| \propto \Delta N = N^{+} - N^{-}
\]

La magnetizzazione è, in ultima analisi, legata al netto di spin paralleli al campo magnetico principale, rispetto a quelli antiparalleli. Per tale motivo è conveniente ragionare in termini di evoluzione della differenza di popolazione, piuttosto che mediante la variazione delle singole popolazioni.

Dalle due equazioni differenziali, che descrivono l'evoluzione temporale di una sola popolazione:

\[
\begin{cases}
\dfrac{dN^{+}}{dt} = N^{-}W_{- \rightarrow +} - N^{+}W_{+ \rightarrow -} \\
\dfrac{dN^{-}}{dt} = N^{+}W_{+ \rightarrow -} - N^{-}W_{- \rightarrow +}
\end{cases} 
\]

sottraendo membro a membro si ha:

\[
\dfrac{d}{dt}\left( N^{+} - N^{-} \right) = N^{-}W_{- \rightarrow +} - N^{+}W_{+ \rightarrow -} - N^{+}W_{+ \rightarrow -} + N^{-}W_{- \rightarrow +}
\]

Da cui:

\[
\dfrac{d\Delta N}{dt} = 2N^{-}W_{- \rightarrow +} - 2N^{+}W_{+ \rightarrow -}
\]

Sia \(N\) il numero totale degli spin contenuto nel volumetto elementare. Siccome il volume non scambia materia con l'esterno, il numero degli spin è costante:

\[
N = N^{+} + N^{-} = const
\]

Per definizione:

\[
\Delta N = N^{+} - N^{-}
\]

Sommando membro a membro sia ha:

\[
N + \Delta N = 2N^{+} \Leftrightarrow N^{+} = \dfrac{1}{2}(N + \Delta N)
\]

Invece, sottraendo membro a membro, si ha:

\[
N - \Delta N = 2N^{+} \Leftrightarrow N^{-} = \dfrac{1}{2}(N - \Delta N)
\]

L'equazione differenziale:

\[
\dfrac{d}{dt}\left( N^{+} - N^{-} \right) = 2N^{-}W_{- \rightarrow +} - 2N^{+}W_{+ \rightarrow -}
\]

Può essere scritta come:

\[
\dfrac{d\Delta N}{dt} = 2\dfrac{1}{2}(N - \Delta N)W_{- \rightarrow +} - 2\dfrac{1}{2}(N + \Delta N)W_{+ \rightarrow -} = (N - \Delta N)W_{- \rightarrow +} - (N + \Delta N)W_{+ \rightarrow -}
\]

Raccogliendo \(N\) e \(\Delta N\) al secondo membro si ha:

\[
\dfrac{d\Delta N}{dt} = N\left( W_{- \rightarrow +} - W_{+ \rightarrow -} \right) - \Delta N\left( W_{- \rightarrow +} + W_{+ \rightarrow -} \right)
\]

Queste equazioni devono essere sempre valide poiché non sono state proposte ipotesi particolari semplificative. Se le equazioni sono sempre valide, lo sono anche all'equilibrio termodinamico, condizione in cui non vi è nessuna variazione di \(\Delta N\) dato che le due popolazioni presentano lo stesso numero di spin, nonostante la variazione della configurazione del sistema.Dal punto di vista matematico, all'equilibrio termodinamico si ha:

\[
\dfrac{d\Delta N}{dt} = 0
\]

Dunque, \(\Delta N = \Delta N_{0} = const\). 

Siano \(N_{0}^{+}\) e \(N_{0}^{-}\) rispettivamente il numero di spin nello stato \(\left| + \right\rangle\) e nello stato \(\left| - \right\rangle\) all'equilibrio termodinamico. Allora:

\[
\Delta N_{0} = N_{0}^{+} - N_{0}^{-}
\]

La quantità \(\Delta N_{0}\) è la differenza di spin nello stato \(\left| + \right\rangle\) rispetto a quelli nello stato \(\left| - \right\rangle\) all'equilibrio termodinamico; ciò produce la magnetizzazione macroscopica \(\vec{M}\).

All'equilibrio, l'equazione differenziale:

\[
\dfrac{d\Delta N}{dt} = N\left( W_{- \rightarrow +} - W_{+ \rightarrow -} \right) - \Delta N\left( W_{- \rightarrow +} + W_{+ \rightarrow -} \right)
\]

Diventa:

\[
N\left( W_{- \rightarrow +} - W_{+ \rightarrow -} \right) - \Delta N_{0}\left( W_{- \rightarrow +} + W_{+ \rightarrow -} \right) = 0
\]

Si ricava \(\Delta N_{0}\):

\[
\Delta N_{0} = N\dfrac{W_{- \rightarrow +} - W_{+ \rightarrow -}}{W_{- \rightarrow +} + W_{+ \rightarrow -}}
\]

Sapendo che il valore all'equilibrio termodinamico del netto di spin è \(\Delta N_{0}\), è possibile ricavare l'andamento temporale di \(\Delta N\). Poiché \(\Delta N_{0}\) è costante, è possibile sottrarre la sua derivata al primo membro dell'equazione:

\[
\dfrac{d\Delta N}{dt} = N\left( W_{- \rightarrow +} - W_{+ \rightarrow -} \right) - \Delta N\left( W_{- \rightarrow +} + W_{+ \rightarrow -} \right)
\]

Ottenendo:

\[
\dfrac{d}{dt}\left( \Delta N - \Delta N_{0} \right) = N\left( W_{- \rightarrow +} - W_{+ \rightarrow -} \right) - \Delta N\left( W_{- \rightarrow +} + W_{+ \rightarrow -} \right)
\]

Al secondo membro si raccoglie il termine \(\left( W_{- \rightarrow +} + W_{+ \rightarrow -} \right)\), ottenendo:

\[
\dfrac{d}{dt}\left( \Delta N - \Delta N_{0} \right) = \left( N\dfrac{W_{- \rightarrow +} - W_{+ \rightarrow -}}{W_{- \rightarrow +} + W_{+ \rightarrow -}} - \Delta N \right)\left( W_{- \rightarrow +} + W_{+ \rightarrow -} \right)
\]

All'equilibrio termodinamico, si ha:

\[
\Delta N_{0} = N\dfrac{W_{- \rightarrow +} - W_{+ \rightarrow -}}{W_{- \rightarrow +} + W_{+ \rightarrow -}}
\]

Per cui, l'equazione differenziale può essere scritta come:

\[
\dfrac{d}{dt}\left( \Delta N - \Delta N_{0} \right) = \left( \Delta N_{0} - \Delta N \right)\left( W_{- \rightarrow +} + W_{+ \rightarrow -} \right)
\]

Poiché \(W_{- \rightarrow +}\) e \(W_{+ \rightarrow -}\) sono probabilità per unità di tempo, si definisce:

\[
\dfrac{1}{T_{1}} = W_{- \rightarrow +} + W_{+ \rightarrow -}
\]

Con questa posizione, l'equazione differenziale si può scrivere:

\[
\dfrac{d}{dt}\left( \Delta N - \Delta N_{0} \right) = \dfrac{1}{T_{1}}\left( \Delta N_{0} - \Delta N \right)
\]

Al fine di avere la stessa quantità, si raccoglie un segno meno al secondo membro:

\[
\dfrac{d}{dt}\left( \Delta N - \Delta N_{0} \right) = - \dfrac{1}{T_{1}}\left( \Delta N - \Delta N_{0} \right)
\]

L'evoluzione temporale di \(\Delta N\) è quindi governata dalla costante di tempo \(T_{1}\), detta tempo di rilassamento longitudinale o \textbf{spin-reticolo} (\textit{spin-lattice}).

La relazione individuata è molto approssimata, poiché le probabilità non sono note a priori, ma permette di descrivere l'evoluzione temporale del netto degli spin \textit{up} nel tempo. La soluzione dell'equazione è di tipo esponenziale crescente verso il valore di regime \(\Delta N_{0}\), con costante di tempo \(T_{1}\).

Generalizzando, in un sistema di spin immerso in un campo magnetico, la componente longitudinale del vettore di magnetizzazione tende a raggiungere il valore di equilibrio, dato dalla legge di Curie, con costante di tempo \(\tau = T_{1}\).

\begin{figure}[ht]
\centering
\includegraphics[width=2.2315in,height=1.68519in,alt={P3237\#yIS1}]{media/6_IntroMRI/image71.pdf}\caption{Evoluzione temporale del netto di vettore di magnetizzazione per effetto del campo principale}
\end{figure}

Se il campo magnetico principale viene disattivato dopo il raggiungimento dell'equilibrio termodinamico, la componente longitudinale del vettore di magnetizzazione si annulla. Il decadimento è esponenziale con costante di tempo \(T_{1}\).


\begin{figure}[ht]
\centering
\includegraphics[width=3.31667in,height=2.56742in,alt={P3240\#yIS1}]{media/6_IntroMRI/image72.pdf}\caption{Evoluzione del vettore di magnetizzazione alla rimozione del campo principale}
\end{figure}

Il tempo \(T_{1}\) riguarda gli scambi energetici tra gli spin e il reticolo, ed è associato alla componente lungo \({\hat{\imath}}_{z}\). L'equazione di Bloch \(dM_{z}/dt = 0\) viene corretta introducendo l'andamento temporale, dettato dalla costante di tempo \(T_{1}\):

\[
\dfrac{dM_{z}}{dt} = \dfrac{1}{T_{1}}\left( M_{0} - M_{z}\  \right)
\]

Questa è detta \textbf{prima equazione di Bloch}. Il tempo \(T_{1}\), o tempo di rilassamento longitudinale, riflette le interazioni tra uno spin e il reticolo, considerato come un serbatoio termico. In generale, \(T_{1}\) è maggiore nei liquidi rispetto ai solidi.

La soluzione della prima equazione di Bloch è del tipo:

\[
M_{z}\left(t\right) = M_{z}\left(0\right)\exp\left( - \dfrac{t}{T_{1}} \right) + M_{0}\left( 1 - \exp\left( - \dfrac{t}{T_{1}} \right) \right)
\]

Questa equazione è fondamentale per descrivere come il vettore di magnetizzazione ritorni all'equilibrio dopo una perturbazione.

Il parametro \(T_{1}\) è essenziale per la caratterizzazione dei tessuti, e quindi per l'\textit{imaging} in risonanza magnetica. Tessuti diversi presentano valori specifici di tempo di rilassamento longitudinale.

\subsection{Iterazioni spin-spin}\label{iterazioni-spin-spin}

Un importante meccanismo che determina il decadimento delle componenti trasversali della magnetizzazione è la generazione di campi locali prodotti dagli spin. Questi campi si combinano con il campo principale applicato, modificando localmente il campo in cui sono immersi gli spin vicini.

Si considera un singolo spin.  il campo prodotto da esso è indicato con \({\vec{B}}_{loc}\). Gli altri spin generano campi differenti. Le componenti verticali di questi campi hanno intensità diverse e non predicibili, poiché non è possibile ricostruire il moto di ogni singolo spin nel volumetto considerato.

Per studiare il campo locale percepito da uno spin, si utilizza il valor quadratico medio:

\[
B_{RMS} = \sqrt{\left\langle \left| {\vec{B}}_{loc} \right|^{2} \right\rangle}
\]

Esiste quindi una componente magnetica media lungo \({\hat{\imath}}_{z}\), che si sovrappone al campo magnetico principale \(B_{0}{\hat{\imath}}_{z}\), determinando una variazione della frequenza di precessione degli spin attorno all'asse \({\hat{\imath}}_{z}\).

A causa dell'agitazione termica, gli spin sono in moto relativo, quindi la loro posizione è dinamica. Ne consegue che il campo locale prodotto da uno spin varia nel tempo, e il valor quadratico medio diventa funzione del tempo:

\[
B_{RMS}\left(t\right)  = \sqrt{\left\langle \left| {\vec{B}}_{loc}\left(t\right)  \right|^{2} \right\rangle}
\]

Sia \(t_{1}\) un primo istante di osservazione: il campo locale prodotto da uno spin vale \({\vec{B}}_{loc}(t_{1})\). In un istante successivo \(t_{2}\), il campo, a causa dei moti termici, sarà diverso:

\[
{\vec{B}}_{loc}\left( t_{1} \right) \neq {\vec{B}}_{loc}\left( t_{2} \right)
\]

Questo risultato è dovuto alla diversa configurazione degli spin tra gli istanti \(t_{1}\) e \(t_{2}\).

a frequenza di precessione intorno all'asse \({\hat{\imath}}_{z}\) di ciascuno spin è quindi diversa e data da:

\[
\Delta \omega\left(t\right) = \gamma{\vec{B}}_{loc}\left(t\right)
\]

Dove \(\Delta \omega = \omega_{0} - \omega_{loc}\left(t\right)\). La differenza tra le frequenze di precessione è, dunque, una funzione del tempo. Dati due istanti temporale, risulta:

\[\
Delta \omega\left( t_{1} \right) = \gamma{\vec{B}}_{loc}\left( t_{1} \right) \neq \Delta \omega\left( t_{2} \right) = \gamma{\vec{B}}_{loc}\left( t_{2} \right)
\]

Ovviamente \(\Delta \omega\left( t_{1} \right) \neq \Delta \omega\left( t_{2} \right)\) poiché i campi locali nei due istanti temporali sono diversi.

Si vuole studiare il fenomeno del campo locale nel sistema di riferimento locale. Trascurando la frequenza principale \(\omega_{0}\), restano solamente le variazioni \(\Delta \omega\) dovute alle iterazioni tra uno spin e i protoni locali. Queste variazioni, nel sistema rotante danno luogo a delle variazioni di fase:

\[
\vartheta = \int_{t_{0}}^{t_{1}}{\Delta \omega\left( t' \right)dt'}
\]

Può capitare che in un istante la fase cresca più rapidamente, altre volte meno rapidamente. In ogni caso la rotazione degli spin avviene sempre, complessivamente, in senso orario. Nel sistema rotante, in particolare, le variazioni di fase possono essere sia positive che negative; di conseguenza, la proiezione dello spin sul piano \(x' - y'\) non è fissa ma varia nel tempo, secondo la sua variazione di fase legate al campo locale.

\begin{figure}[ht]
\centering
\includegraphics[width=4.22917in,height=1.49311in,alt={P3268\#yIS1}]{media/6_IntroMRI/image73.pdf}\caption{Variazione della fase a causa del campo locale}
\end{figure}

La proiezione di tutti gli spine nel piano \(x' - y'\) è casuale, poiché  il campo locale che ogni spin percepisce è  aleatorio. È possibile, quindi, concludere che le proiezioni degli spin sono tutte distribuite uniformemente nel piano \(x' - y'\). Di conseguenza, l'effetto medio, percepito a livello macroscopico nel volumetto contenente un numero di Avogadro di particelle, è l'annullamento delle componenti trasverse del vettore di magnetizzazione. Infatti, le componenti trasversali dei singoli spin, essendo casuali, si elidono a vicenda; ovvero, la risultate media della componente trasversa del vettore di magnetizzazione è nulla, dato l'elevato numero di spin considerato:

\[
\vec{M}_{\perp} \rightarrow 0
\]

\begin{figure}[ht]
\centering
\includegraphics[width=4.40341in,height=3.58333in,alt={P3271\#yIS1}]{media/6_IntroMRI/image74.pdf}\caption{Eliminazione delle componenti trasverse degli spin causa distribuzione uniforme}
\end{figure}

Le componenti longitudinali del campo magnetico prodotto da uno spin inducono sfasamenti e defasamenti degli spin vicini, che precessano con frequenze diverse e casuali. Ne discende che la somma delle componenti trasversali degli spin mediamente è nulla. Da notare che lo sfasamento è legato alla fase diversa per ogni singolo spin.

Il tempo con cui la componente trasversale media va zero può essere stimato, studiando le iterazioni tra due spin. Si suppone che a un certo instante di tempo le fasi dei due spin siano le stesse. In generale, la differenza di fase è data da:

\[
\Delta \phi = \phi_{1} - \phi_{2} = \left( \phi_{0}^{1} - \Delta \omega_{{loc}_{1}}t \right) - \left( \phi_{0}^{2} - \Delta \omega_{{loc}_{2}}t \right)
\]

dove \(\Delta \omega_{{loc}_{1}}\) e \(\Delta \omega_{{loc}_{2}}\) sono le frequenze di precessione relative dei campi magnetici locali rispetto al campo magnetico principale. Se le fasi inziali dei due spin sono le stesse, ovvero \(\phi_{0}^{1} = \phi_{0}^{2}\), la differenza di fase si scrive come:

\[
\Delta \phi = - \left( \Delta \omega_{{loc}_{1}} - \Delta \omega_{{loc}_{2}} \right)
\]

Le frequenze di precessione relative rispetto a quelle del campo magnetico principale possono essere scritte in funzione del campo locale:

\[
\Delta \omega_{{loc}_{\imath}} = \gamma B_{{loc}_{\imath}},\ \ i = 1,2
\]

Per cui la differenza di fase può essere scritta come:

\[
\Delta \phi = - \left( \gamma B_{{loc}_{1}} - \gamma B_{{loc}_{2}} \right)t
\]

Si indica la differenza di campi locali come:

\[
\Delta B_{loc} = B_{{loc}_{1}} - B_{{loc}_{2}}
\]

Si ottiene:

\[
\Delta \phi = - \gamma\Delta B_{loc}t
\]

Per avere una risultate netta nulla nel piano trasverso, le due proiezioni sul piano \(x' - y'\) devono essere uguali e oppure la differenza di fase deve essere \(\Delta \phi = \pi\). È, ora, possibile ottenere il tempo con cui due spin annullano le proprie componenti trasversali. Si indica tale tempo con \(\tau\):

\[
\Delta \phi = \pi = - \gamma\Delta B_{loc}\tau
\]

Da cui:

\[
\tau = - \dfrac{\pi}{\gamma\Delta B_{loc}}
\]

Le differenze dei campi indotti hanno un'intensità proporzionale al fattore \(3\mu/r^{3}\):

\[
B_{loc} \propto \dfrac{3\mu}{r^{3}}
\]

Il tempo di defasamento, al limite (ignorando le costanti), è dato da:

\[
\tau\sim\dfrac{\pi}{\gamma\dfrac{\mu}{r^{3}}} = \dfrac{\pi r^{3}}{\mu\gamma}
\]

Questa stima, sebbene basata su meccanica classica, è coerente con i dati sperimentali. Da questa relazione, noti i valori medi del momento magnetico e della distanza interatomica, è possibile ottenere una stima del tempo di defasamento più o meno concorde ai dati sperimentali. La stima di basa su una descrizione non esatta del fenomeno, in quanto fondata della meccanica classica; tuttavia, fornisce dei risultati vicini ai dati empirici.

Il tempo di defasamento \(\tau\) è indicato con \(T_{2}\) ed è detto tempo di \textbf{rilassamento traversarsele}. Questo tempo rappresenta la costante di tempo con cui la componente trasversale del vettore di magnetizzazione, \({\vec{M}}_{\perp}\), tende a zero:

\[
M_{x} \rightarrow 0,\ \ M_{y} \rightarrow 0
\]

Generalmente, \(T_{2}\) è minore nei solidi rispetto ai liquidi.

\section{Equazioni di Bloch}\label{equazioni-di-bloch}

L'equazione che descrive la magnetizzazione macroscopica di un volumetto elementare è:

\[\dfrac{d\vec{M}}{dt} = \gamma\vec{M} \times {\vec{B}}_{0}\]

A questa vanno aggiunti due termini che tengono conto dei fenomeni di rilassamento: il primo è legato agli scambi energetici tra uno spin e il reticolo (visto come un serbatoio termico), il secondo alle interazioni locali tra spin:

\[
\dfrac{d\vec{M}}{dt} = \gamma\vec{M} \times {\vec{B}}_{0} + \dfrac{1}{T_{1}}\left( M_{0} - M_{z}\  \right){\hat{\imath}}_{z} - \dfrac{1}{T_{2}}{\vec{M}}_{\perp}
\]

La precedente relazione è detta \textbf{equazione di Bloch}. Il segno meno nel termine \({\vec{M}}_{\perp}/T_{2}\) riflette il decadimento verso lo zero della componente trasversale della magnetizzazione con tempo caratteristico \(T_{2}\).

Le equazioni di Bloch forniscono una descrizione fenomenologica del comportamento del vettore di magnetizzazione, poiché non derivano dalla meccanica quantistica, ma sono coerenti con le osservazioni sperimentali. Inoltre, le equazioni di Bloch non possono essere dedotte utilizzando la meccanica quantistica.

In letteratura anglosassone si definiscono le rilassività (o \textit{relaxivity}) longitudinale e trasversale come:

\[
R_{1} = \dfrac{1}{T_{1}},\ \ R_{2} = \dfrac{1}{T_{2}}
\]

L'equazione vettoriale di Bloch può essere anche scritta in termini di relaxivity

\[
\dfrac{d\vec{M}}{dt} = \gamma\vec{M} \times {\vec{B}}_{0} + R_{1}\left( M_{0} - M_{z}\  \right){\hat{\imath}}_{z} - R_{2}{\vec{M}}_{\perp}
\]

I tempi di rilassamento permettono di caratterizzare i tessuti e di identificare lo stato di salute. Essi variano significativamente tra i diversi tessuti molli, consentendo di discriminare i tessuti mediante diverse intensità di segnale (gradi di grigio) nelle immagini.

Le differenze di tempo di rilassamento longitudinale \(T_{1}\) non possono essere determinate sulla base della teoria quantistica, a causa dell'elevata complessità biochimiche dei tessuti umani.

In media, la composizione dei tessuti è simile tra individui, quindi i tempi di rilassamento di un paziente sono generalmente confrontabili con quelli medi. Mediante opportune sequenze di acquisizione, è possibile stimare i tempi di rilassamento e caratterizzare i tessuti.

\begin{longtable}{@{}>{\centering\arraybackslash}p{0.33\linewidth} >{\centering\arraybackslash}p{0.33\linewidth} >{\centering\arraybackslash}p{0.33\linewidth}@{}}
\caption{Tempi di rilassamento di vari tessuti molli con campo principale di \(1.5\ \mathrm{T}\) e temperatura di \(37\,^{\circ}\mathrm{C}\)}\tabularnewline
\toprule
\textbf{Tessuto} & \textbf{Tempo di rilassamento longitudinale \(T_{1}\)} & \textbf{Tempo di rilassamento trasversale \(T_{2}\)} \\
\midrule
\endfirsthead

\toprule
\textbf{Tessuto} & \textbf{Tempo di rilassamento longitudinale \(T_{1}\)} & \textbf{Tempo di rilassamento trasversale \(T_{2}\)} \\
\midrule
\endhead

\bottomrule
\endlastfoot

Materia grigia & \(\sim 950\,\mathrm{ms}\) & \(\sim 100\,\mathrm{ms}\) \\
Tessuto muscolare & \(\sim 900\,\mathrm{ms}\) & \(\sim 50\,\mathrm{ms}\) \\
Grasso & \(\sim 250\,\mathrm{ms}\) & \(\sim 60\,\mathrm{ms}\) \\
Sangue & \(\sim 1200\,\mathrm{ms}\) & \(\sim 100\,\mathrm{ms}\) \\
Materia bianca & \(\sim 600\,\mathrm{ms}\) & \(\sim 80\,\mathrm{ms}\) \\
Fluido cerebrospinale (CSF) & \(\sim 4500\,\mathrm{ms}\) & \(\sim 2200\,\mathrm{ms}\) \\

\end{longtable}

Generalmente, il tempo di rilassamento longitudinale è maggiore di quello trasversale, ovvero \(T_{1} > T_{2}\). Di conseguenza, l'evoluzione longitudinale è più lenta rispetto a quella trasversale. Analogamente, per le rilassività si ha \(R_{2} > R_{1}\).

\subsection{Risoluzione dell'equazione di Bloch}\label{risoluzione-dellequazione-di-bloch}

L'equazione vettoriale di Bloch, che include i termini di rilassamento longitudinale e trasversale, è:


\[\dfrac{d\vec{M}}{dt} = \gamma\vec{M} \times {\vec{B}}_{0} + \dfrac{1}{T_{1}}\left( M_{0} - M_{z}\  \right){\hat{\imath}}_{z} - \dfrac{1}{T_{2}}{\vec{M}}_{\perp}\]

dove:
\begin{itemize}
\item \(\vec{M}\) è il vettore di magnetizzazione;
\item \(\vec{B}_0\) è il campo magnetico principale;
\item \(T_1\) e \(T_2\) sono i tempi di rilassamento longitudinale e trasversale.
\end{itemize}

Questa equazione descrive da un punto di vista classico l'andamento del vettore di magnetizzazione \(\vec{M}\) nel tempo.

Si suppone di forzare il sistema di spin mediante un campo magnetico principale \(B_{0}\) diretto lungo \({\hat{\imath}}_{z}\). Si scrive il prodotto vettorale tra il vettore di magnetizzazione e il campo esterno applicato:

\[
\vec{M} \times \vec{B}_0 =
\begin{vmatrix}
\hat{\imath}_x & \hat{\imath}_y & \hat{\imath}_z \\
M_x & M_y & M_z \\
0 & 0 & B_0
\end{vmatrix}
= B_0 (M_y \hat{\imath}_x - M_x \hat{\imath}_y)
\]

Esplicitando le componenti, l'equazione di Bloch diventa:

\[
\dfrac{d}{dt}\left( M_{x}{\hat{\imath}}_{x} + M_{y}{\hat{\imath}}_{y} + M_{z}{\hat{\imath}}_{z} \right) = \gamma B_{0}\left( M_{y}{\hat{\imath}}_{x} - M_{x}{\hat{\imath}}_{y} \right) + \dfrac{1}{T_{1}}\left( M_{0} - M_{z}\  \right){\hat{\imath}}_{z} - \dfrac{1}{T_{2}}\left( M_{x}{\hat{\imath}}_{x} + M_{y}{\hat{\imath}}_{y} \right)
\]

Per effetto del campo magnetico principale, il vettore di magnetizzazione si sposta da una configurazione iniziale \(\vec{M}\left( t_{0} \right)\) a quella finale, descritta dall'equazione di Bloch. 

Si proietta l'ultima lungo gli assi:

\[
\begin{cases}
\dfrac{dM_{x}}{dt} = \gamma B_{0}M_{y} - \dfrac{1}{T_{2}}M_{x} \\
\dfrac{dM_{y}}{dt} = - \gamma B_{0}M_{x} - \dfrac{1}{T_{2}}M_{y} \\
\dfrac{dM_{z}}{dt} = \dfrac{1}{T_{1}}\left( M_{0} - M_{z}\  \right)
\end{cases} 
\]

Com'è facile osservare, la componente longitudinale \(M_z\left(t\right)\) evolve indipendentemente dalle componenti trasversali \(M_{x}\) e \(M_{y}\); infatti, le relative equazioni non sono accoppiate dopo la proiezione sugli assi. Le componenti trasversa sono, invece, legate tra loro.

Si risolve l'equazione relativa alla componente longitudinale del vettore \(\vec{M}\):

\[
\dfrac{dM_{z}}{dt} = \dfrac{1}{T_{1}}\left( M_{0} - M_{z} \right)
\]

Si passa all'equazione omogenea associata:

\[
\dfrac{dM_{z}}{dt} = \dfrac{1}{T_{1}}M_{z}
\]

La soluzione di questa equazione è del tipo:

\[
M_{z}\left(t\right) = k\exp\left( - \dfrac{t}{T_{1}} \right)
\]

Dove \(k\) è una costante dipendente dalle condizioni iniziali del sistema.

Alla soluzione dell'omogenea, va associata una soluzione particolare. Siccome il forzamento è costante una possibile soluzione particolare è del tipo:

\[
M_{z}\left(t\right) = c = \text{const}
\]

Si sostituisce tale soluzione nell'equazione differenziale:

\[
\left. \ \dfrac{dM_{z}}{dt} \right|_{M_{z} = c} = \dfrac{1}{T_{1}}\left. \ \left( M_{0} - M_{z}\  \right) \right|_{M_{z} = c} \Leftrightarrow \dfrac{dc}{dt} = \dfrac{1}{T_{1}}\left( M_{0} - c \right)
\]

La derivata di una costante è nulla, per cui:

\[
\dfrac{1}{T_{1}}\left( M_{0} - c \right) = 0 \Leftrightarrow c = M_{0}
\]

L'integrale generale dell'equazione differenziale per la componente longitudinale è:

\[
M_{z}\left(t\right) = k\exp\left( - \dfrac{t}{T_{1}} \right) + M_{0}
\]

Si suppone che il vettore di magnetizzazione sia noto all'istante \(t = t_{0}\). Applicando la condizione iniziale è possibile determinare il valore di \(k\):

\[
M_{z}\left( t_{0} \right) = k\exp\left( - \dfrac{t_{0}}{T_{1}} \right) + M_{0} \Leftrightarrow k = \left\lbrack M_{z}\left( t_{0} \right) - M_{0} \right\rbrack\exp\left( \dfrac{t_{0}}{T_{1}} \right)
\]

La soluzione è, dunque:

\[
M_{z}\left(t\right) = \left\lbrack M_{z}\left( t_{0} \right) - M_{0} \right\rbrack\exp\left( \dfrac{t_{0}}{T_{1}} \right)\exp\left( - \dfrac{t}{T_{1}} \right) + M_{0}
\]

Esplicitando la componente transitoria e quella di regime, si ha:

\[
M_{z}\left(t\right) = M_{0}\left\lbrack 1 - \exp\left( \dfrac{t_{0} - t}{T_{1}} \right) \right\rbrack + M_{z}\left( t_{0} \right)\exp\left( \dfrac{t_{0} - t}{T_{1}} \right)
\]

Ovviamente, per tempi sufficientemente lunghi:

\[
M_{z}\left( t_{0} \right)\exp\left( \dfrac{t_{0} - t}{T_{1}} \right) \rightarrow 0,\ \ t \rightarrow \infty
\]

Se \(t_{0} = 0\), risulta:

\[
M_{z}\left(t\right) = M_{0}\left\lbrack 1 - \exp\left( - \dfrac{t}{T_{1}} \right) \right\rbrack + M_{z}\left(0\right)\exp\left( - \dfrac{t}{T_{1}} \right)
\]

Supponendo che il valore iniziale della magnetizzazione sia nullo, la soluzione diventa:

\[
M_{z}\left(t\right) = M_{0}\left\lbrack 1 - \exp\left( - \dfrac{t}{T_{1}} \right) \right\rbrack
\]

Applicando il campo magnetico principale, da una condizione di equilibrio termodinamico, la componente longitudinale tende al valore di regime \(M_0\) con andamento esponenziale e costante di tempo \(T_{1}\). Il valore di regime è dato dall'equazione di Curie:

\[
M_{0} \simeq \ \dfrac{N}{V}\dfrac{\gamma^{2}\hslash^{2}}{4k_{B}T}B_{0}
\]

\begin{figure}[ht]
\centering
\includegraphics[width=4.32769in,height=3.10307in,alt={P3386\#yIS1}]{media/6_IntroMRI/image75.pdf}\caption{Andamento della soluzione longitudinale dell'equazione di Bloch}
\end{figure}

È possibile ritenere che la componente longitudinale raggiunga il regime dopo un tempo uguale a \(4 \div 5\) volte la costante di tempo \(T_{1}\). Generalmente il tempo di rilassamento longitudinale è dell'ordine del secondo, per cui si ritiene che il transitorio abbia una durata di circa \(5\ s\).

In pratica, una volta posizionato il paziente nel gantry della risonanza magnetica, è necessario attendere affinché gli spin raggiungano l'equilibrio termodinamico. L'imaging non può iniziare istantaneamente: in caso contrario, si otterrebbero immagini falsate. Generalmente, si attende un tempo di \(5 \div 10\,\text{s}\) prima di iniziare l'esame.

Per quanto riguarda le componenti trasversali, le equazioni che descrivono la loro evoluzione sono:

\[
\begin{cases}
\dfrac{dM_{x}}{dt} = \gamma B_{0}M_{y} - \dfrac{1}{T_{2}}M_{x} \\
\dfrac{dM_{y}}{dt} = - \gamma B_{0}M_{x} - \dfrac{1}{T_{2}}M_{y}
\end{cases}
\]

Dove \(\omega_{0} = \ \gamma B_{0}\). Il sistema può essere scritto come:

\[
\begin{cases}
\dfrac{dM_{x}}{dt} = \omega_{0}M_{y} - \dfrac{1}{T_{2}}M_{x} \\
\dfrac{dM_{y}}{dt} = - \omega_{0}M_{x} - \dfrac{1}{T_{2}}M_{y}
\end{cases}
\]

Si pone il sistema in forma matriciale:

\[
\dfrac{d}{dt}
\begin{pmatrix}
M_x \\
M_y
\end{pmatrix}
=
\begin{pmatrix}
-\dfrac{1}{T_2} & \omega_0 \\
-\omega_0 & -\dfrac{1}{T_2}
\end{pmatrix}
\begin{pmatrix}
M_x \\
M_y
\end{pmatrix}
\]

La soluzione di questa equazione è del tipo:

\[
{\vec{M}}_{\perp} = \vec{k}\exp\left( \lambda\mathbf{I}t \right)
\]

Sostituendo tale equazione nell'equazione differenziale si ha:

\[
\dfrac{d}{dt}\left\lbrack \vec{k}\exp\left( \lambda\mathbf{I}t \right) \right\rbrack = \begin{pmatrix}
 - \dfrac{1}{T_{2}} & \omega_{0} \\
 - \omega_{0} & - \dfrac{1}{T_{2}}
\end{pmatrix}\vec{k}\exp\left( \lambda\mathbf{I}t \right) \Leftrightarrow \lambda\mathbf{I}\vec{k}\exp\left( \lambda\mathbf{I}t \right) = \begin{pmatrix}
 - \dfrac{1}{T_{2}} & \omega_{0} \\
 - \omega_{0} & - \dfrac{1}{T_{2}}
\end{pmatrix}\vec{k}\exp\left( \lambda\mathbf{I}t \right)
\]

Moltiplicando per la matrice inversa a \(\vec{k}\exp\left( \lambda\mathbf{I}t \right)\) si ottiene:

\[
\lambda\mathbf{I} = \begin{pmatrix}
 - \dfrac{1}{T_{2}} & \omega_{0} \\
 - \omega_{0} & - \dfrac{1}{T_{2}}
\end{pmatrix} \Leftrightarrow \begin{pmatrix}
 - \dfrac{1}{T_{2}} - \lambda & \omega_{0} \\
 - \omega_{0} & - \dfrac{1}{T_{2}} - \lambda
\end{pmatrix} = \mathbf{0}
\]

Si pone il determinate della matrice individuata a zero, al fine di identificare i suoi autovettori. Conoscendo gli autovalori è possibile individuare anche il tipo di moto a cui sono soggette le componenti trasverse:

\[
\det\begin{pmatrix}
 - \dfrac{1}{T_{2}} - \lambda & \omega_{0} \\
 - \omega_{0} & - \dfrac{1}{T_{2}} - \lambda
\end{pmatrix} = 0 \Leftrightarrow \left( - \dfrac{1}{T_{2}} - \lambda \right)\left( - \dfrac{1}{T_{2}} - \lambda \right) + \omega_{0}^{2} = 0 \Leftrightarrow \left( \dfrac{1}{T_{2}} + \lambda \right)^{2} + \omega_{0}^{2} = 0
\]

Svolgendo i prodotti si ha:

\[
\dfrac{1}{T_{2}^{2}} + \dfrac{2\lambda}{T_{2}} + \lambda^{2} + \omega_{0}^{2} = 0 \Leftrightarrow \lambda^{2} + \dfrac{2\lambda}{T_{2}} + \omega_{0}^{2} + \dfrac{1}{T_{2}^{2}} = 0
\]

Si valuta il delta dell'equazione:

\[
\dfrac{\Delta}{4} = \dfrac{1}{T_{2}^{2}} - \left( \omega_{0}^{2} + \dfrac{1}{T_{2}^{2}} \right) = \dfrac{1}{T_{2}^{2}} - \omega_{0}^{2} - \dfrac{1}{T_{2}^{2}} = - \omega_{0}^{2}
\]

Il determinate è negativo per cui l'equazione non ammette soluzioni reali. Ne consegue che l'evoluzione delle componenti trasverse sono di tipo oscillanti smorzate.

Gli autovalori della matrice dei coefficienti sono:

\[
\lambda_{1,2} = - \dfrac{1}{T_{2}} \pm j\omega_{0} \Leftrightarrow \lambda_{1} = - \dfrac{1}{T_{2}} - j\omega_{0},\ \ \lambda_{2} = - \dfrac{1}{T_{2}} + j\omega_{0}
\]

Le soluzioni delle componenti trasverse sono del tipo:

\[
\begin{cases}
M_{x}\left(t\right) = k_{1,x}\exp\left\lbrack \left( - \dfrac{1}{T_{2}} - j\omega_{0} \right)t \right\rbrack + k_{2,x}\exp\left\lbrack \left( - \dfrac{1}{T_{2}} + j\omega_{0} \right)t \right\rbrack \\
M_{y}\left(t\right) = k_{1,y}\exp\left\lbrack \left( - \dfrac{1}{T_{2}} - j\omega_{0} \right)t \right\rbrack + k_{2,y}\exp\left\lbrack \left( - \dfrac{1}{T_{2}} + j\omega_{0} \right)t \right\rbrack
\end{cases}
\]

Per le proprietà degli esponenziali, è possibile scrivere:

\[
\begin{cases}
M_{x}\left(t\right) = k_{1,x}\exp\left( - \dfrac{t}{T_{2}} \right)\exp\left( - j\omega_{0}t \right) + k_{2,x}\exp\left( - \dfrac{t}{T_{2}} \right)\exp\left( j\omega_{0}t \right) \\
M_{y}\left(t\right) = k_{1,y}\exp\left( - \dfrac{t}{T_{2}} \right)\exp\left( - j\omega_{0}t \right) + k_{2,y}\exp\left( - \dfrac{t}{T_{2}} \right)\exp\left( j\omega_{0}t \right)
\end{cases}
\]

Raccogliendo i termini dipendenti da \(T_{2}\), si ottiene:

\[
\begin{cases}
M_{x}\left(t\right) = \left\lbrack k_{1,x}\exp\left( - j\omega_{0}t \right) + k_{2,x}\exp\left( j\omega_{0}t \right) \right\rbrack\exp\left( - \dfrac{t}{T_{2}} \right) \\
M_{y}\left(t\right) = \left\lbrack k_{1,y}\exp\left( - j\omega_{0}t \right) + k_{2,y}\exp\left( j\omega_{0}t \right) \right\rbrack\exp\left( - \dfrac{t}{T_{2}} \right)
\end{cases}
\]

Per individuare le costanti di integrazione bisognerebbe sostituire le equazioni individuate nel sistema di equazioni differenziali per le componenti trasverse e applicare le condizioni al contorno.

Per semplificare la trattazione si nota che la dipendenza da \(T_{2}\) è espressa mediante un fattore esponenziale moltiplicativo, quindi, è possibile scrivere le soluzioni delle componenti trasversali come:

\[
M_{x}\left(t\right) = m_{x}\left(t\right)\exp\left( - \dfrac{t}{T_{2}} \right),\ \ M_{y}\left(t\right) = m_{y}\left(t\right)\exp\left( - \dfrac{t}{T_{2}} \right)
\]

Le due equazioni differenziali si scrivono come:

\[\
\begin{cases}
\dfrac{dM_{x}}{dt} = \omega_{0}M_{y} - \dfrac{1}{T_{2}}M_{x} \\
\dfrac{dM_{y}}{dt} = - \omega_{0}M_{x} - \dfrac{1}{T_{2}}M_{y}
\end{cases} \  \Leftrightarrow \begin{cases}
\dfrac{d}{dt}\left\lbrack m_{x}\exp\left( - \dfrac{t}{T_{2}} \right) \right\rbrack = \omega_{0}m_{y}\exp\left( - \dfrac{t}{T_{2}} \right) - \dfrac{1}{T_{2}}m_{x}\exp\left( - \dfrac{t}{T_{2}} \right) \\
\dfrac{d}{dt}\left\lbrack m_{y}\exp\left( - \dfrac{t}{T_{2}} \right) \right\rbrack = - \omega_{0}m_{x}\exp\left( - \dfrac{t}{T_{2}} \right) - \dfrac{1}{T_{2}}m_{y}\exp\left( - \dfrac{t}{T_{2}} \right)
\end{cases}
\]

Svolgendo le derivate, si ha:

\[
\begin{cases}
\dfrac{dm_{x}}{dt}\exp\left( - \dfrac{t}{T_{2}} \right) - \dfrac{1}{T_{2}}m_{x}\exp\left( - \dfrac{t}{T_{2}} \right) = \omega_{0}m_{y}\exp\left( - \dfrac{t}{T_{2}} \right) - \dfrac{1}{T_{2}}m_{x}\exp\left( - \dfrac{t}{T_{2}} \right) \\
\dfrac{dm_{y}}{dt}\exp\left( - \dfrac{t}{T_{2}} \right) - \dfrac{1}{T_{2}}m_{y}\exp\left( - \dfrac{t}{T_{2}} \right) = - \omega_{0}m_{x}\exp\left( - \dfrac{t}{T_{2}} \right) - \dfrac{1}{T_{2}}m_{y}\exp\left( - \dfrac{t}{T_{2}} \right)
\end{cases}
 \]

Semplificando i termini comuni al primo e al secondo membro e il termine esponenziale dipendente da \(T_{2}\), si ottiene un semplice sistema di equazioni differenziali, la cui soluzione è nota:

\[
\begin{cases}
\dfrac{dm_{x}}{dt} = \omega_{0}m_{y} \\
\dfrac{dm_{y}}{dt} = - \omega_{0}m_{x}
\end{cases}
\]

Si deriva la prima equazione rispetto al tempo:

\[
\dfrac{d^{2}m_{x}}{dt} = \omega_{0}\dfrac{dm_{y}}{dt}
\]

Sostituendo la seconda equazione, si ottiene:

\[
\dfrac{d^{2}m_{x}}{dt} = - \omega_{0}^{2}m_{x}
\]

L'equazione differenziale ha come soluzione:

\[
m_{x} = C_{1}\exp\left( \lambda_{1}t \right) + C_{2}\exp\left( \lambda_{2}t \right) = C_{1}\exp\left( - j\omega_{0}t \right) + C_{2}\exp\left( j\omega_{0}t \right)
\]

Nota \(m_{x}\) è possibile ricavare l'equazione per \(m_{y}\) dalla prima relazione del sistema:

\[
\dfrac{dm_{x}}{dt} = \omega_{0}m_{y} \Leftrightarrow m_{y} = \dfrac{1}{\omega_{0}}\dfrac{dm_{x}}{dt} = \dfrac{1}{\omega_{0}}\dfrac{d}{dt}\left\lbrack C_{1}\exp\left( - j\omega_{0}t \right) + C_{2}\exp\left( j\omega_{0}t \right) \right\rbrack
\]

Svolgendo l'operazione di derivata si ottiene:

\[
m_{y} = \dfrac{1}{\omega_{0}}\left\lbrack - j\omega_{0}C_{1}\exp\left( - j\omega_{0}t \right) + j\omega_{0}C_{2}\exp\left( j\omega_{0}t \right) \right\rbrack = - jC_{1}\exp\left( - j\omega_{0}t \right) + jC_{2}\exp\left( j\omega_{0}t \right)
\]

In definitiva, si è ottenuto:

\[
\begin{cases}
m_{x} = C_{1}\exp\left( - j\omega_{0}t \right) + C_{2}\exp\left( j\omega_{0}t \right) \\
m_{y} = - jC_{1}\exp\left( - j\omega_{0}t \right) + jC_{2}\exp\left( j\omega_{0}t \right)
\end{cases} 
\]

Tornando alle componenti trasverse si ha:

\[
\begin{cases}
M_{x}\left(t\right) = \left\lbrack C_{1}\exp\left( - j\omega_{0}t \right) + C_{2}\exp\left( j\omega_{0}t \right) \right\rbrack\exp\left( - \dfrac{t}{T_{2}} \right) \\
M_{y}\left(t\right) = \left\lbrack - jC_{1}\exp\left( - j\omega_{0}t \right) + jC_{2}\exp\left( j\omega_{0}t \right) \right\rbrack\exp\left( - \dfrac{t}{T_{2}} \right)
\end{cases}
\]

Si suppone che il vettore di magnetizzazione sia noto all'istante \(t_{0}\). In questo modo è possibile individuare le costanti di integrazione:

\[
\begin{cases}
M_{x}\left( t_{0} \right) = \left\lbrack C_{1}\exp\left( - j\omega_{0}t_{0} \right) + C_{2}\exp\left( j\omega_{0}t_{0} \right) \right\rbrack\exp\left( - \dfrac{t_{0}}{T_{2}} \right) \\
M_{y}\left( t_{0} \right) = \left\lbrack - jC_{1}\exp\left( - j\omega_{0}t_{0} \right) + jC_{2}\exp\left( j\omega_{0}t_{0} \right) \right\rbrack\exp\left( - \dfrac{t_{0}}{T_{2}} \right)
\end{cases} 
\]

Si divide per il termine esponenziale per entrambe le equazioni:

\[
\begin{cases}
M_{x}\left( t_{0} \right)\exp\left( \dfrac{t_{0}}{T_{2}} \right) = C_{1}\exp\left( - j\omega_{0}t_{0} \right) + C_{2}\exp\left( j\omega_{0}t_{0} \right) \\
M_{y}\left( t_{0} \right)\exp\left( \dfrac{t_{0}}{T_{2}} \right) = - jC_{1}\exp\left( - j\omega_{0}t_{0} \right) + jC_{2}\exp\left( j\omega_{0}t_{0} \right)
\end{cases}
\]

Si divide per l'unità immaginaria nella seconda equazione:

\[
\begin{cases}
M_{x}\left( t_{0} \right)\exp\left( \dfrac{t_{0}}{T_{2}} \right) = C_{1}\exp\left( - j\omega_{0}t_{0} \right) + C_{2}\exp\left( j\omega_{0}t_{0} \right) \\
 - jM_{y}\left( t_{0} \right)\exp\left( \dfrac{t_{0}}{T_{2}} \right) = - C_{1}\exp\left( - j\omega_{0}t_{0} \right) + C_{2}\exp\left( j\omega_{0}t_{0} \right)
\end{cases}
\]

Sommando membro a membro si ottiene:

\[
M_{x}\left( t_{0} \right)\exp\left( \dfrac{t_{0}}{T_{2}} \right) - jM_{y}\left( t_{0} \right)\exp\left( \dfrac{t_{0}}{T_{2}} \right) = 2C_{2}\exp\left( j\omega_{0}t_{0} \right)
\]

Da cui si ricava \(C_{2}\)

\[
C_{2} = \dfrac{1}{2}\left\lbrack M_{x}\left( t_{0} \right)\exp\left( \dfrac{t_{0}}{T_{2}} \right) - jM_{y}\left( t_{0} \right)\exp\left( \dfrac{t_{0}}{T_{2}} \right) \right\rbrack\exp\left( - j\omega_{0}t_{0} \right)
\]

Sottraendo membro a membro si ricava \(C_{1}\):

\[
M_{x}\left( t_{0} \right)\exp\left( \dfrac{t_{0}}{T_{2}} \right) + jM_{y}\left( t_{0} \right)\exp\left( \dfrac{t_{0}}{T_{2}} \right) = 2C_{1}\exp\left( - j\omega_{0}t_{0} \right)
\]

Infatti, risulta:

\[
C_{1} = \dfrac{1}{2}\left\lbrack M_{x}\left( t_{0} \right)\exp\left( \dfrac{t_{0}}{T_{2}} \right) + jM_{y}\left( t_{0} \right)\exp\left( \dfrac{t_{0}}{T_{2}} \right) \right\rbrack\exp\left( j\omega_{0}t_{0} \right)
\]

La soluzione lungo \({\hat{\imath}}_{x}\) è:

\[
M_{x}\left(t\right) = \left\lbrack C_{1}\exp\left( - j\omega_{0}t \right) + C_{2}\exp\left( j\omega_{0}t \right) \right\rbrack\exp\left( - \dfrac{t}{T_{2}} \right)
\]

Dove per \(C_1\) risulta:

\begin{align*}
C_{1}\exp\left( - j\omega_{0}t \right) &= 
\dfrac{1}{2}\left[ M_{x}\left( t_{0} \right)\exp\left( \dfrac{t_{0}}{T_{2}} \right) 
+ jM_{y}\left( t_{0} \right)\exp\left( \dfrac{t_{0}}{T_{2}} \right) \right] \exp\left( j\omega_{0}t_{0} \right)\exp\left( - j\omega_{0}t \right) = \\
&= \left[ \dfrac{1}{2}M_{x}\left( t_{0} \right) + j\dfrac{1}{2}M_{y}\left( t_{0} \right) \right]
\exp\left( \dfrac{t_{0}}{T_{2}} \right)\exp\left[ - j\omega_{0}\left( t - t_{0} \right) \right] = \\
&= \left[ \dfrac{1}{2}M_{x}\left( t_{0} \right) - \dfrac{1}{2j}M_{y}\left( t_{0} \right) \right]
\exp\left( \dfrac{t_{0}}{T_{2}} \right)\exp\left[ - j\omega_{0}\left( t - t_{0} \right) \right]
\end{align*}


mentre per \(C_2\):

\begin{align*}
C_{2}\exp\left( j\omega_{0}t \right) &= 
\dfrac{1}{2}\left[ M_{x}\left( t_{0} \right)\exp\left( \dfrac{t_{0}}{T_{2}} \right) 
- jM_{y}\left( t_{0} \right)\exp\left( \dfrac{t_{0}}{T_{2}} \right) \right] \exp\left( - j\omega_{0}t_{0} \right)\exp\left( j\omega_{0}t \right) = \\
&= \left[ \dfrac{1}{2}M_{x}\left( t_{0} \right) - j\dfrac{1}{2}M_{y}\left( t_{0} \right) \right]
\exp\left( - j\omega_{0}t_{0} \right)\exp\left[ j\omega_{0}\left( t - t_{0} \right) \right] = \\
&= \left[ \dfrac{1}{2}M_{x}\left( t_{0} \right) + \dfrac{1}{2j}M_{y}\left( t_{0} \right) \right]
\exp\left( - j\omega_{0}t_{0} \right)\exp\left[ j\omega_{0}\left( t - t_{0} \right) \right]
\end{align*}

Quindi, la soluzione lungo \({\hat{\imath}}_{x}\) si scrive come:

\begin{align*}
M_{x}(t) &= \left\{ 
\left[ \dfrac{1}{2}M_{x}(t_{0}) - \dfrac{1}{2j}M_{y}(t_{0}) \right]
\exp\left( \dfrac{t_{0}}{T_{2}} \right)
\exp\left[ -j\omega_{0}(t - t_{0}) \right] \right.+ \\
&\quad + \left. 
\left[ \dfrac{1}{2}M_{x}(t_{0}) - j\dfrac{1}{2}M_{y}(t_{0}) \right]
\exp\left( -j\omega_{0}t_{0} \right)
\exp\left[ j\omega_{0}(t - t_{0}) \right] 
\right\} \exp\left( -\dfrac{t}{T_{2}} \right)
\end{align*}

Raccogliendo opportunamente si ha:

\begin{align*}
M_{x}(t) &= \left\{ 
M_{x}(t_{0}) \dfrac{
\exp\left[ j\omega_{0}(t - t_{0}) \right] + 
\exp\left[ -j\omega_{0}(t - t_{0}) \right]
}{2} \right. +\\
&\quad + \left.
M_{y}(t_{0}) \dfrac{
\exp\left[ j\omega_{0}(t - t_{0}) \right] - 
\exp\left[ -j\omega_{0}(t - t_{0}) \right]
}{2j}
\right\} \exp\left( -\dfrac{t - t_{0}}{T_{2}} \right)
\end{align*}

Per le relazioni di Eulero, risulta:

\[M_{x}\left(t\right) = \left\{ M_{x}\left( t_{0} \right)\cos\left\lbrack \omega_{0}\left( t - t_{0} \right) \right\rbrack + M_{y}\left( t_{0} \right)\sin{\cos\left\lbrack \omega_{0}\left( t - t_{0} \right) \right\rbrack} \right\}\exp\left( - \dfrac{t - t_{0}}{T_{2}} \right)\]

Analogamente, per la componente lungo \({\hat{\imath}}_{y}\) si ha:

\[M_{y}\left(t\right) = \left\lbrack - jC_{1}\exp\left( - j\omega_{0}t \right) + jC_{2}\exp\left( j\omega_{0}t \right) \right\rbrack\exp\left( - \dfrac{t}{T_{2}} \right)\]

Svolgendo i calcoli per \(C_1\):

\begin{align*}
- jC_{1}\exp\left( - j\omega_{0}t \right) &= 
- j\dfrac{1}{2}\left[ M_{x}(t_{0})\exp\left( \dfrac{t_{0}}{T_{2}} \right) 
+ jM_{y}(t_{0})\exp\left( \dfrac{t_{0}}{T_{2}} \right) \right]  \exp\left( j\omega_{0}t_{0} \right)\exp\left( - j\omega_{0}t \right) =  \\
&= \left[ - j\dfrac{1}{2}M_{x}(t_{0}) + \dfrac{1}{2}M_{y}(t_{0}) \right]
\exp\left( \dfrac{t_{0}}{T_{2}} \right)\exp\left[ - j\omega_{0}(t - t_{0}) \right] =\\
&= \left[ \dfrac{1}{2j}M_{x}(t_{0}) + \dfrac{1}{2}M_{y}(t_{0}) \right]
\exp\left( \dfrac{t_{0}}{T_{2}} \right)\exp\left[ - j\omega_{0}(t - t_{0}) \right]
\end{align*}

mentre per \(C_2\)

\begin{align*}
jC_{2}\exp\left( j\omega_{0}t \right) &= 
j\dfrac{1}{2}\left[ M_{x}(t_{0})\exp\left( \dfrac{t_{0}}{T_{2}} \right) 
- jM_{y}(t_{0})\exp\left( \dfrac{t_{0}}{T_{2}} \right) \right] \exp\left( - j\omega_{0}t_{0} \right)\exp\left( j\omega_{0}t \right) = \\
&= \left[ j\dfrac{1}{2}M_{x}(t_{0}) + \dfrac{1}{2}M_{y}(t_{0}) \right]
\exp\left( \dfrac{t_{0}}{T_{2}} \right)\exp\left[ j\omega_{0}(t - t_{0}) \right] = \\
&= \left[ -\dfrac{1}{2j}M_{x}(t_{0}) + \dfrac{1}{2}M_{y}(t_{0}) \right]
\exp\left( \dfrac{t_{0}}{T_{2}} \right)\exp\left[ j\omega_{0}(t - t_{0}) \right]
\end{align*}

Quindi, la soluzione lungo \({\hat{\imath}}_{y}\) si scrive come:

\begin{align*}
M_{y}(t) &= \left\{ 
\left[ \dfrac{1}{2j}M_{x}(t_{0}) + \dfrac{1}{2}M_{y}(t_{0}) \right]
\exp\left( \dfrac{t_{0}}{T_{2}} \right)
\exp\left[ -j\omega_{0}(t - t_{0}) \right] \right. +\\
&\quad + \left. 
\left[ -\dfrac{1}{2j}M_{x}(t_{0}) + \dfrac{1}{2}M_{y}(t_{0}) \right]
\exp\left( \dfrac{t_{0}}{T_{2}} \right)
\exp\left[ j\omega_{0}(t - t_{0}) \right]
\right\} \exp\left( -\dfrac{t}{T_{2}} \right)
\end{align*}

Raccogliendo opportunamente si ha:

\begin{align*}
M_{y}(t) &= \left\{ 
- M_{x}(t_{0}) \dfrac{
\exp\left[ j\omega_{0}(t - t_{0}) \right] - 
\exp\left[ -j\omega_{0}(t - t_{0}) \right]
}{2j} \right. \\
&\qquad + \left.
M_{y}(t_{0}) \dfrac{
\exp\left[ j\omega_{0}(t - t_{0}) \right] + 
\exp\left[ -j\omega_{0}(t - t_{0}) \right]
}{2}
\right\} \exp\left( -\dfrac{t - t_{0}}{T_{2}} \right)
\end{align*}

Per l'equivalenza di Eulero risulta:

\[
M_{y}\left(t\right) = \left\{ - M_{x}\left( t_{0} \right)\sin\left\lbrack \omega_{0}\left( t - t_{0} \right) \right\rbrack + M_{y}\left( t_{0} \right)\cos\left\lbrack \omega_{0}\left( t - t_{0} \right) \right\rbrack \right\}\exp\left( - \dfrac{t - t_{0}}{T_{2}} \right)
\]

In definitiva, le componenti trasverse evolvono secondo le seguenti equazioni:

\[
\begin{cases}
M_{x}\left(t\right) = \left\{ M_{x}\left( t_{0} \right)\cos\left\lbrack \omega_{0}\left( t - t_{0} \right) \right\rbrack + M_{y}\left( t_{0} \right)\sin\left\lbrack \omega_{0}\left( t - t_{0} \right) \right\rbrack \right\}\exp\left( - \dfrac{t - t_{0}}{T_{2}} \right) \\
M_{y}\left(t\right) = \left\{ - M_{x}\left( t_{0} \right)\sin\left\lbrack \omega_{0}\left( t - t_{0} \right) \right\rbrack + M_{y}\left( t_{0} \right)\cos\left\lbrack \omega_{0}\left( t - t_{0} \right) \right\rbrack \right\}\exp\left( - \dfrac{t - t_{0}}{T_{2}} \right)
\end{cases}
\]

Queste equazioni esprimono un moto di \textbf{precessione smorzata} attorno all'asse \(z\), con frequenza angolare \(\omega_0\) e tempo di decadimento \(T_2\), a partire dalle condizioni iniziali \(M_x(t_0)\) e \(M_y(t_0)\).

Se l'istante iniziale coincide con l'origine dei tempi, \(t_{0} = 0\), risulta:

\[
\begin{cases}
M_{x}\left(t\right) = \left\lbrack M_{x}\left(0\right)\cos\left( \omega_{0}t \right) + M_{y}\left(0\right)\sin\left( \omega_{0}t \right) \right\rbrack\exp\left( - \dfrac{t}{T_{2}} \right) \\
M_{y}\left(t\right) = \left\lbrack - M_{x}\left(0\right)\sin\left( \omega_{0}t \right) + M_{y}\left(0\right)\cos\left( \omega_{0}t \right) \right\rbrack\exp\left( - \dfrac{t - t_{0}}{T_{2}} \right)
\end{cases}
\]

Per un tempo sufficientemente lungo, circa \(4 \div 5\) volte \(T_{2}\) è possibile ritenere la risposta transitoria esaurita, per cui le componenti trasversali sono nulle:

\[
\begin{cases}
    M_{x}\left(t\right) \rightarrow 0, & t \rightarrow \infty \\
    M_{y}\left(t\right) \rightarrow 0, & t \rightarrow \infty
\end{cases}
\]

Siccome il tempo di rilassamento trasversale è dell'ordine di \(100\ ms\), il tempo necessario affinché le componenti trasversali del vettore di magnetizzazione si annullino è dell'ordine di \(500\ ms\), ovvero un ordine di grandezza inferiore rispetto al tempo che la componente longitudinale impiega per raggiungere il regime.

È possibile scrivere la soluzione delle componenti trasversali in forma complessa, introducendo il fasore \(M_{+}\), definito come:

\begin{align*}
M_{+}(t) &= M_{x}(t) + jM_{y}(t) =\\
&= \left[
M_{x}(0)\cos(\omega_{0}t) + M_{y}(0)\sin(\omega_{0}t)
- jM_{x}(0)\sin(\omega_{0}t) + jM_{y}(0)\cos(\omega_{0}t)
\right] \exp\left( -\dfrac{t}{T_{2}} \right)
\end{align*}

Raccogliendo si ottiene:

\begin{align*}
M_{+}(t) &= \left\{
M_{x}(0)\left[ \cos(\omega_{0}t) - j\sin(\omega_{0}t) \right]
+ jM_{y}(0)\left[ \cos(\omega_{0}t) - j\sin(\omega_{0}t) \right]
\right\} \exp\left( -\dfrac{t}{T_{2}} \right) \\
&= \left\{
\left[ \cos(\omega_{0}t) - j\sin(\omega_{0}t) \right]
\left[ M_{x}(0) + jM_{y}(0) \right]
\right\} \exp\left( -\dfrac{t}{T_{2}} \right)
\end{align*}

Dove:

\[M_{+}\left(0\right) = M_{x}\left(0\right) + jM_{y}\left(0\right)\]

Inoltre, per l'identità di Eulero e le proprietà delle funzioni trigonometriche:

\[\cos\left( \omega_{0}t \right) - j\sin\left( \omega_{0}t \right) = j\sin\left( - \omega_{0}t \right) + \cos\left( - \omega_{0}t \right) = \exp\left( - j\omega_{0}t \right)\]

Per cui il fasore si scrive come:

\[M_{+}\left(t\right) = M_{+}\left(0\right)\exp\left( - j\omega_{0}t \right)\exp\left( - \dfrac{t}{T_{2}} \right) = M_{+}\left(0\right)\exp\left\lbrack - \left( j\omega_{0}t + \dfrac{t}{T_{2}} \right) \right\rbrack\ \]

Tale relazione fornisce l'evoluzione temporale del vettore di magnetizzazione nel piano complesso. Il moto nel piano trasverso avviene con pulsazione \(\omega_{0}\) e con ampiezza che decade esponenzialmente con constante ti tempo \(T_{2}\). In circa \(500\ ms\) la componente trasversa del vettore di magnetizzazione si annulla.

\begin{figure}[ht]
\centering
\includegraphics[width=2.7037in,height=2.1196in,alt={P3496\#yIS1}]{media/6_IntroMRI/image76.pdf}\caption{Andamento del vettore di magnetizzazione nel piano trasverso}
\end{figure}

Il vettore di magnetizzazione \(\vec{M}\), in definitiva, evolve secondo un movimento rotatorio decrescente nel piano trasverso con constante di tempo \(T_{2}\) e con andamento esponenziale crescente, e costante di tempo \(T_{1}\), lungo la direzione longitudinale. Le componenti del vettore di magnetizzazione sono:

\[
 \begin{cases}
M_{x}\left(t\right) = \left\lbrack M_{x}\left(0\right)\cos\left( \omega_{0}t \right) + M_{y}\left(0\right)\sin\left( \omega_{0}t \right) \right\rbrack\exp\left( - \dfrac{t}{T_{2}} \right) \\
M_{y}\left(t\right) = \left\lbrack - M_{x}\left(0\right)\sin\left( \omega_{0}t \right) + M_{y}\left(0\right)\cos\left( \omega_{0}t \right) \right\rbrack\exp\left( - \dfrac{t - t_{0}}{T_{2}} \right) \\
M_{z}\left(t\right) = M_{0}\left\lbrack 1 - \exp\left( - \dfrac{t}{T_{1}} \right) \right\rbrack
\end{cases} 
\]

La composizione dei due moti determina che il vettore di magnetizzazione ha un modulo variabile nel tempo.

Si considera come condizione iniziale il vettore di magnetizzazione a valle di un ribaltamento nel piano trasverso a opera di un impulso a radiofrequenza. Al tempo \(t = 0\), il vettore di magnetizzazione è:

\[\vec{M}\left(0\right) = M_{x}\left(0\right){\hat{\imath}}_{x} + M_{y}\left(0\right){\hat{\imath}}_{y}\]

La componente lungo \({\hat{\imath}}_{z}\) parte dal valore nullo e si porta al valore di regime in un tempo di \(5 \div 10\ s\), mentre quelle trasversali vanno a zero in un tempo di \(0.5 \div 1\ s\).

La curva descritta dal vettore di magnetizzazione tende a raggiungere il valore di regime \(M_{0}\) sull'asse delle \({\hat{\imath}}_{z}\) mediante un andamento a spirale, convergente sull'asse del campo principale. Il modo elicoidale a raggio variabile lo si ritrova dopo una perturbazione ed è il moto in cui il vettore \(\vec{M}\) torna all'equilibrio termodinamico per effetto del campo principale \(B_{0}\). I tempi di rilassamento, quindi, interessano il ritorno all'equilibrio.

\begin{figure}[ht]
\centering
\includegraphics[width=3.1141in,height=2.12733in,alt={P3505\#yIS1}]{media/6_IntroMRI/image77.pdf}\caption{Evoluzione temporale del vettore di magnetizzazione a valle di un ribaltamento sul piano trasverso}
\end{figure}

\subsection{Vettore magnetizzazione durante una perturbazione}\label{vettore-magnetizzazione-durante-una-perturbazione}

Si analizza l'evoluzione temporale del vettore di magnetizzazione durante l'applicazione di un impulso a radiofrequenza, che perturba l'equilibrio del sistema. La trattazione è condotta nel sistema di riferimento rotante, per una maggiore semplicità interpretativa.

Il campo magnetico efficace visto da uno spin nel sistema rotante, con impulso a radiofrequenza diretto lungo l'asse \({\hat{\imath}}_{x'}\), è:

\[
{\vec{B}_{\text{eff}}} = \left( B_{0} - \dfrac{\omega}{\gamma} \right){\hat{\imath}}_{z} + B_{1}{\hat{\imath}}_{x'}
\]

dove \(\omega\) è la velocità angolare del sistema rotante.

L'equazione di Bloch nel sistema rotante è formalmente analoga a quella nel sistema del laboratorio:

\[
\left( \dfrac{d\vec{M}}{dt} \right)' = \gamma\vec{M} \times {\vec{B}_{\text{eff}}} + \dfrac{1}{T_{1}}\left( M_{0} - M_{z}\  \right){\hat{\imath}}_{z} - \dfrac{1}{T_{2}}{\vec{M}}_{\perp}
\]

Questa equazione è stata ottenuta senza applicare nessuna ipotesi sul sistema di riferimento, quindi, è valida sia in sistemi inerziali che non inerziali. In particolare, la componente longitudinale del vettore di magnetizzazione evolve in maniera indipendente da quelle trasversali, quindi, il modo lungo l'asse \({\hat{\imath}}_{z}\) si conserva.

Si proietta l'equazione vettoriale di Bloch lungo gli assi, svolgendo il prodotto vettoriale:

\[
\vec{M} \times {\vec{B}_{\text{eff}}} =  \begin{vmatrix}
{\hat{\imath}}_{x'} & {\hat{\imath}}_{y'} & {\hat{\imath}}_{z}\  \\
M_{x'} & M_{y'} & M_{z} \\
B_{1} & 0 & B_{0} - \dfrac{\omega}{\gamma}
\end{vmatrix}  = M_{y'}\left( B_{0} - \dfrac{\omega}{\gamma} \right){\hat{\imath}}_{x'} + M_{z}B_{1}{\hat{\imath}}_{y'} - M_{y'}B_{1}{\hat{\imath}}_{z} - M_{x'}\left( B_{0} - \dfrac{\omega}{\gamma} \right){\hat{\imath}}_{y'}
\]

Raccogliendo i termini, si ottiene:

\begin{align*}
\vec{M} \times \vec{B}_{\text{eff}} &= 
M_{y'}\left( B_{0} - \dfrac{\omega}{\gamma} \right)\hat{\imath}_{x'} 
+ M_{z}B_{1}\hat{\imath}_{y'} 
- M_{y'}B_{1}\hat{\imath}_{z} 
- M_{x'}\left( B_{0} - \dfrac{\omega}{\gamma} \right)\hat{\imath}_{y'} = \\
&= M_{y'}\left( B_{0} - \dfrac{\omega}{\gamma} \right)\hat{\imath}_{x'} 
- \left[ M_{x'}\left( B_{0} - \dfrac{\omega}{\gamma} \right) - M_{z}B_{1} \right]\hat{\imath}_{y'} 
- M_{y'}B_{1}\hat{\imath}_{z}
\end{align*}

L'equazione vettoriale di Bloch si scrive:

\begin{align*}
\left( \dfrac{d\vec{M}}{dt} \right)' &= 
\gamma \vec{M} \times \vec{B}_{\text{eff}} 
+ \dfrac{1}{T_{1}}\left( M_{0} - M_{z} \right)\hat{\imath}_{z} 
- \dfrac{1}{T_{2}} \vec{M}_{\perp} = \\
&= \gamma \left\{
M_{y'}\left( B_{0} - \dfrac{\omega}{\gamma} \right)\hat{\imath}_{x'} 
- \left[ M_{x'}\left( B_{0} - \dfrac{\omega}{\gamma} \right) - M_{z}B_{1} \right]\hat{\imath}_{y'} 
- M_{y'}B_{1}\hat{\imath}_{z}
\right\} + \\
&\quad + \dfrac{1}{T_{1}}\left( M_{0} - M_{z} \right)\hat{\imath}_{z} 
- \dfrac{1}{T_{2}}\left( M_{x'}\hat{\imath}_{x'} + M_{y'}\hat{\imath}_{y'} \right)
\end{align*}

Scomponendo lungo gli assi del sistema rotante si ottiene:

\[
\begin{cases}
\left( \dfrac{dM_{x'}}{dt} \right)' = \gamma M_{y'}\left( B_{0} - \dfrac{\omega}{\gamma} \right) - \dfrac{1}{T_{2}}M_{x'} \\
\left( \dfrac{dM_{y'}}{dt} \right)' = \gamma M_{z}B_{1} - \gamma M_{x'}\left( B_{0} - \dfrac{\omega}{\gamma} \right) - \dfrac{1}{T_{2}}M_{y'} \\
\left( \dfrac{dM_{z}}{dt} \right)' = \dfrac{1}{T_{1}}\left( M_{0} - M_{z}\  \right) - \gamma M_{y'}B_{1}
\end{cases} 
\]

Definendo \(\omega_0 = \gamma B_0\), \(\omega_1 = \gamma B_1\), e \(\Delta \omega = \omega_0 - \omega\), si ha:

\[
\begin{cases}
\left( \dfrac{dM_{x'}}{dt} \right)' = \left( \omega_{0} - \omega \right)M_{y'} - \dfrac{1}{T_{2}}M_{x'} \\
\left( \dfrac{dM_{y'}}{dt} \right)' = \omega_{1}M_{z} - \left( \omega_{0} - \omega \right)M_{x'} - \dfrac{1}{T_{2}}M_{y'} \\
\left( \dfrac{dM_{z}}{dt} \right)' = \dfrac{1}{T_{1}}\left( M_{0} - M_{z}\  \right) - \omega_{1}M_{y'}
\end{cases}
\]

Nelle equazioni \(\omega_{0}\) è la frequenza di Larmor, \(\omega_{1}\) è la frequenza del campo a radiofrequenza, mentre \(\omega\) è la frequenza con cui ruota il sistema di riferimento.

Si definisce \(\Delta \omega = \omega_{0} - \omega\) la deviazione dalla condizione ideale in cui \(\omega_{0} = \omega\). Questa deviazione è legata alle disomogeneità del campo o alla variazioni delle frequenze dell'impulso utilizzato. Affinché il vettore di magnetizzazione ruoti dell'angolo desiderato, è necessario che l'impulso a radiofrequenza abbia una durata di qualche \(ms\). Nel dettaglio, il periodo di applicazione del campo a radiofrequenza, che genera la processione intorno a \({\hat{\imath}}_{x'}\), è di qualche millisecondo ed è legato alla pulsazione \(\omega_{1}\) dalla relazione:

\[
\omega_{1} = \dfrac{2\pi}{T}
\]

Dato che il tempo di rilassamento longitudinale \(T_{1}\) è dell'ordine dei secondi, risulta che:

\[T \ll T_{1} \Leftrightarrow \dfrac{1}{T} \gg \dfrac{1}{T_{1}}\]

A meno di un fattore \(2\pi\), risulta:

\[\dfrac{2\pi}{T} = \omega_{1} \gg \dfrac{1}{T_{1}}\]

Analogo discorso vale per il tempo di rilassamento trasversale \(T_{2}\), dell'ordine dei \(500\ ms\). Rispetto alla pulsazione \(\omega_{1}\), i termini che evolvono con costanti di tempo \(T_{1}\) e \(T_{2}\) possono essere trascurati, in quanto molto più lenti. L'evoluzione del vettore di magnetizzazione, in ultima analisi, non dipende dai tempi di rilassamento:

\[
\begin{cases}
\left( \dfrac{dM_{x'}}{dt} \right)' = \left( \omega_{0} - \omega \right)M_{y'} \\
\left( \dfrac{dM_{y'}}{dt} \right)' = \omega_{1}M_{z} - \left( \omega_{0} - \omega \right)M_{x'} \\
\left( \dfrac{dM_{z}}{dt} \right)' = - \omega_{1}M_{y'}
\end{cases}
\]

Se il sistema ruota con una pulsazione angolare molto prossima a quella di Larmor, ovvero in condizione di risonanza, risulta:

\[
\omega_{0} \simeq \omega \Leftrightarrow \Delta \omega = \omega_{0} - \omega \simeq 0
\]

È possibile, quindi, trascurare i termini contenenti \(\Delta \omega\), semplificando il sistema:

\[
\begin{cases}
\left( \dfrac{dM_{x'}}{dt} \right)' = 0 \\
\left( \dfrac{dM_{y'}}{dt} \right)' = \omega_{1}M_{z} \\
\left( \dfrac{dM_{z}}{dt} \right)' = - \omega_{1}M_{y'}
\end{cases}
\]

Questo sistema descrive una \textbf{precessione del vettore di magnetizzazione attorno all'asse} \({\hat{\imath}}_{x'}\), indotta dall'impulso a radiofrequenza.

Per le sequenze applicate normalmente nella pratica, in definitiva, si ritiene che la rotazione del vettore di magnetizzazione avvenga senza l'influenza dei tempi di rilassamento, poiché l'evoluzione temporale legata all'impulso a radiofrequenza è molto più veloce di quella relativa ai fenomeni di rilassamento. Si approssima, inoltre, la frequenza di risonanza con quella del campo a radiofrequenza. Anche in presenza di tali approssimazioni, i risultati ottenuti sono attendibili, nel senso che concordi ai dati sperimentali. Il ritorno all'equilibrio è, invece, caratterizzato dai tempi di rilassamento del tessuto.

L'evoluzione del vettore di magnetizzazione, legato all'applicazione dell'impulso a radiofrequenza, si compone di due fasi:

\begin{itemize}
    \item \textbf{Ribaltamento}: Il vettore di magnetizzazione, inizialmente all'equilibrio lungo \(\hat{\imath}_z\), viene ruotato nel piano trasverso (\(x'y'\)) tramite un impulso a radiofrequenza a polarizzazione lineare o circolare. Questo processo è molto rapido rispetto ai tempi di rilassamento, per cui si trascurano \(T_1\) e \(T_2\).
\end{itemize}

\begin{figure}[ht]
  \centering
  \begin{minipage}[b]{0.45\textwidth}
    \centering
    \includegraphics[width=3.17in,height=3.20in]{media/6_IntroMRI/image78.pdf}
    \caption*{(a)}
  \end{minipage}
  \hfill
  \begin{minipage}[b]{0.45\textwidth}
    \centering
    \includegraphics[width=2.91in,height=3.20in]{media/6_IntroMRI/image79.pdf}
    \caption*{(b)}
  \end{minipage}
  \caption{L'applicazione dell'impulso RF porta a una precessione intorno all'asse \(y'\). 
  (a) e (b) mostrano due fasi della rotazione.}
  \label{fig:rf_precession}
\end{figure}

\begin{itemize}
    \item \textbf{Recupero}: Dopo la perturbazione, il vettore di magnetizzazione torna all'equilibrio termodinamico. La componente longitudinale \(M_z\) cresce esponenzialmente con costante di tempo \(T_1\), mentre le componenti trasversali \(M_{x'}, M_{y'}\) decadono con costante \(T_2\).
\end{itemize}

\subsection{Sequenza FID}\label{sequenza-fid}

Per ottenere una misura del vettore di magnetizzazione, è necessario perturbare l'equilibrio raggiunto dagli spin nei tessuti del paziente. Dal segnale registrato è possibile ricavare informazioni sui tempi di rilassamento \textit{spin-lattice} (\(T_1\)) e \textit{spin-spin} (\(T_2\)), così da caratterizzare il tessuto \cite{BrownMRI}.

La sequenza più semplice per perturbare il sistema è detta FID (\textit{Free Induction Decay}). In essa si applica una radiazione a radiofrequenza, rappresentata nei diagrammi con un impulso rettangolare (\(\text{rect}\left(t\right)\)) o con un pacchetto di onde sinusoidali, a frequenza \(\omega \simeq \omega_0\).

Successivamente, si registra il segnale \(s\left(t\right)\) dovuto al ritorno all'equilibrio del vettore di magnetizzazione. Le antenne in ricezione iniziano ad acquisire il segnale subito dopo la fine dell'impulso.

Il segnale registrato è proporzionale alla componente trasversa del vettore di magnetizzazione. Le antenne ricevono una sinusoide a frequenza \(\omega_0\), smorzata esponenzialmente con costante di tempo \(T_2\). In notazione complessa, il segnale è proporzionale al fasore \(M_+(t)\):


\[
M_{+}\left(t\right) = M_{+}\left(0\right)\exp\left( - \dfrac{t}{T_{2}} \right)\exp\left( - j\omega_{0}t \right)
\]


dove \(M_{+}\left(0\right)\) è proporzionale al valore iniziale del vettore di magnetizzazione \(M_0\), che dipende dalla densità protonica \(\rho\), indicante il numero di protoni nel volumetto \(V\).

Poiché non è presente alcun gradiente di campo, ogni spin nel volume del paziente precessa alla stessa frequenza di Larmor \(\omega_0\). L'applicazione del campo rotante a frequenza prossima a \(\omega_0\) eccita tutti gli spin nel corpo del paziente. In questo caso, il volume del paziente è considerato come un unico blocco, non suddiviso in volumetti elementari.

La sequenza FID non fornisce informazioni sulla composizione chimica dei tessuti, ovvero non permette di determinare i tempi di rilassamento di ciascun volumetto. In altre parole, la sequenza FID non è utile per l'\textit{imaging}, ma consente di correggere errori introdotti dalle disomogeneità del campo magnetico principale. Le misure FID possono fornire un'indicazione su quanto il campo magnetico principale si discosta dall'ideale, cioè uniforme nello spazio occupato dal paziente.

Il tempo di rilassamento trasversale \(T_2\) è determinato dalle interazioni spin-spin, che alterano localmente il campo magnetico percepito da ciascuno spin. Se il campo magnetico esterno è disomogeneo, il campo locale varia con la posizione, introducendo un'ulteriore fonte di disturbo.

In generale, i produttori di apparecchiature per risonanza magnetica garantiscono un'omogeneità del campo nel gantry dell'ordine di una parte per milione (ppm), all'interno di una sfera centrata nel gantry con raggio di circa \(20\,\text{cm}\).


\begin{figure}[ht]
\centering
\includegraphics[width=3.73125in,height=2.04747in,alt={P3558\#yIS1}]{media/6_IntroMRI/image80.pdf}\caption{Regione di spazio del gantry in cui il campo è omogeneo}
\end{figure}

Se il valore nominale del campo è \(1.5\,\text{T}\), la variazione all'interno della sfera è dell'ordine di \(10^{-6}\), ovvero una variazione massima di circa \(1.5\,\mu\text{T}\). Queste disomogeneità si sommano ai campi locali percepiti dai singoli spin, contribuendo a una riduzione del tempo di rilassamento trasversale \(T_2\).

Si introduce quindi il tempo \(T_2^*\), che tiene conto sia delle disomogeneità del campo principale sia delle interazioni spin-spin. Il primo fenomeno è quantificato da un tempo \(T_2'\), legato alla tecnologia costruttiva del magnete e alla qualità del campo \(B_0\).


Il tempo \(T_{2}^{*}\) è definito come:

\[
\dfrac{1}{T_{2}^{*}} = \dfrac{1}{T_{2}} + \dfrac{1}{T_{2}'}
\]


Con la sequenza FID è possibile stimare \(T_2^*\) e, di conseguenza, valutare quanto il campo magnetico principale varia nello spazio. Conoscendo \(T_2^*\), è possibile applicare algoritmi di correzione alle immagini, migliorando l'affidabilità nella stima dei tempi di rilassamento \(T_1\) e \(T_2\).
