\begin{center}
\vfill
    \chapter{Segnale della risonanza magnetica}
    \label{blx:MRISignal\therefsection}
\vfill

\minitoc
\newpage
\end{center}
\justify


\section{Segnale registrato in RMI}\label{segnale-registrato-in-rmi}

Per ottenere una misura del vettore di magnetizzazione, è necessario perturbare l'equilibrio raggiunto dagli spin nei tessuti del paziente. Una volta perturbato il sistema mediante un impulso a radiofrequenza, il vettore di magnetizzazione ritorna al valore di equilibrio secondo un'evoluzione dettata dai tempi di rilassamento spin-reticolo (\(T_1\)) e spin-spin (\(T_2\)).

All'interno del gantry sono presenti antenne (\textit{RF coil}) disposte ortogonalmente tra loro, in modo da generare un campo a polarizzazione circolare. In linea di principio, queste antenne possono essere pensate come due spire percorse da corrente, una posta nel piano verticale e l'altra nel piano orizzontale rispetto al corpo del paziente.

\begin{figure}[ht]
\centering
\includegraphics[width=6.27171in,height=4.22976in,alt={P3570\#yIS1}]{media/7_MRISignal/image81.pdf}\caption{Schema strutturale di gantry per risonanza magnetica}
\end{figure}

Nel gantry sono presenti anche bobine dedicate alla generazione di gradienti di campo, utili per variare la frequenza di Larmor lungo la direzione \({\hat{i}}_{z}\)\ secondo un andamento noto. Questo consente di selezionare una singola \textit{slice} del corpo umano.


Una volta terminata l'erogazione del campo a radiofrequenza, la magnetizzazione si rilassa, ovvero torna all'equilibrio termodinamico con il campo principale. Durante questa fase, è possibile captare il segnale con apposite antenne, spesso le stesse utilizzate per la trasmissione, ma talvolta dedicate esclusivamente alla ricezione.

Le antenne possono essere pensate come spire percorse da corrente, sulle quali il rilassamento della magnetizzazione induce una forza elettromotrice (f.e.m. o e.m.f) secondo la legge di Faraday-Neumann-Lenz. Tramite la f.e.m. indotta è possibile ricostruire immagini di sezioni del corpo umano.

\section{Valutazione f.e.m. indotta sulle antenne riceventi}\label{valutazione-f.e.m.-indotta-sulle-antenne-riceventi}

Per ricostruire immagini di sezioni del paziente, è necessario comprendere come la forza elettromotrice (f.e.m.) indotta sull'antenna ricevente sia legata alle variazioni del vettore di magnetizzazione durante il ritorno all'equilibrio termodinamico.

Secondo la legge di Faraday-Neumann-Lenz, la f.e.m. indotta sull'antenna ricevente è:

\[
\text{emf} = - \dfrac{d}{dt} \Phi_S\left(B\right)
\]

dove \(\Phi_S(B)\) è il flusso del campo magnetico concatenato con la superficie \(S\) dell'antenna:

\[
\Phi_S\left(B\right) = \int_S \vec{B} \cdot d\vec{S}
\]

Considerando un volumetto \(V'\) contenente spin, il vettore di magnetizzazione \(\vec{M}\) di tale volumetto si concatena con la spira di area \(S\). Le variazioni di \(\vec{M}\) inducono una f.e.m. sulla spira.

Il campo magnetico è legato al potenziale vettore \(\vec{A}\) dalla relazione:

\[
\vec{B} = \vec{\nabla} \times \vec{A}
\]

Da cui segue:

\[
\text{emf} = - \dfrac{d}{dt} \Phi_S\left(B\right) =- \dfrac{d}{dt} \int_S \vec{B} \cdot d\vec{S} = - \dfrac{d}{dt} \int_S \vec{\nabla} \times \vec{A} \cdot d\vec{S}
\]

Si dimostra che il potenziale vettore in un punto \(\vec{r}\), esterno alla regione contenente le sorgenti, è dato da:

\[
\vec{A}(\vec{r}) = \dfrac{\mu_0}{4\pi} \int_{V'} \dfrac{1}{|\vec{r} - \vec{r}'|} \vec{J}(\vec{r}') \, dV'
\]

dove \(\vec{r}'\) è un punto nel volumetto elementare e \(\vec{J}(\vec{r}')\) è la densità di corrente.

\begin{figure}[ht]
\centering
\includegraphics[width=3.32351in,height=3.05in,alt={P3589\#yIS1}]{media/7_MRISignal/image82.pdf}\caption{Distanze tra volumetto elementare e antenna}
\end{figure}

Il sistema di riferimento è scelto in modo tale che in \({\vec{r}}'\) vi sia una certa densità di corrente vincolata \(\vec{J}\left( {\vec{r}}' \right)\), sorgente del campo infinitesimo \(d\vec{B}\), indotto nella posizione \(\vec{r}\) sull'antenna.

L'integrale che permette di calcolare il potenziale vettore è valutato su \({\vec{r}}'\) ed è risolvibile note le sorgenti \(\vec{J}\), ovvero la densità di corrente nel volumetto. La quantità \(\left| \vec{r} - {\vec{r}}' \right|\) rappresenta la distanza tra il punto di osservazione \(\vec{r}\) rispetto i punti del volumetti \({\vec{r}}'\), soggetti alla densità di corrente \(\vec{J}\).

È noto che, per effetto dei campi magnetici applicati, si generano delle densità di correnti vincolate date dalla relazione:

\[
\vec{J}_{\text{vinc}}(\vec{r}') = \vec{\nabla}' \times \vec{M}(\vec{r}')
\]

dove la notazione \({\vec{\nabla}}' \times\) indica che il rotore è valutato rispetto la coordinata \({\vec{r}}'\).

All'espressione della f.e.m. indotta si applica il teorema di Stokes, il quale afferma che il flusso del rotore può essere scritto come la circuitazione del potenziale vettore \(\vec{A}\) sulla linea che rappresenta il contorno \(\partial S\), della superficie \(S\) della spira:

\[
\text{emf}  = - \dfrac{d}{dt}\int_{S}{\vec{\nabla} \times \vec{A} \cdot d\vec{S}} = - \dfrac{d}{dt}\oint_{l}{\vec{A} \cdot d\vec{l}}
\]

dove  \(l = \partial S\). Si sostituisce l'espressione per il potenziale vettore in funzione delle densità di correnti:

\[
\text{emf} = - \dfrac{d}{dt}\oint_{l}{\vec{A} \cdot d\vec{l}} = - \dfrac{d}{dt}\oint_{l}{\left\lbrack \dfrac{\mu_{0}}{4\pi}\int_{V'}{\dfrac{1}{\left| \vec{r} - {\vec{r}}' \right|}\vec{J}\left( {\vec{r}}' \right)dV'} \right\rbrack \cdot d\vec{l}}
\]

Poichè le densità di corrente nel volumetto sono di tipo vincolato, \(\vec{J} = \vec{\nabla}' \times \vec{M}\), l'espressione per la f.e.m. si può scrive come:

\[
\text{emf} = - \dfrac{\mu_{0}}{4\pi}\dfrac{d}{dt}\oint_{l}{\left\lbrack \int_{V'}{\dfrac{1}{\left| \vec{r} - {\vec{r}}' \right|}\vec{J}\left( {\vec{r}}' \right)dV'} \right\rbrack \cdot d\vec{l}} = - \dfrac{\mu_{0}}{4\pi}\dfrac{d}{dt}\oint_{l}{\left\lbrack \int_{V'}{\dfrac{{\vec{\nabla}}' \times \vec{M}\left( {\vec{r}}' \right)}{\left| \vec{r} - {\vec{r}}' \right|}dV'} \right\rbrack \cdot d\vec{l}}
\]

Applicando l'identità vettoriale:

\[
\vec{\nabla} \times \left\lbrack f\left( \vec{r} \right)\vec{a}\left( \vec{r} \right) \right\rbrack = \nabla f\left( \vec{r} \right) \times \vec{a}\left( \vec{r} \right) + f\left( \vec{r} \right)\nabla \times \vec{a}\left( \vec{r} \right)
\]

dove \(f\left( \vec{r} \right)\) è una qualsiasi funzione scalare e \(\vec{a}\left( \vec{r} \right)\) una funzione vettoriale. Si ricava \(f\left( \vec{r} \right)\nabla \times \vec{a}\left( \vec{r} \right)\)

\[
\dfrac{1}{|\vec{r} - \vec{r}'|} \vec{\nabla}' \times \vec{M}(\vec{r}') = \vec{\nabla}' \times \left[ \dfrac{1}{|\vec{r} - \vec{r}'|} \vec{M}(\vec{r}') \right] - \vec{\nabla}' \left( \dfrac{1}{|\vec{r} - \vec{r}'|} \right) \times \vec{M}(\vec{r}')
\]


Ponendo \(f(\vec{r}) = \left|\vec{r} - \vec{r}'\right|^{-1}\) e \(\vec{a}\left(\vec{r}\right) = \vec{M}(\vec{r}')\),  l'identità può essere scritta come:

\[
\dfrac{1}{\left| \vec{r} - {\vec{r}}' \right|}\vec{\nabla} \times \vec{M}\left( {\vec{r}}' \right) = \vec{\nabla} \times \left\lbrack \dfrac{1}{\left| \vec{r} - {\vec{r}}' \right|}\vec{M}\left( {\vec{r}}' \right) \right\rbrack - \vec{\nabla}\left( \dfrac{1}{\left| \vec{r} - {\vec{r}}' \right|} \right) \times \vec{M}\left( {\vec{r}}' \right)
\]

L'espressione per l'\textit{electromotive force} può essere scritta come:

\begin{align}
\text{emf} &= - \dfrac{\mu_{0}}{4\pi}\dfrac{d}{dt}\oint_{l}\left[\int_{V'}\!\left\{{\vec{\nabla}}' \times \left(\dfrac{1}{\left| \vec{r} - {\vec{r}}' \right|}\vec{M}({\vec{r}}')\right)-\vec{\nabla}'\!\left(\dfrac{1}{\left| \vec{r} - {\vec{r}}' \right|}\right)\times \vec{M}({\vec{r}}')\right\} dV' \right]\!\cdot d\vec{l}\nonumber\\
&= - \dfrac{\mu_{0}}{4\pi}\dfrac{d}{dt}\oint_{l}\left[\int_{V'} {\vec{\nabla}}' \times \left(\dfrac{1}{\left| \vec{r} - {\vec{r}}'\right|}\vec{M}({\vec{r}}')\right) dV'- \int_{V'} \vec{\nabla}'\!\left(\dfrac{1}{\left| \vec{r} - {\vec{r}}' \right|}\right)\times \vec{M}({\vec{r}}')\, dV'\right]\!\cdot d\vec{l}
\end{align}

Questa espressione lega la variazione temporale della magnetizzazione alla f.e.m. indotta sull'antenna ricevente, e rappresenta il principio fisico alla base della rilevazione del segnale in risonanza magnetica.

Si considera l'integrale:

\[
\int_{V'} \vec{\nabla}' \times \left[ \dfrac{1}{|\vec{r} - \vec{r}'|} \vec{M}(\vec{r}') \right] dV'
\]

Per un teorema analogo a quello di Stokes, tale integrale può essere trasformato in un integrale di superficie sulla frontiera \(\partial V'\) del volumetto \(V'\):

Si può dimostrare che, per un teorema analogo a quello di Stokes, l'integrale considerato è uguale all'integrale calcolato sulla frontiera del volumetto (\(\partial V'\)) della funzione di cui si applica il rotore, vettor la normale della frontiera. In altre parola, è valida la relazione:

\[
\int_{V'} \vec{\nabla}' \times \left[ \dfrac{1}{|\vec{r} - \vec{r}'|} \vec{M}(\vec{r}') \right] dV' =
\int_{\partial V'} \hat{n} \times \left[ \dfrac{1}{|\vec{r} - \vec{r}'|} \vec{M}(\vec{r}') \right] dS'
\]

Si ottiene così un integrale superficiale. La relazione è sempre valida, quindi, vale anche per il volumetto \(V\), su cui effettuare l'integrale, leggermente più grande del volumetto \(V'\), contenente le sorgenti del campo. All'esterno del volumetto la magnetizzazione è nulla, di conseguenza l'integrale di flusso è nullo, in quanto il vettore di magnetizzazione è nullo sulla superficie.

\begin{figure}[ht]
\centering
\includegraphics[width=1.46875in,height=1.26687in,alt={P3617\#yIS1}]{media/7_MRISignal/image83.pdf}\caption{Volume \(V\) su cui calcolare l'integrale, leggermente più grande di \(V'\) contenente gli spin}
\end{figure}

La forza elettromotrice si riduce a:

\[
\text{emf}  = - \dfrac{\mu_{0}}{4\pi}\dfrac{d}{dt}\oint_{l}{\left\lbrack - \int_{V'}{\vec{\nabla'}\left( \dfrac{1}{\left| \vec{r} - {\vec{r}}' \right|} \right) \times \vec{M}\left( {\vec{r}}' \right)dV'} \right\rbrack \cdot d\vec{l}} = \dfrac{\mu_{0}}{4\pi}\dfrac{d}{dt}\oint_{l}{\int_{V'}{\vec{\nabla'}\left( \dfrac{1}{\left| \vec{r} - {\vec{r}}' \right|} \right) \times \vec{M}\left( {\vec{r}}' \right)dV'} \cdot d\vec{l}}
\]

Per la linearità è possibile invertire gli integrali su \(d\vec{l}\) e \(dV'\):

\[
\text{emf}= \dfrac{\mu_{0}}{4\pi}\dfrac{d}{dt}\int_{V'}{\oint_{l}{\vec{\nabla'}\left( \dfrac{1}{\left| \vec{r} - {\vec{r}}' \right|} \right) \times \vec{M}\left( {\vec{r}}' \right) \cdot d\vec{l}}dV'}
\]

Applicando le identità del prodotto misto e considerando che \(d\vec{l}\) è esterno al volumetto, si ottiene:

Nell'integrale della f.e.m. compare un prodotto misto, per cui valgono le identità:

\[
\vec{\nabla'}\left( \dfrac{1}{\left| \vec{r} - {\vec{r}}' \right|} \right) \times \vec{M}\left( {\vec{r}}' \right) \cdot d\vec{l} = \vec{M}\left( {\vec{r}}' \right) \times d\vec{l} \cdot \vec{\nabla'}\left( \dfrac{1}{\left| \vec{r} - {\vec{r}}' \right|} \right) = d\vec{l} \times \vec{\nabla'}\left( \dfrac{1}{\left| \vec{r} - {\vec{r}}' \right|} \right) \cdot \vec{M}\left( {\vec{r}}' \right)
\]

L'ultimo termine può essere scritto come:

\[
\vec{\nabla'}\left( \dfrac{1}{\left| \vec{r} - {\vec{r}}' \right|} \right) \times \vec{M}\left( {\vec{r}}' \right) \cdot d\vec{l} = - \vec{M}\left( {\vec{r}}' \right) \cdot \vec{\nabla'}\left( \dfrac{1}{\left| \vec{r} - {\vec{r}}' \right|} \right) \times d\vec{l}
\]

Dalla relazione \(\vec{\nabla}f\left( \vec{r} \right) \times \vec{a}\left( \vec{r} \right) = \vec{\nabla} \times \left\lbrack f\left( \vec{r} \right)\vec{a}\left( \vec{r} \right) \right\rbrack - f\left( \vec{r} \right)\vec{\nabla} \times \vec{a}\left( \vec{r} \right)\) è possibile scrivere:

\[
\vec{\nabla'}\left( \dfrac{1}{\left| \vec{r} - {\vec{r}}' \right|} \right) \times d\vec{l} = \vec{\nabla} \times \left( \dfrac{1}{\left| \vec{r} - {\vec{r}}' \right|}d\vec{l} \right) - \dfrac{1}{\left| \vec{r} - {\vec{r}}' \right|}\vec{\nabla} \times \ d\vec{l}
\]

La f.e.m. può essere riformulata come:

\[
\text{emf} = \dfrac{\mu_{0}}{4\pi}\dfrac{d}{dt}\int_{V'}{\oint_{l}\left\lbrack - \vec{M}\left( {\vec{r}}' \right) \cdot {\vec{\nabla}}' \times \left( \dfrac{1}{\left| \vec{r} - {\vec{r}}' \right|}d\vec{l} \right) + \vec{M}\left( {\vec{r}}' \right) \cdot \dfrac{1}{\left| \vec{r} - {\vec{r}}' \right|}{\vec{\nabla}}' \times \ d\vec{l} \right\rbrack}dV'
\]

\(d\vec{l}\) è esterno al volumetto elementare, poiché è il versore che agisce sulla spira, dunque, \({\vec{\nabla}}' \times \ d\vec{l} = \vec{0}\):

\[
\text{emf} = - \dfrac{\mu_{0}}{4\pi}\dfrac{d}{dt}\int_{V'}{\vec{M}\left( {\vec{r}}' \right) \cdot {\vec{\nabla}}' \times \oint_{l}{\dfrac{1}{\left| \vec{r} - {\vec{r}}' \right|}d\vec{l}}}\,dV'
\]

La circuitazione non opera all'interno del volumetto, dunque, su \({\vec{r}}'\) ma solo sulla spira, quindi su \(\vec{r}\). Di conseguenza, è possibile portare \(\vec{M}\left( {\vec{r}}' \right)\) all'esterno del simbolo di circuitazione.

Nell'espressione della f.e.m. compare il termine:

\[
\dfrac{\mu_{0}}{4\pi}\oint_{l}{\dfrac{1}{\left| \vec{r} - {\vec{r}}' \right|}d\vec{l}}
\]

È noto che il potenziale vettore generato da una corrente filiforme \(I\), che scorre nell'elemento infinitesimo di antenna \(d\vec{l}\), è dato da:

\[
\vec{A}\left( {\vec{r}}' \right) = \dfrac{\mu_{0}}{4\pi}\oint_{l}{\dfrac{I}{\left| \vec{r} - {\vec{r}}' \right|}d\vec{l}}
\]

L'integrale nel calcolo della f.e.m. è, quindi, il potenziale vettore nel punto \({\vec{r}}'\) della spira, usata come antenna quando in essa scorre una corrente unitaria. L'espressione della f.e.m. si scrive come:

\[
\text{emf} = - \dfrac{\mu_{0}}{4\pi}\dfrac{d}{dt}\int_{V'}{\vec{M}\left( {\vec{r}}' \right) \cdot {\vec{\nabla}}' \times \oint_{l}{\dfrac{1}{\left| \vec{r} - {\vec{r}}' \right|}d\vec{l}}dV'} = - \dfrac{\mu_{0}}{4\pi}\dfrac{d}{dt}\int_{V'}{\vec{M}\left( {\vec{r}}' \right) \cdot {\vec{\nabla}}' \times \vec{A}\left( {\vec{r}}' \right)dV'}
\]

Il campo \({\vec{B}}_{ric}\left( {\vec{r}}' \right) = {\vec{\nabla}}' \times \vec{A}\left( {\vec{r}}' \right)\) è il campo magnetico che sarebbe prodotto dalla spira ricevente in \({\vec{r}}'\) se in essa scorresse una corrente unitaria. Il campo \({\vec{B}}_{\text{ric}}\) è detto ricevente e l'equazione per la f.e.m. si scrive come:

\[
\text{emf} = - \dfrac{\mu_0}{4\pi} \dfrac{d}{dt} \int_{V'} \vec{M}\left(\vec{r}'\right) \cdot \vec{B}_{\text{ric}}\left(\vec{r}'\right) \, dV'
\]

Questa relazione lega la f.e.m. indotta alla variazione temporale del prodotto scalare tra la magnetizzazione \(\vec{M}(\vec{r}')\) e il campo ricevente \(\vec{B}_{\text{ric}}(\vec{r}')\), ovvero il campo che sarebbe generato dalla spira ricevente in \(\vec{r}'\) se in essa scorresse una corrente unitaria.

Il progettista del sistema di risonanza magnetica realizza l'antenna ricevente in modo da generare un campo noto in ogni punto \(\vec{r}'\). Pertanto, misurando la f.e.m. indotta, l'unica incognita nella relazione è la magnetizzazione \(\vec{M}(\vec{r}')\), che può così essere ricostruita.

Nell'equazione individuata per la f.e.m.:

\[
\text{emf} = - \dfrac{\mu_{0}}{4\pi}\dfrac{d}{dt}\int_{V'}{\vec{M}\left( {\vec{r}}' \right) \cdot {\vec{B}}_{ric}\left( {\vec{r}}' \right)dV'}
\]

assumendo che il volume \(V'\) non vari nel tempo, è possibile scambiare derivata e integrale:


\[
\text{emf} = - \dfrac{\mu_{0}}{4\pi} \int_{V'} \dfrac{\partial \vec{M}(\vec{r}', t)}{\partial t} \cdot \vec{B}_{\text{ric}}(\vec{r}') \, dV'
\]

Il campo \(\vec{B}_{\text{ric}}\) è fisso e non dipende dal tempo.

Durante l'\textit{}{imaging}, un impulso a radiofrequenza ribalta la magnetizzazione. Nel successivo ritorno all'equilibrio, la variazione temporale di \(\vec{M}\) induce una f.e.m. sulla spira ricevente, proporzionale alla velocità di variazione della magnetizzazione.

Esplicitando il prodotto scalare tra magnetizzazione e campo ricevente si ottiene:

\[
\text{emf} = - \dfrac{\mu_{0}}{4\pi} \int_{V'} \left( \dfrac{\partial M_x}{\partial t} B_{\text{ric},x} + \dfrac{\partial M_y}{\partial t} B_{\text{ric},y} + \dfrac{\partial M_z}{\partial t} B_{\text{ric},z} \right) dV'
\]

Le componenti trasversali \(M_x\) e \(M_y\) evolvono nel sistema fisso con andamento oscillatorio smorzato. Le loro derivate temporali, a meno di costanti di fase, sono del tipo:

\[
\dfrac{\partial M_y}{\partial t}\propto\omega_{0}M_{0}\exp\left( j\omega_{0}t \right)
\]

La componente longitudinale \(M_{z}\) evolve esponenzialmente con costante di tempo \(T_{1}\), quindi la derivata di tale componente rispetto al tempo sarà del tipo:

\[
\dfrac{\partial M_z}{\partial t}\propto\dfrac{1}{T_{1}}\exp\left( - \dfrac{t}{T_{1}} \right)
\]

Dato che \(\omega_{0}\) è dell'ordine di \(64 \cdot 10^{6}\ rad/s\) mentre il tempo di rilassamento \textit{spin-lattice} è di \(1\ s\), risulta che:

\[
\omega_{0} \gg \dfrac{1}{T_{1}}
\]

il contributo delle componenti trasversali alla f.e.m. e è molto maggiore del contributo longitudinale, che può essere trascurato. In altre parole, la f.e.m. è proporzionale approssimativamente solo alle derivate delle componenti trasverse:

\[
\text{emf} \propto \dfrac{\partial M_x}{\partial t} B_{\text{ric},x} + \dfrac{\partial M_y}{\partial t} B_{\text{ric},y}
\]

trascurando il contributo della componente longitudinale.

In conclusione, il segnale ricevuto dall'antenna, indotto dalle variazioni della magnetizzazione macroscopica di un volumetto elementare del paziente, è proporzionale a:

\[
\text{emf} \propto \int_{V'} \left( \dfrac{\partial M_x}{\partial t} B_{\text{ric},x} + \dfrac{\partial M_y}{\partial t} B_{\text{ric},y} \right) dV'
\]

dove \(\vec{B}_{\text{ric}}(\vec{r}')\) è il campo magnetico che sarebbe generato dalla spira ricevente in \(\vec{r}'\) se in essa scorresse una corrente continua unitaria. Questo campo è noto e costante nel tempo. Il volumetto \(V'\) deve essere sufficientemente grande da contenere un numero statisticamente significativo di spin, ma anche sufficientemente piccolo da garantire una buona risoluzione spaziale dell'immagine.

\begin{figure}[ht]
\centering
\includegraphics[width=3.21667in,height=2.45459in,alt={P3664\#yIS1}]{media/7_MRISignal/image84.pdf}\caption{Rappresentazione del vettore di magnetizzazione e del campo ricevuto}
\end{figure}


L'evoluzione temporale nel piano trasversale al campo principale del vettore di magnetizzazione può essere studiata mediante il fasore \(M_+\), definito come:

\[
M_+\left(\vec{r}, t\right) = M_x\left(\vec{r}, t\right) + j M_y\left(\vec{r}, t\right)
\]

Da cui:

\[
M_x(\vec{r}, t) = \Re\{M_+(\vec{r}, t)\}, \quad M_y(\vec{r}, t) = \Im\{M_+(\vec{r}, t)\}
\]

Ne risulta che \(M_{x}\left( \vec{r},t \right) = Re\left\{ M_{+}\left( \vec{r},t \right) \right\}\) e \(M_{x}\left( \vec{r},t \right) = Im\left\{ M_{+}\left( \vec{r},t \right) \right\}\). Inoltre, si è visto che il fasore \(M_{+}\) può essere scritto come:


\[
M_{+}\left( \vec{r},t \right) = M_{+}\left( \vec{r},\ 0 \right)\exp\left( - j\omega_{0}t \right)\exp\left( - \dfrac{t}{T_{2}} \right) = \left| M_{+}\left( \vec{r},\ 0 \right) \right|\exp\left\lbrack j\phi_{0}\left( \vec{r} \right) \right\rbrack\exp\left( - j\omega_{0}t \right)\exp\left( - \dfrac{t}{T_{2}} \right)
\]

dove:
\begin{itemize}
    \item \(\left|M_+(\vec{r}, 0)\right|\) è l'ampiezza iniziale,
    \item \(\phi_0\left(\vec{r}\right)\) è la fase iniziale,
    \item \(\omega_0\left(\vec{r}\right)\) è la frequenza di Larmor locale,
    \item \(T_2\left(\vec{r}\right)\) è il tempo di rilassamento trasversale.
\end{itemize}

Il tempo di rilassamento \(T_{2}\) dipende dal particolare volumetto considerato, poiché tra i diversi volumetti cambiano le proprietà chimico-fisiche dei tessuti. Analogo discorso vale per la fase iniziale, infatti, ogni vettore di magnetizzazione presenta una fase iniziale nel piano \(x - y\) diversa in base alla posizione del volumetto. Le quantità \(\left| M_{+}\left( \vec{r},\ 0 \right) \right|\) e \(\phi_{0}\left( \vec{r},\ 0 \right)\) dipendono dalle condizioni iniziali del volumetto.

Siccome \(\omega_{0} \gg T_{1}^{- 1}\) è possibile trascurare la componente longitudinale nella valutazione della f.e.m. Si suppone di applicare una perturbazione a opera di un campo magnetico a radiofrequenza. Il vettore di magnetizzazione viene ribaltato lungo il piano trasverso lungo uno degli assi. Si osservi che è possibile avere anche delle perturbazioni che portano la magnetizzazione a giacere sul piano trasverso e non su uno degli assi.

Interrotta la perturbazione, il vettore di magnetizzazione evolve secondo un andamento elicoidale con raggio variabile, descritto delle equazioni di Bloch. In questo contesto non è possibile trascurare i tempi di rilassamento, in quanto le evoluzioni delle componenti del vettore di magnetizzazione sono dettate dai tempi \(T_{1}\) e \(T_{2}\):

\[
\dfrac{d\vec{M}}{dt} = \gamma\vec{M} \times {\vec{B}}_{0} + \dfrac{1}{T_{1}}\left( M_{0} - M_{z}\  \right){\hat{i}}_{z} - \dfrac{1}{T_{2}}{\vec{M}}_{\bot}
\]

Il movimento del vettore di magnetizzazione induce sull'antenna un segnale f.e.m. dipendente prevalentemente dalle componenti trasverse del vettore \(\vec{M}\), descritte dal fasore \(M_{+}\left( \vec{r},t \right)\). Questa quantità tiene conto della dipendenza dallo spazio, dal tempo di rilassamento trasversale, dalla fase e dalla frequenza di Larmor alla quale risuonano gli spin nel volumetto elementare. La frequenza di Larmor dei diversi volumetti può essere diversa a causa delle disomogeneità di campo o dei gradienti di campo applicato.

Trascurando l'evoluzione longitudinale, la f.e.m. indotta è proporzionale a:

\[
\text{emf} \propto \int_{V'}{\left( \dfrac{\partial M_{x}}{\partial t}\vec{B}_{\text{ric}} + \dfrac{\partial M_{y}}{\partial t}B_{ric,y} + \dfrac{\partial M_{z}}{\partial t}B_{ric,z} \right)dV'} \simeq \int_{V'}{\left( \dfrac{\partial M_{x}}{\partial t}\vec{B}_{\text{ric}} + \dfrac{\partial M_{y}}{\partial t}B_{ric,y} \right)dV'}
\]

Portando il simbolo di derivata all'esterno dell'integrale, nell'ipotesi che il volumetto elementare non vari nel tempo, è possibile scrivere:

\[
text{emf} \propto \dfrac{d}{dt}\int_{V'}{\left( M_{x}\vec{B}_{\text{ric}} + M_{y}B_{ric,y} \right)dV'}
\]

La funzione integranda può essere espressa in espressa in termini del fasore:

\[
M_{x}\vec{B}_{\text{ric}} + M_{y}B_{ric,y} = {\vec{M}}_{\bot} \cdot {\vec{B}}_{ric} = \left(\Re\{M_+(\vec{r}, t)\} B_{\text{ric},x}{\hat{i}}_{x} + \Im\{M_+(\vec{r}, t)\} B_{\text{ric},y}{\hat{i}}_{y} \right) \cdot {\vec{B}}_{ric}\left( \vec{r} \right)
\]

Si svolgono i passaggi matematici per la parte reale:

\[\begin{aligned}
\Re\left\{ M_{+}(\vec{r}, t) \right\} 
&= \text{Re}\left\{ 
\left| M_{+}(\vec{r}, 0) \right| 
\exp\left[ -j\left( \phi_{0}(\vec{r}) - \omega_{0}(\vec{r})t \right) \right] 
\exp\left( -\dfrac{t}{T_{2}(\vec{r})} \right) 
\right\} \\
&= \left| M_{+}(\vec{r}, 0) \right| 
\exp\left( -\dfrac{t}{T_{2}(\vec{r})} \right) 
\cos\left[ \phi_{0}(\vec{r}) - \omega_{0}(\vec{r})t \right] \\
&= \left| M_{+}(\vec{r}, 0) \right| 
\exp\left( -\dfrac{t}{T_{2}(\vec{r})} \right) 
\cos\left[ \omega_{0}(\vec{r})t - \phi_{0}(\vec{r}) \right]
\end{aligned}\]

Analogamente, per la parte immaginaria

\[\begin{aligned}
\Im\left\{ M_{+}(\vec{r}, t) \right\} 
&= \text{Im}\left\{ 
\left| M_{+}(\vec{r}, 0) \right| 
\exp\left[ -j\left( \phi_{0}(\vec{r}) - \omega_{0}(\vec{r})t \right) \right] 
\exp\left( -\dfrac{t}{T_{2}(\vec{r})} \right) 
\right\} \\
&= \left| M_{+}(\vec{r}, 0) \right| 
\exp\left( -\dfrac{t}{T_{2}(\vec{r})} \right) 
\sin\left[ \phi_{0}(\vec{r}) - \omega_{0}(\vec{r})t \right] \\
&= -\left| M_{+}(\vec{r}, 0) \right| 
\exp\left( -\dfrac{t}{T_{2}(\vec{r})} \right) 
\sin\left[ \omega_{0}(\vec{r})t - \phi_{0}(\vec{r}) \right]
\end{aligned}\]

Per cui la funzione integranda è data da:

\[\begin{aligned}
\vec{M}_{\bot} \cdot \vec{B}_{\text{ric}}(\vec{r}) 
&= \left( \Re\left\{ M_{+}(\vec{r}, t) \right\} \hat{i}_{x} 
+ \Im\left\{ M_{+}(\vec{r}, t) \right\} \hat{i}_{y} \right) 
\cdot \vec{B}_{\text{ric}}(\vec{r}) \\
&= \left| M_{+}(\vec{r}, 0) \right| 
\exp\left( -\dfrac{t}{T_{2}(\vec{r})} \right) 
\left[ B_{\text{ric},x}(\vec{r}) \cos\left( \omega_{0}(\vec{r})t - \phi_{0}(\vec{r}) \right) 
- B_{\text{ric},y}(\vec{r}) \sin\left( \omega_{0}(\vec{r})t - \phi_{0}(\vec{r}) \right) \right]
\end{aligned}\]


La f.e.m. indotta sull'antenna è data dalla derivata della quantità appena individuata:


\[\begin{aligned}
\text{emf}\!\left(t\right) \propto \dfrac{d}{dt} \int_{V'} & \left|M_+\!\left(\vec{r}, 0\right)\right| \exp\!\left(-\dfrac{t}{T_2\!\left(\vec{r}\right)}\right)
\big[B_{\text{ric},x}\!\left(\vec{r}\right)
\cos\!\left(\omega_0\!\left(\vec{r}\right)t - \phi_0\!\left(\vec{r}\right)\right) + \\
&-B_{\text{ric},y}\!\left(\vec{r}\right) \sin\!\left(\omega_0\!\left(\vec{r}\right)t -\phi_0\!\left(\vec{r}\right)\right)\big] dV'
\end{aligned}\]

In ipotesi di volumetto costante nel tempo, è possibile scrivere:


\[\begin{aligned}
\text{emf}\left( t \right) \propto \int_{V'} \dfrac{\partial}{\partial t} \Bigg\{ 
& \left| M_{+}\left(\vec{r}, 0\right) \right| \exp\left( -\dfrac{t}{T_{2}\left(\vec{r}\right)} \right) \ \left[ \vec{B}_{\text{ric}}\left(\vec{r}\right) \cos\left( \omega_{0}\left(\vec{r}\right) t - \phi_{0}\left(\vec{r}\right) \right) \right. + \\
& \left.- B_{\text{ric},y}\left(\vec{r}\right) \sin\left( \omega_{0}\left(\vec{r}\right) t - \phi_{0}\left(\vec{r}\right) \right)\right]\Bigg\} \, dV' =
\end{aligned}\]

Svolgendo la derivata temporale, si ha:

\[\begin{aligned}
= \int_{V'}\! \Bigg\{ 
& - \dfrac{1}{T_{2}\!\left(\vec{r}'\right)} \left| M_{+}\!\left(\vec{r}, 0\right) \right| 
\exp\!\left( - \dfrac{t}{T_{2}\!\left(\vec{r}\right)} \right)
\Big[
B_{\text{ric},x}\!\left(\vec{r}\right) \cos\!\left( \omega_{0}\!\left(\vec{r}\right) t - \phi_{0}\!\left(\vec{r}\right) \right) - B_{\text{ric},y}\!\left(\vec{r}\right) \sin\!\left( \omega_{0}\!\left(\vec{r}\right) t - \phi_{0}\!\left(\vec{r}\right) \right)
\Big] + \\
& - \omega_{0}\!\left(\vec{r}\right) \left| M_{+}\!\left(\vec{r}, 0\right) \right| \exp\!\left( - \dfrac{t}{T_{2}\!\left(\vec{r}\right)} \right)\Big[
B_{\text{ric},x}\!\left(\vec{r}\right) \sin\!\left( \omega_{0}\!\left(\vec{r}\right) t - \phi_{0}\!\left(\vec{r}\right) \right)  +\\
&+  B_{\text{ric},y}\!\left(\vec{r}\right) \cos\!\left( \omega_{0}\!\left(\vec{r}\right) t - \phi_{0}\!\left(\vec{r}\right) \right)\Big]\Bigg\}\, dV' \simeq
\end{aligned}\]

Questa espressione mostra che il segnale ricevuto è una sinusoide smorzata, modulata spazialmente dalla fase iniziale, dalla frequenza di Larmor locale e dal campo ricevente.

Poiché \(\omega_{0} \gg T_{2}^{- 1}\), è possibile trascurare il termine dipendente dal tempo di rilassamento trasversale:

\[\begin{aligned}
\simeq \int_{V'} \Big\{ 
& \omega_{0}\left( \vec{r} \right)
\left| M_{+}\left( \vec{r},\ 0 \right) \right|
\exp\left( - \dfrac{t}{T_{2}\left( \vec{r} \right)} \right)\left[
- \vec{B}_{\text{ric}}\left( \vec{r} \right)
\sin\left( \omega_{0}\left( \vec{r} \right)t - \phi_{0}\left( \vec{r} \right) \right)
\right. \\
& \left.
- B_{\text{ric},y}\left( \vec{r} \right)
\cos\left( \omega_{0}\left( \vec{r} \right)t - \phi_{0}\left( \vec{r} \right) \right)
\right]
\Big\} \, dV' =
\end{aligned}\]

All'interno dell'integrale sono presenti solo quantità dipendenti dalla posizione, come la fase iniziale, la frequenza di larmor, tempo di rilassamento e magnetizzazione iniziale.

Se il campo ricevente è del tipo:

\[
\vec{B}_{\text{ric}}(\vec{r}) =B_\perp(\vec{r}) \cos\left( \vartheta(\vec{r}) \right) \hat{\imath}_x + B_\perp(\vec{r}) \sin\left( \vartheta(\vec{r}) \right) \hat{\imath}_y
\]

Il segnale ricevuto può essere scritto come:

\[\begin{aligned}
= \int_{V'} \Big\{ 
& \omega_{0}\left( \vec{r} \right)
\left| M_{+}\left( \vec{r},\ 0 \right) \right|
B_{\bot}\left( \vec{r} \right)
\exp\left( - \dfrac{t}{T_{2}\left( \vec{r} \right)} \right) \left[
- \cos\left( \vartheta\left( \vec{r} \right) \right)
\sin\left( \omega_{0}\left( \vec{r} \right)t - \phi_{0}\left( \vec{r} \right) \right)
\right. \\
& \left.
- \sin\left( \vartheta\left( \vec{r} \right) \right)
\cos\left( \omega_{0}\left( \vec{r} \right)t - \phi_{0}\left( \vec{r} \right) \right)
\right]
\Big\} \, dV'
\end{aligned}\]


Per le formula di addizione del seno:

\[
\cos \vartheta \sin(\omega_0 t - \phi_0) + \sin \vartheta \cos(\omega_0 t - \phi_0) =
\sin(\omega_0 t - \phi_0 + \vartheta)
\]

Dunque, trascurando il segno meno, si ottiene:


\[
\text{emf} \propto \int_{V'}{\left\{ \omega_{0}\left( \vec{r} \right)\left| M_{+}\left( \vec{r},\ 0 \right) \right|B_{\bot}\left( \vec{r} \right)\exp\left( - \dfrac{t}{T_{2}\left( \vec{r} \right)} \right)\sin\left\lbrack \omega_{0}\left( \vec{r} \right)t - \phi_{0}\left( \vec{r} \right) + \vartheta\left( \vec{r} \right) \right\rbrack \right\} dV'}
\]

Poiché le variazioni spaziali di \(\omega_{0}\!\left( \vec{r} \right)\) sono piccole rispetto al valore medio, è possibile esprimere la frequenza di precessione di Larmor come:

\[
\omega_0\left(\vec{r}\right) = \omega_0 + \Delta \omega_0\left(\vec{r}\right), \quad \text{con} \quad \Delta \omega_0 \ll \omega_0
\]

Le variazioni tipiche della frequenza \(\Delta f = \Delta\omega_{0}/2\pi\), sono molto piccole se confrontate con la quantità \(\gamma B_{0}/2\pi\). Infatti, tale quantità è dell'ordine di qualche \(kHz\), mentre la frequenza di risonanza è dell'ordine di \(60\ MHz\), per cui:

\[\omega_{0} \gg \Delta\omega_{0}\left( \vec{r} \right)\]

Per cui è possibile portare all'esterno dell'integrale \(\omega_{0}\):

\[
\text{emf} \propto \omega_{0}\int_{V'}{\left\{ \left| M_{+}\left( \vec{r},\ 0 \right) \right|B_{\bot}\left( \vec{r} \right)\exp\left( - \dfrac{t}{T_{2}\left( \vec{r} \right)} \right)\sin\left\lbrack \omega_{0}\left( \vec{r} \right)t - \phi_{0}\left( \vec{r} \right) + \vartheta\left( \vec{r} \right) \right\rbrack \right\} dV'}
\]

All'esterno dell'integrale le variazioni della pulsazione di Larmor possono essere trascurate, mentre all'interno della funzione trigonometrica è importante, in quanto tiene conto che la funzione sinusoidale può variare significativamente anche per piccole variazioni di \(\omega_{0}\).

Il segnale misurato dipende essenzialmente da:
\begin{itemize}
    \item il tempo di rilassamento trasversale \(T_2\left(\vec{r}\right)\);
    \item la fase iniziale \(\phi_0\left(\vec{r}\right)\);
    \item la magnetizzazione iniziale \(\left|M_+\left(\vec{r}, 0\right)\right|\);
    \item dal campo ricevente \(B_\perp\left(\vec{r}\right)\), noto e progettato opportunamente
\end{itemize}

La dipendenza spaziale può essere trascurata se la magnetizzazione proviene da un piccolo campione omogeneo, come ad esempio un bicchiere d'acqua. In questo caso, l'integrale è di semplice risoluzione poiché nessuna quantità, come \(T_2\) e \(\omega_0\), dipende dalla posizione spaziale del volumetto elementare \(\vec{r}\). Sia \(V_s\) il volume del campione, il segnale ottenuto è proporzionale a:

\[
\text{emf}\left(t\right) \propto \omega_0 \int_{V'} \left| M_+ \right| B_\perp \exp\left( - \dfrac{t}{T_2} \right) \sin\left( \omega_0 t - \phi_0 + \vartheta \right) dV'
= \omega_0 \left| M_+ \right| V_s B_\perp \exp\left( - \dfrac{t}{T_2} \right) \sin\left( \omega_0 t - \phi_0 + \vartheta \right)
\]


Affinché la frequenza di precessione sia costante, è necessario che il campo principale sia uniforme in tutto il volume del campione. Questa ipotesi è verificata se il campione ha dimensioni ridotte e composizione omogenea.

Le costanti di fase sono estremamente importanti per il confronto tra segnali provenienti da sorgenti di magnetizzazione diverse, in cui è necessario che si verifichino abbattimenti tra le varie fasi dei segnali. Le quantità \(\omega_0\), \(|M_+|\) e \(B_\perp\) sono imposte dall'esterno dall'operatore.


Per un campione omogeneo è semplice ottenere informazioni sulla fase. In particolare, avendo un segnale sinusoidale a frequenza \(\omega_0\) nota, il processo più semplice da utilizzare è la \textit{demodulazione coerente}, ovvero una demodulazione in cui si estraggono le informazioni di fase della sinusoide.

\subsection{Demodulazione del segnale registrato}\label{demodulazione-del-segnale-registrato}

Le rapide oscillazioni alla frequenza \(\omega_0\), contenute nel segnale registrato dalle antenne nella risonanza magnetica, sono rimosse mediante una circuiteria elettronica che si occupa della demodulazione. Questo processo equivale ad analizzare il segnale proveniente dal sistema di riferimento rotante a frequenza di Larmor.

La demodulazione consiste nella moltiplicazione del segnale \(s\left( t \right)\) per un'oscillazione sinusoidale a frequenza di Larmor \(\omega_0\), in fase con il segnale da demodulare. Successivamente, un filtro passa-basso (LPF) estrae le componenti di interesse.


\begin{figure}[ht]
\centering
\includegraphics[width=6.67083in,height=2.07625in,alt={Immagine che contiene nero, oscurità}]{media/7_MRISignal/image85.pdf}\caption{Demodulazione del segnale acquisito dalle antenne}
\end{figure}

Il segnale \(s\left( t \right)\) possiede uno spettro centrato sulla frequenza \(\omega_{0}\). Si ha, quindi:

\[
s\left( t \right) \propto V_s M_\perp B_\perp \sin(\omega_0 t - \phi_0 + \vartheta)
\]

Se il segnale proviene da un piccolo campione omogeneo, lo spettro è composto da due impulsi centrati su \(\pm \omega_0\). 

\begin{figure}[ht]
\centering
\includegraphics[width=3.87452in,height=3.425in]{media/7_MRISignal/image86.pdf}\caption{Spettro di materiale omogeneo}
\end{figure}

In questo caso, la frequenza a cui risuonano gli spin del campione, in generale, può risulta diversa dall'oscillazione che produce il sistema di demodulazione. La frequenza del segnale \(s\left( t \right)\) registrato può essere espressa come:

\[
\omega = \omega_{0} + \delta\omega_{0}
\]

dove \(\delta\omega_0\) è la frequenza di offset rispetto a \(\omega_0\). Dal punto di vista dello spettro, gli impulsi non sono centrati a \(\omega_{0}\), ma piuttosto a \(\omega\)

Trascurando la dipendenza da \(\exp\left( - t/T_{2} \right)\), il segnale registrato \(s\left( t \right)\) dipende solo dalla sinusoide:


\[
s\left( t \right) \propto \sin\left( \omega_{0}t + \delta\omega_{0}t + \vartheta - \phi_{0} \right)
\]

In uscita al moltiplicatore, si ottiene un segnale proporzionale a:

\[
s\left( t \right)\sin\left( \omega_{0}t \right) \propto \sin\left( \omega_{0}t + \delta\omega_{0}t + \vartheta - \phi_{0} \right)\sin\left( \omega_{0}t \right)
\]

Per le formule di prostaferesi, il segnale in uscita è proporzionale, a meno di un fattore \(1/2\), a:

\[
s\left( t \right)\sin\left( \omega_{0}t \right) \propto \cos\left( 2\omega_{0}t + \delta\omega_{0}t + \vartheta - \phi_{0} \right) - \cos\left( \delta\omega_{0}t + \vartheta - \phi_{0} \right)
\]

Ovvero il segnale in uscita dal moltiplicatore è dato dalla somma di un termine a frequenza \(2\omega_{0} + \delta\omega_{0}\) e un termine a frequenza \(\delta\omega_{0}\), differenza tra la frequenza di risonanza di Larmor e dell'oscillatore.

Generalmente, risulta:

\[
2f_{0} + \delta f_{0} \gg \ \delta f_{0}
\]

Infatti, \(f_{0} \simeq 64\ MHz\), mentre \(\delta f_{0}\) è dell'ordine di qualche \(kHz\).

È possibile eseguire un processo di filtraggio, attraverso il quale il termine a frequenza \(2f_{0} + \delta f_{0}\) è completamente rimosso, mentre il termine a bassa frequenza è lasciato inalterato.

\begin{figure}[ht]
\centering
\includegraphics[width=4.225in,height=1.9202in]{media/7_MRISignal/image87.pdf}\caption{Filtraggio a valle del moltiplicatore}
\end{figure}

Per la presenza del tempo di rilassamento \(T_{2}\), lo spettro non è più di tipo impulsivo ma è slargato e centrato sulle frequenze \(\delta f_{0}\) e \(2f_{0} + \delta f_{0}\). Mediante il processo di filtraggio, si conserva solamente il termine in bassa frequenza, centrato su \(\delta f_{0}\).

Nel dominio del tempo, il segnale registrato è una sinusoide ad alta frequenza modulata in ampiezza da \(\exp(-t/T_2)\). Dopo la demodulazione, la frequenza risultante è \(\delta f_0\), con ampiezza modulata da \(T_2\).

\begin{figure}[ht]
\centering
\includegraphics[width=6.69306in,height=3.69861in]{media/7_MRISignal/image88.pdf}\caption{Segnale registrato dalle antenne}
\end{figure}

Dopo la demodulazione del segnale registrato, la frequenza della sinusoide modulata da \(\exp\left( - t/T_{2} \right)\) è data dalla bassa frequenza \(\delta f_{0}\), sovrapposta a un'eventuale differenza tra oscillatore locale e segnale registrato.

\begin{figure}[ht]
\centering
\includegraphics[width=6.69306in,height=3.69861in,alt={Immagine che contiene diagramma, linea, Diagramma Descrizione generata automaticamente}]{media/7_MRISignal/image89.pdf}\caption{Segnale a valle della modulazione}
\end{figure}

La demodulazione viene effettuata su due canali:

\begin{itemize}
\item Il primo canale moltiplica \(s\left( t \right)\) per \(\sin\left(\omega_0 t\right)\), ottenendo la parte reale del fasore \(M_+\).
\item Il secondo canale moltiplica \(s\left( t \right)\) per un'oscillazione in quadratura al primo, ovvero \(\cos\left(\omega_0 t\right)\), ottenendo la parte immaginaria.
\end{itemize}

Questo tipo di modulazione è detta coerente.

\begin{figure}[ht]
\centering
\includegraphics[width=5.79167in,height=3.14174in]{media/7_MRISignal/image90.pdf}\caption{Demodulazione a due canali}
\end{figure}

Il segnale in uscita dal primo canale è proporzionale al \(\cos\left( \delta\omega_{0} + \vartheta - \phi_{0} \right)\), ovvero alla parte reale della quantità \(M_{+} = \exp\left( j\omega_{0}t \right)\exp\left\lbrack j\left( \vartheta - \phi_{0} \right) \right\rbrack\). Questo canale è detto reale e, studiando l'inviluppo del segnale ottenuto a frequenza \(\delta\omega_{0}\), è possibile ottenere informazioni sulla quantità \(T_{2}\), ovvero la costante di tempo con cui decade l'ampiezza dell'oscillazione.

Il secondo canale è detto immaginario ed è ottenuto moltiplicando il segnale registrato dalle antenne per una cosinusoide a frequenza \(\omega_{0}\). Il segnale a valle del moltiplicatore è, quindi, proporzionale a:

\[
s\left( t \right)\cos\left( \omega_{0}t \right) \propto \sin\left( \omega_{0}t + \delta\omega_{0}t + \vartheta - \phi_{0} \right)\cos\left( \omega_{0}t \right)
\]

Per le formule di prostaferesi, a meno di un fattore moltiplicativo \(1/2\), si ha:

\[
s\left( t \right)\cos\left( \omega_{0}t \right) \propto \sin\left( \delta\omega_{0}t + \vartheta - \phi_{0} \right) - \sin\left( 2\omega_{0}t + \delta\omega_{0}t + \vartheta - \phi_{0} \right)
\]

Il segnale in uscita dal canale immaginario è la parte immaginaria della quantità \(M_{+} = \exp\left( j\omega_{0}t \right)\exp\left\lbrack j\left( \vartheta - \phi_{0} \right) \right\rbrack\). Anche qui, il filtro passa-basso elimina il termine ad alta frequenza, lasciando la parte immaginaria del fasore.

La demodulazione a due canali consente di ricostruire l'evoluzione sia della parte immaginaria che della parte immaginaria del segnale \(M_{+}\), ovvero le componenti longitudinali e trasversali del vettore di magnetizzazione \(\vec{M}\).

Poiché il segnale in ingresso ai due canali è lo stesso e poiché i due canali sono posti in parallelo, il rumore sovrapposto al segnale parte reale e parte immaginaria è correlato. In generale, si sfruttano altri schemi che sfruttano delle antenne poste in quadratura, ovvero orientate in maniera ortogonale tra loro, rispetto al campione di cui si vuole eseguire l'\textit{imaging}.

\begin{figure}[ht]
\centering
\includegraphics[width=3.87533in,height=4.13095in]{media/7_MRISignal/image91.pdf}\caption{Antenne in quadratura}
\end{figure}

Il segnale registrato da una delle due antenne è in quadratura col segnale registrato dalla seconda, quindi, è possibile ricavare la parte reale e immaginaria del fasore \(M_{+}\) ma con rumore incorrelato. Con questa soluzione è possibile aumentare il rapporto segnale/rumore.

\section{Acquisizione con sequenza Free Induction Dacay}\label{acquisizione-con-sequenza-free-induction-dacay}

Il segnale ricevuto dall'antenna, usata come detettore, \(s\left( t \right)\), non dipende solamente dai parametri del tessuto, ma anche da come vengono applicati i campi magnetici al corpo in esame, secondo una precisa sequenza.

La sequenza più semplice è l'esperimento di decadimento libero dell'induzione o \textit{Free Induction Decay} (FID), in cui si irradia il campione con un singolo impulso a radiofrequenza.

Si suppone di avere un campione omogeneo contenente un numero di Avogadro di spin. Immediatamente dopo l'impulso a radiofrequenza \(\vec{B}_1\), la magnetizzazione, secondo l'equazione di Bloch, si porta dall'asse \(\hat{x}'\) o \(\hat{y}'\) al valore di equilibrio mediante una traiettoria ellittica.

Il segnale registrato dalle antenne è il decadimento delle componenti trasversali del vettore di magnetizzazione durante il ritorno all'equilibrio, con costante di tempo \(T_2\).

\begin{figure}[ht]
\centering
\includegraphics[width=4.83512in,height=2.48333in]{media/7_MRISignal/image92.pdf}\caption{Sequenza FID nel sistema fisso del laboratorio}
\end{figure}

Nel sistema rotante, il campo \(\vec{B}_1\) appare come un impulso costante lungo \(\hat{x}'\) o \(\hat{y}'\). Dopo l'esaurimento dell'impulso, la magnetizzazione trasversale si riduce in modulo con costante \(T_2\).

\begin{figure}[ht]
\centering
\includegraphics[width=4.125in,height=2.07423in]{media/7_MRISignal/image93.pdf}\caption{Sequenza FID nel sistema rotante}
\end{figure}

Il segnale visto nel sistema rotante equivale al segnale acquisito nel sistema fisso dopo la demodulazione. Se la velocità di rotazione del sistema rotante \(\omega\) è diversa dalla frequenza di Larmor \(\omega_0\), nel sistema rotante si osserva un'oscillazione a frequenza \(\delta\omega = \omega_0 - \omega\).

\begin{figure}[ht]
\centering
\includegraphics[width=4.02924in,height=2.45833in]{media/7_MRISignal/image94.pdf}\caption{Sequenza FID nel sistema rotante con \(\delta\omega = \omega_{0} - \omega \neq 0\)}
\end{figure}

Si vuole studiare il comportamento della fase del segnale registrato. Il segnale di tensione indotto sull'antenna ricevente, in ipotesi di campione omogeneo, è proporzionale a:


\[
s\left( t \right) \propto \omega_{0}\exp\left( - \dfrac{t}{T_{2}} \right)\exp\left( - j\left( \left( \omega_{0} - \omega \right)t + \phi_{0} - \vartheta_{B} \right) \right)\int_{V}{{\vec{B}}_{\bot}\left( \vec{r} \right) \cdot {\vec{M}}_{\bot}\left( \vec{r} \right)dV}
\]

Per un campione omogeneo, è possibile trascurare il prodotto scalare tra le componenti traverse del vettore di magnetizzazione e del campo prodotto dall'antenna ricevente nel punto \(\vec{r} \in V\) se fosse percorsa da una corrente unitaria, in quanto si vuole studiare l'evoluzione della fase del segnale registrato. In altre parole, solamente l'argomento dell'esponenziale complesso è di interesse.

La fase totale accumulata è data dalla fase della magnetizzazione \( \phi_{0}\left( \vec{r}\right) \) a cui si sottrae la fase del campo ricevente \(\vartheta_{B}\left( \vec{r}\right) \). Quest'ultimo denota l'orientamento spaziale del campo ricevente nel piano trasversale. L'iterazione fisica fa sì che questi angoli si combinino nella fase netta, indicata con abuso di notazione \(\phi_{0}\) secondo la relazione:

\[
\phi_{0} = \phi_{0} - \vartheta_{B}
\]

Il termine \(\vartheta_{B}\) è la \textit{field angle}, ovvero l'angolo oltre il quale l'intensità della sorgente si riduce del \(10\%\).

Con questa posizione la fase del segnale registrato è:

\[
\phi\left( t \right) = \left( \omega_{0} - \omega \right)t + \phi_{0}
\]

Dopo la demodulazione, le componenti ad alta frequenza, maggiori di \(\delta\omega_{0}\), sono eliminate, e la fase diventa:


\[
\phi\left( t \right) = \delta\omega_{0}t + \phi_{0}
\]

Si osservi che l'attenuazione del segnale avviene con constante di tempo \(T_{2}\), la quale varia anche in base alle disomogeneità di campo magnetico. Il tempo \(T_2\) è legato allo sfasamento degli spin nel piano trasversale, causato da interazioni spin-spin e disomogeneità del campo magnetico.

Il campo magnetico principale non è omogeneo in tutto lo spazio occupato dal paziente, ma presenta una certa variabilità dell'ordine di grandezza di \(1\ ppm\). Il campo magnetico avvertito da uno spin può, quindi, essere espresso come somma del campo magnetico principale e di un certo termine \(\Delta B\) legato alle disomogeneità:

\[
B\left( \vec{r} \right) = B_{0} + {\Delta}B\left( \vec{r} \right)
\]

Le differenze locali dei campi magnetici sono percepite dai vari spin e si traducono in differenti frequenze di precessione. Dati due spin, che precedono rispettivamente alle frequenze \(\omega_{1}\) e \(\omega_{2}\), le frequenze con cui gli spin precedono possono essere espresse come:

\[
\omega_1 = \omega_0 + \Delta\omega_1, \quad \omega_2 = \omega_0 + \Delta\omega_2
\]

Dunque, oltre alle iterazioni spin-spin, si ha un ulteriore effetto legato ai limiti costruttivi dei meccanismi di generazione del campo. All'interno del volumetto elementare di circa \(1\ mm\) di lato e contente un numero di Avogadro di protoni, gli effetti elle impurità del campo si traducono in una riduzione significativa del tempo di rilassamento traversale, poiché le fasi si distribuiscono ancor più casualmente. Mentre il tempo di rilassamento traversale \(T_{2}\) è legato alle interazioni spin-spin, il tempo di rilassamento \(T_{2}'\) è dovuto alle disomogeneità del campo principale.

Il tempo complessivo con cui la componente trasversale \({\vec{M}}_{\bot}\) si porta a zero è indicato con \(T_{2}^{*}\) ed è legato ai due tempi \(T_{2}\) e \(T_{2}'\) dalla relazione:

\[
\dfrac{1}{T_{2}^{*}} = \dfrac{1}{T_{2}} + \dfrac{1}{T_{2}'}
\]

La fase del segnale registrato dalle antenne in ricezione è:

\[
\phi\left( t \right) = \left( \omega_{0} - \omega \right)t + \phi_{0}
\]


dove \(\omega = \gamma B_{0} + \gamma \Delta B\) e $\Delta B=B_{\text{loc}} - B_{0}$. In definitiva, la fase del segnale è proporzionale a \(\gamma \Delta Bt\):

\[
\phi\left( t \right) \propto \gamma \Delta Bt\]

Il termine \(\phi\left( t \right)\) tende rapidamente a zero, quando si considera un volumetto elementare del paziente. Ciò determina un decadimento molto più rapido del segnale registrato dalle antenne.

\begin{figure}[ht]
\centering
\includegraphics[width=4.85833in,height=2.57017in]{media/7_MRISignal/image95.pdf}\caption{Andamento dello sfasamento al variare del tempo di rilassamento}
\end{figure}

L'esperimento FID consente di effettuare lo \emph{shimming} del campo magnetico, ovvero la compensazione delle disomogeneità tramite regolazioni meccaniche ed elettriche (bobine di shimming).



L'esperimento FID permette di effettuare lo \textit{shimming} del capo magnetico, ovvero la compensazione delle disomogeneità del campo principale tramite una serie di operazioni meccaniche ed elettriche, come la regolazione della corrente di alimentazione di apposite bobine dette di \textit{shimming}. Questa regolazione è fondamentale per garantire precisione nell'\textit{imaging} e nella localizzazione dei volumetti, che dipende dai gradienti di campo. L'esperimento FID è effettuato a monte di altri al fine di omogeneizzare il campo principale.

Un campo principale non omogeneo porta a una serie di problematiche nella valutazione dei tempi di rilassamento, ciò determina una forte incertezza durante l'\textit{imaging} sulla localizzazione dei volumetti. Il posizionamento dei volumetti è legato ai gradienti di campo.

\subsubsection[Origine del tempo T2*]{Origine del tempo $\mathbf{T}_{\mathbf{2}}^{\mathbf{*}}$}
\label{origine-del-tempo-mathbft_mathbf2mathbf}

Si vuole determinare l'origine della relazione tra il tempo di rilassamento trasversale \(T_2\) e il tempo \(T_2^*\), legato alle disomogeneità del campo magnetico.

Il segnale captato dalle antenne è del tipo:

\[
s\left( t \right) \propto \omega_0 \int_V \exp\left( - \dfrac{t}{T_2(\vec{r})} \right) \vec{B}_\perp(\vec{r}) \cdot \vec{M}_\perp(\vec{r}) \sin\left( \omega_0(\vec{r}) t \right) dV
\]

In generale, per un materiale qualsiasi, sia il tempo di rilassamento traversale\(T_2\) che la frequenza di precessione di Larmor \(\omega_0\) dipendono dalla posizione \(\vec{r}\) del volumetto elementare nel corpo, anche a causa delle disomogeneità del campo \(\Delta \vec{B}\).

Il processo di demodulazione coerente avviene a frequenza \(\omega\), generata dalla strumentazione e corrispondente alla velocità angolare del sistema rotante.

\begin{figure}[ht]
\centering
\includegraphics[width=4.14306in,height=3.19048in]{media/7_MRISignal/image96.pdf}\caption{Demodulazione coerente a frequenza \(\omega\)}
\end{figure}

Il segnale demodulato può essere espresso in forma complessa:

\[
s\left( t \right) = \Re\left\{ s\left( t \right) \right\} + j\Im\left\{ s\left( t \right) \right\}
\]

Si pone \(s_{R}\left( t \right) = \Re\left\{ s\left( t \right) \right\}\) e \(s_{I}\left( t \right) = \Im\left\{ s\left( t \right) \right\}\), per cui:

\[
s\left( t \right) = s_{R}\left( t \right) + js_{I}\left( t \right)
\]

Si è visto che, a valle del processo di demodulazione, il segnale registrato è dato da:

\[
s\left( t \right) \propto \omega_{0}\int_{V}{\exp\left( - \dfrac{t}{T_{2}\left( \vec{r} \right)} \right){\vec{B}}_{\bot}\left( \vec{r} \right) \cdot {\vec{M}}_{\bot}\left( \vec{r},0 \right)\exp\left( - j\left( \left( \omega - \omega_{0} \right)t + \phi_{0}\left( \vec{r} \right) \right) \right) dV}
\]

Il segnale \(s\left( t \right)\) è proporzionale a un'oscillazione complessa \(\exp(j\Delta\omega t)\). Si suppone che la fase iniziale dell'oscillazione sia nulla, ovvero \(\phi_{0}\left( \vec{r} \right) = 0\). Si considera, inoltre, un campione di materiale omogeneo come un bicchiere d'acqua. In questo caso, il tempo di rilassamento trasversale \(T_{2}\) non dipende dalla posizione. Si ritiene, infine, che l'antenna sia ideale, ovvero che il campo irradiato non dipende dalla posizione \(\vec{r}\). Il segnale registrato, sotto queste ipotesi, è proporzionala a:

\[
s\left( t \right) \propto \omega_{0}\exp\left( - \dfrac{t}{T_{2}} \right)B_{\bot}\int_{V}{M_{\bot}\left( \vec{r},0 \right)\exp(j{\Delta}\omega t)dV}
\]

Le disomogeneità del campo introducono variazioni nella frequenza di precessione , dell'ordine di decine di kHz. La distribuzione di \(\Delta \omega\) può essere modellata come gaussiana o lorentziana.


Le disomogeneità di campo principale \(\Delta B\) introducono delle variazioni della frequenza di precessione dei vari isocromi \(\Delta \omega\), dell'ordine della decina di \(kHz\). La distribuzione della variazione della differenze tra le frequenze di oscillazione del sistema e di precessione degli spin può essere modellata come una distribuzione gaussiana o lorentziana.

La distribuzione guassiana che meglio descrive le variazioni di frequenza \({\Delta}\omega\) presenta una media nulla e una varianza dipendente dalla disomogeneità del campo magnetico. Gli spin che presentano una differenza di frequenza positiva, ovvero \(\Delta\omega > 0 \Leftrightarrow \omega_{0} - \omega > 0\), presentano una velocità maggiore rispetto al sistema di riferimento; mentre quelli per cui la differenza di frequenza è negativa, ovvero \(\Delta\omega < 0 \Leftrightarrow \omega_{0} - \omega < 0\), presentano una velocità inferiore dell'oscillatore.

\begin{figure}[ht]
\centering
\includegraphics[width=3.22727in,height=2.60327in]{media/7_MRISignal/image97.pdf}\caption{Distribuzione gaussiana}
\end{figure}

Un'altra possibile distribuzione, che può essere attribuita alla variazione della frequenza di precessione, è la lorentziana, descritta dalla \(pdf\):

\[
p(\Delta\omega) = \dfrac{2T_{2}'}{1 + (2\pi{\Delta}f)^{2}{T_{2}'}^{2}}
\]

dove \(T_{2}'\) è un parametro della lorentziana che, in questo caso, descrive le disomogeneità di campo magnetico, sorgenti della distribuzione stessa.

\begin{figure}[ht]
\centering
\includegraphics[width=4.12847in,height=2.5514in]{media/7_MRISignal/image98.pdf}\caption{Distribuzione lorentziana}
\end{figure}

Prove sperimentali hanno dimostrato che la reale distribuzione delle frequenze di processione presenta valori intermedi tra le due distribuzioni citate. Per semplicità, si ritiene che la distribuzione delle frequenze di processione sia di tipo lorentziana.

Il segnale prelevato, a seguito della demodulazione, ricorrendo alla \(pdf\) lorentziana, può essere espresso come:

\[
s\left( t \right) \propto \omega_{0}\exp\left( - \dfrac{t}{T_{2}} \right)B_{\bot}\int_{- \infty}^{+ \infty}{p({\Delta}\omega)\exp(j{\Delta}\omega t)d\omega}
\]

La percentuale degli isocromati è espressa dalla \(pdf\) lorentziana, ovvero:

\[
s\left( t \right) \propto \omega_{0}\exp\left( - \dfrac{t}{T_{2}} \right)B_{\bot}\int_{- \infty}^{+ \infty}{\dfrac{2T_{2}'}{1 + (2\pi{\Delta}f)^{2}{T_{2}'}^{2}}\exp(j2\pi{\Delta}ft)df}
\]

La percentuale degli isocromati è massima per \(\Delta f = 0\), ed è uguale a \(2T_{2}'\); mentre tende a zero per \(\Delta f \rightarrow \pm \infty\).

L'integrale rappresenta l'antitrasformata di Fourier della \(pdf\) della lorentziana. È noto che:

\[
\int_{- \infty}^{+ \infty}{\dfrac{2T_{2}'}{1 + (2\pi{\Delta}f)^{2}{T_{2}'}^{2}}\exp(j2\pi{\Delta}ft)df} = \exp\left( - \dfrac{|t|}{T_{2}'} \right)
\]

Poiché si considerano tempi positivi, \(t > 0\), il segnale registrato è proporzionale a:

\[
s\left( t \right) \propto \omega_{0}\exp\left( - \dfrac{t}{T_{2}} \right)\exp\left( - \dfrac{t}{T_{2}'} \right) = \omega_{0}\exp\left( - t\left( \dfrac{1}{T_{2}} + \dfrac{1}{T_{2}'} \right) \right)
\]

Si definisce:

\[
\dfrac{1}{T_{2}^{*}} = \dfrac{1}{T_{2}} + \dfrac{1}{T_{2}'}
\]

Il tempo \(T_2^*\) dipende dalla distribuzione delle frequenze di precessione degli isocromi, dovuta alle disomogeneità del campo (quantificate da \(T_2'\)) e dalle proprietà intrinseche del materiale (quantificate da \(T_2\)).

L'uso della distribuzione di Lorentz rende l'analisi del tempo di rilassamento \(T_{2}^{*}\) molto semplice dal punto di vista analitico, ma non descrive in modo molto accurato i fenomeni reali. Al contrario, la distribuzione gaussiana rende l'analisi più complessa, poiché non esiste una primitiva della \(pdf\). Nei modelli simulati si preferisce utilizzare la distribuzione gaussiana in quanto di più semplice implementazione.

\section{Sequenza spin-echo}\label{sequenza-spin-echo}

Anche in presenza di opportune compensazioni, il campo magnetico principale presenta una disomogeneità dell'ordine di \(1\ ppm\), ovvero:

\[
\Delta B \simeq 10^{-6} B_0
\]

Se \(B_0 = 1.5\, T\), la disomogeneità è dell'ordine di \(1.5\,\mu T\). Sebbene tale variazione sia molto bassa, non può essere trascurata, poiché è confrontabile con le variazioni locali del campo causate dagli spin.

Il tempo di rilassamento \(T_2^*\) è strettamente legato alle disomogeneità di campo. Infatti, per \(\Delta B = 1 \ ppm\), il tempo \(T_2\) di un tessuto può passare da \(100\ ms\) a \(5 \ ms\) o meno. Inoltre, le disomogeneità possono ridurre fortemente il segnale misurato. Per attenuare questi effetti, sono state proposte sequenze di impulsi, tra cui la sequenza \textit{spin-echo}, composta da due impulsi:

\begin{itemize}
    \item Il primo impulso a radiofrequenza ruota la magnetizzazione sul piano trasverso,  allineandola all'asse lungo cui è diretto l'impulso.  Se, ad esempio, l'asse di rotazione è \(x'\), l'impulso è indicato con \((\pi/2)_{x'}\), in quanto ruota il vettore di magnetizzazione di un angolo pari a \(\pi/2\), procedendo intorno all'asse \(x'\);
    \item Il secondo impulso può essere diretto sia lungo \({\hat{i}}_{x'}\) sia lungo \({\hat{i}}_{y'}\) ed è di tipo \(\pi\), ovvero ruota la magnetizzazione nel piano trasverso di un angolo pari a \(\pi\). Nel primo caso l'impulso è indicato con $\pi_{x'}$ e $\pi_{y'}$ nell'altro.
\end{itemize}

\begin{figure}[ht]
\centering
\includegraphics[width=5.43021in,height=1.3332in,alt={Immagine che contiene linea, diagramma, bianco Descrizione generata automaticamente}]{media/7_MRISignal/image99.pdf}\caption{Sequenza spin-echo}
\end{figure}

Dopo il primo impulso a $\pi/2$, si registra un segnale \(s\left(t\right)\) ad alta frequenza (nella banda delle onde radio), la cui ampiezza decresce con costante di tempo $T_2^*$.  


Nel sistema di riferimento rotante, gli spin del volumetto elementare si sfasano rapidamente a causa delle disomogeneità del campo magnetico principale, e la componente trasversale della magnetizzazione si annulla in un tempo dell’ordine di pochi millisecondi (circa $5\ ms$).

Si applica quindi un secondo impulso di ampiezza $\pi$ lungo l’asse $x'$. Si consideri uno spin generico: esso si trova nel piano trasverso rispetto al campo principale $B_0$ e precessa attorno all’asse $x'$ con frequenza:

\[
\omega = \omega_0 + \Delta \omega,
\]

dove $\omega_0$ è la frequenza di Larmor media e $\Delta \omega$ rappresenta la deviazione dovuta alle disomogeneità di campo.

Nel sistema rotante, gli spin con $\Delta \omega > 0$ (più veloci) precessano in senso antiorario rispetto al sistema (trovandosi a sinistra di $x^{'}$) mentre quelli con $\Delta \omega < 0$ (più lenti) precessano in senso orario (trovandosi a destra dell’asse \(x'\)).  Infine, gli spin con \(\Delta \omega = 0\) precessano alla frequenza \(\omega_{0}\) e risultano quindi stazionari nel sistema di riferimento rotante.

Dopo un certo intervallo di tempo $\tau$, gli spin risultano sfasati tra loro nel piano $x'y'$, e la magnetizzazione trasversa totale si riduce quasi a zero.

L’applicazione dell’impulso a $\pi$ intorno all’asse \(x'\) provoca una rotazione di mezzo giro (pari a una semicirconferenza) attorno all’asse di applicazione, portando ciascuno spin di magnetizzazione nella posizione diametralmente opposta rispetto a tale asse. In tal modo, il segno della deviazione di fase di ciascuno spin viene invertito: gli spin che erano in ritardo di fase diventano in anticipo e viceversa.  

La frequenza di precessione di ciascuno spin non cambia, poiché $\Delta \omega$ dipende unicamente dalle disomogeneità statiche di $B_0$ e non dagli impulsi di radiofrequenza.

Dopo l’impulso a $\pi$, gli spin continuano a precessare con la stessa distribuzione di frequenze ma con le fasi invertite.  Di conseguenza, gli spin che prima precessavano più lentamente tendono a recuperare la posizione iniziale, procedendo in senso antiorario. Analogamente, gli spin che prima precessavano più velocemente si muovo in modo da ritornare nella situazione di partenza, procedendo in senso orario. Trascorso un intervallo di tempo pari a $\tau$ dall’impulso a $\pi$, le differenze di fase accumulate prima dell’impulso vengono compensate, e tutti gli spin risultano nuovamente in fase lungo l’asse $x'$. Questo fenomeno è noto come \textbf{rifocalizzazione} e dà origine al segnale di \textbf{spin-echo}, che si osserva al tempo $t = 2\tau$ dall’applicazione del primo impulso.

\begin{figure}[ht]
\centering
\includegraphics[width=4.62912in,height=4.42403in,alt={Immagine che contiene diagramma, disegno, schizzo, Disegno tecnico Descrizione generata automaticamente}]{media/7_MRISignal/image100.pdf}\caption{Movimento degli spin nella sequenza spin-echo}
\end{figure}

Nel sistema di riferimento fisso, le componenti trasversali dei vari spin si sommano producendo un progressivo aumento della magnetizzazione trasversa.  
Questo avviene perché, dopo l’impulso a $\pi$, gli spin si stanno gradualmente rifocalizzando: le loro fasi tendono nuovamente ad allinearsi, e la somma vettoriale delle componenti trasversali assume valori via via più elevati.  
La magnetizzazione trasversa diventa quindi sempre più intensa fino a raggiungere un massimo, corrispondente al tempo di eco.  
Successivamente, si osserva un comportamento analogo a quello che segue l’impulso a $\pi/2$: gli spin veloci e lenti nel sistema rotante tendono nuovamente a sfasarsi, e la magnetizzazione trasversa decresce progressivamente verso zero, ristabilendo la condizione di equilibrio termodinamico.

\begin{figure}[ht]
\centering
\includegraphics[width=3.00295in,height=0.95921in,alt={Immagine che contiene schizzo, disegno, linea, arte Descrizione generata automaticamente}]{media/7_MRISignal/image101.pdf}\caption{Segnale registrato dopo l'applicazione dell'impulso a \(\pi\)}
\end{figure}

Questo processo avviene poiché ciascuno spin mantiene invariata la propria frequenza di precessione. Pertanto, esiste un istante in cui tutti gli spin attraversano simultaneamente l’asse $x'$, determinando il massimo della magnetizzazione trasversale. Dopo tale istante, la magnetizzazione si riduce con costante di tempo $T_2^*$, poiché le disomogeneità del campo magnetico causano nuovamente uno sfasamento tra gli spin. Si noti che anche il processo di rifocalizzazione avviene con la stessa costante di tempo apparente $T_2^*$, essendo determinato proprio dalle disomogeneità di campo.

La sequenza \textit{spin-echo}, composta dai due impulsi di ampiezza $\pi/2$ e $\pi$, sfrutta la rifocalizzazione della magnetizzazione per stimare sperimentalmente il tempo di rilassamento trasversale $T_2$.

Analiticamente, è possibile determinare l’istante in cui la magnetizzazione si rifocalizza lungo l’asse $x'$.  La fase di ciascuno spin nel sistema di riferimento rotante è legata alla disomogeneità del campo magnetico principale dalla relazione:
\[
\phi\left(t\right) = -\gamma \Delta B\, t,
\]
dove $\gamma$ è il rapporto giromagnetico e $\Delta B$ rappresenta la variazione locale del campo magnetico rispetto al valore medio $B_0$.

Nel tempo, la fase decresce finché non si applica il secondo impulso. In altre parole, dopo l'applicazione del primo impulso a \(\pi/2\), per un certo spin, la fase decresce con legge lineare nel tempo.

Sia $t = 0\ \text{s}$ il tempo di fine dell’impulso a $\pi/2$; dopo un intervallo $\tau$ si applica l’impulso a $\pi$. La fase dello spin immediatamente dopo quest'ultimo impulso sarà opposta a quella che lo spin possedeva all'istante di tempo appena precedente a \(\tau\). Ciò è dovuto al fatto che gli spin sono ribaltati rispetto all'asse su cui è applicativo l'impulso a \(\pi\).

Analiticamente, sia \(\phi\left( \tau^{-} \right)\) la fase un istante prima dell'applicazione dell'impulso a \(\pi\):

\[
\phi(\tau^-) = -\gamma \Delta B\, \tau.
\]

Poiché l’impulso a $\pi$ ribalta la magnetizzazione rispetto all’asse $x'$, la fase subito dopo l’impulso risulta opposta:

\[
\phi\left(\tau^+\right) = -\phi\left(\tau^-\right) = \gamma \Delta B\, \tau.
\]

Terminato l’impulso, l’evoluzione della fase prosegue linearmente nel tempo, mantenendo la stessa pendenza (poiché $\Delta B$ è costante), ma con la nuova condizione iniziale $\phi_0 = \gamma \Delta B\, \tau$:

\[
\phi\left( t \right) = -\gamma \Delta B \left(t - \tau\right) + \gamma \Delta B\, \tau = \gamma \Delta B \left(2\tau - t\right).
\]

Ciò accade poiché l'applicazione degli impulsi non provoca una variazione nella disomogeneità del campo principale, dunque, la frequenza di precessione degli spin non varia a causa degli impulsi a radiofrequenze.

Nel diagramma delle fasi, si distinguono due tratti rettilinei e paralleli: 
\begin{itemize}
    \item il primo, da $t=0$ a $t=\tau$, mostra l’evoluzione della fase dallo zero a $\phi(\tau^-) = -\gamma \Delta B \tau$, legata all'applicazione dell'impulso a \(\pi/2\);  
    \item il secondo, da $t=\tau$ a $t=2\tau$, rappresenta l’evoluzione dopo l’impulso a $\pi$, con pendenza invariata ma fase iniziale $\phi_0 = \gamma \Delta B \tau$.
\end{itemize}

\begin{figure}[ht]
\centering
\includegraphics[width=5.13386in,height=3.25197in]{media/7_MRISignal/image102.pdf}\caption{Diagrammi della fase dopo l'applicazione dell'impulso a \(\pi/2\)}
\end{figure}

Ogni spin percepisce una disomogeneità del campo magnetico principale $\Delta B$ diversa, e di conseguenza la propria fase evolve con una pendenza differente.  
Nonostante ciò, tutte le fasi degli spin convergono a un valore nullo nello stesso istante, in virtù del parallelismo tra l’andamento della fase prima e dopo l’applicazione dell’impulso a $\pi$.

\begin{figure}[ht]
\centering
\includegraphics[width=5.12992in,height=3.25197in]{media/7_MRISignal/image103.pdf}\caption{Per tutte gli spin le fasi si annullano nello stesso istante}
\end{figure}

L’istante in cui la fase di ciascuno spin si annulla è determinato dalla condizione:

\[
\phi\left( t \right) = 0 \Leftrightarrow \Delta B(2\tau - t) = 0 \Leftrightarrow t = 2\tau
\]

Il tempo necessario affinché la fase complessiva si annulli, dopo l’applicazione dell’impulso a $\pi$, è dunque esattamente $2\tau$, dove $\tau$ rappresenta l’intervallo di tempo tra i due impulsi.  
All’istante $t = 2\tau$, tutte le componenti di magnetizzazione trasversa sono nuovamente in fase e il segnale di echo raggiunge la sua massima ampiezza.

In sintesi, applicando un impulso a $\pi/2$ si genera un primo sfasamento della magnetizzazione.  
Dopo un intervallo $\tau$, l’impulso a $\pi$ inverte le fasi dei singoli spin, dando origine a una fase di rifocalizzazione seguita da un nuovo defasamento.  
In entrambe le fasi, gli inviluppi del segnale presentano un andamento esponenziale caratterizzato dalla costante di tempo apparente $T_2^*$, dovuta alle disomogeneità del campo magnetico principale.

\begin{figure}[ht]
\centering
\includegraphics[width=4.81835in,height=3.11733in,alt={Immagine che contiene schizzo, diagramma, disegno, linea Descrizione generata automaticamente}]{media/7_MRISignal/image104.pdf}\caption{Sequenza spin-echo e segnale registrato}
\end{figure}

% lista delle immagini (senza virgole!)
\def\imagelist{
image105, image106, image107, image108, image109, image110, image111, image112,
image113, image114, image115, image116, image117, image118, image119, image120,
image121, image122, image123, image124, image125}

% --- corpo principale ---
\setcounter{imgcount}{0}

\begin{center}
\foreach \image in \imagelist {%
    \includegraphics[width=\imgwidth]{media/7_MRISignal/\image.pdf}%
    \stepcounter{imgcount}%
    \ifnum\value{imgcount}<\imagesperrow
        \hspace{0.02\textwidth}% piccolo spazio tra immagini
    \else
        \par\vspace{0cm}% fine riga
        \setcounter{imgcount}{0}% reset del contatore
    \fi
}
\captionof{figure}{Andamento degli spin nella sequenza spin-echo}
\end{center}

Il punto in cui si ha il recupero della magnetizzazione trasversa, ovvero dove il segnale misurato raggiunge il valore massimo, corrisponde al \textbf{tempo di echo}:

\[
T_E = 2\tau.
\]

Durante l’evoluzione libera, l’inviluppo del segnale decresce con costante di tempo apparente $T_2^*$, poiché risente delle disomogeneità del campo magnetico principale $B_0$.  Tuttavia, tale tempo $T_2^*$ è un parametro \textbf{effettivo}, non un vero tempo di rilassamento, in quanto include effetti di dephasing dovuti a variazioni spaziali del campo.  L’unico vero tempo caratteristico del rilassamento trasversale è $T_2$, che dipende esclusivamente dalle interazioni spin-spin.

Poiché l’interazione spin-spin non è influenzata dalle disomogeneità di campo, il processo di rilassamento trasversale con costante $T_2$ continua ad agire durante tutta la sequenza, sia nella fase di sfasamento sia in quella di rifocalizzazione.  Di conseguenza, anche l’ampiezza dell’echo risulta modulata da un termine esponenziale di decadimento con costante di tempo $T_2$:

\[
M_{\bot}(t) \propto \exp\!\left(-\dfrac{t}{T_2}\right).
\]

In definitiva, la magnetizzazione trasversa al tempo di eco risulta ridotta di un fattore \(\exp\!\left(-{T_E}/{T_2}\right)\) rispetto alla magnetizzazione iniziale misurata subito dopo l’impulso a $\pi/2$.

In altre parole, mentre il processo di rifocalizzazione compensa gli effetti del dephasing dovuti alle disomogeneità statiche del campo (cioè a $T_2^*$), non può annullare il decadimento intrinseco dovuto alle interazioni spin-spin, per cui l’ampiezza degli echi successivi risulta progressivamente attenuata in base al tempo $T_2$.

Un primo metodo per ottenere una stima sperimentale del tempo $T_2$ mediante una sequenza \textit{spin-echo} consiste nell’acquisire il segnale in corrispondenza di diversi tempi di eco $T_E = 2\tau$.  L’intensità del segnale di eco, registrata dalle antenne, segue infatti un andamento esponenziale decrescente con costante di tempo $T_2$.

Il segnale ricevuto dalle bobine di risonanza viene demodulato e campionato, così da poter essere elaborato digitalmente.  
Dall’analisi dell’ampiezza degli echi in funzione del tempo di eco si ricava sperimentalmente il valore del tempo di rilassamento trasversale $T_2$.

\begin{figure}[ht]
\centering
\includegraphics[width=2.68333in,height=2.01894in]{media/7_MRISignal/image126.pdf}\caption{Elaborazione del segnale registrato}
\end{figure}

È possibile misurare il picco del segnale di echo ricevuto, ottenendo così informazioni sulla magnetizzazione trasversa, che a sua volta dipende dal tempo di rilassamento $T_2$:

\[
M_{\bot} \propto \exp\!\left(-\dfrac{T_E}{T_2}\right).
\]

In questo modo si ricavano informazioni dirette su $T_2$, eliminando gli effetti dovuti al tempo apparente $T_2^*$.  Tuttavia, la misura di un singolo segnale di echo non è sufficiente per determinare con precisione il tempo di rilassamento trasversale: è necessario acquisire più echi a diversi tempi $T_E$ e analizzarne l’andamento esponenziale.

La modulazione del segnale registrato secondo la legge \(\exp\left( - t/T_{2} \right)\) deriva dal fatto che il vettore di magnetizzazione $\vec{M}$ evolve nel tempo in accordo con le equazioni di Bloch.  All’interno del campione, gli spin che precessano alla stessa frequenza (\textbf{isocromati}) presentano fasi inizialmente casuali, legate a \(\Delta B\). Le disomogeneità statiche del campo magnetico principale $B_0$ fanno sì che differenti regioni del campione abbiano frequenze di precessione leggermente diverse, producendo un rapido sfasamento tra gli isocromati. L’effetto complessivo è una magnetizzazione trasversa che, in assenza di rifocalizzazione, decresce come:

\[
M_{\bot}(t) \propto \exp\!\left(-\dfrac{t}{T_2^*}\right),
\]

a causa delle disomogeneità di campo.

Il vettore di magnetizzazione, secondo le equazioni di Bloch, subisce inoltre un decadimento intrinseco della sua componente trasversale con costante di tempo $T_2$:
\[
M_{\bot}(t) \propto \exp\!\left(-\dfrac{t}{T_2}\right).
\]
Questo fenomeno rappresenta il vero processo di rilassamento trasversale, dovuto alle interazioni spin-spin, e consente di misurare sperimentalmente il tempo $T_2$ mediante sequenze di tipo \textbf{spin-echo}.

\subsection{Multiple spin-echo}\label{multiple-spin-echo}

Esistono due principali strategie per la stima del tempo di rilassamento trasversale $T_{2}$ mediante la sequenza \textit{spin-echo}, entrambe basate sull’acquisizione di segnali con tempi di eco diversi.

\subsection{Applicazione multipla della sequenza spin-echo}\label{applicazione-multipla-della-sequenza-spin-echo}

Una prima strategia per ottenere una misura del tempo di rilassamento trasversale \(T_{2}\) consiste nell'applicazione di più sequenze \textit{spin-echo}, caratterizzate da impulsi a \(\pi\) separati da differenti tempi di echo. In particolare, si considerino due sequenze: la prima con tempo di echo \(T_{E_{1}}\) e la seconda con tempo di echo \(T_{E_{2}}\).

In questa configurazione, dopo il decadimento della componente trasversa della magnetizzazione a seguito della prima sequenza \textit{spin-echo}, la seconda sequenza viene applicata partendo dalle stesse condizioni iniziali della prima e segue la stessa legge di decadimento esponenziale \(\exp\left( - t/T_{2} \right)\).

\begin{figure}[ht]
\centering
\includegraphics[width=5.52083in,height=2.62445in]{media/7_MRISignal/image127.pdf}\caption{Sequenze spin-echo con tempi di eco diversi}
\end{figure}

Durante le finestre di acquisizione si registrano due echi: il primo al tempo \(T_{E_{1}}\) e il secondo al tempo \(T_{E_{2}}\), con \(T_{E_{2}} > T_{E_{1}}\). Poiché entrambe le sequenze presentano lo stesso andamento di decadimento, il segnale acquisito al tempo \(T_{E_{1}}\) risulta proporzionale alla componente trasversa della magnetizzazione:


\[
s\left( T_{E_{1}} \right) \propto M_{\bot} \propto \exp\left( - \dfrac{T_{E_{1}}}{T_{2}} \right)
\]

Nella seconda finestra di acquisizione si registra un segnale al tempo di echo \(T_{E_{2}}\), proporzionale alla magnetizzazione trasversa:

\[
s\left( T_{E_{2}} \right) \propto M_{\bot} \propto \exp\left( - \dfrac{T_{E_{2}}}{T_{2}} \right)
\]

I due segnali sono correlati, poiché la magnetizzazione iniziale \(M_{\bot}(\vec{r},0)\) è la stessa per entrambe le sequenze. Calcolando il rapporto tra i due segnali, i fattori di proporzionalità si semplificano:

\[
\dfrac{s\left( T_{E_{1}} \right)}{s\left( T_{E_{2}} \right)} = \dfrac{\exp\left( - \dfrac{T_{E_{1}}}{T_{2}} \right)}{\exp\left( - \dfrac{T_{E_{2}}}{T_{2}} \right)}
\]

Definendo per semplicità di notazione \(s_{1} = s\left(T_{E_{1}}\right)\) e \(s_{2} = s\left(T_{E_{2}}\right)\), si ottiene:

\[
\dfrac{s_{1}}{s_{2}} = \dfrac{\exp\left( - \dfrac{T_{E_{1}}}{T_{2}} \right)}{\exp\left( - \dfrac{T_{E_{2}}}{T_{2}} \right)} = \exp\left( - \dfrac{T_{E_{1}} - T_{E_{2}}}{T_{2}} \right)
\]

Da cui, invertendo la relazione, si ricava un’espressione per la stima del tempo di rilassamento trasversale \(T_{2}\):

\[
T_{2} = \dfrac{T_{E_{2}} - T_{E_{1}}}{\log\left( \dfrac{s_{1}}{s_{2}} \right)} = \dfrac{T_{E_{2}} - T_{E_{1}}}{\log\left( s_{1} \right) - \log\left( s_{2} \right)}
\]

In questo modo si ottiene una misura sperimentale del tempo \(T_{2}\).

Storicamente, le prime misure del tempo di rilassamento trasversale nei tessuti biologici umani utilizzavano proprio questa metodica. Tuttavia, poiché i segnali \(s_{1}\) e \(s_{2}\) vengono acquisiti con un'incertezza sperimentale, anche la stima di \(T_{2}\) risulta affetta da errore per effetto della propagazione dell’incertezza.

L’accuratezza della stima di \(T_{2}\) dipende dunque dalla precisione con cui vengono misurati i segnali ai tempi di echo \(T_{E_{1}}\) e \(T_{E_{2}}\). Tale errore può risultare significativo, soprattutto se il rapporto segnale/rumore è basso. Inoltre, la tecnica si basa sull’ipotesi che la magnetizzazione iniziale delle due sequenze \textit{spin-echo} sia identica. Affinché tale condizione sia soddisfatta, tra una sequenza e la successiva deve trascorrere un intervallo temporale almeno pari a \(5T_{1}\), così da permettere alla magnetizzazione longitudinale di ritornare all’equilibrio.

\subsection[Applicazione multipla dei gradienti a pi]{Applicazione multipla dei gradienti a $\mathbf{\pi}$}
\label{applicazione-gradienti-pi}

Una soluzione per stimare il tempo di rilassamento trasversale $T_{2}$ con maggiore precisione prevede l’acquisizione di più echi mediante una sequenza nota come \textit{multiple spin-echo}. In questa configurazione, si applica un primo impulso a $\pi/2$ che perturba l’equilibrio del campione, seguito da una serie di impulsi a $\pi$ distanziati da un tempo $\tau = T_{E}/2$.

\begin{figure}[ht]
\centering
\includegraphics[width=4.64394in,height=2.92077in]{media/7_MRISignal/image128.pdf}\caption{Sequenza multiple spin-echo}
\end{figure}

Con questa metodologia, mediante una sola eccitazione iniziale a $\pi/2$, si ottengono più misure della magnetizzazione trasversa, consentendo una valutazione più accurata del tempo di rilassamento $T_{2}$.

Dal punto di vista del sistema di riferimento rotante, l’impulso a $\pi/2$ ribalta la magnetizzazione sul piano trasverso, ponendo gli spin in precessione attorno a uno degli assi $x'$ o $y'$.  
Se, ad esempio, l’impulso è applicato lungo l’asse $y'$, la magnetizzazione si focalizza inizialmente su tale asse. Gli spin che precessano più velocemente della rotazione del sistema si muovono in senso orario rispetto a $y'$, mentre quelli più lenti in senso antiorario.

Nella sequenza a treno di echi (\textit{echo train}), gli impulsi di rifocalizzazione ($\pi$) vengono applicati in serie. Poiché ciascun impulso $\pi$ inverte la fase accumulata e ribalta il vettore $M_{\perp}$ nel piano trasversale, gli echi successivi si formano alternativamente lungo le direzioni opposte (ad esempio lungo $\hat{\imath}_{y'}$ e $-\hat{\imath}_{y'}$), campionando in modo efficace il decadimento associato a $T_{2}$.

Questo meccanismo è noto come sequenza di \textbf{Carr–Purcell–Meiboom–Gill (CPMG)} ed è progettato per minimizzare gli errori dovuti alle imperfezioni degli impulsi a $\pi$ e alle inhomogeneità residue del campo.

Analogamente, se gli impulsi a $\pi$ sono applicati lungo $y'$, la magnetizzazione trasversa si focalizza alternativamente su $x'$ e $-x'$ ad ogni echo.

% lista delle immagini (senza virgole!)
\def\imagelist{
image129, image130, image131, image132, image133, image134, image135, image136,
image137, image138, image139, image140, image141, image142, image143, image144,
image145, image146, image147, image148, image149, image150, image151, image152,
image153, image154, image155, image156, image157, image158, image159, image160,
image161, image162, image163, image164, image165, image166, image167, image168,
image169, image170, image171, image172, image173, image174, image175, image176,
image177, image178, image179, image180, image181, image182, image183, image184,
image185, image186, image187, image188, image189, image190, image191, image192,
image193, image194, image195, image196, image197, image198, image199, image200,
image201, image202, image203, image204, image205, image208}

% --- corpo principale ---
\setcounter{imgcount}{0}

\begin{center}
\foreach \image in \imagelist {%
    \includegraphics[width=\imgwidth]{media/7_MRISignal/\image.pdf}%
    \stepcounter{imgcount}%
    \ifnum\value{imgcount}<\imagesperrow
        \hspace{0.02\textwidth}% piccolo spazio tra immagini
    \else
        \par\vspace{0cm}% fine riga
        \setcounter{imgcount}{0}% reset del contatore
    \fi
}
\captionof{figure}{Movimento degli spin nella sequenza multiple spin-echo}
\end{center}

Gli impulsi di rifocalizzazione consentono di misurare un valore del segnale proporzionale alla magnetizzazione in corrispondenza dei tempi di echo. Tutte le misure condividono la stessa magnetizzazione iniziale, garantendo coerenza nella stima.

In definitiva, il processo di misura equivale a un campionamento del segnale di magnetizzazione trasversale:

\[
M_{\bot}(t) \propto \exp\!\left(-\dfrac{t}{T_{2}}\right),
\]

con periodo di campionamento pari a $T_{E}$.  Questo approccio consente una valutazione più robusta e precisa del tempo di rilassamento trasversale $T_{2}$.

Tipicamente, $T_{2}$ è dell’ordine di alcune centinaia di millisecondi. Affinché il segnale $\exp\!\left(-t/T_{2}\right)$ possa considerarsi estinto, è necessario attendere un intervallo temporale di circa $5T_{2} \sim 500~\text{ms}$.  
Scegliendo un tempo di eco di $T_{E} = 10~\text{ms}$, è possibile acquisire fino a 50 echi, ricostruendo così con buona risoluzione temporale l’evoluzione della magnetizzazione trasversale $M_{\bot}(t)$.

\subsection[Stima del tempo T2 da n misurazioni]{Stima del tempo $\mathbf{T}_{\mathbf{2}}$ da $\mathbf{n}$ misurazioni}
\label{stima-tempo-T2-n-misurazioni}

Si suppone di applicare una sequenza multiple spin-echo. Il segnale misurato, a causa delle disomogeneità di campo, decresce con costante di tempo \(T_2^*\). Per l'applicazione degli impulsi a \(\pi\) ripetuti, l'ampiezza del picco massimo dell'echo si riduce come \(\exp(-t/T_2)\). Al tempo \(t_n = nT_E\), gli isocromi sono rifocalizzati sull'asse \(x'\) o \(y'\), e il segnale registrato ha un massimo proporzionale a \(\exp\!\left(-nT_E/T_2\right)\).

Valutando i segnali registrati ai tempi di echo, si ottengono dei campioni del segnale \(\exp\left( - t/T_{2} \right)\), da cui è possibile ricavare il tempo di rilassamento trasversale \(T_{2}\).

Il segnale \(s\left( t \right)\) misurato è uguale al modello scelto per descrivere il comportamento del vettore di magnetizzazione, a cui si somma un termine di errore \(\varepsilon\), supposto essere additivo:

\[
s\left( t \right) = \exp\left( - \dfrac{t}{T_{2}} \right) + \varepsilon
\]

Il modello non corrisponde esattamente alle misure sperimentali. Si valutano i campioni nei tempi di echi \(t_{n} = nT_{E}\):

\[
s\left( nT_{E} \right) = s_{n} = \exp\left( - \dfrac{nT_{E}}{T_{2}} \right) + \varepsilon_{n}
\]

Il termine di errore \(\varepsilon_{n}\) dipende dalla misura, in quanto ogni misura è affetta da un errore diverso dagli altri e statisticamente indipendenti.

Per stimare \(T_2\) si applica il metodo dei minimi quadrati (\textbf{Least Squares}, LS). Si linearizza il modello applicando il logaritmo:


Per stimare il tempo \(T_{2}\) dalle \(n\) misurazioni si applica il metodo dei minimi quadrati o \textit{Least Squares} (LS) o \textit{Ordinary Least Squares} (OLS) ideato da Gauss. A tale scopo, per rendere la relazione tra la misura e la quantità da valutare lineare, si applica il logaritmo a ambo i membri dell'equazione per \(s_{n}\):

\[
\log\left( s_{n} \right) = - \dfrac{nT_{E}}{T_{2}} + \varepsilon_{n}'
\]

dove \(\varepsilon_{n}'\) è un termine di errore additivo dipendente dal logaritmo dell'errore \(\varepsilon_{n}\). Questa relazione è del tipo:

\[
y = mx + q + \varepsilon
\]

con \(y = \log\left( s_{n} \right)\), \(m = - nT_{E}/T_{2}\) e \(q \neq 0\) se non normalizzato.

Mediante \(n\) misurazioni si ottiene una popolazione di \(n\) coppie \(\left( y_{i},x_{i} \right)\), legati dalla relazione:
\[
y_i = mx_i + q + \varepsilon_i, \quad i = 1, \ldots, n
\]

Per ogni campione si ottiene una relazione lineare tra \(y\) e i coefficienti \(m\) e \(q\) della regressione. Il sistema di equazioni può essere scritto in forma matriciale:

\[
\begin{cases}
y_{1} = mx_{1} + q + \varepsilon_{1} \\
y_{2} = mx_{2} + q + \varepsilon_{2} \\
\dots \\
y_{n} = mx_{n} + q + \varepsilon_{n}
\end{cases} 
\]

Si introduce il vettore delle misure:

\[
\vec{y} =
\begin{pmatrix}
y_1 \\
y_2 \\
\vdots \\
y_n
\end{pmatrix}
\]

Si introduce la matrice dei coefficienti o di design:

\[
\mathbf{X} =
\begin{pmatrix}
x_1 & 1 \\
x_2 & 1 \\
\vdots & \vdots \\
x_n & 1
\end{pmatrix}
\]

si definisce il vettore delle incognite, spesso indicato con il simbolo \(\vec{\vartheta}\), con due sole componenti:

\[
\vec{\vartheta} =
\begin{pmatrix}
m \\
q
\end{pmatrix}
\]

Infine, si introduce il vettore degli errori:

\[
\vec{\varepsilon} =
\begin{pmatrix}
\varepsilon_1 \\
\varepsilon_2 \\
\vdots \\
\varepsilon_n
\end{pmatrix}
\]

La relazione di regressione può essere scritta come:

\[
\vec{y} = \mathbf{X} \vec{\vartheta} + \vec{\varepsilon}
\]

In generale, questa equazione è valida per ogni tipologia di regressione lineare e, dunque, anche il metodo proposto per valutare \(T_{2}\) è valido.

Si osservi che \(\mathbf{X}\) è una matrice \(2 \times n\), quindi, non può essere invertita per ottenere la soluzione. In generale, la matrice dei coefficienti nel metodo dei minimi quadrati è di \(m \times n\), dove \(m\) è il numero delle incognite ed \(n\) quello delle misure.

Sia \({\vec{\vartheta}}^{*}\) il valore vero dei parametri incogniti; la seguente equazione:

\[
{\vec{y}}^{*} = \mathbf{X}{\vec{\vartheta}}^{*}
\]

rappresenta il valore vero delle osservazioni.

I vari elementi del vettore delle misure \(\vec{y}\) sono affetti da rumore approssimabile come variabili aleatori \(\varepsilon_{i}\). Si vuole trovare il vettore \(\vec{\vartheta}\) tale da minimizzare l'errore quadratico medio, ovvero:

\[
\hat{\vec{\vartheta}} = {\arg{\min_{\vec{\vartheta}}\left\| \vec{\varepsilon} \right\|}}^{2} = \arg{\min_{\vec{\vartheta}}\left\| \vec{y} - \mathbf{X}\vec{\vartheta} \right\|^{2}}
\]

I parametri \(\vec{\vartheta}\) devono rendere minima la distanza tra le misure \(\vec{y}\) e le previsioni del modello \(\mathbf{X}\vec{\vartheta}\).

La quantità \(\left\| \vec{y} - \mathbf{X}\vec{\vartheta} \right\|^{2}\) può essere scritta in forma matriciale come:

\[
\left\| \vec{y} - \mathbf{X}\vec{\vartheta} \right\|^{2} = \left( \vec{y} - \mathbf{X}\vec{\vartheta} \right)^{T}\left( \vec{y} - \mathbf{X}\vec{\vartheta} \right)
\]

Si indica con \(S\left( \vec{\vartheta} \right) = \left\| \vec{y} - \mathbf{X}\vec{\vartheta} \right\|^{2}\), l'ultima relazione si scrive come:

\[
S\left( \vec{\vartheta} \right) = \left( \vec{y} - \mathbf{X}\vec{\vartheta} \right)^{T}\left( \vec{y} - \mathbf{X}\vec{\vartheta} \right)
\]

Svolgendo il trasposto e i prodotti si ha:

\[
S\left( \vec{\vartheta} \right) = \left( \vec{y} - \mathbf{X}\vec{\vartheta} \right)^{T}\left( \vec{y} - \mathbf{X}\vec{\vartheta} \right) = \left( {\vec{y}}^{T} - {\vec{\vartheta}}^{T}{\mathbf{X}}^{T} \right)\left( \vec{y} - \mathbf{X}\vec{\vartheta} \right) = {\vec{y}}^{T}\vec{y} - {\vec{y}}^{T}\mathbf{X}\vec{\vartheta} - {\vec{\vartheta}}^{T}{\mathbf{X}}^{T}\vec{y} + {\vec{\vartheta}}^{T}{\mathbf{X}}^{T}\mathbf{X}\vec{\vartheta}
\]

Le quantità \({\vec{y}}^{T}\mathbf{X}\vec{\vartheta}\) e \({\vec{\vartheta}}^{T}{\mathbf{X}}^{T}\vec{y}\) sono degli scalari, quindi, possono essere sommati tra loro:

\[
S\left( \vec{\vartheta} \right) = {\vec{y}}^{T}\vec{y} + {\mathbf{X}}^{T}\mathbf{X}{\vec{\vartheta}}^{2} - 2{\vec{y}}^{T}\mathbf{X}\vec{\vartheta}
\]

Minimizzare il valore quadratico medio equivale a uguagliare a zero la derivata della quantità \(S\!\left( \vec{\vartheta} \right)\):

\[
\dfrac{\partial S\left( \vec{\vartheta} \right)}{\partial\vec{\vartheta}} = 0 \Leftrightarrow 2{\mathbf{X}}^{T}\mathbf{X}\vec{\vartheta} - 2{\vec{y}}^{T}\mathbf{X} = 0 \Leftrightarrow {\mathbf{X}}^{T}\mathbf{X}\vec{\vartheta} = {\vec{y}}^{T}\mathbf{X}
\]

Risulta che:

\[
{\vec{y}}^{T}\mathbf{X} = {\mathbf{X}}^{T}\vec{y}
\]

Per cui si ottiene:

\[
{\mathbf{X}}^{T}\mathbf{X}\vec{\vartheta} = {\mathbf{X}}^{T}\vec{y}
\]

Moltiplicando a destra e a sinistra per \(\left( {\mathbf{X}}^{T}\mathbf{X} \right)^{- 1}\), si ottiene l'equazione per la soluzione \textit{ordinary least squares}:

\[
\hat{\vec{\vartheta}} = \left( {\mathbf{X}}^{T}\mathbf{X} \right)^{- 1}{\mathbf{X}}^{T}\vec{y}
\]

Si suppone che il rumore sia a media nulla, ovvero:

\[
E\left[ \vec{\varepsilon} \right] = \vec{0}
\]

La matrice della covarianza è data da:

\[
E\left[ \vec{\varepsilon}{\vec{\varepsilon}}^{T} \right] = \sigma^{2}\mathbf{I}
\]

Si dimostra che la media della soluzione OLS tende al valore vero dei parametri incogniti \({\vec{\vartheta}}^{*}\), infatti la media di \(\hat{\vec{\vartheta}}\) è data da:

\[
E\left[ \hat{\vec{\vartheta}} \right] = E\left[ \left( {\mathbf{X}}^{T}\mathbf{X} \right)^{- 1}{\mathbf{X}}^{T}\vec{y} \right]
\]

Ma \(\vec{y} = \mathbf{X}{\vec{\vartheta}}^{*} + \vec{\varepsilon}\); inoltre la matrice di design non contiene variabili aleatorie, quindi, può essere portato fuori dall'operazione di media statistica:

\[
E\left[ \hat{\vec{\vartheta}} \right] = E\left[ \left( {\mathbf{X}}^{T}\mathbf{X} \right)^{- 1}{\mathbf{X}}^{T}\vec{y} \right] = \left( {\mathbf{X}}^{T}\mathbf{X} \right)^{- 1}{\mathbf{X}}^{T}E\left[ \vec{y} \right] = \left( {\mathbf{X}}^{T}\mathbf{X} \right)^{- 1}{\mathbf{X}}^{T}E\left[ \mathbf{X}{\vec{\vartheta}}^{*} + \vec{\varepsilon} \right] =
\]

Per la linearità dell'operatore valor medio si ha:

\[
= \left( {\mathbf{X}}^{T}\mathbf{X} \right)^{- 1}{\mathbf{X}}^{T}\left( \mathbf{X}E\left[ {\vec{\vartheta}}^{*} \right] + E\left[ \vec{\varepsilon} \right] \right)
\]

Per ipotesi il rumore è a media nulla, per cui:

\[
E\left[ \hat{\vec{\vartheta}} \right] = \left( {\mathbf{X}}^{T}\mathbf{X} \right)^{- 1}{\mathbf{X}}^{T}\mathbf{X}E\left[ {\vec{\vartheta}}^{*} \right]
\]

Per definizione di matrice inversa risulta:

\[
E\left[ \hat{\vec{\vartheta}} \right] = E\left[ {\vec{\vartheta}}^{*} \right]
\]

La stima di \(T_{2}\) con questo metodo è detto non polarizzata o \textit{unbiased}.

Si calcola, ora, la matrice di covarianza; secondo la definizione si ha:

\[
E\left[ \left( \hat{\vec{\vartheta}} - {\vec{\vartheta}}^{*} \right)\left( \hat{\vec{\vartheta}} - {\vec{\vartheta}}^{*} \right)^{T} \right] = E\left[ \left( \left( {\mathbf{X}}^{T}\mathbf{X} \right)^{- 1}{\mathbf{X}}^{T}\left( \mathbf{X}{\vec{\vartheta}}^{*} + \vec{\varepsilon} \right) - {\vec{\vartheta}}^{*} \right)\left( \left( {\mathbf{X}}^{T}\mathbf{X} \right)^{- 1}{\mathbf{X}}^{T}\left( \mathbf{X}{\vec{\vartheta}}^{*} + \vec{\varepsilon} \right) - {\vec{\vartheta}}^{*} \right)^{T} \right] =
\]

Svolgendo i prodotti si ha:

\[
= E\left[ \left( \left( {\mathbf{X}}^{T}\mathbf{X} \right)^{- 1}{\mathbf{X}}^{T}\mathbf{X}{\vec{\vartheta}}^{*} + \left( {\mathbf{X}}^{T}\mathbf{X} \right)^{- 1}{\mathbf{X}}^{T}\vec{\varepsilon} - {\vec{\vartheta}}^{*} \right)\left( \left( {\mathbf{X}}^{T}\mathbf{X} \right)^{- 1}{\mathbf{X}}^{T}\mathbf{X}{\vec{\vartheta}}^{*} + \left( {\mathbf{X}}^{T}\mathbf{X} \right)^{- 1}{\mathbf{X}}^{T}\vec{\varepsilon} - {\vec{\vartheta}}^{*} \right)^{T} \right] =
\]

Per la proprietà della matrice inversa, si ha:

\[
= E\left[ \left( {\vec{\vartheta}}^{*} + \left( {\mathbf{X}}^{T}\mathbf{X} \right)^{- 1}{\mathbf{X}}^{T}\vec{\varepsilon} - {\vec{\vartheta}}^{*} \right)\left( {\vec{\vartheta}}^{*} + \left( {\mathbf{X}}^{T}\mathbf{X} \right)^{- 1}{\mathbf{X}}^{T}\vec{\varepsilon} - {\vec{\vartheta}}^{*} \right)^{T} \right]=
\]

Semplificando e svolgendo i prodotti si ha:

\[\begin{aligned}
&= E\left[ 
  \left( (\mathbf{X}^{T}\mathbf{X})^{-1}\mathbf{X}^{T}\vec{\varepsilon} \right)
  \left( (\mathbf{X}^{T}\mathbf{X})^{-1}\mathbf{X}^{T}\vec{\varepsilon} \right)^{T} 
\right] = E\left[ 
  \left( (\mathbf{X}^{T}\mathbf{X})^{-1}\mathbf{X}^{T}\vec{\varepsilon} \right)
  \left( \vec{\varepsilon}^{T}\mathbf{X}(\mathbf{X}^{T}\mathbf{X})^{-T} \right)
\right] \\
&= 
\left(\mathbf{X}^{T}\mathbf{X}\right)^{-1}\mathbf{X}^{T}
E\!\left[ \vec{\varepsilon}\vec{\varepsilon}^{T} \right]
\mathbf{X}(\mathbf{X}^{T}\mathbf{X})^{-T}
\end{aligned}\]


Ma, per ipotesi \(E\left[ \vec{\varepsilon}{\vec{\varepsilon}}^{T} \right]\mathbf{X} = \sigma^{2}\mathbf{I}\), per cui:

\[
E\left[ \left( \hat{\vec{\vartheta}} - {\vec{\vartheta}}^{*} \right)\left( \hat{\vec{\vartheta}} - {\vec{\vartheta}}^{*} \right)^{T} \right] = \left( {\mathbf{X}}^{T}\mathbf{X} \right)^{- 1}{\mathbf{X}}^{T}\sigma^{2}\mathbf{I}\mathbf{X}\left( {\mathbf{X}}^{T}\mathbf{X} \right)^{- T} = \sigma^{2}\left( {\mathbf{X}}^{T}\mathbf{X} \right)^{- 1}{\mathbf{X}}^{T}\mathbf{X}\left( {\mathbf{X}}^{T}\mathbf{X} \right)^{- T}
\]

Dove \({\mathbf{X}}^{T}\mathbf{X}\) è una matrice simmetrica, per cui l'inversa è anch'essa simmetrica, ovvero:

\[
\left[ \left( {\mathbf{X}}^{T}\mathbf{X} \right)^{- 1} \right]^{T} = \left[ \left( {\mathbf{X}}^{T}\mathbf{X} \right)^{T} \right]^{- 1} = \left( {\mathbf{X}}^{T}\mathbf{X} \right)^{- 1}
\]

Per cui:

\[
\left( {\mathbf{X}}^{T}\mathbf{X} \right)^{- 1}{\mathbf{X}}^{T}\mathbf{X} = \mathbf{I}
\]

Di conseguenza:

\[
E\left[ \left( \hat{\vec{\vartheta}} - {\vec{\vartheta}}^{*} \right)\left( \hat{\vec{\vartheta}} - {\vec{\vartheta}}^{*} \right)^{T} \right] = \sigma^{2}\left( {\mathbf{X}}^{T}\mathbf{X} \right)^{- 1}
\]

La curva stimata per ricostruire le misure può differire da quella teorica sia per eccesso che per difetto; tuttavia, non vi è modo di prevedere né di conoscere esattamente il parametro stimato. Ovviamente, minore è l'errore additivo e più precisa sarà la stima di \(T_{2}\) mediante il metodo dei minimi quadrati.

\begin{figure}[ht]
\centering
\includegraphics[width=3.78269in,height=3.17361in,alt={Exponential Fitting Using OriginLab 2021 \textbar{} \textbar{} Drawing/Graphing-27}]{media/7_MRISignal/image212.pdf}\caption{Fitting della curva esponenziale}
\end{figure}

\subsection{Inversion Recovery}\label{inversion-recovery}

Le sequenze FID e \textit{spin-echo} sono molto utili per determinare i tempi di rilassamento trasversale \(T_2\) e \(T_2^*\),
ma non permettono di valutare il tempo di rilassamento longitudinale \(T_1\).

L’esperimento noto come \textit{inversion recovery} consente di ottenere una misura precisa di \(T_1\) mediante una sola acquisizione.
La sequenza è simile alla \textit{spin-echo}, ma l’impulso a \(\pi/2\) è preceduto da un impulso a \(\pi\),
separato da un intervallo temporale \(T_I\).

Nella misura di \(T_1\) attraverso l’\textit{inversion recovery} è necessario analizzare l’evoluzione della componente longitudinale
del vettore di magnetizzazione, regolata dall’equazione di Bloch:

\[
\dfrac{dM_z}{dt} = \dfrac{1}{T_1}\left( M_0 - M_z \right)
\]

Dopo l’impulso a \(\pi\), la magnetizzazione è ruotata in modo da essere orientata lungo la direzione negativa dell’asse \(z\).

\begin{figure}[ht]
    \centering
    \includegraphics[width=0.6\textwidth]{media/7_MRISignal/image213.pdf}
    \caption{Inversione della magnetizzazione dopo l'impulso a \(\pi\).}
\end{figure}

Sia \(t = 0~\text{s}\) l’istante in cui l’impulso a \(\pi\) è interrotto. 
La magnetizzazione, in questo istante, è ribaltata di \(\pi\), per cui:

\[
M(0) = -M_0
\]

dove \(M_0\) è il valore della magnetizzazione all’equilibrio termodinamico.

Subito dopo l’applicazione dell’impulso a \(\pi\), il vettore di magnetizzazione evolve secondo un andamento esponenziale del tipo:

\[
M_z(t) = M_0 \left[ 1 - 2\exp\left( -\dfrac{t}{T_1} \right) \right]
\]

Al tempo \(t = T_I\), si applica l’impulso a \(\pi/2\) lungo uno degli assi trasversali (\(x'\) o \(y'\)).
In questo modo la magnetizzazione longitudinale è ribaltata nel piano trasversale.

\begin{figure}[ht]
    \centering
    \includegraphics[width=0.8\textwidth]{media/7_MRISignal/image214.pdf}
    \caption{Andamento della componente longitudinale nel tempo.}
\end{figure}

Dall’equazione per \(M_z(t)\) si nota che esiste un certo istante temporale in cui la magnetizzazione longitudinale si annulla.
Se si applica in questo istante l’impulso a \(\pi/2\), non si registrerebbe alcun segnale,
poiché il vettore di magnetizzazione è nullo e quindi non può essere ribaltato.

Ogni tessuto presenta un tempo di rilassamento \(T_1\) specifico.
Applicando un impulso a \(\pi/2\) nel momento opportuno è possibile ottenere informazioni selettive:
i tessuti che presentano una magnetizzazione longitudinale nulla al tempo \(T_I\) (\(M_z(T_I) = 0\)) non contribuiscono al segnale e risultano quindi soppressi. Questa tecnica è utilizzata, ad esempio, per la soppressione selettiva del grasso (\textit{fat suppression}).

Conoscendo \(T_1\) del tessuto da sopprimere, si può scegliere \(T_I\) in modo che la magnetizzazione longitudinale sia nulla, escludendo così il contributo del tessuto all’immagine.

Per ottenere la misura di \(T_1\) attraverso la sequenza \textit{inversion recovery}, si consideri che al tempo \(t = 0^{+}\) (fine dell’impulso a \(\pi\)) la magnetizzazione è orientata lungo \(-\hat{\mathbf{i}}_z\):

\[
M(0^{+}) = -M_0
\]

Prima del tempo \(T_I\) (istante dell’impulso a \(\pi/2\)), la magnetizzazione longitudinale ritorna all’equilibrio secondo:

\[
M_z(t) = M_0\left[ 1 - 2\exp\left( -\dfrac{t}{T_1} \right) \right], \quad 0 < t < T_I
\]

Al tempo di inversione, la magnetizzazione longitudinale ha ampiezza:

\[
M_z(T_I) = M_0\left[ 1 - 2\exp\left( -\dfrac{T_I}{T_1} \right) \right]
\]

La componente trasversale evolve di conseguenza come:

\[
M_{\bot}(t) =
\left| M_0\left[ 1 - 2\exp\left( -\dfrac{T_I}{T_1} \right) \right] \right|
\exp\left( -\dfrac{t - T_I}{T_2^*} \right)
\]

L’ampiezza del segnale misurato è modulata da un fattore dipendente dal tempo di rilassamento longitudinale \(T_1\), dato da \(\left| M_0\left[ 1 - 2\exp\left( -T_I/T_1 \right) \right] \right|\). Il segnale registrato si annulla quando:

\[
\left| M_0\left[ 1 - 2\exp\left( -\dfrac{T_I}{T_1} \right) \right] \right| = 0\Leftrightarrow 
\exp\left( -\dfrac{T_I}{T_1} \right) = \dfrac{1}{2}
\]

Da cui si ricava il tempo di inversione in funzione del tempo di rilassamento longitudinale:

\[
-\dfrac{T_I}{T_1} = \log\!\left( \dfrac{1}{2} \right)
\quad \Leftrightarrow \quad
\dfrac{T_I}{T_1} = \log(2)
\]

e quindi:

\[
T_I = T_1 \log(2)
\]

\begin{figure}[ht]
    \centering
    \includegraphics[width=0.7\textwidth]{media/7_MRISignal/image215.pdf}
    \caption{Andamento della magnetizzazione trasversale nel tempo nell’inversion recovery.}
\end{figure}

Il segnale registrato \(s(t)\) è funzione di \(T_1\),
e la scelta del tempo \(T_I\) è cruciale per ottenere informazioni selettive. La sequenza \textit{inversion recovery}, seguita da una \textit{spin-echo}, consente di scegliere il tempo di eco in modo che la magnetizzazione trasversale del tessuto da sopprimere sia nulla.

Conoscendo le caratteristiche del tessuto, è quindi possibile calcolare \(T_I\) per annullarne il contributo alla magnetizzazione trasversale,
ottenendo così la soppressione selettiva del segnale indesiderato.
