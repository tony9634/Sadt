\begin{figure}[h!]
    \centering
\begin{tikzpicture}[>=stealth, thick]

% Punti A e B
\node[label=left:{$\left(\vec{A}, t_0\right)$}] (A) at (0,0) {};
\node[label=right:{$\left(\vec{B}, t_1\right)$}] (B) at (6,2) {};

% Prima curva (senza freccia)
\draw[line width=1.2pt] (A) to[out=60, in=180] (2,0);

% Seconda curva (con freccia verso B)
\draw[->, line width=1.2pt] (2,0) to[out=0, in=150] (B);

% Punti visibili
\fill (A) circle (2pt);
\fill (B) circle (2pt);

\end{tikzpicture}

    \caption{Esempio di moto su traiettoria a azione stazionaria tra due punti nel tempo}
    \label{fig:1_Traiettoria}
\end{figure}
