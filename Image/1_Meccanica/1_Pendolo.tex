\begin{figure}[h!]
    \centering

\begin{tikzpicture}[scale=3, thick, >=stealth]

% --- Parametri ---
\coordinate (B) at (0,-1);
\coordinate (O) at (0,0);       % punto di sospensione
\def\L{2}                       % lunghezza del filo
\def\ang{35}                    % angolo di deviazione

% --- Massa ---
\coordinate (M) at (\ang:-\L);

% --- Linea verticale di riferimento ---
\draw[dashed, gray] (O) -- (0,-\L-0.3) node[below] {Verticale};

% --- Filo del pendolo ---
\draw[black] (O) -- (M) node[midway, right=2pt] {$L$};

% --- Massa (pallina) ---
\filldraw[fill=gray!30] (M) circle (0.08) node[right=5pt] {$m$};

% --- Angolo θ ---
\pic [draw, "$\theta$", angle radius=1cm] {angle = M--O--B};

% --- Forze agenti ---
% Tensione
\draw[->, blue, thick] (M) -- ++(\ang:0.7) node[above right=-1pt and -6pt] {$\vec{T}$};

% Peso totale
\draw[->, red!80!black, thick] (M) -- ++(270:0.9) node[right] {$\vec{P}=m\vec{g}$};

% --- Componenti del peso ---
% Radiale (verso O)
\draw[->, red!60!black, thick] (M) -- ++(\ang-90:0.7)
  node[below right=-15pt and 1pt] {$\vec{P_r}=mg\cos\theta$};

% Tangenziale (perpendicolare al filo)
\draw[->, red!60!black, thick] (M) -- ++(\ang-180:0.4)
  node[above left=2pt and -2pt] {$\vec{P_t}=mg\sin\theta$};

% --- Angolo tra P e Pr ---
\coordinate (Pr) at ($(M)+(\ang-90:0.7)$);
\coordinate (P)  at ($(M)+(0,-0.9)$);

\pic [draw, "$\theta$", angle radius=1cm] {angle = P--M--Pr};


% --- Punto di sospensione ---
\fill (O) circle (0.04);

\end{tikzpicture}

    \caption{Pendolo semplice}
    \label{fig:1_Pendolo}
\end{figure}
