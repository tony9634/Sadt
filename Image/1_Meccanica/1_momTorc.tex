\begin{figure}[h!]
    \centering
\begin{tikzpicture}[scale=2]
    % Coordinate
    \coordinate (O) at (0,0);
    \coordinate (A) at (2,0);
    \coordinate (B) at (3,1);

    % Vettori
    \fill (O) circle (1pt);
    \node[left] at (O) {$O$};
    \draw[->, thick] (O) -- (A) node[midway, below] {$\vec{r}$};
    \draw[->, thick] (A) -- (B) node[midway, above] {$\vec{F}$};

    % Disegno dell'angolo tra i due vettori
    \draw (A) -- ++(-0.5,0) coordinate (aux); % punto ausiliario per l'arco
    \pic [draw, "$\theta$", angle radius=0.4cm] {angle = B--A--O};

\end{tikzpicture}
    \caption{Definizione del momento torcente rispetto a un punto fisso}
    \label{fig:1_momTorc}
\end{figure}
