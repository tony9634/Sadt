\begin{tikzpicture}[scale=1.3]

  \shade[top color=black!60,bottom color=black!80,shading angle=10] % background
    (7:1) arc (7:355:1);
  
  \fill[thick,black,postaction=decorate, % rough inner surface
    decoration={markings,mark=between positions 0.55 and 1 step 0.03 with {
                  \node[transform shape,inner sep=1pt]
                  (hit\pgfkeysvalueof{/pgf/decoration/mark info/sequence number}) {};
    }}]
    (7:1) arc (7:353:1) --++ (-7:-0.18)
    decorate[decoration={random steps,segment length=2,amplitude=1pt}]
        {arc (-7:-353:0.82)} -- cycle;

% Raggi che entrano e colpiscono la circonferenza
\draw[red,thick,->] (2.3,0) -- (0.9,0); % ingresso
\draw[red,thick,->] (0.9,0) -- (-0.1,-0.8); % primo rimbalzo
\draw[red,thick,->] (-0.1,-0.8) -- (0.6,0.5); % secondo rimbalzo
\draw[red,thick,->] (0.6,0.5) -- (-0.8,-0.1); % terzo rimbalzo
\draw[red,thick,->] (-0.8,-0.1) -- (-0.2,-0.51); % quarto rimbalzo

% Etichetta raggio
\node[red] at (3.5,0) {Raggio incidente};

\end{tikzpicture}