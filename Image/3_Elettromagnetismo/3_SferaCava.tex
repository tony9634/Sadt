\begin{tikzpicture}[scale=2.5, >=stealth, line cap=round, line join=round]
    % Definizioni di base
    \def\R{1.6} % Raggio della sfera
    \def\thetaValue{60} % Angolo polare \vartheta in gradi
    
    % *** Parametri per l'effetto 3D (prospettiva) ***
    \def\yAxisScale{0.4} % Raggio y dell'ellisse per la profondità (0.4 è un buon compromesso)

    % Coordinate sferiche (in base a \thetaValue)
    \pgfmathsetmacro\r{\R*sin(\thetaValue)} % raggio dell'anello: R*sin(\vartheta)
    \pgfmathsetmacro\z{\R*cos(\thetaValue)} % coordinata z: R*cos(\vartheta)

    % --- Parametri per lo spessore infinitesimo (d\vartheta) ---
    \def\dTheta{3} % Metà della variazione d\vartheta per l'anello in gradi
    \pgfmathsetmacro\thetaUpper{\thetaValue - \dTheta} % Angolo superiore
    \pgfmathsetmacro\zUpper{\R*cos(\thetaUpper)}
    \pgfmathsetmacro\rUpper{\R*sin(\thetaUpper)}
    \pgfmathsetmacro\thetaLower{\thetaValue + \dTheta} % Angolo inferiore
    \pgfmathsetmacro\zLower{\R*cos(\thetaLower)}
    \pgfmathsetmacro\rLower{\R*sin(\thetaLower)}
    
    % --- Asse e Origine ---
    
    % Asse z (asse di rotazione)
    \draw[->, very thick] (0, 0) -- (0, \R + 0.5) node[above] {$z$};
    \draw[dashed, gray] (0, 0) -- (0, -\R-0.1);
    % Asse x 
    \draw[->, very thick] (0, 0) -- (\R + 0.5, 0) node[below right] {$x$};
    \draw[dashed, gray] (0, 0) -- (-\R-0.1, 0);

    % Centro della sfera (Origine O)
    \coordinate (O) at (0, 0);
    \draw[fill] (O) circle (1.5pt) node[below right] {$O$};

    % Linea per \omega (velocità angolare)
    \draw[->, very thick] (0.2, \R + 0.1) -- (0.2, \R + 0.4);
    \node[right] at (0.2, \R + 0.25) {$\vec{\omega}=\omega\,\hat{z}$};
    
    % --- Disegno della Sfera 3D (Simulata) ---

    % 1. Parte nascosta (posteriore) della sfera
    \draw[thick, gray] (-\R, 0) arc (180:360:\R cm and \R*\yAxisScale cm); 
    \draw[dashed, gray] (-\R, 0) arc (180:0:\R cm and \R*\yAxisScale cm); 

    % 2. Contorno verticale (circonferenza)
    \draw[thick, gray] (0, 0) circle (\R);

    % --- Anello Elementare (Guscio) ---
    
    % 1. Parti nascoste (posteriori)
    % Bordo superiore posteriore
    \draw[dashed, blue!70!cyan] (\rUpper, \zUpper) arc (0:180:\rUpper cm and \rUpper*\yAxisScale cm); 
    % Bordo inferiore posteriore
    \draw[dashed, blue!70!cyan] (-\rLower, \zLower) arc (180:0:\rLower cm and \rLower*\yAxisScale cm); 

% ============================
%   2. RIEMPIMENTO POSTERIORE (nascosto)
% ============================
\path[fill=blue!30, opacity=0.7]
 % Arco posteriore superiore (da -rUpper a rUpper)
 (-\rUpper, \zUpper)
   arc[start angle=180, end angle=0,
       x radius=\rUpper, y radius=\rUpper*\yAxisScale]
  --
  (\rLower, \zLower)
  % Arco posteriore inferiore (da rLower a -rLower)
   arc[start angle=0, end angle=180,
       x radius=\rLower, y radius=\rLower*\yAxisScale]
  -- cycle;


% ============================
%  3. BORDI ANTERIORI (visibili)
% ============================
% Bordo superiore anteriore
\draw[line width=2pt, blue!70!cyan] 
  (\rUpper, \zUpper) 
    arc[start angle=0, end angle=-180, 
      x radius=\rUpper, y radius=\rUpper*\yAxisScale];

% Bordo inferiore anteriore
\draw[line width=2pt, blue!70!cyan] 
  (-\rLower, \zLower)
    arc[start angle=-180, end angle=0, 
        x radius=\rLower, y radius=\rLower*\yAxisScale];

% ============================
%   4. RIEMPIMENTO ANTERIORE (visibile)
% ============================
\path[fill=blue!40, opacity=0.7]
  % Bordo superiore anteriore
  (\rUpper, \zUpper)
    arc[start angle=0, end angle=-180,
        x radius=\rUpper, y radius=\rUpper*\yAxisScale]
  -- % Linea sinistra
  (-\rLower, \zLower)
  % Bordo inferiore anteriore (in senso orario, da -rLower a rLower)
    arc[start angle=-180, end angle=0,
        x radius=\rLower, y radius=\rLower*\yAxisScale]
  -- cycle;
        
    % Linee di collegamento laterali (visibili)
    \draw[line width=2pt, blue!70!cyan] (\rUpper, \zUpper) -- (\rLower, \zLower);
    \draw[line width=2pt, blue!70!cyan] (-\rUpper, \zUpper) -- (-\rLower, \zLower);

    % --- Etichettatura dei parametri ---
    
    % Raggio 'r' dell'anello (al punto medio \z)
    \draw[<->, dashed, red] (0, \z) -- (\r, \z);
    \node[above, xshift=10pt] at (\r*0.5, \z) {Raggio: $r=R\sin\vartheta$};
    
    % Raggio R della sfera (sul piano xz, passa attraverso il punto medio)
    \draw[->, very thick] (O) -- (\r, \z); 
    \node[right] at (\r*0.5 - 0.2, \z*0.5 + 0.1) {$R$};
    
    % Angolo \vartheta
    \draw[thick] (0.3, 0) arc (0:\thetaValue:0.3);
    \node[right] at (0.3, 0.1) {$\vartheta$};

    % Larghezza R d\vartheta (sulla superficie)
    
    % Coordinate per la freccia di larghezza (all'esterno della sfera)
    \coordinate (P_Top_Ext) at (\rUpper+0.1, \zUpper);
    \coordinate (P_Bottom_Ext) at (\rLower+0.1, \zLower);
    
    % Frecce per la larghezza R d\vartheta
    \draw[<->, very thick, red!70!black, shorten >= 1pt, shorten <= 1pt] 
        (P_Top_Ext) -- (P_Bottom_Ext);
    
    % Etichetta centrata sullo spessore
    \node[right, xshift=5pt] at ($(P_Top_Ext)!0.5!(P_Bottom_Ext)$) {Larghezza: $R\,d\vartheta$};

    % Elemento di area (etichette testuali)
    \node[below right, align=left, text=black!80!green, font=\small] at (\R + 0.05, -\R + 0.1) {
        \textbf{Elemento di area $dS$:}\\
        $\text{Circonf.} \times \text{Larghezza}$
    };
    \node[below right, align=left, text=black!80!green, font=\small] at (\R + 0.05, -\R + 0.5) {
        $dS = (2\pi r)(R\,d\vartheta)$\\
        $dS = 2\pi R^2\sin\vartheta\,d\vartheta$
    };
    
\end{tikzpicture}