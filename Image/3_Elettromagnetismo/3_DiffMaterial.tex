\begin{tikzpicture}[font=\sffamily]

% General styles
\tikzset{
  sample/.style={
    draw, thick,
    minimum width=6cm,
    minimum height=1.8cm,
    align=center,
    fill=#1!20
  },
  Bfield/.style={-Stealth, line width=0.9pt},
  mfield/.style={-Stealth, line width=0.8pt, red},
  domainline/.style={line width=0.9pt, blue, -Stealth},
  labeltop/.style={font=\large, yshift=3mm},
  labelbottom/.style={font=\footnotesize, yshift=-6mm, align=center}
}

%%%%%%%%%%%%%%%%%%%%%%%%%%%%
%% LAYOUT con due colonne
%%%%%%%%%%%%%%%%%%%%%%%%%%%%

\def\colsep{8cm}   % distanza tra colonne
\def\rowsep{6.5cm} % distanza tra righe


%%%%%%%%%%%%%%%%%%%%%%%%%%%%%%%%%%%%%%%%%%%%%%%%
%%%%%%%%%%%% 1) DIAMAGNETICO (colonna sinistra)
%%%%%%%%%%%%%%%%%%%%%%%%%%%%%%%%%%%%%%%%%%%%%%%%
\begin{scope}[xshift=-\colsep/2]

\node (dia) at (0,0) [sample=cyan] {};
\node[labeltop] at (dia.north) {\textbf{Diamagnetico}};
\node[labelbottom] at (dia.south)
{Momenti $m$ indotti opposti a $B$ \\ ($\chi < 0$)};

\foreach \y in {0.7,0.35,0,-0.35,-0.7}{
  \draw[Bfield] (-3,\y) -- (3,\y);
}

\foreach \x in {-1.5,-0.5,0.5,1.5}{
  \draw[mfield] (\x,0) -- ++(-0.5,0);
}

\begin{scope}[yshift=-3.5cm, scale=0.8]
  \draw[->] (0,0) -- (2,0) node[right]{\(B\)};
  \draw[->] (0,0) -- (0,1.2) node[above]{\(M\)};
  \draw[thick] (0.2,1.0) -- (1.8,0.3);
\end{scope}

\end{scope}


%%%%%%%%%%%%%%%%%%%%%%%%%%%%%%%%%%%%%%%%%%%%%%%%
%%%%%%%%%%%% 2) PARAMAGNETICO (colonna destra)
%%%%%%%%%%%%%%%%%%%%%%%%%%%%%%%%%%%%%%%%%%%%%%%%
\begin{scope}[xshift=\colsep/2]

\node (para) at (0,0) [sample=green] {};
\node[labeltop] at (para.north) {\textbf{Paramagnetico}};
\node[labelbottom] at (para.south)
{Momenti $m$ debolmente allineati a $B$ \\ ($\chi > 0$)};

\foreach \y in {0.7,0.35,0,-0.35,-0.7}{
  \draw[Bfield] (-3,\y) -- (3,\y);
}

\foreach \x in {-1.5,-0.5,0.5,1.5}{
  \draw[mfield] (\x,0) -- ++(0.5,0);
}

\begin{scope}[yshift=-3.5cm, scale=0.8]
  \draw[->] (0,0) -- (2,0) node[right]{\(B\)};
  \draw[->] (0,0) -- (0,1.2) node[above]{\(M\)};
  \draw[thick] (0.2,0.2) -- (1.8,1.0);
\end{scope}

\end{scope}


%%%%%%%%%%%%%%%%%%%%%%%%%%%%%%%%%%%%%%%%%%%%%%%%
%%%%%%% 3) FERROMAGNETICO (riga sotto, centrato)
%%%%%%%%%%%%%%%%%%%%%%%%%%%%%%%%%%%%%%%%%%%%%%%%
\begin{scope}[yshift=-\rowsep]

\node (ferro) at (0,0) [sample=orange] {};
\node[labeltop] at (ferro.north) {\textbf{Ferromagnetico}};
\node[labelbottom] at (ferro.south)
{Domini magnetici allineati; remanenza e isteresi};

\foreach \y in {0.7,0.35,0,-0.35,-0.7}{
  \draw[Bfield] (-3,\y) -- (3,\y);
}

\foreach \x in {-1.5,-0.5,0.5,1.5}{
  \draw[domainline] (\x,0) -- ++(0.5,0);
}

\foreach \x in {-1.0,-0.5,0,0.5}{
  \draw[dashed] (\x,-1) -- ++(0,2);
}

\begin{scope}[yshift=-3.5cm, scale=0.8]
  \draw[->] (-1.5,0) -- (1.5,0) node[right]{\(B\)};
  \draw[->] (0,-1.2) -- (0,1.4) node[above]{\(M\)};
  \draw[thick, rounded corners=3pt]
    (-1.1,-0.45) .. controls (-0.4,-0.9) and (0.4,-0.9) .. (1.1,-0.45)
    .. controls (0.45,0) and (0.5,0.6) .. (1.1,0.95)
    .. controls (0,1.2) and (-0,1.2) .. (-1.1,0.95)
    .. controls (-0.5,0.6) and (-0.5,0) .. (-1.1,-0.45);
\end{scope}

\end{scope}

\end{tikzpicture}