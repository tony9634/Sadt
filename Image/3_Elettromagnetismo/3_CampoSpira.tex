\begin{tikzpicture}[scale=2]

% Assi con prospettiva
\draw[->] (0,0) -- (2,0) node[below right] {$x$};
\draw[->] (0,0) -- (-1,1) node[above left] {$y$};
\draw[->] (0,0) -- (0,1.5) node[left] {$z$};

% Piano XY (in prospettiva)
\fill[cyan!20,opacity=0.5] (-1.2,1.2) -- (2,0) -- (1.2,-1.2) -- (-2,0) -- cycle;

% Spira come ellisse
\draw[red,thick] (0,0) ellipse [x radius=1, y radius=0.4];

% Centro C
\node at (0,0) [below left] {$C$};

% Punto sull'asse z
\coordinate (Pz) at (0,1.2); % punto a quota z
\fill (Pz) circle (0.02);
\node[right] at (Pz) {$z$};

% Elemento infinitesimo dl sulla spira
\coordinate (dl) at (0.8,0.32); % punto sulla ellisse
\draw[<->,blue,thick] (dl) ++(-0.25,0.01) -- ++(0.26,-0.1); % piccolo segmento tangente
\node[xshift=6pt,yshift=1pt] at (dl) {$d\vec{l}$};

% Vettore r dal punto z a dl
\draw[->,orange,thick] (Pz) -- (dl) node[midway,right] {$\vec{r}$};

% Raggio R
\draw[->] (0,0) -- ($(0,0)!0.85!(dl)$) node[midway,above] {$R$};

% Angolo alpha (tra z e r)
\draw (0,0.8) arc[start angle=-90,end angle=-35,radius=0.3];
\node at (0.1,0.95) {$\alpha$};

% Angolo theta tra asse x e raggio verso dl
\draw (0.3,0) arc[start angle=0,end angle=22,radius=0.3]
    node[pos=0.7,right] {$\theta$};
    
% Formula nel centro
\node[fill=yellow!30,draw,rounded corners] at (-1.5,1.5)
{$\displaystyle B = \dfrac{\mu_0 I}{2R}$};

% Formula sull'asse
\node[fill=yellow!50,draw,rounded corners,text width=4cm] at (0,-1.2)
{\centering $\displaystyle B_{\text{asse}} = \dfrac{\mu_0 i}{2} \dfrac{R^2}{(R^2 + z^2)^{3/2}}$};

\end{tikzpicture}